 **NYTimes.com no longer supports Internet Explorer 9 or earlier. Please
upgrade your browser.
\href{http://www.nytimes3xbfgragh.onion/content/help/site/ie9-support.html}{LEARN
MORE »}

**Sections

**Home

**Search

\hypertarget{the-new-york-times}{%
\subsection{\texorpdfstring{\href{http://www.nytimes3xbfgragh.onion/}{The
New York Times}}{The New York Times}}\label{the-new-york-times}}

\hypertarget{-magazine-}{%
\subsubsection{\texorpdfstring{
\href{https://www.nytimes3xbfgragh.onion/section/magazine}{Magazine}
}{ Magazine }}\label{-magazine-}}

 \href{https://www.nytimes3xbfgragh.onion/section/magazine}{Magazine}
\textbar{}25 Songs That Tell Us Where Music Is Going

**Close search

\hypertarget{site-search-navigation}{%
\subsection{Site Search Navigation}\label{site-search-navigation}}

Search NYTimes.com

**Clear this text input

Go

\url{https://nyti.ms/2m5QOKu}

\hypertarget{site-navigation}{%
\subsection{Site Navigation}\label{site-navigation}}

\hypertarget{site-mobile-navigation}{%
\subsection{Site Mobile Navigation}\label{site-mobile-navigation}}

\hypertarget{25-songs-that-tell-us-where-music-is-going}{%
\section{25 Songs That Tell Us Where Music Is
Going}\label{25-songs-that-tell-us-where-music-is-going}}

\includegraphics{https://static01.graylady3jvrrxbe.onion/newsgraphics/2017/03/03/music2017/assets/icons/arrow.png}

\hypertarget{-25-songs}{%
\subsubsection[ \protect\hyperlink{ux2f}{25
Songs}]{\texorpdfstring{\href{http://www.nytimes3xbfgragh.onion/magazine}{\protect\includegraphics{https://static01.graylady3jvrrxbe.onion/newsgraphics/2017/03/03/music2017/assets/icons/nytmag_short.svg}}
\protect\hyperlink{ux2f}{25 Songs}}{ 25 Songs}}\label{-25-songs}}

\begin{itemize}
\tightlist
\item
  \protect\hyperlink{ux2fadele-send-my-love-to-your-new-lover}{ 1 Send
  My Love (to Y... Adele }
  \href{https://open.spotify.com/track/3LGxef3inmn5jMLGDz9lqw}{\includegraphics{https://static01.graylady3jvrrxbe.onion/newsgraphics/2017/03/03/music2017/assets/icons/spotify.svg}Full
  Track}
\item
  \protect\hyperlink{ux2fleonard-cohen-you-want-it-darker}{ 2 You Want
  It Darker Leonard Cohen }
  \href{https://open.spotify.com/track/5zb7npjQqoJ7Kcpq4yD9qn}{\includegraphics{https://static01.graylady3jvrrxbe.onion/newsgraphics/2017/03/03/music2017/assets/icons/spotify.svg}Full
  Track}
\item
  \protect\hyperlink{ux2fmissy-elliott-im-better}{ 3 I'm Better Missy
  Elliott }
  \href{https://open.spotify.com/track/2Kf9fwIOwZwd6Aw7OxfkF0}{\includegraphics{https://static01.graylady3jvrrxbe.onion/newsgraphics/2017/03/03/music2017/assets/icons/spotify.svg}Full
  Track}
\item
  \protect\hyperlink{ux2ffuture-mask-off}{ 4 Mask Off Future }
  \href{https://open.spotify.com/track/3rOSwuTsUlJp0Pu0MkN8r8}{\includegraphics{https://static01.graylady3jvrrxbe.onion/newsgraphics/2017/03/03/music2017/assets/icons/spotify.svg}Full
  Track}
\item
  \protect\hyperlink{ux2fpentatonix-jolene}{ 5 Jolene Pentatonix }
  \href{https://open.spotify.com/track/60PbWX0IWXg10YaE6VlywH}{\includegraphics{https://static01.graylady3jvrrxbe.onion/newsgraphics/2017/03/03/music2017/assets/icons/spotify.svg}Full
  Track}
\item
  \protect\hyperlink{ux2fa-tribe-called-quest-we-the-people-}{ 6 We the
  People .... A Tribe Called Quest }
  \href{https://open.spotify.com/track/3wUX7HpXS382s1DHTRouVk}{\includegraphics{https://static01.graylady3jvrrxbe.onion/newsgraphics/2017/03/03/music2017/assets/icons/spotify.svg}Full
  Track}
\item
  \protect\hyperlink{ux2flil-yachty-one-night}{ 7 One Night Lil Yachty }
  \href{https://open.spotify.com/track/40yJGQyEaBbIkXpwMFkXip}{\includegraphics{https://static01.graylady3jvrrxbe.onion/newsgraphics/2017/03/03/music2017/assets/icons/spotify.svg}Full
  Track}
\item
  \protect\hyperlink{ux2fkelela-rewind}{ 8 Rewind Kelela }
  \href{https://open.spotify.com/track/2adF8MLoBFPSh1mxXYgK9T}{\includegraphics{https://static01.graylady3jvrrxbe.onion/newsgraphics/2017/03/03/music2017/assets/icons/spotify.svg}Full
  Track}
\item
  \protect\hyperlink{ux2fkungs-vs-cookin-on-3-burners-this-girl}{ 9 This
  Girl Kungs vs. Cookin' on 3 Burners }
  \href{https://open.spotify.com/track/1A8j067qyiNwQnZT0bzUpZ}{\includegraphics{https://static01.graylady3jvrrxbe.onion/newsgraphics/2017/03/03/music2017/assets/icons/spotify.svg}Full
  Track}
\item
  \protect\hyperlink{ux2fchurch-of-misery-make-them-die-slowly-john-george-haigh}{
  10 Make Them Die Slow... Church of Misery }
  \href{https://open.spotify.com/track/08O2pHuPdcJ5PoxFqbpxLO}{\includegraphics{https://static01.graylady3jvrrxbe.onion/newsgraphics/2017/03/03/music2017/assets/icons/spotify.svg}Full
  Track}
\item
  \protect\hyperlink{ux2fmica-levi-oliver-coates-barok-main}{ 11 Barok
  Main Mica Levi \& Oliver Coates }
  \href{https://open.spotify.com/track/2Dy1MaumORP1kh1sHDXi0w}{\includegraphics{https://static01.graylady3jvrrxbe.onion/newsgraphics/2017/03/03/music2017/assets/icons/spotify.svg}Full
  Track}
\item
  \protect\hyperlink{ux2fka-mourn-at-night}{ 12 Mourn at Night Ka }
  \href{https://open.spotify.com/track/6zao3czBFKk1zbmIe3azJd}{\includegraphics{https://static01.graylady3jvrrxbe.onion/newsgraphics/2017/03/03/music2017/assets/icons/spotify.svg}Full
  Track}
\item
  \protect\hyperlink{ux2fshirley-caesar-hold-my-mule}{ 13 Hold My Mule
  Shirley Caesar }
  \href{https://open.spotify.com/track/0JjsI172ped2hF5v27o28J}{\includegraphics{https://static01.graylady3jvrrxbe.onion/newsgraphics/2017/03/03/music2017/assets/icons/spotify.svg}Full
  Track}
\item
  \protect\hyperlink{ux2frufus-wainwright-a-womans-face-reprise-sonnet-20}{
  14 A Woman's Face --- R... Rufus Wainwright }
  \href{https://open.spotify.com/track/5D12h7kioqQdCKzoDiUfvJ}{\includegraphics{https://static01.graylady3jvrrxbe.onion/newsgraphics/2017/03/03/music2017/assets/icons/spotify.svg}Full
  Track}
\item
  \protect\hyperlink{ux2fjames-mcmurtry-copper-canteen}{ 15 Copper
  Canteen James McMurtry }
  \href{https://open.spotify.com/track/2jG8qg7KDYWKm9smYCXz1i}{\includegraphics{https://static01.graylady3jvrrxbe.onion/newsgraphics/2017/03/03/music2017/assets/icons/spotify.svg}Full
  Track}
\item
  \protect\hyperlink{ux2fsolange-fubu}{ 16 F.U.B.U. Solange }
  \href{https://open.spotify.com/track/2aR6SYRJxy2QsBxA459ATF}{\includegraphics{https://static01.graylady3jvrrxbe.onion/newsgraphics/2017/03/03/music2017/assets/icons/spotify.svg}Full
  Track}
\item
  \protect\hyperlink{ux2fariana-grande-side-to-side}{ 17 Side to Side
  Ariana Grande }
  \href{https://open.spotify.com/track/1pKeFVVUOPjFsOABub0OaV}{\includegraphics{https://static01.graylady3jvrrxbe.onion/newsgraphics/2017/03/03/music2017/assets/icons/spotify.svg}Full
  Track}
\item
  \protect\hyperlink{ux2fkanye-west-fade}{ 18 Fade Kanye West }
  \href{https://open.spotify.com/track/3cCxoOgfi6hgt8MNteuiiD}{\includegraphics{https://static01.graylady3jvrrxbe.onion/newsgraphics/2017/03/03/music2017/assets/icons/spotify.svg}Full
  Track}
\item
  \protect\hyperlink{ux2fccile-mclorin-salvant-trolley-song}{ 19 Trolley
  Song Cécile McLorin Salvant }
  \href{https://open.spotify.com/track/5j78Obp9QYGopCUlsDd3nQ}{\includegraphics{https://static01.graylady3jvrrxbe.onion/newsgraphics/2017/03/03/music2017/assets/icons/spotify.svg}Full
  Track}
\item
  \protect\hyperlink{ux2flady-gaga-grigio-girls}{ 20 Grigio Girls Lady
  Gaga }
  \href{https://open.spotify.com/track/6Xxvdy1m9TX9HTxOzQcfuV}{\includegraphics{https://static01.graylady3jvrrxbe.onion/newsgraphics/2017/03/03/music2017/assets/icons/spotify.svg}Full
  Track}
\item
  \protect\hyperlink{ux2fyoung-ma-ooouuu}{ 21 Ooouuu Young M.A }
  \href{https://open.spotify.com/track/0D21XvHcVsIvJM6FcGY2BT}{\includegraphics{https://static01.graylady3jvrrxbe.onion/newsgraphics/2017/03/03/music2017/assets/icons/spotify.svg}Full
  Track}
\item
  \protect\hyperlink{ux2fcharles-bradley-changes}{ 22 Changes Charles
  Bradley }
  \href{https://open.spotify.com/track/3i85UIF4nR7qZXTnEwqdPr}{\includegraphics{https://static01.graylady3jvrrxbe.onion/newsgraphics/2017/03/03/music2017/assets/icons/spotify.svg}Full
  Track}
\item
  \protect\hyperlink{ux2ffrank-ocean-seigfried}{ 23 Seigfried Frank
  Ocean }
  \href{https://open.spotify.com/track/1BViPjTT585XAhkUUrkts0}{\includegraphics{https://static01.graylady3jvrrxbe.onion/newsgraphics/2017/03/03/music2017/assets/icons/spotify.svg}Full
  Track}
\item
  \protect\hyperlink{ux2fmitski-your-best-american-girl}{ 24 Your Best
  American Girl Mitski }
  \href{https://open.spotify.com/track/172rW45GEnGoJUuWfm1drt}{\includegraphics{https://static01.graylady3jvrrxbe.onion/newsgraphics/2017/03/03/music2017/assets/icons/spotify.svg}Full
  Track}
\item
  \protect\hyperlink{ux2fmigos-bad-and-boujee}{ 25 Bad and Boujee Migos
  }
  \href{https://open.spotify.com/track/4Km5HrUvYTaSUfiSGPJeQR}{\includegraphics{https://static01.graylady3jvrrxbe.onion/newsgraphics/2017/03/03/music2017/assets/icons/spotify.svg}Full
  Track}
\item
  \protect\hyperlink{ux2fmetallica-spit-out-the-bone}{ + Spit Out the
  Bone Metallica }
  \href{https://open.spotify.com/track/7MHkweinOceprzkRWIYTTc}{\includegraphics{https://static01.graylady3jvrrxbe.onion/newsgraphics/2017/03/03/music2017/assets/icons/spotify.svg}Full
  Track}
\item
  \protect\hyperlink{ux2fthe-ep-the-music-issue-podcast}{ ++ The EP }
\end{itemize}

\href{https://open.spotify.com/user/nytmag/playlist/4FrPqvd2eD17G3slCaIAYG}{\includegraphics{https://static01.graylady3jvrrxbe.onion/newsgraphics/2017/03/03/music2017/assets/icons/spotify_playlist.svg}}

Lettering by Bráulio Amado

\hypertarget{introduction}{%
\subsubsection{Introduction}\label{introduction}}

By Nitsuh Abebe

A strange thing you learn about American popular music, if you look back
far enough, is that for a long time it didn't much have ``genres'' ---
it had ethnicities. Vaudeville acts, for instance, had tunes for just
about every major immigrant group: the Italian number, the Yiddish
number, the Irish one, the Chinese. Some were sung in a spirit of abuse;
others were written or performed by members of those groups themselves.
And of course there were the minstrel shows, in which people with
mocking, cork-painted faces sang what they pretended were the songs of
Southern former slaves. This was how we reckoned with our melting pot:
crudely, obliviously, maybe with a nice tune and a beat you could dance
to.

Sometime in the 1950s, the mainstream saw its last great gasp of this
habit. A nation that considered itself very space-age and worldly
enjoyed quaint spins on sentimental Italian music (``That's Amore'' and
its pizza pies) and Trinidadian calypso songs about hard, simple labor
(``Day-O'' and its bananas). You had your ``Latin'' numbers, your
Hawaiian ones, your ``Asian'' songs --- light ethnic pastiches laid out
cheerily, like an international buffet that serves falafel one day and
schnitzel the next, never too bothered about how accurate the recipes
are.

There was a simple notion behind all this stuff, and it was the belief
that music, like food, came from someplace, and from some people. Even
when it was played in a condescending ethnic-joke burlesque of who those
people actually were --- even when it was pretty aggressively racist ---
the notion remained: Different styles sprang from different people. Then
all of this changed, and we decided to start thinking of pop music not
as a folk tradition but as an art; we started to picture musicians as
people who invented sounds and styles, making intellectual decisions
about their work.

But music is still, pretty obviously, tied to people. How else do you
create a situation in which, after decades of hip-hop's being the main
engine of pop music, it can still be a little complicated when nonblack
people rap? That vexed thing we call ``identity'' leans its considerable
weight on all kinds of questions: which sounds comfort us or excite us;
where and how we listen to them; how we move our bodies as they play.
Watch a mere silhouette of a human being dancing to music, and you can
immediately guess things about who they are and where they came from.

In 2017, identity is the topic at the absolute center of our
conversations about music. There may be times when this fact grates at
us, when it feels as though there must be other dimensions of the world
to attend to; ``surely,'' you moan, ``there are songs that speak to
basic human emotions in ways that transcend the particulars of who we
are!'' But if you look through the essays in this magazine, you may
notice two things. One is that, unbidden and according to no plan, they
find themselves continually reckoning with questions of identity. The
other is that they're doing this because the musicians are, too.

A Japanese-American musician writes a song called ``Your Best American
Girl.'' An R.\&B. singer titles one ``F.U.B.U.'' --- or, ``for us, by
us.'' Are you part of her ``us''? The house music in Kanye West's
``Fade'': Does it make you picture the black Chicagoans who helped
invent it or the club-going Europeans who embraced it? How does it work
when a queer woman matches the sexual braggadocio of male rappers, when
L.G.B.T. activists sing a country song for a restaurant chain that once
fired gay employees, when Leonard Cohen revisits his childhood religious
inheritance?

This is what we talk about now, the music-makers and the music-listeners
both. Not the fine details of genre and style --- everyone, allegedly,
listens to everything now --- but the networks of identity that float
within them. Maybe decades ago you could aim your songs at a mass
market, but music does not really have one of those anymore. Artists
have to figure out whom they're speaking to and where they're speaking
from. The rest of us do the same. For better or worse, it's all identity
now. ♦

\emph{Nitsuh Abebe is a story editor for the magazine.}

Read more

\hypertarget{1}{%
\subsection{1.}\label{1}}

\hypertarget{--send-my-love-to-your-new-lover}{%
\subsection{\texorpdfstring{ Send My Love (to Your New
Lover)}{  Send My Love (to Your New Lover)}}\label{--send-my-love-to-your-new-lover}}

\hypertarget{adele}{%
\paragraph{Adele}\label{adele}}

All I want to do is dance --- but it's complicated.

By Wesley Morris

\href{https://open.spotify.com/track/3LGxef3inmn5jMLGDz9lqw}{\includegraphics{https://static01.graylady3jvrrxbe.onion/newsgraphics/2017/03/03/music2017/assets/icons/spotify.svg}Full
Track}

Loving Adele shouldn't be that hard. When a chorus brings her voice to
its cruising altitude, it's like you're up there, flying with it. Down
here, on earth, where her third album, ``25,'' made Adele the
top-selling artist of 2015 and 2016, she has that realness we say we
value in the people we elevate to stardom; last month, during the
Grammys telecast, she cursed as she interrupted a laconic version of
George Michael's ``Fastlove'' in order to get the tempo right.

But even Adele knows that loving Adele is complicated. At the Grammys,
``25'' won album of the year, and a poignant portion of her acceptance
speech was a tribute to Beyoncé, whose album ``Lemonade'' broke the
cultural Richter scale --- and didn't win any of the big awards. The
moment was poignant because it was earnest: Adele stood just a few feet
in front of the woman she called her ``idol'' and spoke of how
``Lemonade'' had empowered her and ``my black friends.'' This was the
sort of candor you usually have to wait for Kanye West to deliver, only
with none of West's biliousness, recrimination and, however myopic,
sense of history. Adele didn't have to acknowledge that history --- of a
white industry's crowning preference for white artists. She was living
it.

\includegraphics{https://static01.graylady3jvrrxbe.onion/images/2017/03/12/magazine/12music10-adele/12music10-adele-master180-v4.jpg}

Black people have never been necessary to make black music. But it has
become obligatory for white artists who do (and who win prizes for it)
to pay a public contrition tax to their black peers, whether it's Adele
to Beyoncé or, three years earlier, Macklemore texting (then
publicizing) an apology to Kendrick Lamar for having won (with Ryan
Lewis) the Grammys in the rap category. That's the future of music:
recognizing, in the present, that you're permanently indentured to the
past.

Setting aside its enormous sales, ``25'' is not the artistically
catholic landmark that ``Lemonade'' is but an old-fashioned record,
built around the bloom and flare of Adele's singing. Some of the hooks,
though, could catch a whale. And isn't whaling pop's whole point?

Yes, certain cultural institutions have a habit of setting traps that
trigger trauma. But when it's just me and Adele --- very good Adele,
catchy-as-hell Adele --- the triggers lock. ``Send My Love (to Your New
Lover),'' the second track on ``25,'' makes you mad that we live in a
world where what happens at the Grammys can't not matter. It makes you
mad that we put a political price tag on this kind of perfection. I must
have danced to this song 200 times, in blocks of repeats. (Why doesn't
this woman make more fast songs?)

\includegraphics{https://static01.graylady3jvrrxbe.onion/newsgraphics/2017/03/08/podcast-promo/2439f132b82793e8f551cf5ef1af8b55f031e643/NYTMAG_EP_152.png}

It starts with her saying, ``Just the guitar. O.K., cool.'' Then comes
the rhythmic plink of a guitar Lindsey Buckingham might have picked. The
plink is married to a kick drum's heartbeat. Then comes The Voice, at a
low smolder, the smoke still rising from a crater of disillusionment. By
the prechorus, her voice is flanked by other Adeles swooping in, on
multiple tracks, to dispel the dismay of having dated someone with cold
feet and to wish the best for this person's next girlfriend. ``Send my
love to your new lover. Treat her better,'' she sings, going up a note
and adding an extra, addictive breath to ``lover.''

Is this a black song? It moves in dance-hall time. The swelling
repetitions are chillingly churchy. And the voice itself has what can be
only called soul. It's in the pews, the rafters and the aisles. I love
this song because it makes me feel strong --- as strong as singing ``We
gon' slay'' any time Beyoncé does. In other words, ``Send My Love'' sets
out to catch a whale. This song makes me feel ridiculous for reacting to
institutional biases that pressure us into calling Adele a trespasser.
All I want to do when I hear it is call her Ishmael.♦

\emph{Wesley Morris is a critic at large for The Times and a staff
writer for the magazine.}

Read more

\hypertarget{2}{%
\subsection{2.}\label{2}}

\hypertarget{--you-want-it-darker}{%
\subsection{\texorpdfstring{ You Want It
Darker}{  You Want It Darker}}\label{--you-want-it-darker}}

\hypertarget{leonard-cohen}{%
\paragraph{Leonard Cohen}\label{leonard-cohen}}

After a long life, he was ready.

By Jonathan Mahler

\href{https://open.spotify.com/track/5zb7npjQqoJ7Kcpq4yD9qn}{\includegraphics{https://static01.graylady3jvrrxbe.onion/newsgraphics/2017/03/03/music2017/assets/icons/spotify.svg}Full
Track}

It wasn't an email from God, but it was close. Leonard Cohen had written
to ask if Gideon Zelermyer, the cantor of Congregation Shaar Hashomayim
near Montreal --- Cohen's childhood synagogue --- was interested in
recording with him.

Zelermyer was soon sitting inside the synagogue's sanctuary with a few
members of Shaar's all-male choir, playing with different arrangements
for ``You Want It Darker,'' the title track of Cohen's 14th and final
studio album. Some of the words Cohen had given them to work with were
familiar; they were borrowed from two of Judaism's holiest prayers. One
is the Kaddish, recited by mourners after the death of a loved one. The
other is the High Holy Days prayer Hineni --- literally, ``Here I am''
--- a personal entreaty to God, the worshiper asking plaintively for
mercy. The choir's voices are the first sounds you hear on the album,
their ethereal harmonics giving way to sparse instrumentation and
Cohen's weary, subterranean growl, then returning to back up the song's
choruses and final movement.

This was hardly the first time that Cohen had drawn on his Judaism for
his music. Though he had a complicated relationship with his religious
inheritance, it provided a natural vocabulary for him; it was what he
knew, and its stories of human suffering and, occasionally, redemption
suited his poet's pull toward the existential. But never before have
Cohen's biblical references felt so charged, so dark, so pointed.
``Magnified, sanctified, be thy holy name,'' he sings. ``Vilified,
crucified, in the human frame. A million candles burning for the help
that never came. You want it darker.'' Then, echoing the words that
Abraham spoke as he answered God's command to sacrifice his only son:
``Hineni, I'm ready, my Lord.''

\includegraphics{https://static01.graylady3jvrrxbe.onion/images/2017/03/12/magazine/12music13-cohen/12music13-cohen-master180.jpg}

You can't listen to these words without thinking about the fact that
Cohen was dying when he recorded them. It's one thing to meditate on
faith and mortality when death is an abstraction. It is surely another
when you can feel it bearing down on you. And yet the choir's harmonies
manage to transform the song, lifting Cohen's solitary struggle into
something universal, even eternal.

Cohen once said that he did not think of himself as a religious person,
but his life was in many ways an extended spiritual journey. Buddhism,
Scientology, kabbalah, Hare Krishna, Hinduism --- Cohen sampled them
all. Yet in his final years, he found himself drawn back to the
171-year-old synagogue where he had become a bar mitzvah, where both his
grandfather and great-grandfather served as presidents, where a
photograph of his Hebrew-school class taken in 1949 still hangs on the
wall. Cohen was living in Los Angeles, but a cousin in Montreal sent him
a recording of Zelermyer and his choir, reuniting Cohen with sounds that
had never stopped echoing in his head. He and the cantor struck up an
email correspondence. ``May your voice reach that Place and bring down
the blessings,'' Cohen wrote Zelermyer in 2008, before the High Holy
Days. (``He can't write anything normally, can he?'' the cantor
remembers thinking.) And then several years later came the note, asking
for help with a new record. As Cohen put it, ``I'm looking for a sound
like the Shaar choir and cantor of my youth.''

In October, the record-release event for ``You Want It Darker'' was held
at the residence of the Canadian consul general in Los Angeles.
Zelermyer was seated with the other V.I.P.s in the front row. It was the
first time he had met Cohen in person. It would also be the last. Weeks
later, Cohen's coffin was lowered into the earth at Shaar Hashomayim's
cemetery. Zelermyer stood next to Cohen's family as they recited the
Kaddish.♦

\emph{Jonathan Mahler is a staff writer for the magazine.}

Read more

\hypertarget{3}{%
\subsection{3.}\label{3}}

\hypertarget{--im-better}{%
\subsection{\texorpdfstring{ I'm
Better}{  I'm Better}}\label{--im-better}}

\hypertarget{missy-elliott}{%
\paragraph{Missy Elliott}\label{missy-elliott}}

A rap legend bends the present into a hopeful future.

By Julianne Escobedo Shepherd

\href{https://open.spotify.com/track/2Kf9fwIOwZwd6Aw7OxfkF0}{\includegraphics{https://static01.graylady3jvrrxbe.onion/newsgraphics/2017/03/03/music2017/assets/icons/spotify.svg}Full
Track}

When Missy Elliott divines the future in her science-fiction-inflected
videos, she never envisions dystopia. Even in the oversize jacket
bearing the slogan ``Save the Humans'' that she wears in the video for
``I'm Better,'' her optimism supersedes the plea. Positivity feels like
an intrinsic part of her message and her temperament, and in a climate
of uncertainty and fear, ``I'm Better'' is an alien message of hope,
full of sneaky inspiration meant to help us get up in the morning,
rise-and-grind style. ``It's another day, another chance,'' the Miami
producer Lamb raps, helping Elliott out on the chorus. ``I wake up, I
wanna dance. So as long as I got my friends ... I'm better, I'm better,
I'm better!''

Released in the last week of January, the signature futurism of ``I'm
Better'' was a welcome jolt. The beat is so stark it's almost jarring,
with a tiptoeing synth melody that makes Elliott sound like some kind of
gumshoe on the case, tracking down dudes in the act of creeping.
Characteristically slick, the video features dancers in headgear that
mimics light-therapy acne masks. Elliott gleams in outlandish feathered
adornment and lip gloss the color of a vinyl record, delivering cheeky
brags about her coterie of admirers, who watch her ``like he watchin'
`Scandal' --- but I'm just here with my girls.''

Elliott's approach has always been an antidote to conservatism, both
within music and sometimes outside it. She's unwilling to abide by any
perceived rules of language. For years, she has bent rhythmic parameters
and willfully warped nouns so that they rhyme, as if to prove that
there's always another, probably far cooler way of going about life and
language than the rest of us have recognized. On ``I'm Better,'' she
rhymes the Spanish word for ``fire'' with an English word for a car ---
``He say I'm hot, I'm so fue-eh-go/Pull up on him in my veh-heh-co.''
These tricks carry such uncomplicated joy that they give us permission
to celebrate too. ``He say I'm pretty, I'm pretty, you must be from
Brazil, I must be from México,'' she concludes, rhyming the Spanish
pronunciation --- and giving dap to those of us with origins in the
global South, who could use the shout-out these days.

``I'm Better'' also positions Elliott in the American South, the
Virginia of her upbringing, as well as Orlando, Miami, Atlanta. Its
sound adheres to current trends in Southern hip-hop --- which is
curious, because Elliott sets trends but rarely follows them. Yet by
employing the syrupy, stripped-down delivery so many young Southern
rappers favor, she establishes a lineage, from her work to theirs:
Stylistically, it is difficult to imagine a Migos without a Missy
Elliott.

For Elliott to position herself within a style that's popular and yet
often derided, usually by Northerners and hip-hop fans over the age of
25 --- for her to sprinkle her flavor on the 2017 iteration of music
made for the subwoofers in your Jeep --- does a real service to youth
culture. More than most rappers, she seems to bend time to her will. And
if she can't stop it, she will swerve around it.♦

\emph{Julianne Escobedo Shepherd is the culture editor for Jezebel.}

Read more

\hypertarget{4}{%
\subsection{4.}\label{4}}

\hypertarget{--mask-off}{%
\subsection{\texorpdfstring{ Mask Off}{  Mask Off}}\label{--mask-off}}

Party music for the age of opiates.

By Amos Barshad

\href{https://open.spotify.com/track/3rOSwuTsUlJp0Pu0MkN8r8}{\includegraphics{https://static01.graylady3jvrrxbe.onion/newsgraphics/2017/03/03/music2017/assets/icons/spotify.svg}Full
Track}

\hypertarget{future}{%
\paragraph{Future}\label{future}}

I spent a large chunk of 2016 trying to talk to Future. I hounded and
pressed his P.R. team. Around the top of the year, my nagging paid off;
as instructed, I flew from New York to London for an audience with the
rapper. I was set to join his tour and follow him for a few days through
Europe. Very soon, I would find, things would not break my way.

On the first night I found myself at a chicken shop called Nando's,
directly across the street from his overflowing concert venue, rather
than backstage as planned. With great envy, I stared at the crowd
flowing in as I munched my breast-and-wing combo.

Future had just sat down for an interview with the BBC's Charlie Sloth,
who asked him about his relationship with Blac Chyna, the
Kardashian-affiliated reality-TV personality with whom he'd possibly
been romantically involved. ``Are yous two still cool?'' Sloth asked, in
a punchy London rumble. ``We great,'' Future responded, in his trademark
flat-affect reserve.

Privately, though, the entreaty into his personal life enraged him. He
declared an immediate media blackout. I was in line for his concert when
I got the call from P.R.: The interview was decisively off. I spent a
weekend eating delicious Pakistani food, watching Tottenham play
Leicester City, hoping for a change of mind that never came.

\includegraphics{https://static01.graylady3jvrrxbe.onion/images/2017/03/12/magazine/12music9-FUTURE/12music9-FUTURE-master180.jpg}

At that point, Future was roughly two years into a radical public and
artistic reimagining. It started in the fall of 2014, not long after his
breakup with the R.\&B. singer Ciara and the soft landing of his
pop-friendly sophomore album, ``Honest.'' The failure became an
important inflection point. Over the next few years, he created a
swelling mass of music with a cloaking grandness to it: Take a step
inside, and you were entombed. The songs were lean and incessant and
almost completely devoid of any other voice but Future's. And what that
voice was intimating to us, from behind the thickest of blackout
curtains, was that our man had given up on his conscience and that he
was guzzling the prescription cough syrup Promethazine and downing Xanax
and that he was having sex with women he did not really care about and
that this was neither making him feel good nor bad but rather it was
making him feel nothing.

And then, the really weird part: Suddenly, rightfully, Future was
considered an artist who could not be ignored, our best next hope for
rap-star transcendence. Embracing personal destruction took him there.
Was it a meltdown or a rise? What were we to make of a man who made
party music out of a death rattle? How should I know? I was stuck at
Nando's.

\textbf{This February,} after a period of uncharacteristic dormancy,
Future --- born Nayvadius Wilburn in 1983 in Atlanta --- returned with a
barrage. He released two albums in two weeks, and there are rumors of a
third. On the heartbroken ``HNDRXX,'' he gushed and apologized and
balladeered. Future has always had a cockeyed crooner alter-ego; here,
it takes the whole stage, suggesting one tantalizing path forward for
his discography. And on ``Future,'' he boasted and bragged and sounded
weirdly content.

Take ``Mask Off,'' a down-tempo track built, by the elite producer Metro
Boomin, around a bizarre but lovely woodwind sample. The song hints at a
certain kind of violence and ruthlessness, the kind suggested by a
criminal setting off into the night and choosing to leave the ski mask
at home. And yet it's the kind of song you would want the D.J. to slip
on right when you've lost count of your drinks and you're feeling buzzy
and smiley and warm.

Historically, M.C.s have treated narcotics as product to be moved;
today's younger, party-happy rappers give drugs a gleeful knucklehead
spin. But when Future describes his voluminous intake, he does so with
all the zeal of a man popping open a days-of-the-week pill organizer. On
the hook to ``Mask Off,'' Future rattles off drugs, unsentimentally:
``Percocets/Molly, Percocets.'' For him, sometimes the drugs are great;
sometimes, not so much. On ``Mask Off,'' amid rhymes about how totally
fun and good his life is, he calls Promethazine his ``guillotine.''

It feels reductive to try to pin an artist down on the sins of his
persona. Hip-hop's greatest running trick has been blurring the lines of
``real life'' and art. But with the rate at which Future was rapping
about drugs, one question was inevitably posed: Is this an addiction? If
so, it was a new spin on a classic trope. The arc of pretty much every
drug movie mimics the whiz-bang of the initial high and the
eye-blackening horror of the inevitable comedown. Future's music
acknowledged that drug addiction isn't that cinematically neat: It's the
high and the comedown over and over again.

\subsection{}

\hypertarget{-1}{%
\paragraph{}\label{-1}}

After London, Future's P.R. staff and I got back into our little dance.
Emails, calls, texts, pleadings. Soon, I received word that Future was
ready to talk again. It was in Toronto that we actually met, and where
it was so cold that the streets had a kind of a permafrost hue. The
pavement felt as if it could, at any point, shatter. For a few days, I
tagged along with Future and his affable crew. The first order of
business was an interview with a TV station on the 19th floor of a
high-end hotel.

The interviewer, a friendly reporter in all black, was drinking a glass
of white wine. She showed Future the tattoos on her arm; I couldn't
quite see them, but they were apparently inspired by his music. ``Oh!''
he said in delight, then waved away Shooter, his ever-present personal
photographer. ``Let us have this moment.'' Apparently emboldened, the
reporter shared more. ``I've been drinking 'cause I'm nervous,'' she
told him. She had ended a long-term relationship, she said, because of
his music.

In person, Future provides no outward signs that you should approach him
with confessionals. He's imposingly tall and more than a little grave.
He is also beautiful. (L.A. Reid, chairman of Epic Records, who signed
Future, told me of their initial meeting: ``Usually I ask people to
audition. Future, I didn't even want him to move. `Let's get you signed
while you're sitting there looking like that.' '') But the TV reporter
went for it, and it was brave. And almost immediately, Future went back
to thumbing through his phone. He either hadn't heard what she said or
he chose to ignore it. After a few beats of silence he finally looked
up. ``Ay, what's the name of this hotel?''

The next day, I finally had my chance to connect. We were upstairs at a
middlebrow bistro with a lot of bare wood, and Future had just finished
off an impromptu date. His partner had off-white blond hair tucked under
an actually white baseball cap and was wearing a combination
bodysuit/tunic (also white). She'd brought him a late Valentine's Day
gift, a nice puffy coat: ``It's that Chanel swag!'' she announced. They
ate sushi, chicken wings and steak salad. She told him that when she
travels, she likes to stay at Airbnbs because that way you get
``immersed'' in local culture.

And I know this because during the totality of the date, the team and I
were sitting at the adjoining table. Eventually his date left, and
Future announced his verdict on the holiday, to grins from the crew:
``Man this Valentine's Day {[}expletive{]} a setup.''

Finally, we talked. I brought up London. He smiled. I guess you could
call it a sheepish smile. I told him it really didn't seem as if he
wanted to do press at all. I asked him why he was going through with it.
``I don't wanna do it,'' he said, maybe even relieved to say it out
loud. ``My publicist like: `Man, why you got a publicist if you don't
wanna do press?' I'm trying to give you the real me, but they want me to
be fake, so I'd rather not even say nothing.''

The conversation rolled on, meandered. It even clicked into gear at a
few points. He talked about his itinerant childhood, how he never wanted
to have a fixed address so no one with an antagonistic agenda would ever
be able to find him. He talked about the love and care of the family
members that sorted him out. He remembered the joy of playing ``Racks,''
an early hit, for the first time and how the D.J. loved it so much he
didn't want to give the CD back. And he said that it all, eventually,
changed everything. ``Back then, I had no feelings,'' he told me. ``It
wasn't until I started doing music that I started to really have a
conscience.''

It was nice, and fleeting. But I never was able to get a hook into him.
I never could formulate a question that made him want to really talk.
When I called DJ Spinz, one of Future's regular collaborators, he told
me about Future's work ethic, his remarkable ability to unfurl a whole
song after 20 minutes of hearing a beat roll. But nothing he said felt
as relevant as when he told me this: ``Future doesn't speak much.''

I was reminded of a moment back in London. I had stuck around after
Nando's long enough to try to finagle my way into the show. My move was
to sidle close to the stage door, in the alley, hoping for an opening.
It never came. Upon Future's arrival, his luxury sedan idled until
minutes before his set time. Then he exited the back seat and walked
directly through the stage door, surrounded by an imposing security
detail, with the massive hood of an arctic parka over his head. I never
even saw his face.

I chased Future through two separate sovereign nations and walked away
remembering one thing: I love rappers. They never break character.♦

\emph{Amos Barshad is a senior writer for The Fader magazine.}

Read more

\hypertarget{5}{%
\subsection{5.}\label{5}}

\hypertarget{--jolene}{%
\subsection{\texorpdfstring{ Jolene}{  Jolene}}\label{--jolene}}

\hypertarget{pentatonix}{%
\paragraph{Pentatonix}\label{pentatonix}}

Can choir nerds bring harmony to a divided country?

By Amy Phillips

\href{https://open.spotify.com/track/60PbWX0IWXg10YaE6VlywH}{\includegraphics{https://static01.graylady3jvrrxbe.onion/newsgraphics/2017/03/03/music2017/assets/icons/spotify.svg}Full
Track}

``A cappella is cool again,'' declared the Cracker Barrel Twitter
\href{https://twitter.com/crackerbarrel/status/659404256785616896}{account},
an unexpected authority on such matters, in October 2015. The occasion
was a new sponsorship deal with Pentatonix, the astonishingly popular
vocal quintet. Much like Cracker Barrel, Pentatonix is one of those
cultural institutions whose existence you could go your whole life not
noticing, until you do, when you realize it is everywhere. ``3x Grammy
Award Winning Multi-Platinum Selling Choir Nerds,'' the group's Twitter
bio boasts, and it's true: Over the course of almost six years,
Pentatonix has sold more than six million records, put out the first a
cappella album to debut at No. 1 on Billboard's Top 200 chart and
amassed over two billion YouTube views.

The third of those Grammys, for Best Country Duo/Group Performance, came
in February, for a collaboration with Dolly Parton on a new version of
her '70s heartbreak classic, ``Jolene.'' The world has Cracker Barrel to
thank for this team-up: Both Parton and Pentatonix have partnerships
with the chain, which brought them together to record ``Jolene'' as a
promotional single. The result sounds like a barbershop quartet singing
at an old-timey barn raising. ``We're both very family-centric and very
wholesome,'' Kirstin Maldonado, the Pentatonix mezzo-soprano, said in
2015, explaining why the group's image meshes so well with the brand's,
``and I think our demographics really integrate well.''

Pentatonix's music is indeed relentlessly wholesome --- just five voices
cooing and trilling and humming cheerily along. There is nothing
dangerous or dark or threatening in their work, which consists mostly of
chaste covers of pop hits and Christmas songs. No sex, only kissing. No
bad behavior, no cursing and certainly no politics. The five members of
Pentatonix, though, represent a rainbow coalition of historically
marginalized groups. Maldonado, the group's lone woman, is Hispanic. One
of the male lead singers, Mitch Grassi, is openly gay. He and the other
male lead, Scott Hoying, have a side-project YouTube series called
``\href{https://www.youtube.com/user/sup3rfruit/featured}{Superfruit},''
which sells tank tops that say ``Marriage Is So Gay.'' Avi Kaplan, the
basso profundo, is Jewish, and Kevin Olusola, the beat-boxer, is black
and a practicing Seventh-day Adventist.

This ``Sesame Street'' version of American harmony makes the group's
association with Cracker Barrel --- which has a history of
discrimination --- particularly fascinating. Despite looking like a
United Colors of Benetton ad styled by the Kardashians, the members of
Pentatonix sound like the jukebox at a heavily chaperoned sock hop;
through them, Cracker Barrel can dip its toes in the waters of inclusion
without fearing any backlash. Indeed, if you're looking for criticism of
Pentatonix, you're more likely to find it on liberal music websites
trashing its ``nightmarishly hammy'' sound (to quote Rolling Stone) than
on conservative sites attacking its politics. Members have appeared on
``All Things Considered'' and ``Fox and Friends,'' and nobody batted an
eye. Could it be that five choir nerds hold the secret to bridging a
divided nation?♦

\emph{Amy Phillips is the news director for Pitchfork.}

Read more

\hypertarget{6}{%
\subsection{6.}\label{6}}

\hypertarget{---we-the-people-}{%
\subsection{\texorpdfstring{ We the People
....}{   We the People ....}}\label{---we-the-people-}}

\hypertarget{a-tribe-called-quest}{%
\paragraph{A Tribe Called Quest}\label{a-tribe-called-quest}}

Who makes the protest music?

By Greg Tate

\href{https://open.spotify.com/track/3wUX7HpXS382s1DHTRouVk}{\includegraphics{https://static01.graylady3jvrrxbe.onion/newsgraphics/2017/03/03/music2017/assets/icons/spotify.svg}Full
Track}

Of course it would be A Tribe Called Quest who gave the nation's pop
body politic its first acidly anthemic counterassault on Donald Trump's
anti-immigrant sentiments. This is, after all, the Golden Age vintage
hip-hop band with respective Trinidadian-American and Muslim founders in
Phife Dawg and Ali Shaheed Muhammad, one half of the original team; the
one that gave dap to the plight of Haitian immigrants on its 1990 debut
album, ``People's Instinctive Travels and the Paths of Rhythm''; the
group whose very name is a panegyric in praise of the hyphenated
American dream. Of course they did it.

Their performance of that snarky resistance song --- ``We the People
....'' --- at last month's
\href{https://www.grammy.com/videos/a-tribe-called-quest-anderson-paak-grammy-performance-medley}{Grammy
Awards}, alongside their Jamaican-American fellow traveler Busta Rhymes,
didn't just split ownership of the whole showcase with Beyoncé's
fertility-goddess spectacular. It had Busta addressing the president by
a pungent new name: ``Agent Orange.'' Months earlier, their ``Saturday
Night Live'' debut of the song (and its parent project, ``We Got It From
Here ... Thank You 4 Your Service'') came in the same week the president
blustered his way into the Oval Office, lost the popular vote and set
off a resistance movement in song and dance and sardonic album titling.
That moment made New York rap iconoclasm --- and A Tribe Called Quest
--- matter again in one epic, epochal heartbeat: Who else are you gonna
call when the dirty work of radical-oppositional boom-bap needs to be
done, live and direct, in irony-redolent rhyme?

\includegraphics{https://static01.graylady3jvrrxbe.onion/images/2017/03/12/magazine/12music11-atcq/12music-ATCQ-master180.jpg}

By February, A Tribe Called Quest's Grammy performance was sustaining
the same Black Panther- and Public Enemy-powered momentum that Beyoncé
brought to last year's Super Bowl --- the same kind in Kendrick Lamar's
own fire-breathing, flame-throwing, plantation-to-penitentiary salvo at
last year's Grammys.

To the extent that America's current protest movement has provoked any
pop paeans worthy of Bob Dylan or Curtis Mayfield, they have emerged out
of Black Lives Matter --- and in remarkably short order. It has been 31
months since ``we the people'' of Ferguson, and then Baltimore, inspired
not just those cities' urban commandos but a generation of
youthful-and-truthful hip-hop and R.\& B. standard-bearers. Whether the
white alt-rock left will seize this moment's baton as frankly or as
fruitfully remains to be seen.

We need not hold our breath waiting, though. ``We the People ....'' is
full of ready-to-rumble pushback. Its lyrics name and gather together
all the targeted --- Mexicanfolk, Muslimfolk, gayfolk, womenfolk,
\#BlackLivesMatterfolk --- under one force field. And under one
intersectional, Queens-bred guerrilla meal plan:

\begin{quote}
``We don't believe you 'cause we the people

Are still here in the rear, ayo, we don't need you

You in the `killing off good young nigga' mood

When we get hungry we eat the same {[}expletive{]} food

The ramen noodle.''♦
\end{quote}

\emph{Greg Tate is a writer and musician who lives in Harlem.}

Read more

\hypertarget{7}{%
\subsection{7.}\label{7}}

\hypertarget{--one-night}{%
\subsection{\texorpdfstring{ One
Night}{  One Night}}\label{--one-night}}

\hypertarget{lil-yachty}{%
\paragraph{Lil Yachty}\label{lil-yachty}}

Rap music fully unmoored from its past.

By JAMIE LAUREN KEILES

\href{https://open.spotify.com/track/40yJGQyEaBbIkXpwMFkXip}{\includegraphics{https://static01.graylady3jvrrxbe.onion/newsgraphics/2017/03/03/music2017/assets/icons/spotify.svg}Full
Track}

There are two different ways you can keep up with pop. The first is by
drifting along with the current, bobbing immersed in the changing of the
charts --- so lost from any point of reference on the shore that minor
fluctuations (the downfall of an air horn, the outflow of a sound)
hardly register. From there, in the tide, you don't ask, ``How did pop
get here?'' because you were with it the whole time. You, most likely,
are in high school, or college, or somewhere that music flows like water
all around. Pop, in such places, is understood by osmosis.

The rest of us --- less lucky --- must accept the second system.
Somewhere along the way we get busy with work, or prioritize movies, or
decide to have kids and look up to find we've lost the thread. We emerge
from our hiatus and turn on the radio only to wonder, ``How did pop get
here?'' We have no idea. It is then that we begin to study pop trends by
rote --- by reading reviews and listening to podcasts, by looking up
songs on Shazam in the drugstore, by turning to Google to ask it: ``Who
is Lil Yachty?''

Before I heard Yachty's music, I heard that he was awful. I heard he had
no flow and couldn't rap, was a meme and a poseur, wasn't repping the
culture. Wiz Khalifa called his style ``mumble rap.'' His sound, I
heard, was dinky. When I finally relented and listened to ``One Night''
--- Yachty's top-charting song of last year --- nothing I heard could
dispel what I'd been told. Flow? Clumsy. Lyrics? Bland. The beat on the
track was the inverse of a banger --- tinny and thin, compulsively
looping, like something churned out with a really cool toy. In a voice
that was somehow both droning and singsong, the 19-year-old Atlantan
wanly shrugged off commitment. It was an anthem for the player who can't
be tied down, run a thousand times though a Xerox machine. Catchy like a
backing track in a commercial, it was sticky for all the texture it
lacked. I listened on repeat with car-crash infatuation. It wasn't good
rap by any technical metric --- or at least not by any metric I knew ---
but for some unknown reason, I wanted to like it. Like any pop émigré, I
started to study.

\includegraphics{https://static01.graylady3jvrrxbe.onion/images/2017/03/12/magazine/12music14-lilyachty/12mag-12lilyachty-t_CA0-master180.jpg}

The hip-hop establishment had little to offer. Ebro Darden of Hot 97
took to Twitter to whine about Yachty; '90s production legend Pete Rock
posted online, ``He sucks mud on a rainy day!'' Yachty, meanwhile,
seemed to pay little mind to the genre conventions he'd supposedly
betrayed. Of hip-hop forefathers Biggie and Tupac, he told Billboard,
``I honestly couldn't name five songs.'' Instead, his influences were
Fall Out Boy and Coldplay. He signed an endorsement deal with Target,
alongside the Canadian pop star Carly Rae Jepsen. In Nautica shirts and
plastic-beaded braids, he was an ungraceful hybrid of your grandpa and
your niece. In the style of Warhol backstage at WrestleMania --- awkward
and quiet, but nonetheless enthused --- he made the rounds, telling
press outlets, ``I'm not a rapper.'' Instead, he claimed he was an
artist, a brand. As old-schoolers and gatekeepers scratched their heads
and wept, Yachty continued to rise through the ranks, buoyed by fans who
had no trouble understanding.

His come-up was something straight outta LinkedIn, an origin uncaring
toward the rap plot as we know it. Before releasing a single track ---
before perhaps even rapping a single bar --- he spent a summer
networking with influencers in New York, befriending the internet
cool-teen Luka Sabbat and A\$AP-affiliate-slash-stylist Ian Connor. It
wasn't a mixtape or even SoundCloud that brought ``One Night'' to fame
but instead the track's appearance in a viral
\href{https://youtu.be/251cxou3yR4}{comedy clip on YouTube}.

The subsequent music video made a mockery of an older generation's
rap-vid fantasies. What begins as Yachty on a yacht with three women
quickly descends into maritime madness --- jump-cuts from hammerhead
sharks and harpoons, to dress-up in wet suits and other nautical garb,
to glitch-art graphics of slow-swimming fish, calling to mind the early
days of home computers. It's a sloppy pastiche of what's cool right now:
a surging nostalgia for the '90s, blended with the net-art aesthetic of
today, wrapped in the cachet of an Atlanta pedigree. It might all be a
gimmick, but it doesn't feel random. If rap music is founded on a
process of layering --- autobiography with references and samples ---
then Yachty does the same for the rap career itself, mixing what's
trending with the right group of friends and tying it together with the
loose thread of a sound. If it isn't good rap, then it's the perfect
kind of music for those of us barely treading water in the zeitgeist,
who can't understand new music by ear. Yachty had to do his research,
just like the rest of us.♦

\emph{Jamie Lauren Keiles is a writer based in Los Angeles.}

Read more

\hypertarget{8}{%
\subsection{8.}\label{8}}

\hypertarget{--rewind}{%
\subsection{\texorpdfstring{ Rewind}{  Rewind}}\label{--rewind}}

She has found her own corner of R.\&B.

By JENNA WORTHAM

\href{https://open.spotify.com/track/2adF8MLoBFPSh1mxXYgK9T}{\includegraphics{https://static01.graylady3jvrrxbe.onion/newsgraphics/2017/03/03/music2017/assets/icons/spotify.svg}Full
Track}

\hypertarget{kelela}{%
\paragraph{Kelela}\label{kelela}}

The 33-year-old musician Kelela favors the kind of fashion aesthetic
that science-fiction films sometimes use to signify characters from the
future: gravity-defying materials in iridescent or metallic colors. For
a recent rainy night in Strasbourg, the small city in the northeastern
corner of France, she strode onstage dressed like a lieutenant in an
anime cartoon, in an oversize gray bomber jacket, matching shorts and
heels made from white fabric that stretched above her knees. She raised
her hands and gave a hard stare to the crowd. ``My mission tonight,''
she said, ``is to show the breadth of R.\&B. music. It has influenced
every genre, pretty much, so anyone who thinks it is basic or
rudimentary has another thing coming.''

There were no whoops, claps or even smiles. The audience remained
passive. Kelela likes to keep an eye out for the edges of the crowd,
where her core fans (``the queer black and brown weirdos'' as she put it
to me) usually congregate. But tonight, the scene was homogeneous in a
very European way: Women favored striped boatnecks, red lips and messy
topknots; the men, zipped-up pullovers and spotless white trainers.
Kelela nodded at her D.J., Loric Sih, a sweet-faced boy with bleached
blond hair and wire-rimmed Harry Potter glasses, and they dove into her
set. True to her word, amid the switchbacks of her feathery falsetto
voice, there was no mistaking the roots of classic R.\&B. --- all set to
spacey electronic beats far outside the traditional canon. The room
became a sound installation of Kelela's reverb-y vocals and synthetic
'90s-era Miami bass.

Kelela's stage was minimally adorned, but her lighting team is adept at
creating James Turrell-like lightscapes that drape her figure in rich
reds, purples and blues. At one point, her face and body were
illuminated by an electric shade of cyan, while the background remained
shaded in dark azure. The effect made Kelela look as ethereal and
spectral as the music radiating from the speakers. The handful of times
I'd seen her perform in the United States, the audience was rapt for the
entire performance --- reverent during her atmospheric songs, breaking
into exuberant, feverish dance during her fast-paced ones. (Her music
can keep the lovesick company in bed just as easily as it can shepherd a
party past sunrise.)

\includegraphics{https://static01.graylady3jvrrxbe.onion/images/2017/03/12/magazine/12music5-kelela1/12music5-kelela1-master180-v2.jpg}

But that night the concertgoers remained inscrutable. When she
transitioned into a new song --- ``Blue Light,'' the first single from
her long-awaited debut album --- I pulled out my phone and sent the
recording to some friends back home. Some 4,000 miles away, they seemed
more excited than the people physically present in the concert hall.
Finally, about 30 minutes into her set, Sih began playing ``Rewind,''
the closest thing Kelela has to a pop song. The audience, charmed at
last, succumbed to the irresistible beat and danced along. The moment
was buoyant but short-lived: It was her last number. She thanked the
crowd and then bounded offstage.

When she was back in her dressing room, the composure Kelela had
projected to the audience quickly dissipated. She stood with her hands
on her hips, chewing on her lip. Her boyfriend --- a filmmaker named
Cieron Magat, with whom she shares an apartment in London --- murmured
words of reassurance and handed her a cup of homemade ginger tea. ``That
was one of the worst ones,'' she said, sighing and taking off her
earrings. Magat told her not to worry, but Kelela wanted to deconstruct
the performance.

``The thing I'm always looking for are the eyes, or even the face that's
like, I don't know what this is but I'm into it,'' she said. ``But I got
nothing.'' She peeled her clothes off absent-mindedly and paced around
her dressing room. ``I was this guy'' --- she threw her arms up in
imitation of the shruggie emoticon --- and sighed again. ``But it's
unrealistic in this context.''

By context, Kelela meant that she wasn't the headliner --- most people
were there to see the main act, the moody British band the xx. Earlier
in the day, while roaming around Strasbourg, I noticed that the posters
advertising the show didn't even mention her name. That night, in the
nearly sold-out venue, a space that could hold 4,000 people, only a few
attendees were black; her ``queer black and brown weirdos'' were
missing. In the United States, Kelela is part of the vanguard of black
female musicians who make emotional soul, women like Solange, SZA and
\href{https://www.nytimes3xbfgragh.onion/interactive/2016/03/10/magazine/25-songs-that-tell-us-where-music-is-going.html\#/syd-tha-kyd-the-internet-get-away}{Syd
tha Kyd}. The music of these women is aimed squarely at the heart chakra
of young black women; it legitimizes as much as it asserts the value of
being yourself --- even if that self is thought to be a little
off-center. Kelela, in particular, explodes the notion that blackness is
monolithic, a single Pantone square instead of untold variations. Her
music is geared to a generation that lives for juxtapositions and
unexpected arrangements, sonically and visually.

In 2012, Kelela was performing at a show that Solange's manager happened
to see. She asked for a demo and gave the song to Solange, who asked
Kelela to come on tour with her later that year, introducing Kelela to
an audience who could appreciate her innovations in R.\&B. In October
2013, Kelela released ``Cut 4 Me,'' an impressive mixtape composed of 13
songs that were initially free online. At the time, Kelela wanted to see
how far she could push herself as an artist and play with the boundaries
of R.\&B. Kelela's uninhibited experimentation, as well as the rich
latticework on songs like ``Send Me Out,'' impressed critics. Pitchfork
gave the collection a
\href{http://pitchfork.com/reviews/albums/18661-kelela-cut-4-me/}{rare
8.3 rating out of 10} and said she had ``the talent to make herself
heard, and the intelligence to get it all together''; Spin called the
collection ``stunning'' and said the singer could ``go anywhere from
here.'' That November, Solange chose two of Kelela's songs for ``Saint
Heron,'' a compilation album released by Solange's label of the same
name. In 2015, she released a six-song EP called ``Hallucinogen.'' Her
sound on the EP somehow managed to evoke Jimmy Jam and Terry Lewis,
Björk and Donna Summer all at the same time. It felt like a sonic relic
of the past unearthed 100 years in the future. Since then, fans have
been waiting for her first full-length album, which Kelela expects to
release this year.

In her dressing room, Kelela folded herself into a pretzel on the couch
next to me. A candle burned in the background. She knew it had been an
off night, but because she loves performing so much, she was still
buzzing from the energy. ``The goal,'' she explained, ``is just --- how
many people can I put on, just so they can name it, even if they don't
know what it is yet?''

\textbf{Kelela Mizanekristos} was born in 1983 to Mizanekristos Yohannes
and Neghist Girma, students who escaped war-torn Ethopia and immigrated
separately to the United States. She was raised in Gaithersburg, Md.
Kelela's parents introduced her to the violin when she was a child, and
she practiced singing along to the radio in her bedroom at night and
composing medleys in her head. Her father was fond of Blues Alley, an
all-ages jazz supper club in Georgetown frequented by Dizzy Gillespie
and Sarah Vaughan. He often took Kelela with him, and she fell in love
with the culture of music. She listened to Kirk Franklin on the radio
and learned to sing in ge'ez, an ancient language used primarily in the
Ethiopian church, which she attended with her mother. You can still
catch the influence in her voice --- the way she turns sounds into
sacred geometry, almost unconsciously stairstepping through the vowels
and consonants.

In her early to mid 20s, she would go to a Washington bar called 18th
Street Lounge for its Sunday-night house sessions. A D.J. named Sam
Burns played eclectic soul and deep house music, and after a few drinks,
if she heard a bit of music that reminded her of another song, she would
jump on the microphone and blend the two in real-time. ``I would run to
the microphone and figure out a way to sing it. I would create a flip,''
she says. ``That is where I live.''

Her first boyfriend, Kris Funn, whom she met when she was 19, played the
upright bass, and she sat in bars for hours, watching him and his
friends play. She joked to me that she was a ``jazz wife'' but also
admitted that she received an unexpected education: She learned to
listen to music, to get a feel for it. Eventually the couple broke up,
but Funn encouraged Kelela to trust her instincts and not be intimidated
by her lack of formal music training.

By that time, Kelela was a student at American University, studying
international studies and sociology. ``I was the only brown girl, all
the time, talking about African politics,'' she told me. ``I thought I
was going to be an academic. In my head, I am supposed to be a college
graduate. I wanted to finish. But I was not motivated to sit there and
do that paper. I had a lot of resistance.'' She felt alienated by the
program. She dropped out.

This was in 2006, and synthpop, epitomized by bands like the Knife, was
trending. She began recording in a punk house in Washington, a city with
a hard-core lineage that included acts like Fugazi and Bad Brains. She
thrived in an environment devoid of rules. ``What's so beautiful about
punk culture,'' Kelela says, ``is that the whole idea is that you don't
have to be perfect. Just try. Let's just try to make it ourselves.'' At
first she sang over indie rock, but it didn't feel authentic to her. She
wanted to experiment with electronic music --- ``not real instruments,''
she says.

She spent hours on MySpace, scrolling through pages of music and
listening to instrumentals. She recorded herself singing over sounds she
liked. Then she would send the artist her sample, along with an
invitation to collaborate. Two notable electronic producers agreed,
including Daedelus, who featured her on a track. At the same time, a
friend introduced her to the electro duo Teengirl Fantasy, and they
created a song. Will Boston, a founder of the music label Fade to Mind,
heard their collaboration and was impressed by what he described to me
as her ``wealth of startling honesty, felt through her vocals.''

By then, Kelela was living in Los Angeles, and Boston brought her a
thumb drive of sounds from the label and its British counterpart, Night
Slugs. Kelela spent the next several days poring over the files,
improvising lyrics over the sounds she liked, turning them into songs.
She loved the otherworldliness of the instrumentals --- staccato mixes
that used sound effects like tinkling glass and guns reloading over drum
machines. The music complemented the gossamer scales she likes to sing
in. It was ``exactly what I'd been looking for,'' she says. Two of the
songs she produced during this time were on the mixtape she released in
2013. Kelela took care to describe her role in developing the mixtape,
to make certain I knew how active she was in it --- perhaps to counter
the idea that she isn't self-made or in charge of her own sound. ``The
first thing people want to take away from me is my agency,'' she told
me.

\includegraphics{https://static01.graylady3jvrrxbe.onion/images/2017/03/12/magazine/12music5-kelela2/12music5-kelela2-master180-v2.jpg}

Alex Sushon, an electronic producer who goes by Bok Bok, was one of the
artists Kelela met as she was working on her mixtape, and these days,
she tends to work mostly with British producers like him, possibly
because they've been pushing the boundaries of R.\&B. more aggressively
by blending it with grime, the East London style of dance music.
Electronica, Sushon told me, is referential in the same way that R.\&B.
tends to be. ``That's how we think about music, inserting small samples
from other genres, little shortcuts that are saying something by playing
something,'' he says. Because of the internet, he explained, musicians
can share references more easily than they did in the past. Google,
YouTube and SoundCloud make it easy. Sushon compares this dynamic to
``memes, but in musical form.''

The closest analogy is sampling, but it's more complex than that. I
watched Kelela and her D.J. perform this trick onstage, during one of
their songs in Strasbourg. While she was singing ``The High,'' a
fuzzed-out, gritty ode to desire, Sih mixed in a few seconds of a 2002
song by Tweet called ``Oops (Oh My),'' and Kelela segued into its
vocals. Here, suddenly, was the thrilling flicker of a decade-old hit
that had almost entirely faded from popular culture, tucked into her own
noir love song. It was the perfect encapsulation of Kelela's odd
21st-century gift for taking familiar sounds and repurposing them in a
new context, a kind of lyrical déjà vu.

\textbf{After the show}, back at her Strasbourg Airbnb, Kelela changed
into oversize gray sweatpants and a black button-down crop top, and
padded into the kitchen in white slippers. She plugged in an electric
kettle and made another cup of ginger tea as our conversation turned to
her debut album. I expected her to talk about its sound, but she wanted
to speak about the intention behind it. She described the album as
``something that speaks to power.'' When I asked her to elaborate, she
paused, then said: ``I'm not talking to black people, to be honest. I
don't need to tell black people that R.\&B. is deep. It's about the
music industry. When I say the white gaze, I'm not just talking about
white people at my show. I like that. I like playing to mixed crowds.
For me, it's really just the way I'm treated by people of positions of
power who seemingly hold the key.''

The dynamics that bothered Kelela in college didn't disappear when she
became an artist --- if anything, they intensified. As a student, she
was introduced to black academics and feminists like Audre Lorde, Bell
Hooks and Angela Davis. These women helped her make sense of the racial
and sexist forces that shape the world, and she still turns to them to
navigate the music industry. She internalized their insistence to not be
apologetic for her womanhood or blackness and not be debilitated by
exclusion.

Sprawled on the bed in the apartment's master bedroom, Kelela told me
that Beyoncé's loss at the Grammys depressed her and made her question
the metrics of success for mainstream musicians. ``If that's the highest
we can achieve in that framework, it doesn't even make sense to try,''
she told me. A notebook was open to a blank page on the night stand,
alongside ``B Jenkins,'' a Fred Moten poetry book that explores the
interstitial space among jazz, black aesthetics and politics. Moten's
work concerns itself with the ``resistance of the object'' and the way
that blackness refuses commodification. Kelela is aware of how artists
like her get co-opted, morphed into something symbolic that they no
longer control, and is determined to avoid it.

I had already heard the lengths to which she would go to prevent this
from happening. The first night we met, I asked her how she managed
expectations as an artist in an age of hyperconsumption. I mostly meant
her reserve on social media, despite the disturbingly insistent demands
in her Twitter and Instagram mentions for her next release. Instead, she
described an encounter with Fendi, the Italian luxury brand, which
invited her to perform at its new headquarters in Rome to celebrate the
start of a new website aimed at millennials. Researching the
headquarters online, she discovered that the Palazzo della Civiltà
Italiana, the impressive white marble colosseum that Fendi had taken
over, was built to be a propaganda tool under Benito Mussolini. ``I'm
Ethiopian,'' she reminded me: Mussolini invaded Ethiopia in 1935, just
before World War II. She didn't want her body, her blackness, to be used
by a brand, even a big fashion label like Fendi, as a proxy for cool in
this context. She asked Fendi representatives to agree to release a
statement addressing her concerns as a condition of her involvement.
``That flipped the whole damn script,'' she said. They're still
negotiating the partnership, but the interaction speaks to how Kelela
views her role: ``If an artist says something, people listen,'' she told
me. She sees herself as someone who can wield her status as a celebrity
to catalyze change.

As the evening wound down, Kelela invited me to get comfortable and
listen to some of her new tracks. She gave me earbuds and left me alone
to listen. (When I pressed her about a release date, she made a
coquettish face and demurred, saying the songs were still being mixed.
In reality, she just signed with Warp Records, which will take over the
release of the album.) Earlier in the night, she said that as
politically aware as she feels she is, she didn't make an album that
addresses those views. ``There was a period after I finished all the
songs, I was nervous because I was like: There's nothing about my
experience as a black woman overtly in this. But I could never not make
anything from any other place.'' The songs instead deal with the agony
of falling out of love and the ecstasy of finding it anew. Her voice is
as pretty as ever, rising and crashing like cresting waves over beats
that swing from a druggy drone to throbbing bass lines perfect for
dance-floor grinding. In their own way, they are a quiet protest: They
feel radical in the way a Kerry James Marshall painting or a Ntozake
Shange poem expresses the humanity and beauty of black life. ``All of
that is happening in a world that does not want me,'' she said. ``It is
a safe space for us to feel, and not necessarily for anyone else.''♦

\emph{Jenna Wortham is a staff writer for the magazine.}

Read more

\hypertarget{9}{%
\subsection{9.}\label{9}}

\hypertarget{--this-girl}{%
\subsection{\texorpdfstring{ This
Girl}{  This Girl}}\label{--this-girl}}

\hypertarget{kungs-vs-cookin-on-3-burners}{%
\paragraph{Kungs vs. Cookin' on 3
Burners}\label{kungs-vs-cookin-on-3-burners}}

An earworm that nearly eluded its audience.

By LYDIA KIESLING

\href{https://open.spotify.com/track/1A8j067qyiNwQnZT0bzUpZ}{\includegraphics{https://static01.graylady3jvrrxbe.onion/newsgraphics/2017/03/03/music2017/assets/icons/spotify.svg}Full
Track}

The song, ``This Girl,'' is a result of the (at the time) 19-year-old
French D.J. and producer Valentin Brunel, alias Kungs, trawling YouTube
videos in his bedroom and finding an obscure 2009 soul-revival tune by
the Australian band Cookin' on 3 Burners. Brunel remixed the song,
speeding it up into what he calls a ``happy house track.'' With the aid
of Spotify, the song soon became a viral hit, topping charts across the
globe and leaving no one more surprised than its respective creators.

Kungs's remix was conceived on YouTube, and YouTube also contains a
record of its charms. \href{https://youtu.be/2Y6Nne8RvaA}{The video},
which has been viewed over 200 million times and depicts a summer
romance on a Greek isle, is followed by hundreds of comments from
jubilant global citizens who have finally trapped their earworm. ``This
Girl'' now hovers around No.20 on the Shazam Hall of Fame, meaning that
more than 13 million users of the app were similarly desperate to find
it. For nine weeks, it was the most Shazammed song in the world.

What is it about ``This Girl''? The rich vocals of Kylie Auldist, the
singer for Cookin' on 3 Burners? The retro, cheerful, almost cloying
guitar riff? Cynics might say that Kungs left the heavy lifting to
Auldist and her bandmates, but this diminishes Kungs's inspired
additions: his trumpets and his house beat, crucial to the ecstatic
danciness of the new version. The result is youthful magic, the aural
version of dancing until dawn with a boy you just met. Of smoking
cigarettes on a rooftop all hot summer night.

These days, an enterprising 19-year-old can browse YouTube, find
something that catches his fancy, transform it and broadcast it to the
world. An algorithm can anticipate my taste so precisely that it serves
up the song I've been dying to hear. There's something about the timing:
``This Girl'' is upbeat and joyful and nostalgic when the news has been
a grim parade. The same year the video's young lovers frolicked on the
Mediterranean, some 5,000 people died trying to cross the same body of
water. Our atmosphere is on track to become one long hot summer night.
In harrowing times, this earworm asks little and gives a lot. It's an
echo of better times --- and one you can dance to.♦

\emph{Lydia Kiesling is the editor of The Millions.}

Read more

\hypertarget{10}{%
\subsection{10.}\label{10}}

\hypertarget{--make-them-die-slowly-john-george-haigh}{%
\subsection{\texorpdfstring{ Make Them Die Slowly (John George
Haigh)}{  Make Them Die Slowly (John George Haigh)}}\label{--make-them-die-slowly-john-george-haigh}}

\hypertarget{church-of-misery}{%
\paragraph{Church of Misery}\label{church-of-misery}}

Very good music for very bad feelings.

By JOHN DARNIELLE

\href{https://open.spotify.com/track/08O2pHuPdcJ5PoxFqbpxLO}{\includegraphics{https://static01.graylady3jvrrxbe.onion/newsgraphics/2017/03/03/music2017/assets/icons/spotify.svg}Full
Track}

Sometimes you just want to kill somebody, you know? Really end their
life: make mourners of their friends and family, make orphans of their
children, leave a hole in the world where a person once was. You don't
feel this way all the time --- you're not a monster. But sometimes you
do. Maybe this didn't always happen; maybe you can remember a time when
you not only didn't want to kill anybody but also hoped, moreover, to
never even know what it felt like to want such an awful thing. But you
were a child then, probably, and poets are always making hay about how
you can't get back to that condition in this world. You're not a kid
anymore, and every once in a while --- not every day, except maybe
sometimes for several days in a row --- you want to kill somebody.

Maybe you don't. I don't know. But if you do, when you do, maybe
sometimes it kind of gets away from you, right? Obviously you won't kill
this person, because you can't kill them, because you're not that sort
of person, and besides, you'd get caught, and you're pretty sure you
couldn't hack it in prison. Would you cover your tracks? Try to hide the
body? Go into hiding and hear about yourself on the news? Walk through
the doors of the police station and turn yourself in? You think about
these things when you want to kill somebody. You have the occasional
dream about them. They're not pleasant thoughts, and it's not a pleasant
dream.

What's your damage? How did you get like this? Brain chemistry? Trauma?
Read too many stories about Ozzy Osbourne biting the heads off bats when
you were a kid? It doesn't matter. Two things matter: First, you're not
going to kill anybody, because you are a decent person who wishes no one
any real harm; second, you need to do something a little more creative
with your feelings than just pretending they're not there. For some
people, that means hitting the gym. For others, it means a stereo with a
volume knob.

\includegraphics{https://static01.graylady3jvrrxbe.onion/images/2017/03/12/magazine/12music15-misery/12mag-12misery-t_CA0-master180.jpg}

Heavy metal has been providing people with catharsis for nearly 50
years. There's more to metal than that, of course, which ought to go
without saying --- not every metalhead harbors some hideous inner
sadist. When I listen to death metal, it's not for the splatter-gore
violence of its themes but for the breathtaking proficiency and
relentless musical curiosity of the musicians who play it: To even get
your foot in death metal's door, you have to be awfully good at your
instrument. But Church of Misery doesn't play death metal; it plays
stoner doom (the band itself prefers ``doom,'' without the modifier),
and I don't listen to it for the pleasure of immersion in technical
wizardry. I listen to it because of how it makes me feel.

Who's Church of Misery? Chiefly, they're the brainchild of the bassist
Tatsu Mikami. They've been putting out records since the late 1990s.
They have six studio efforts, numerous EPs and a live album to their
credit, and every song on every album except one takes, as its theme, a
known serial killer. (Their first studio recording was made before they
settled on their thematic focus and wasn't officially released for
years.) Songs indicate their subjects parenthetically: ``Düsseldorf
Monster (Peter Kürten),'' ``Shotgun Boogie (James Oliver Huberty),''
``Red Ripper Blues (Andrei Chikatilo).'' Some of their subjects are
widely infamous --- Charles Manson, Ted Bundy, the Heaven's Gate cult.
Others are so obscure that only true crime buffs are likely to recognize
their names. Look them up at your peril: These are people whose crimes
will give you nightmares.

``Make Them Die Slowly'' is the second song on the band's latest album,
``And Then There Were None ... ,'' and it is impossibly heavy. It begins
with a thudding kick drum all alone, with the central guitar riff
ambling in murderously after two bars --- a figure that lurches
methodically through three five-note patterns to resolve on three
descending chords that land like boulders being dropped on a house. My
iTunes play count shows that I listened to it more than I listened to
any song in 2016 except for drafts of songs I was writing myself. Scott
Carlson of the legendary Repulsion sings it; the 2016 incarnation of the
band was essentially a reboot, with Mikami the only original member.
Carlson's feel for metal forms is instinctive and inerrant; when he
sings the initial chorus --- ``I'll hang you high and drain your
blood/sip on the nectar while your bones reduce to sludge'' --- it's as
if all that stuff about Ozzy had been true: Here's a guy gleefully
recounting obscure, unthinkable real-world crimes as if they were fond
memories of his time in varsity football. It's seductive and
irresistible. It has a cowbell. You can bang your head and sing along.

I have spent a fair bit of idle time over the years wondering what it
says about me that I want to indulge this mood at least a few times a
week for the rest of my life, occasionally at earsplitting volumes in
clubs. I don't think I'm secretly a murderer; I'm a liberal softy. I
can't even eat animals. When I was young, if I heard something that
sounded too celebratory of death, it terrified me. I'll never forget
seeing the clip for ``Bohemian Rhapsody'' on ``Don Kirshner's Rock
Concert'' when I was very young, maybe 9 years old, up too late. I
listened to Freddie Mercury sing the verse about how he'd just killed a
man by shooting him point blank in the head --- although I thought he
was speaking in the third person rather than the vocative, that it was
his ``mama'' who'd just killed a man --- and, horrified, I turned off
the television. Over time, though, that feeling --- the shock, the
revulsion --- became an object of interior curiosity: What's this about?
How much time can I spend with it? What part of me is it? What does it
look like up close?

The cheap answer is something about the cathartic value of
transgression, etc., all old hoary stuff I don't really buy into
anymore. The truer answer, for me, is that sometimes you really wanna
kill somebody. Naturally, you won't. You mustn't. It would be wrong. You
try not to do wrong. But if you spend a little time in the presence of a
perfect groove contemplating the wrong directly without moralizing about
it, you can ride the feeling in safety and go in as deep as you want,
emerging later not wanting to kill anybody. When this has become a
pattern, it's an immense release, as addictive as the pellets they would
give us every so often if we were rats in labs.♦

\emph{John Darnielle is a musician and the author of ``Universal
Harvester.''}

Read more

\hypertarget{11}{%
\subsection{11.}\label{11}}

\hypertarget{--barok-main}{%
\subsection{\texorpdfstring{ Barok
Main}{  Barok Main}}\label{--barok-main}}

Two composers beyond category.

By RYAN BRADLEY

\href{https://open.spotify.com/track/2Dy1MaumORP1kh1sHDXi0w}{\includegraphics{https://static01.graylady3jvrrxbe.onion/newsgraphics/2017/03/03/music2017/assets/icons/spotify.svg}Full
Track}

\hypertarget{mica-levi--oliver-coates}{%
\paragraph{Mica Levi \& Oliver Coates}\label{mica-levi--oliver-coates}}

On a gray winter day in London, the composers Mica Levi and Oliver
Coates met in a high-beamed former hat factory --- the offices of Levi's
manager --- and climbed up to a loft space with an excellent stereo
system. They had agreed to listen to ``Remain Calm,'' an album of
experimental music that, more than a year ago, they improvised over the
course of a single day. Each had moved on to new gigs since the album's
November release, writing new work for different orchestras around
England and Europe, and this was the first time since then that they'd
listened to it together. Coates sat on the edge of a couch; Levi took a
chair; each looked expectant, borderline anxious.

It had been a busy year. Levi is also a producer and D.J. who sings and
plays guitar, most recently in the pop-art trio Micachu and the Shapes.
She has written two movie scores; the second, for the biopic ``Jackie,''
was nominated for an Academy Award. Coates has scored films, too, but is
better known for his work as a cellist. He contributed to Radiohead's
``A Moon Shaped Pool,'' released last May, and in the same month put out
his own ``Upstepping,'' a very danceable record of house music performed
almost entirely on his cello.

\includegraphics{https://static01.graylady3jvrrxbe.onion/images/2017/03/12/magazine/12music4-LEVI/12music4-LEVI-master180-v2.jpg}

``Remain Calm'' is an odd creation, even by their standards. Its 13
tracks, some less than a minute in length, jump from beat-heavy, densely
layered and looped orchestrations to atmospheric and spacey noodlings.
Often, Coates's cello seems to emerge, pure and acoustic, out of a stew
of sonic burbles. It is a sketchbook in which every figure gestures
toward newer, more exciting ideas to come, outlining musical rules (a
key, a beat, a melody) one minute only to abandon them in the next.

Before they listened to the record, Coates reached into his backpack and
pulled out a coloring book. He showed Levi one of the images he had
colored in, a mandala filled with bright blues and greens, thin wisps of
gold, bursts of coral pink. Levi leaned in for a closer look, drawing
her finger across the page. ``Bruv, this is nuts,'' she said.

``It's what I wish composing was,'' he said. ``The form preordained.''
Levi nodded in agreement. It was a visual cantus firmus, she said: a
fixed melody providing a structure for a limited range of improvisation.
They had each studied classical composition: Coates at the Royal Academy
of Music, where he received the highest marks in the school's history,
Levi at Guildhall School of Music and Drama, where, at 21, she was
commissioned to compose a piece for the London Philharmonic.

The first song began, Coates playing a solo figure, then Levi's
electronic textures trickling in, like an orchestra tuning up. The pair
sat in silence, pleased enough but also distracted. A few tracks later,
Coates looked up at Levi, who was looking at his mandala. ``I'd like for
it to be music,'' he said. It seemed like a familiar conversation. Levi
massaged her temples, thinking, listening. ``You just get a set of
rules,'' she said. ``Each color could be a different pitch.'' Coates
frowned. The idea of ``algorithmic music'' troubled him. Maybe, Levi
said, you set up the rules and then find a way to break them; color
inside the lines, so to speak, and then scribble a face over the
results. Coates liked it. The breaking of the rule, he said, is ``what
makes a moment, musically.''

In fact, he added, his coloring was loaded with mistakes already, but
the mistakes were what made the thing come together, in a subtle way. He
turned the page, exposing the blots where the pen ink bled through to
the other side and the sharp lines of the pattern were barely visible.
He said they reminded him of one of Levi's works.

Coates and Levi met almost a decade ago. Coates had come to perform
student string quartets for a class Levi was taking, and he was struck
by her compositions. They were written so well, he said, ``that you knew
immediately what you needed to do to make them come to life.'' From then
on, the two stayed in touch. Coates sent Levi a video by the electronic
producer Daniel Lopatin, also known as Oneohtrix Point Never; Levi sent
Coates a mixtape she made with some tracks by Harry Partch, a composer
who created new musical scales and built his own instruments.

In 2011, Coates was working on a commission for the London Sinfonietta
and suggested a collaboration with Micachu and the Shapes, who were
already trying to, in Levi's words, ``smudge a whole lot of things
together.'' For the Sinfonietta collaboration, Levi and her bandmates,
following in Partch's footsteps, deployed a new instrument she had built
with a friend called the chopper --- a hollow box with strings on top
and a rotating wheel to pluck them, like a turntable that strummed. The
live album that resulted, filled with wobbly, off-kilter strings, was
called ``Chopped and Screwed'' (a nod to screw music, the Houston
hip-hop style). It landed in the hands of the filmmaker Jonathan Glazer,
who was looking for someone to score ``Under the Skin,'' his film about
an alien exploring earth in a woman's body. He wanted an experienced
composer who had never written music for a movie, someone who would come
at the task differently. ``I think he wanted someone who was, you know,
cheap,'' Levi said.

\hypertarget{-2}{%
\subsection{}\label{-2}}

\hypertarget{-3}{%
\paragraph{}\label{-3}}

For 10 months, she worked on almost nothing else, worried that if she
listened to anything --- particularly another soundtrack --- she would
unintentionally steal from it. She focused instead on the challenge of
scoring the emotional development of the film's unknowable alien, played
by Scarlett Johansson. Levi's stark themes all spring from a dense void,
abuzz in discordant strings (including Coates on cello), then synthetic
and hollow with muffled drums and distant drones. As the film
progresses, though, the music becomes more melodic, reflecting the alien
creature's struggle with humanity, with the untidiness of human
emotions. The soundtrack is unsettling, but also strangely empathetic.

Levi describes much of her work as mixtapes. For ``Jackie,'' she
imagined classy melancholy, loads of strings, the occasional flute,
clarinet and piano --- the kind of music that Jacqueline Kennedy might
listen to, a mixtape for private mourning made for a woman in the center
of so much public chaos. The day Levi and Coates recorded ``Remain
Calm,'' Levi arrived with two CD turntables and thumb drives full of
samples and premade digital files --- all of which, in one way or
another, had reminded her of Coates. She was thinking of music not in
terms of classical or hip-hop or any other genre, but in terms of
people. Some music was Oliver Coates music. Some music was Mica Levi
music.

Listening to ``Barok Main,'' one of the longest tracks on ``Remain
Calm,'' Coates and Levi agreed that it was also probably the most
unified piece. ``This was a nice peaceful moment, just after lunch,''
Coates said, recalling the rainy day they recorded it.

Levi said: ``I was thoughtful of what could fit here. Something that
wasn't too thick, too full of information.''

They listened a bit longer.

The track swelled and echoed, and it was hard to know where Coates's
cello stopped and Levi's sound began. ``It's just happy music making,''
Coates said, and they sat, listening silently, waiting for their track
to end.♦

\emph{Ryan Bradley is a writer based in Los Angeles.}

Read more

\hypertarget{12}{%
\subsection{12.}\label{12}}

\hypertarget{--mourn-at-night}{%
\subsection{\texorpdfstring{ Mourn at
Night}{  Mourn at Night}}\label{--mourn-at-night}}

Putting out fires by day, music by night.

By JODY ROSEN

\href{https://open.spotify.com/track/6zao3czBFKk1zbmIe3azJd}{\includegraphics{https://static01.graylady3jvrrxbe.onion/newsgraphics/2017/03/03/music2017/assets/icons/spotify.svg}Full
Track}

\hypertarget{ka}{%
\paragraph{Ka}\label{ka}}

If you buy a record on brownsvilleka.com, the website of the rapper Ka,
it is Ka himself who packs up your purchase and puts it in the mail.
Every few days, Ka sits in a study in his home near Prospect Park in
Brooklyn and goes through the orders on his site. He was there on a
morning not long ago, with a MacBook propped on his knees. ``This guy,
from California, he bought the whole discography,'' Ka said. On the
floor were cardboard boxes holding copies of his five full-length
albums. He placed five CDs in a padded envelope. ``I'll try to get those
out tomorrow,'' he said. A message on the website warns: Please have
patience with your order, I don't go to the post office every day.

There was a time when Ka took a guerrilla approach to promoting his
music. In 2007, he had 1,000 CDs made of his debut album, ``Iron
Works.'' ``I gave them to my cousins, my friends. I still had, like, 990
CDs left. So I started giving them away. I'd drive around the city, and
if I heard music coming from the next car at a red light ---
boom-boom-boom --- I'd say, `You like hip-hop?' Then I'd toss a CD in
their window.'' By 2012, when Ka released his second album, ``Grief
Pedigree,'' he had a sizable-enough following to make an announcement on
Twitter and hold a curbside sale in Greenwich Village. This has become a
tradition: On the day that Ka drops a new album, he tweets, turns up on
a street corner and sells a few dozen records out of the trunk of his
car.

\includegraphics{https://static01.graylady3jvrrxbe.onion/images/2017/03/12/magazine/12music1-KA/12music1-KA-master180-v2.jpg}

When music writers use the word ``indie,'' they generally have a certain
kind of recording artist in mind: a white bohemian, with a battered
guitar and a tricky haircut, who dwells on the distant outskirts of the
pop-industrial complex. Ka, whose given name is Kaseem Ryan, doesn't
quite fit that description, but he may be the platonic ideal of an
independent musician in 2017. It would be hard to find a more
thoroughgoing D.I.Y. musical enterprise. Ka is the rare rapper who
handles both rhymes and beats, writing his lyrics and producing the
music that accompanies them. He has directed most of his videos, and he
self-releases his music, on his own label. It is not a profitable
venture. Ka spends as much as \$15,000 per album and has never broken
even. ``I make music, but I'm not in the music business,'' he says.
``It's my hobby.''

``It's my hobby'' is one of Ka's mantras. It's accurate, but also an
understatement. Over the past several years, Ka has released some of the
most gripping music in any genre. His records offer a poignant,
distinctive take on classic New York hip-hop: vivid stories of street
life and struggle narrated in virtuosic rhymes over music of bleak
beauty. His output has won him a small but passionate fan base and
critical raves in Pitchfork and Spin. The magazine The Fader, an
influential arbiter of musical cool,
\href{http://www.thefader.com/2016/10/04/ka-interview-new-york-post}{proclaimed
Ka} ``New York rap's greatest living treasure.'' Praise has also come
from well-known musicians. In 2015, the Los Angeles M.C. Earl Sweatshirt
took to Twitter to exult in caps-lock mode: ``REAL TALK, I HAVE NO
PROBLEM ADMITTING THAT @BrownsvilleKa IS THE BEST RAPPER.''

For Ka to have won even modest recognition is an improbable underdog
triumph. He spent much of the 1990s trying to make it as a rapper, quit
music altogether and returned a decade later, releasing his solo debut
at age 35. Today he is 44. This career trajectory defies one of the
seemingly immutable laws of pop, and of hip-hop in particular, a genre
in which the cult of youth and novelty is especially pronounced.

Then there is the matter of Ka's day job. He is a captain in the New
York City Fire Department. It's a story that has the ring of folklore:
One of the great recording artists of the current decade is a musical
moonlighter, a middle-aged man who earns his living as the decorated
company commander of Engine 235 in Bedford-Stuyvesant. ``I try to keep
my job and music separate,'' Ka says. ``I never wanted be `The Rapping
Captain.' I try to be a good firefighter. And when I come home, I try to
make some dope music.''

Last Aug. 21, Ka's photo appeared on the cover of
\href{http://nypost.com/2016/08/21/fdny-veteran-bad-mouths-cops-in-double-life-as-rapper/}{The
New York Post} under a lurid headline: ``FLAME THROWER: FDNY Captain
moonlights as anti-cop rapper.'' The article purported to be an exposé,
but the facts of what The Post termed Ka's ``double life'' have long
been known to his Fire Department colleagues and to readers of the music
press. Ka put it succinctly: ``The Post don't like rap. They don't like
firemen. And they don't like black people. With me, they had all
three.''

In fact, Ka's music bears little resemblance to the ``gangsta rap''
caricature of tabloid fulminations. In Ka's songs, crime and violence
are omnipresent but deglamorized. It's here that Ka's age shows: He raps
in the voice of a world-weary O.G., looking back at ``the hustle'' with
regret and dismay, and reckoning with its spiritual toll. Ka's lyrics
are cinematic, training a grit-caked lens on scenes of poverty and
desperation. In ``No Downtime'' (2012), he raps: ``I admit, not from
environment that let a child flourish/In the street, slim physique, wild
courage/Five deep, trying to eat, looking malnourished.''

\includegraphics{https://static01.graylady3jvrrxbe.onion/images/2017/03/12/magazine/12music2-KA/12music2-KA-master180-v2.jpg}

These stories aren't fictional. Ka grew up poor, in Brownsville. As a
teenager, he drifted into the drug trade, dealing crack and selling
firearms. It was a ``scary and lonely'' existence, Ka says, one that
left a legacy of trauma and a wealth of material, a burden that's also a
muse. ``This man has experienced so much pain, seen so much death,''
says Ka's wife, Mimi Valdés. ``It's all over his music.''

If Ka is not in the music business, his wife definitively is. Valdés is
the former editor in chief of Vibe, the music magazine. Today she is
chief creative officer for i am OTHER, a multimedia company founded by
Pharrell Williams, the superstar rapper-singer-producer. Valdés's
professional circle includes some of the world's most famous musicians.
It is conceivable that Ka could trade on his wife's connections to raise
his profile. But a commercial breakthrough is far-fetched, and a
prospect for which Ka seems constitutionally ill equipped. He has
performed just a few live shows and professes little interest in playing
more. He's a homebody who prefers to sit in a closet-size work space and
make records.

Those records are, in the best sense, strange. Critics often call Ka a
revivalist of rap's '90s golden age. But he is by no means ``retro.'' He
takes the rudiments of '90s rap and fashions them into an eccentric
other thing. His albums have idiosyncratic conceptual frames: ``The
Night's Gambit'' (2013) is full of chess imagery; ``Days With Dr. Yen
Lo'' (2015), a collaboration with the producer Preservation, is an
elaborate gloss on the novel ``The Manchurian Candidate.''

But it is the sound, the vibe, that sets Ka's music apart. His songs are
unnervingly quiet and still; they hold a listener in thrall because they
hold so much back. Ka's beats do not boom, they slither and slink and
creep. Often the songs discard drums altogether, opening vast spaces
that are filled by samples in brooding minor keys. The most arresting
sound is Ka's voice, a guttural rasp that sometimes dips into a whisper.
It is an unshakable voice of experience, delivering hard-boiled tales
and hard-won wisdom.

\hypertarget{-4}{%
\subsection{}\label{-4}}

\hypertarget{-5}{%
\paragraph{}\label{-5}}

``Mourn at Night,'' from the 2016 album ``Honor Killed the Samurai,'' is
a typical Ka mood piece --- a rap-noir, burnished to lustrous dark hue.
The song's title establishes the terms: The mood is elegiac, the setting
nocturnal. ``Mourn at Night'' has more of a beat than many Ka songs,
which is not to say that it has much of one. A snare drum knocks out a
steady pulse, like the sound of an unseen man's footfalls trailing down
a low-lit street, or a time bomb ticking to an explosion. The music is
nudged forward by guitar and keyboard arpeggios, which chime and toll
behind a somber refrain: ``He gone/We mourn at night.'' In the final
verse, the action erupts: Gunfire blasts out of the shadows, strafing a
street corner crowded with young drug dealers.

\begin{quote}
On the block with the rock

Up nights, dribbling

Time spent getting bent

Hard to be upright citizen

It's more than troubling

Same corners are bubbling

The pawns are huddling

Then all of a sudden it's

\emph{Pop, pop, pop}
\end{quote}

Ka excels at this kind of writing, brisk storytelling that unfolds in a
pileup of rhymes and puns. What you won't find much of on his records is
the meat-and-potatoes of rap: trash talk. ``What do I have to brag
about?'' Ka says. ``I ain't winning. I'm a working man, with a
blue-collar job. I'm running into burning buildings for a living. So I
speak about the things that I did, the things I pray I never have to do
again. How do I finish my life in grace? That's what I rap about.''

On those occasions when Ka does slip in a mild boast, it's about his
skills as an M.C. In ``Mourn at Night'' he permits himself one such
moment, the testimony of a bootstrapper who has survived untold trials
to ply his trade --- as he would have it, his hobby --- in the
soundproof booth of a recording studio. ``We was born in the thorns, few
arose,'' Ka raps. ``Once in town's noose/Now in soundproofs/Pursuing
goals.''♦

\emph{Jody Rosen is a contributing writer for the magazine}.

Read more

\hypertarget{13}{%
\subsection{13.}\label{13}}

\hypertarget{--hold-my-mule}{%
\subsection{\texorpdfstring{ Hold My
Mule}{  Hold My Mule}}\label{--hold-my-mule}}

\hypertarget{shirley-caesar}{%
\paragraph{Shirley Caesar}\label{shirley-caesar}}

When gospel goes viral --- with an abundance of pure joy.

By GREG HOWARD

\href{https://open.spotify.com/track/0JjsI172ped2hF5v27o28J}{\includegraphics{https://static01.graylady3jvrrxbe.onion/newsgraphics/2017/03/03/music2017/assets/icons/spotify.svg}Full
Track}

It was 29 years ago that the legendary gospel artist Shirley Caesar
first recorded the song ``Hold My Mule.'' It was a banger then, and it
still is. She has performed it many times, and at least once, 10 years
ago, someone filmed her in a church. About midway through the
nine-and-a-half-minute video, the band and the organ, which riff all the
way through, fall quiet. Caesar hunches over, gazes into the distance
and, playing the role of the song's protagonist, Shouting John --- a
farmer who has been reprimanded by the deacons at his local church for
celebrating too much, dancing too wildly, speaking too loudly (and
sometimes in tongues) --- enumerates the many blessings God has bestowed
on him and his farm. She counts each blessing off on her fingers and
raps: ``I've got beans, greens, potatoes, tomatoes, lambs, rams, hogs,
dogs, chickens, turkeys, rabbits'' --- and then she shouts, ``you name
it!''

The band kicks in again, and a slew of sonic histrionics, pyrotechnics
and acrobatics follows. But that list is the climax, the moment when
Caesar basks in the warmth of God's grace, when she and you both feel
closest to him.

This was the moment that stood out to a musician called DJ Suede the
Remix God, who just before Thanksgiving took that snippet --- just eight
seconds in all --- and
\href{https://www.youtube.com/watch?v=UR8p9z8AK0Q}{laid it over a
trap-style hip-hop} beat of his own making. The story of Shouting John
and his fussy deacons disappeared, leaving behind nothing but that list
of blessings and Caesar's euphoric ``you name it!'' completely stripped
of context.

\includegraphics{https://static01.graylady3jvrrxbe.onion/images/2017/03/12/magazine/12music16-caesar/12mag-12caesar-t_CA0-master180.jpg}

Suede then offered the beat to the internet, calling it the U Name It
Challenge and inviting others to put their own spins on it. The singer
Chris Brown recorded \href{https://youtu.be/FKBnETEiRVk}{a video}
dancing to it. Countless other dancers and rappers followed him. The
challenge went megaviral. The trick was that the snippet Suede chose had
Caesar talking about food --- and about giving thanks for that food ---
convincingly, joyously and at the exact right time of year. Her ecstatic
cry made it universal.

The best came a few days before Thanksgiving, when an Atlanta-based
rapper named Grey responded to Suede's challenge with
\href{https://www.instagram.com/p/BMzFwj9jQyP/}{a verse of his own}.
Grey is a vegan, unlike Shouting John, but a quick jaunt through his
social media identifies him as every bit the evangelist of his
philosophy that Caesar is for Christ. From inside a parked car, Grey
unleashed a verse describing the Thanksgiving feast he was preparing for
his friends and family: ``Jump into the whip and hit Whole Foods
early/Get everything to get that soul food working.'' There would be
vegan cornbread, vegan stuffing, ``greens with no ham in it,'' pies
whipped up with almond milk, ``mac and cheese to collard greens to
black-eyed peas/Get a roast of that tofu turkey.'' Then the sample of
Caesar's list, with a little amendment from Grey: He endorsed the
``beans, greens, potatoes, tomatoes'' part, then noted that he wasn't
messing with the ``lambs, rams, hogs, dogs.'' It's almost certainly the
best and funniest verse of his life.

While taking a gospel sample and warping it into a trap song not about
Christ but about the wonders of soy and fake meat scrubs Caesar's joy of
its context, the joy embedded in her voice still translates, is still
familiar. I'm not religious. But I grew up going to church, dancing and
singing to raucous gospel bands and choirs nearly every Sunday. Once,
after a particularly rousing concert, I walked from my seat to the front
of the auditorium to be baptized and join the church, only to come to my
senses once I got to the altar. I never truly believed in Caesar's God,
but I was swept away as so many nonbelievers can be, in flashes of
euphoria ignited by family and tofu and trap and everything else that
sustains us.♦

\emph{Greg Howard is a David Carr fellow for The New York Times.}

Read more

\hypertarget{14}{%
\subsection{14.}\label{14}}

\hypertarget{--a-womans-face--reprise-sonnet-20}{%
\subsection{\texorpdfstring{ A Woman's Face --- Reprise (Sonnet
20)}{  A Woman's Face --- Reprise (Sonnet 20)}}\label{--a-womans-face--reprise-sonnet-20}}

\hypertarget{rufus-wainwright}{%
\paragraph{Rufus Wainwright}\label{rufus-wainwright}}

He soothes the anxieties of life as a gay man.

By ALEXANDER CHEE

\href{https://open.spotify.com/track/5D12h7kioqQdCKzoDiUfvJ}{\includegraphics{https://static01.graylady3jvrrxbe.onion/newsgraphics/2017/03/03/music2017/assets/icons/spotify.svg}Full
Track}

For more than a year now, I have listened to little else in my car other
than the albums of Rufus Wainwright. This obsession began when my
husband and I bought a car for weekend trips: a 2009 AWD Subaru
hatchback with what in retrospect seems like an ancient playback
machine, a 5-disc CD player. We pulled all our old CD wallets out to the
car, loaded the changer and set off for our first drive. And what had
been my self-imposed exile from music came to an end that day.

Sometime between 2009 and 2015, without ever noticing it was happening,
I stopped listening to music regularly. I was alienated, in part, by the
new ways of listening: I don't like streaming music, or ``social'' music
platforms, my private whims made public. Around the same time, my doctor
told me I had mild depression, which would respond to exercise and a
change in habits. But this mild depression did not feel mild. I felt
trapped at the bottom of a swimming pool, immobilized. Everything I had
to do, everything I needed to take care of, was up at the surface, and
the soundtrack to this situation was silence.

During one of our first drives, we loaded the changer solely with Rufus
Wainwright albums, and I still haven't changed those CDs out. I was
always a fan of Wainwright's, but I'm something more now: His
lighthearted way of penetrating my anxieties about gay life became
essential. In my year of carbound listening, I played through his
songbook over and over. I gained a new appreciation for his
extraordinary voice, and the way its nasal timbre humanizes him, as if
someone ordinary had been given extraordinary powers, midnote. I am
often tricked into trying to sing along, but I'm always outclassed. I
never care. Wainwright is a storyteller, and his albums work on my
imagination the same way short-story collections do --- poetically,
dramatically. Singing along with his secrets became like telling mine to
myself, and somehow, this helped me up from the bottom of that pool.

A change of habits, then.

\includegraphics{https://static01.graylady3jvrrxbe.onion/images/2017/03/12/magazine/12music-rufuswainwright/12music-wainwright-master180.jpg}

Joan Didion has written of how the ability to make a note ``when
something came to mind'' was the ``difference between being able to
write and not being able to write'' --- a sign of life. For me, looking
for new music from favorite musicians is also a sign of life. This fall,
I finally thought to look for a new Wainwright album. What I found was
``Take All My Loves,'' released early last year. Wainwright adapts nine
Shakespearean sonnets into a kind of variety show, with William Shatner;
Helena Bonham Carter; Carrie Fisher; Florence Welch; the soprano Anna
Prohaska; his sister, Martha Wainwright; the BBC Orchestra; and the
Berlin String Section. He is on the cover in full Queen Elizabeth I
drag, flowers in his hair. My favorite track is ``A Woman's Face ---
Reprise (Sonnet 20),'' which Wainwright first interpreted in 2010 on
``All Days Are Nights: Songs for Lulu.'' That first version is a
would-be lover's complaint, sung softly over slow piano; it's
elliptical, haunting, slight. A new version on ``Take All My Loves'' is
almost a torch song --- teasing, raunchy, openly seductive, as if
Wainwright were seated on a piano in a slinky gown and Elizabethan
collar, all made up and beckoning.

The poem is, famously, the locus of academic debates over Shakespeare's
queerness. His misogyny too. The sonnet is about an impossible love for
a young man who the poet claims has all the beauty of a woman, but for
the ``one thing.'' I had read it, but I never really understood it the
way I did when I heard Wainwright sing it: It's a refrain many gay men
know from waiting out the passion of a man who doesn't ``think of
himself that way,'' despite his feelings for you. ``If only you were a
girl'' becomes ``You're better than a girl'' before either fear or
desire wins.

Wainwright knows this territory well. His second album, ``Poses,'' was
dedicated to one of these reluctant, not-quite-queer lovers, stories
from when Wainwright was the kind of seductive youth the sonnet
describes. Now, for the duration of this song, he is the reluctant
lover.

As a storyteller, Wainwright has always been more of a memoirist than a
novelist. He made his reputation singing wise songs of impossible loves
and rejection, turning personal pain into public art. Wainwright is now
married to Jörn Weisbrodt, an arts impresario, and he has a daughter
with his friend Lorca Cohen, a photographer and the daughter of Leonard
Cohen. Listening to ``A Woman's Face,'' I wonder if happiness has made
him more private, happier telling fictional stories than true ones. His
next project is an opera about the Roman emperor Hadrian, who, out of
grief at the death of his lover Antinous, created a religion around him.

I intend to be there when it opens, wearing a tuxedo in an opera box ---
no Subaru this time. I'll try not to sing along.♦

\emph{Alexander Chee is the author of the novels ``Edinburgh'' and ``The
Queen of the Night.''}

Read more

\hypertarget{15}{%
\subsection{15.}\label{15}}

\hypertarget{--copper-canteen}{%
\subsection{\texorpdfstring{ Copper
Canteen}{  Copper Canteen}}\label{--copper-canteen}}

The danger of being a singer-songwriter.

By RUTH GRAHAM

\href{https://open.spotify.com/track/2jG8qg7KDYWKm9smYCXz1i}{\includegraphics{https://static01.graylady3jvrrxbe.onion/newsgraphics/2017/03/03/music2017/assets/icons/spotify.svg}Full
Track}

\hypertarget{james-mcmurtry}{%
\paragraph{James McMurtry}\label{james-mcmurtry}}

Fifteen minutes after finishing an acoustic concert one evening in
January, the Texas singer-songwriter James McMurtry was backstage
talking about his great-grandfather. The man had lost his own father and
grandfather to post-Civil War skirmishes in Missouri, McMurtry said, so
he and his wife fled the state. They settled in Denton County, Tex.,
40-some miles northwest of the renovated Art Deco theater in Dallas that
their great-grandson headlined that night. ``They had to time it just
right, because if you went West at that time, you could get into more
violence because the Comanche were still active,'' he said. ``They
farmed there for five or six years, waiting to make sure the Comanche
weren't coming back.''

The story was a good example of how forces like politics and war can
affect regular people's lives, which is a running theme in McMurtry's
work. McMurtry, who has released 12 albums over a 28-year career, has a
reputation in some quarters as a political songwriter, in part because
one of his most popular songs is an angry-lefty anthem. That song, ``We
Can't Make It Here Anymore,'' laments that minimum wage ``won't pay for
a roof, won't pay for a drink,'' and that the children of the poor are
the ones who end up fighting in rich men's wars. Released shortly before
the 2004 election, the song swept through an America hollowed out by
departed manufacturing jobs and the middle-class stability that went
with them. A few years after its release, the critic Robert Christgau
named it the best song of the decade.

But McMurtry more often writes about how seemingly distant political
concerns nudge his characters' choices and prod at their psyches: the
stretched budget of the Veterans Affairs Department, or the birth of a
new national park's consuming the neighbors' land through eminent
domain. In ``Sixty Acres,'' the narrator laments that when his
grandmother died, he inherited a plot of unpromising farmland while his
cousin got ``the good land,'' zoned commercial: ``Looks like a Walmart
waiting to happen/I mean to tell you it's a pot of gold.''

\includegraphics{https://static01.graylady3jvrrxbe.onion/images/2017/03/12/magazine/12music7-MCMURTRY/12music7-MCMURTRY-master180-v2.jpg}

McMurtry's father is the great Texas novelist Larry McMurtry, and his
mother, Jo Scott McMurtry, is a former English professor specializing in
Shakespeare. McMurtry calls his parents ``first-generation-off-the-farm
academics.'' He was raised in secular urban homes and even attended
boarding school for a spell. But he dropped out of the University of
Arizona, and he remains fluent in his extended family's dialect of
ranches, oil fields and Jesus Christ. He has been on tour almost
constantly since the late 1980s, and he just takes note of what he sees
through the windshield, he said, like banners welcoming home soldiers in
small towns.

At an upscale barbecue restaurant near his hotel in Dallas, where we met
before his concert, our talk turned to tribalism and
anti-intellectualism. His grandparents wanted their children to go to
college to get better jobs, he said, but they didn't want them to become
intellectuals. Later he referred to the broader middle-American ``hatred
of anyone who is perceived to be getting a free ride of any kind'' and
the racism woven into that ethos: ``The image of the black welfare queen
driving the Cadillac while good white folk drove Oldsmobiles was the
boogeyman of the '80s. Now I suppose it's Mexican immigrants getting
free health care.'' Over the course of an hour, he talked about the
military-industrial complex, ``embedded'' war journalism and the role of
charisma in electoral politics. I didn't know whether to believe him
when he said he didn't read many books.

McMurtry had ordered black coffee and a plate of fried oysters. In a few
hours, he would take the stage alone with his guitar, and in a few
weeks, he and his band would leave for a European tour that would carry
them from Ireland to Italy, playing 33 nights in a row. McMurtry's music
is usually classified as ``Americana''; it's at turns jagged and
raucous, and at others deliberative and wistful. Other singers have
smoother voices. But a movie star's perfect face can work against his
authenticity as an actor, and McMurtry's unassuming vocal style and
stage presence bring to the foreground the voices populating his songs.
``There's a danger in being a singer-songwriter,'' he said. ``You're
writing a character's point of view, and you've got to sing it yourself.
It comes out of your voice, so everybody thinks it's your opinion.''

Although I've been listening to him for years, I have begun to think of
him as interpreter of the places ``out here in the middle,'' as he puts
it one song. He has written about Cheyenne, Wyo.; Tulsa, Okla.; and
Wewahitchka, Fla; about crab fishermen, soldiers and Walmart stockers.
His songs tap into resentments about things like coastal attitudes of
superiority and political correctness. His narrators are often white men
who know the Bible, own guns and give their kids a nip of vodka in their
Cherry Coke to get through long road trips. A Texan friend of mine likes
to say that McMurtry writes as though he has spent time eavesdropping on
conversations in every Dairy Queen in America. Stephen King, who owns a
classic-rock station in Maine, has written that he ``may be the truest,
fiercest songwriter of his generation.''

McMurtry has seen things change in rural America over the last few
decades, he said, the curdling of patriotism and self-reliance into
something uglier. Gun ownership, for example, has become an identity, or
even like a cult. He's a hunter --- deer, turkey, wild hogs --- and he
said gun shows and shops used to be friendly spaces. But now there's a
suspicion in the air if ``you don't have the right hat or the right
haircut.'' McMurtry himself has both, but he also has the kind of
detachment required for skeptical reportage.

\hypertarget{-6}{%
\subsection{}\label{-6}}

\hypertarget{-7}{%
\paragraph{}\label{-7}}

``Copper Canteen,'' the opening track on his 2015 album, ``Complicated
Game,'' is the song I've returned to most since Election Day. The
narrator is a hunter, a fisherman and a small-business owner. He doesn't
go to church, but his wife does. Although retirement is in sight, and he
has a pension, he hasn't been able to save as much money as he would
like, in part because the store he owns is getting squeezed by ``the big
boxes out on the bypass.''

But like most of McMurtry's best songs, ``Copper Canteen'' eventually
resolves into a portrait of a relationship. The long marriages and old
love affairs he favors eschew the hot and cold dramatics of contemporary
country music, a world in which you're either swooning over your one
true love or bashing in your ex's windshield with a baseball bat. The
``Copper Canteen'' narrator is singing to his wife; they're grandparents
now, reflecting back on what it meant to ``grow up hard'' in a life that
has moved by in a blur: ``This life that we craved so little we saved
between the grandparents' graves and the grandchildren's toys.'' Like
many McMurtry characters, he sounds nostalgic even though he knows that
the past was often bad. The song opens with a bang, or the suggestion of
one: ``Honey, don't you go yellin' at me while I'm cleaning my gun,'' he
drawls. ``I'll wash the blood off the tailgate when deer season's
done.'' Defying Chekhov, the gun never goes off, and the song sweetens
--- barely --- as it moves along. Though that line about the gun got a
big laugh when McMurtry played it in Dallas, I still don't know whether
to hear it as a joke or a threat, and McMurtry has never been one to
offer the easy comfort of a straight answer.♦

\emph{Ruth Graham is a contributing writer for Slate.}

Read more

\hypertarget{16}{%
\subsection{16.}\label{16}}

\hypertarget{--fubu}{%
\subsection{\texorpdfstring{ F.U.B.U.}{  F.U.B.U.}}\label{--fubu}}

\hypertarget{solange}{%
\paragraph{Solange}\label{solange}}

How a song can make its audience feel seen.

By ANGELA FLOURNOY

\href{https://open.spotify.com/track/2aR6SYRJxy2QsBxA459ATF}{\includegraphics{https://static01.graylady3jvrrxbe.onion/newsgraphics/2017/03/03/music2017/assets/icons/spotify.svg}Full
Track}

A song should not shut a door in a listener's face. Of course, many
songs do drive listeners away --- through objectionable lyrics, a
vocalist's nasal intonation, a grating hook --- but generally not on
purpose. Still rarer is a song that identifies its audience in explicit,
demographic terms. But in the first line of Solange Knowles's song
``F.U.B.U.,'' she welcomes black listeners --- ``one for us,'' she
proclaims --- and she shows white listeners the door.

Once this metaphorical point of entry is closed, an important
conversation ensues. ``All my niggas in the whole wide world/Made this
song to make it all y'all's turn,'' Knowles sings in the opening chorus.
The word ``nigga'' (along with its ``-er'' counterpart) is so fraught
with negative history that the N.A.A.C.P. held a funeral for it in 2007.
Not every black person can hear it in a song and feel the exultation
that Knowles intends. Still, words take their own sweet time to die, and
right now, ``nigga'' remains a go-to, often-affectionate phrase in many
black circles. In ``F.U.B.U.'' Knowles uses it to draw a line in the
sand: If you aren't supposed to say it, then this song isn't for you.
``For us,'' she sings --- ``this {[}expletive{]} is for us.''

The song's title references the clothing line of the same name ---
F.U.B.U., For Us By Us. At its height in the late '90s, F.U.B.U. was
synonymous with black cool: I wore it in high school despite the clothes
being impractical for my too-tall, too-thin frame. And the song appears
on Knowles's recent album, ``A Seat at the Table,'' after an interlude
featuring the rapper Master P., whose stories of success as an
independent music artist and executive help provide thematic shape to
the project. For many of us who were young and black, or young and black
and Southern, like Knowles, during the years when he dominated the rap
charts, his story has always been inspirational. ``If you don't
understand my record, you don't understand me, so this is not for you,''
he says.

Songs that can be described as ``black-empowerment music'' make up a
genre as diverse as that of black music itself. There are the
time-tested tracks that focus on pride, like James Brown's ``Say It Loud
--- I'm Black and I'm Proud,'' and Nina Simone's ``Young, Gifted and
Black.'' There are anthems that speak to struggle and protest, like Bob
Marley's ``Get Up Stand Up'' and Kendrick Lamar's ``Alright.'' And there
are a plethora of doggedly sunny ditties like Nas's ``I Can'' and
Jaheim's ``Fabulous.'' Knowles's ``F.U.B.U.'' hews closer to the first
two categories, and a majority of its messages of pride and protest are
nuanced, centering the individual over the collective. She allows that
not every black person will relate to the experiences she recounts:
``When you feeling all alone/And you can't even be you up in your
home/When you even feeling it from your own.'' The song pushes past one
litmus test for the black-empowerment anthem --- that it present itself
as made for black listeners, broadly --- and tries to reach an even more
elusive audience: the black listener who maybe hasn't felt included in
such messages before.

Art finds who it finds, and the white gaze lands where it lands. The
more you try to ignore it, the more it seems keen on dissecting you.
Knowles is aware of this. The deliberate rejection of white scrutiny is
part of a long tradition of black art-making. In his famous 1926 essay,
``The Negro Artist and the Racial Mountain,'' which would serve as the
young poet's artistic manifesto, Langston Hughes declared that younger
black artists should create ``without fear or shame. If white people are
pleased, we are glad. If they are not, it doesn't matter.'' It's the
sort of thing emerging black writers --- me included --- have to remind
themselves when putting pen to paper, lest we succumb to pandering for
white attention.

In ``F.U.B.U.'' Knowles doesn't just eschew the white gaze for the sake
of creating her art; she performs that rejection in song, to pull her
black listeners close. It's a rhetorical invitation for black folks to
grab a chair and settle in for real talk about lives like theirs. To her
white listeners (who are listening, after all), she offers a
tongue-in-cheek indictment as consolation: ``Don't feel bad if you can't
sing along/Just be glad you got the whole wide world.''

The verses in ``F.U.B.U.'' outline microaggressions, but the song isn't
``about'' microaggression; it's about recognition. ``When a nigga tryna
board the plane/And they ask you, `What's your name again?'/'Cause they
thinking, Yeah, you're all the same,'' Knowles sings. This is the sonic
equivalent of shaking out your hair after long hours of wearing it
pulled back and tied down for work, putting on your sweatpants and
calling your girlfriend to tell her about the day you had. Between black
folks, recounting microaggressions can offer valuable confirmation that
you're not crazy, that someone else sees you. It's commiseration, but
without the misery, because the slights are both hurtful and numbingly
commonplace. Knowles isn't listing racial slights on ``F.U.B.U.'' to
educate those who perpetuate them; she does it to foster intimacy with
her listeners.

I first heard the song at the gym, on a cheap pair of headphones, so the
bass on the record wasn't at the forefront as it should have been. The
warbled notes of the piano and organ sounded muffled, as if underwater.
On even very good headphones, the song's dragged-down pace is
reminiscent of screw music --- the slow-motion, purposefully warped
aesthetic born in Knowles's native Houston. The featured artists are a
nod to R.\&B. over the past two decades. The-Dream sings in an airy
falsetto, making the line ``I didn't come back down to earth to die''
sound resolute instead of bitter. The relative newcomer BJ the Chicago
Kid's delivery --- by turns exuberant and plaintive --- is like that of
a soloist during Sunday praise and worship. The backing vocals by Tweet
are a special treat for those of us who sometimes shake our fists at the
sky, wondering what ever happened to that singer and her hypnotizing
voice.

For all its angst and exasperation, ``F.U.B.U.'' is a party song, just
not one for the nightclub. Its measured cadence and dragging bass are
perfect for a spontaneous, low-key house party. You might be able to
pull off a drawn-out, stylized two-step to ``F.U.B.U.,'' but really it's
a rhythm tailored to the simple head-nod --- an appropriate move for a
song about recognition. Its boisterous horns call to mind the New
Orleans second line, those musical parades marched both for celebration
and for mourning.

Knowles wrote lyrics for ``A Seat at the Table'' in New Iberia, La.,
where her mother's family hails from. Her mother, Tina Lawson, has said
that her family was essentially run out of town following a salt-mine
collapse involving her father. In the contentious aftermath, a Molotov
cocktail was thrown through the window of their home. They didn't go far
--- less than 300 miles west to Galveston, Tex. Hasty separation leaves
a generational longing all the same. It follows that an album created in
a town once fled for fear of white violence should include ``F.U.B.U.,''
a song that creates a space where black listeners might feel prized just
for being seen.♦

\emph{Angela Flournoy is the author of ``The Turner House'' and a fellow
at the New York Public Library's Cullman Center.}

Read more

\hypertarget{17}{%
\subsection{17.}\label{17}}

\hypertarget{--side-to-side}{%
\subsection{\texorpdfstring{ Side to
Side}{  Side to Side}}\label{--side-to-side}}

\hypertarget{ariana-grande}{%
\paragraph{Ariana Grande}\label{ariana-grande}}

Bubble-gum-pop queen turns tongue-in-cheek sex bomb.

By HAZEL CILLS

\href{https://open.spotify.com/track/1pKeFVVUOPjFsOABub0OaV}{\includegraphics{https://static01.graylady3jvrrxbe.onion/newsgraphics/2017/03/03/music2017/assets/icons/spotify.svg}Full
Track}

When a young pop star outgrows the wholesome, teeny-bopper sound that
made her career, we tend to watch the results while peeking through our
fingers, terrified of what we'll see. This is how we watched as
Christina Aguilera put on her ``Dirrty'' leather chaps, as Britney
Spears stepped out of her ``little girl'' world on ``I'm a Slave 4 U,''
as Miley Cyrus awkwardly fraternized with a stripper pole at the Teen
Choice Awards. What makes these moments so cringeworthy is their
self-seriousness, their declaration that to be adult is to be sexy ---
and to be sexy is to be straight-faced, preferably with a well-oiled
body, writhing and pining for male approval.

Ariana Grande, pop music's tiny, adorable diva, could easily have found
herself in such a position. A former Broadway actress and the former
star of the Nickelodeon show ``Sam \& Cat,'' she has an unabashed
theater-kid spirit and a flair for costume --- you often find her
wearing cat ears and thigh-high vinyl boots. Her songs are brassy, retro
numbers that deal with gushy PG-13 love, and she has a tendency to
oversell them with the zeal of a collegiate a cappella singer, her
impressive vocal range pushing against the edges of her bubble-gum hits
as if trying to pop them entirely. Grande, still straddling the line
between child star and adult hitmaker, is in the prime risk group for
hypersexed transition songs.

\includegraphics{https://static01.graylady3jvrrxbe.onion/images/2017/03/12/magazine/12music17-grande/12music16-grande-master180.jpg}

And on ``Side to Side,'' the biggest single from her excellent third
album, ``Dangerous Woman,'' she's aiming for something grown-and-sexy.
``Side to Side'' has her tapping the same watered-down dancehall trend
that fueled songs like Justin Bieber's ``Sorry'' and Drake's
``Controlla,'' with lyrics about sleeping with a bad secret boyfriend
she keeps hidden from her friends. ``I've been here all night/I've been
here all day/And boy got me walking side to side,'' she sings. Walking
that way, Grande hinted in an interview last year with MTV --- just in
case anyone didn't get it --- is a result of having too much sex, too
vigorously.

It's a huge hit. The song, in which Nicki Minaj coolly raps about riding
a bicycle as if it were a male member (or vice versa), as if she were
starring in some psychosexual Cronenberg horror film, has more than 420
million plays on Spotify. But ``Side to Side'' isn't exactly sexy. With
its coy innuendo, as well as a video that uses SoulCycle-esque bikes as
a metaphor for Grande's sexual workouts, ``Side to Side'' plays more
like a giggly inside joke.

We're in a golden age for things like this: media made by women, aimed
at women, in which sex is explicit, messy and, most important, funny.
Listening to ``Side to Side,'' I'm reminded of Issa Rae's HBO series
``Insecure,'' in which Rae's eyes continually wander to an ex named
Daniel, whom her best friend describes as Rae's Achilles' heel ---
except instead of ``heel,'' she uses a certain nickname for a phallus.
On shows like ``Broad City'' or Amazon's ``Fleabag,'' female
protagonists dig into their worst and raunchiest sexual impulses. Last
year, the Swedish pop star Tove Lo named her record ``Lady Wood,'' a
goofy, equal-opportunity reinvention of stereotypical male arousal.

With its brazenly over-the-top premise, ``Side to Side'' joins them,
beating you to the punch line about Grande's revamp as Sexy Adult. It
doesn't feel as if it's designed to stir anyone's arousal --- only their
laughter. Does this song sound ridiculous? Well, that's the point. In a
genre that fetishizes sexual passion that lasts for eternity, Grande's
song happily delivers comic realism, with a surprisingly smart wink.♦

\emph{Hazel Cills is a staff writer for MTV News.}

Read more

\hypertarget{18}{%
\subsection{18.}\label{18}}

\hypertarget{--fade}{%
\subsection{\texorpdfstring{ Fade}{  Fade}}\label{--fade}}

\hypertarget{kanye-west}{%
\paragraph{Kanye West}\label{kanye-west}}

What would black music sound like in an alternate universe?

By THOMAS CHATTERTON WILLIAMS

\href{https://open.spotify.com/track/3cCxoOgfi6hgt8MNteuiiD}{\includegraphics{https://static01.graylady3jvrrxbe.onion/newsgraphics/2017/03/03/music2017/assets/icons/spotify.svg}Full
Track}

The genius of ``Fade,'' the penultimate track on Kanye West's living
work of art, ``The Life of Pablo,'' is evident from the opening lines, a
sample from the white Motown group Rare Earth. But it was a half-minute
in, at that first unmistakable rip of bass, that I lost my mind. Like
many of West's songs, ``Fade'' is built around several commingling
samples. Its rhythmic backbone is the deceptively simple arrangement
from the 1985 classic ``Mystery of Love,'' by Larry Heard, better known
as Mr. Fingers. That track, along with a handful of others, marks a
seminal moment in the history of deep house --- a rich and criminally
neglected chapter in the book of black music.

Today it's easy to forget that in the early and mid-'80s there existed a
window when New York rap, Chicago house and Detroit techno --- as well
as a slew of other fledgling genres and subcultures --- functioned more
or less as equals, each as likely as the next to flounder or thrive. New
York won the contest handily, and now hip-hop has so thoroughly subsumed
mainstream black culture that it often feels as if earlier artistic
forms have either been eradicated or retrofitted to its preferences
(see: funk, R.\&B. and jazz). House music --- much like West himself ---
is unabashedly black and Chicago-bred, but somewhere along the line, it
grew cozy in Europe and came to be seen as catering to white people. And
though it has only ever managed to find significant audiences overseas,
this transfixing style of minimal electronic dance music was pioneered
by Midwestern D.J.s spinning mainly for black and gay audiences looking
to ``jack'' their bodies at Windy City nightclubs like the Muzic Box and
the Warehouse (where, under the stewardship of Frankie Knuckles, the
style was birthed and named). While trailblazers like Mr. Fingers --- a
virtuosic multi-instrumentalist --- are worshiped in London, Paris and
Berlin, they are barely remembered back at home.

``Fade'' sets out to correct this. Onto the wide-open surface ``Mystery
of Love'' provides, West spreads out his own sparse raps alongside what
grows into an aural smorgasbord of samples, allusions and guest
appearances spanning eras and ethnicities --- '90s Nuyorican house, the
white rapper/singer Post Malone --- a subtle reminder of the outsize
influence of black aesthetics on all manner of American and global
culture.

Which is why, as a radio-friendly hit (with an awe-inspiring video to
boot), ``Fade'' feels not only generous but subversive: In the span of a
little over three minutes, it gives the lie to simplistic conceptions of
musical borders. West has always displayed a rare encyclopedic and
intuitive grasp of both mainstream and regional black sounds, from
traditional gospel and R.\&B. to college-inflected spoken word and even
black Greek stepping, not to mention dance, reggae, trap and drill
music. He knows that, glimpsed from the proper vantage, these are but
facets of the same, constantly shifting whole. I don't think there is
another pop star who could conceive of such a medley, let alone bring it
to life in a way that coheres. Yet ``Fade'' doesn't just cohere; it
functions as a sly and infectious meditation on the variety of formal
possibilities of black sound --- as invented and interpreted by black
people themselves as well as Latinos and white people. It also serves as
a bittersweet thought experiment: Things could have been otherwise.
Imagine, if you will, a world in which Mr. Fingers got his due.♦

\emph{Thomas Chatterton Williams is a contributing writer for the
magazine.}

Read more

\hypertarget{19}{%
\subsection{19.}\label{19}}

\hypertarget{--trolley-song}{%
\subsection{\texorpdfstring{ Trolley
Song}{  Trolley Song}}\label{--trolley-song}}

The counter-diva who brings a comedian's timing and sensibility to jazz.

By MARGO JEFFERSON

\href{https://open.spotify.com/track/5j78Obp9QYGopCUlsDd3nQ}{\includegraphics{https://static01.graylady3jvrrxbe.onion/newsgraphics/2017/03/03/music2017/assets/icons/spotify.svg}Full
Track}

\hypertarget{cuxe9cile-mclorin-salvant}{%
\paragraph{Cécile McLorin Salvant}\label{cuxe9cile-mclorin-salvant}}

How can you tell the singer from the song? On a good day, you can't ---
so let's start with the singer.

She's Cécile McLorin Salvant, a 27-year-old with an exhilarating command
of the jazz vocal tradition, which has long been dominated by women,
many of them black. Salvant's parents are French-Guadeloupean and
Haitian; she grew up in Miami studying classical music, then moved to
Paris and added jazz to her studies. In 2010, she won the Thelonious
Monk Institute International Jazz Vocals Competition; three albums
later, ``For One to Love'' won the jazz vocal album of the year at the
2016 Grammy Awards.

I said she's in ``command of'' the jazz tradition. Better to say she's
in communion with it. I like how she listens. I like how she tests
herself and learns as she performs. Salvant has a supple, well-trained
voice with spot-on pitch. (No vibrato-teases; no meandering warbles
passing as melisma.) Her low notes go from husky to full-bodied; her
high notes float purely and cleanly. When she scats, it's not an ego
trip but a musical game, where notes and syllables get to shape-shift.

\includegraphics{https://static01.graylady3jvrrxbe.onion/images/2017/03/12/magazine/12music6-salvant1/12music6-salvant1-master180-v3.jpg}

Like many young jazz singers, she does the Great American Songbook ---
the Gershwins, Rodgers and Hart or Hammerstein, Sondheim, Ellington. The
risk? Sounding decorous and derivative. Like some other young jazz
singers, she does the black vaudeville hits of Bert Williams and Bessie
Smith, even some of the exotica that female musicians once tossed out to
keep their fans tantalized. Here the risk is archness: the knowing
postmodern wink.

I wouldn't be writing about Salvant if she had fallen into either trap.
But mainstream success has other traps. And while it has been a long
time since jazz was at the center of pop commerce, the star- (or cult-)
making machinery still labors to produce familiar types. Especially for
women, and even more especially for black women.

For black women in pop music, the dominant and preferred model remains
the Diva. Here's my personal theory. The black diva's obverse is the
black matriarch, that forceful mammy, maid or housekeeper, whose mythic
reign lasted through a century of film and television. And yes, it's a
long way from mammy to diva, but there's one constant. Racially and
socially, this figure is considered lower and lesser. Theatrically, then
--- it's the law of fantasy compensation --- she must appear greater. In
the old days, that meant she was literally bigger, louder, bossier. Now
she's symbolically bigger: more expansive, more intense, more
outrageous.

Call the roll of the last half-century, from Patti LaBelle to Fantasia,
Jennifer Holliday to Jennifer Hudson. My favorites are Aretha Franklin
and Tina Turner because of their contradictions. There's a core reserve,
even sadness in Franklin, while Turner channeled her lust-goddess
intensity through the dance moves of a gritty pugilist. Beyoncé and
Rihanna are the grandest of contemporary divas. Their social, financial
and cultural riches are vast; they command global kingdoms. They aren't
prey to traditional diva narratives of abuse or self-destruction; they
turn all vulnerabilities into victories. Beyoncé is the gracious
sovereign, Rihanna the cocky bad girl turned It Girl. But they share a
mandate: The Black Diva must flaunt, court and rule.

Jazz divas have tended to have alter egos. Billie Holiday was a laconic
wit before she was the Lady with the Gardenia; Sarah Vaughan could
counter her Divine One with up-tempo Sassy. Dinah (The Queen) Washington
was also a salty good-time gal. I hear their traces in Salvant's
singing, and it gives me great pleasure to watch her revise the tropes
of black divadom. Don't misunderstand --- she's not an anti-diva. I'd
call her a counter-diva: the heroine as ebullient comedian.

\includegraphics{https://static01.graylady3jvrrxbe.onion/images/2017/03/12/magazine/12music6-salvant2/12music-salvant2-master180.jpg}

\textbf{Listen to her} version of ``The Trolley Song'' from the 1944 hit
film ``Meet Me In St. Louis.'' This is a song that could give any singer
acute Anxiety of Influence symptoms. It began as Judy Garland's clarion
love call, then entered the jazz repertory to be taken up by Sarah
Vaughan and Betty Carter. Vaughan's version is speedy and pristinely
seductive; Carter's is a daredevil blend of swing and abstract
expressionism. Salvant pays homage to all three with touches of loving
parody --- Garland's girlishness, Vaughan's love for her own silky
vibrato, Carter's near-manic tempo changes --- finding her own
interpretation. The train rhythms stop, start and stutter; so does
Salvant's voice. Our heroine is thinking and feeling her way, note by
note, word by word, into exuberant infatuation, fashioning a
romantic-comedy monologue in which the woman surprises herself with each
turn of phrase and tempo.

Salvant, like all counter-divas, constructs her look with care. Hers is
gamine glam: Her face is round, her hair close-cropped. She wears big,
blocky white glasses --- a droll trademark, like Fats Waller's derby; a
red fascinator with feathers that look like insect feelers.

The black counter-diva is now making her way into the culture at large.
I hear her when the mezzo-soprano Alicia Hall Moran builds a song around
nothing but a repeated ``Shhhh'' or arranges to fuse John Dowland's
``Flow My Tears'' and Smokey Robinson's ``Cruisin'.'' She's in the
mischief that snakes through Corinne Bailey Rae's rueful feathery songs
too.

I see her when Tracee Ellis Ross goes goofy on ``black-ish.'' (It's as
if she has reclaimed the soubrette wit her mother renounced for lofty
grandeur.) I like how Rashida Jones uses cheerful slapstick to counter
her sweet-girl prettiness in ``Angie Tribeca.'' As Lucca Quinn, the
clever lawyer on ``The Good Wife'' and ``The Good Fight,'' Cush Jumbo
seems to be letting us in on a private joke; a tiny pause, the hint of a
double-take, suggests a quizzical ``Can I get away with this?'' and a
gleeful ``OMG, I just did.'' And won't it be fun to watch Lupita Nyong'o
in the film version of Chimamanda Adiche's ``Americanah''? Freed from
period dramas that stretch from slavery to the Empire galaxy, she will
bring her talents to a here-and-now comedy of morals and manners.

John Gielgud once said that style is knowing what play you're in. It's
true for all performers who work with varied traditions and repertories.
And it means that the performer must know what age she is performing in,
and why. Along with a merry band of counter-divas, Salvant is thinking
hard about what conventions, habits and desires need revising --- in her
art, and in her audience.♦

\emph{Margo Jefferson is the author of ``Negroland: A Memoir.''}

Read more

\hypertarget{20}{%
\subsection{20.}\label{20}}

\hypertarget{--grigio-girls}{%
\subsection{\texorpdfstring{ Grigio
Girls}{  Grigio Girls}}\label{--grigio-girls}}

\hypertarget{lady-gaga}{%
\paragraph{Lady Gaga}\label{lady-gaga}}

A boozy toast to friendship.

By BOBBY FINGER

\href{https://open.spotify.com/track/6Xxvdy1m9TX9HTxOzQcfuV}{\includegraphics{https://static01.graylady3jvrrxbe.onion/newsgraphics/2017/03/03/music2017/assets/icons/spotify.svg}Full
Track}

When Lady Gaga, draped in what looked like the silver uniform for a
sport that hadn't yet been invented,
\href{https://youtu.be/txXwg712zw4}{jumped off the roof} of NRG Stadium
at the Super Bowl this year, she reaped the benefits of a
safe-but-exciting performance almost immediately. Fireworks shows and
rip-roaring acrobatics and dancing in front of a group of second-rate
backup dancers in capes are the kind of broadly appealing theatrics that
generate awe without shock, intrigue without the threat of controversy.
Her albums shot to the top of the music charts, her performance was
roundly praised by fans and critics and more than 111 million viewers
were reminded that her 2009 single ``Bad Romance'' still bangs.

This wasn't Lady Gaga's comeback, as many claimed: She never really left
us, nor had she abandoned the music that made her famous.

Two of Gaga's early albums, ``The Fame'' (2008) and ``Born This Way''
(2011), were bursting at the seams with instantly memorable dance-pop
hits. But after her fourth solo album, the both fairly and unfairly
maligned ``Artpop'' (2013), failed to deliver that kind of instant
success, Gaga redirected her focus from an ostentatious existence ---
she wore a dress of raw beef to the 2010 MTV Video Music Awards, after
all --- to less-bombastic acts of musical bravura that showcased her
virtuosic skills as both a singer and a pianist. Last October, she
released her much-anticipated fifth album, the deeply personal and
country-tinged ``Joanne.''

So when she fell to Earth during last month's Super Bowl and performed
the hits that made her famous, fans who thought ``Joanne,'' with its
stripped-down production and abundance of twang, was yet another sign
that she had abandoned her old ways were able to breathe a sigh of
relief: Gaga's verve hadn't gone anywhere. It's a message she made clear
in ``Grigio Girls,'' a midtempo acoustic bonus track and the album's
best song.

In ``Grigio Girls,'' Gaga recalls a moment in her early 20s when she was
``spiraling out,'' only to be saved by the contagious spirit of her
30-something friend Sonja Durham, who was battling breast cancer at the
time. ``Grigio Girls'' is an ode to friendship and singing along to your
favorite songs off-key. But through some strange alchemy involving
pop-culture references and cheap wine, it manages to do more than just
celebrate sisterhood: It's proof that Gaga is still fundamentally Gaga,
whether she's wrapped in meat or the arm of her recent collaborator,
Tony Bennett. Though explicitly about Dunham and the inspirational
``tough girl'' persona she shared with Joanne Germanotta (Gaga's aunt,
who died in 1974 and for whom the album is named), ``Grigio Girls''
becomes something more universal once Gaga's own voice is joined by the
backing harmony. ``Sisters never pack up,'' the singers reassure us.
``We always run back.''♦

\emph{Bobby Finger is a staff writer for Jezebel and co-host of ``Who?
Weekly.''}

Read more

\hypertarget{21}{%
\subsection{21.}\label{21}}

\hypertarget{--ooouuu}{%
\subsection{\texorpdfstring{ Ooouuu}{  Ooouuu}}\label{--ooouuu}}

A woman's unabashed appreciation of women.

By HANNAH GIORGIS

\href{https://open.spotify.com/track/0D21XvHcVsIvJM6FcGY2BT}{\includegraphics{https://static01.graylady3jvrrxbe.onion/newsgraphics/2017/03/03/music2017/assets/icons/spotify.svg}Full
Track}

\hypertarget{young-ma}{%
\paragraph{Young M.A}\label{young-ma}}

Young M.A's music sounds like summer: It buzzes, bumps, burns. It's hot.
``OOOUUU,'' the Brooklyn artist's single from last May, both
soundtracked the sweltering season and cemented its place in party
rotations well into the colder months, and well beyond New York. It
finds the confident East New York-born newcomer relaxed, but not sedate.
Her opening line is ``Yo, bro, I think I had too much Hennessy, man''
--- and M.A has said she was intentionally ``smizz and drizz'' (that is,
drunk) while recording it, ``so it feels authentic.'' Hennessy is not
known for encouraging rest, and M.A rides the beat with an energy that
is at once frenetic and hazy. A slew of clever punch lines invite you to
rap alongside her --- to feel yourself too. And with spacey, futuristic
production from U-Dub of NY Bangers, ``OOOUUU'' insists you respect its
groove by dancing with your whole body. When you hear it in the club,
muscle memory takes over.

Like some of the best New York rap that preceded it, the track
emboldens. M.A told the website Genius that the song wasn't modeled
after fellow Brooklyn rapper Bobby Shmurda's ``Hot Nigga,'' but the two
share similar beats --- and ``OOOUUU'' lends itself perfectly to the
``Shmoney Dance'' that took over Vine in 2014 and turned Shmurda into a
viral star. The city's rappers took note: Less than a year since its
release, the track has been remixed by local heavyweights like Nicki
Minaj, French Montana, 50 Cent, Jadakiss and Remy Ma.

\includegraphics{https://static01.graylady3jvrrxbe.onion/images/2017/03/12/magazine/12music8-YOUNGMA/12music8-YOUNGMA-master180-v2.jpg}

The original song is full of references to M.A's own life in Brooklyn,
where she's ``in these streets more than Sesame.'' While the track's
narrative might span only one day, it feels firmly grounded in the
habitual: Young M.A raps what she knows. She's not only a woman rapping
but also a masculine-presenting queer woman in an industry that is
dominated by heterosexual men. Queer hip-hop artists are hardly new;
rappers like Le1f, Zebra Katz, Mykki Blanco, Cakes Da Killa and many
others have been at the forefront of shifting the genre's landscape in
recent years, and rumors of sexual fluidity have always surrounded many
of hip-hop's biggest stars, both despite and because of the industry's
reputation for homophobia. (Queer women in particular must grapple with
intersecting axes of misogyny and homophobia.) Still, M.A seems
refreshing, and quintessentially New York, in her frankness. She is who
she is.

The ``OOOUUU'' punch lines that inspire the most raucous singalongs are
those in which she speaks, openly and matter-of-factly, about sex with
another woman. Rapping about women is not the entirety of her catalog,
nor is it a gimmick; she simply traffics in the same kind of braggadocio
the industry's men are often known for, which is to say she's your
average Bed-Stuy stud. M.A is seductive --- even to women who don't
identify as lesbian, queer or bisexual --- and she knows it. If M.A's
contemporary Remy Ma is conceited, M.A is cocky: ``If that's your chick,
then why she textin' me?/Why she keep calling my phone speaking
sexually?''

The lines that follow are salacious, but they're not performative. M.A
may be a commanding, bite-her-lip-when-she-looks-in-your-direction queer
woman, but her music neither starts nor ends there. What's more
significant than M.A's queerness is the way she deftly weaves references
to sex with women into her music without overstating or explaining them.
If you know, you know. And if you don't, you can still enjoy her music
--- but M.A isn't going to waste her bars enlightening you.♦

\emph{Hannah Giorgis is the special-projects editor for The Ringer.}

Read more

\hypertarget{22}{%
\subsection{22.}\label{22}}

\hypertarget{--changes}{%
\subsection{\texorpdfstring{ Changes}{  Changes}}\label{--changes}}

\hypertarget{charles-bradley}{%
\paragraph{Charles Bradley}\label{charles-bradley}}

A '70s rock star's troubles, and a 68-year-old soul singer's.

By CHARLES AARON

\href{https://open.spotify.com/track/3i85UIF4nR7qZXTnEwqdPr}{\includegraphics{https://static01.graylady3jvrrxbe.onion/newsgraphics/2017/03/03/music2017/assets/icons/spotify.svg}Full
Track}

A chord from a Hammond B3 organ quavers, and Charles Bradley's head
snaps up, his eyes locking with the camera. When the woozy accompaniment
hits, his deep-creased face wrenches; he shakes his head, places his
palms together and gazes Godward. So begins the remarkable video for
Bradley's soul rendition of Black Sabbath's 1972 gloom-folk lament
``Changes'' --- no lip-syncing, just the 68-year-old singer in a black
turtleneck against a blank backdrop. For six minutes, as his sculpted
rasp wails, Bradley uses his face and hands to silently act out a vast
grief, one at which the plain-spoken lyrics, originally written about
the Sabbath drummer Bill Ward's breakup with his first wife, only hint.
Bradley conceived his version while comforting and reconciling with his
dying mother, but even if you didn't know that, his pantomime would
still rip you up.

``Changes'' was originally the product of rock's early-'70s commercial
largess. After three albums spent forging heavy metal in the unglamorous
English Midlands, Sabbath was afforded a Bel Air mansion to record its
fourth. The house was filled with equipment and enough drugs to get
``coked out of our brains every day,'' as the bassist Geezer Butler
recalled in ``Black Sabbath: Doom Let Loose,'' Martin Popoff's
illustrated history of the band. One debauched night, the guitarist Tony
Iommi wandered over to a grand piano in the manse's ballroom and tapped
out the melody of ``Changes.'' Butler wrote the lyrics, and Ozzy
Osbourne, the singer, found some gentle nuance in them, despite being in
such a drug-freaked state that he was ``dreaming there was, like, a tape
machine coming into my room and eating me.''

When Bradley's guitarist, Tom Brenneck, raised the idea of covering
``Changes,'' he had never heard of Black Sabbath. ``But as I was
learning the song, my mom was very ill, and she was talking to me a lot,
and what she was saying fit the lyrics,'' he told me. ``Before that
song, I was just trying to forget about her dying. But how can you
forget the deepness of your mom finally telling you about her personal
life and what she'd gone through?'' Osbourne was so impressed by
Bradley's version that he sent an appreciative email.

Bradley's relationship with his mother was deeply fraught. She left her
8-month-old son to move from Gainesville, Fla., to Brooklyn, then sent
for him when he was 8. He left at 14, saying she was abusive. Decades
later, she begged him to come home to Brooklyn and take care of her. He
agreed, reluctantly, but time didn't soften his resentment. It was only
when she finally shared the terror she experienced in the Jim Crow South
of the 1930s and '40s --- and her struggles as a single mother of three
in Bed-Stuy --- that he was able to see his mother clearly.

By 2011, when Bradley released the first of three top-shelf albums for
the Brooklyn label Daptone, he was 62 and badly scarred by his own
misfortunes. Even his ensuing, 11th-hour success has come at a cost: He
has been pressured to support family and friends, and because his
artistic method demands a deep emotional connection with his material,
he has found himself revisiting buried memories of homelessness and
violence. Late last year, he underwent a debilitating operation for
stomach cancer. When I spoke with him in February, he was making plans
to sing again, but his voice was practically a whisper. Still, he
eagerly testified about ``Changes.''

``Oh, my God,'' he said, ``when I heard the middle of the song, where
it's, `It took so long/To realize/That I can still hear/Her last
goodbyes,' that really touched my soul, truthfully. I used to think my
mom was evil, but we were able to find forgiveness at the end of her
life. Now I can go out into the world without animosity or anger and
show people the love in my soul.''♦

\emph{Charles Aaron is a contributing writer for MTV News.}

Read more

\hypertarget{23}{%
\subsection{23.}\label{23}}

\hypertarget{--seigfried}{%
\subsection{\texorpdfstring{
Seigfried}{  Seigfried}}\label{--seigfried}}

\hypertarget{frank-ocean}{%
\paragraph{Frank Ocean}\label{frank-ocean}}

On the boldness and cowardice of love.

By JASON PARHAM

\href{https://open.spotify.com/track/1BViPjTT585XAhkUUrkts0}{\includegraphics{https://static01.graylady3jvrrxbe.onion/newsgraphics/2017/03/03/music2017/assets/icons/spotify.svg}Full
Track}

Five years ago, on a sticky July afternoon, I stood in the kitchen of a
friend's apartment, face jammed to an iridescent iPhone screen, poring
over Frank Ocean's ``coming out letter.'' What seemed radical then now
feels so achingly true to Ocean's uncategorizable spirit: He'd posted
\href{http://frankocean.tumblr.com/post/26473798723}{a screen shot} of a
TextEdit file to his personal Tumblr page and, in the echo of a Lucille
Clifton or Countee Cullen poem, detailed his summers-long relationship
with another man, his first love. Five years later, one passage in it
has yet to unhook itself from me: ``HUMAN BEINGS SPINNING ON BLACKNESS.
ALL WANTING TO BE SEEN, TOUCHED, HEARD, PAID ATTENTION TO.''

Whether you have the emotional bandwidth to admit it or not, all of us,
at some point in our volatile existence on Earth, want to be
acknowledged by another human being --- seen, touched, heard and paid
attention to, and not just platonically. Maybe those two lines jolted me
because I live in an ebb-and-flow of denial about my own romantic
relationships. Of the four semi-serious ones I've had, not one has
lasted longer than a year. (Once I asked an ex, with whom I'd remained
on good terms, to send me a favorite sentence of hers for a
collaborative writing project I was working on; maybe half-jokingly, but
probably not, she responded with the St. Vincent lyric ``Your heart is a
strange little orange to peel.'') And if music, like all art, should be
an exercise in just how honest you are willing to be with yourself, then
here are two truths: One is that I have never once known the texture of
love --- its nirvana-inducing high, its valley of heartbreak. The other
is the logical, dead-on explanation for this --- I am a coward. I refuse
to take my guard down and open up.

Cowardice is also something you innately recognize in another. And on
``Seigfried,'' a moody, astral slow burner from Frank Ocean's
``Blonde,'' I see an ideal palette for Ocean to mask cowardice as love.
Has self-confliction ever sounded this divine? Over a bed of drowned-out
guitar chords, he sings: ``Maybe I'm a fool/To settle for a place with
some nice views/Maybe I should move/And settle down, two kids and a
swimming pool/I'm not brave.'' But it's the end of the song that finds
Ocean sounding his most tormented. ``I'd do anything for you,'' he
croons in the final minute, pausing for a beat and then adding, ``in the
dark.'' He repeats the line five or six times. It's such a pure
exchange, and a beautiful one too. Feelings run up against one another:
vulnerability and intimacy, the possibility of love. But there's also
that undercurrent of cowardice. It's a deep, genuine promise with a
built-in provision. I am yours, but only when the lights are low and no
one is looking.

Perhaps it's not a surprising admission in a song that borrows from the
singer-songwriter Elliott Smith, himself an effigy of melancholia, who
died in 2003. But Ocean has always seemed so unafraid to me, both in his
personal life and in the way his music emerges from the inside out. I
suppose the bravery we see in him is evidenced by his very promise:
though cloaked in a haze of emotional discord, it is one offered on his
terms (Ocean has insisted on this position of self-authorship and
ownership with increasing brazenness ever since the success of his first
LP, ``Channel Orange''). Maybe he has become more of the man he has
always wanted to be in the years since the summer of 2012. I certainly
haven't.

The truth is, many of us are still ``spinning on blackness'' in our
day-to-day lives --- at times lost, afraid, uncertain. But tonight, here
in the dark and away from everybody else, Ocean's orchestral psalm is a
reminder that you can still be in full control, even if that means
retreating to a place you vowed never to return to. It is only with this
kind of startling humanness --- wading through life's moral gray ---
that Ocean could deliver such a promise. Because, really, this isn't
about love or lust; it's about power. Harness it. Hold it close. And
don't let go.♦

\emph{Jason Parham is a senior editor for The Fader magazine.}

Read more

\hypertarget{24}{%
\subsection{24.}\label{24}}

\hypertarget{--your-best-american-girl}{%
\subsection{\texorpdfstring{ Your Best American
Girl}{  Your Best American Girl}}\label{--your-best-american-girl}}

Growing up wondering if the American dream is for you.

By JENNY ZHANG

\href{https://open.spotify.com/track/172rW45GEnGoJUuWfm1drt}{\includegraphics{https://static01.graylady3jvrrxbe.onion/newsgraphics/2017/03/03/music2017/assets/icons/spotify.svg}Full
Track}

\hypertarget{mitski}{%
\paragraph{Mitski}\label{mitski}}

Growing up in America, I experienced two puberties. The first opened me
up to the possibilities of adulthood. The second reinforced that for
someone like me --- an immigrant, a minority, an Asian-American ---
there were limits. In this second coming-of-age, I had to contend with
the pain of wanting a beautiful white body, not out of some misguided
vanity, but because I saw over and over how whiteness conferred an
instant legitimacy.

So when the indie musician Mitski Miyawaki titled her fourth studio
album ``Puberty 2,'' I felt an immediate flicker of recognition. I felt
another when she released the first single, ``Your Best American Girl,''
last March. Coming from someone like Mitski, who was born in Japan to a
Japanese mother and an American father, the title alone was powerful
enough to reopen a wound that had been rotting inside me ever since I
came of age as an Asian girl in America: an old hatred for myself, my
culture, the way I looked and the way I was raised.

\includegraphics{https://static01.graylady3jvrrxbe.onion/images/2017/03/12/magazine/12music3-MITSKI/12music-madigan2-master180.jpg}

The simple bass line that opens the song should feel familiar to anyone
who ever bobbed along with '90s indie rock while indulging in self-pity
(as I most certainly did). Mitski murmurs and sighs and sings about a
lover. ``If I could, I'd be your little spoon/And kiss your fingers
forever more.'' Did it mean something to me to hear another
Asian-American woman refer to herself in the diminutive? Of course it
did, and even more to hear her sing, ``But big spoon, you have so much
to do/And I have nothing ahead of me.'' The guitar and drum beat join
in, adding bulk, while Mitski's voice remains at a steady, low ache,
elevating him to the status of a god, unable to imagine that anything
could be off-limits to him: ``You're the sun, you've never seen the
night.'' She denigrates herself again: ``Well, I'm not the moon, I'm not
even a star.'' Then, after the power chords kick in and the chorus lifts
off: ``Your mother wouldn't approve of how my mother raised me/But I do,
I think I do.''

Mothers have always held such symbolic weight in determining a person's
worth. Your mother tongue, your motherland, your mother's values ---
these things can qualify or disqualify you from attaining myriad
American dreams: love, fluency, citizenship, legitimacy, acceptance,
success, freedom. ``You're an all-American boy,'' Mitski sings. ``I
guess I couldn't help trying to be your best American girl.''

I wanted to hear Mitski's story in this song, but I only heard my own.
Listening brought me back to the fuzzy '90s D.I.Y. scene of my
adolescence in the suburbs of Long Island, back when no one much
questioned why a subculture that saw itself as rebelling against the
establishment was quite so dominated by white men. I grew up in that
scene, as far as someone like me could --- a self-hating Asian kid with
strict immigrant parents who wouldn't let me go over to white people's
houses for fear that their values (laziness, selfishness, drug abuse,
sexual promiscuity) would infect me. My mother had two unshakable
beliefs that she tried to drill into me. The first was that I had to
study and work twice as hard as my white peers if I wanted to survive in
America, and the second was that it was delusional and dangerous to
believe I possessed the same freedom white people had to pursue my
dreams.

We never actually said ``white people''; we used the Chinese term for
``American,'' \emph{mei guo ren}. But it was clear which Americans we
were referring to. For over a decade, she and my father worked two and
sometimes three jobs, on top of night classes, until they saved enough
to move us into an upper-middle-class, mostly white neighborhood.
``You've trapped me,'' I used to argue to her in high school. ``You
brought me to this place but you won't let me be a part of it.''

When I told my mother I wanted to be a writer, she reacted as if I had
said I wanted to kill myself. She vowed that she would do everything
humanly possible to stop me. Compared with how she had behaved as a
teenager, I was ungrateful; compared with the white mothers I knew, she
was a dictator. At her most cutting, she asked me, ``Who's ever going to
read your stories?'' I was stumped. I couldn't say. At that point in my
life, kids were still shouting my last name back to me in the hallways
as if it were a punch line and asking me if Chinese people really ate
dogs. For the first time, it occurred to me that maybe my mother wasn't
trying to stop me so much as she was trying to protect me.

\hypertarget{-8}{%
\subsection{}\label{-8}}

\hypertarget{-9}{%
\paragraph{}\label{-9}}

There's a line in the second verse of ``Your Best American Girl'':
``You're the one/You're all I ever wanted/I think I'll regret this.'' It
hints at what so many nonwhite Americans have long suspected: that our
desire to be loved may be truly masochistic. It goes unquestioned that
the non-American girl would want to assimilate into the all-American
boy's world. But she cannot. That world is not for her.

Racism thrives when the people being harmed by it blame themselves and
one another. As a teenager, I blamed my parents for failing to secure me
admission into whiteness, which I was certain was a prerequisite to
being loved. I was mad at them, not at the cruelty of the American dream
or the ways in which white supremacy had warped each of us. My
privileged upbringing and education and linguistic fluency gave me such
proximity to whiteness that it stung all the more to still find myself
outside of it. My mother, on the other hand, not only accepted that she
would always be an outsider in this country but also believed it to be a
finer fate and home than any other she could have had.

As I grow older, I find myself trying to turn to her more, to ask the
questions I never asked when I was too busy feeling sorry for myself:
why she came here, what it was like to uproot a life and move across the
world, whether it was hard to raise me here. Does Mitski's narrator
really approve of how her mother raised her? It's unclear, but you can
hear in her voice just how much she wants to.♦

\emph{Jenny Zhang is the author of the forthcoming short-story
collection ``Sour Heart.''}

Read more

\hypertarget{25}{%
\subsection{25.}\label{25}}

\hypertarget{--bad-and-boujee}{%
\subsection{\texorpdfstring{ Bad and
Boujee}{  Bad and Boujee}}\label{--bad-and-boujee}}

\hypertarget{migos}{%
\paragraph{Migos}\label{migos}}

Raindrop. Drop top. Is rap music the new punk rock?

By NAOMI ZEICHNER

\href{https://open.spotify.com/track/4Km5HrUvYTaSUfiSGPJeQR}{\includegraphics{https://static01.graylady3jvrrxbe.onion/newsgraphics/2017/03/03/music2017/assets/icons/spotify.svg}Full
Track}

Ever since the election in November, I've gone out more and stayed out
later. This doesn't feel glamorous or especially healthy. People who go
to clubs regularly are usually there to work --- whether they make or
manage or write about music or just labor to look great. You spend more
money than you have and try not to think about it, because late at
night, positive energy is the most valuable currency. If anything made
the hangover the next morning worth the trouble, it was Migos's ``Bad
and Boujee,'' which dominated parties of all kinds this freakishly warm
winter.

The Atlanta rap trio --- made up of the rappers Quavo, Offset and
Takeoff --- have been around for years, but ``Bad and Boujee''
represents their most significant foray into the mainstream, and it's
still a slightly awkward fit. For a recent music-video shoot, they wore
heavy-metal tees and studded leather; Quavo's jacket was painted with
the skull logo of the Misfits. This is an increasingly common trope in
rap these days; many hip-hop artists insist they are rock stars to
signal that they are at the center of today's counterculture --- which
is true. The trio records constantly and has worked the live circuit
hard. First the rappers conquered Atlanta's strip-club scene, then the
V.I.P. booths of the surrounding states and then the money of
corporations. On their recent album ``Culture,'' Quavo reflects on their
lifestyle: ``All this pain, we can live through it. It's called
success.''

\includegraphics{https://static01.graylady3jvrrxbe.onion/images/2017/03/12/magazine/12music12-migos/12music-MIGOS-master180.jpg}

``Culture'' triumphed on the back of ``Bad and Boujee,'' which hit No. 1
on the Billboard Hot 100 chart in January after months of gestation,
thanks in part to internet memes that parroted back its most memorable
parts. Migos is known for building songs around catchy lines, and ``Bad
and Boujee'' is expertly crammed with them, from its opening preamble to
its abstract poetic ``raindrop, drop top'' hook and the five ecstatic
woos that announce the beginning of Offset's verse. The song is nearly
six minutes long, and these noises are like souvenirs collected by
everyone who has traveled through it, a trip that usually happens at the
pinnacle of a night out, when the dance floor is packed and the women in
the crowd are ready. ``You can just tell,'' one D.J. friend said, when I
asked how he decides when to play it. ``It's almost as if someone
whispers in your ear, `It's time.' ''

With all its parts to latch onto, ``Bad and Boujee'' draws people
together. When its bass shakes, people do, too; you could cut the music
off and everyone would still be yelling in unison. Surrounded by that
kind of collective precision, you might find it difficult not to feel
less alone. If Migos has to pose in punk-rock outfits to be recognized
as a voice for its generation, at least it's an authentic pose: The
punks I know love their music deeply. Whether they meant to or not, with
``Bad and Boujee,'' Migos created a space of refuge for people who
needed a break and were willing to work for it. The song offered a
sanctuary: Settle in, stay a while, belong here for as long as you
need.♦

\emph{Naomi Zeichner is the editor in chief of The Fader.}

Read more

\hypertarget{bonus-track}{%
\subsection{BONUS TRACK.}\label{bonus-track}}

\hypertarget{--spit-out-the-bone}{%
\subsection{\texorpdfstring{ Spit Out the
Bone}{  Spit Out the Bone}}\label{--spit-out-the-bone}}

\hypertarget{metallica}{%
\paragraph{Metallica}\label{metallica}}

You play you.

By Steven Hyden

\href{https://open.spotify.com/track/7MHkweinOceprzkRWIYTTc}{\includegraphics{https://static01.graylady3jvrrxbe.onion/newsgraphics/2017/03/03/music2017/assets/icons/spotify.svg}Full
Track}

Last November, Metallica did something extraordinary: It put out a new
album, ``Hardwired . . . to Self-Destruct,'' that actually sounded like
Metallica. On paper, a band acting like itself seems normal. But
consider Metallica's output since its megaselling self-titled LP in
1991, better known as ``The Black Album.'' There was the ``alt-rock
haircut'' phase of ``Load and Re-Load.'' The ``Let's stop playing guitar
solos'' phase of ``St. Anger.'' And, most regrettable, the ``Collaborate
with Lou Reed on a confounding art-rock opus about a 19th-century German
femme fatale'' phase of ``Lulu.''

Contrast those digressions with ``Spit Out the Bone,'' the kinetic
seven-minute closer of ``Hardwired.'' Dueling riffs by the guitarists
James Hetfield and Kirk Hammett race forward at a frightening velocity.
Lars Ulrich bashes out a martial beat that sounds like a velociraptor
trapped inside a washing machine. And Robert Trujillo's melodically
sinister bass snakes below the maelstrom, a demon lurking in the
shadows. ``Spit Out the Bone'' epitomizes the thrash-metal formula that
Metallica helped invent in the early '80s and then tried to avoid for
the past quarter-century. On ``Hardwired ... to Self-Destruct,'' it's
almost as if those 25 years never happened.

\includegraphics{https://static01.graylady3jvrrxbe.onion/images/2017/03/12/magazine/12endpaper/12endpaper-master180.jpg}

Not giving the people what they want has been admired as a gesture of
artistic fortitude in music, going back at least to Igor Stravinsky,
whose premiere of his dissonant masterwork, ``The Rite of Spring,''
caused a riot at the Théâtre des Champs-Élysées in Paris in 1913. More
recently, a boxed set collecting every known concert recording from Bob
Dylan's 1966 tour --- 36 discs in all --- was released last fall. The
exhaustive document chronicles a pivotal period for rock's poet
laureate, who angered folk fans by ``going electric.'' Now it's possible
to hear nearly every instance of an audience jeering loudly as Dylan
introduced ``Like a Rolling Stone'' and other songs into his live
repertoire.

Metallica once embraced a foundational idea of ``progressive'' popular
music: Change constantly or become instantly irrelevant, even if those
stylistic shifts alienate your fans. But in an age when a plurality of
niches has supplanted the mainstream that Metallica once conquered,
there's a new kind of street cred that comes with sticking to what
you're known for. Now it's listeners who contain multitudes. It's up to
us to create eclectic playlists designed for a variety of moods and
situations. For a veteran band like Metallica, that means being rewarded
for playing to type. ``Hardwired ... to Self-Destruct'' debuted at No. 1
and was the best-selling rock album in years.

Is it antithetical to art to suggest that ``know your brand'' is the
future not just of music, but also all of entertainment? George Lucas
sold the ``Star Wars'' franchise to Disney in 2012, and now J. J. Abrams
makes the sort of ``fun'' ``Star Wars'' movies that Lucas was trying to
get away from with his little-loved, self-serious prequels. Lucas never
recovered after making his versions of ``Load,'' ``St. Anger'' and
``Lulu.'' Metallica should be grateful that it still gets to make
Metallica records.♦

\emph{Steven Hyden is cultural critic for Uproxx.}

Read more

\hypertarget{bonus-track-1}{%
\subsection{BONUS TRACK.}\label{bonus-track-1}}

\hypertarget{--the-music-issue-podcast}{%
\subsection{\texorpdfstring{ The Music Issue
Podcast}{  The Music Issue Podcast}}\label{--the-music-issue-podcast}}

We talk with some of the great writers in this issue about the songs
that spoke to them over the past year, and what those songs might tell
us about the world right now.

Listen:
\href{https://itunes.apple.com/us/podcast/the-ep-13-songs-that-capture-our-moment/id1213030309?mt=2\&ign-mpt=uo\%3D4}{iTunes}
\textbar{}
\href{http://www.stitcher.com/podcast/the-new-york-times/the-ep-13-songs-that-capture-our-moment?refid=stpr}{Stitcher}
\textbar{}
\href{https://static01.graylady3jvrrxbe.onion/newsgraphics/2017/03/03/music2017/assets/audio/nytmag_ep_2017.mp3}{Download}

\hypertarget{the-ep}{%
\paragraph{The EP}\label{the-ep}}

13 songs, one audio experience.

Read more

Digital Design: Rodrigo de Benito Sanz and Linsey Fields.

\hypertarget{-10}{%
\subsection{}\label{-10}}

\begin{itemize}
\item
  \href{https://www.nytimes3xbfgragh.onion/2017/03/08/podcasts/the-ep-13-songs-that-capture-our-moment.html}{}

  \includegraphics{https://static01.graylady3jvrrxbe.onion/images/2017/03/10/podcasts/the-EP-album-art/the-EP-album-art-mediumThreeByTwo225.jpg}

  \hypertarget{the-ep-13-songs-that-capture-our-moment}{%
  \subsection{`The EP': 13 Songs That Capture Our
  Moment}\label{the-ep-13-songs-that-capture-our-moment}}

  Jan. 20, 2018
\item
  \href{https://www.nytimes3xbfgragh.onion/interactive/2016/03/10/magazine/25-songs-that-tell-us-where-music-is-going.html}{}

  \includegraphics{https://static01.graylady3jvrrxbe.onion/images/2016/03/13/magazine/13mag-music-promo/13mag-music-promo-mediumThreeByTwo225.png}

  \hypertarget{25-songs-that-tell-us-where-music-is-going-1}{%
  \subsection{25 Songs That Tell Us Where Music Is
  Going}\label{25-songs-that-tell-us-where-music-is-going-1}}

  June 5, 2017
\item
  \href{https://www.nytimes3xbfgragh.onion/2016/12/08/magazine/great-performers-la-noir-the-years-most-captivating-film-performances.html}{}

  \includegraphics{https://static01.graylady3jvrrxbe.onion/images/2016/12/11/magazine/11cover2/11cover2-mediumThreeByTwo225.jpg}

  \hypertarget{the-years-most-captivating-film-performances}{%
  \subsection{The Year's Most Captivating Film
  Performances}\label{the-years-most-captivating-film-performances}}

  Jan. 25, 2019
\item
  \href{https://www.nytimes3xbfgragh.onion/2017/02/16/magazine/the-dirty-projectors-go-solo.html}{}

  \includegraphics{https://static01.graylady3jvrrxbe.onion/images/2017/02/19/magazine/19dirtyprojectors1/19mag-19dirtyprojectors-t_CA0-mediumThreeByTwo225.jpg}

  \hypertarget{the-dirty-projectors-go-solo}{%
  \subsection{The Dirty Projectors Go
  Solo}\label{the-dirty-projectors-go-solo}}

  Jan. 20, 2018
\item
  \href{https://www.nytimes3xbfgragh.onion/2017/02/21/magazine/john-legend-cant-pretend-times-are-normal.html}{}

  \includegraphics{https://static01.graylady3jvrrxbe.onion/images/2017/02/26/magazine/26talk1/26talk1-mediumThreeByTwo225-v3.jpg}

  \hypertarget{john-legend-cant-pretend-times-are-normal}{%
  \subsection{John Legend Can't Pretend Times Are
  Normal}\label{john-legend-cant-pretend-times-are-normal}}

  Jan. 20, 2018
\end{itemize}

Advertisement

\hypertarget{site-information-navigation}{%
\subsection{Site Information
Navigation}\label{site-information-navigation}}

\begin{itemize}
\tightlist
\item
  \href{https://help.nytimes3xbfgragh.onion/hc/en-us/articles/115014792127-Copyright-notice}{©
  2020 The New York Times Company}
\item
  \href{https://www.nytimes3xbfgragh.onion}{Home}
\item
  \href{https://www.nytimes3xbfgragh.onion/search/}{Search}
\item
  Accessibility concerns? Email us at
  \href{mailto:accessibility@NYTimes.com}{\nolinkurl{accessibility@NYTimes.com}}.
  We would love to hear from you.
\item
  \href{https://help.nytimes3xbfgragh.onion/hc/en-us/articles/115015385887-Contact-Us}{Contact
  Us}
\item
  \href{https://www.nytco.com/careers/}{Work with us}
\item
  \href{https://nytmediakit.com/}{Advertise}
\item
  \href{https://help.nytimes3xbfgragh.onion/hc/en-us/articles/115014892108-Privacy-policy\#pp}{Your
  Ad Choices}
\item
  \href{https://help.nytimes3xbfgragh.onion/hc/en-us/articles/115014892108-Privacy-policy}{Privacy}
\item
  \href{https://help.nytimes3xbfgragh.onion/hc/en-us/articles/115014893428-Terms-of-service}{Terms
  of Service}
\item
  \href{https://help.nytimes3xbfgragh.onion/hc/en-us/articles/115014893968-Terms-of-sale}{Terms
  of Sale}
\end{itemize}

\hypertarget{site-information-navigation-1}{%
\subsection{Site Information
Navigation}\label{site-information-navigation-1}}

\begin{itemize}
\tightlist
\item
  \href{https://spiderbites.nytimes3xbfgragh.onion}{Site Map}
\item
  \href{https://help.nytimes3xbfgragh.onion/hc/en-us}{Help}
\item
  \href{https://help.nytimes3xbfgragh.onion/hc/en-us/articles/115015385887-Contact-Us?redir=myacc}{Site
  Feedback}
\item
  \href{https://www.nytimes3xbfgragh.onion/subscription?campaignId=37WXW}{Subscriptions}
\end{itemize}
