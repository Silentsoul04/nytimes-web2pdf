 **NYTimes.com no longer supports Internet Explorer 9 or earlier. Please
upgrade your browser.
\href{http://www.nytimes3xbfgragh.onion/content/help/site/ie9-support.html}{LEARN
MORE »}

**Sections

**Home

**Search

\hypertarget{the-new-york-times}{%
\subsection{\texorpdfstring{\href{http://www.nytimes3xbfgragh.onion/}{The
New York Times}}{The New York Times}}\label{the-new-york-times}}

\hypertarget{-magazine-}{%
\subsubsection{\texorpdfstring{
\href{https://www.nytimes3xbfgragh.onion/section/magazine}{Magazine}
}{ Magazine }}\label{-magazine-}}

 \href{https://www.nytimes3xbfgragh.onion/section/magazine}{Magazine}
\textbar{}View Finder

**Close search

\hypertarget{site-search-navigation}{%
\subsection{Site Search Navigation}\label{site-search-navigation}}

Search NYTimes.com

**Clear this text input

Go

\url{https://nyti.ms/2rAwIh2}

\hypertarget{site-navigation}{%
\subsection{Site Navigation}\label{site-navigation}}

\hypertarget{site-mobile-navigation}{%
\subsection{Site Mobile Navigation}\label{site-mobile-navigation}}

\hypertarget{view-finder}{%
\section{View Finder}\label{view-finder}}

After being forced to brick over his loft's beautiful view of the Hudson
River, a man devises a clever replacement. As drawn by Tom Gauld.

\includegraphics{https://static01.graylady3jvrrxbe.onion/newsgraphics/2017/05/18/ny-issue/assets/texts/about-300.png}

\hypertarget{new-york-stories}{%
\section{New York Stories}\label{new-york-stories}}

The magazine's first-ever all-comics issue, with 12 tales of the city
based on stories from The Times's metro desk.

By Kevin Huizenga

\hypertarget{view-finder-1}{%
\section{View Finder}\label{view-finder-1}}

After being forced to brick over his loft's beautiful view of the Hudson
River, a man devises a clever replacement. As drawn by Tom Gauld.

By Tom Gauld

This was the situation Dr. Richard Sloan found himself in a few years
ago. Most people in his place would curse their luck and move on. Dr.
Sloan, a biomedical researcher at Columbia University, is not like most
people.

It is a sad fact of life and real estate that you cannot control

what your neighbors do. One day you command a fourth-story view of the
Hudson River out a window that fills your loft with sunshine and fresh
breeze. The next day you're staring at a brick wall.

The loft had a wall of north-facing windows. But the most direct view of
the river was out a large west-facing window.

He and his wife bought the loft at the west end of Canal Street in 2005.
The only property between them and the West Side Highway was a gas
station that had been in continuous operation for decades.

We thought that since Exxon Mobil owned the property and made nearly 40
billion dollars in profit that year, it was a safe bet that they would
not need to sell it.

In 2012 Exxon Mobil sold the gas-station property. The new owners later
announced plans for a nine-story condo tower.

Dr. Sloan's north-facing windows were unaffected, but his west-facing
window was right up against the property line, and he was ordered to
brick it over.

We had dinner parties all the time. We'd sit at the table and look at
the Hudson.

Sometimes at sunset in

the summer, the light was absolutely blinding.

Over the years the view became even more impressive:

As the skyline on the west

side of the Hudson grew, the number of reflective buildings increased,
and at sunrise in

the winter, the sun would reflect off the buildings. So we got the
sunrise and the sunset.

Dr. Sloan duly walled off his window. But he had another idea.

Then he installed an 85-inch monitor where the west window used to be

and sent the camera image there.

I decided the next best thing to a real window would be a digital
window.

He mounted a wide-angle camera out a north-facing window and positioned
it to take in a view as close as possible to the vanished one out the
west window.

I gave up the conceit that this was a real window. It now just evokes a
window, so that's fine. It evokes the image that was lost.

No one would mistake the view coming off the pixels of the monitor for
the view coming off the river itself. Especially in the late afternoon,
when the colors grow washed-out rather than glorious. As dusk falls, the
image gets snowy.

I have to adjust the settings.

Dr. Sloan has thought about things he could do to make the fake window
more like a real window.

I'm very happy with it.

But if you could, magically, in a world of alternative facts, go back to
the real window, I would do so.

I was toying with the idea of putting a fan behind it, to get a breeze
for the window,

But it's too much trouble.

Instead, he has found himself embracing the possibilities of artifice.
Recently, Dr. Sloan has started Chromecasting college basketball games
onto the window.

I realized: 85-inch TV,

I might as well use it.

It is a sad fact of life and real estate that you cannot control what
your neighbors do. One day you command a fourth-story view of the Hudson
River out a window that fills your loft with sunshine and fresh breeze.
The next day you're staring at a brick wall.

This was the situation Dr. Richard

Sloan found himself in a few years ago. Most people in his place would
curse their luck and move on. Dr. Sloan, a biomedical researcher at
Columbia University, is not like most people.

He and his wife bought the loft at the west end of Canal Street in 2005.
The only property between them and the West Side Highway was a gas
station that had been in continuous operation for decades.

We thought that since Exxon Mobil owned the property and made nearly 40
billion dollars in profit that year, it was a safe bet that they would
not need to sell it.

The loft had a wall of north-facing windows. But the most direct view of
the river was out a large west-facing window.

We had dinner parties all the time. We'd sit at the table and look at
the Hudson.

Sometimes at sunset in

the summer, the light was absolutely blinding.

Over the years the view became even more impressive:

As the skyline on the west

side of the Hudson grew, the number of reflective buildings increased,
and at sunrise in the winter, the sun would reflect off the buildings.
So we got the sunrise and the sunset.

In 2012 Exxon Mobil sold the gas-station property. The new owners later
announced plans for a nine-story condo tower.

Dr. Sloan's north-facing windows were unaffected, but his west-facing
window was right up against the property line, and he was ordered to
brick it over.

Dr. Sloan duly walled off his window. But he had another idea.

I decided the next best thing to a real window would be a digital
window.

He mounted a wide-angle camera out a north-facing window and positioned
it to take in a view as close as possible to the vanished one out the
west window.

Then he installed an 85-inch monitor where the west window used to be

and sent the camera image there.

No one would mistake the view coming off the pixels of the monitor for
the view coming off the river itself. Especially in the late afternoon,
when the colors grow washed-out rather than glorious. As dusk falls, the
image gets snowy.

I have to adjust the settings.

I gave up the conceit that this was a real window. It now just evokes a
window, so that's fine. It evokes the image that was lost.

I'm very happy with it.

But if you could, magically, in a world of alternative facts, go back to
the real window, I would do so.

Dr. Sloan has thought about things he could do to make the fake window
more like a real window.

I was toying with the idea of putting a fan behind it, to get a breeze
for the window,

But it's too much trouble.

Instead, he has found himself embracing the possibilities of artifice.
Recently, Dr. Sloan has started Chromecasting college basketball games
onto the window.

I realized: 85-inch TV,

I might as well use it.

This collaboration between the Metro Desk and the Magazine appears for
the first time here.

\hypertarget{the-window-gazers}{%
\section{The Window Gazers}\label{the-window-gazers}}

Every summer, some New Yorkers take up spots at their windows and spend
their days watching the streets below. As drawn by Bill Bragg.

By Bill Bragg

\href{http://www.nytimes3xbfgragh.onion/2007/08/18/nyregion/18summer.html}{Based
on ``Raising Sash for a View of the City's Daily Ballet,'' by Andy
Newman and Cassi Feldman.}

\hypertarget{an-amiable-child}{%
\section{An Amiable Child}\label{an-amiable-child}}

The story of the small, precious, two-century-old monument tucked away
in a corner of Riverside Park. As drawn by Robert G. Fresson.

By Robert G. Fresson

\href{https://cityroom.blogs.nytimes3xbfgragh.onion/2008/08/12/to-the-memory-of-an-amiable-child/comment-page-1/}{Based
on ``To the Memory of an Amiable Child,'' by Corey Kilgannon.}

\hypertarget{twin-flames}{%
\section{Twin Flames}\label{twin-flames}}

A fortuneteller winds up costing a lovesick man a fortune. As drawn by
Tillie Walden.

By Tillie Walden

\href{https://www.nytimes3xbfgragh.onion/2015/06/06/nyregion/he-went-to-the-fortuneteller-now-his-fortune-is-gone.html}{Based
on ``A Manhattan Fortuneteller Cost Him Fortune After Fortune,'' by
Michael Wilson.}

\hypertarget{hot-fun}{%
\section{Hot Fun}\label{hot-fun}}

In the summer of 1991, New Yorkers flock to the Bronx's Orchard Beach.
As drawn by Wesley Allsbrook.

By Wesley Allsbrook

\href{http://www.nytimes3xbfgragh.onion/1991/07/01/nyregion/orchard-beach-offers-slice-of-paradise.html}{Based
on ``Orchard Beach Offers Slice of Paradise,'' by David Gonzalez.}

\hypertarget{missing}{%
\section{Missing}\label{missing}}

A Brooklyn neighborhood stops to search for a lost dog. As drawn by
Bianca Bagnarelli.

By Bianca Bagnarelli

\href{https://www.nytimes3xbfgragh.onion/2016/11/13/nyregion/world-or-at-least-brooklyn-stops-for-lost-dog.html}{Based
on ``World (or at Least Brooklyn) Stops for Lost Dog,'' by Andy Newman.}

\hypertarget{terror-and-mystery}{%
\section{Terror and Mystery}\label{terror-and-mystery}}

The last New Yorker killed on Sept. 11, 2001, was Henryk Siwiak --- shot
on a Brooklyn street, for reasons still unknown. As drawn by Sammy
Harkham.

By Sammy Harkham

\href{http://www.nytimes3xbfgragh.onion/2011/09/10/nyregion/brooklyn-murder-on-sept-11-2001-remains-unsolved.html}{Based
on ``Killed on Sept. 11, 2001, but Destined to be Mourned Only Quietly,
Only by a Few,'' by Michael Wilson.}

\hypertarget{the-laid-back-break-in}{%
\section{The Laid-Back Break-In}\label{the-laid-back-break-in}}

In Williamsburg, a low-energy intruder enjoys the pleasures of a
high-tech home. As drawn by K. L. Ricks.

By K.L. Ricks

\href{https://www.nytimes3xbfgragh.onion/2017/03/05/nyregion/intruder-brooklyn-apartment.html}{Based
on ``Low-Energy Intruder Hangs Out in a High-Tech Home,'' by Michael
Wilson.}

\hypertarget{the-birdmen-of-queens}{%
\section{The Birdmen of Queens}\label{the-birdmen-of-queens}}

On Sundays, men gather for a high-stakes battle: the singing
competitions of the chestnut-bellied seed finch. As drawn by Andrew Rae.

By Andrew Rae

\href{https://www.nytimes3xbfgragh.onion/2015/08/02/nyregion/tiny-birds-big-drama-inside-the-world-of-the-birdmen-of-queens.html}{Based
on ``Tiny Birds, Big Drama: Inside the World of the Birdmen of Queens,''
by Emily S. Rueb.}

\hypertarget{enemies-among-us}{%
\section{Enemies Among Us}\label{enemies-among-us}}

In 1942, Nazi saboteurs sneaked into New York and strolled through
Midtown, waiting to launch a terror plot. As drawn by Francesco
Francavilla.

By Francesco Francavilla

\href{http://www.nytimes3xbfgragh.onion/2002/01/17/nyregion/terrorists-among-us-1942-detecting-the-enemy-wasn-t-easy-then-either.html}{Based
on ``Terrorists Among Us (1942); Detecting the Enemy Wasn't Easy Then,
Either,'' by Andy Newman.}

\hypertarget{fake-notes}{%
\section{Fake Notes}\label{fake-notes}}

Two men (and a stack of counterfeit bills) head out for a big night in
the Village. As drawn by David Mazzucchelli.

By David Mazzucchelli

\href{https://www.nytimes3xbfgragh.onion/2015/03/28/nyregion/a-west-village-bar-crawl-leaves-behind-a-trail-of-counterfeit-green.html}{Based
on ``A West Village Bar Crawl Leaves Behind a Trail of Counterfeit
Green,'' by Michael Wilson.}

\hypertarget{more-on-nytimescom}{%
\subsection{More on NYTimes.com}\label{more-on-nytimescom}}

Advertisement

\hypertarget{site-information-navigation}{%
\subsection{Site Information
Navigation}\label{site-information-navigation}}

\begin{itemize}
\tightlist
\item
  \href{https://help.nytimes3xbfgragh.onion/hc/en-us/articles/115014792127-Copyright-notice}{©
  2020 The New York Times Company}
\item
  \href{https://www.nytimes3xbfgragh.onion}{Home}
\item
  \href{https://www.nytimes3xbfgragh.onion/search/}{Search}
\item
  Accessibility concerns? Email us at
  \href{mailto:accessibility@NYTimes.com}{\nolinkurl{accessibility@NYTimes.com}}.
  We would love to hear from you.
\item
  \href{https://help.nytimes3xbfgragh.onion/hc/en-us/articles/115015385887-Contact-Us}{Contact
  Us}
\item
  \href{https://www.nytco.com/careers/}{Work with us}
\item
  \href{https://nytmediakit.com/}{Advertise}
\item
  \href{https://help.nytimes3xbfgragh.onion/hc/en-us/articles/115014892108-Privacy-policy\#pp}{Your
  Ad Choices}
\item
  \href{https://help.nytimes3xbfgragh.onion/hc/en-us/articles/115014892108-Privacy-policy}{Privacy}
\item
  \href{https://help.nytimes3xbfgragh.onion/hc/en-us/articles/115014893428-Terms-of-service}{Terms
  of Service}
\item
  \href{https://help.nytimes3xbfgragh.onion/hc/en-us/articles/115014893968-Terms-of-sale}{Terms
  of Sale}
\end{itemize}

\hypertarget{site-information-navigation-1}{%
\subsection{Site Information
Navigation}\label{site-information-navigation-1}}

\begin{itemize}
\tightlist
\item
  \href{https://spiderbites.nytimes3xbfgragh.onion}{Site Map}
\item
  \href{https://help.nytimes3xbfgragh.onion/hc/en-us}{Help}
\item
  \href{https://help.nytimes3xbfgragh.onion/hc/en-us/articles/115015385887-Contact-Us?redir=myacc}{Site
  Feedback}
\item
  \href{https://www.nytimes3xbfgragh.onion/subscription?campaignId=37WXW}{Subscriptions}
\end{itemize}
