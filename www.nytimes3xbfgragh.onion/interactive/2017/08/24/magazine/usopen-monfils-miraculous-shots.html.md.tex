 **NYTimes.com no longer supports Internet Explorer 9 or earlier. Please
upgrade your browser.
\href{http://www.nytimes3xbfgragh.onion/content/help/site/ie9-support.html}{LEARN
MORE »}

**Sections

**Home

**Search

\hypertarget{the-new-york-times}{%
\subsection{\texorpdfstring{\href{http://www.nytimes3xbfgragh.onion/}{The
New York Times}}{The New York Times}}\label{the-new-york-times}}

\hypertarget{-magazine-}{%
\subsubsection{\texorpdfstring{
\href{https://www.nytimes3xbfgragh.onion/section/magazine}{Magazine}
}{ Magazine }}\label{-magazine-}}

 \href{https://www.nytimes3xbfgragh.onion/section/magazine}{Magazine}
\textbar{}Gaël Monfils Hits Miraculous Shots. Why Can't He Win?

**Close search

\hypertarget{site-search-navigation}{%
\subsection{Site Search Navigation}\label{site-search-navigation}}

Search NYTimes.com

**Clear this text input

Go

\url{https://nyti.ms/2vqLvc4}

\hypertarget{site-navigation}{%
\subsection{Site Navigation}\label{site-navigation}}

\hypertarget{site-mobile-navigation}{%
\subsection{Site Mobile Navigation}\label{site-mobile-navigation}}

\hypertarget{gauxebl-monfils-hits-miraculous-shots-why-cant-he-win}{%
\section{Gaël Monfils Hits Miraculous Shots. Why Can't He
Win?}\label{gauxebl-monfils-hits-miraculous-shots-why-cant-he-win}}

Of all the players struggling to win titles in this unkind era of the
so-called Big Four, he is the most captivating, maddening and
misunderstood.

\includegraphics{https://static01.graylady3jvrrxbe.onion/images/2017/08/17/magazine/27monfils1/27monfils1-master180.jpg}

Miracle Man

\textbf{In Goran Ivanisevic Stadium}, at an exhibition two nights before
the start of July's Croatia Open, Gaël Monfils made sure that fans
received the full entertainment value of their tickets. The 30-year-old
tennis star pretended he couldn't see lobs arcing above him, only to
leap into the air and spin, a complete 360, smashing the ball at the
peak of his turn, as if this were the most natural way to hit an
overhead. He raced from the baseline to the net, long galloping strides,
and slid low the last dozen feet on the red clay to gently kiss a drop
shot off his strings. He returned balls behind his back, between his
legs, off his head, against the grip end of his racket. He cranked back
his muscular right arm and uncorked a 110-miles-an-hour forehand winner.
Three thousand or so spectators were there in the tennis village along
the Adriatic Sea, in the resort town of Umag, and for the shot's seeming
impossibility they laughed.

Monfils (``Moan-feese'') was joined on center court by two Croatian pros
and David Goffin, an undersize Belgian with laser groundstrokes who was
seeded No. 1 at the tournament to Monfils's No. 2. But all eyes were
fixed on Monfils. He is a player who crackles with the possibility that,
at any instant, he may do something beyond the limits of physical laws
or human capabilities or merely the respectable conventions of tennis.
Even given the hyperathleticism of today's game, he is very likely the
fastest and strongest man on the tour --- perhaps also the most
coordinated and balanced and precise, with an expert marksman's aim. On
YouTube, a profusion of videos attest to his acrobatic and
improvisational feats, a mix of the comic and the sublime: ``Top 30 Only
Gaël Monfils Moments,'' ``Top 20 Miraculous Points,'' ``Top 10 Jumping
Winner Shots.'' The previous week, I was watching from home as he played
in the third round of Wimbledon. A serve pulled him far off the ad court
to his left. Monfils hit a two-handed backhand return and scurried
rightward toward the center line. His opponent, a lower-ranked French
player named Adrian Mannarino, took the return in the air, volleying it
behind Monfils. Tennis players reverse direction hundreds of times on
court, in every point, drill and side-shuffling exercise; it's as much a
part of their muscle memory as it is for any of us to sit or stand. But
Monfils whirled around clockwise, blindly away from the court, emerging
from this pirouette in one fluid motion with his backhand drawn back and
swinging, the shot sent down the line past the lunging Mannarino. It was
not a real-life tennis move. It was a move from a fast-cut Hollywood
fight scene.

Roger Federer has his own arsenal of tricky shots --- dunk-contest
flying overheads, wristy squash sidewinders and lob-chasing
backward-between-the-legs swipes, punctuated now and then with a flip of
his bangs or a modest fist pump, but usually executed with the placid
composure of a waiter in a high-end restaurant. And yet Federer also
sustains a general level of otherworldly virtuosity that has won him 93
career tournaments, including a men's record 19 in the four biggest
annual contests, known as the Grand Slams. For all his outsize talents,
Monfils's career, so far, could be rated a disappointment. He was the
sixth-best tennis player in the world last year (recently, he has been
in the teens and 20s), and he has won more than 400 matches and earned
some \$13.6 million in prize money and probably that much again in
endorsements and appearance fees. But he has never reached a Grand Slam
final and has won a total of only six professional tournaments. After he
hit that Jason Bourne backhand at Wimbledon, he lost the next point to
Mannarino on a lusterless return of serve. In fact, for stretches of the
match Monfils's magical elasticity and power simply disappeared. He
couldn't turn his shoulders or move his feet properly. A man who hits
some of the sport's hardest forehands managed only awkward defensive
chops. He lost in five sets to Mannarino.

Of all the players struggling to win titles in this unkind era of the
so-called Big Four --- Federer, Rafael Nadal, Novak Djokovic and Andy
Murray, who, along with the Swiss Stan Wawrinka, have hoarded 51 of the
last 55 Grand Slam trophies --- Monfils is the most captivating,
maddening and misunderstood. His critics see his dips in play and deride
him as unprofessional, saying he doesn't care. ``That's the time I care
most,'' Monfils told me. ``I'm a warrior. I'm trying to figure it out.''
I heard from tennis officials and coaches that he stays out all night
with friends, but he has never drunk alcohol or smoked a thing. He does
have hobbies --- he likes to make music, work on watches, play
basketball and read history --- but these pastimes, he says, have kept
him sane. ``They're not seeing the work and training,'' his coach,
Mikael Tillstrom, told me. Sure, Tillstrom would like it if Monfils
didn't switch to a left-handed overhead in a tiebreaker or hit look-away
half-volleys or feint that he's given up on a shot before tracking it
down to dupe his opponent. But just as often he pulls off these
miraculous shots. ``There is a thin line between genius and stupidity,''
Tillstrom added. ``Gaël is trying to win the point. Those shots just
come to him; he reacts.'' The player on tour who perhaps knows Monfils
best, his fellow Frenchman Gilles Simon, said: ``Gaël has to enjoy what
he's doing. It's who he is. It will never be effective if he is not
taking any pleasure in what he's doing.''

Monfils wants to be remembered as more than the ``human highlight
reel.'' He has a dream, modulated to the unyielding epoch in which he
has happened to play: He wants the fortnight's run of perfection and
luck needed to win at least a single Grand Slam. ``I think one time in
my life I can put myself in the position to be the best for two weeks,''
Monfils said. ``The best all time is Roger. My dream, my challenge, why
I'm training is to have within a year the chance to be the best for two
weeks.''

\subsection{}

In Ivanisevic Stadium for the exhibition, he was, arguably, the ultimate
professional, delighting the crowd as he did a running Gene Kelly vault
over the light-bulb mascot of the Croatian national power company, HEP.
A Balkan breakdancer windmilled on the clay (exactly what you think that
looks like), and Monfils took him on in a dance-off. If Federer is the
tennis Superman, with powers so great they can make plot lines a little
too predictable, Monfils is something else entirely. He is consumed by a
quest many doubt he is even on, and the phenomenal beauty and mastery of
his game may end up not being enough to overcome opponents or himself. I
bumbled through this theory the first time I sat down with Monfils on
the Istrian Peninsula. He chuckled and said: ``Welcome to my mind, my
head.''

\textbf{Monfils is French} but has lived in Switzerland since turning
pro in 2004. He considers himself also to be Caribbean; his mother is
from Martinique, and his father is from Guadeloupe. ``They gave me an
island education,'' he said. (Tattooed on his arms are the outlines of
both islands and the key to his grandmother's front door in Martinique.)
Monfils grew up in the 19th Arrondissement of Paris, a largely immigrant
neighborhood in the northeast of the city. He says he inherited his
athleticism from his father, Rufin, who came to France to play
professional soccer and later worked for a telecommunications company.
But Monfils chose tennis for its individuality, despite the children
outside their high-rise taunting him for playing what they called a
rich, white sport. ``I like to win and lose on my own,'' he explained.
From his mother, Sylvette, a retired nurse who is about to move back to
her native island, Monfils says he learned that tennis is only a game.
At the end of even the most bitter losses, he embraces his opponents as
if genuinely happy for them. He is known to crack up the stoic umpires
during matches and correct calls in favor of his opponent.

Monfils says the French Tennis Federation at first refused to provide
him with coaching and other financial help --- he never figured out why
--- but at 13 he was ``the best, so they had to help me.'' He left home
and never returned, going first to one federation center outside Paris
and then another. He grew up at these tennis schools with Simon, Richard
Gasquet and Jo-Wilfried Tsonga, all of them turning pro (and eventually
becoming Swiss residents) and remaining close friends. In 2004, at 17,
Monfils won the junior singles events at the Australian Open, the
\href{http://topics.nytimes3xbfgragh.onion/topics/reference/timestopics/subjects/f/french_open_tennis/index.html?inline=nyt-classifier}{French
Open} and Wimbledon. When I asked him what he considered his greatest
professional match, he answered before I finished the question. It was
later in 2004, at the Paris Masters, against Lleyton Hewitt, then the
world's No. 3. Monfils lost 6-3, 7-6 and hit 46 unforced errors. But the
energy of the huge home crowd, the closeness of the second set --- ``it
was unreal,'' he said. ``That was why I have to play against those guys
every time.'' He beat his first Top 10 player soon after that, and by
the next year he was ranked in the Top 30 himself.

The commentators during his matches almost always joke that Monfils must
have driven his former coaches crazy. But like his peers on the tour,
his coaches adore him. If Monfils is quirky, he is also kind, loyal and
eager for the guidance and strenuous work. ``He wasn't contacting me
because I was available,'' said Roger Rasheed, an Australian who coached
Hewitt, referring to the cold call he received in 2008, when Monfils's
ranking had stalled around 30. ``He was looking for something that would
make him improve. He knew I was a hard-ass.'' The work they did together
over the next three years was as much mental as it was physical and
strategic. ``I get people to understand what elite behavior looks
like,'' Rasheed explained. `` `Am I behaving in an elite way that helps
me use my skill set in the best way and use it as often as I can?' I
teach them to live in that space.'' He guided Monfils into the Top 10,
working him as if he were in the finals of a Grand Slam and making him
understand how intimidated opponents would be if he was completely
committed to every point.

I played junior tennis and lower-division college, and many times I
wished for someone like Roger Rasheed, someone who would train me like
an elite player. But at other, clearer moments, I knew I was lying to
myself. ``Not a lot of people want to go in that space,'' Rasheed said.
It is a dark place. The single-minded devotion required to master the
technique and timing is not only life-sapping; it makes the stakes of
failure so much higher. Trying to concentrate shot after shot is a kind
of mania in itself. In tennis, you're alone on the court, exposed,
battling your own body and mind even as you're battling your opponent
across the court, someone else who has been taught to live in that elite
space since the age of 9. After an especially long point, your legs
wobbly and your lungs pleading for air, you tell yourself you'll play as
many points like that as it takes to win, but more frequently a tiny
part of you concedes. Tennis is like boxing in that you have to be
willing to take a punch to deliver a blow yourself. The natural reaction
is to recoil from the pain.

\includegraphics{https://static01.graylady3jvrrxbe.onion/images/2017/08/27/magazine/27monfils2/27monfils2-master180.jpg}

``I think some players have put the cue in the racks and decided to be
comfortable around the Big Four,'' Rasheed said. Unconsciously and not,
they have been stripped of conviction, questioning whether it is worth
the herculean effort to commit like those guys. On every single shot,
Nadal looks as if he's hurling a life preserver a thousand feet to
rescue his drowning mother. To challenge for slams, Djokovic adopted an
asceticism that had him forgo gluten, dairy and sugar. And Murray in his
pursuit stalks the court between points in a muttering and barking rage.
Monfils told me that he has always believed in himself, but you can hear
how that belief has been tempered for this era. ``I played the French
Open 12 times,'' he told me. ``Nine times it's Rafa who wins. Roger just
caught his 19th slam.'' He went on: ``They're legends. Sometimes people
don't like the truth. The truth is these guys were just better than
me.''

\textbf{The day after} the exhibition was the Wimbledon final, but
Monfils took to the courts beyond the walls of the stadium to practice.
Other players filled the courts around him. Croatians on holiday sat in
their swimwear on the concrete slabs along the sidelines watching as
Monfils alternated side to side, hitting two forehands, two backhands.
He directed each ball back to a 20-year-old named Marc Polmans, No. 207
in the Association of Tennis Professionals rankings, an Australian by
way of South Africa. Monfils's forehand is a big, menacing wallop, with
his right leg planted far behind him and his right arm yanked back above
his shoulder like the exaggerated windup of a Vaudeville slap. With
lashing uppercuts, he can topspin the ball into lovely parabolic flights
or flatten out the arc, hitting harder with less margin for error,
sometimes jumping with a balletic scissor kick to generate even more
force. His two-handed backhand is simpler, his swiveled torso remarkably
still as his arms swiftly barrel-toss forward. Monfils was getting his
feel and wind back after a few days of rest following his loss to
Mannarino; jogging along the Adriatic that morning, I passed him running
intervals. He was acclimating to the clay as well. (The Croatian Open is
a lower-level A.T.P. tournament, and it's also part of a blip in the
regimented tennis year --- a three-week mini clay season in Europe
between the grass of Wimbledon and the hard courts in North America that
culminate at the
\href{http://topics.nytimes3xbfgragh.onion/top/reference/timestopics/subjects/u/united_states_open_tennis/index.html?inline=nyt-classifier}{U.S.
Open}.) In one drill, Monfils sprinted the 78 feet from baseline to
baseline, sliding as he made his turn, sprinted back and then jumped
into the point against Polmans.

Monfils has doleful saucer eyes, a high forehead and a riot of short
dreadlocks that seem to be pointing him in every direction. At 6-foot-4,
he is big even for, say, an N.F.L. wide receiver. As large and powerful
as he is, though, he plays a defensive style of tennis. He is
comfortable seven feet behind the baseline, chasing down balls, waiting
for his opponent to miss or leave himself open to attack. And because
he's so fast and agile, Monfils gets to balls no one else can, turning
desperation into aggression. Many players in today's power baseline game
take short balls and back up, in the same way the N.B.A.'s Stephen Curry
flashes to the three-point line rather than head to the rim for a layup.
Why go where you're not at your best when you can crush an inside-out
forehand harder than you hit an overhead? ``All my career they say, `You
have to be more aggressive, come to the net. You're so athletic,' ''
Monfils said. Very few players serve and volley with regularity. The
opponents are too strong, the rackets and strings too advanced, the
returns too fast. Federer epitomizes the modern offensive style; he
plants himself on the baseline and moves inexorably forward with each
shot, looking to strike a winner. That's what coaches have wanted
Monfils to do. `` `Buddy, calm down, I cannot volley,' '' Monfils said,
recalling his years of exasperated responses. `` `I'm not like that. I
won't change for you.' ''

Tillstrom, whom Monfils hired at the end of 2015, told me by phone that
he has been trying to get Monfils to put more pressure on his opponents,
to not settle for trading shots from deep in the court. He should use
his powerful serve and groundstrokes, Tillstrom said, to set up his
improved net game and to put away short forehands. Tillstrom wants
Monfils to focus more intently during the early rounds of tournaments,
which could shorten his matches against lesser players and help him
retain strength for when he finally faces the guys at the top. It is
also a way to avoid injury for a player who has been sidelined by bad
knees, ankles and wrists, some of the damage due to sliding on hard
courts and launching himself fully horizontal on dives.

Gilles Simon, who first met Monfils when he was 14 and Monfils was 12,
has an amazing mind for tennis and will almost certainly be a coach one
day. He can remember the final score of nearly every pro match Monfils
has played, and in each case he can offer an analysis of how his friend
might have better utilized his huge strengths and protected his
weaknesses. (He has beaten Monfils seven of the nine times they've
played as professionals.) ``The emotional part with Gaël is much more
important than with any other player: It's almost all of it,'' he told
me in the players' locker room in Umag, as Monfils was passing by. ``He
doesn't like to win 6-1, 6-1. He needs some drama at some point. The
showman who likes to play tricky shots will show up, and he forgets to
win the match.'' His advice was similar to Tillstrom's. `` `Bro, kill
the guy when you have to kill them, and don't play your {[}expletive{]}
on court. Because you are so super strong, everyone is afraid to play
you. But everyone knows they are going to come back against you because
you are going to make them come back in the match.' ''

\includegraphics{https://static01.graylady3jvrrxbe.onion/images/2017/08/27/magazine/27monfils3/27monfils2-master180.jpg}

After two hours of practicing with Polmans, Monfils called it quits. But
first he rallied with a 12-year-old top junior who was taking in the
tournament. The boy's father had asked the night before, and of course
Monfils agreed. After 15 minutes, Monfils even shouted, ``First to 11,''
and they played groundstroke games. ``Am I going to win?'' Monfils
clowned. ``I don't know.'' I asked him later if there was a pro who hit
with him when he was that age. He hit one shot against Venus Williams,
he said. He was 11, and she 17 and already a U.S. Open finalist, and
Monfils's father took him to a clinic she was conducting. ``I drop-shot
her,'' Monfils announced proudly. ``I was so happy. Obviously, she
didn't run after it. I remind her of it, I say: `Yeah, I won a point off
Venus Williams.' ''

\textbf{Monfils will turn} 31 during this year's U.S. Open. No longer is
that old for professional tennis. The top five men are all 30 or older;
Federer is 36. Monfils has added injury-prevention exercises with
resistance bands and balance balls to his daily routine, and Asics just
extended its endorsement deal with him for another five years. ``What
goes down over time is the traveling part and the separation from
private life,'' he said. ``This is getting tougher.'' To remedy this in
Croatia, he was staying in a villa with one his best friends from
Geneva; next door was his brother, Daryl, who was the top singles player
at Virginia Commonwealth University last year (they have a sister, too,
who is a promising junior in France), and his cousin Jeremy Filet. When
we talked at the villa, Monfils beckoned over his cousin, an aspiring
hip-hop artist whose nom de rap is Yung Fille. Monfils played a beat he
created, and Filet rapped over it, with Monfils adding exclamations at
the end of each line.

A conversation with Monfils is a physical activity. He speaks with his
entire body, bobbing in and out as if dodging jabs. After making a
point, he exhaled loudly, a Gallic fuhgeddaboudit, and placed his cupped
hands on the table in front of us and slid them around for emphasis, as
if he were a sharpie doing the shell game. He took out his phone and
showed me the stream of racist posts on social media he received after
every loss. ``It's the way people think and the way life is,'' he said,
shrugging. His tennis idol is Arthur Ashe. He didn't mean Ashe's style
of play, which he'd never seen. ``The legacy he left behind,'' Monfils
said. ``What he did for Yannick Noah,'' whose father was from Cameroon
and who is the last Frenchman to win a slam, back in 1983. (Ashe was on
a good-will tour in Africa when he hit with Noah, then 11, and soon
helped enroll him in a French tennis academy.)

Tillstrom phoned, and Monfils excused himself. It turned out they were
talking about a personal matter, not a tennis one --- Monfils needed
some advice about a gift for his girlfriend. That redirected our
conversation to watchmaking. His Swiss girlfriend works for an auction
house dealing in fine watches, and Monfils has been training as a
watchmaker, thinking he might work in the luxury-watch industry after he
retires from tennis. He searched for the English names for the pieces of
the watch assembly. \emph{Le rouage}? \emph{L'échappement}? He is
fascinated by the role of each part in the mechanism, the wheels,
levers, springs, every component interlocking and moving in sync.
``Everybody thinks I'm not focused, and I'm very focused when I do my
watch,'' he said.

The next day I spoke on the phone with Eric Winogradsky of the French
Tennis Federation, who coached Monfils for a couple of months in 2013.
It was a low point for Monfils, who was coming back from a knee injury
and had fallen out of the Top 100. The two of them went for long walks,
discussing life as well as the game. ``Gaël's feelings are not common,''
Winogradsky told me. ``He is a nice and really sensitive person. He has
to try all these things. That's what makes Gaël so interesting and
complex.'' To win a Grand Slam, Winogradsky believes a player must
commit to it religiously for a year to a year and a half, building a
rhythm and intensity, going deep in every tournament to get used to
competing against the top guys time and again. For that stretch, the
off-court pastimes needed to be ignored.

\hypertarget{-1}{%
\subsection{}\label{-1}}

``Are you saying he would be a better tennis player if he was less
interesting?'' I asked Winogradsky. The question seemed to pain him, and
he thought in silence on the other end of the line.

``That we won't be able to answer until the very end of his career. If
next month he wins a Grand Slam. . . .''

\textbf{One afternoon in Umag,} Monfils and Goffin played a practice set
on center court. The stands were empty save for the men power-washing
the seats before the night's matches, and for me. Watching these two up
close, I could see the utter freakishness of their foot and racket
speed. They glided around the court, sliding ``Matrix''-like to catch up
with balls that were already past them. They scooted around backhands to
unload on forehands and took full home-run cuts on their returns of
speeding, skidding serves. It was hot and dry, and clouds of red clay
swirled around them. Near the end of the set, Monfils couldn't miss. He
hit winners from both wings. He ran around one backhand and, crouching,
waited to hit a forehand with his back foot in the doubles alley,
leaving the rest of his side of the court exposed. But he yanked the
ball across his body, jumping as he hit a down-the-line screecher so
obscenely hard that the scrambling, sliding Goffin wasn't within five
feet when the winner caromed off the back wall. This was kill-the-guy,
no-nonsense Gaël. This was competing-in-Grand-Slams Monfils.

Two days later, after a first-round bye, Monfils played his first match
of the Croatia Open. His opponent was Rogerio Dutra Silva, a 33-year-old
Brazilian ranked No. 64 in the world who lost only three games in his
first-round win. Clay is Dutra's best surface, and by the second game he
and Monfils were already locked in 25-shot rallies, the pendulum
patterns of their groundstrokes making them look as if they were running
suicides sideline to sideline. Ten minutes in, they each struggled for
breath, with runnels of sweat rolling off them. On the pro tour, it's
not just the Big Four or the Top 20 who are a threat. Monfils drifted
farther behind the baseline, relying on his side-to-side speed, sliding
almost into the splits as he wound up for his huge forehand, sending the
lines judges fleeing. The fans cheered him, chanting: ``Guy-el!
Guy-el!'' The first set went into a tiebreaker. Monfils netted an
inside-out forehand and lost the breaker 5-7.

Monfils has many on-court oddities. On his serve, he hurries to start
each point, as if late for an appointment. On his return of serve, he
bends so far at the waist that he has to look up, hunched lower than the
line judges behind him, like someone nearsighted trying to read the
prices of the meats in the deli counter. After long points, he doubles
over, leaning on his racket like a cane. He ties and reties his
shoelaces with single knots. On changeovers, he sits on his side of the
net, his eyes scanning the stands, watching people as they moved about
--- the opposite of tunnel vision. At a couple of moments during the
match, after endurance-test rallies, he spotted me in my seat along the
far sideline. Feeling awkward, worried about jinxing him, I was the one
to break eye contact. Or maybe I imagined this, and it's just that he
makes everyone feel as if they're struggling through the drama with him.

\hypertarget{-2}{%
\subsection{}\label{-2}}

Between the sets, the clay was dragged, the lines were swept, the court
was made pristine like an unblemished beach of cinnamon. When Monfils
stepped onto it for the second set, the player who appeared was the one
from the practice set with Goffin. He jumped inside the baseline on
Dutra's serve, pounding the balls down the line, gaining the advantage.
He used his shots not to extend the rallies but as one-two combinations,
setting up the knockout blows. He went ahead 4-0 and held on to win the
set 6-4.

The second point of the deciding set lasted 34 shots, Monfils ending it
by winding up for a high backhand from the corner and stopping his swing
to slink the ball delicately over the net. ``Allez, Monfils!'' people
shouted. I have to imagine that what draws people to Monfils is not only
his superhuman ability but also the fickleness of his powers. On a
ho-hum rally ball, he hit a forehand long. Then he charged to the net to
put away a volley. Then he double-faulted. Then he froze Dutra on a drop
shot that seemed to break Newton's laws of motion. Then he tossed in an
84-miles-an-hour first serve that Dutra destroyed. Then he aced Dutra at
123 m.p.h. That was just one game. It was the human condition.

After more than two hours of play, with the score tied 3-3 in the last
set, Monfils had a break point to win the decisive game. He batted his
return of serve crosscourt at an extreme angle. It was a perfect shot
that Dutra had to stretch and reach low for on his one-handed backhand
10 feet off the court. Monfils moved forward to guard against a
desperation crosscourt chop. Instead, Dutra dug up the ball and sent it
up over the highest part of the net, hitting the sideline deep on the
other side. It was video-game tennis, and Monfils, unable to reach the
ball, screeched and jerked back as if he'd been electrocuted. He clapped
a hand on his racket face, and Dutra acknowledged it with a quick
thumbs-up reply. Dutra hit lines on two more points, and Monfils,
flummoxed, planted himself on the edge of the stands, staring imploring
at his family and friends.

Monfils double-faulted three consecutive times to fall behind 3-5, then
failed to convert on three more break points the following game. He lost
7-6 (5), 4-6, 6-3. He draped an arm around Dutra and congratulated him.
He smiled as he shook the umpire's hand. ``Today wasn't a good day,''
Monfils would say in a news conference later. ``I gave my best.'' But
before that, he asked the tournament director to turn on the lights on
one of the side courts, the one closest to the Dance Arena, the Istria
Gourmet Zone and the gleaming white stalls selling cigarettes. He
practiced serves, cracking one after the other. His cousin popped open a
case of balls, filling the box with them, and stood at half court,
feeding forehands. Monfils blasted each one harder than the last. A
hundred people gathered in the darkness surrounding the court, watching
in silence at what looked like punishment. With each of his shots, the
wheezing exhaust from Monfils's lungs was audible. He rested
momentarily, his head slung low, then hit some more. I felt embarrassed,
but also moved, to see him so exposed.

After 30 minutes more, Monfils limped off the court. He had nowhere to
go but through the crowd, which now demanded the exhibition man.
Children held up giant novelty tennis balls for autographs. Monfils,
red-eyed, signed them all. People pressed against him for selfies,
smiling at the phones lofted in their outstretched arms, twisting their
faces into goofy expressions. ``Straight away after the match, I wanted
to address everything and work on it,'' Monfils explained at the news
conference. The summer would be long, and he headed next to Washington
and Montreal and on to the U.S. Open. It was there, in 2014, that he was
perhaps closest to fulfilling his dream. He won his early rounds without
dropping a set. Nadal was out with an injury, and Murray, Djokovic and
Wawrinka would all be beaten on the opposite side of the draw. In the
quarterfinals against Federer, Monfils held two match points. He was
unable to convert.

``Some people will say, `I don't want to lose time,' '' Monfils told me
when he described his preoccupation with watches and their hold on our
sense of our lives. ``They are chasing something. You have all the time
to do it. Or you're wasting time.''

I asked him if he was also talking about himself chasing a Grand Slam,
that his own time to do it was running out.

``Yeah, that can be true,'' he said. He held up his open hands and
smiled. ``Or not.''

Ben Austen is the author of ``High-Risers: Cabrini Green and the Fate of
American Public Housing,'' to be published in February. He last wrote
for the magazine about
\href{https://www.nytimes3xbfgragh.onion/2017/02/23/magazine/the-new-working-class.html}{new
American jobs}.

\emph{\href{http://www.nytimes3xbfgragh.onion/newsletters/magazine}{Sign
up for our newsletter} to get the best of The New York Times Magazine
delivered to your inbox every week.}

\hypertarget{more-from-the-us-open-issue}{%
\subsubsection{More from the U.S. Open
Issue}\label{more-from-the-us-open-issue}}

\href{https://www.nytimes3xbfgragh.onion/interactive/2017/08/24/magazine/usopen-grand-slam-wimbledon-pleasures-pitfalls.html}{}

\includegraphics{https://static01.graylady3jvrrxbe.onion/images/2017/08/27/magazine/27tournaments1/27mag-27tournaments.t_CA1-master180.jpg}

\hypertarget{courtside-chronicles}{%
\subsection{Courtside Chronicles}\label{courtside-chronicles}}

The Peculiar Pleasures of the Grand Slam Stands

\href{https://www.nytimes3xbfgragh.onion/interactive/2017/08/24/magazine/usopen-federer-nadal-backhand-wonder-year.html}{}

\includegraphics{https://static01.graylady3jvrrxbe.onion/images/2017/08/27/magazine/27federer1/27federer1-master180.jpg}

\hypertarget{wonder-year}{%
\subsection{Wonder Year}\label{wonder-year}}

How Roger Federer Upgraded His Game

\href{https://www.nytimes3xbfgragh.onion/interactive/2017/08/24/magazine/usopen-tall-players-zverev-tennis-future.html}{}

\includegraphics{https://static01.graylady3jvrrxbe.onion/images/2017/08/27/magazine/27tallplayers1/27tallplayers1-master180.jpg}

\hypertarget{altitude-adjustment}{%
\subsection{Altitude Adjustment}\label{altitude-adjustment}}

Will Tall Players Dominate the Future of Tennis?

\hypertarget{-3}{%
\subsection{}\label{-3}}

\begin{itemize}
\item
  \href{https://www.nytimes3xbfgragh.onion/interactive/2017/08/24/magazine/usopen-federer-nadal-backhand-wonder-year.html}{}

  \includegraphics{https://static01.graylady3jvrrxbe.onion/images/2017/08/27/magazine/27federer1/27federer1-mediumThreeByTwo225.jpg}

  \hypertarget{how-roger-federer-upgraded-his-game}{%
  \subsection{How Roger Federer Upgraded His
  Game}\label{how-roger-federer-upgraded-his-game}}

  Oct. 25, 2017
\end{itemize}

Advertisement

\hypertarget{site-information-navigation}{%
\subsection{Site Information
Navigation}\label{site-information-navigation}}

\begin{itemize}
\tightlist
\item
  \href{https://help.nytimes3xbfgragh.onion/hc/en-us/articles/115014792127-Copyright-notice}{©
  2020 The New York Times Company}
\item
  \href{https://www.nytimes3xbfgragh.onion}{Home}
\item
  \href{https://www.nytimes3xbfgragh.onion/search/}{Search}
\item
  Accessibility concerns? Email us at
  \href{mailto:accessibility@NYTimes.com}{\nolinkurl{accessibility@NYTimes.com}}.
  We would love to hear from you.
\item
  \href{https://help.nytimes3xbfgragh.onion/hc/en-us/articles/115015385887-Contact-Us}{Contact
  Us}
\item
  \href{https://www.nytco.com/careers/}{Work with us}
\item
  \href{https://nytmediakit.com/}{Advertise}
\item
  \href{https://help.nytimes3xbfgragh.onion/hc/en-us/articles/115014892108-Privacy-policy\#pp}{Your
  Ad Choices}
\item
  \href{https://help.nytimes3xbfgragh.onion/hc/en-us/articles/115014892108-Privacy-policy}{Privacy}
\item
  \href{https://help.nytimes3xbfgragh.onion/hc/en-us/articles/115014893428-Terms-of-service}{Terms
  of Service}
\item
  \href{https://help.nytimes3xbfgragh.onion/hc/en-us/articles/115014893968-Terms-of-sale}{Terms
  of Sale}
\end{itemize}

\hypertarget{site-information-navigation-1}{%
\subsection{Site Information
Navigation}\label{site-information-navigation-1}}

\begin{itemize}
\tightlist
\item
  \href{https://spiderbites.nytimes3xbfgragh.onion}{Site Map}
\item
  \href{https://help.nytimes3xbfgragh.onion/hc/en-us}{Help}
\item
  \href{https://help.nytimes3xbfgragh.onion/hc/en-us/articles/115015385887-Contact-Us?redir=myacc}{Site
  Feedback}
\item
  \href{https://www.nytimes3xbfgragh.onion/subscription?campaignId=37WXW}{Subscriptions}
\end{itemize}
