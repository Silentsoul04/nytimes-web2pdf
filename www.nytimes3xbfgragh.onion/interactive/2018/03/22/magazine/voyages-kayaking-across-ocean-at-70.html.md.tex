 **NYTimes.com no longer supports Internet Explorer 9 or earlier. Please
upgrade your browser.
\href{http://www.nytimes3xbfgragh.onion/content/help/site/ie9-support.html}{LEARN
MORE »}

**Sections

**Home

**Search

\hypertarget{the-new-york-times}{%
\subsection{\texorpdfstring{\href{http://www.nytimes3xbfgragh.onion/}{The
New York Times}}{The New York Times}}\label{the-new-york-times}}

\hypertarget{-magazine-}{%
\subsubsection{\texorpdfstring{
\href{https://www.nytimes3xbfgragh.onion/section/magazine}{Magazine}
}{ Magazine }}\label{-magazine-}}

 \href{https://www.nytimes3xbfgragh.onion/section/magazine}{Magazine}
\textbar{}Why He Kayaked Across the Atlantic at 70 (for the Third Time)

**Close search

\hypertarget{site-search-navigation}{%
\subsection{Site Search Navigation}\label{site-search-navigation}}

Search NYTimes.com

**Clear this text input

Go

\url{https://nyti.ms/2FWJhva}

\hypertarget{site-navigation}{%
\subsection{Site Navigation}\label{site-navigation}}

\hypertarget{site-mobile-navigation}{%
\subsection{Site Mobile Navigation}\label{site-mobile-navigation}}

\hypertarget{why-he-kayaked-across-the-atlantic-at-70-for-the-third-time}{%
\section{Why He Kayaked Across the Atlantic at 70 (for the Third
Time)}\label{why-he-kayaked-across-the-atlantic-at-70-for-the-third-time}}

For Aleksander Doba, pitting himself against the wide-open sea ---
storms, sunstroke, monotony, hunger and loneliness --- is a way to feel
alive in old age.

\hypertarget{voyages}{%
\subsection{Voyages}\label{voyages}}

\begin{itemize}
\tightlist
\item
  \href{https://www.nytimes3xbfgragh.onion/interactive/2018/03/22/magazine/voyages-kayaking-across-ocean-at-70.html?action=click\&module=MagazineModule\&pgtype=Article\&contentCollection=Magazine\&region=Header}{}
\item
  \href{https://www.nytimes3xbfgragh.onion/interactive/2018/03/22/magazine/voyages-worlds-greatest-hitchhiker.html?action=click\&module=MagazineModule\&pgtype=Article\&contentCollection=Magazine\&region=Header}{}
\item
  \href{https://www.nytimes3xbfgragh.onion/interactive/2018/03/22/magazine/voyages-nathalie-cabrol-searching-mars-life-on-earth.html?action=click\&module=MagazineModule\&pgtype=Article\&contentCollection=Magazine\&region=Header}{}
\item
  \href{https://www.nytimes3xbfgragh.onion/interactive/2018/03/22/magazine/voyages-joshua-tree-lost-hiker.html?action=click\&module=MagazineModule\&pgtype=Article\&contentCollection=Magazine\&region=Header}{}
\end{itemize}

2018

\includegraphics{https://static01.graylady3jvrrxbe.onion/images/2018/03/25/magazine/25mag-kayaker1/25mag-kayaker1-master180.jpg}

\hypertarget{why-he-kayaked-across-the-atlantic-at-70-for-the-third-time-1}{%
\section{Why He Kayaked Across the Atlantic at 70 (for the Third
Time)}\label{why-he-kayaked-across-the-atlantic-at-70-for-the-third-time-1}}

\subsection{}

Alone at Sea

By ELIZABETH WEIL\\
Photographs by JOAKIM ESKILDSEN MARCH 22, 2018

\textbf{When Aleksander Doba} kayaked into the port in Le Conquet,
France, on Sept. 3, 2017, he had just completed his third --- and by far
most dangerous --- solo trans-Atlantic kayak trip. He was a few days shy
of his 71st birthday. He was unaccustomed to wearing pants. He'd been at
sea 110 days, alone, having last touched land that May at New Jersey's
Barnegat Bay. The trip could have easily ended five days earlier, when
Doba was just a few hundred feet off the British coast. But he had
promised himself, when he left New Jersey, that he would kayak not just
to Europe but to the Continent proper. So he stayed on the water nearly
another week, in the one-meter-wide boat where he'd endured towering
waves, in the coffinlike cabin where he spent almost four months not
sleeping more than three hours at a stretch, where he severely tried his
loved ones' patience in order to be lonely, naked and afraid. Then he
paddled to the French shore.

\hypertarget{kayaking-the-ocean}{%
\subsection{Kayaking the Ocean:}\label{kayaking-the-ocean}}

First Expedition

\includegraphics{https://static01.graylady3jvrrxbe.onion/newsgraphics/2018/03/19/voyages-2018-maps/assets/kayaker1-300.png}

North

Atlantic

Ocean

Dakar,

SENEGAL

Africa

1

AcaraÚ,

Brazil

2

South

America

South

Atlantic

Ocean

\hypertarget{second-expedition}{%
\subsection{Second Expedition}\label{second-expedition}}

\includegraphics{https://static01.graylady3jvrrxbe.onion/newsgraphics/2018/03/19/voyages-2018-maps/assets/kayaker2-300.png}

Atlantic

Ocean

North

America

Lisbon,

Portugal

1

Bermuda

Florida

2

3

Africa

South

America

\hypertarget{third-expedition}{%
\subsection{Third Expedition}\label{third-expedition}}

\includegraphics{https://static01.graylady3jvrrxbe.onion/newsgraphics/2018/03/19/voyages-2018-maps/assets/kayaker3-300.png}

Atlantic

Ocean

North

America

2

Le Conquet,

France

1

New Jersey

Africa

South

America

Kayaking is an absurd form of long-distance ocean travel. All the big
muscles in the body are useless. ``A real \emph{katorga},'' says Doba,
who is Polish --- \emph{katorga} being the Polish word for forced labor
in Siberia. But by \emph{katorga}, Doba does not mean an activity he
does not wish to do. What most of us experience as suffering he
repurposes as contrarian self-determination, and that gives him an
existential thrill. Among Doba's bigger regrets in life are the times
when he has succumbed, when he has perceived and reacted to suffering in
conventional ways --- for instance, the night in April 1989 when he
built a fire in order to make tea and dry his clothes while paddling on
the Vistula River near the city of Plock, in central Poland. Or the
afternoon, a week later, on that same river, when he succumbed to the
temptation of eating pancakes, tomato soup and rice at the Milk Bar
restaurant when he should have been at his campsite, by his kayak,
eating cold canned goulash in order to condition his body for arctic
temperatures. Doba had promised himself he would be tougher than all
that.

Doba maintains that his need to cross the Atlantic in a kayak did not
originate within him. ``With my hand on my heart, it wasn't my idea,''
Doba told me when I met him in Poland in January. (Doba doesn't speak
much English, so we communicated through a translator.) ``I was infected
with a virus.'' In 2003, when he was already the most seasoned kayaker
in Poland, a Polish professor approached him to get advice on his quest
to kayak across the Baltic Sea. The professor eventually persuaded Doba
to cross the southern Atlantic with him from Ghana to Brazil in separate
one-man kayaks, lashing those kayaks together at night to make a
platform on which to sleep. The trip was a fiasco. Forty-two hours after
leaving, they washed up back on the beach.

Doba flew back to Poland; returned to his hometown, Police, in the
country's northwest, where he had been managing maintenance and repairs
at an enormous chemical factory; and swore off kayaking with a partner
ever again. Then Doba sketched out a design for a new boat that could
make the trip. He knew that the kayak needed to be unsinkable, as well
as self-righting, in the event that it capsized, and that it needed
lockers to store food and a cabin in which to sleep. Sketch in hand,
Doba drove from Police to Szczecin, the regional capital, and approached
a yacht-builder named Andrzej Arminski. Arminski agreed to build the
boat, and in spring of 2010, Olo, as Doba named his kayak --- after his
own nickname, Olek --- was complete. Doba told his wife that he was
going to try to cross the Atlantic again.

\includegraphics{https://static01.graylady3jvrrxbe.onion/images/2018/03/25/magazine/25mag-kayaker2/25mag-25kayaker-t_CA2-master180.jpg}

Only one person had ever crossed the Atlantic in a kayak using solely
muscle power, and he traveled island to island, from Newfoundland to
Ireland. Doba's goal was to go continent to continent between the
mainlands, from Senegal to Brazil, unsupported. This time Doba's trip
was far more successful, which is not to say it was pleasant. The
weather was disgusting --- humid and hot. Doba tried to sleep during the
day but couldn't, so he tried to paddle during the day and nearly got
sunstroke. He kept no schedule. ``I am not German --- always 9 a.m.
paddle,'' he explained. ``I am Polish. I paddle when I would like.'' His
skin broke out in salt-induced rashes, including blisters in his armpits
and groin. His eyes blew up with conjunctivitis. His fingernails and
toenails just about peeled off. His clothes, permeated with salt,
refused to dry. The fabric smelled horrendous and aggravated his skin,
so he abandoned clothes.

Ocean kayaking is catastrophically monotonous. The primary challenge is
not physical. Doba describes the tedium as a form of dementia:
``Hundreds, thousands, maybe millions of repetitions. The brain is
removed from the process.'' Alone at sea without his hearing aids, Doba
joked, he grew so disoriented that he started shouting at himself ``so
that I could hear.'' (Doba is fairly deaf but didn't bring the aids
along because they're expensive and not waterproof, and there was no one
to talk to anyway.) He intended to keep muscle tone in his legs by
swimming, but he had to abort that plan because his body in the water
attracted sharks. He was assaulted by hailstorms of flying fish. ``Do
you know how fast they go?'' he said. ``This does not feel good.''

When he couldn't sleep, because of the unrelenting stuffiness of his
cabin and the waves crashing through the portal onto his head, Doba
thought about his wife, children and his young granddaughter. He thought
about his dead parents. He communed with the turtles, whose shells he
tapped while they swam alongside him to make sure they were alive, and
the birds, who landed on Olo for a rest and often entered his cabin and
did not want to leave. He did have a satellite phone, and he texted with
Arminski, who, as his trip navigator, sent a regular forecast for wind
and weather. Doba also called his wife, twice. But after she got the
bill for \$500, she says, ``the desire to talk'' decreased.

Doba rotated through three kinds of freeze-dried porridge for breakfast,
four kinds of freeze-dried soup for lunch and an assortment of a dozen
freeze-dried entrees. (He ate all the meat options first.) He also
snacked on dried fruit and his wife's plum jam, but he ran out of that
halfway across the ocean. Every time he closed his eyes, Doba told me,
``I dreamed I was paddling in the winter in Poland.'' He lost 45 pounds.
Still, the trip was perfect. Ninety-nine days after leaving Senegal,
Doba arrived in Brazil. He was greeted by one journalist and the Polish
ambassador. Nobody cares if you cross the Atlantic in a kayak. The fact
that Doba knows this is clear in his eyes. In photos from the ends of
his trips, he looks ecstatic and feral, in the best possible sense,
intrinsically wild and free.

\hypertarget{-1}{%
\subsection{}\label{-1}}

\textbf{The day I} arrived in Warsaw, a very chic woman named Martyna
Wojciechowska, the host of a Polish documentary TV show called ``Woman
at the End of the World,'' showed up at my hotel to explain Doba to me.
I was not in a great state. To be precise: I was a Jew with the flu
about to go kayaking in Poland in January --- not a setup likely to end
well. But still, I was so happy I'd gotten away. I'd been feeling
buried, by stuff exactly as predictable as you'd imagine for a working
mother of two kids. (Honestly, you don't need the details.)
Wojciechowska drank a double cappuccino and told me that she'd been
engaged five times but never married --- she felt it would be impossible
to follow her dreams with a husband. She also left her daughter at age 8
months to go climb a mountain in Antarctica because she was trying to
complete the Seven Summits, the highest mountains on each continent, and
achieving that goal, at that time, felt like a matter of life or death.
She felt bad about it, she said, and she felt judged, but here we were,
weren't we? Then she sat me down at a nearby restaurant, ordered
pierogies and borscht and told me a joke.

``The devil, a German man, a Frenchman and a Polish man are all in a
hot-air balloon,'' Wojciechowska said. ``They are falling, falling --- a
catastrophe is about to happen. So the devil says to the German man:
`You must jump. This is an order.' And the German man jumps.

``Then the devil says to the Frenchman, `You must jump.' The Frenchman
says, `What does this mean?'

``The devil says, `It means that life is meaningless but when you jump
you will look very chic, very modern.' So the Frenchman jumps.

\includegraphics{https://static01.graylady3jvrrxbe.onion/images/2018/03/25/magazine/25mag-kayaker3/25mag-25kayaker-t_CA1-master180.jpg}

``Then the devil gets to the Polish man.'' He tries the reasons he used
on the German and French men, without success. `` `Shoot,' the devil
says. `I know you will not jump.' And the Polish man jumps.''

Wojciechowska looked at me squarely, to make sure I understood. ``The
more you don't believe in Polish people, the more determined we are. To
prove themselves, Polish people will endure everything. If you aren't
willing to suffer, you can do nothing. You can sit and die. This is the
only one thing you can do.''

Doba has a deep, almost performance-art-like sense of this. You can be
made small by life or rage against it. ``\emph{Nie chce byc malym szarym
czlowiekiem,}'' he told me. ``I do not want to be a little gray man.''
This is a common expression in Poland --- and a good motto for us all.

Doba was born in 1946 in Swarzedz, Poland, right after the end of World
War II, when the country had been run over by the Soviets and the
Germans, then bombed to dust. The whole nation was starving. Previous
generations had not fared better. Doba's maternal grandfather, a
high-ranking officer in the czarist army, was poisoned in the Bolshevik
Revolution in 1917. His maternal grandmother was sent to Siberia. Their
three sons, Doba's uncles, disappeared.

\includegraphics{https://static01.graylady3jvrrxbe.onion/images/2018/03/25/magazine/25mag-kayaker4/25mag-25kayaker-t_CA0-master180.jpg}

And yet Doba describes his youth as magical, almost fantastical. Near
his house, he said, there was a pond, behind the pond was a forest, in
the forest there were mushrooms and behind all that was a small airport
with gliders and storks, which migrated there for the summer. Doba's
father, a master at building and repairing things from scraps, made his
son a bike. At 15, Doba rode around Poland by himself.

After he graduated from Poznan University of Technology, where he
studied mechanical engineering, Doba met Gabriela Stucka, his future
wife, on a backpacking trip. He did explain to her how to boil water,
which to this day she finds ridiculous. But she was charmed by his
impish streak --- his stories of smuggling ski boots back from East
Germany by walking across the border in them; the beard he grew over a
summer holiday, so he could return to university and his mandatory
military training as the only soldier in Poland with facial hair.

Doba started kayaking in 1980, after he and Gabriela moved to Police,
where Doba took his equipment-repair job at the chemical factory and
Gabriela was a social worker. One day, a colleague of Doba's said the
factory's kayaking club was going on a two-week expedition. Doba had
been into hang gliding and sky diving, but the government often closed
the airspace because Police was near the East German border. Doba signed
up for the factory kayaking trip and went again the following year.
After, on weekends Doba would put a kayak on a train, debark as close to
a river as possible and walk the kayak, on a wheeled contraption built
from bicycle and stroller parts, 25 miles if necessary to put in.
Sometimes he took his young sons, Bartek and Czesiek, born in 1979 and
1982. Before they left, Gabriela would make Doba state for the record
the condition of the children (``Bartek has a small sniffle and is tired
but otherwise is well. ...'') so that she could compare it with the
condition of her children upon their return. After a while, the boys
began to mutiny. ``He took us on long trips, only him and us both, in
the middle of nowhere, and we hated that,'' Bartek recalled. ``It was
raining. There were spiders. There were small rivers, and we were
squeezing through the bushes.''

Freed from the ballast of his sons, Doba started setting goals for
himself: surpass the record of the greatest number of days paddled by a
Polish man in a single year (108); kayak the longest possible route in
Poland (1,189 kilometers). Before the Iron Curtain fell, the Communist
Party declared it illegal to kayak on the Baltic Sea, lest anyone
defect. Doba did anyway. ``I was just paddling down the river,'' he told
the border patrol when he inevitably was caught. ``I don't know how I
ended up here.'' The soldiers told Doba he had broken so many laws that
they didn't know how to charge him. Doba responded, ``In that case, I'll
be on my way.''

\includegraphics{https://static01.graylady3jvrrxbe.onion/images/2018/03/25/magazine/25mag-kayaker5-cov/25mag-kayaker5-cov-master180-v2.jpg}

After the transition, as the Poles call their country's change to a
democratic government in the early 1990s, Doba's trips became more
extreme. He spent 100 days paddling the circumference of the Baltic Sea.
He kayaked the coast of Norway to the Arctic Circle. Along the way, he
was thrown from his kayak in a storm. The rope tethering Doba to the
kayak came undone. He passed out. He woke up on shore to the sound of
screaming --- his own. Doba does not regret this experience. He has no
interest in dying in his bed.

\textbf{One afternoon Doba} parked his Mazda 5 minivan in the lot beside
his apartment building, pointed to a nearby forest of Scots pines
standing rigid and competing for light and said, amused, ``Sometimes we
have the great adventure of going out there to drink a beer.''

Doba's apartment building is a five-story cube of blank concrete, with
blank concrete stairwells. It looks straight out of a dystopian
young-adult novel in which the state is intent on creating little gray
men. But the inside of the apartment feels much better: shoes in the
entrance hall; pencil marks of grandchildren's heights in the kitchen
door jamb; atlases, kayaking trophies and snow globes of family photos
in the living room. Doba's stinking foam sleeping pad and other gear
from Olo remain, at Gabriela's request, all these months later, on the
balcony. Even here at home, Doba's physicality is sui generis. His body
appears to be assembled from parts belonging to people of vastly
different ages. His skin looks 71. His chest looks 50. His hands and
forearms look 30, straight off a Montana roper. His hair and beard
appear to be taken from a Michelangelo painting of God.

Gabriela joined us to talk at the dining table, where she had set bowls
of peanuts, raisins, chocolates, cookies and cake and French presses of
coffee and tea. At age 64, after 42 years of marriage, she still adores
Doba, and her acceptance of him is absolute --- easier than it used to
be, in fact. ``When people are young, sometimes they are very jealous,''
she said, and want to keep a piece of their partner for themselves.
Gabriela's accommodations to Doba's lifestyle started in the late 1980s,
when Doba became obsessed with kayaking and began insisting that
Gabriela drop him and his kayak by the side of the road on the way to
her mother's house for Christmas, and then pick up him and his kayak at
a prearranged point on her way back home after a few days. The plan was
that if Doba was late, Gabriela would write with a stick in the dirt: I
WAS HERE. Then she'd leave.

\hypertarget{-2}{%
\subsection{}\label{-2}}

After Gabriela told me about Christmas, Doba began telling a story. It
started with a small diatribe on the tyranny of politeness that dictates
a houseguest must always say he likes everything his host serves him and
the problem that results: You're fed food you don't like again and
again. ``The first time I visited my future mother-in-law, she served
duck's blood soup. I don't like it at all. I was in a pickle!'' To avoid
the nightmare of quiet conformity, Doba told me he said to his
mother-in-law, ``This is a good soup! It looks good, it smells good,
probably everybody likes it. But this specific soup I don't like.'' That
was a shock to his mother-in-law, he said, but a short one. ``The next
time I visited, my mother-in-law did not prepare that soup. She knew I
didn't like it.''

Nonetheless, Gabriela was not prepared for Doba's first trans-Atlantic
expedition. He had been taking long trips for decades, but he always
slept on land. So, Gabriela told me, she laid out for her husband all
the reasons trans-Atlantic kayaking was stupid. She threatened divorce.
``If you have a crisis in the middle of the Atlantic and the closest
land is the bottom,'' she asked, ``what will you do then?'' Doba said,
``There will be no such crisis.'' She knew she couldn't stop him.

Still, Gabriela did not want to be Penelope to Doba's Odysseus. She did
not want to spend her life stressed, waiting for his return. She threw
herself into her work, which had become far more rewarding than she ever
imagined it could when she trained as a social worker in the 1970s. Once
Poland became a democracy, social structures and attitudes changed.
Gabriela traveled to Denmark to study how European Union countries
handled problems like unemployment, alcoholism and lonely young mothers.
Her department grew to 100 people from five. ``I never expected it would
be so big,'' she said. I joked with Gabriela that this was her Atlantic.
``Bigger than the Atlantic!'' Doba chimed in. ``This is Gabi's
Pacific!''

Later that afternoon, Bartek, the couple's older son, stopped by the
apartment with his two elementary-school-age daughters. He told me that
he took his father to the airport for his most recent crossing. Over the
years, he, like Gabriela, had made internal accommodations to his
father's adventures. But this trip felt different. ``When he gave me a
last hug, maybe he didn't cry, but I saw his eyes,'' Bartek said. ``I'd
never seen him like that. So I thought: Well, he might not come back.''

\includegraphics{https://static01.graylady3jvrrxbe.onion/images/2018/03/25/magazine/25mag-kayaker6/25mag-kayaker6-master180.jpg}

\textbf{Doba started talking} about his second trans-Atlantic trip a few
days after he returned from the first. He had always planned to go on
three crossings: South, Mid- and North Atlantic. Gabriela did not
believe he would. The departure for the second trip, from Portugal to
Florida, turned out to be very abrupt. In mid-September 2013, Doba still
did not possess all the gear he wanted to take on his trip. ``No retired
man in Poland can afford to do these things,'' Doba explained. His
pension is \$700 a month. But the community raised money for him, and on
Oct. 3, 2013, off he went. Neither Gabriela nor Bartek took him to the
airport.

Olo is 300 kilograms empty. At the start of that trip, she weighed 700
kilograms, and she carried (among other things) two carbon-fiber
paddles, five spray skirts to cover Doba's body while he paddled, two
pairs of kayaking gloves, three pairs of sunglasses, two pairs of
reading glasses, 10 boxes of waterproof matches, two big butcher's
knives, one electric desalinator, two manual desalinators, three
floating smoke signals, nine red flares, a cooking stove and propane
canisters, two flashlights, two headlamps, an iPad, a GoPro camera, 320
lithium batteries (for Olo's satellite signaling devices), mineral
supplements, a sewing kit, 175 chocolate bars, three toothbrushes, seven
bottles of sunscreen, three liters of homemade wine, a pumice stone for
the calluses on his hands, two pairs of sandals and an emergency fishing
kit. Doba also brought two sailing harnesses, which resemble climbing
harnesses but are worn on the chest instead of the legs; he uses these
to attach himself with a carabiner to the boat. Doba's sleeping cabin,
which he reaches by sliding himself through a portal the size of a
laptop, affords him, when he's lying down, 15 inches of headroom. He
relieved himself by leaning off the side or the stern.

For a while, once again, on this second expedition, all was smooth. Doba
paddled naked. He drank five cups a day of instant coffee. He ate
freeze-dried chicken tikka masala and the occasional raw tiny flying
fish that landed on his deck. He texted via satellite phone with his
family and Arminski, the boat builder, who once again served as trip
navigator, sending wind and weather reports. Then, on Dec. 19, Doba's
phone stopped working. He waited three days and still had no signal, so
in hopes of getting Gabriela or Arminski to look into his phone
problems, he pressed the help button on his SPOT device, a piece of gear
that has become de rigueur among adventurers, as it can send emergency
messages and GPS coordinates when a person is far off the grid.

Shortly after, a huge Greek ship steamed alongside Doba, seeking to
rescue him. Doba had only wanted his phone repaired. He did not want to
be saved, so he waved the Greeks off. ``Me, fine,'' he shouted in
English, pointing to himself and giving the thumbs up. ``Phone, bad.''
He gave the thumbs down. Doba declined the ropes the crew offered to
throw him. Thus, most likely assuming that Doba was deranged --- he was,
after all, a tiny speck bobbing in a vast universe of water --- the ship
turned around and tried to rescue him again. Still, Doba refused. During
the third rescue attempt he yelled a Polish vulgarity. The Greeks got
the message and left.

\includegraphics{https://static01.graylady3jvrrxbe.onion/images/2018/03/25/magazine/25mag-kayaker7/25mag-kayaker7-master180.jpg}

Forty-seven days after it stopped working, the phone came back on.
Someone had forgotten to fund the prepaid credit card that covered the
bill.

Doba did finally accept help two weeks later. He was, by then, 90
percent of the way across the ocean but had spent the last six weeks
paddling in circles in the Bermuda Triangle, trapped by wind and
currents. A storm had damaged his rudder. He tried to fix it and failed.
He texted Arminski and Piotr Chmielinski, a Polish kayaker who in the
mid-1980s, shortly after he'd defected to the United States, was part of
the first team to paddle the length of the Amazon. Chmielinski also had
been coordinating Doba's logistics. Doba accepted his suggestion that he
paddle to Bermuda for a repair.

On the island, after 143 days at sea, Doba could hardly walk, and by the
time he was ready to return to the Atlantic, it was late March, and the
weather was turning bad. Chmielinski had a hard time finding a ship
captain on the island who was willing, in dangerous weather, to put Doba
back out in the ocean near where he'd gone off course. Eventually they
found him a ride, and Doba paddled away during a storm. Three weeks
later, he reached Florida. He put on a clean red-and-white shirt, the
colors of the Polish flag, and lay down on the grass.

Doba returned to Police to a hero's welcome and 14 months later flew to
Washington to receive an award from the National Geographic Society as
the 2015 People's Choice Adventurer of the Year. Event producers asked
him to just walk onstage and say, in English, ``Thank you very much.''
Doba, who wore jeans to the June ceremony, walked onstage and said,
``\emph{Polacy nie gesi i swoj jezyk maja}.'' Polish people are not
geese and have their own language.

\hypertarget{-3}{%
\subsection{}\label{-3}}

\textbf{Suffering tends to} do more for the flagellant than it does for
the flagellant's loved ones. Everyone who cared about Doba opposed his
third expedition --- the cold and stormy North Atlantic, from New Jersey
to France. Doba began planning it, of course, a few days after he
returned from Florida. Arminski, the boatbuilder, spent many hours
trying to argue Doba out of it. The risk of kayaking the North Atlantic,
Arminski believed, was irresponsibly, even amorally high. ``It is
impossible to design a kayak that won't capsize on the North Atlantic,''
Arminski told me in his British-accented English, sitting behind his
desk in Szczecin in a tidy V-neck sweater and a pressed shirt. ``This is
a matter of the size and energy of the wave compared to the mass of the
kayak.'' Beyond the boat rolling over, there is an even more
catastrophic problem: A breaking wave transfers all the potential energy
in the wave's height into turbulent, violent kinetic energy. It is
basically an avalanche of water. ``A breaking wave can do whatever it
wants to a kayak.'' There was no engineering Olo for this. ``When Doba
finally said, `I will go on the North Atlantic,' I said, `I will not
participate in this,' '' Arminski told me. `` `Absolutely no. It is
simply too dangerous. Every three weeks there is a storm with waves that
can capsize the kayak. How many capsizes can you survive?' ''

Gabriela, she says, was ``to be honest, a little pissed'' that Doba was
going again. Nonetheless, on May 29, 2016, Doba kayaked out from the New
Jersey shore near the Statue of Liberty. The weather report was bad, but
Chmielinski had arranged for a lot of press, and an entourage of
kayakers had come to paddle out with Doba a ways, and he felt he had a
duty to them. In video footage of the hour before he left, Doba looks
wrecked --- exhausted and on the verge of tears. He had slept only three
hours the night before and had not even had time to double-check all his
gear. Once he was out past the Statue of Liberty, Doba turned on his
GPS. It didn't work. Four days later, after capsizing repeatedly, Doba
finally cleared the Sandy Hook peninsula and left lower New York Bay. He
was pummeled again by wind and waves. (Leaving shore is one of the
trickiest parts of a trans-Atlantic journey.) Water trapped in Olo's
storage lockers short-circuited the electric desalinator. Doba was blown
to the beach and called the trip off.

A year later, he tried again.

Three days in, Doba received his first storm warning. He was still close
to the New Jersey shore, so he landed at Barnegat Bay, ate a steak,
slept at a hotel and continued on his way. For a while the crossing was
drama-free. Doba's electric desalinator broke, as it did before, and he
had to spend several hours a day manually pumping seawater through an
extremely fine filter to produce the five to nine liters of fresh water
he needed to stay hydrated and prepare his freeze-dried food. But so
what? Doba wasn't on a schedule. He paddled and he paddled and he
paddled and he paddled, a real \emph{katorga}. Then, about three weeks
into the trip, more storms arrived: Winds up to 55 knots created outsize
waves, mountains of water racing toward the kayak, the entire surface of
the sea lurching as if a stampede of elephants were running across a
giant water bed. Rowers who've paddled the Atlantic --- an easier
proposition because rowing allows a person to recruit leg and back
muscles --- have strapped themselves to the floors of their cabins to
avoid being bounced around during storms and concussed, or worse.
Imagine a plane the size of a kayak hitting turbulence, but in water
rather than air; at sea level, water is hundreds of times denser.

The key to surviving a serious storm, Doba says, was to keep his kayak
oriented with the stern perpendicular to the waves, so that those waves
hit the narrow end of the vessel, instead of broadsiding the boat and
rolling it over. The tool he used to accomplish this was a sea anchor,
basically a parachute, secured to the stern of the kayak with a rope,
that he set underwater to increase drag. During a storm, the water below
the ocean's surface stays relatively still; what moves radically are the
waves. The most serious trouble came for Doba when he hit a storm that
lasted two days. During the worst of it, the rope on his sea anchor
broke. If a breaking wave hit Olo with no sea anchor, Doba said, ``I
knew I would roll over many times. The kayak would break into many
pieces.'' So, wearing only his harness, Doba exited the cabin, crawled
across the bucking deck with a spare sea anchor attached to a rope, tied
the rope to his kayak and threw the anchor off the stern. Doba returned
to his cabin somewhat shocked to be alive. ``I did it with no
stuntman,'' Doba told me proudly. ``But I don't have a film. But even if
I did, censorship would not let it go.''

\includegraphics{https://static01.graylady3jvrrxbe.onion/images/2018/03/25/magazine/25mag-kayaker8/25mag-kayaker8-master180.jpg}

After a storm on June 16, 2017, Doba emerged from his cabin to find that
the large pin that attached Olo's rudder to the stern was seriously
bent. Doba tried to heat the pin with his camping stove and smash it
straight. (That didn't work.) He tried to drill holes in the rudder and
attach the rudder to the stern with plastic ties. (This did work, in
that Doba managed to drill the holes, but he couldn't steer.) Then Doba
cut the bent portion off the pin with a hacksaw blade he'd brought along
and jury-rigged a way to control the rudder with carabiners. (This
worked a little, but not entirely.)

Doba alerted Chmielinski, who alerted Gabriela --- who didn't really
want to know anything about how the trip was going. But this voyage was
too dangerous, she told me, to keep her head where she liked it to be,
stuck in the sand. Doba drifted around for days eating freeze-dried
goulash and chocolate bars. Chmielinski and others came up with what
Doba describes as ``exotic and expensive methods'' of rescue. These
involved such things as catamarans sailing in from the Bahamas and
helicopters dropping rudder-repair gift packages out of the sky. Doba
did not consent. ``I was the one who would have to pay!'' he says. He
didn't particularly want to be rescued, anyway. The idea was to do the
crossing unsupported. He wanted to tough out his original plan.

Within a week, however, Doba let go and agreed to have a freighter that
was passing nearby pluck him out of the water and repair his rudder with
its welding tools. Aboard Baltic Light, he ate a hot meal and took a
bunch of selfies with the Filipino crew. The ship's captain didn't think
it was a great idea to set this rash-covered, sleep-deprived 70-year-old
man back in the ocean alone. But Doba insisted.

``I did not think I could find myself in something so downing,'' he
texted to Chmielinski, somewhat incomprehensibly, two weeks later. ``My
attempt to get out of this terrible conflict was a `signal strike.' ''
Doba then turned off all his communications, including his SPOT device,
which uploaded his coordinates every 10 minutes. Without the SPOT,
nobody would be able to find the kayak if they needed to rescue him.
Chmielinski interpreted this communication blackout as a sign that Doba
was deeply despairing, perhaps because he was alone again after being
with the crew on Baltic Light, perhaps because, now that he had accepted
help, he worried that he would not earn a spot in the Guinness Book of
World Records for kayaking solo from continent to continent unsupported.
Doba was ready, Chmielinski suspected, to just drift off into the
eternal vastness and go down with his ship. But then Doba turned his
SPOT and his phone back on. Before he arrived in France, he made a few
GoPro videos. We watched them in his living room together with Gabriela.
``In three weeks, I'll be 71,'' he says in one. ``If I survive.''

\includegraphics{https://static01.graylady3jvrrxbe.onion/images/2018/03/25/magazine/25mag-kayaker9/25mag-kayaker9-master180.jpg}

\textbf{One unseasonably balmy} January morning, Doba, dressed in baggy
Gore-Tex pants and rain boots, handed me a blue plastic paddle, waved me
into the front of a red double kayak, took out his hearing aids and
declared, happily, shoving us off, ``Now I am in the silent zone.'' We
had gone to a river-veined lowland of central Poland for the 53rd annual
International Winter Kayaking Congregation, an event that sounds grand
but was really just 60 Polish people, at or past middle age, spending a
weekend in January together kayaking, drinking and hanging out at a
lodge. The air that day hovered around 40 degrees --- about 4 degrees
Celsius. Doba was a little disappointed. One year during the kayaking
convention, Doba giddily reported, ``It was minus 18 degrees, and the
number of people who showed up was the same!''

Yet, at least for me, the river was magic --- calm, mysterious and
alive. Bright green moss covered the trees, egrets stretched like sheets
blowing on a line and the occasional ray of sun broke through the low
clouds. In truth, my own trip had not been as peaceful as I'd hoped. My
flu lingered. My phone exploded with hysterical texts from home. But I
was my own little speck in the vast universe. It felt good.

\textbf{This story is one of four extraordinary journeys published in
the magazine's Voyages issue.\\
Read the rest here:}
\href{https://www.nytimes3xbfgragh.onion/interactive/2018/03/22/magazine/voyages-kayaking-across-ocean-at-70.html?action=click\&module=MagazineModule\&pgtype=Article\&region=Body\&contentCollection=Magazine\%0A}{The
70-Year-Old Master Kayaker}
\href{https://www.nytimes3xbfgragh.onion/interactive/2018/03/22/magazine/voyages-nathalie-cabrol-searching-mars-life-on-earth.html?action=click\&module=MagazineModule\&pgtype=Article\&region=Body\&contentCollection=Magazine\%0A}{Looking
for Mars on Earth}
\href{https://www.nytimes3xbfgragh.onion/interactive/2018/03/22/magazine/voyages-worlds-greatest-hitchhiker.html?action=click\&module=MagazineModule\&pgtype=Article\&region=Body\&contentCollection=Magazine\%0A}{The
World's Greatest Hitchhiker}
\href{https://www.nytimes3xbfgragh.onion/interactive/2018/03/22/magazine/voyages-joshua-tree-lost-hiker.html?action=click\&module=MagazineModule\&pgtype=Article\&region=Body\&contentCollection=Magazine\%0A}{The
Search for Joshua Tree's Missing Hiker}

Behind me, steering us among the 60 other kayakers and dozens of fallen
trees, Doba fidgeted in his seat, like an ADHD kid in class. Around
noon, we pulled onto a riverbank along with the rest of the group to
toast sausages and bread over a fire and eat little cookies shaped like
croissants. Kind, gracious, enthusiastic Poles kept bringing me shots of
alcohol, which I tried to wave off. Then a woman who noticed me coughing
approached with a little glass of thick red liquid that looked like
cough syrup. ``Medicine!'' she said.

After I downed it, she laughed. ``Raspberry vodka! It's good?''

The next day, hearing aids restored, Doba drove us to Police, and a few
days later, he addressed a question that had been nagging at me: Why did
he go on that third trip? How did he justify it to himself? ``To go into
the sea and die --- that would be a problem for my family and a problem
for Andrzej Arminski, too,'' Doba acknowledged. ``I came very close to
the line of my possibility and human possibility. But. ...'' What he did
not say, what was left hanging in the air, was that it would not be a
big problem for him to go into the sea and die. He has made his own
bargain with the human condition. Doba's response to Gabriela's concern
about what he'd do in a crisis, if the closest land was the bottom, was
to say there will be no crisis. He did not say this because he's naïve.
He said it because he has reimagined the concept of crisis, just as he
has reimagined the concept of suffering. A crisis, in Doba's worldview,
is an opportunity for triumph. So Doba moves toward the crisis, just as
he moves toward the suffering. By choosing it, he casts himself in the
role of hero, not victim. He gives himself control.

He does love Gabriela, however, and he does not want to cause her more
pain. Will he cross the Atlantic again? Doba, for now, has a carefully
constructed answer. ``An expedition,'' he told me. ``On the kayak.
Through the ocean. So far I don't plan.'' Then he added: ``But I do like
to sail.''

\emph{Sign up for
\href{http://www.nytimes3xbfgragh.onion/newsletters/magazine}{our
newsletter} to get the best of The New York Times Magazine delivered to
your inbox every week.}

\textbf{Elizabeth Weil} is a contributing writer for the magazine. She
is also co-author of ``The Girl Who Smiled Beads: A Story of War and
What Comes After,'' which will be published next month. Her last article
for the magazine was about
\href{https://www.nytimes3xbfgragh.onion/2015/10/18/magazine/shrimp-boys-day-in-court.html}{a
Chinese mafia don}.\\
\textbf{Joakim Eskildsen} is a Danish photographer based outside Berlin.
He is currently working on two books, ``Home Works'' and ``Cuba.'' He
last shot
\href{https://www.nytimes3xbfgragh.onion/interactive/2017/09/21/magazine/voyages-issue-photographs-family-vacations-around-world.html\#estonia}{a
smoke sauna in Estonia} for the magazine.

\textbf{Correction: March 23, 2018} An earlier version of this article
misstated the timing of Piotr Chmielinski's navigation of the Amazon. He
paddled the length of the river in the mid-1980s --- not 1980 --- and it
was after he moved to the United States, not before.

\hypertarget{more-voyages}{%
\subsection{More Voyages}\label{more-voyages}}

\begin{itemize}
\tightlist
\item
  \href{https://www.nytimes3xbfgragh.onion/interactive/2018/03/22/magazine/voyages-worlds-greatest-hitchhiker.html?action=click\&module=MagazineModule\&pgtype=Article\&contentCollection=Magazine\&region=Footer}{}
\item
  \href{https://www.nytimes3xbfgragh.onion/interactive/2018/03/22/magazine/voyages-nathalie-cabrol-searching-mars-life-on-earth.html?action=click\&module=MagazineModule\&pgtype=Article\&contentCollection=Magazine\&region=Footer}{}
\item
  \href{https://www.nytimes3xbfgragh.onion/interactive/2018/03/22/magazine/voyages-joshua-tree-lost-hiker.html?action=click\&module=MagazineModule\&pgtype=Article\&contentCollection=Magazine\&region=Footer}{}
\end{itemize}

2018

\hypertarget{more-on-nytimescom}{%
\subsection{More on NYTimes.com}\label{more-on-nytimescom}}

Advertisement

\hypertarget{site-information-navigation}{%
\subsection{Site Information
Navigation}\label{site-information-navigation}}

\begin{itemize}
\tightlist
\item
  \href{https://help.nytimes3xbfgragh.onion/hc/en-us/articles/115014792127-Copyright-notice}{©
  2020 The New York Times Company}
\item
  \href{https://www.nytimes3xbfgragh.onion}{Home}
\item
  \href{https://www.nytimes3xbfgragh.onion/search/}{Search}
\item
  Accessibility concerns? Email us at
  \href{mailto:accessibility@NYTimes.com}{\nolinkurl{accessibility@NYTimes.com}}.
  We would love to hear from you.
\item
  \href{https://help.nytimes3xbfgragh.onion/hc/en-us/articles/115015385887-Contact-Us}{Contact
  Us}
\item
  \href{https://www.nytco.com/careers/}{Work with us}
\item
  \href{https://nytmediakit.com/}{Advertise}
\item
  \href{https://help.nytimes3xbfgragh.onion/hc/en-us/articles/115014892108-Privacy-policy\#pp}{Your
  Ad Choices}
\item
  \href{https://help.nytimes3xbfgragh.onion/hc/en-us/articles/115014892108-Privacy-policy}{Privacy}
\item
  \href{https://help.nytimes3xbfgragh.onion/hc/en-us/articles/115014893428-Terms-of-service}{Terms
  of Service}
\item
  \href{https://help.nytimes3xbfgragh.onion/hc/en-us/articles/115014893968-Terms-of-sale}{Terms
  of Sale}
\end{itemize}

\hypertarget{site-information-navigation-1}{%
\subsection{Site Information
Navigation}\label{site-information-navigation-1}}

\begin{itemize}
\tightlist
\item
  \href{https://spiderbites.nytimes3xbfgragh.onion}{Site Map}
\item
  \href{https://help.nytimes3xbfgragh.onion/hc/en-us}{Help}
\item
  \href{https://help.nytimes3xbfgragh.onion/hc/en-us/articles/115015385887-Contact-Us?redir=myacc}{Site
  Feedback}
\item
  \href{https://www.nytimes3xbfgragh.onion/subscription?campaignId=37WXW}{Subscriptions}
\end{itemize}
