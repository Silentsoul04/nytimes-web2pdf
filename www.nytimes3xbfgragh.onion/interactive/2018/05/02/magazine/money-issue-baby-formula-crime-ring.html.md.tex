 **NYTimes.com no longer supports Internet Explorer 9 or earlier. Please
upgrade your browser.
\href{http://www.nytimes3xbfgragh.onion/content/help/site/ie9-support.html}{LEARN
MORE »}

**Sections

**Home

**Search

\hypertarget{the-new-york-times}{%
\subsection{\texorpdfstring{\href{http://www.nytimes3xbfgragh.onion/}{The
New York Times}}{The New York Times}}\label{the-new-york-times}}

\hypertarget{-magazine-}{%
\subsubsection{\texorpdfstring{
\href{https://www.nytimes3xbfgragh.onion/section/magazine}{Magazine}
}{ Magazine }}\label{-magazine-}}

 \href{https://www.nytimes3xbfgragh.onion/section/magazine}{Magazine}
\textbar{}The Baby-Formula Crime Ring

**Close search

\hypertarget{site-search-navigation}{%
\subsection{Site Search Navigation}\label{site-search-navigation}}

Search NYTimes.com

**Clear this text input

Go

\url{https://nyti.ms/2HLsMmY}

\hypertarget{site-navigation}{%
\subsection{Site Navigation}\label{site-navigation}}

\hypertarget{site-mobile-navigation}{%
\subsection{Site Mobile Navigation}\label{site-mobile-navigation}}

\hypertarget{the-baby-formula-crime-ring}{%
\section{The Baby-Formula Crime
Ring}\label{the-baby-formula-crime-ring}}

It's pricey, it's portable, its users need it constantly, and retailers
love to buy it at a discount. All of which makes it a perfect product to
steal.

\hypertarget{the-money-issue}{%
\subsection{The Money Issue}\label{the-money-issue}}

\begin{itemize}
\tightlist
\item
  \href{https://www.nytimes3xbfgragh.onion/interactive/2018/05/02/magazine/money-issue-baby-formula-crime-ring.html?action=click\&module=MagazineModule\&pgtype=Article\&contentCollection=Magazine\&region=Header}{}
\item
  \href{https://www.nytimes3xbfgragh.onion/interactive/2018/05/02/magazine/money-issue-insys-opioids-kickbacks.html?action=click\&module=MagazineModule\&pgtype=Article\&contentCollection=Magazine\&region=Header}{}
\item
  \href{https://www.nytimes3xbfgragh.onion/interactive/2018/05/03/magazine/money-issue-bangladesh-billion-dollar-bank-heist.html?action=click\&module=MagazineModule\&pgtype=Article\&contentCollection=Magazine\&region=Header}{}
\item
  \href{https://www.nytimes3xbfgragh.onion/interactive/2018/05/03/magazine/money-issue-iowa-lottery-fraud-mystery.html?action=click\&module=MagazineModule\&pgtype=Article\&contentCollection=Magazine\&region=Header}{}
\item
  \href{https://www.nytimes3xbfgragh.onion/interactive/2018/05/03/magazine/money-issue-white-collar-crimes-cheat-sheet.html?action=click\&module=MagazineModule\&pgtype=Article\&contentCollection=Magazine\&region=Header}{}
\end{itemize}

it.

\includegraphics{https://static01.graylady3jvrrxbe.onion/images/2018/05/05/magazine/05mag-baby-image1/05mag-baby-image1-master180-v3.png}

\hypertarget{the-baby-formula-crime-ring-1}{%
\section{The Baby-Formula Crime
Ring}\label{the-baby-formula-crime-ring-1}}

By CHRIS POMORSKI\\
Illustrations by FRANCESCO FRANCAVILLA MAY 2, 2018

It's pricey, it's portable, its users need it constantly, and retailers
love to buy it at a discount. All of which makes it a perfect product to
steal.

New Port Richey, perched on a knuckle of Gulf Coast 35 miles northwest
of Tampa, is a typically Floridian enclave of strip malls, subdivisions
and brackish waterways. During the 1920s, it enjoyed a brief period of
glamour when professional golfers and silent-film actors bought land,
built handsome homes and socialized with visiting stars from Broadway
and vaudeville at the Hacienda Hotel. But the town owed much of its
success to the first of Florida's many real estate bubbles, and the
fantasy ended around 1925, dashing forever New Port Richey boosters'
hopes of its becoming a kind of Hollywood East.

By November 2005, when Alexis and Ronald Dattadeen bought a home there
--- a cozy ranch house shaded by a generous oak --- it was just another
Tampa suburb. Alexis and Ronald, who were in their mid-20s, had recently
welcomed their first child, A., and within a few years they had another
son, D. (The children in this article are identified by their first
initial only.) Like his brother, D. had dark hair and big brown eyes.
But D. also had a rare genetic disorder that would require many visits
over the years to gastroenterologists, neurologists and hospitals.

Ronald worked long hours for low pay, maintaining pools during the day
and scrubbing operating rooms at night. Alexis, who had held jobs
sterilizing medical equipment and wrangling phone lines at a hospital,
found D.'s unpredictable need for medical attention incompatible with
even part-time employment. You could change schedules only so many times
before exasperating even the most understanding manager. And she soon
found upsides to staying home --- napping with D. during the day; the
satisfactions of a clean car, folded laundry, the waft of pot roast from
the oven. But homemaking could be boring and lonely --- and Dattadeen
wasn't earning any money, further limiting her life outside the house.

In 2011, when D. was 2, Dattadeen had a realization: She could be
selling her excess baby formula. D. needed an expensive brand called
EleCare, which was designed for easy digestion. Dattadeen got it through
Medicaid, and it arrived at her home automatically from a medical
supplier. She always ended up with extra cans, so she posted an ad on
Craigslist and quickly found a buyer, a man who indicated that the
formula would go to a child in need. But before long, a second buyer ---
a buyer who would prove far more lucrative --- responded to the ad. Her
name was Alicia Tondreau-Leve.

Dattadeen and Tondreau-Leve first met in a McDonald's parking lot about
an hour's drive from New Port Richey. Soon Tondreau-Leve became a repeat
client, meeting Dattadeen every few weeks to buy some of the EleCare.
During one of their first meetings, Tondreau-Leve explained that she had
a business distributing excess powdered formula to needy families.
Dattadeen expressed interest in the venture, and Tondreau-Leve soon
agreed to bring her aboard --- emphasizing, though, that she wasn't an
employer. Dattadeen would be an independent contractor, sourcing formula
for resale to Tondreau-Leve. As in a multilevel marketing operation, her
success would depend on her ability to create a large network. Still,
for Dattadeen, the benefits would be manifold: She could set her own
hours, earning an income while tending to D.'s appointments.

Like Mary Kay or Amway, Tondreau-Leve provided a start-up guide. She
showed Dattadeen how to replicate a Craigslist ad she had been posting
to promote the business. It had an elegant logo --- a silhouette of a
woman with flowing hair holding an infant aloft. ``Formula Mom,'' it
read. ``Helping Other Moms ... Helps You!'' The ad outlined Formula
Mom's services: free pickups and cash payments for brands including
Gerber, Enfamil and Similac. Dattadeen would swap in her own phone
number and a Formula Mom email address that Tondreau-Leve suggested she
create. There was an invoice template and a price guide with photos of
the formulas Tondreau-Leve accepted. She instructed Dattadeen not to
stray far from the Tampa area and advised her to meet sellers in public
places. Later, she even provided business cards.

To help her get started, Tondreau-Leve gave Dattadeen a few leads.
Craigslist ads --- answering and posting them --- yielded additional
sellers, and Dattadeen's first weeks were successful. Attentive and
patient, Tondreau-Leve guided her by phone and text. ``That's great that
you are so willing to help,'' Dattadeen texted her in early 2012. ``We
will make a great team.''

``Yes we will,'' Tondreau-Leve replied, ``You have the same drive that I
have.''

Dattadeen was included in business decisions, providing input on flyer
design and marketing. One day, she had an idea of her own. To make
things more official and to better track sales, she thought, they should
provide sellers with receipts. Taking initiative, she went to an Office
Depot and designed a prototype. But before returning to print copies,
she mentioned the idea to Tondreau-Leve, who saw things differently.
Receipts wouldn't work for Formula Mom, Tondreau-Leve said, because that
would leave a paper trail.

\textbf{Some \$4.3 billion} worth of infant formula was sold in the
United States last year, a vast majority of it in powdered form. Between
factory and baby aisle, its cheap ingredients (dehydrated milk and
vitamins) become steeply, even mysteriously expensive. Basic types run
about \$15 for a 12.5-ounce can, amounting to perhaps \$150 a month for
a fully formula-fed infant. Specialty recipes like EleCare can cost two
or three times as much. Strict Food and Drug Administration regulations
govern formula production, and three companies dominate. Abbott
Laboratories, which makes Similac, and Mead Johnson, which makes
Enfamil, each control about 40 percent of the market. The Nestlé-owned
brand Gerber holds a roughly 15-percent share.

A market with so little competition is bound to have generous margins,
and formula makers have grown richer still because a single buyer
accounts for roughly half of all domestic sales: the United States
government. The Special Supplemental Nutrition Program for Women,
Infants and Children, commonly known as WIC, provides needy mothers with
cash assistance for certain foods, including powdered formula. When it
began, in 1972, WIC represented a fresh, lush source of inelastic
demand, by effectively eliminating from the formula market those
customers most sensitive to price. During the '80s, formula prices rose
by more than 150 percent, vastly outpacing increases in milk costs. By
the middle of that decade, formula was absorbing 40 percent of WIC's
food budget, prompting shortfalls that shunted many eligible families to
a waiting list.

In the '90s, the Senate Subcommittee on Antitrust, Monopolies and
Business Rights; the Federal Trade Commission; and attorneys general
from 19 states pursued formula manufacturers for price-fixing and
illicit marketing. Multimillion-dollar fines were assessed, but no firm
admitted wrongdoing. Even today, formula prices bear the imprint of
yesteryear's state-enabled gouging; according to a 2009 report by the
Notre Dame economist David Betson, ``the WIC program accounts for 91
percent of the increase in the growth of real formula prices'' between
1981 and 2002.

Products like formula --- expensive but with slim retail margins --- are
vulnerable to black-marketeers. Independent store owners, for example,
don't buy enough formula to qualify for the bulk discounts that
manufacturers offer big chains. But if they can acquire off-market
formula at subwholesale prices and resell it for the usual rates, they
can improve their bottom lines.

By the middle of 2012, Dattadeen's formula business began to thrive. But
certain irregularities emerged in her supply chain. Some of her
``customers,'' as she called the people from whom she bought formula,
would sell her three to 10 cans every few weeks. That seemed about
right. Why, after all, would anyone have more leftover formula than
that? But others routinely had much more. There was a woman named Julie
who met Dattadeen regularly, sometimes with more than \$1,000 worth of
formula. A couple named Krystal and Chris offered 80 to 150 cans at each
transaction. Another couple, Brian and Jessica, often showed up to
meetings with so much formula that they had to load it into their trunk.

Dattadeen listened to explanations from such sellers about their
sources: an aunt with a baby store, a warehouse-based wholesaling
business. But when Dattadeen voiced reservations about them to
Tondreau-Leve, Tondreau-Leve suggested that she accept the stories at
face value, or adopt a don't-ask-don't-tell attitude. This all seemed
out of step with a business predicated on buying up surplus product, but
it soon became clear that Formula Mom did not function quite as
advertised; Tondreau-Leve seemed most interested in ``customers'' who
could offer formula in bulk. To ensure that Dattadeen could handle these
large purchases, Tondreau-Leve had her set up an account with Bank of
America, and wired her cash in advance of sales. Tondreau-Leve even
began making requests. ``I need 35 Similac Advance cans,'' she wrote in
a text message. ``Can you see if Julie can get that for us?''

Dattadeen quickly found good uses for her new income. In June 2012, she
traveled to a medical conference in Illinois for D. That July, she
bought a new minivan, texting a photo to Tondreau-Leve. ``I still can't
believe my dream is parked up in my driveway,'' she wrote. She bought a
walker for D. and enrolled A. in the Cub Scouts. A. ``is going to call
you tonight, to try and sell you popcorn for boyscouts,'' Dattadeen
texted Tondreau-Leve soon after. ``Its more of a teaching thing for him
so don't feel obligated to buy. I just had him make a list of everyone
we are close to.''

Dattadeen's social life had been limited, and perhaps as much as her
job, she seemed to value the friendship that had blossomed between her
and Tondreau-Leve. ``I missed talking to you today!!!! GIRLFRIEND,'' she
texted in October 2012. ``Do u realize we've known each other for like a
year now.''

Fifteen years older than Dattadeen, Tondreau-Leve had studied computer
science at college; she was logical and organized --- everything
Dattadeen wasn't. But like Dattadeen, she had two sons. Her husband, a
salesman, was often away from home, so she, too, was frequently alone.
In Dattadeen she'd found an openhearted protégée, ever eager to please.
Dattadeen sometimes called her Momma. Once, when Tondreau-Leve was
swindled by an out-of state seller, she called Dattadeen in tears. When
Dattadeen fought with her husband, Tondreau-Leve offered counsel.
``Don't feel bad for pushing him away,'' she texted. ``He has to see
that you still love him, but he has to build the relationship again.''
Outside work, they met for dinner at the Cheesecake Factory, took their
boys camping and saw a Trans-Siberian Orchestra concert. In 2013,
Dattadeen took her first real vacation in years, staying not far from
Tondreau-Leve's home on the Atlantic coast so that they could spend time
together. On her last night in town, Dattadeen sent her a text: ``You
are the best friend a woman can have.''

\textbf{In spring 2012,} Kevin Shultz, a loss-prevention manager for
Publix Supermarkets, received the first of what would be many reports
that year concerning a mysterious plague of thefts. Initially, the
missives trickled in from stores around Tampa, where Shultz is based.
But Publix has more than 1,000 locations scattered through the
Southeast. Some 400, mostly in Florida, fall under his purview, and
soon, he was getting similar reports from all over the state: Cape
Coral, Fort Myers, Orlando, Miami. The thieves seemed to be multiplying
--- and all they wanted was baby formula.

Loss prevention is the rare topic about which competing retailers will
trade intelligence, and from talking to his counterparts at Walmart,
Target and Walgreens, Shultz learned that they were losing large volumes
of formula, too. That sort of overlap tended to rule out the employee
theft that often accounts for large-scale pilferage. To store-security
officers and local cops, who were addressing the crimes on a
case-by-case basis, the incidents didn't appear related. But Shultz had
a unique perch, and as he dug deeper, the thefts began to take on a
pattern --- the work, he believed, of organized crime.

Shultz joined Publix in 2006 after a long, varied career in law
enforcement --- time on the homicide squad in Plant City, Fla.; stints
embedded with the United States Marshals and the Drug Enforcement
Administration --- and was soon assigned to a newly created position,
Organized Retail Crime (O.R.C.) Investigator, to help stem what
retailers had come to recognize as a growing problem. Bob Moraca, the
vice president of loss prevention at the National Retail Federation
(N.R.F.), a trade group, says that in the last few years, O.R.C., which
accounts for \$30 billion in annual losses in the United States, has
overtaken ``internal shrink'' --- that is, employee theft --- as the
greatest threat to retailers' bottom line. A 2016 survey by the N.R.F.
found that 100 percent of respondents --- a sample representing
thousands of stores --- had been victimized in the last 12 months by
organized crime, more than in any other year.

Retail-crime groups generally share a pyramidal structure: a boss up
top, captains and lieutenants below and professional shoplifters, or
``boosters,'' at the base. The internet has been catalytic, providing
those without underworld ties new ways of fencing goods: auction sites
and payment methods that facilitate anonymity. Prime targets include
clothing and handbags --- expensive, high-demand goods that are
relatively easy to conceal --- as well as top-shelf liquor, pain-killers
and laptops. Pharmacy wares are favorites, too. In 2008, Shultz helped
neutralize a theft ring dealing in health and beauty products. The
following year, he broke up a group of middle-aged Florida men who
sometimes feigned infirmity --- affecting a limp, using a motorized cart
--- to help them relieve retailers of countless razor blades. The
third-most-targeted item, according to the 2016 N.R.F. report, was
infant formula.

Shultz's formula inquiry began in the usual way, with information
flowing to him from lower-level personnel --- mostly video and photo
stills depicting shoplifters in various stages of the act. Shultz noted
that the thefts did not seem attributable to people stealing for
personal use --- they were taking way too much. And they were craftier
than your run-of-the-mill smash-and-grabber. One man, in Orlando, liked
to select an opaque, lidded storage bin from a sales display, fill it
with formula, then proceed through the exit doors, brandishing a phony
receipt for the bin. Others worked in teams. One couple used their
children as camouflage, stowing their take in a specialized diaper bag
that retained its shape empty or full. Another hid formula in a stroller
with a spacious undercarriage. Many thieves favored reusable Walmart
bags, which had the advantage of a substantial, precise capacity: 18
12.5-ounce cans of formula (three layers of six), or nine 1.45-pound
tubs (three layers of three).

As Shultz identified repeat offenders, he circulated the footage to
other stores he considered likely targets for the same thieves. But
clever boosters are difficult to apprehend; store-security officers
serve a largely deterrent purpose, and cashiers and stock clerks cannot
be expected to confront criminals. To improve its prospects, Publix
stepped up inventory checks in formula aisles, enabling the company to
zero in on the window during which a theft occurred, alert nearby stores
and swiftly fold the data into the case file.

One night in July 2012, Shultz got a break --- a phone call from a
Publix in Pinellas County, about 20 miles from Tampa. Police officers
there had apprehended a couple trying to steal 13 cans of formula.
Shultz recognized them from security footage. They made a striking pair;
slender, with brown hair, the woman looked slight beside her partner, a
lumbering man with an ex-lineman's physique. Now, under arrest, they had
names: Jessica Gordon and Brian Oliver, both about 30. Shultz hoped they
could provide some clue about the organization he suspected they were
working for. In an interview with Gordon arranged by the Pinellas County
Sheriff's Office, he learned that the couple had two buyers. One proved,
after a brief investigation, to be an online business trading in
quantities too small to explain the losses retailers were seeing.

The other was a 32-year-old woman named Alexis Dattadeen.

\textbf{Dattadeen read online} about the arrests. She was frightened ---
by then, she'd bought formula from Gordon and Oliver at least 10 times.
But when she told Tondreau-Leve what had happened, Tondreau-Leve seemed
unruffled, concerned mostly with coaxing Dattadeen back to work --- and
Dattadeen didn't want to disappoint her. Encouraging and sympathetic at
times, Tondreau-Leve could also be stern. She had become increasingly
focused on high productivity, bristling at mistakes. A sizable haul was
one thing that Dattadeen could be sure would make her happy, and so,
when Oliver contacted her some weeks later, offering to sell more
formula, she agreed to meet.

One evening in September 2012, Dattadeen pulled her minivan into a
Walmart parking lot in Palm Harbor, a community south of New Port
Richey. Soon, Oliver arrived, driven by someone Dattadeen had never met.
She got out of her car and greeted Oliver, who hulked over her in a
green polo shirt. ``Nice to see you again,'' she said, hugging him. The
other man was introduced as Donnie. His real name was George Moffett ---
and he was a Pinellas County sheriff's deputy. From an unmarked car,
another officer filmed the exchange. ``I've stolen a lot of
{[}expletive{]} in my day,'' Moffett says in the video. ``But this
{[}expletive{]} was hard.'' Dattadeen smiles bashfully, but doesn't
otherwise respond. Moffett presents himself as Oliver's source. In his
trunk are 90 tubs and 18 cans of Similac formula, which had been
provided to the sheriff's department by Publix. ``So you're the guy
who's been getting all this?'' Dattadeen asks. ``Do you have a number
--- I could call you?''

In the coming weeks, Dattadeen met ``Donnie'' three more times. In
recordings of their interactions, she speaks in a high, soft voice, with
the scattered animation of a teenager. Details about Formula Mom pour
forth: Dattadeen has a partner, near Orlando, whom she meets on Mondays;
the formula is shipped to Massachusetts, and also to China. ``She's not
as easygoing as I am,'' she said once, referring to Tondreau-Leve. ``I
can tell you that.''

In October, Shultz drove to a gas station off Interstate 4, which
connects Orlando and Tampa. He had been guided there by the Pinellas
County Sheriff's Office, which had relayed intelligence to him about
Dattadeen's movements gleaned from a GPS tracker that it had attached to
her van. At the rear of the parking lot, he found Dattadeen's van backed
against a white fence, with slats spaced widely enough that he could
shoot video through them. Beside it, another van had backed in. Both
trunks yawned open. Dattadeen and a woman unknown to Shultz moved
formula from one van to the other. ``White female, appears to be
middle-aged --- late 40s, early 50s, shorter hair,'' Shultz narrates,
describing the stranger over a Rascal Flatts song playing from his
radio. ``Multiple cans of formula,'' he says. ``Multiple, multiple
stacks.'' After the transfer was complete, Shultz followed the older
woman to a house about 15 miles away, taking down a license-plate number
that he traced to a rental contract signed by one Alicia Tondreau-Leve.

\textbf{Tondreau-Leve arrived in} Florida with her two adolescent sons
in July 2011, becoming one of the roughly 800 new residents the state
absorbs per day, an aspirant lured by its redblooded interpolation of
Californian mythos: sun, surf, opportunity. Having recently endured a
difficult period, the family had come from Freetown, Mass., a quaint
village of about 9,000, where Tondreau-Leve's husband, Alan Leve,
remained in his sales job while he looked for work in Florida. In 2010,
Alan realized a long-held dream of opening a restaurant. But the
business failed, and the family filed for bankruptcy. Sonoma at Viera,
the subdivision 45 miles southeast of Orlando where the Leves settled,
might reasonably have seemed an auspicious place to start again.
Diminutive palms lined its streets, which bumpered neat lawns garnished
by flowering shrubs. Backyards were generously puddled with artificial
ponds. Everything was new.

Not long before Tondreau-Leve moved to Florida, Alan's cousin, Michael,
presented her with an opportunity. He knew the secondary retail market
well, having for years run a store in Massachusetts that took a
commission for selling customers' goods on eBay. He suggested that
formula might earn her some extra money: She could buy unused cans from
Florida moms and ship them to him. Initially, Tondreau-Leve had little
sense of how to go about things, so she simply did what she could; if
she had to drive 15 miles to buy two cans, so be it. Gas costs could
exceed the value of her purchases, but she viewed the legwork as an
investment. Eventually, she learned to schedule all her deals in a given
metro area --- Tampa, say --- on the same day. When she'd gathered
enough formula to fill a moving box, she shipped it to Michael,
profiting about \$2 per can.

If Tondreau-Leve ever began to sense that her business was mutating from
a wholesome D.I.Y. venture into something malignant, there was little in
her outward presentation to suggest it. Soon after moving in, she told
Cindy Lashomb, who lived next door with her husband, about a company
she'd started, dealing in surplus infant formula. There was nothing
secret about it, Lashomb told me; Tondreau-Leve even affixed a magnet
advertising the business to her car. Another neighbor, Donald Egan, knew
about the business, too; he discerned in Tondreau-Leve the makings of a
mogul, once describing her as ``the pants in the family. She ran
everything. You could tell by her attitude and the way she approached
you that she was the alpha male, so to speak.''

In the spring of 2012, Lashomb began to suspect that something was not
quite right next door. Suddenly, it seemed, trucks were always coming
and going, picking things up or dropping them off. Lashomb complained to
Tondreau-Leve, but the traffic worsened --- UPS deliveries gave way to
tractor-trailers. Once, an 18-wheeler blocked the Lashombs' driveway,
trapping them at home.

Tondreau-Leve eventually shifted operations to a storage facility,
renting a single unit, then a double, filling it to capacity with
head-high stacks of formula. By early 2013, Formula Mom had statewide
reach. A woman named April covered Fort Myers. A man named Angel, who
owned a consignment store, handled Miami. Giulyanna took care of
Orlando. To each subcontractor, Tondreau-Leve provided the same start-up
guide. Each set up a Bank of America account. Having some experience
with consignment stores, Dattadeen voiced concerns about Angel's supply,
which arrived in huge quantities --- a fact she happened to notice
during a drop-off at Tondreau-Leve's storage unit. But Tondreau-Leve
dismissed them. Having expanded her network, Tondreau-Leve was less
dependent on Dattadeen, and she had cooled on her, berating her over
botched invoices, and comparing her unfavorably with her other deputies.

Demand had meanwhile grown alongside supply. That year, Tondreau-Leve
stopped selling to Michael in favor of a California woman named
Lissette, who ran a formula-wholesale business. Lissette had contacted
Tondreau-Leve online, offering her a much better price than Michael did.
By playing Lissette off yet another buyer, in Wisconsin, Tondreau-Leve
negotiated an even better rate. Buyers from New York and New Jersey
materialized, too. Tondreau-Leve did not much trouble herself about what
became of her formula after she sold it, but the buyers mostly seemed to
be wholesalers, redistributing the product to small stores. At least one
owned a Brooklyn bodega, and Tondreau-Leve knew that Lissette had
customers in China, where a tainted-formula scare in 2008 had stoked
demand for foreign brands, which were seen as safer.

Between January 2012 and July 2014, records from several freight
carriers indicate that Tondreau-Leve shipped a total of roughly 60 tons
of product. During that period, she received nearly \$300,000 from her
California client, \$300,000 from buyers in Wisconsin and nearly \$1
million from New York City buyers. No longer a neophyte, she proved
herself an able businesswoman, adept at navigating complex logistics. In
missives to underlings, she presents as a sober, precise manager. ``I
appreciate everyone's patience over the last couple of months,'' she
once wrote. ``Because of the Chinese market not taking formula from the
U.S., it has left a surplus of Nutramigen and Alimentum in everyone's
hands. Hopefully now that the doors are open again in China, we will see
some of this product move.''

\textbf{Shultz continued} to watch formula disappear from his shelves in
2013, but the mechanics of the operation remained unclear. Late that
year, he approached Jeff Newcomb, an agent with the Orange County
Sheriff's Office, which covers metropolitan Orlando, with a lead that he
believed was connected to much of the area's missing baby formula: a
Formula Mom Craigslist ad, with contact information for a ``Julie.'' By
then, the Florida Department of Law Enforcement, as Florida's state
police is known, had joined the case in a supervisory role,
synchronizing police efforts in different jurisdictions. F.D.L.E. agents
had coordinated additional surveillance, following Dattadeen to meetings
beyond Pinellas County, and trailing Tondreau-Leve through Central
Florida. A prosecutor from the attorney general's office had also been
assigned, adding subpoena power to an investigation that would produce
mountains of paper evidence: bank and phone records, wire transfers,
emails. Bit by bit, law enforcement was working toward a case against
Formula Mom under Florida's RICO Act, a version of the 1970 federal
anti-racketeering law designed to cripple the mafia.

Having worked several previous RICO cases, Newcomb was familiar with the
methods of the genre --- turning lowly button men against capos and so
on. But this time, rather than work up the chain of command, Newcomb
would establish a pattern of criminality at the base of the pyramid,
sketching a fuller portrait of the enterprise. Using a criminal
informant to conduct a controlled sale, Newcomb quickly identified
``Julie'' as 30-year-old Giulyanna Guzman, Tondreau-Leve's Orlando
deputy.

A single mother with a 2-year-old daughter, Guzman was organized and
hardworking, often waking early and retiring late. She spent the hours
between seeking out formula sellers online and distributing the business
cards Tondreau-Leve had given her. Since starting with Formula Mom in
early 2013, she had developed a stable of fruitful sources. But like
Dattadeen, she had concerns about a number of them. Guzman was meeting
some people every day, sometimes more than once. They would demand to
see her on short notice, urgently and at odd hours. In some cases, she
did not feel right about their being around her daughter; they seemed
\emph{off.}

One such supplier was Janine Piccirillo, who'd learned the formula
racket from a friend: what to look for; how to get it; and, most
important, that there was a woman named Julie who would always buy it.
Piccirillo could often manage to make off with two Walmart bags full of
formula at a time --- roughly \$550 worth at retail. She worked with a
driver, Jennifer Day, who would obscure her license plate with a sheet
of paper and wait outside. They might hit three stores in a day. After
selling their haul to Guzman, Piccirillo and Day usually drove straight
to see their dealer.

One day in 2013, Day was dope-sick, and went to work alone, visiting a
Publix in the morning. Noticing someone filming her with a phone, she
left and drove to Walmart. Her symptoms were worsening, and she resolved
to steal as much as she could, leaving the store with a bag on each arm.
But when she reached the parking lot, she was swarmed by masked police
officers with their guns drawn.

When I met him recently at an Orange County police station, Newcomb
grinned at the memory, explaining that they'd thought Day might be armed
--- she'd been surveilled for about a month, and had once been observed
leaving a gun shop. During the arrest, Guzman, who was listed in Day's
phone as Formula Mom, called repeatedly --- Day was late for their
meeting. She also had numbers saved for Alicia and Alan Leve. In all,
Newcomb said, he identified about 15 boosters associated with Guzman. I
asked how many were drug users. He paused, considering. ``I don't think
there was a single one who wasn't,'' he said. Most if not all were
addicted to opioids --- using heroin, prescription painkillers or both.

\textbf{In the summer} of 2014, the F.D.L.E. staged a dramatic final
scene in Operation Baby Burp, as their investigation ultimately came to
be known. They had by then developed an intricate understanding of
Formula Mom. But much of the evidence against Tondreau-Leve was
circumstantial. The police could barely ever connect her directly to
thefts. Several officers who worked the case suggested to me that ---
like many dons before her --- Tondreau-Leve had ``insulated herself.''
If they got the case to trial, the question of her guilt would turn
largely on what she knew or should have known --- Florida's standard for
dealing in stolen property --- and on a legal concept known as willful
blindness, or the ``deliberate avoidance of positive knowledge.'' For
juries, though, such esoterica does not play nearly as well as one last,
spectacular bust. (Tondreau-Leve declined to be interviewed for this
article. Most of her and Dattadeen's story comes from court documents.)

That April, an F.D.L.E. agent named Shawn Sloan went to the Dattadeens'
home, accompanied by Shultz, hoping to turn Dattadeen against
Tondreau-Leve. They did not lack leverage; Dattadeen had repeatedly
discussed buying stolen property on tape. When her predicament was made
clear to her, she quickly agreed to cooperate. Following Sloan's
instructions, in a recorded meeting later that month --- a meeting
during which she was terribly nervous --- Dattadeen told Tondreau-Leve
that she didn't want to work with her anymore. Some of her suppliers,
she said, were making her uncomfortable, particularly a man named Steve,
who had said he could get whole shipping pallets' worth of formula.

The ploy worked. Soon after their conversation, Tondreau-Leve contacted
Dattadeen to ask for Steve's number. Dattadeen obliged. The phone number
she provided belonged to William Powell, an F.D.L.E. agent posing as a
supplier. ``Good morning,'' Tondreau-Leve texted him in early May. ``My
name is Alicia, and I received your number from Alexis.'' Powell/Steve
wrote back: ``I can get it pretty regular, and Alexis said you're
reliable.''

The next day, Powell met Alicia and Alan in a CVS parking lot near their
home, selling them 65 Similac tubs, which he indicated he'd gotten from
his brother, a Publix employee. As represented to Powell,
Tondreau-Leve's vision of Formula Mom had changed markedly; she made no
mention of needy mothers, of redistributing formula in the community.
``I'm the go-to person here in Florida,'' she says in a recording of the
meeting. ``I have an endless amount of money basically.'' Over the next
month, they met twice more. Tondreau-Leve repeatedly sought assurances:
``Please confirm that these cans are acquired legally!'' she texted when
she first made contact. But when Powell equivocated, as invariably he
did, she didn't seem deterred. ``I'd love to know where he's getting it
from,'' she says during their second meeting, meaning the brother. When
Powell tells her, ``Everything's good, Alicia,'' for a moment on the
recording, she appears flustered: ``No, not `everything's good,'
because, you know, I don't steal --- I don't buy stolen.'' Then she
resumes loading her purchase.

On July 2, Tondreau-Leve arranged to meet Powell in a Lowe's parking lot
for what was to be the largest deal of her career: 3,300 tubs of formula
--- six pallets' worth. The load had a retail value of about \$85,000.
Tondreau-Leve had agreed to pay \$33,000 in cash, which she stored for
some days before the transaction in a coffee can on her kitchen counter.
In the days beforehand, she seemed to overcome her trepidation about the
formula's origins, texting Powell requests for future orders: 2,000
Similac Advance cans, 400 Enfamil boxes, Similac Go-and-Grow. But on the
morning of the deal, after the pallets were transferred from Powell's
truck to the one Tondreau-Leve had rented, she and her husband, who
accompanied her that day, were arrested. Amid a blare of sirens from
unmarked cars stationed in the lot, the police descended from all sides.
Still in character, Powell cried out, ``What'd you do to us, Alicia?''

\textbf{In August 2016,} a jury convicted Tondreau-Leve of charges
including racketeering, dealing in stolen property and money laundering.
Alan was convicted of conspiracy to commit racketeering. The judge,
Wayne Durden, sentenced Tondreau-Leve to 20 years in prison, citing,
among other factors, her refusal to acknowledge wrongdoing. Alan
received a seven-year term. (At trial, Alexis Dattadeen and Giulyanna
Guzman gave extensive, notably contrite testimony; both received
probation.) The Leves' sentences struck me as severe, and I asked Pam
Bondi, the Florida attorney general, if she agreed. ``I wish she'd been
locked up for as long as humanly possible for what she did,'' she said
of Tondreau-Leve. ``Had she used her wits to start a legit business, she
could have been incredibly successful --- a true entrepreneur.''

We were sitting in a sparsely decorated corner suite in Bondi's Tampa
office with Paul Dontenville, the lead prosecutor in the case, and
Nicholas Cox, Florida's statewide prosecutor. Before us on a table lay a
binder of Formula Mom price lists and invoices. The documents made a
polished, professional presentation. They suggested pride of ownership,
looking to me not at all like the work of someone trying to hide
something. I asked Dontenville if, for an organized crime boss,
Tondreau-Leve hadn't gone about things in a rather unusual fashion. She
had registered her company with the state --- displaying related
paperwork inside her rented storage unit --- and handled payments via
commercial wire transfer. She had printed business cards and sent email
that amounted to evidence of conspiracy. Clean cut and rigidly
matter-of-fact, Dontenville declined to ruminate on the subject. He'd
been surprised, though, when he received discovery materials from the
defense, that they included Tondreau-Leve's text messages, which she
hadn't been required to provide. Tondreau-Leve evidently considered them
exculpatory; Dontenville found them useful in making his case.

Before trial, the prosecutors presented a deal that would have meant
much less prison time for Alicia and mere probation for Alan. But the
charges to which they were asked to plead apparently did not comport
with their sense of their actions, or of themselves. They turned the
deal down. Soon after, the Leves met for a voluntary pretrial interview
with prosecutors, during which they insisted stubbornly that Alicia had
run a legitimate business. She had worked hard and become successful. If
she'd erred, it had been through naïveté, trusting the wrong people. She
was a business owner like any other, victimized by dishonest underlings.

Last July, I went to see Dattadeen at her house in New Port Richey. When
I pulled up, she was outside, hanging wet towels and bathing suits. She
wore a tank top and had a reddish tan, her hair damp and straight. From
the corner of her mouth, she smoked a cigarette. She spoke in a high,
soft voice --- the voice from the surveillance footage --- but she'd
grown less trusting than she was then, regarding me warily and declining
to speak in detail about her experiences with Tondreau-Leve. Formula Mom
had been a mistake, and she was moving on with her life. Ronald came
outside and we talked in the deepening dusk under a full yellow moon.
When Dattadeen went to make a call, Ronald asked me, a little bashfully,
about ``the lady,'' meaning Tondreau-Leve. When I told him she was in
prison, he looked sorry. Then he said: ``The others admitted they were
wrong. She was in denial.''

When Dattadeen and Tondreau-Leve were still close, Tondreau-Leve often
spoke to her of her anger at losing the home she and her husband had
owned in Massachusetts --- a stately four-bedroom Colonial on an acre of
land. Visiting from Florida, while Alan was still living there, she
found it so upsetting to be in the house, which she knew by then that
they would lose, that she sometimes preferred to spend time in a rented
R.V. Formula Mom became a means of redemption and reinvention. The Leves
had been renting in Florida, but about five months before their arrest,
they began thinking of buying property: Lot 6HH at Tralee Bay Estates, a
new development where the homes had granite countertops, Bosch
appliances and ``Tuscan-inspired architecture.''

When law-enforcement officers spoke of Tondreau-Leve ``insulating''
herself, they meant it in a legal sense, that the layered structure of
Formula Mom had made prosecuting her more difficult. But Tondreau-Leve's
regional deputies also largely buffered her from the sad, grimy
underpinnings of her business: the furtive meetings in parking lots; the
twitching, nauseated victims of dope-sickness. In a way, the
architecture of her organization had insulated her too from a damning
kind of self-knowledge, even as her daily duties --- arranging
cross-country shipments, negotiating rates, collating spreadsheets,
tracking policy changes --- encouraged her to indulge an entrepreneurial
delusion. It seemed likely that what Tondreau-Leve knew or should have
known had been at least partly occluded by what she badly wanted to
believe.

On July 2, 2014, when she drove to the Lowe's parking lot, Tondreau-Leve
took with her an immaculate-looking receipt made out to Steve Riley, the
full name provided to her by Agent Powell. It had been perhaps two years
since she'd told Dattadeen not to pick up her prototype receipt from
Office Depot, and Formula Mom had become a different kind of company. In
the pretrial interview, she would tell Dontenville that in her last
transaction, she believed she was buying from Publix at a wholesale
rate. At the top of the receipt for Riley, she'd printed the Formula Mom
logo --- the woman with flowing hair holding an infant aloft.

\emph{Sign up for}
\href{http://www.nytimes3xbfgragh.onion/newsletters/magazine}{\emph{our
newsletter}} \emph{to get the best of The New York Times Magazine
delivered to your inbox every week.}

Chris Pomorski is a freelance writer who has contributed to The
Guardian, New York, and Bloomberg Businessweek. This is his first
feature for the magazine.

\hypertarget{more-crime-stories}{%
\subsection{More Crime Stories}\label{more-crime-stories}}

\begin{itemize}
\tightlist
\item
  \href{https://www.nytimes3xbfgragh.onion/interactive/2018/05/02/magazine/money-issue-baby-formula-crime-ring.html?action=click\&module=MagazineModule\&pgtype=Article\&contentCollection=Magazine\&region=Footer}{}
\item
  \href{https://www.nytimes3xbfgragh.onion/interactive/2018/05/02/magazine/money-issue-insys-opioids-kickbacks.html?action=click\&module=MagazineModule\&pgtype=Article\&contentCollection=Magazine\&region=Footer}{}
\item
  \href{https://www.nytimes3xbfgragh.onion/interactive/2018/05/03/magazine/money-issue-bangladesh-billion-dollar-bank-heist.html?action=click\&module=MagazineModule\&pgtype=Article\&contentCollection=Magazine\&region=Footer}{}
\item
  \href{https://www.nytimes3xbfgragh.onion/interactive/2018/05/03/magazine/money-issue-iowa-lottery-fraud-mystery.html?action=click\&module=MagazineModule\&pgtype=Article\&contentCollection=Magazine\&region=Footer}{}
\item
  \href{https://www.nytimes3xbfgragh.onion/interactive/2018/05/03/magazine/money-issue-white-collar-crimes-cheat-sheet.html?action=click\&module=MagazineModule\&pgtype=Article\&contentCollection=Magazine\&region=Footer}{}
\end{itemize}

it.

\hypertarget{more-on-nytimescom}{%
\subsection{More on NYTimes.com}\label{more-on-nytimescom}}

Advertisement

\hypertarget{site-information-navigation}{%
\subsection{Site Information
Navigation}\label{site-information-navigation}}

\begin{itemize}
\tightlist
\item
  \href{https://help.nytimes3xbfgragh.onion/hc/en-us/articles/115014792127-Copyright-notice}{©
  2020 The New York Times Company}
\item
  \href{https://www.nytimes3xbfgragh.onion}{Home}
\item
  \href{https://www.nytimes3xbfgragh.onion/search/}{Search}
\item
  Accessibility concerns? Email us at
  \href{mailto:accessibility@NYTimes.com}{\nolinkurl{accessibility@NYTimes.com}}.
  We would love to hear from you.
\item
  \href{https://help.nytimes3xbfgragh.onion/hc/en-us/articles/115015385887-Contact-Us}{Contact
  Us}
\item
  \href{https://www.nytco.com/careers/}{Work with us}
\item
  \href{https://nytmediakit.com/}{Advertise}
\item
  \href{https://help.nytimes3xbfgragh.onion/hc/en-us/articles/115014892108-Privacy-policy\#pp}{Your
  Ad Choices}
\item
  \href{https://help.nytimes3xbfgragh.onion/hc/en-us/articles/115014892108-Privacy-policy}{Privacy}
\item
  \href{https://help.nytimes3xbfgragh.onion/hc/en-us/articles/115014893428-Terms-of-service}{Terms
  of Service}
\item
  \href{https://help.nytimes3xbfgragh.onion/hc/en-us/articles/115014893968-Terms-of-sale}{Terms
  of Sale}
\end{itemize}

\hypertarget{site-information-navigation-1}{%
\subsection{Site Information
Navigation}\label{site-information-navigation-1}}

\begin{itemize}
\tightlist
\item
  \href{https://spiderbites.nytimes3xbfgragh.onion}{Site Map}
\item
  \href{https://help.nytimes3xbfgragh.onion/hc/en-us}{Help}
\item
  \href{https://help.nytimes3xbfgragh.onion/hc/en-us/articles/115015385887-Contact-Us?redir=myacc}{Site
  Feedback}
\item
  \href{https://www.nytimes3xbfgragh.onion/subscription?campaignId=37WXW}{Subscriptions}
\end{itemize}
