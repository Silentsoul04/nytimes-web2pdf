 **NYTimes.com no longer supports Internet Explorer 9 or earlier. Please
upgrade your browser.
\href{http://www.nytimes3xbfgragh.onion/content/help/site/ie9-support.html}{LEARN
MORE »}

**Sections

**Home

**Search

\hypertarget{the-new-york-times}{%
\subsection{\texorpdfstring{\href{http://www.nytimes3xbfgragh.onion/}{The
New York Times}}{The New York Times}}\label{the-new-york-times}}

\hypertarget{-magazine-}{%
\subsubsection{\texorpdfstring{
\href{https://www.nytimes3xbfgragh.onion/section/magazine}{Magazine}
}{ Magazine }}\label{-magazine-}}

 \href{https://www.nytimes3xbfgragh.onion/section/magazine}{Magazine}
\textbar{}The Prisoners Who Care for the Dying and Get Another Chance at
Life

**Close search

\hypertarget{site-search-navigation}{%
\subsection{Site Search Navigation}\label{site-search-navigation}}

Search NYTimes.com

**Clear this text input

Go

\url{https://nyti.ms/2IohSDJ}

\hypertarget{site-navigation}{%
\subsection{Site Navigation}\label{site-navigation}}

\hypertarget{site-mobile-navigation}{%
\subsection{Site Mobile Navigation}\label{site-mobile-navigation}}

\hypertarget{the-prisoners-who-care-for-the-dying-and-get-another-chance-at-life}{%
\section{The Prisoners Who Care for the Dying and Get Another Chance at
Life}\label{the-prisoners-who-care-for-the-dying-and-get-another-chance-at-life}}

In a California prison hospice, inmates become caregivers to fellow
convicts who will never make it out alive.

\hypertarget{the-health-issue}{%
\subsection{The Health Issue}\label{the-health-issue}}

\begin{itemize}
\tightlist
\item
  \href{https://www.nytimes3xbfgragh.onion/interactive/2018/05/15/magazine/health-issue-my-adventures-with-hallucinogenic-drugs-medicine.html?action=click\&module=MagazineModule\&pgtype=Article\&contentCollection=Magazine\&region=Header}{}
\item
  \href{https://www.nytimes3xbfgragh.onion/interactive/2018/05/16/magazine/health-issue-convicted-prisoners-becoming-caregivers.html?action=click\&module=MagazineModule\&pgtype=Article\&contentCollection=Magazine\&region=Header}{}
\item
  \href{https://www.nytimes3xbfgragh.onion/interactive/2018/05/16/magazine/health-issue-what-we-lose-with-data-driven-medicine.html?action=click\&module=MagazineModule\&pgtype=Article\&contentCollection=Magazine\&region=Header}{}
\item
  \href{https://www.nytimes3xbfgragh.onion/interactive/2018/05/16/magazine/health-issue-reinvention-of-primary-care-delivery.html?action=click\&module=MagazineModule\&pgtype=Article\&contentCollection=Magazine\&region=Header}{}
\end{itemize}

TINGLEY

\hypertarget{the-prisoners-who-care-for-the-dying-and-get-another-chance-at-life-1}{%
\section{The Prisoners Who Care for the Dying and Get Another Chance at
Life}\label{the-prisoners-who-care-for-the-dying-and-get-another-chance-at-life-1}}

In a California prison hospice, inmates become caregivers to fellow
convicts who will never make it out alive.

By SULEIKA JAOUAD\\
Photographs by KATY GRANNAN MAY 16, 2018

Inked in tattoos from neck to knuckle, Kevion Lyman rose from his bunk
at dawn, pulled scrubs over his skinny frame, stepped out of his cell
and set out for work. The 27-year-old strolled down the long central
hallway connecting the different wings of the prison, past the dining
hall, the solitary-confinement unit for violent offenders and the psych
ward. Pushing open the big steel doors, he reported for his morning
shift in the hospice.

Great efforts have been made to differentiate the hospice from the rest
of the prison: The windows have white shutters, root-beer floats are
occasionally served, the walls are plastered in artwork and a plastic
tree, left over from Christmas with green-and-red tinsel looping through
its branches, lights up the entrance. These attempts to add cheer go
only so far, of course. Shutters open onto iron bars. Correctional
officers escort nurses as they make rounds with a medication cart.
Inmate workers are frisked at the start and the end of their shifts. And
until recently, the only outdoor space available to patients was a small
chain-link-fenced patio nicknamed ``the dog run.'' The California
Medical Facility, a medium-security prison in Vacaville, midway between
San Francisco and Sacramento, houses general-population inmates as well
as those with health conditions and specialized medical needs. It is
home to 2,400 men --- some young and healthy, others disabled and sick,
and then those in the hospice, who are dying.

Later that January morning, Lyman and two co-workers, Fernando Murillo
and Kao Saephanh, smoothed clean sheets and a red, flowery quilt onto an
empty bed. They were ushering in their newest patient: a lanky man clad
in a navy jumpsuit named Jimmy Figueroa. Finding a spot on the edge of
his new mattress, Figueroa held a carton of fortified milk, appearing
dazed as he slurped on a straw through the large gap where his teeth had
once been. With a deep tan, a full head of silvery-black hair elegantly
parted in the middle and knockoff Ray-Ban sunglasses perched on the
bridge of his nose, he looked, as one of the men observed, like an
Italian hit man in a movie.

``I believe we've done business together,'' he said to Saephanh.

``Oh, yeah?''

``I sold you some nuclear weapons a few years back. You're Kim
Jong-il,'' said Figueroa, as though he were greeting an old colleague.
He did not appear to be joking.

Saephanh laughed, exchanging a raised eyebrow with his co-workers.
Lyman, who stood to his right, was toting a welcome box filled with
toiletries, flip-flops, a blue-and-white-pinstriped robe and a urine
bottle for emergencies. He held the bin out to Figueroa. ``Nice to meet
you, Big Dog. That's for you.''

``Religion?'' Saephanh asked, filling out the intake sheet.

``I've got five religions.'' Figueroa's face brightened into a grin as
he counted them on each of his fingertips --- ``Muslim, Catholic,
Buddhist, Christian. ... ''

``I feel you. Me, too,'' Lyman said, giving him a fist bump. ``This is
the TV, a'ight?'' He pointed to the plastic monitor on the dresser,
where Dr. Oz and Donald Trump were deep in conversation.

Murillo got down on one knee on the linoleum floor so that he was at eye
level with the patient. He put tan socks on Figueroa's feet, which
dangled off the side of the bed, and gave them a squeeze. ``We're here
for you --- anything you need. Now, it's time to get some rest.''

``I've got a big brain!'' Figueroa exclaimed. ``Did you know I'm a very
important man? I'm the son of Hitler, and I work for the U.S. government
on very important projects.''

Murillo placed a gloved hand on Figueroa's shoulder and stood up to
leave. ``Even people with big brains need to rest.''

The hospice at the California Medical Facility is one of the nation's
first and the only licensed hospice unit inside a California prison.
Built in 1993 in response to the AIDS crisis and inmate-led demands for
more humane care, the hospice was originally populated with young men
dying of complications of the disease. Today, the 17-bed unit is filled
with a different demographic: graying men with everything from end-stage
cancer to Alzheimer's shuffle around with walkers, sit in wheelchairs
watching television or lie curled up under heavy blankets.

Prisoners older than 55 serving time in federal and state prisons make
up the fastest-growing age group behind bars, increasing more than 500
percent since the 1990s, from 26,300 aging inmates in 1993 to 164,800 at
the end of 2016. Criminal-justice experts point to a mix of policies
that landed us here: long sentences from get-tough-on-crime laws, a
steady increase of older adults entering prison and challenges with the
timely issuing of compassionate release and medical parole. One result
is a different kind of death penalty for violent and nonviolent
offenders alike. As one patient said to me when I first arrived at the
hospice, ``Welcome to death row.''

Most prisons were never built to be nursing homes. Correctional officers
often aren't equipped with the necessary training, and medical staff can
be spread thin. Here at the California Medical Facility, that's where
men like Lyman, Saephanh and Murillo come in. They are part of a cohort
of about two dozen men called the Pastoral Care Service Workers. Most of
them are convicted murderers serving life sentences who have been
granted an unusual role: providing dignified deaths to their fellow
inmates.

A job in the hospice is not easy to come by. To qualify, Lyman and the
others first have to pass a series of interviews and disciplinary checks
and agree to random drug tests. They do 70 hours of preliminary training
in the psychological and spiritual dynamics of end-of-life care, bedside
etiquette and the bereavement process. But the real education comes with
the patients. Keith Knauf, a Presbyterian chaplain who oversees the
program, believes that caring for the dying teaches compassion and
changes these men in profound ways. Of some 250 workers who have been
released from prison since the program began, he says, none has returned
for a felony and only three have returned for minor parole offenses.
Knauf's estimates put the program's recidivism rate at 1.2 percent.
Nationally, around 25 percent of federal inmates return to prison within
eight years.

Seven days a week, the workers pull 10- to 15-hour shifts, often longer.
It's one of the lowest paid jobs available, making just 15 to 32 cents
an hour. They brush patients' teeth, massage sore limbs, read books out
loud, strip soiled mattresses and assist the medical staff. Trust is a
rare currency in prison, and some patients whisper conspiracies that the
hospice doctors and nurses prioritize the interests of the
criminal-justice system over their well-being. The workers can serve as
the trusted middlemen between the patients and medical staff. When
patients are in their final hours, it is the workers who sit bedside,
holding round-the-clock vigils. They pride themselves on their policy:
No prisoner here dies alone.

Whether you're in prison or on the outside, the person being cared for
or the caretaker, dying happens in unpredictable ways. Some patients go
within hours, like a skeletal man in his 70s who was admitted one
morning while I was visiting the unit and zipped into a body bag by
sundown. Others the workers get to know well over many weeks. One
patient named Lamerrill Dawson had been there for nearly five months, so
long that he wondered if he even belonged in the hospice. Known for
terrorizing the ward with tantrums, Dawson treated the workers like his
personal butlers. He reserved a special kind of torment for Lyman,
badgering him with a never-ending list of demands. ``Change my diaper,
{[}expletive{]}!'' he was known to holler. ``This food is
{[}expletive{]}.''

The workers make a point not to find out what the patients have done.
They worry that knowing too much could affect the quality of care. When
a patient's past sins cross over into the realm of the horrific, it can
be hard to keep creeping judgments and questions at bay. How do you
reconcile the dissonance between the serial killer and the elderly
patient, bedridden, incontinent and lost in the fog of dementia? The
workers are also in prison for crimes, but that doesn't make them immune
to judgment. ``Death can be an equalizer,'' Lyman said. The past falls
aside. Time is grounded in the shifting demands of the body as it begins
its decay.

Saephanh is the hospice's self-appointed barber, and on a sunny,
cloudless day, he promised to give Ralph Martinez, a patient with
cirrhosis of the liver, a haircut. Martinez sat in a rust-red barber
chair outside in the dog run. He tipped his head back and closed his
eyes, letting the noon sun graze his sallow skin. Saephanh got to work
with the clippers, sending snippets of black hair skittering onto the
pavement beneath their feet (the dog run has since been turned into a
garden). According to Saephanh, in most other prisons, a Latino would
never get a haircut from an Asian barber, or vice versa. Invisible
boundaries carve up the cellblocks, and consorting with the ``wrong
kind,'' especially for gang members like Martinez, who belonged to
Nuestra Familia, can get you ``got.'' But within the walls of the
hospice, these unspoken rules don't seem to matter as much. Black men
give meal trays to white men with swastika tattoos on their faces, Crips
play cards with Bloods and everyone here --- regardless of creed, race
or politics --- gets his hair cut by Saephanh.

Saephanh says he started hospice work because he was looking for
redemption. ``I've done a lot of bad things in the past, and I feel like
I can do some good to try and make it right with him. It's the Christian
thing to do.'' He grew up in Merced, a nearby city, going to church
every Sunday with his five siblings. His father, a landscaper, and his
mother, a janitor, emigrated here from the mountains of Laos after the
Vietnam War. In eighth grade, Saephanh started hanging around some older
cousins who were involved in ganglike activities. He thought they were
cool and wanted to be more like them --- ``drinking, drugs, partying and
guns, you know.'' Then, at age 17, Saephanh was at a Halloween party,
and a brawl broke out. ``I'd never seen anybody really die except for
the time that I did it, that I shot and killed the guy. Even then I
didn't watch, I just did it, and I left.''

That was 12 years ago. Now, at 29, Saephanh watches people die
regularly. His friends give him a hard time about working in the
hospice. ``They're like, `Aw, man, you're in there wiping butt!' ''
Saephanh told me. He doesn't mind much. He keeps looking for small ways
to make patients smile and feel a little more comfortable before they
go. He spends his wages on ice-cream cones and vending-machine snacks to
pass around. He pays special attention to the more isolated patients,
like Ernest Marin Jr., a 79-year-old who quickly earned a bad rap in the
hospice for stealing from the bunks of others. Saephanh had heard
disturbing rumors about the gruesome crime he committed on the outside.
He didn't know if it was true, but he was determined to treat Marin the
same way he would any of the others. Having your freedom taken away,
having to die in prison, as most here did, with gutting guilt,
monolithic regret or the suspicion that the system was rigged against
them all along was punishment enough, he believed.

Saephanh gave Marin a gray trucker hat, which he wore every day. When he
noticed that Marin's shoes were falling apart, he asked around to see if
anyone had a pair of size 10 sneakers that they might be willing to
donate. These gestures caught Marin by surprise. Slowly, he stopped
misbehaving as much and started airing out old jokes. By the time I met
Marin, he had become the resident comic, spontaneously breaking out into
what he called the ``boogie-woogie'' as he danced in his wheelchair,
snapping his dentures in the air like castanets. ``He trusts me to make
his bed and to go into his area without him there,'' Saephanh told me,
beaming.

Most of the other workers were just as candid and forthcoming about
their pasts --- it was what led them to the hospice. Murillo, who is 38,
has wavy black hair, an olive-cast complexion and dark-lashed eyes. He
was quick to cry, especially when he talked about his life before. Born
in Berkeley to young parents who couldn't provide for him, Murillo stole
food and clothes in elementary school to get by. Bullied at school and
beaten at home, he began acting out. ``It felt better to be viewed as an
aggressive individual who was tough, as opposed to a kid who had shame
for getting hit at home, for not getting fed, for living in a place that
he wasn't wanted. It was much easier for me to hang around with people
who accepted me for being violent.''

Murillo has been locked up since age 16, more than half of his life, and
is serving a life sentence for murder in the second degree. A landmark
California law, passed in 2013, allowing juvenile offenders with life
sentences the chance to earn parole, gave Murillo hope that he might get
out of prison one day. He had a parole hearing a few years ago but
received a seven-year denial, meaning that for the time being, his life
sentence stands.

Each of the workers has his own style of caregiving, but if there is one
trait that stands out about Murillo, it is the tenderness with which he
handles the patients. When Jimmy Figueroa needed a shower, Murillo stood
in the stall with him to make sure he didn't fall, fidgeted with the
water temperature until it was just right and gently helped towel him
off. A few days later, when Ralph Martinez's health took a sudden turn
for the worse and he began sobbing on his bed, it was Murillo who sat
down next to him and put an arm around his shoulders. ``I'm just
returning something I didn't get as a kid,'' Murillo told me, rocking
back and forth in his chair, punching his hands together. ``All I wanted
was kindness and to be held as a boy. Now I get to do that for somebody
else. There's also the regret of not being able to do that for my
victims, for the people in my community who I hurt.''

It took Lyman a while to tell me why he was there. He was shy in a way
that could easily be misread as disinterest. Eyes superglued to the
ground, slender shoulders tense and hunched, he responded to my
questions in as few words as possible. When I asked why he decided to
work in the hospice, he told me he didn't like his previous job in the
prison kitchen. Then, in a voice quiet enough that I was forced to crane
my neck to hear him, he mumbled, ``Also, I had cancer before, so I
wanted to work with people who went through something I went through.''

``Me, too,'' I said. ``Leukemia.''

Lyman's eyes widened in surprise. ``Lymphoma, non-Hodgkin,'' he shot
back. His shoulders relaxed, and for the first time since we met, he
lifted his head and looked me in the eyes.

Lyman grew up in the projects near Compton with a teenage mother and a
father who was sentenced to life in prison when he was 5. ``I was the
kid my mom learned how to raise kids with,'' he said. ``None of my other
siblings got in trouble, just me. I'm the black sheep. I'm my daddy's
son.'' Lyman didn't have what it took to become a rapper, his first
career plan. By the time he was 13, he'd become a member of the Compton
Crips. He loved everything about gang life --- the chaos and girls, the
booze and drugs, the tempting of fate each day on the streets --- until
one afternoon, two weeks after his 18th birthday, he was involved in a
shootout with a rival gang in broad daylight. Lyman missed, his gun
jamming, but it was too late. A fleet of police cars swerved into view
before he could run, the wail of sirens slicing through the residential
block. ``That's when I knew it was down, down, baby.'' He got 19 years
for attempted murder.

In Lyman's first few weeks at the Los Angeles County Jail, he noticed a
lump on his right elbow. He didn't think anything of it at first. At 18,
his body was as strong and fast as ever. But over the course of the next
few months, the lump grew to the size of a hard-boiled egg, his already
slim frame withered and he began to lose sensation in one hand. After
the diagnosis, Lyman was escorted each day by officers to a nearby
medical center, where he received radiation treatments while handcuffed
to a hospital bed. The hair on his arm singed off, his skin charred to a
crisp and he grew so weak that he needed a wheelchair. Lyman didn't
expect to live --- wasn't sure he even wanted to --- but after dozens of
treatments, the cancer went into remission.

It was strange for Lyman to know that without his arrest and the access
to medical care that had come with it, he probably wouldn't be alive.
And once he was well, he didn't know what to make of it all --- or what
to make of himself. ``I was lost. I was ignorant to who I could be.'' He
married a girl he knew from Compton in a no-fuss ceremony in the
visiting yard of the prison and became a father to her two young
children. But he was still trying to figure it out.

``Before, I was numb. Death didn't hold the weight that it should have
held.'' Now, through his work, he had the opportunity to treat death
with respect. ``Most people are scared to death of death, but I get the
chance to be with people --- to impart what I have, and they get to
impart what they have,'' Lyman said. ``It gives me a chance to live.''

It was Lyman whom Lamerrill Dawson asked for when his condition began to
worsen. When Dawson became delusional, babbling to imaginary gods, the
doctors grew concerned enough to put him on vigil status. Lyman and the
other workers took turns sitting by his bedside until Dawson suddenly
regained consciousness. ``How come I'm on vigil? I want to get up in my
wheelchair so that I can breathe!'' Dawson threw such a fit that the
staff agreed to take him off vigil, and the workers got up to leave.

The next day, when Lyman went to work, he walked straight to the cell in
the rear right corner. Dawson lay in bed with eyes wide open. Lyman
braced himself for the usual morning's invective but instead found only
silence. He took a step closer. Dawson looked strange, he thought to
himself --- skin waxen, lips bluish, pupils vacant. Then Lyman noticed
the body bag left behind by the nurses. While delivering breakfast trays
around the unit, they had been the first to discover his cooling body.

Lyman was upset that Dawson died alone, and even more upset with himself
for not pushing harder to stay by his side. But after months of
explosive outbursts, everyone had learned it was easier to do as Dawson
said. Now, without Dawson there to tell him what to do, Lyman stood,
surveying the quiet room. He called Murillo, and they got to work,
helping to wash Dawson's body with a sponge and preparing it for the
morgue. When it came time to tie a toe tag onto Dawson's foot, Lyman
felt his throat narrow and tighten. As difficult as Dawson was, he had
grown on him over the months. It was moments like those when Lyman was
reminded how flimsy the divide between the healthy and the sick really
was. None of the workers knew when they might end up here in the hospice
--- not as worker, but as patient.

The last dusky rays of sun sifted through the shuttered windows of the
hospice. Lyman sat in a recliner, taking a break, before starting his
second shift of the day. He picked up the silver receiver of a pay phone
and pressed it close to his cheek. His son, only 3 at the time, was on
the other side.

``What you doing?''

``I'm at work,'' Lyman told him.

``You not at work --- you in prison,'' his son replied with innocent,
blunt honesty. Lyman's face crumpled. He told him he loved him, promised
to call back soon and placed the phone back in its cradle.

Within 24 hours, another patient moved into Dawson's room, and Lyman
helped him get settled. The patient taped pictures to the wall --- a
poster of the ocean in Hawaii where he surfed as a kid, the silhouette
of a pinup model torn from a magazine and a photograph of his mother
hugging her service dog, a pit bull named Cuddles. Lyman started to
clean the room with sanitizing wipes. Before he could finish, he was
called next door where another patient had been put on vigil.

\hypertarget{correction-may-28-2018}{%
\subparagraph{\texorpdfstring{\textbf{Correction} May 28,
2018}{Correction May 28, 2018}}\label{correction-may-28-2018}}

An article on May 20 about California's first prison hospice ward
incorrectly stated the age of a patient at the time the reporter visited
the facility. Ernest Marin Jr. was 79, not 80.

\textbf{Suleika Jaouad} wrote the Life, Interrupted column for the New
York Times Well section and is the author of the forthcoming memoir
``Between Two Kingdoms.''

\hypertarget{more-in-this-issue}{%
\subsection{More in this issue}\label{more-in-this-issue}}

\begin{itemize}
\tightlist
\item
  \href{https://www.nytimes3xbfgragh.onion/interactive/2018/05/15/magazine/health-issue-my-adventures-with-hallucinogenic-drugs-medicine.html?action=click\&module=MagazineModule\&pgtype=Article\&contentCollection=Magazine\&region=Footer}{}
\item
  \href{https://www.nytimes3xbfgragh.onion/interactive/2018/05/16/magazine/health-issue-what-we-lose-with-data-driven-medicine.html?action=click\&module=MagazineModule\&pgtype=Article\&contentCollection=Magazine\&region=Footer}{}
\item
  \href{https://www.nytimes3xbfgragh.onion/interactive/2018/05/16/magazine/health-issue-reinvention-of-primary-care-delivery.html?action=click\&module=MagazineModule\&pgtype=Article\&contentCollection=Magazine\&region=Footer}{}
\end{itemize}

TINGLEY

\hypertarget{more-on-nytimescom}{%
\subsection{More on NYTimes.com}\label{more-on-nytimescom}}

Advertisement

\hypertarget{site-information-navigation}{%
\subsection{Site Information
Navigation}\label{site-information-navigation}}

\begin{itemize}
\tightlist
\item
  \href{https://help.nytimes3xbfgragh.onion/hc/en-us/articles/115014792127-Copyright-notice}{©
  2020 The New York Times Company}
\item
  \href{https://www.nytimes3xbfgragh.onion}{Home}
\item
  \href{https://www.nytimes3xbfgragh.onion/search/}{Search}
\item
  Accessibility concerns? Email us at
  \href{mailto:accessibility@NYTimes.com}{\nolinkurl{accessibility@NYTimes.com}}.
  We would love to hear from you.
\item
  \href{https://help.nytimes3xbfgragh.onion/hc/en-us/articles/115015385887-Contact-Us}{Contact
  Us}
\item
  \href{https://www.nytco.com/careers/}{Work with us}
\item
  \href{https://nytmediakit.com/}{Advertise}
\item
  \href{https://help.nytimes3xbfgragh.onion/hc/en-us/articles/115014892108-Privacy-policy\#pp}{Your
  Ad Choices}
\item
  \href{https://help.nytimes3xbfgragh.onion/hc/en-us/articles/115014892108-Privacy-policy}{Privacy}
\item
  \href{https://help.nytimes3xbfgragh.onion/hc/en-us/articles/115014893428-Terms-of-service}{Terms
  of Service}
\item
  \href{https://help.nytimes3xbfgragh.onion/hc/en-us/articles/115014893968-Terms-of-sale}{Terms
  of Sale}
\end{itemize}

\hypertarget{site-information-navigation-1}{%
\subsection{Site Information
Navigation}\label{site-information-navigation-1}}

\begin{itemize}
\tightlist
\item
  \href{https://spiderbites.nytimes3xbfgragh.onion}{Site Map}
\item
  \href{https://help.nytimes3xbfgragh.onion/hc/en-us}{Help}
\item
  \href{https://help.nytimes3xbfgragh.onion/hc/en-us/articles/115015385887-Contact-Us?redir=myacc}{Site
  Feedback}
\item
  \href{https://www.nytimes3xbfgragh.onion/subscription?campaignId=37WXW}{Subscriptions}
\end{itemize}
