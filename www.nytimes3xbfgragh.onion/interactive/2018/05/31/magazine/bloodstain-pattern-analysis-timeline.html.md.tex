 **NYTimes.com no longer supports Internet Explorer 9 or earlier. Please
upgrade your browser.
\href{http://www.nytimes3xbfgragh.onion/content/help/site/ie9-support.html}{LEARN
MORE »}

**Sections

**Home

**Search

\hypertarget{the-new-york-times}{%
\subsection{\texorpdfstring{\href{http://www.nytimes3xbfgragh.onion/}{The
New York Times}}{The New York Times}}\label{the-new-york-times}}

\hypertarget{-magazine-}{%
\subsubsection{\texorpdfstring{
\href{https://www.nytimes3xbfgragh.onion/section/magazine}{Magazine}
}{ Magazine }}\label{-magazine-}}

 \href{https://www.nytimes3xbfgragh.onion/section/magazine}{Magazine}
\textbar{}How an Unproven Forensic Science Became a Courtroom Staple

**Close search

\hypertarget{site-search-navigation}{%
\subsection{Site Search Navigation}\label{site-search-navigation}}

Search NYTimes.com

**Clear this text input

Go

\url{https://nyti.ms/2LIqUJQ}

\hypertarget{site-navigation}{%
\subsection{Site Navigation}\label{site-navigation}}

\hypertarget{site-mobile-navigation}{%
\subsection{Site Mobile Navigation}\label{site-mobile-navigation}}

\hypertarget{how-an-unproven-forensic-science-became-a-courtroom-staple}{%
\section{How an Unproven Forensic Science Became a Courtroom
Staple}\label{how-an-unproven-forensic-science-became-a-courtroom-staple}}

A timeline of a niche, unproven discipline that gained a hold in the
American justice system.

\hypertarget{how-an-unproven-forensic-science-became-a-courtroom-staple-1}{%
\section{How an Unproven Forensic Science Became a Courtroom
Staple}\label{how-an-unproven-forensic-science-became-a-courtroom-staple-1}}

A timeline of a niche, unproven discipline that gained a hold in the
American justice system.

By LEORA SMITH, PROPUBLICA MAY 31, 2018

\textbf{Testimony from bloodstain-pattern} analysts is now accepted in
courts throughout the country. But in recent years, some scientists and
legal scholars have questioned the training of these experts, as well as
the validity of the field itself. How did a niche, unproven discipline
gain a hold in the American justice system and proliferate state by
state?

The modern era of bloodstain-pattern analysis began when a small group
of scientists and forensic investigators started testifying in cases, as
experts in a new technique. Some of them went on to train hundreds of
police officers, investigators and crime-lab technicians --- many of
whom began to testify as well. When defendants appealed the legitimacy
of the experts' testimony, the cases made their way to state appeals
courts. Once one court ruled such testimony admissible, other states'
courts followed suit, often citing their predecessors' decisions. When
discussing the reliability or accuracy of the technique, judges
typically relied on their own --- or the testifying expert's own ---
assessment. Rarely, if ever, have courts required objective proof of
bloodstain-pattern analysis' accuracy.

\hypertarget{1954}{%
\paragraph{1954}\label{1954}}

Sam Sheppard, an Ohio doctor, is convicted of murdering his wife in a
case that attracts widespread attention. Paul Leland Kirk, a renowned
scientist and criminalist who worked on the Manhattan Project, studies
the bloodstains in the Sheppard home and, the following year, offers an
interpretation of events that the defense believes exonerates Sheppard.

\hypertarget{1957}{%
\paragraph{1957}\label{1957}}

\textbf{People v. Carter}

The Supreme Court of California affirms that bloodstain-pattern analysis
is a proper area for expert testimony and that Kirk is a qualified
expert in the field. The case is possibly the earliest instance of an
appellate court explicitly accepting bloodstain-pattern analysis as an
appropriate field of expertise.

\emph{• Sets precedent in California.}

\hypertarget{1966}{%
\paragraph{1966}\label{1966}}

Sam Sheppard is retried, and Kirk's findings play a central role in
Sheppard's defense. Sheppard is found not guilty.

\textbf{Pedersen v. State}

The Supreme Court of Alaska holds that bloodstain-pattern analysis is an
acceptable area for expert testimony. An Alaska State Police officer
uses blood spatters on a crab-fishing ship to determine where the victim
was standing when shot.

\emph{• Sets precedent in Alaska.}

\hypertarget{1971}{%
\paragraph{1971}\label{1971}}

The United States Department of Justice publishes a report, ``Flight
Characteristics and Stain Patterns of Human Blood,'' by Herbert Leon
MacDonell, an instructor at a two-year college in New York with a
master's degree. It becomes a foundational text in the field of
bloodstain-pattern analysis.

\hypertarget{1973}{%
\paragraph{1973}\label{1973}}

MacDonell teaches his first Bloodstain Institute, a weeklong workshop,
in Jackson, Miss., training police officers as bloodstain-pattern
analysts. Months later, he teaches a second institute in Elmira, N.Y.
Over the next few decades, MacDonell will train more than 1,000 new
analysts.

\hypertarget{1979}{%
\paragraph{1979}\label{1979}}

\textbf{Compton v. Commonwealth}

The Supreme Court of Virginia rules that bloodstain-pattern analysis is
a proper area for expert testimony. A Danville police officer testifies
that bloodless circles on the floor helped determine that the victim was
probably sitting at a table when shot.

\emph{• Sets precedent in Virginia.}

\hypertarget{1980}{%
\paragraph{1980}\label{1980}}

\textbf{People v. Erickson}

An Illinois appellate court upholds a man's conviction for murdering his
wife. MacDonell testifies that blood patterns on the defendant's clothes
suggest that his wife's blood spattered on him while he attacked her. On
appeal, the court doesn't rule on the issue of admissibility of experts
in bloodstain-pattern analysis, because the defendant did not
specifically appeal MacDonell's admission as an expert.

\emph{• The case does not set bloodstain-pattern-analysis precedent in
Illinois, but it is later cited by courts in Michigan and Texas to
support the admission of experts in the field.}

\textbf{State v. Hall}

The Supreme Court of Iowa affirms the admission of bloodstain-pattern
evidence and the admission of MacDonell as an expert. The court refers
to MacDonell's field as ``relatively uncomplicated'' and, as a result,
does not require extensive proof of its reliability. The judges write,
``The evidence offered to show the reliability of the bloodstain
analysis included: (1) Professor MacDonell's considerable experience and
his status as the leading expert in the field; (2) the existence of
national training programs; (3) the existence of national and state
organizations for experts in the field; (4) the offering of courses on
the subject in several major schools; (5) use by police departments
throughout the country in their day-to-day operations; (6) the holding
of annual seminars; and (7) the existence of specialized publications.''
The court does not acknowledge that MacDonell himself is the source of
almost all these indicators of reliability.

\emph{• Sets precedent in Iowa.}

\hypertarget{1981}{%
\paragraph{1981}\label{1981}}

\textbf{State v. Hilton}

The Supreme Judicial Court of Maine discusses the testimony of an expert
in bloodstain-pattern analysis but does not actually rule on the issue
of admissibility of such testimony.

\emph{• The case does not set precedent in Maine, but it is later cited
by courts in Idaho and Texas to support the admission of experts in
bloodstain-pattern analysis.}

\hypertarget{1982}{%
\paragraph{1982}\label{1982}}

\textbf{State v. Melson}

The Supreme Court of Tennessee affirms a man's murder conviction and
death sentence and finds that bloodstain-pattern-analysis testimony of
MacDonell was properly admitted.

\emph{• Sets precedent in Tennessee.}

\hypertarget{1983}{%
\paragraph{1983}\label{1983}}

MacDonell holds his first Advanced Bloodstain Institute. Twenty-two
graduates of the advanced class go on to form the International
Association of Bloodstain Pattern Analysts, and Tom Bevel, an Oklahoma
police officer and former student of MacDonell's, becomes its first
president.

\textbf{Farris v. State}

The Court of Criminal Appeals of Oklahoma rules that bloodstain-pattern
analysis is a proper area for expert testimony. In a murder case under
review, Bevel had testified that the bloodstains showed that the
defendant struck the victim multiple times after shooting him and that
the victim tried to defend himself.

\emph{• Sets precedent in Oklahoma, citing cases in Alaska and
California.}

\hypertarget{1984}{%
\paragraph{1984}\label{1984}}

\textbf{People v. Knox}

An appellate court in Illinois accepts bloodstain-pattern analysis as a
proper area of expertise and holds that a police officer who takes a
three-week course with MacDonell is qualified to testify as an expert.

\emph{• Sets precedent in Illinois.}

\hypertarget{1985}{%
\paragraph{1985}\label{1985}}

\textbf{Jordan v. State}

The Supreme Court of Mississippi holds that even if a court erred in
admitting a bloodstain-pattern analyst, the testimony was not
influential enough to merit a retrial. The expert in the case attended a
one-week institute with MacDonell in 1973, the first year the course was
offered.

\hypertarget{1987}{%
\paragraph{1987}\label{1987}}

\textbf{Lewis v. State}

An appellate court in Texas affirms the reliability of
bloodstain-pattern analysis and MacDonell's qualification to testify as
an expert witness.

\emph{• Sets precedent in Texas, citing cases in California, Illinois,
Maine and Tennessee.}

\textbf{Fox v. State}

The Supreme Court of Indiana upholds a conviction and finds that the
expert in bloodstain-pattern analysis who testified in a murder case is
qualified because he met Indiana's requirements for expert witnesses:
His knowledge exceeded that of an ``average layperson,'' and he had
sufficient knowledge and skill to aid the judge and jury at trial. The
detective who testified had attended one course on bloodstain-pattern
analysis in 1980 and had never testified about such evidence before.

\emph{• Sets precedent in Indiana.}

\hypertarget{1990}{%
\paragraph{1990}\label{1990}}

\textbf{State v. Moore}

The Supreme Court of Minnesota rules that bloodstain-pattern analysis is
a proper area for expert testimony and that a serology expert who has
never taken a course on bloodstain-pattern analysis is qualified enough
to testify as an expert.

\emph{• Sets precedent in Minnesota, citing cases in Illinois, Iowa and
Texas.}

\hypertarget{1991}{%
\paragraph{1991}\label{1991}}

\textbf{State v. Rodgers}

The Supreme Court of Idaho decides that bloodstain-pattern analysis is
an appropriate area for expert testimony. The two experts in the case
had trained with MacDonell and Bevel. The court affirms the men's
admission as experts, saying they each meet Idaho's legal standard,
which requires only that experts be more knowledgeable on a topic than
an average juror.

\emph{• Sets precedent in Idaho, citing cases in Maine, Oklahoma,
Tennessee and Texas.}

\hypertarget{1995}{%
\paragraph{1995}\label{1995}}

\textbf{State v. Goode}

The Supreme Court of North Carolina rules that bloodstain-pattern
analysis is admissible as expert testimony and upholds the death
sentence for a defendant convicted of two counts of first-degree murder.
The expert, Duane Deaver, worked in forensics for the State Bureau of
Investigation and had studied bloodstain-pattern analysis with former
students of MacDonell and Kirk. Deaver testified that the defendant, who
appeared to have no blood on his clothes, actually had minuscule blood
spots on his boots.

\emph{• Sets precedent in North Carolina, citing cases in Idaho,
Indiana, Iowa, Oklahoma, Tennessee and Virginia. People v. Haywood}

\textbf{People v. Haywood}

The Michigan Court of Appeals affirms the admission of
bloodstain-pattern-analysis testimony and holds, among other reasons,
that because bloodstain-pattern analysis is not a novel technique,
extensive proof of its reliability is not required.

\emph{• Sets precedent in Michigan, citing cases in California and
Texas.}

\hypertarget{2001}{%
\paragraph{2001}\label{2001}}

\textbf{State v. Halake}

The Tennessee Court of Criminal Appeals overturns a conviction of
first-degree murder, finding that a police officer who had never
attended a full course in bloodstain-pattern analysis was improperly
admitted as an expert at trial. The court contrasts the officer's
credentials with experts admitted by other courts, including MacDonell.

\hypertarget{2004}{%
\paragraph{2004}\label{2004}}

\textbf{Holmes v. State}

The Texas Court of Appeals holds that a detective who attended one
weeklong bloodstain-pattern-analysis institute is sufficiently qualified
to testify as an expert witness and upholds Texas' longstanding
acceptance of bloodstain-pattern analysis as a reliable technique fit
for expert testimony.

\emph{• Sets precedent in Texas, citing cases from 15 other states.}

\hypertarget{2009}{%
\paragraph{2009}\label{2009}}

A groundbreaking National Academy of Sciences study finds serious
deficiencies in the field of forensic science in the United States and
notes that ``the uncertainties associated with bloodstain-pattern
analysis are enormous.'' The authors note that ``in general, the
opinions of bloodstain-pattern analysts are more subjective than
scientific.''

A federal court finds that Duane Deaver, who testified as a bloodstain
analyst in the 1995 North Carolina case State v. Goode, had performed
inadequate testing. An audit of the state forensics lab where he worked
will find the next year that Deaver provided misleading information on
his reports for years. By the time these findings are public, judges in
Tennessee and Texas will have already have cited State v. Goode in
deciding to admit bloodstain-pattern analysis as a reliable field.

\hypertarget{2018}{%
\paragraph{2018}\label{2018}}

In January, the Texas Forensic Science Commission holds a one-day
hearing, prompted in part by its investigation into the role that a
minimally trained bloodstain-pattern analyst played in the convictions
of Joe Bryan. In February, the commission decides to create an
accreditation requirement for bloodstain-pattern-analysis experts in
Texas.

\hypertarget{related-coverage}{%
\subsection{Related Coverage}\label{related-coverage}}

\begin{itemize}
\tightlist
\item
  \href{https://www.nytimes3xbfgragh.onion/interactive/2018/05/23/magazine/joe-bryan-blood-forensics-murder.html}{}
\item
  \href{https://www.nytimes3xbfgragh.onion/interactive/2018/05/31/magazine/joe-bryan-part-2-blood-spatter-analysis-faulty-evidence.html}{}
\item
  \href{https://www.nytimes3xbfgragh.onion/interactive/2018/05/31/magazine/bloodstain-pattern-analysis-timeline.html}{}
\end{itemize}

2018

\hypertarget{more-on-nytimescom}{%
\subsection{More on NYTimes.com}\label{more-on-nytimescom}}

Advertisement

\hypertarget{site-information-navigation}{%
\subsection{Site Information
Navigation}\label{site-information-navigation}}

\begin{itemize}
\tightlist
\item
  \href{https://help.nytimes3xbfgragh.onion/hc/en-us/articles/115014792127-Copyright-notice}{©
  2020 The New York Times Company}
\item
  \href{https://www.nytimes3xbfgragh.onion}{Home}
\item
  \href{https://www.nytimes3xbfgragh.onion/search/}{Search}
\item
  Accessibility concerns? Email us at
  \href{mailto:accessibility@NYTimes.com}{\nolinkurl{accessibility@NYTimes.com}}.
  We would love to hear from you.
\item
  \href{https://help.nytimes3xbfgragh.onion/hc/en-us/articles/115015385887-Contact-Us}{Contact
  Us}
\item
  \href{https://www.nytco.com/careers/}{Work with us}
\item
  \href{https://nytmediakit.com/}{Advertise}
\item
  \href{https://help.nytimes3xbfgragh.onion/hc/en-us/articles/115014892108-Privacy-policy\#pp}{Your
  Ad Choices}
\item
  \href{https://help.nytimes3xbfgragh.onion/hc/en-us/articles/115014892108-Privacy-policy}{Privacy}
\item
  \href{https://help.nytimes3xbfgragh.onion/hc/en-us/articles/115014893428-Terms-of-service}{Terms
  of Service}
\item
  \href{https://help.nytimes3xbfgragh.onion/hc/en-us/articles/115014893968-Terms-of-sale}{Terms
  of Sale}
\end{itemize}

\hypertarget{site-information-navigation-1}{%
\subsection{Site Information
Navigation}\label{site-information-navigation-1}}

\begin{itemize}
\tightlist
\item
  \href{https://spiderbites.nytimes3xbfgragh.onion}{Site Map}
\item
  \href{https://help.nytimes3xbfgragh.onion/hc/en-us}{Help}
\item
  \href{https://help.nytimes3xbfgragh.onion/hc/en-us/articles/115015385887-Contact-Us?redir=myacc}{Site
  Feedback}
\item
  \href{https://www.nytimes3xbfgragh.onion/subscription?campaignId=37WXW}{Subscriptions}
\end{itemize}
