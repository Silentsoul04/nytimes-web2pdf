 **NYTimes.com no longer supports Internet Explorer 9 or earlier. Please
upgrade your browser.
\href{http://www.nytimes3xbfgragh.onion/content/help/site/ie9-support.html}{LEARN
MORE »}

**Sections

**Home

**Search

\hypertarget{the-new-york-times}{%
\subsection{\texorpdfstring{\href{http://www.nytimes3xbfgragh.onion/}{The
New York Times}}{The New York Times}}\label{the-new-york-times}}

\hypertarget{-magazine-}{%
\subsubsection{\texorpdfstring{
\href{https://www.nytimes3xbfgragh.onion/section/magazine}{Magazine}
}{ Magazine }}\label{-magazine-}}

 \href{https://www.nytimes3xbfgragh.onion/section/magazine}{Magazine}
\textbar{}Lady Gaga Isn't Done Shape-Shifting Yet

**Close search

\hypertarget{site-search-navigation}{%
\subsection{Site Search Navigation}\label{site-search-navigation}}

Search NYTimes.com

**Clear this text input

Go

\url{https://nyti.ms/2P4POVb}

\hypertarget{site-navigation}{%
\subsection{Site Navigation}\label{site-navigation}}

\hypertarget{site-mobile-navigation}{%
\subsection{Site Mobile Navigation}\label{site-mobile-navigation}}

\hypertarget{lady-gaga-isnt-done-shape-shifting-yet}{%
\section{Lady Gaga Isn't Done Shape-Shifting
Yet}\label{lady-gaga-isnt-done-shape-shifting-yet}}

Lady Gaga wants to wear every costume, live out every type of known
stardom. ``A Star Is Born'' is just her latest reinvention.

\hypertarget{the-shape--shifter}{%
\section{The Shape- Shifter}\label{the-shape--shifter}}

Lady Gaga wants to wear every costume, live out every type of known
stardom. ``A Star Is Born'' is just her latest reinvention.

By \href{https://www.nytimes3xbfgragh.onion/by/rachel-syme}{RACHEL SYME}
OCT. 3, 2018

Lady Gaga did not so much arrive at the Venice Film Festival this August
as she floated into it, a platinum Aphrodite borne on the waves, black
stilettos skimming the sea foam. Which is to say, she took a water taxi.

An image of her zooming across the canal --- perched precariously on the
side of the lacquered motorboat in a little black dress, her legs
elegantly entwined, her hair shaped into three victory rolls like a
crown of croissants, holding a single red rose in one hand and blowing
kisses with the other --- immediately became a meme. Of course she
couldn't just \emph{walk} up to the premiere of ``A Star Is Born,'' the
first feature film in which she has a leading role, playing the titular
supernova. Walking is for rubes. Sailing, on the other hand, is
timeless. It is an activity for sirens, of both the mythological and
screen persuasions. It is also joyfully, unapologetically hammy: high
camp on the high seas, a playful pastiche of all the celebrity cruisers
who came before. In mere hours, several internet sleuths began to post
pictures of Gaga on the boat along with photos of classic Hollywood
stars, including Marilyn Monroe in a one-piece black bathing suit. The
next day, Gaga and Bradley Cooper, her director and co-star, arrived
hand in hand to a screening; she was wearing a swingy white dress, the
kind made for walking over subway grates. The wink was complete.

We could have seen this coming. Lady Gaga is our pop laureate of the
grand entrance, our patron saint of operatic ingress. She has never, in
a decade of global fame, been content to simply appear in a room; she
has to plummet into it, shimmying down a cable like a diamond-encrusted
spider. Or she hobbles in, a fembot on fake crutches, a high-fashion
Tiny Tim. Before performing at the 2011 Grammys, she claimed to have
slept in an oversize translucent egg for 72 hours, so that when she
finally emerged, she could feel that she had experienced total
``creative, embryonic incubation.'' For the first decade of her career,
she was often at least seminude when descending every staircase. In her
younger, more tenderized years, she trotted into the MTV Video Music
Awards in a now-infamous gown and snow boots made of raw beef, not just
a sight gag but a full-on olfactory happening, abattoir fabulous. Gaga
once described herself as ``a show with no intermission,'' but it might
be more accurate to view her career as a glorious series of overtures;
her curtain is always rising. This is why her water ride in Venice
elicited such collective delight in the form of vigorous retweeting. She
may now be a serious actress, but she hasn't lost her sense of play.

When I met Lady Gaga on a hazy afternoon a few days after her Venice
tour, at her house so high up in the Hollywood Hills that I broke
through the fog line before I reached it, she was still in full Marilyn
mode. Her duckling-blond hair was molded into a halo around her face.
Her lips were matte red, slightly overdrawn, an enthusiastic valentine.
She was wearing the same towering patent-leather stilettos from the boat
and a brown tiger-print wiggle dress, a midcentury silhouette favored by
celluloid bombshells that vacuum-seals the calves into place. Her
earrings, obsidian chandelier dangles heavy as hood ornaments, cast
prismatic shadows on her clavicle and seemed to threaten the general
integrity of her otherwise regal posture.

Having seen ``A Star Is Born'' the day before, in which Gaga gives a
notably stripped down, unbleached performance, I was slightly jarred as
I watched her shuffle through her house (which also happened to have
been the house of the avant-garde rocker Frank Zappa before she bought
it from his family trust in 2016) in a full face and spike heels. In the
film, her character, Ally, starts off makeup-free, a frustrated waitress
with mud-puddle hair (Gaga's natural hue) who long ago abandoned her
songwriting dreams and has settled for crooning live covers one night a
week at a drag bar, the only woman on the bill. One night, Bradley
Cooper, as the shambling, alcoholic rock star Jackson Maine, stumbles
into the bar looking for a nightcap and instead discovers a muse --- he
is bewitched by her performance of ``La Vie en Rose'' in an Edith Piaf
costume complete with thin eyebrows fashioned from electrical tape.

\emph{\href{https://www.nytimes3xbfgragh.onion/2018/10/03/movies/a-star-is-born-review-lady-gaga-bradley-cooper.html}{{[}Read
our review of ``A Star Is Born.''{]}}}

Later that night, Jackson asks Ally why she doesn't pursue a music
career. She tells him that she tried, she really did. She just couldn't
find any industry types who could get past her face. They loved the way
she sounded, hated the way she looked. Hearing this, Jackson reaches out
with a single finger and traces the contours of her nose. While this is
on its own an erotic gesture, it is Ally's reaction that makes the
scene: She just breathes as he gently outlines the organ she feels worst
about. It's an arresting moment, in which she seems both receptive and
completely assured.

Now, as we toured her house, Gaga was as opaque as Ally is transparent.
She spoke carefully, in a breathy tone, as if she were in an active
séance with an old movie star whose press agent advised her to remain
enigmatic and demure. She showed me a bizarre bathroom, where she had
found a bed over the shower; she gestured delicately at her backyard,
announcing: ``Some beautiful lemon trees. It's a nice place to come and
just create.'' When we got into the studio, she tiptoed through the
cavernous live room, pointing out a grand piano in a voice so quiet I
could barely hear her. We made our way to a small alcove with
whitewashed walls and 20-foot ceilings, which looked like the storage
room of an art museum --- an echo chamber, she explained. I asked about
the acoustics, in part because it seemed the polite thing to do, but in
part because I was trying to open any conversational tap I could find.
Whether she was feeling legitimately shy or was simply method-acting as
a restrained ingénue, she had yet to speak at full volume.

Suddenly, she broke into song. A cappella, unprompted, \emph{voce
forte}, her arms flung out to full wingspan, her head tossed back to
bare her throat. She was singing the chorus of ``Shallow,'' the song she
co-wrote for ``A Star Is Born'' that has become the de facto theme song
for the movie. It is sung at the cathartic apex of the trailer (which
has been viewed almost 10 million times on YouTube) --- the moment when
Ally reluctantly steps onto an arena stage for the first time to sing
with Jackson. Gaga plays this moment with incredible restraint; it's
hard to imagine her not wanting to storm a stage, but she really sells
it. Ally has been down for so long that she hesitates, not fully
believing that this is her shot. But then something shifts. She
straightens her shoulders, struts out to the microphone and sends her
voice soaring over the crowd.

In the echo chamber, the words of the song ricocheted, shaking the room:
``I'm off the deep end! Watch as I dive in! I'll never meet the
ground!'' When Gaga sings, her whole body vibrates. She clenches her
fists, squeezes her eyes shut.

After she finished belting, Gaga looked beatific, almost giddy, having
answered my banal question with undeniable certainty. The acoustics in
here, we agreed, were very good.

The title of ``A Star Is Born'' is misleading and always has been. It
implies spontaneous generation, Athena popping fully formed out of Zeus'
forehead. In reality, it is a story about hard work, about the grueling
machinations behind celebrity. In each version of the film, fame can
destroy (by enabling addiction or worsening self-destructive behaviors),
but it can also be a sacred rite; it anoints the truly worthy with
laurels and fragrant oils, no matter how aquiline her nose. The
narrative takes a nobody and brings her together with a fading legend.
He falls in love with her and her artistic potential, and thrusts her
straight into the crucible of mass popularity. It is a love story as
unshakably perennial as ``Romeo and Juliet,'' except slightly less
crushing, because only the man is doomed and the leading lady gets to
walk away from her tragedy triumphant, her suffering noble, her name in
neon lights.

``A Star Is Born'' has never really been a film about an unknown actress
shooting across the screen like a rare comet. Instead, from the very
beginning, it has always been a film about an already superfamous woman
shooting a movie. That's the real reason the franchise works: It comes
with a built-in insurance policy. In 1937, when Janet Gaynor stepped
into the role of the farm girl Esther Blodgett in the first version
(which was itself a remix of a 1932 drama called ``What Price
Hollywood?''), she was making a comeback, but she had been a box-office
titan of the silent era, the first woman to ever win an Academy Award
for acting. Judy Garland, who tackled Esther in 1954 (a studio executive
quickly changes her name to Vicki Lester in the film), was a household
name at 17, no longer a vaudevillian striver but a minted studio girl,
kept on a steady infusion of amphetamines and barbiturates and praise.
In 1976, Barbra Streisand, whose character's name was Esther Hoffman (we
have to believe she goes from \emph{mieskeit} to swan), was already an
Oscar winner for playing Fanny Brice, and fresh off another nomination,
for ``The Way We Were.'' These actresses were all at least a decade into
their careers, and they used the material less as a coming-out party and
more as a victory lap. Of course the Esthers would succeed; their
real-life counterparts had already pushed through every obstacle.

This is why the lead role is so alluring to divas who want to explore
the boundaries of their fame and what they had to endure to lasso it.
These actresses, in drag as younger versions of themselves, get to
wrestle with their flaws and air out their darkest fears. But we don't
fear for them, not really, because we know how the story turns out.
Garland, who always felt so intimidated by the leggy army of MGM blondes
that she spent her life making self-deprecating jokes, fashioned herself
into the world's most beloved brunette. Streisand, whose line ``Hello,
gorgeous'' was soaking in wry irony, turned a prominent bridge into a
locus of desire.

Gaga's innate New York City toughness brings a different flavor to the
role than her predecessors. Where Janet Gaynor plays the starlet as pure
and cornfed, Garland plays her as a plucky troubadour in pert ribbon bow
ties and Streisand plays her as a wisecracking prima donna in colorful
ponchos (hey, it was the '70s), Gaga's Ally is more world-weary and
knowing. She is the kind of woman who gets into fistfights, who
alternately sasses and fusses over her father (Andrew Dice Clay), a
chauffeur who once had showbiz aspirations himself but never had a lucky
break. When Cooper offered Gaga the role, he told her that ``this is
what it would be like if you were 31 and had never made it,'' and she
readily embodies the ferocious hunger of the would-be famous. She's no
innocent when she walks onstage to sing. She knows exactly what to do,
and exactly what this will mean for her career. She's ready to go.

Ally's journey is not about a singer developing her talent --- that's
already there. It is about finding her way toward an aesthetic once she
has the world's attention. She dyes her hair Tang orange, begins working
with a choreographer and sings springy pop songs about butts, all of
which she does without wavering, even when Jackson drunkenly criticizes
her for being inauthentic. Some viewers may read a rock-versus-pop
hierarchy into Ally's transformations --- that she is more ``real'' when
she is harmonizing with Jackson's twangy melodies or sitting at her
piano --- but Gaga's onscreen mastery over both genres is a pre-emptive
rebuttal to what is essentially a gendered bias. What ``A Star Is Born''
makes clear about Lady Gaga is that she possesses the dexterity to make
whatever kind of music she likes.

Cooper told me that he cast Gaga after seeing her perform ``La Vie en
Rose'' at a cancer benefit. The very next day, he drove to her Malibu
home to test their chemistry. They bonded right away about their
families (both East Coast, both Italian) and ate spaghetti on her porch.
``She was completely illuminated by the sun,'' he said. ``So
charismatic. I thought inside my head, Oh, gosh. If she is like this on
film, if that's the worst case scenario that she's this present on film,
the movie will work.''

It is difficult to pinpoint exactly when Lady Gaga, the international
superstar, was born. Past a certain level of fame, the origin stories of
pop artists begin to tilt into the mythological. ``I have a nerve inside
of me to do this,'' Gaga said, sitting on a swivel chair in her basement
studio, when I asked what drives her. She kept her legs crossed at the
ankles and her spine rod-straight, with her shell-pink nails gingerly
intertwined in her lap, as if she were practicing to meet Queen
Elizabeth (side note: When Gaga did meet the queen, after performing in
the Royal Variety Show in 2009, she curtsied while wearing a
floor-length, puff-sleeved dress made entirely of slick red latex).
``And I have no idea where it comes from, except that it might come from
God. No one knows.''

What she does know is that at some point, she felt free: to drop her
birth name (Stefani Joanne Angelina Germanotta), to turn herself into an
event, to keep shedding old skins.

Lady Gaga's early career was a study in this invitational freedom:
\emph{Look how free I am, look how free you could be.} This is what she
was selling, at 21, with her platinum oversize hairbows and gigantic
sunglasses and skyscraper shoulder pads. This is the realization that
led her, after growing up on the Upper West Side, attending a private
Catholic girls' school and studying piano minuets, to move downtown in
2004, first to study theater arts at N.Y.U. (she dropped out during
sophomore year) and then to sing in grungy bars on the Lower East Side
while she sent her demos to record labels. She read Andy Warhol's books
and realized that what most people want, when they dream of fame, is not
necessarily wealth or power but limitlessness: the ability to change. So
many artists start out gritty and homegrown but calcify into hardened
personae over time; when Lady Gaga adopted her new name (sometime around
2006, most likely from a Queen song), she decided to flip the formula.
What if she began with the character, and the character was the physical
embodiment of flux? What if she never wore the same outfit twice, or
gave an interview out of costume, or claimed to be a paragon of creative
authenticity?

Gaga's debut album, ``The Fame'' (quickly reissued with extra songs as
``The Fame Monster''), came out in August 2008, a season of optimism and
political overhaul, when young people were ready to accept jangly pop
hooks from a chimerical sprite who told them they could continually
redefine themselves. Her first recordings may not have been too deep ---
``Poker Face,'' still her second-biggest single to date, after ``Just
Dance,'' is an ode to mirrored surfaces, to remaining willfully
inscrutable --- but they were catchy (she changed the way an entire
generation hears the phrase ``ooh la la''), and their agile lightness
was intentional. Much of her early music was thumping and linear: big
synths, big hooks, the beats clinking together like a wristful of silver
bangles. The music was a tool for propagating her radiant image, which
was continually surprising to behold.

When Gaga first emerged onto the pop scene, she was a phenomenon --- a
kooky amalgam of New York club-kid toughness, art-school
experimentation, record-label grooming, classical vocal training and
bona fide radio hits. She clearly took her cues from previous
incarnations of major pop stardom (David Bowie's amphibious glam,
Madonna's blond ambition, Michael Jackson's dual love of sparkles and
precision), but she was even more focused than her predecessors on the
live event, on the coup de théâtre. She started pushing boundaries and
stopped wearing pants; she became a walking billboard for avant-garde
fashion (Alexander McQueen's ankle-bending hoof heels, a jacket covered
in felt Kermit the Frogs, several gowns made of human hair, that meat
dress), a fact that served to make every other artist at the time who
wasn't rolling around onstage in a pool of fake blood seem, frankly,
dull.

Gaga's initial obsession with masquerade predicted the double lives we
all live now, our simultaneous existences as living, breathing people
and disembodied avatars. But instead of seeing those identities as
segmented --- the real person, the facade --- she put forth the concept
that it's possible, and ultimately adaptive, in a fractured world to try
to free yourself from old boundaries. You can be an insider and an
outsider at the same time, a human and an alien. All that is solid melts
into Gaga. If this seems paradoxical, it is; but the paradox is where
Gaga shines. Postmodern double truths are her milieu.

She started calling herself a monster, not just to embrace a kind of
outré bizarreness that had mainly been the province of male pop icons
like Bowie or Prince, but also because she \emph{was} monstrous, a pop
creation that devoured the zeitgeist and then gleefully regurgitated it.
She mock-hanged herself with a noose onstage, she dreamed up a hat
filled with live cockroaches, she sucked on a rosary in the
``Alejandro'' video, she hired a ``vomit artist'' to spew lime-green
milk on her outfit at South by Southwest, she delivered an awards-show
speech as her male alter-ego, Jo Calderone. Her whole project was a
Technicolor dream ballet, a gauzy hallucination. And it sold records
(over 27 million, worldwide) and won awards (six Grammys).

``I do keep transforming into a new shell of me,'' she told me. ``So
sure, there is an acting component to what I do, or a showbiz component
to what I do. But the word `acting,' it's hard for me to talk about in
that way, because `acting' to me almost implies faking it.'' She
insisted to me that all her iterations form an unbroken line, that the
performance is the reality.

Gaga has, over the last decade, arguably moved the entire pop apparatus
toward forceful weirdness. Her influence is everywhere --- she opened
the doors for more female hitmakers to be cheekily bizarre (Miley Cyrus
grinding on a wrecking ball, Katy Perry with her sniper-rifle bra filled
with whipped cream, Sia living under her wig, even St. Vincent's indie
Fritz Lang affect) --- but as a result, Gaga's early maximalism began to
feel less vital to the cultural conversation. In 2011, Adele's ``21''
cemented a new austerity in pop; all she had to do to sell 11 million
records was stand in one place and sing plaintively about heartbreak.

So Gaga swerved again, and again, and again. She made a jazz record with
Tony Bennett. She made a crunchier, heavy-metallish album called
``Artpop'' that mostly failed to connect with the public, at least on
the large Gagagian scale she was used to (it sold fewer than a million
copies). When she turned 30, she released a more minimalist fifth record
called ``Joanne,'' after an aunt who died young of complications from
lupus. She promoted the album in ripped T-shirts and a plain, pink felt
hat. She toured dive bars before the arenas. She also released the
Netflix documentary ``Gaga: Five Foot Two,'' a vérité glimpse into her
daily life as she prepped for the 2017 Super Bowl, produced and promoted
``Joanne'' and spoke openly about the debilitating pain caused by her
fibromyalgia (something she had been dealing with privately for years).
The documentary presents Gaga with a striking lack of vanity. She
appears on camera with dirty hair and a bare face. This is Gaga the
Vulnerable, Gaga the Sensitive Soul.

That film ends with her performance at the Super Bowl, where she sang
all the karaoke staples of her back catalog --- ``Bad Romance,''
``Telephone,'' even ``Just Dance'' --- with gusto in a sequined
bodysuit, thrusting through the jangly disco beats of ``Born This Way''
in high-heel boots, surrounded by an army of dancers in iridescent
capes. It was a blistering set, a Greatest Hits Cardio Workout and a
truly impressive display of her cultural dominance. But it also felt
elegiac, as if it belonged to a different era, when Gaga was giving
stump speeches about overturning ``Don't Ask, Don't Tell'' and the song
became an anthem of the fight for gay marriage on a national scale. In
recent years, queer culture has become more anti-institutional, less
about normalizing and more about resisting norms. In a way, Gaga's
galactic fame, which once gave her such a huge platform as an advocate
for equality, became a liability when the conversation became more
intimate and nuanced. Pop is not entirely post-spectacle (Beyoncé's
recent Coachella performance was a multiact extravaganza), but it is
evolving into a less bombastic space. It is getting more raw, smaller.
And Gaga is doing the same.

Two artists explore the idea of transformation. Video by Marilyn Minter

She has not given up on the power of an audacious live show (this
winter, she will put on a pyrotechnic Las Vegas residency called
``Enigma''), but in making ``A Star Is Born,'' she is entering into a
softer conversation with the public --- about talent, about ambition,
about her own trajectory. Ally is the most human of all of Gaga's
creations, and offering her to us --- her fear, her loyalty, her
shattered heart after tragedy --- is a different kind of gamble than
stepping out in front of millions dressed like a holographic Muppet. She
is, in essence, making exploratory autofiction on a grand scale, even as
she is playing yet another character.

Lady Gaga bought Frank Zappa's eccentric woodland estate not as a place
to live while in Los Angeles --- she already has a Mediterranean-style
villa on an isolated, craggy cliffside in Malibu for that --- but as a
work retreat, the new nerve center of her countless creative pursuits.
She wants to paint here, write music here (she told me that currently
she is feverishly writing songs on a white piano upstairs; literally on
the surface of the piano, with a black Sharpie) and plan her Vegas
spectacular from here with her production team, like a war council
plotting some dazzling siege. In her recording studio, after her
``Shallows'' serenade, Gaga played me five tracks from the coming film
soundtrack. As the music blasted, she began to loosen up --- this was
her turf, her major contribution to the film. She lip-synced to her own
songs from her swivel chair, looking straight into my eyes and drawing
me insistently into her joy.

The studio is her sanctuary, and one of the main reasons she felt she
had to have the property. She is also working to preserve as many of the
home's oddities as possible: the vintage submarine doors with thick
portholes, a giant dragon mural, the library floor painted to look like
a lily pond. She told me she loves the house's ``intricate chaos.''

Gaga is an auction shopper --- she likes to acquire iconic objects,
created by iconic personalities --- and as I toured Zappa's house it
occurred to me that we were standing in a giant collector's item of
sorts, an 8,000-square-foot twig in the magpie's nest of pop cultural
artifacts that she has been building for a decade. In 2012, she
purchased 55 items from Michael Jackson's private archive, including his
leather ``Bad'' jacket and a crystal glove. That same year, she bought
an eggshell silk Alexander McQueen gown from the collection of the
British fashion maven Daphne Guinness. In 2016, for her Dive Bar Tour in
support of ``Joanne,'' Gaga rolled up to a show in Elvis's pink 1955
Cadillac Fleetwood (she was just borrowing it).

Perhaps this collector's impulse is what she absorbed from her early
study of Warhol. Gaga is an artist of accrual, of remixing and
reimagination, of pulling her heroes into her gravitational orbit. She
once told an interviewer that her ``whole career is a tribute to David
Bowie,'' but her career is really a tribute to all the different ways a
person can be monstrously famous: She wants to wear every costume, live
out every type of stardom to its maximalist extreme.

If she was going to be a movie star, she couldn't just step into a role,
or a film, that no one had ever heard of --- she wanted to waltz into a
lineage. When she was younger, she told me, she used to watch ``The
Wizard of Oz,'' over and over, convinced that Judy Garland was the
greatest entertainer alive. ``Judy, I just think she's tremendous,'' she
told me. ``There's a vulnerability behind her eyes, the way she speaks,
she has big features. I just always wanted to be like her. It's as
simple as that.'' And now she is standing on the very same stage.

Earlier in the afternoon, she showed me a room that was empty save for a
gigantic photograph of her own face, at least 15 feet across, in a
gilded frame. ``It was a gift from Bradley,'' she said. ``It's the last
frame of the movie. Do you know the scene?''

I did know it. It is the moment when Ally is standing on the stage of
the Shrine Auditorium --- where Garland shot her final scene --- in an
ice blue evening gown, singing a homage to her late husband. She starts
out timid and drained of expression, explaining to the audience that she
is going to sing the last song that Jackson wrote for her, and that
maybe with their support, she can get through it. But as the ballad goes
on, her voice swells and becomes an avalanche. It's a bravura
performance in extreme close-up, a sort of symphonic summoning of every
woman who has played the part. Gaga channels both the way Garland sang
(wounded, tonally bright, barely holding it together) and the way
Streisand did (forceful, sweeping, with a diffident jutting of the jaw).
But Gaga adds something of her own: a sensual, earthy confidence, like
gasoline in her veins.

When she finishes, a single elephant tear rolls down her face.
Magically, the moment somehow avoids bathos --- the tear feels truly
earned. After watching her perform this scene, I felt elated by what
Gaga managed to do, not just for her character but for herself. You
desperately want to know what her future holds after the curtain falls.

I asked Gaga later what we can expect from her next phase. Of course,
there's Vegas and a new record on the way, and she's reading piles of
scripts. But she really didn't want to discuss any of that. Instead, she
just smiled enigmatically. ``Oh,'' she sighed. ``I'm just shape-shifting
again.''

Rachel Syme is a writer and cultural critic whose work has appeared in
The Times, The New Yorker and The New Republic. She last wrote for the
magazine about
\href{https://www.nytimes3xbfgragh.onion/2017/11/28/magazine/rachel-brosnahans-comic-timing.html}{the
actor Rachel Brosnahan.}

Stylists: Sandra Amador and Tom Eerebout. Hair: Frederic Aspiras.
Makeup: Sarah Tanno.

\hypertarget{the-culture-issue}{%
\subsection{The Culture Issue}\label{the-culture-issue}}

\begin{itemize}
\tightlist
\item
  \href{https://www.nytimes3xbfgragh.onion/interactive/2018/10/03/magazine/lady-gaga-movie-star-is-born.html}{}
\item
  \href{https://www.nytimes3xbfgragh.onion/interactive/2018/10/03/magazine/morality-social-justice-art-entertainment.html}{}
\item
  \href{https://www.nytimes3xbfgragh.onion/interactive/2018/10/04/magazine/good-place-michael-schur-philosophy.html}{}
\item
  \href{https://www.nytimes3xbfgragh.onion/interactive/2018/10/04/magazine/barry-jenkins-james-baldwin-if-beale-street-could-talk.html}{}
\item
  \href{https://www.nytimes3xbfgragh.onion/interactive/2018/10/05/magazine/instagram-cindy-sherman-ugly-beauty.html}{}
\item
  \href{https://www.nytimes3xbfgragh.onion/interactive/2018/10/05/magazine/culture-moments-2018.html}{}
\end{itemize}

\hypertarget{related-coverage}{%
\subsection{Related Coverage}\label{related-coverage}}

\begin{itemize}
\tightlist
\item
  \href{https://www.nytimes3xbfgragh.onion/interactive/2018/10/03/magazine/lady-gaga-movie-star-is-born.html}{}
\item
  \href{https://www.nytimes3xbfgragh.onion/interactive/2018/10/03/magazine/morality-social-justice-art-entertainment.html}{}
\item
  \href{https://www.nytimes3xbfgragh.onion/interactive/2018/10/04/magazine/good-place-michael-schur-philosophy.html}{}
\item
  \href{https://www.nytimes3xbfgragh.onion/interactive/2018/10/04/magazine/barry-jenkins-james-baldwin-if-beale-street-could-talk.html}{}
\item
  \href{https://www.nytimes3xbfgragh.onion/interactive/2018/10/05/magazine/instagram-cindy-sherman-ugly-beauty.html}{}
\item
  \href{https://www.nytimes3xbfgragh.onion/interactive/2018/10/05/magazine/culture-moments-2018.html}{}
\end{itemize}

2018

\subsection{}

\begin{itemize}
\item
  \href{https://www.nytimes3xbfgragh.onion/2018/09/27/movies/bradley-cooper-a-star-is-born.html}{}

  \includegraphics{https://static01.graylady3jvrrxbe.onion/images/2018/09/26/style/30-bradley-cooper-oak-promo/30-bradley-cooper-oak-promo-mediumThreeByTwo225.jpg}

  \hypertarget{bradley-cooper-is-not-really-into-this-profile}{%
  \subsection{Bradley Cooper Is Not Really Into This
  Profile}\label{bradley-cooper-is-not-really-into-this-profile}}

  Nov. 13, 2019
\item
  \href{https://www.nytimes3xbfgragh.onion/2018/10/03/movies/a-star-is-born-review-lady-gaga-bradley-cooper.html}{}

  \includegraphics{https://static01.graylady3jvrrxbe.onion/images/2018/10/05/arts/05astarisborn1/05astarisborn1-mediumThreeByTwo225-v3.jpg}

  \hypertarget{review-the-gorgeous-heartbreak-of-a-star-is-born}{%
  \subsection{Review: The Gorgeous Heartbreak of `A Star Is
  Born'}\label{review-the-gorgeous-heartbreak-of-a-star-is-born}}

  July 22, 2019
\end{itemize}

Advertisement

\hypertarget{site-information-navigation}{%
\subsection{Site Information
Navigation}\label{site-information-navigation}}

\begin{itemize}
\tightlist
\item
  \href{https://help.nytimes3xbfgragh.onion/hc/en-us/articles/115014792127-Copyright-notice}{©
  2020 The New York Times Company}
\item
  \href{https://www.nytimes3xbfgragh.onion}{Home}
\item
  \href{https://www.nytimes3xbfgragh.onion/search/}{Search}
\item
  Accessibility concerns? Email us at
  \href{mailto:accessibility@NYTimes.com}{\nolinkurl{accessibility@NYTimes.com}}.
  We would love to hear from you.
\item
  \href{https://help.nytimes3xbfgragh.onion/hc/en-us/articles/115015385887-Contact-Us}{Contact
  Us}
\item
  \href{https://www.nytco.com/careers/}{Work with us}
\item
  \href{https://nytmediakit.com/}{Advertise}
\item
  \href{https://help.nytimes3xbfgragh.onion/hc/en-us/articles/115014892108-Privacy-policy\#pp}{Your
  Ad Choices}
\item
  \href{https://help.nytimes3xbfgragh.onion/hc/en-us/articles/115014892108-Privacy-policy}{Privacy}
\item
  \href{https://help.nytimes3xbfgragh.onion/hc/en-us/articles/115014893428-Terms-of-service}{Terms
  of Service}
\item
  \href{https://help.nytimes3xbfgragh.onion/hc/en-us/articles/115014893968-Terms-of-sale}{Terms
  of Sale}
\end{itemize}

\hypertarget{site-information-navigation-1}{%
\subsection{Site Information
Navigation}\label{site-information-navigation-1}}

\begin{itemize}
\tightlist
\item
  \href{https://spiderbites.nytimes3xbfgragh.onion}{Site Map}
\item
  \href{https://help.nytimes3xbfgragh.onion/hc/en-us}{Help}
\item
  \href{https://help.nytimes3xbfgragh.onion/hc/en-us/articles/115015385887-Contact-Us?redir=myacc}{Site
  Feedback}
\item
  \href{https://www.nytimes3xbfgragh.onion/subscription?campaignId=37WXW}{Subscriptions}
\end{itemize}
