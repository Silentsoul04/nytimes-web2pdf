 **NYTimes.com no longer supports Internet Explorer 9 or earlier. Please
upgrade your browser.
\href{http://www.nytimes3xbfgragh.onion/content/help/site/ie9-support.html}{LEARN
MORE »}

**Sections

**Home

**Search

\hypertarget{the-new-york-times}{%
\subsection{\texorpdfstring{\href{http://www.nytimes3xbfgragh.onion/}{The
New York Times}}{The New York Times}}\label{the-new-york-times}}

\hypertarget{-magazine-}{%
\subsubsection{\texorpdfstring{
\href{https://www.nytimes3xbfgragh.onion/section/magazine}{Magazine}
}{ Magazine }}\label{-magazine-}}

 \href{https://www.nytimes3xbfgragh.onion/section/magazine}{Magazine}
\textbar{}Barry Jenkins's Films of Love, Pain and Black Male
Vulnerability

**Close search

\hypertarget{site-search-navigation}{%
\subsection{Site Search Navigation}\label{site-search-navigation}}

Search NYTimes.com

**Clear this text input

Go

\url{https://nyti.ms/2PdRPid}

\hypertarget{site-navigation}{%
\subsection{Site Navigation}\label{site-navigation}}

\hypertarget{site-mobile-navigation}{%
\subsection{Site Mobile Navigation}\label{site-mobile-navigation}}

\hypertarget{barry-jenkinss-films-of-love-pain-and-black-male-vulnerability}{%
\section{Barry Jenkins's Films of Love, Pain and Black Male
Vulnerability}\label{barry-jenkinss-films-of-love-pain-and-black-male-vulnerability}}

The ``Moonlight'' director brings his intuitive approach to an
adaptation of James Baldwin's ``If Beale Street Could Talk.''

\hypertarget{heart-doctor}{%
\section{Heart Doctor}\label{heart-doctor}}

The ``Moonlight'' director brings his intuitive approach to an
adaptation of James Baldwin's ``If Beale Street Could Talk.''

By ANGELA FLOURNOY OCT. 4, 2018

Outside, the back house looked like a charming Silver Lake vacation
rental: bright purple bougainvillea, picnic table in front. Inside, its
rustic-tiled bathroom read misplaced Mediterranean. The refrigerator
bore cans of LaCroix and not much else. The place felt tenuously
inhabited, fratty, as if a pledge named Connor might emerge from a back
room to offer everyone cheap beer. And in the bedroom nearest the
kitchen, with its lurid red curtain and indigo wall, Barry Jenkins, one
of the most thoughtful filmmakers of his generation, stood peering at a
panel of computer monitors. He had driven to this editing suite ---
north on the 101 freeway from his downtown Los Angeles apartment --- so
he could see about adding 16 seconds to his newest film.

``I want the music to hit exactly where it hit before, you know?'' He
waited for a response. ``You know what I mean, Joi?''

Joi McMillon laughed, and kept her back to Jenkins as she clicked
through the file. She is the first African-American woman to have been
nominated for the Academy Award for film editing, for her work on
Jenkins's ``Moonlight.'' They have known each other since their college
days, in the film program at Florida State University. ``Yessir,'' she
said. ``It definitely will hit where it was before.''

The room was too warm, and just as I thought this, the ceiling vent
belched out even warmer air. Male voices down the hall discussed
character motivation in some other project. McMillon made the change to
the music and played the pass for Jenkins again. The penultimate scene
of his film.

``It's almost like ... ,'' Jenkins began. ``Not like jazz, because it's
clichéd to say. But yeah, sometimes one movement needs to de-escalate
before the next movement can escalate. This is the place where I feel
it. It just sucks, because making this choice at this moment means that
when we screen in Toronto we won't have this adjustment in it.'' It was
the first Wednesday in September, and the Toronto International Film
Festival was only four days away. The film, ``If Beale Street Could
Talk,'' would have its premiere there without these 16 seconds, though
it's hard to say if anyone not named Barry Jenkins would have noticed.

This was my second meeting with Jenkins, and --- more than during the
first --- his demeanor seemed bifurcated, one foot still in the shaping
of the work, the other stepping into the role of a presenter, devising
talking points that would carry him from Toronto through the film's
limited release on Nov. 30, perhaps into the awards season beyond. ``I
learned the hard way on `Moonlight' that it's best to be at the
forefront of the conversation,'' he told me that morning. ``It's almost
like tennis, you know? You want to be leading, and not playing off the
back foot.'' Still, he was eager to show me outtakes, moments that might
have been. He spoke of the film as if it possessed an immune system, one
with a long list of irritants it couldn't abide. ``The film rejected
it,'' he said wistfully of one deleted scene.

A work of cinema, like a novel, is a collage. The finished product is a
result of hundreds of moments of cut and paste: this take moved and
placed next to another one, this voice recorded to change the feel of a
scene shot with no voice-over in mind. A shift to the arrangement or
size of one small piece can make the whole collage feel unstable,
particularly to the collagist. Each time McMillon played the pass again,
the score, composed by Nicholas Britell, swelled up and embraced us
before shrinking away on command. The scene --- on its own, compellingly
composed --- slid closer to rapturous with the addition of music. A
novelist would kill for such a tool.

``What do you notice about that sequence?'' Jenkins asked me. I'd
watched it several times at this point, but I wasn't sure what in
particular I was supposed to notice. ``O.K.,'' he said. ``Well, I'll
call it. That's me imitating Baldwin. The voice-over, that's the one bit
of voice-over not taken from the book.''

\subsection{}

Before the start of production on ``If Beale Street Could Talk,''
Jenkins received an unexpected package from Gloria Karefa-Smart, James
Baldwin's sister and the executor of his estate. It contained a brown
leatherbound notebook dated 1978. In it were Baldwin's ideas about how
to approach a film adaptation of the novel, some of which suggest that
Jenkins made choices the writer would have endorsed --- using the words
of the novel's young narrator as voice-over, excluding a certain church
scene. ``This is the cast he wanted,'' Jenkins pointed out, showing me
the book in his apartment. There's Ruby Dee, a mysteriously initialed
``D.B.,'' the singer and actress Rosalind Cash. Baldwin's list of dream
directors include the photographer and filmmaker Gordon Parks, the
French New Wave director François Truffaut and Lloyd Richards, who
staged the original production of Lorraine Hansberry's ``A Raisin in the
Sun.'' ``And then I ended up directing the damn thing,'' Jenkins said.
Baldwin's handwriting covers the first two pages of the notebook, but
the rest is blank, as if the writer started jotting down ideas before
being called away to other work.

Jenkins's apartment, with its wall-length windows, film-related art and
preponderance of books, seems an apt place to house this small part of
Baldwin's literary footprint. In his living room I spotted Marilynne
Robinson's ``Gilead'' (``I've read it at least six times''), Vladimir
Nabokov's ``Pale Fire'' and, of course, plenty of books by Baldwin
himself. On the cement floor --- under a bookshelf, behind a vintage
cream couch --- I spotted Jenkins's Oscar for best adapted screenplay.
Jenkins saw me looking and said he was working on a better spot for it.
But he did have the envelope revealing the best-picture win for
``Moonlight'' framed and hung on the wall, alongside a congratulatory
note from Warren Beatty --- one of the presenters who mistakenly
announced that ``La La Land'' had won --- telling him the honor was
``much deserved.'' After the ensuing hubbub subsided, Jenkins went to
Mérida, Mexico, for a few days to clear his mind. ``My favorite thing to
do is to sit in the window, the front window of a cafe looking out at
the sidewalk,'' he said, ``which is the worst thing to do, or at least
in the immediate aftermath of what happened.''

In addition to this new film's being the much-anticipated follow-up to
``Moonlight,'' it is also the first big-screen English-language film
adaptation of a novel by Baldwin, a writer whose works are closely
guarded by his estate. Much of the country, owing to our current
political reality and Raoul Peck's 2016 documentary, ``I Am Not Your
Negro,'' has recently become better acquainted with a truth black
readers grasped long ago: James Baldwin was right about everything.
Jenkins began his adaptation of Baldwin's fifth novel back in 2013,
writing a faithful screen version of the 1974 book, in which a pregnant
19-year-old woman named Tish works, alongside her family, to prove the
innocence of her child's jailed father, a young sculptor. This being
Baldwin, of course, there's more to it: a meditation on the radical
implications of declaring yourself an artist while black, on what it
means to be poor in New York, on the power and limitations of romantic
and filial love.

According to Jenkins, this would be the first of his feature films to
directly grapple with political themes, but I begged to differ, bringing
up the role that gentrification played in his first feature, the 2008
morning-after romance ``Medicine for Melancholy'' --- it's set in San
Francisco, and the two leads even stumble onto a round-table discussion
of rising housing prices. In ``Moonlight'' the ravages of the war on
drugs and the brutality of poverty are not at all apolitical. True,
Jenkins said, but with that film, ``the whole movie is created to almost
force the audience to confront what this character is feeling. And so
it's really easy to sit outside the film and just want to hug the film,
to hug the main character. But this is Baldwin. In Baldwin, everyone's
implicated, including himself. So I think there's not a passive path
through this film.'' It didn't appear to bother Jenkins that this might
make it harder for some ``Moonlight'' devotees to stay along for the
ride to ``Beale Street.'' There were still plenty of reasons to want to
hug this film too.

Barry Jenkins is 38, under six feet tall and solidly built, with a
curated nerdiness: translucent glasses, brightly printed oxford shirts,
slip-on sneakers, a canvas tote bag. He is willfully kind, and has a
plethora of ways to put others at ease. He might call you ``bruh'' or
``boss'' or ``my dear'' or ``boss lady,'' though when adamant he's apt
to call anyone, regardless of gender, ``man.'' When he listens to you
speak, he lets you know he's there via a steady cadence of mm-hmms,
about 10 per minute, sometimes more. He has the sort of smile that if
glimpsed across the room at a cocktail party might compel you to walk
over and stand near him. ``I will not annoy, rebuff or humiliate you,''
this smile says.

``One of the things he says to me a lot is, you know, `flies with
honey,' '' his producer, Adele Romanski, another former film-school
classmate, told me. ``I think that's a life philosophy that he extends
to the work.'' This is a useful stance if you are interested in getting
actors to open up, to bare themselves. The stereotype of the American
director-auteur may be of the demanding genius who bullies his actors
into transcendent performances, but Jenkins prefers a nurturing
approach: ``I try to find a language that the actor can participate
in,'' he says. ``There has to be a language for every single person on
set.'' These efforts translate onto the screen. But what exactly is it
that's being translated?

It begins with vulnerability. I've spent my life loving black men, and I
understand black masculinity to be malleable, its fabled rigidity
overblown. After work, in the dark, I've heard whispered secrets, the
wanderings of restless minds. And yet all of my moviegoing life I'd
never seen this quotidian vulnerability so accurately rendered in film
--- not without a wink, a glance away, some posturing that distances ---
until I saw Jenkins's ``Moonlight.'' An adult son tries to keep
emotional distance from his mother, a recovering crack-cocaine addict,
and cannot; tears stream down his face instead. A drug dealer confirms
his profession to a boy (the same son, but younger), thereby admitting
to playing a part in what holds the mother captive; the boy leaves, and
the drug dealer (a father figure, not a monster) stares straight ahead,
defeated. In ``Beale Street,'' we watch the main character, Fonny,
listen to his friend Daniel describe the psychological horror of prison.
Daniel begins the conversation nonchalant, swigging a beer, and ends it
with his shoulders stooped forward, the light drained from his eyes. His
honesty and helpless frustration is so familiar from my actual life that
it is nearly too much to bear --- a perfect moment of cinema.

I asked Jenkins about his ability to capture a rarely depicted kind of
black male vulnerability as he drove us down Vermont Avenue in his white
Toyota Prius. He said nothing for a while, a first in our hours of
conversation. A chopped-and-screwed version of Solange's ``A Seat at the
Table'' drifted from the speakers. ``I will say that's not a goal, but I
would hope that's true,'' he said finally. ``I think I know what you're
talking about, especially these last two films.''

Jenkins is open, specific and effusive when talking about craft,
narrative, culture, sports or politics. Less so when discussing himself.
He tends to speak in long bursts, turning an idea around out loud before
hitting on the declarative statement he likes. He started talking about
his work adapting ``In Moonlight Black Boys Look Blue,'' the
autobiographical script by Tarell Alvin McCraney that became his
breakthrough feature. The story had uncanny overlaps with Jenkins's own
biography: an addicted mother, an upbringing in the same Miami ghetto,
subjects Jenkins would not have been comfortable exploring without
McCraney's words as an intermediary. ``I can take Tarell's experience,''
Jenkins said, ``which is much more open than mine, and I can sew
cinematic storytelling into it.'' McCraney, whose gay coming-of-age
story informs the plot of ``Moonlight,'' describes Jenkins, who has a
girlfriend, as approaching emotionality with something akin to a queer
imagination: ``I think he has moments where his imagination is not
normative. Just look at his work and see. Where there should be broad
strokes there are polka dots.''

``Barry loves love,'' Romanski says. ``If you look at his work, there
are three different types of love stories from `Medicine' to `Moonlight'
to `Beale,' but they're all love stories.'' Love, yes, but also pain.
Nat Sanders, who edited both ``Moonlight'' and ``Beale Street'' with
McMillon, says there's ``a duality'' to Jenkins: ``Any time he wants to,
he's so charismatic and exuberant. When I do a pass of a scene that
really excites him, you get the best reaction you could ever hope to
get. He'll do airplanes around the room and jump up and down, yell
`Bruh! Bruh!' and it's the best. But also, he's very private and very
quiet.'' In ``Chlorophyl,'' a short film Jenkins directed in 2011,
working without a formal script, the main character, played by Ana Laura
Treviño, reads a James Baldwin quote, translating into Spanish: ``Love
does not begin and end the way we seem to think it does. Love is a
battle, love is a war; love is a growing up.'' In a moment of
improvisation, Jenkins had handed her his leather cellphone case, on
which he'd paid to have these words embossed.

Although he values romance in film, Jenkins is hesitant to say he places
as much importance on romantic love in his own life. ``I've allowed
myself to experience growth in love once in my life, but it was so long
ago,'' he says. ``I don't know if I've become accustomed to being alone,
or if because of the way I grew up I became conditioned to being alone.
But I guess my identity has been formed around being alone.''

We were back downtown now, in the Arts District, pulling up to a
bookstore on the hunt for a novel I'd recommended to him. It turned out
that this one, the second of three we'd try, did not sell fiction.
Jenkins did a lap anyway, searching for a book of photography by Roy
DeCarava, whose iconic images of midcentury Harlem served as a visual
reference for ``Beale Street'' and appear at the beginning and end of
the film. On our way back to the car, we passed a boutique where Jenkins
stopped to inquire after the owner: Had she had her baby yet? She had.
Jenkins conveyed his congratulations. For a man who considers being
alone a comfortable mode of existence, he makes reaching out to others
look easy.

Jenkins was born on Nov. 19, 1979, and raised in the Liberty City
neighborhood of Miami. When he was 6 months old, riots in response to
the acquittal of police officers involved in the death of a black man
named Arthur McDuffie worsened blight in his segregated neighborhood.
``I'm the baby, by a mile,'' he says. ``I have one sister, one
brother.'' His mother gave birth to his older siblings at 15 and 16
years old. Nine years later, she gave birth to him. ``I don't think any
of us were planned, but I was definitely a mistake,'' he says. ``I have
no idea who my father is.''

He grew up in Knight Manor (also known as the Village), a housing
project built close to the larger Liberty Square projects. His mother
was addicted to crack, a fact that Jenkins is both open about and firm
in not expounding upon. When ``Moonlight'' had its premiere, he told me,
``We rented a theater just for her to go see the film, and on the day
of, she decided she could not see it. Probably my fault. I should have
flown back.'' Something previously open about Jenkins's posture --- the
set of his shoulders, the steadiness of his gaze --- closed off as he
told me this. Later, he says, his mother happened upon a bootleg copy of
the film at a corner store; she bought it and watched at home. ``What I
see in that,'' Jenkins says, ``is she just wanted to watch it by herself
in an environment she could control. She actually complimented Naomie's
performance'' --- Naomie Harris, who plays the main character's troubled
mother. ``Say of that what you will.''

Knight Manor was torn down, and a new campus for Miami Northwestern high
school built on top of it. Jenkins would attend classes there, preparing
for the future atop his own demolished childhood home --- though he
doesn't recall giving this symbolism much thought. Of his home life
during these years he reveals little, though he alludes to it often. ``I
grew up in a certain kind of way,'' he told me. ``It's a crutch. I
started therapy, I stopped. I should start it again. It's a crutch.''
His reluctance to explain further may also reflect his general
disinterest in creating a public persona, and in the self-mythologizing
that goes with it. ``It was what was so difficult about everything that
happened with `Moonlight,' '' he says. ``I'm not smart enough to say as
much as I have to say about the film without revealing things about
myself. But that's the last thing I ever want to do.''

In high school Jenkins played football, got good grades, made it to
senior year. He had been admitted to the University of Florida, but
after a trip to visit F.S.U. in Tallahassee, ``I thought: This is the
blackest place in America. I gotta be here.'' In Tallahassee, he lived
with high school friends and intended to pursue a creative-writing
degree --- he used to be obsessed with the Iowa Writers' Workshop ---
until one day he walked by a sign for the film school and had a change
of heart. He began pursuing a dual degree. ``I remember my first short
film,'' he says. ``I put my roommate in it, and it was a strange film.
We shot it on a Bolex'' --- a spring-wound camera associated with
avant-garde film --- ``so it wasn't like a narrative. It was more
experimental.'' His roommates, he says, ``just looked at me like they
thought I had lost my mind.'' In the film department, though, Jenkins
began making connections with the collaborators he'd work with over the
next two decades: McMillon and Sanders; James Laxton, the
cinematographer for all his feature-length films; and Romanski and Mark
Ceryak, two of his partners at the production company Pastel.

After watching other students' work, Jenkins didn't think he had the
technical ability to compete. ``Digital filmmaking wasn't anywhere near
what it is now,'' he says; working on film, ``you couldn't see what you
were doing. There's no playback on set. You can't shoot it and then
immediately pull it up on the memory card.'' The filmmaker needed solid
knowledge from the start. Jenkins took a year away from the film program
to study the craft on his own, finishing his creative-writing credits in
the meantime. He moved in with Laxton. The popular directors to emulate
in their program were those who made big-budget, Spielbergian features.
Jenkins watched foreign art-house cinema in hopes of finding an
alternative aesthetic approach. ``I remember we'd have running DVDs
playing in the background,'' Laxton told me. ``And as people would come
in and out the house, we'd kind of sit down and talk about it.'' Before
that year, Jenkins would have named ``Die Hard'' as his favorite film.
After, it was the Hong Kong filmmaker Wong Kar-wai's stylized, lovesick
``Chungking Express.''

When Jenkins returned to the program he wrote and directed the short
film ``My Josephine,'' a mostly subtitled account of an Arab immigrant
couple who own a laundromat and wash American flags for free after the
Sept. 11 attacks. Over wine at a bar near his apartment, Jenkins
recalled making it: ``That's me as a dude who grew up in the hood!'' he
said, his face alight once again. ``That's not me now, 20 years removed
from the hood. That's me, you know what I mean? And it's taking my
experience of being somebody from the hood who now is a fish out of
water in this academic setting, in this film consortium, taking that
feeling and putting it into these things. It was empowering.''

\hypertarget{-1}{%
\subsection{}\label{-1}}

After graduation, the director Darnell Martin gave Jenkins his first
film job, as her assistant on the set of the TV movie ``Their Eyes Were
Watching God.'' His work on ``My Josephine,'' she told me, had ``so much
poetry in it. Visually it was really stunning.'' Here was an avenue
through which Jenkins could grapple with feelings he otherwise preferred
to leave unexamined. ``I think when I found film,'' he said, ``I found a
way --- I still hide a bit --- but a way to not hide as much. I felt
like I could put these things into the work because it's the movie. It's
not me.''

On a very hot August day, Jenkins sat in a color-grading studio in
Greenwich Village with Laxton and the colorist Alex Bickel. The big
screen at the other end of the room showed a scene at Showmans Jazz Club
in Harlem. The characters Joseph and Frank, played by the veteran actors
Colman Domingo and Michael Beach, sat at the far end of the bar. The
room was realistically smoke-filled for the early '70s, but too smoky
for the purposes of the film. Jenkins and Laxton took turns guiding
Bickel on how to fix the problem. They moved on to a close-up of Beach's
face, its topography made more intense by his look of dejection. Jenkins
made a noise of approval. ``It was like Take 3 of the zoom,'' he said,
``and he just locked in.''

Jenkins enjoys moments when his actors make direct eye contact with the
camera. He and Laxton are in agreement on the power of this sustained
looking, how holding the camera on an actor can bring out a host of
emotions in the viewer. ``If you're in a dark theater with 300 people
sitting next to you,'' Laxton said, ``and you have someone looking at
you from a big screen, I think it does something to you as an audience
member.'' Alfred Hitchcock employed these sorts of shots, as did
Jonathan Demme (who can forget Hannibal Lecter's stare?), but unlike
those filmmakers, Jenkins and Laxton rarely shoot theirs during moments
of great emotional agitation. Instead they catch their characters at
ease, quiet. ``Barry captures silence in a way that we don't see much,
and we especially don't see that much in the African-American film
experience,'' Mahershala Ali, who won a best-supporting-actor Oscar for
playing the drug-dealer-cum-father-figure in ``Moonlight,'' told me.
``You usually don't see black people holding peace and occupying
silence, having to fill those voids in that way.''

Jenkins shoots these moments intuitively, waiting until he feels
something. ``I'm not directing them,'' he says. ``They are just giving
me this thing. And sometimes you can look at an actor and see, Oh,
there's the soul.'' And if they're comfortable enough, he says, they can
look directly into the camera without losing that soul. ``Instead
they're going to give it to the audience.'' KiKi Layne, whose starring
role as Tish in ``Beale Street'' marks her first foray into film from
theater, described it as looking into a black hole: ``I think at one
point I told him that `Man, this feels so strange' and he was like, `I
know, but I need it, I need it.' '' The actors don't know where these
shots will wind up in the film, and neither, necessarily, does Jenkins,
at the time. Later, though, the emotions viewers read on the actors'
faces --- a close-up of Fonny near the end of the film goes from anxious
and unsure to settled --- feel made for the precise moment when they
appear on screen.

These looks don't quite break the fourth wall, because the actors are
not regarding the audience. In ``Beale Street,'' they're most often
gazing at someone they love. For nonblack audience members, it might be
the first time they've had a black person direct such a gaze their way;
Jenkins offers a glimpse at a world previously hidden to them. For a
black viewer, there's more likely a kind of recognition: I know that
face, although I have never seen this actor before. Or, if the actor is
one you're familiar with, it can go the opposite direction, letting you
see the person anew. Regina King, who plays Tish's mother in the film,
has played a mother or wife as many times as I have fingers, over
decades. But who was this woman on the screen, staring at her reflection
in a mirror, summoning her courage, while also staring at me?
Typecasting actors isn't simply about having them play a role they have
played before; it's about locking them into the same aesthetic
representation of that role. ``He knows that it's not just his film,''
King told me. ``He can't do this without the talent of other people, and
he allows those talents to shine.''

During the festival season leading up to the theatrical release of
``Moonlight,'' Jenkins directed an episode of the first season of the
Netflix series ``Dear White People,'' revolving around an actor named
Marque Richardson; his character has a gun pulled on him by campus
police officers, and by the episode's end he breaks down, crying. I have
personally known Marque Richardson for 15 years and have watched him act
in all sorts of roles. This episode was the first time he ever felt a
stranger to me on screen.

Jenkins prefers to remain on a last-name basis with older women he
respects --- ``Ms. Winfrey,'' ``Ms. Morrison'' --- but when it comes to
the woman whose work in film he admires most, the 72-year-old French
director Claire Denis, he calls her ``Claire,'' with possessiveness. He
gushes about her on social media (``I JUST MET CLAIRE DENIS AND MAYBE
GOT A LIL TEARY EYED'') and describes himself as a ``Claire Denis stan''
in his Twitter bio. ``Having watched so much of her work, and read so
many of her interviews, and ingested the principles of how she
approaches art,'' he says, ``I think it does affect the way I make the
things I make. But they're not like the things she makes.'' A framed
poster for ``Beau Travail,'' Denis's 1999 drama of male camaraderie,
desire and rivalry within a French Foreign Legion outpost in Djibouti,
hangs in his living room. ``She's unconcerned with the audience, I would
say. It doesn't occur to her, and because of that I think her movies are
pure artistic curiosity, artistic impulse, artistic reckoning and
reasoning.''

This level of artistic freedom is hard to come by in the world of
Hollywood-financed film, he thinks, but the possibility of leading such
a life --- not feeling obligated to have one foot planted in the role of
presenter --- captivates him. He could also see himself teaching film to
others in 15 years. In the short term, though, he is working on a
forthcoming limited-series adaptation of Colson Whitehead's ``The
Underground Railroad'' for Amazon, and a screenplay based on the life of
the gold-medal-winning boxer Claressa Shields. ``I wish I had taken a
break after the Oscars and all that stuff, but I just went right into
this movie, and right into the writers' room for `The Underground
Railroad.' I didn't really process all that stuff. And now this movie is
finally ending, and I think it's time for a moment to process.''

It was our last few minutes together, and I wanted to know --- having
heard him speak of a more unfettered artistic life, and witnessing him
bring happiness to others with such ease --- what exactly made him
happy. ``I'm never happy, man --- I mean, that's just the case,'' he
said. He used to think he didn't deserve to be. Then: ``I think making
other people happy makes me happy.'' He recalled helping Alex Hibbert,
who plays the child version of the main character in ``Moonlight,'' find
his footing as an actor. ``When other people do things on my sets that
either they didn't believe they could do, or understand they could do,
or that surprises them?'' That, he said, was amazing. It was a matter of
helping them tap into the ineffable. ``What I say to the actors, if they
ask, is that it's not about a certain meaning; it's definitely about a
certain feeling. How can we work together to create this feeling, even
if the feeling is not knowable?

``I really want that simple life, you know?'' Jenkins said. ``I think
that's why I fell in love with this book.'' He quoted the moment in
Baldwin's novel when Fonny, the young sculptor jailed for a crime he
swears he didn't commit, says he only needs art and the love of his
woman to survive: `` `I got two things in my life, man --- I got my wood
and stone, and I got Tish.' '' Despite the increasing complexity that
comes with a successful Hollywood career, Jenkins desires the same. ``I
have my wood and stone now,'' he said.

Angela Flournoy is the author of the novel ``The Turner House'' and a
2018 National Endowment for the Arts fellow. She last wrote for the
magazine about
\href{https://www.nytimes3xbfgragh.onion/interactive/2018/03/08/magazine/25-songs-future-of-music.html\#/big-shaq}{Big
Shaq's song ``Mans Not Hot.''}

\hypertarget{the-culture-issue}{%
\subsection{The Culture Issue}\label{the-culture-issue}}

\begin{itemize}
\tightlist
\item
  \href{https://www.nytimes3xbfgragh.onion/interactive/2018/10/03/magazine/lady-gaga-movie-star-is-born.html}{}
\item
  \href{https://www.nytimes3xbfgragh.onion/interactive/2018/10/03/magazine/morality-social-justice-art-entertainment.html}{}
\item
  \href{https://www.nytimes3xbfgragh.onion/interactive/2018/10/04/magazine/good-place-michael-schur-philosophy.html}{}
\item
  \href{https://www.nytimes3xbfgragh.onion/interactive/2018/10/04/magazine/barry-jenkins-james-baldwin-if-beale-street-could-talk.html}{}
\item
  \href{https://www.nytimes3xbfgragh.onion/interactive/2018/10/05/magazine/instagram-cindy-sherman-ugly-beauty.html}{}
\item
  \href{https://www.nytimes3xbfgragh.onion/interactive/2018/10/05/magazine/culture-moments-2018.html}{}
\end{itemize}

\hypertarget{related-coverage}{%
\subsection{Related Coverage}\label{related-coverage}}

\begin{itemize}
\tightlist
\item
  \href{https://www.nytimes3xbfgragh.onion/interactive/2018/10/03/magazine/lady-gaga-movie-star-is-born.html}{}
\item
  \href{https://www.nytimes3xbfgragh.onion/interactive/2018/10/03/magazine/morality-social-justice-art-entertainment.html}{}
\item
  \href{https://www.nytimes3xbfgragh.onion/interactive/2018/10/04/magazine/good-place-michael-schur-philosophy.html}{}
\item
  \href{https://www.nytimes3xbfgragh.onion/interactive/2018/10/04/magazine/barry-jenkins-james-baldwin-if-beale-street-could-talk.html}{}
\item
  \href{https://www.nytimes3xbfgragh.onion/interactive/2018/10/05/magazine/instagram-cindy-sherman-ugly-beauty.html}{}
\item
  \href{https://www.nytimes3xbfgragh.onion/interactive/2018/10/05/magazine/culture-moments-2018.html}{}
\end{itemize}

2018

\hypertarget{more-on-nytimescom}{%
\subsection{More on NYTimes.com}\label{more-on-nytimescom}}

Advertisement

\hypertarget{site-information-navigation}{%
\subsection{Site Information
Navigation}\label{site-information-navigation}}

\begin{itemize}
\tightlist
\item
  \href{https://help.nytimes3xbfgragh.onion/hc/en-us/articles/115014792127-Copyright-notice}{©
  2020 The New York Times Company}
\item
  \href{https://www.nytimes3xbfgragh.onion}{Home}
\item
  \href{https://www.nytimes3xbfgragh.onion/search/}{Search}
\item
  Accessibility concerns? Email us at
  \href{mailto:accessibility@NYTimes.com}{\nolinkurl{accessibility@NYTimes.com}}.
  We would love to hear from you.
\item
  \href{https://help.nytimes3xbfgragh.onion/hc/en-us/articles/115015385887-Contact-Us}{Contact
  Us}
\item
  \href{https://www.nytco.com/careers/}{Work with us}
\item
  \href{https://nytmediakit.com/}{Advertise}
\item
  \href{https://help.nytimes3xbfgragh.onion/hc/en-us/articles/115014892108-Privacy-policy\#pp}{Your
  Ad Choices}
\item
  \href{https://help.nytimes3xbfgragh.onion/hc/en-us/articles/115014892108-Privacy-policy}{Privacy}
\item
  \href{https://help.nytimes3xbfgragh.onion/hc/en-us/articles/115014893428-Terms-of-service}{Terms
  of Service}
\item
  \href{https://help.nytimes3xbfgragh.onion/hc/en-us/articles/115014893968-Terms-of-sale}{Terms
  of Sale}
\end{itemize}

\hypertarget{site-information-navigation-1}{%
\subsection{Site Information
Navigation}\label{site-information-navigation-1}}

\begin{itemize}
\tightlist
\item
  \href{https://spiderbites.nytimes3xbfgragh.onion}{Site Map}
\item
  \href{https://help.nytimes3xbfgragh.onion/hc/en-us}{Help}
\item
  \href{https://help.nytimes3xbfgragh.onion/hc/en-us/articles/115015385887-Contact-Us?redir=myacc}{Site
  Feedback}
\item
  \href{https://www.nytimes3xbfgragh.onion/subscription?campaignId=37WXW}{Subscriptions}
\end{itemize}
