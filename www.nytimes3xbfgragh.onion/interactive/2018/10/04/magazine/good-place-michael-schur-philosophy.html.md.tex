 **NYTimes.com no longer supports Internet Explorer 9 or earlier. Please
upgrade your browser.
\href{http://www.nytimes3xbfgragh.onion/content/help/site/ie9-support.html}{LEARN
MORE »}

**Sections

**Home

**Search

\hypertarget{the-new-york-times}{%
\subsection{\texorpdfstring{\href{http://www.nytimes3xbfgragh.onion/}{The
New York Times}}{The New York Times}}\label{the-new-york-times}}

\hypertarget{-magazine-}{%
\subsubsection{\texorpdfstring{
\href{https://www.nytimes3xbfgragh.onion/section/magazine}{Magazine}
}{ Magazine }}\label{-magazine-}}

 \href{https://www.nytimes3xbfgragh.onion/section/magazine}{Magazine}
\textbar{}What Makes `The Good Place' So Good?

**Close search

\hypertarget{site-search-navigation}{%
\subsection{Site Search Navigation}\label{site-search-navigation}}

Search NYTimes.com

**Clear this text input

Go

\url{https://nyti.ms/2O8an6L}

\hypertarget{site-navigation}{%
\subsection{Site Navigation}\label{site-navigation}}

\hypertarget{site-mobile-navigation}{%
\subsection{Site Mobile Navigation}\label{site-mobile-navigation}}

\hypertarget{what-makes-the-good-place-so-good}{%
\section{What Makes `The Good Place' So
Good?}\label{what-makes-the-good-place-so-good}}

NBC gave Michael Schur total freedom. So the TV impresario made a sitcom
that's also a profound work of philosophy.

\hypertarget{the-ultimate-sitcom}{%
\section{The Ultimate Sitcom}\label{the-ultimate-sitcom}}

NBC gave Michael Schur total freedom. So the TV impresario made a sitcom
that's also a profound work of philosophy.

By \href{https://www.nytimes3xbfgragh.onion/by/sam-anderson}{SAM
ANDERSON} OCT. 4, 2018

How do hands move in heaven? Ted Danson knows. Watch him in ``The Good
Place,'' NBC's circle-squaring philosophical sitcom about life, death,
good, evil, redemption and frozen yogurt. As Danson speaks, his hands
flutter and hover in front of him like a pair of trained birds. They
poke and swirl, pinch and twist. They snap suddenly ahead to accent a
word as if they're plucking a feather from a passing breeze. Danson is
tall and slim --- he was a basketball star growing up --- and his hands
are expressively large. He can move them, when he needs to, with the
long-fingered languor of Michelangelo's God reaching out to touch Adam.
On the show, Danson plays an ``architect'' of the afterlife named
Michael, a sort of immortal Willy Wonka who dresses in bright suits and
bow ties. He is always flying into spasms of delight over the
fascinating novelties of human culture --- paper clips, suspenders,
karaoke, Skee-Ball --- and in one scene he gets so celestially excited
that he lunges into a squat, holds his arms out in front of him and
gyrates his wrists like an electric mixer on full blast. ``How do you
pump your fist again?'' he asks. ``Is this it?''

Danson is now 70, roughly twice the age he was when he started on
``Cheers,'' and he carries his seniority around as if it were the
funniest thing in the world. In recent years, he has put together a
virtuosic run of performances, including supporting roles in shows such
as ``Damages,'' ``Bored to Death'' and ``Fargo.'' His turn on ``The Good
Place'' seems like a culmination of this late-phase greatness: a role he
was born, and then very carefully aged, to play. Although Danson still
has the seductive good looks of Sam Malone, his hair has gone chalk
white, and he delivers his lines with an ease so deep it looks
effortless and instinctive --- the kind of thing you couldn't practice
if you tried.

In May, I happened to be on the set of ``The Good Place,'' watching the
making of an episode of Season 3, when Danson arrived. They were filming
that day inside an obscenely large mansion in Beverly Hills --- the kind
of gaudy faux-French superpalace you would find, in fact, by Googling
the words ``obscenely large mansion.'' Danson walked in wearing a white
suit jacket, munching on a bag of snacks. All around him was chaos.
Bearded men with multicolored rolls of tape hanging from their belts
hauled clattering carts past extras in formal wear holding glasses of
fake red wine. Danson worked his way cheerfully through the noise. He
wove past the ballroom's titanic staircase and the giant kitchen with
its archipelago of separate islands and the fireplace wide enough to
consume whole small trees. I was excited to watch him step in front of
the camera and do what I imagined would be his automatic Ted Danson
thing.

Instead, still munching his snacks, Danson slipped into one of the
mansion's back rooms. It, too, was crowded, full of people chatting and
laughing, looking at their phones, watching monitors, ignoring the set's
constant calls of \emph{``shhhhhhhh.''} Danson found an empty folding
chair in the far corner and sat. He became very still. He started to
speak softly, under his breath, whispering inaudible words into the
space directly in front of him. Soon his large hands rose, poking and
swirling, and then they fell, and then they rose, and then they fell.
Danson was fine-tuning, move by move, the components of his next scene.
He gripped the air in front of him and folded it like dough. It looked,
at times, as if he were conducting his own invisible orchestra. I
watched Danson do this, silently, for nearly half an hour. He may have
gone on longer --- he was still rehearsing, in total focus, when I had
to go off and do something else.

Maybe it's faint praise to call a show the best sitcom on TV. It's like
calling a vehicle the best horse buggy on the autobahn. Sitcoms no
longer sit anywhere near the vital center of American culture. We are
not a nation of families squeezing onto couches to watch can't-miss
programming. Our entertainment metabolisms have sped up and scattered in
a thousand directions.

And yet the fact remains. ``The Good Place'' --- the story of a group of
recently deceased earthlings navigating the afterlife --- is the best
sitcom on TV. It is, in many ways, charmingly old-fashioned: a 22-minute
glossy confection of family-friendly LOLs, featuring beautiful
multicultural characters with perfect hair trading tightly scripted
jokes as they learn life lessons. Grandchildren and grandparents could
watch it together in relative comfort, laughing in heartwarming unison.

But the show is also, by network standards, quite radical. It attempts a
clever gambit. The American sitcom, since its inception, has struggled
with a fundamental tension at its core. Let's call it ``jester vs.
guru.'' We expect half-hour comedies to pull off an impossible double
duty: to both inject jokes into the national bloodstream and to
enlighten us with high-minded moral instruction. We want not only zany
catchphrases but wise life lessons. The history of the form has been a
constant tug of war between these two contradictory demands. Early
sitcoms tended toward Very Special Episodes --- morality plays in which
we learned to honor our parents, say no to drugs and rat out even our
most charming friends. The sitcoms that followed rebelled against such
ham-fisted piety, replacing it with ironic cynicism. ``Seinfeld''
famously rejected the moral duties of the sitcom altogether; ``30 Rock''
was a pure fire hose of laughs. The control knob turned, further and
further, from wisdom toward jokes.

``The Good Place'' tries, improbably, to fulfill both functions at once.
It wants to sit at both ends of the control knob simultaneously. Like
any good modern comedy, the show is a direct IV of laughs, but the trick
is that all of those laughs are explicitly \emph{about} morality.

The premise of ``The Good Place'' is absurdly high concept. It sounds
less like the basis of a prime-time sitcom than an experimental puppet
show conducted, without a permit, on the woodsy edge of a large public
park. The show's action begins in a candy-colored heaven in which new
residents are welcomed to find their perfect soul mate, an ideal home
and an eternal supply of frozen yogurt. (Flavors include Double Rainbow,
Four-Day Weekend, Full Cellphone Battery, Panoply of Exuberance and
Beyoncé Compliments Your Hair.) There is just one problem: Eleanor
Shellstrop, our foulmouthed protagonist, does not belong anywhere near
any kind of paradise. Eleanor is a comically awful person --- in
flashbacks, we see her refusing to be a designated driver, ruining a
stranger's \emph{quinceañera} and selling fake medicine to the elderly.
Her arrival at the Good Place seems to be a result of some kind of
existential clerical error. Eleanor is understandably reluctant to
confess this, particularly when she learns about the many horrors of the
Bad Place: bees with teeth, four-headed bears, volcanoes full of
scorpions and --- unfortunately --- ``butthole spiders.'' Out of sheer
desperation, she decides to try something drastic: to improve herself.
Eleanor manages to persuade her alleged soul mate, a Senegalese
professor of ethics and moral philosophy named Chidi, to teach her how
to be good. ``How do we do it?'' she asks. ``Is there a pill I can take
or something I can vape?''

This is the trick of ``The Good Place.'' Ethics is not some kind of
moralistic byproduct; it's baked into the very premise. The show is
entirely life lessons. Every episode is Very Special. It synthesizes
those old contradictory impulses --- jester vs. guru --- so completely
that they cease to be in tension. If ``Seinfeld'' was a show about
nothing, ``The Good Place'' is a show about everything --- including,
and especially, growing and learning. By all rights, it should probably
be awful --- preachy, awkward, tedious, wooden, labored and out of
touch. Instead, it is excellent: a work of popular art that hits on many
levels at once. It has been not only critically acclaimed but also
widely watched, especially on streaming services, where its twists and
intricate jokes lend themselves to bingeing and rebingeing. The modern
world, perhaps, is hungrier for ethics than we have been led to believe.

Back in the mansion, sometime after his private rehearsal session, Ted
Danson thanked me, very sincerely, for coming to write about ``The Good
Place.'' Then he corrected himself. ``Well,'' he said, ``thanks for
coming to write about Mike.'' He was referring to Michael Schur, the
show's creator and driving force. I told Danson I wasn't there to write
just about Schur; I was there to write about the whole show. But Danson
waved this off with one of his big, elegant hands. ``It's all Mike,'' he
told me. ``We're all just a bunch of little Mikes.''

As Michael is to the Good Place, Michael Schur is to ``The Good Place''
--- the architect with everything on the line. (It is surely no
coincidence that Schur named Ted Danson's character after himself.)
While the episode was being shot at the mansion, Schur was sitting five
miles away, in his bungalow on the back lot of Universal Studios,
worrying. He was thinking forward and backward, checking and rechecking
his work, trying to exist in 10 different time signatures at once.
Putting a TV show together --- any TV show --- is a crushing logistical
nightmare. There are pitches and drafts and scripts and casting calls
and table reads and revisions and rehearsals and budgets. You have to
scout locations and then unleash huge quasi-military mobilizations of
equipment. On top of which, ``The Good Place'' offers its own unique
challenges. Its plot is an ongoing logic puzzle with twists that have to
work across several different dimensions, and the integrity of that
puzzle sometimes keeps Schur up at night.

Schur grew up in Connecticut and has a classic New England comedy
pedigree: president of The Harvard Lampoon, staff writer on ``Saturday
Night Live.'' He is now 42. He has a boyish face on a squarish head, and
his thick, dark hair is brushed at the temples with white. He speaks in
earnest, open paragraphs, with the clear pleasure of someone who enjoys
exploring his own brain. ``To be totally honest,'' he says as soon as I
step into his office, ``I'm having a hard time gauging whether I'm O.K.
with being a writer who has a vintage typewriter on his coffee table.''
(It was a gift.) Our conversation covered the prodigious gifts of
Kristen Bell --- ``She just has a really low center of gravity for how
she approaches her job; you can give her 40 notes on a line, and she'll
go, `Yep, got it,' and she'll do all 40 of those notes at once'' --- and
Schur's own struggle with what he calls directional insanity: ``If I
don't put a tremendous amount of concentration into it, I will get lost
even going from my office to my house.'' He lamented the persistence of
toxic masculinity in show business. ``People in general are far too
tolerant of bad behavior,'' he told me, ``because they think it's
necessary for creativity. But I don't think you should ever think they
are one and the same.''

Schur is famous in the industry for his policy of --- as he puts it in
polite company --- ``no jerks.'' This applies to every level of every
project, from writers to directors to actors, and people say it is
life-changing; there is a dedicated group of talent that follows Schur
from show to show. His rise to network power has corresponded with a new
tone in prime-time comedy, an era of good-hearted humanistic warmth. One
of Schur's signature early achievements was helping NBC, as part of the
writing team, transform the original British version of ``The Office''
--- the revered but bitterly cynical BBC mockumentary --- into its
softer American incarnation. The show flipped the ratio of cringing to
pleasure and ran for nine seasons. Afterward, Schur co-created ``Parks
and Recreation,'' which took the humanistic impulse even further. It was
a sort of course correction from the shows Schur grew up on, the
irony-drenched sitcoms of the 1990s.

``I'm not a huge fan of the `Seinfeld' era, personally,'' he told me.
``I liked it when it was on --- it broke the form and put it back
together in this new way, and it was revolutionary and wonderful, and I
consumed it like candy. But I don't find the urge to go back and watch
it again. It's like doing a crossword puzzle for the second time. I
think there's a reason that shows where there is growth and learning are
more rewatchable.''

A few years ago, after all of Schur's success, NBC offered him a dream
opportunity: total freedom for his next project. He was ambitious enough
to see this not only as an opportunity but as a perverse creative
challenge. Schur is a big literary-fiction reader, and he loves formal
experimentation, and he especially reveres the late American writer
David Foster Wallace --- another innovator obsessed with goodness. In
his office, Schur keeps various Wallace quotes for inspiration. One,
from a 1993 interview, reads like a mission statement for ``The Good
Place'':

\begin{quote}
Look, man, we'd probably most of us agree that these are dark times, and
stupid ones, but do we need fiction that does nothing but dramatize how
dark and stupid everything is? In dark times, the definition of good art
would seem to be art that locates and applies CPR to those elements of
what's human and magical that still live and glow despite the times'
darkness. Really good fiction could have as dark a worldview as it
wished, but it'd find a way both to depict this world and to illuminate
the possibilities for being alive and human in it.
\end{quote}

The idea that excited Schur, for his next sitcom, was both simple and
infinitely complex: what it means to be a good person. It was an idea he
had been obsessed with in different forms for many years --- and that
had crystallized for him back in 2005, when Jennifer Philbin, who is now
his wife, got into a very minor traffic accident with a man driving a
Saab. No one was hurt, and no visible damage was done, and yet the
incident would become, Schur later wrote, ``one of the most interesting
and complicated events of my adult life.'' When the Saab driver filed
what Schur thought was an unnecessary insurance claim and demanded \$836
for bumper damage, Schur countered with a grandly high-minded
alternative. If the man would drop his claim, Schur said, he would
donate the \$836 to victims of Hurricane Katrina. Schur's plan went
viral, and friends and supporters jumped in to pledge more than \$30,000
--- an incredible philanthropic victory --- and yet Schur began to feel
a growing sense of unease. He suspected that his mission was not,
perhaps, entirely righteous. There was an element of grandstanding to
the gesture, of moral one-upmanship, and Schur spoke about it with his
family and colleagues and even professors of ethics. He became
fascinated by the ways people can rack up ethical credits and debits all
at the same time. This, eventually, would become the subject of his
show.

Goodness is a notoriously difficult topic --- a tangled knot at which
religions and philosophers have been picking for all of human history. A
22-minute network comedy seems like exactly the wrong tool for the job.
It's like trying to hammer a nail with a banana peel. And yet that was
the tool that Michael Schur had. So he was going to try.

Schur's initial premise, in true sitcom fashion, was a kind of joke.
What if life was, ethically speaking, a sort of video game --- if every
action had a point value, positive or negative, and the goal was to rack
up the highest score? This is where ``The Good Place'' started. In the
show's first episode, during Michael's orientation session, you can read
dozens of scored behaviors, positive and negative, on his celestial
presentation board: everything from ``pet a lamb'' (+0.89 points),
``remember sister's birthday'' (+15.02) and ``save a child from
drowning'' (+1,202.33) to ``stiff a waitress'' (-6.83), ``disturb coral
reef with flipper'' (-53.83) and ``poison a river'' (-4,010.55).

Then things got complicated. As Schur began to plan the show, he
embarked on an intense program of philosophical self-education. He read
the classics (Aristotle, Mill, Bentham, Rawls) and hunted new academic
papers online. He compiled thick reading packets and gave them to
colleagues. He started to wish that he could go back to college to study
philosophy full time. Philbin, who is also a writer, was overwhelmed by
the number of boxes arriving at their front door. One night, she walked
into the bedroom, brushing her teeth, to find Schur in bed studying a
slim black book with a raven on its cover --- the title of which was
simply ``Death.'' (``What do we make of ourselves,'' the book asks, ``if
the death that undermines us is a necessary feature of our lives being
worthwhile?'')

``This show is going to ruin my life, isn't it?'' she asked.

One day out of the blue, Pamela Hieronymi, a professor at U.C.L.A., got
an email from Schur, asking if she would speak to him about ethics.
Hieronymi is not a TV watcher and had no idea who Schur was, but she
agreed, and they ended up talking for three hours, largely about whether
it is possible to become a good person by trying --- about how intention
and motivation color our moral behavior. Hieronymi was impressed by
Schur's earnestness and curiosity. It was clear that he didn't just want
to make jokes about philosophy; he wanted to actually understand the
ideas. Eventually, Schur asked Hieronymi to join the show as a
``consulting philosopher'' --- surely a first in sitcom history. Later
he brought on Todd May, the author of that slim book about death. The
consultants spoke not only to Schur but also to the writers' room,
giving lectures on existentialism and the famous thought experiment
known as the Trolley Problem, ideas which were later woven into the
show. All of which is to say ``The Good Place'' is not about philosophy
in the way that ``The Big Bang Theory'' is about science --- as a set of
clichés to tap for silly jokes. A sitcom is not a grad school seminar,
obviously, so the philosophy is highly abridged. But it is not
insubstantial, and philosophical ideas actually determine and shape the
plot.

At the beginning of Episode 6, Chidi holds up a book: a thick academic
paperback with one of those devastatingly quiet covers (earth tones,
Morandi still-life) that make you feel as if you will never be allowed
to leave the library again.

Eleanor reads its title aloud --- ``What We Owe to Each Other'' --- and
gasps.

``I saw this movie!'' she says. ``Laura Linney cries in a lake house
because Jude Law left her for his ex-wife's ghost.''

This synopsis, of course, is incorrect. The book is actually a dense
work of philosophy by the Harvard emeritus professor T.M. Scanlon. It
introduces an idea called ``contractualism.'' As Chidi explains it to
Eleanor: ``Imagine a group of reasonable people are coming up with the
rules for a new society. ... But anyone can veto any rule that they
think is unfair.'' (``Well, my first rule would be that no one can veto
my rules,'' Eleanor responds, to which Chidi counters, ``That's called
tyranny, and it's generally frowned upon.'')

The book seeks to explain how human societies might find moral authority
without appealing to a deity or inherited laws. The answer comes from a
sort of idealized social negotiation --- the process of thinking, in
good faith, with a community of other good-faith thinkers. As Scanlon
puts it: ``Thinking about right and wrong is, at the most basic level,
thinking about what could be justified to others on grounds that they,
if appropriately motivated, could not reasonably reject.''

Pamela Hieronymi introduced Schur to ``What We Owe to Each Other'';
Scanlon was her dissertation adviser at Harvard. It was the perfect way
to deepen the show's original premise --- that mechanistic notion of an
ethical points system. It was richer, Hieronymi argued, to think of
morality in terms of cooperative human relationships --- the way
networks of people, with their interdependencies and conflicts, have to
find a way to coexist and sacrifice and treat one another with respect.
In such messy human environments, ethical choices rarely map directly
onto obvious results. There are no leader boards. The problems can be
almost infinitely complex.

Schur loved not only the central thesis of ``What We Owe to Each Other''
but also the book's title. ``It \emph{assumes} that we owe things to
each other,'' he told me. ``It starts from that place. It's not like: Do
we owe anything to each other? It's like: Given that we owe things to
each other, let's try to figure out what they are. It's a very quietly
subversive idea.''

It is, in a way, deeply un-American --- an affront to our central
mythology of individual rights, self-interest and the sanctity of the
free market. As an over-the-top avatar of all our worst impulses,
Eleanor is severely allergic to any notion of community. And yet her
salvation will turn out to depend on the people around her, all of whom
will in turn depend on her. What makes us good, Chidi tells her, is
``our bonds to other people and our innate desire to treat them with
dignity.'' As the show progresses, ``What We Owe to Each Other'' becomes
a recurring character, popping up onscreen at several crucial plot
points. This amazed Hieronymi --- the last thing she had expected to see
was her dissertation adviser's book featured prominently on a network
sitcom.

Watching at home, Hieronymi was pleased with the show's evolution.
``What's going to save the characters is the relationships they have
with one another,'' she said. ``That seems exactly right to me.''

The cast of ``The Good Place'' is its own little experiment in
contractualism. Ted Danson and Kristen Bell are, in different ways,
Hollywood royalty --- experienced, award-winning, brand-name
celebrities. By contrast, the show's other core actors are so new to the
business that everyone on set refers to them, affectionately, as ``the
babies.'' Jameela Jamil, who plays a magnetically self-absorbed British
socialite named Tahani, had never acted before her audition. Manny
Jacinto was a hip-hop dancer before he landed the part of Jason, a
scene-stealing doofus from Florida. William Jackson Harper and D'Arcy
Carden, who play Chidi and a heavenly version of Siri named Janet, were
both in their mid 30s and had given up on dreams of mainstream acting
success. (Carden says she still has frequent ``daymares,'' driving home
from the set, that none of this is really happening.) And yet the actors
on ``The Good Place'' seem to coexist in a spirit of radical
egalitarianism. The younger actors notice Danson arriving early to
obsessively work on his scenes, despite the fact that he is literally
Ted Danson. And they notice that Kristen Bell memorizes not only her own
lines but everyone else's too, and that she reads all the crazy
philosophical course packets Schur sends out, so that she can discourse
at length about moral particularism vs. the categorical imperative.

Making a sitcom is incredibly tedious. Every scene is repeated infinite
times, with tiny variations --- and then the cameras are moved and the
scene is repeated infinitely again. On set, Bell is a technician,
precise and rational and systematic. She has a bright, quick energy, and
she can change direction, multiple times, in a single line. In the
mansion, I watched her working and reworking a scene in which Eleanor
has a meltdown --- a moment of insecurity that escalates into an angry
retreat. Bell is small, barely five feet tall, and she was dressed that
day in jeans and a pink sweatshirt. In the middle of the mansion's
chaos, she seemed like the rock-solid center around which everything
else rotated.

I watched her perform the same scene over and over, take by take, as if
she were slowly turning up an emotional dial. The writer of the episode,
Joe Mande, stood a few yards away, with one headphone on and one off,
mouthing the script as Bell performed it. D'Arcy Carden watched on a
monitor and shook her head. ``What a lil' squirt,'' she muttered.
``She's so good.'' After many takes, Bell reached what she called her
``big'' take --- one in which her rising anger basically blew the gilded
doors off the room. (Later, editors would patch pieces of these
performances together, deciding which precise emotional shades served
the episode best.) Before storming out of the room, Bell had to rip out
a chunk of a giant cake with her hands, and she approached that task
with similar precision. She tried it standing on one side of the table
and then the other, with one hand and then two, pulling the chunk
violently up or dragging it straight back. ``I want a rough rip,'' she
declared finally. The cake was huge, and there were only two backups, so
it was important not to mess it up. But Bell needed only one cake. The
rip, indeed, was rough.

Why is such a strange show so popular? There are, of course, many
reasons. There are the hundreds of hours of effort that go into Ted
Danson's fleeting hand gestures, as well as the hundreds of hours of
reading that undergird Schur's understanding of philosophical ideas. But
a bigger reason might just be about timing. The sitcom is arguably the
defining commercial art form of the American 20th century. Here in the
ugly adolescence of the 21st century, ``The Good Place'' is using that
old artistic form to take an honest moral accounting of the modern
American soul. In doing so, it raises questions that have always been
essential but that now glow with a special radioactivity.

Imagine, if you will, that a person, or a group of people, have done
something bad. They have knowingly chosen to gain an advantage at the
expense of someone else. Perhaps they lied to potential investors about
a building's actual value. Maybe they bullied the vulnerable and then
laughed. Or maybe it was something worse: They took an entire continent
from its indigenous peoples. They enslaved, tortured, kidnapped,
murdered, lied, stole. Maybe the people who benefited from all these
various crimes are also willing to argue, strenuously, that none of it
actually happened --- or that if it did happen, it didn't matter, or
even that it was all actually for the good. Maybe these people have
managed, through further bad actions, to put themselves into positions
of power, where they will have outsize influence on everything that
happens next.

Can a person, or a nation, that finds itself in such a situation --- can
it ever change? Is such deep badness redeemable? Is there any hope?

In a low moment, I mentioned to Hieronymi that American culture seems to
have abandoned ethics. She disagreed strongly. ``It's amazing to me how
moralized and moralistic we seem to be,'' she said, ``especially right
now. It's just a cultural blamefest.'' All the arguments that rage every
day across social media and cable news --- racism, reverse racism,
statutes of limitations, reparations --- are fundamentally about ethics.
Even the top-down distractions meant to derail these conversations are
conducted under the guise of earnest concern for right and wrong. ``I
think that's got to be part of the popularity of `The Good Place,' ''
Hieronymi said.

Schur told me he wants to stress, in his show, the hard work of
morality. So much of our ethical life is about thankless grinding
drudgery, daily feats of internal strength, a constant invisible
resistance.

``It feels, all the time in life, like a bad decision is right in front
of you,'' Schur said. ``No matter who you are, there's the opportunity
to make bad decisions and hurt people. And it takes work just to keep
not making those bad decisions. It takes a lot of concentrated effort to
do the right thing all the time. Hopefully, you get so used to it, and
it becomes such a part of who you are, that it doesn't take work ---
you're on autopilot making good decisions. But not always, and for a lot
of people, not ever. You don't have to look very hard to see a group of
people in this country who have given in and are just making the worst
decisions you can make. Like the most selfish, the most corrupt, the
most evil decisions --- and they're just doing it as a matter of course.
And it's way too late. They're never gonna go the other way.''

In the face of so much badness, Schur said, it is always tempting to
give up. But the heroic thing is simply to try.

``You have to work at it, every day,'' he said. ``It's so hard. The
temptation will always be there to go: `Oh, no one's watching. No one's
looking. I'll just do this.' Whatever `this' is. If you throw a coffee
cup at a garbage can and you miss, you could just walk away. The amount
of bad you put into the universe is very minimal. But someone else is
gonna have to come along and pick that thing up, and it sucks. It's not
that person's problem, it's your problem. And it's a very slippery
slope. As you get accustomed to one kind of bad action as permissible,
then the Overton window shifts, right?

``And now the next thing is like, whatever --- you cheat on your taxes.
And you get away with it, because government bureaucracy is bad at
picking up on tiny errors people make. And you're like: All right,
nobody got hurt. Because you're not thinking about the school 82 miles
away that couldn't afford new textbooks because they didn't get enough
tax revenue and had to lower the school budget. All you're thinking
about is, I saved \$400 by cheating on my taxes, that's pretty cool. The
window just keeps shifting, and eventually you become the kind of person
who is making the bad, selfish, wrong decision by default instead of the
good one. And then 15 years have gone by.''

As he wrote this new season of ``The Good Place,'' Schur couldn't get an
image out of his head. It was a scene from the end of ``Saving Private
Ryan'' in which Tom Hanks, moments after being shot in the chest out in
the middle of the battlefield, turns to see a German tank rumbling
toward him. Bullets are pinging all around him, and the battle is
clearly lost, and he is almost certainly going to die. He is absolutely
not going to stop a tank. But in that moment, he decides to try. Hanks
takes out his little handgun and, arm trembling, fires bullet after
bullet at the oncoming tank.

``It's the weirdest source material,'' Schur said, ``but it is a good
analogy. That's the essence of `The Good Place' --- to put people in a
very difficult situation and have them say: `What's the next thing we
can do? What's the next thing we can do?' To point out that there's more
value in trying than in not trying, basically. I don't want to spoil
anything, but in the third season the characters get to a point where
they have a choice. Do you give up or do you try? And they decide to
try. And that is what the whole season is like. We'll keep trying as
long as we can. We'll keep trying. No one is perfect. No one will ever
win the race to be the best person. It's impossible. But, especially
since starting this show, I just think everyone should try harder.
Including me.''

Sam Anderson is a staff writer for the magazine who frequently writes
the New Sentences column. He recently wrote the
\href{https://www.nytimes3xbfgragh.onion/interactive/2018/06/06/magazine/love-city-new-york-introduction.html}{introductory
essay to the magazine's New York issue.}

\hypertarget{the-culture-issue}{%
\subsection{The Culture Issue}\label{the-culture-issue}}

\begin{itemize}
\tightlist
\item
  \href{https://www.nytimes3xbfgragh.onion/interactive/2018/10/03/magazine/lady-gaga-movie-star-is-born.html}{}
\item
  \href{https://www.nytimes3xbfgragh.onion/interactive/2018/10/03/magazine/morality-social-justice-art-entertainment.html}{}
\item
  \href{https://www.nytimes3xbfgragh.onion/interactive/2018/10/04/magazine/good-place-michael-schur-philosophy.html}{}
\item
  \href{https://www.nytimes3xbfgragh.onion/interactive/2018/10/04/magazine/barry-jenkins-james-baldwin-if-beale-street-could-talk.html}{}
\item
  \href{https://www.nytimes3xbfgragh.onion/interactive/2018/10/05/magazine/instagram-cindy-sherman-ugly-beauty.html}{}
\item
  \href{https://www.nytimes3xbfgragh.onion/interactive/2018/10/05/magazine/culture-moments-2018.html}{}
\end{itemize}

\hypertarget{related-coverage}{%
\subsection{Related Coverage}\label{related-coverage}}

\begin{itemize}
\tightlist
\item
  \href{https://www.nytimes3xbfgragh.onion/interactive/2018/10/03/magazine/lady-gaga-movie-star-is-born.html}{}
\item
  \href{https://www.nytimes3xbfgragh.onion/interactive/2018/10/03/magazine/morality-social-justice-art-entertainment.html}{}
\item
  \href{https://www.nytimes3xbfgragh.onion/interactive/2018/10/04/magazine/good-place-michael-schur-philosophy.html}{}
\item
  \href{https://www.nytimes3xbfgragh.onion/interactive/2018/10/04/magazine/barry-jenkins-james-baldwin-if-beale-street-could-talk.html}{}
\item
  \href{https://www.nytimes3xbfgragh.onion/interactive/2018/10/05/magazine/instagram-cindy-sherman-ugly-beauty.html}{}
\item
  \href{https://www.nytimes3xbfgragh.onion/interactive/2018/10/05/magazine/culture-moments-2018.html}{}
\end{itemize}

2018

\hypertarget{more-on-nytimescom}{%
\subsection{More on NYTimes.com}\label{more-on-nytimescom}}

Advertisement

\hypertarget{site-information-navigation}{%
\subsection{Site Information
Navigation}\label{site-information-navigation}}

\begin{itemize}
\tightlist
\item
  \href{https://help.nytimes3xbfgragh.onion/hc/en-us/articles/115014792127-Copyright-notice}{©
  2020 The New York Times Company}
\item
  \href{https://www.nytimes3xbfgragh.onion}{Home}
\item
  \href{https://www.nytimes3xbfgragh.onion/search/}{Search}
\item
  Accessibility concerns? Email us at
  \href{mailto:accessibility@NYTimes.com}{\nolinkurl{accessibility@NYTimes.com}}.
  We would love to hear from you.
\item
  \href{https://help.nytimes3xbfgragh.onion/hc/en-us/articles/115015385887-Contact-Us}{Contact
  Us}
\item
  \href{https://www.nytco.com/careers/}{Work with us}
\item
  \href{https://nytmediakit.com/}{Advertise}
\item
  \href{https://help.nytimes3xbfgragh.onion/hc/en-us/articles/115014892108-Privacy-policy\#pp}{Your
  Ad Choices}
\item
  \href{https://help.nytimes3xbfgragh.onion/hc/en-us/articles/115014892108-Privacy-policy}{Privacy}
\item
  \href{https://help.nytimes3xbfgragh.onion/hc/en-us/articles/115014893428-Terms-of-service}{Terms
  of Service}
\item
  \href{https://help.nytimes3xbfgragh.onion/hc/en-us/articles/115014893968-Terms-of-sale}{Terms
  of Sale}
\end{itemize}

\hypertarget{site-information-navigation-1}{%
\subsection{Site Information
Navigation}\label{site-information-navigation-1}}

\begin{itemize}
\tightlist
\item
  \href{https://spiderbites.nytimes3xbfgragh.onion}{Site Map}
\item
  \href{https://help.nytimes3xbfgragh.onion/hc/en-us}{Help}
\item
  \href{https://help.nytimes3xbfgragh.onion/hc/en-us/articles/115015385887-Contact-Us?redir=myacc}{Site
  Feedback}
\item
  \href{https://www.nytimes3xbfgragh.onion/subscription?campaignId=37WXW}{Subscriptions}
\end{itemize}
