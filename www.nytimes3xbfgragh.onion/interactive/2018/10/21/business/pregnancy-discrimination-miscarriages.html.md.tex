 **NYTimes.com no longer supports Internet Explorer 9 or earlier. Please
upgrade your browser.
\href{http://www.nytimes3xbfgragh.onion/content/help/site/ie9-support.html}{LEARN
MORE »}

**Sections

**Home

**Search

\hypertarget{the-new-york-times}{%
\subsection{\texorpdfstring{\href{http://www.nytimes3xbfgragh.onion/}{The
New York Times}}{The New York Times}}\label{the-new-york-times}}

\hypertarget{-business-}{%
\subsubsection{\texorpdfstring{ \href{/section/business}{Business}
}{ Business }}\label{-business-}}

 \href{/section/business}{Business} \textbar{}Miscarrying at Work: The
Physical Toll of Pregnancy Discrimination

**Close search

\hypertarget{site-search-navigation}{%
\subsection{Site Search Navigation}\label{site-search-navigation}}

Search NYTimes.com

**Clear this text input

Go

\url{https://nyti.ms/2R3GRMA}

\hypertarget{site-navigation}{%
\subsection{Site Navigation}\label{site-navigation}}

\hypertarget{site-mobile-navigation}{%
\subsection{Site Mobile Navigation}\label{site-mobile-navigation}}

\hypertarget{miscarrying-at-work-the-physical-toll-of-pregnancy-discrimination}{%
\section{Miscarrying at Work: The Physical Toll of Pregnancy
Discrimination}\label{miscarrying-at-work-the-physical-toll-of-pregnancy-discrimination}}

Women in strenuous jobs lost their pregnancies after employers denied
their requests for light duty, even ignoring doctors' notes, an
investigation by The New York Times has found.

\hypertarget{miscarrying-at-work-the-physical-toll-of-pregnancy-discrimination-1}{%
\section{Miscarrying at Work: The Physical Toll of Pregnancy
Discrimination}\label{miscarrying-at-work-the-physical-toll-of-pregnancy-discrimination-1}}

Women in strenuous jobs lost their pregnancies after employers denied
their requests for light duty, even ignoring doctors' notes, an
investigation by The New York Times has found.

By
\href{https://www.nytimes3xbfgragh.onion/by/jessica-silver-greenberg}{JESSICA
SILVER-GREENBERG} and
\href{https://www.nytimes3xbfgragh.onion/by/natalie-kitroeff}{NATALIE
KITROEFF} OCT. 21, 2018

MEMPHIS --- If you are a Verizon customer on the East Coast, odds are
good that your cellphone or tablet arrived by way of a beige, windowless
warehouse near Tennessee's border with Mississippi.

Inside, hundreds of workers, many of them women, lift and drag boxes
weighing up to 45 pounds, filled with iPhones and other gadgets. There
is no air-conditioning on the floor of the warehouse, which is owned and
operated by a contractor. Temperatures there can rise past 100 degrees.
Workers often faint, according to interviews with 20 current and former
employees.

One evening in January 2014, after eight hours of lifting, Erica Hayes
ran to the bathroom. Blood drenched her jeans.

She was 23 and in the second trimester of her first pregnancy. She had
spent much of the week hoisting the warehouse's largest boxes from one
conveyor belt to the next. Ever since she learned she was pregnant, she
had been begging her supervisor to let her work with lighter boxes, she
said in an interview. She said her boss repeatedly said no.

She fainted on her way out of the bathroom that day. The baby growing
inside of her, the one she had secretly hoped was a girl, was gone.

``It was the worst thing I have ever experienced in my life,'' Ms. Hayes
said.

Three other women in the warehouse also had miscarriages in 2014, when
it was owned by a contractor called New Breed Logistics. Later that
year, a larger company, XPO Logistics, bought New Breed and the
warehouse. The problems continued. Another woman miscarried there this
summer. Then, in August, Ceeadria Walker did, too.

The women had all asked for light duty. Three said they brought in
doctors' notes recommending less taxing workloads and shorter shifts.
They said supervisors disregarded the letters.

Pregnancy discrimination
\href{https://www.nytimes3xbfgragh.onion/interactive/2018/06/15/business/pregnancy-discrimination.html}{is
widespread in corporate America}. Some employers deny expecting mothers
promotions or pay raises; others fire them before they can take
maternity leave. But for women who work in physically demanding jobs,
pregnancy discrimination often can come with even higher stakes.

The New York Times reviewed thousands of pages of court and other public
records involving workers who said they had suffered miscarriages, gone
into premature labor or, in one case, had a stillborn baby after their
employers rejected their pleas for assistance --- a break from flipping
heavy mattresses, lugging large boxes and pushing loaded carts.

They worked at a hospital, a post office, an airport, a grocery store, a
prison, a fire department, a restaurant, a pharmaceutical company and
several hotels.

But refusing to accommodate pregnant women is often completely legal.
Under federal law, companies don't necessarily have to adjust pregnant
women's jobs, even when lighter work is available and their doctors send
letters urging a reprieve.

The Pregnancy Discrimination Act is the only federal law aimed at
protecting expecting mothers at work. It is
\href{https://www.congress.gov/bill/95th-congress/senate-bill/995}{four
paragraphs long} and 40 years old. It says that a company has to
accommodate pregnant workers' requests only if it is already doing so
for other employees who are ``similar in their ability or inability to
work.''

That means that companies that do not give anyone a break have no
obligation to do so for pregnant women. Employees say that is how the
warehouse's current owner, XPO Logistics, operates.

For example, last October, a 58-year-old woman died of cardiac arrest on
the warehouse floor after complaining to colleagues that she felt sick,
according to a police report and current and former XPO employees.
\href{https://www.facebookcorewwwi.onion/kandie.hicks.14/posts/108370139925270}{In
Facebook posts} at the time and in recent interviews, employees said
supervisors told them to keep working as the woman lay dead.

If companies ``treat their nonpregnant employees terribly, they have
every right to treat their pregnant employees terribly as well,'' said
Representative Jerrold Nadler, Democrat of New York, who has pushed for
stronger federal protections for expecting mothers.

In every congressional session since 2012, a group of lawmakers has
introduced a bill that would do for pregnant women what the Americans
With Disabilities Act does for disabled people: require employers to
accommodate those whose health depends on it. The legislation has never
had a hearing.

``We are deeply troubled by these allegations,'' said a Verizon
spokesman, Rich Young. ``We have no tolerance --- zero tolerance --- for
this sort of alleged behavior.'' He said the company opened an internal
investigation in response to The Times's inquiry. ``None of these
allegations are consistent with our values or the expectations and
demands of contractors that work directly for us or have any affiliation
with us.''

Erin Kurtz, an XPO spokeswoman, said: ``We're surprised by the
allegations of conduct that either predate XPO's acquisition of the
Memphis facility or weren't reported to management after we acquired it
in 2014.'' She said the allegations ``are unsubstantiated, filled with
inaccuracies and do not reflect the way in which our Memphis facility
operates.'' The company also disputed that the warehouse was windowless,
noting that there were a number of interior windows.

Ms. Kurtz said XPO prioritized the safety of its workers, had ``no
tolerance for any type of discriminatory behavior'' and has enhanced pay
and benefits for employees in recent years.

Those improvements didn't help Ceeadria Walker when she got pregnant.
The 19-year-old said she gave her XPO supervisor a doctor's letter from
OB/GYN Centers of Memphis saying she should not lift more than 15
pounds. She said she asked to be assigned to an area with lighter items.
Ms. Walker said her supervisor regularly sent her to a conveyor belt
line where she had to lift more than she was supposed to. She miscarried
the day after spending her shift handling those heavier boxes.

``We're saddened that Ms. Walker had a miscarriage over the summer,''
Ms. Kurtz said. ``We're investigating these newly raised claims.''

\hypertarget{the-risks-of-lifting}{%
\paragraph{The Risks of Lifting}\label{the-risks-of-lifting}}

For most women, it is safe to work while pregnant.

But there is ``a slight to modest increased risk of miscarriage'' for
women who do extensive lifting in their jobs,
\href{https://www.acog.org/Clinical-Guidance-and-Publications/Committee-Opinions/Committee-on-Obstetric-Practice/Employment-Considerations-During-Pregnancy-and-the-Postpartum-Period}{according
to guidelines published} this year by the American College of
Obstetricians and Gynecologists. The recommendations are intended to
inform doctors about best practices.

Two decades of medical research have established a link between
physically demanding work and fetal death, though there is debate about
how strong the connection is. Part of the difficulty in measuring the
relationship, researchers say, is that it's impossible to design a study
that isolates the impact of heavy lifting versus other risk factors,
like pre-existing conditions.

In a \href{https://www.ncbi.nlm.nih.gov/pubmed/23207454}{peer-reviewed
study from 2013}, researchers in Denmark found that the risk of fetal
death increased as women lifted heavier objects more frequently. The
researchers theorized that lifting and bending could reduce blood flow
to the uterus.

Another possibility, doctors said, is that extreme physical exertion
diverts blood from a woman's womb to her muscles.

The potential dangers are greatest for women whose pregnancies are
already classified as high risk, which is why doctors often advise that
they be given easier tasks.

``When employers ignore these medical recommendations, they are
potentially jeopardizing patients' health,'' said Rebecca Jackson, the
chief of obstetrics and gynecology at San Francisco General Hospital.
``It's especially bothersome to me that this is occurring for women in
strenuous jobs, given that they are at the most risk of injuring
themselves or the pregnancy.''

Warehouses are among the fastest growing workplaces in the country,
employing more than a million Americans. Retailers, competing against
the likes of Amazon, demand high speed at low costs.

On Memphis's east side, these are often the highest-paying jobs
available for people without college degrees. Drawn by the proximity to
rail lines and highways, some of the country's largest companies have
set up distribution centers here. One dispatches Nike shoes. Another
handles Disney toys. And a short drive from Graceland, Verizon has its
hub.

XPO runs all of those warehouses. The Verizon facility, which XPO took
over when it bought New Breed Logistics in 2014, is the only one where
The Times interviewed workers about pregnancy discrimination. Shifts
there can last 12 hours. Workers get 30 minutes for lunch and as many as
three other 15-minute breaks.

XPO's 2017 employee handbook warns that taking unapproved breaks,
arriving to work late or leaving early can result in ``immediate
termination,'' unless the reasons for the departures are ``legally
protected.'' The Pregnancy Discrimination Act does not guarantee women
such protections.

\hypertarget{a-job-on-the-line}{%
\paragraph{A Job on the Line}\label{a-job-on-the-line}}

It was the fall of 2013, and Erica Hayes was convinced that she was
having a girl. She daydreamed about the clothes she would buy and made a
list of favorite names. Her friend was pregnant with a boy, and they
talked about raising their children together.

At first, Ms. Hayes was processing individual shipments to Verizon
customers --- one phone, one charger, onto the next. Then, a crush of
holiday orders hit the warehouse in December. She said that her boss
began dispatching her to the area of the warehouse that handled bulk
shipments, often destined for Verizon stores, where the warehouse was
struggling to keep up. She often spent up to 12 hours a day lifting huge
boxes, some with 20 iPads and 20 accessories.

She said she could have handled paperwork or stayed in the section of
the warehouse devoted to small shipments. But she said her supervisor
kept ordering her to work with the largest boxes. Ms. Hayes's mother
said that her daughter talked to her about the rejected requests at the
time.

Ms. Hayes said she began to bleed regularly at work. She sometimes left
early to go to the hospital. Each time, she said, her supervisor wrote
her up. As the demerits accumulated, she stopped leaving. Instead, she
bled through four maxi pads a day.

``My job was on the line,'' she said. At the end of a long shift in
January 2014, she felt blood gushing into her jeans.

A co-worker fetched her a black peacoat to wrap around her waist to
cover the spreading stain. Another grabbed plastic bags to line the
leather driver's seat of her 2003 Hyundai. Ms. Hayes fainted before she
could get to the car. An ambulance took her to the hospital.

A couple of weeks later, she said, her supervisor handed her a \$300
invoice for the cost of the ambulance ride. (Ms. Hayes, who still works
at the warehouse and is hoping for a promotion, said she never paid the
bill.)

That spring, two more women had miscarriages at the warehouse. Both said
that their supervisors rejected their requests to pack lighter boxes.

One of the women, who still works at the warehouse, declined to be
identified for fear of losing her job.

The other was Tasha Murrell. She already had two boys and was praying
for a girl. She planned to name the baby Dallas, after the Cowboys, her
favorite football team. Ms. Murrell said that she told her boss she was
pregnant and asked to leave work early one day that spring because the
lifting had become painful.

Her supervisor told her to get an abortion, according to a
discrimination complaint she filed with the Equal Employment Opportunity
Commission in April 2018. Ms. Murrell woke up the next morning to find
her mattress stained with blood. Her husband drove her to the emergency
room, where doctors told her she had miscarried. Ms. Murrell left the
job last year and is now an organizer with the Teamsters, which is
trying to organize a union at the warehouse.

Ms. Hayes and Ms. Murrell had the same supervisor: Amela Bukvic. Through
her lawyer, Ms. Bukvic denied telling anyone to get an abortion. ``I
would never make such a horrible statement to anyone, especially an
employee under my supervision,'' she said.

Ms. Bukvic said that she made sure that the pregnant women whom she
managed had workloads that were not excessive. She said she never denied
help to the pregnant employees. ``If they had any work restrictions, I
always took all steps to make sure their work duties never exceeded
those restrictions,'' she said.

A few months later, in September 2014, it happened to another woman.

Chasisty Bee, 33, was four months pregnant. Hoping for a girl, she
bought a newborn's blanket from Burlington Coat Factory.

Ms. Bee had miscarried in 2008 while working at the Verizon warehouse.
This time, she said, she brought in a doctor's note recommending that
she work shorter shifts, be given a chair and light duty. Supervisors
rejected her requests. One afternoon, after almost 14 hours on her feet,
she started feeling dizzy and crumpled to the warehouse floor. Her
physician told her that she had miscarried.

After Ms. Bee got pregnant again in 2015, she found a new job. ``I
couldn't bear to lose another child,'' she said. The next February, she
gave birth to a healthy girl.

Ms. Kurtz, the XPO spokeswoman, said: ``The false and misleading
allegations directed at our Memphis facility are fueled by the Teamsters
and are part of their ongoing, but unsuccessful, attempts at
organizing.''

\hypertarget{stalled-momentum}{%
\paragraph{Stalled Momentum}\label{stalled-momentum}}

A bipartisan group of lawmakers has proposed upgrading the 1978
Pregnancy Discrimination Act. The bill would compel companies to
accommodate pregnant women --- for example, by offering extra breaks or
the option of light duty --- as long as it does not impose an ``undue
hardship'' on their business. That is the same language used in the
Americans With Disabilities Act.

Women ``shouldn't have to choose between keeping a doctor appointment or
their job,'' said Senator Dean Heller, Republican of Nevada, one of 125
co-sponsors of the Pregnant Workers Fairness Act in the House and
Senate.

In 2015, it looked as if the bill might gain traction. The Supreme Court
\href{https://www.nytimes3xbfgragh.onion/2015/03/26/us/ups-workers-pregnancy-discrimination-suit-reinstated-by-supreme-court.html}{had
just ruled} in favor of Peggy Young, a UPS driver who was denied light
duty after getting pregnant. Pregnancy discrimination was suddenly
grabbing headlines.

But some Republicans, including Senator Lamar Alexander of Tennessee,
where the XPO warehouse is, viewed that bill as adding a confusing new
layer of regulations, according to Senate aides. Mr. Alexander, who is
chairman of the Senate committee on health and labor, co-sponsored a
competing bill. It expanded protections for pregnant women in some
cases. But it still allowed employers to deny accommodations if they
weren't being provided to other workers in similar situations.

``It was a useful mechanism in order to divert some of the momentum that
was building,'' said Emily Martin of the National Women's Law Center, an
advocacy group focused on women's rights.

Both bills stalled.

Outside Washington, there have been fewer roadblocks. At least 23 states
have passed laws that are stronger than current federal protections
(Tennessee is not among them). In Utah, Delaware, Colorado and New York,
Republicans led the charge. In Nebraska, an anti-abortion Democrat
pushed the measure.

``Women have lost their children due to the lack of robust pregnancy
protections in the workplace,'' said Catherine Glenn Foster, the
president of Americans United for Life, an anti-abortion group. ``Anyone
who can't get behind this or uses it as a political game --- it's a
travesty.''

\hypertarget{i-watched-her-die}{%
\paragraph{`I Watched Her Die'}\label{i-watched-her-die}}

The problems extend beyond the warehouse floor --- to hotels,
restaurants, fire stations and stores.

At the Albertsons grocery store in Atascadero, Calif., Reyna Garcia had
one of the toughest jobs. She pushed 200-pound carts, dragged sacks of
cat litter and climbed 10-foot ladders to stock goods.

Ms. Garcia got pregnant in July 2012, found out she was having a girl
and decided to name the baby Jade.

Ms. Garcia told her boss that her pregnancy was high risk --- she had
previously given birth prematurely. She presented a doctor's note saying
she should not lift more than 15 pounds. The boss ignored the
recommendation, according to a lawsuit she filed against Albertsons in
federal court in Los Angeles.

``She was feeling like she wasn't getting any response from her
supervisor,'' her doctor, Mareeni Stanislaus, said in an interview. She
said the restrictions were ``even more important'' because Ms. Garcia
had a high-risk pregnancy. Heavy lifting can prompt smooth muscles like
the uterus to contract, potentially inducing preterm labor, Dr.
Stanislaus said.

The regular twisting and hoisting caused intense pain, but Ms. Garcia
needed the paycheck and the health insurance. She requested any other
position --- in the Albertsons bakery or at the meat counter or as a
fruit cutter or in the pharmacy or at the customer service desk.

Her boss turned her down, according to the lawsuit, which included
corroborating statements from her colleagues.

Her doctors sent two more notes. ``She should avoid prolonged standing
without a break and should avoid excessive bending and reaching and
balancing,'' Dr. Stanislaus wrote. Copies of the letters were included
in the suit.

About three weeks later, in the middle of her shift, Ms. Garcia began
feeling ``pelvic pressure,'' according to her lawsuit. She asked her
boss for permission to leave early; he gave her a long list of tasks
that she needed to finish first.

Ms. Garcia ended up working overtime. By the time she got home, she
could feel her amniotic sac bulging between her legs. It felt ``like a
balloon coming out of me,'' she said in a sworn statement. She went to
the emergency room. She could feel something scratching from inside; her
doctor told her it was the baby's fingernails. She was 20 weeks
pregnant.

Five days later, Ms. Garcia gave birth to Jade. The baby lived less than
10 minutes.

``My husband and I watched her die,'' Ms. Garcia said in her statement.
``It was the most painful thing I have ever experienced.''

In 2014, Albertsons settled Ms. Garcia's lawsuit for an undisclosed
amount; the deal prohibited her from speaking publicly about what
happened. ``The company has a policy against pregnancy discrimination,
and we accommodate employees with pregnancy-related disabilities in
accordance with state and federal law,'' said Christine Wilcox, an
Albertsons spokeswoman.

Ms. Garcia's boss demoted her when she returned to work.

``I lost my baby for this job,'' Ms. Garcia said, according to her
lawsuit. ``Why didn't you give me help when I was pregnant and asked for
it?''

\hypertarget{hard-to-breathe}{%
\paragraph{Hard to Breathe}\label{hard-to-breathe}}

Workers at the Memphis warehouse thought conditions might improve when
XPO Logistics acquired New Breed in September 2014.

At the time, XPO and its chief executive, Bradley S. Jacobs, were on a
buying spree. His strategy was simple: slash expenses and deliver quick
profits. Today XPO, based in Greenwich, Conn., has a market value of
\$12 billion.

The company has had multiple run-ins with regulators. In one instance,
the California Labor Commissioner's Office last year awarded four XPO
truck drivers a total of \$855,000 after finding that the company had
misclassified them as independent contractors.

In Memphis, XPO's ownership changed the warehouse, but not the way
workers had hoped. Over the next couple of years, pressure on employees
intensified and working conditions deteriorated, according to the 20
current and former employees. Supervisors began demanding that they pack
120 boxes an hour instead of 60. Some bosses penalized employees for
spending too much time in the bathroom or on breaks.

Summer temperatures inside the warehouse regularly exceeded 100 degrees,
the current and former employees said. It was so hot and humid that
workers sometimes found it hard to breathe.

Every week from June through September 2017, at least one worker
collapsed on the warehouse floor, the employees said. Supervisors took
to wheeling the sick workers to the break room, where they sometimes
received an ice pop.

On the morning of Oct. 17, 2017, Linda Neal crashed to the warehouse
floor, dead of cardiac arrest.

Ms. Neal, 58, had suffered from heart problems. She previously had
complained to her son, Dean Turner, that XPO supervisors would not let
her leave early when she was feeling unwell, Mr. Turner said in an
interview. That day, she had told managers that she was short of breath
and asked for an extra break, but her supervisor rejected the request,
according to Lakeisha Nelson, who witnessed the exchange and still works
there. When Mr. Turner arrived at the warehouse, he said, three of his
mother's colleagues told him the same thing.

Managers told workers to keep moving boxes as her body lay on the floor,
according to four employees who were at the warehouse that day, as well
as contemporaneous
\href{https://www.facebookcorewwwi.onion/kandie.hicks.14/posts/108370139925270}{Facebook}
\href{https://www.facebookcorewwwi.onion/kandie.hicks.14/posts/108360299926254?comment_id=108528973242720}{posts}.

Ms. Kurtz, the XPO spokeswoman, said the company did allow workers to
leave for the day after Ms. Neal died. She added that the Teamsters
``continue to shamefully exploit the passing of our colleague through
the spreading of false information'' in their effort to organize at the
warehouse.

That same month, Ceeadria Walker joined XPO. The pay was decent, about
\$12 an hour, an upgrade from her last job at a car-parts warehouse. At
first, she didn't mind the frequent 12-hour shifts. But she learned she
was pregnant in June 2018 and started feeling nauseated and dizzy at the
end of the long days.

Ms. Walker said she gave her supervisor a doctor's note saying she
should reduce the number of hours on her feet to avoid hurting herself
or endangering her pregnancy. ``They looked at it like it was nothing,''
she said.

There were days when XPO let her work in the ``pit,'' dealing with the
paperwork for orders, but she said she spent most of July on the
conveyor-belt line hoisting 45-pound boxes.

Ms. Walker toyed with leaving. But, she said, ``I couldn't just quit my
job.'' She was near the end of her first trimester and needed money. She
planned to name the baby, her first child, after her boyfriend ---
Octavia for a girl, Octavius for a boy.

By the end of her shift on Aug. 4, Ms. Walker said she had handled
hundreds of Verizon boxes. When she woke up bleeding at 6 a.m. the next
day, her mother drove her to the doctor.

Two days later, Ms. Walker spoke with The Times. She was distraught.
``This was going to be my first.''

Doris Burke contributed research.

\hypertarget{related-coverage}{%
\subsection{Related Coverage}\label{related-coverage}}

\begin{itemize}
\tightlist
\item
  \href{https://www.nytimes3xbfgragh.onion/interactive/2018/06/15/business/pregnancy-discrimination.html}{}
\end{itemize}

Doris Burke contributed research.

\hypertarget{more-on-nytimescom}{%
\subsection{More on NYTimes.com}\label{more-on-nytimescom}}

Advertisement

\hypertarget{site-information-navigation}{%
\subsection{Site Information
Navigation}\label{site-information-navigation}}

\begin{itemize}
\tightlist
\item
  \href{https://help.nytimes3xbfgragh.onion/hc/en-us/articles/115014792127-Copyright-notice}{©
  2020 The New York Times Company}
\item
  \href{https://www.nytimes3xbfgragh.onion}{Home}
\item
  \href{https://www.nytimes3xbfgragh.onion/search/}{Search}
\item
  Accessibility concerns? Email us at
  \href{mailto:accessibility@NYTimes.com}{\nolinkurl{accessibility@NYTimes.com}}.
  We would love to hear from you.
\item
  \href{https://help.nytimes3xbfgragh.onion/hc/en-us/articles/115015385887-Contact-Us}{Contact
  Us}
\item
  \href{https://www.nytco.com/careers/}{Work with us}
\item
  \href{https://nytmediakit.com/}{Advertise}
\item
  \href{https://help.nytimes3xbfgragh.onion/hc/en-us/articles/115014892108-Privacy-policy\#pp}{Your
  Ad Choices}
\item
  \href{https://help.nytimes3xbfgragh.onion/hc/en-us/articles/115014892108-Privacy-policy}{Privacy}
\item
  \href{https://help.nytimes3xbfgragh.onion/hc/en-us/articles/115014893428-Terms-of-service}{Terms
  of Service}
\item
  \href{https://help.nytimes3xbfgragh.onion/hc/en-us/articles/115014893968-Terms-of-sale}{Terms
  of Sale}
\end{itemize}

\hypertarget{site-information-navigation-1}{%
\subsection{Site Information
Navigation}\label{site-information-navigation-1}}

\begin{itemize}
\tightlist
\item
  \href{https://spiderbites.nytimes3xbfgragh.onion}{Site Map}
\item
  \href{https://help.nytimes3xbfgragh.onion/hc/en-us}{Help}
\item
  \href{https://help.nytimes3xbfgragh.onion/hc/en-us/articles/115015385887-Contact-Us?redir=myacc}{Site
  Feedback}
\item
  \href{https://www.nytimes3xbfgragh.onion/subscription?campaignId=37WXW}{Subscriptions}
\end{itemize}
