 **NYTimes.com no longer supports Internet Explorer 9 or earlier. Please
upgrade your browser.
\href{http://www.nytimes3xbfgragh.onion/content/help/site/ie9-support.html}{LEARN
MORE »}

**Sections

**Home

**Search

\hypertarget{the-new-york-times}{%
\subsection{\texorpdfstring{\href{http://www.nytimes3xbfgragh.onion/}{The
New York Times}}{The New York Times}}\label{the-new-york-times}}

\hypertarget{-magazine-}{%
\subsubsection{\texorpdfstring{
\href{https://www.nytimes3xbfgragh.onion/section/magazine}{Magazine}
}{ Magazine }}\label{-magazine-}}

 \href{https://www.nytimes3xbfgragh.onion/section/magazine}{Magazine}
\textbar{}The Top 25 Songs That Matter Right Now

**Close search

\hypertarget{site-search-navigation}{%
\subsection{Site Search Navigation}\label{site-search-navigation}}

Search NYTimes.com

**Clear this text input

Go

\url{https://nyti.ms/html\%3E}

\hypertarget{site-navigation}{%
\subsection{Site Navigation}\label{site-navigation}}

\hypertarget{site-mobile-navigation}{%
\subsection{Site Mobile Navigation}\label{site-mobile-navigation}}

\hypertarget{the-top-25-songs-that-matter-right-now}{%
\section{The Top 25 Songs That Matter Right
Now}\label{the-top-25-songs-that-matter-right-now}}

\hypertarget{--25-songs}{%
\subsubsection[ \protect\hyperlink{ux2f}{25
Songs}]{\texorpdfstring{\href{http://www.nytimes3xbfgragh.onion/magazine}{\protect\includegraphics{https://static01.graylady3jvrrxbe.onion/newsgraphics/2019/02/26/music/2654e69ed27287529e314df4aede0297716f8c07/icons/nytmag-square.svg}
\protect\includegraphics{https://static01.graylady3jvrrxbe.onion/newsgraphics/2019/02/26/music/2654e69ed27287529e314df4aede0297716f8c07/icons/nytmag-long.svg}}
\protect\hyperlink{ux2f}{25 Songs}}{  25 Songs}}\label{--25-songs}}

\begin{itemize}
\tightlist
\item
  \protect\hyperlink{ux2fbruce-springsteen}{1 `Born in the U.S.A.'
  (2018) Bruce Spring... }
  \href{https://open.spotify.com/track/6TGhSdj9CaXcxTHLjbNNYL}{\includegraphics{https://static01.graylady3jvrrxbe.onion/newsgraphics/2019/02/26/music/2654e69ed27287529e314df4aede0297716f8c07/icons/spotify-white.svg}Full
  Track}
\item
  \protect\hyperlink{ux2fariana-grande}{2 `Thank u, Next' Ariana Grande
  }
  \href{https://open.spotify.com/track/3e9HZxeyfWwjeyPAMmWSSQ}{\includegraphics{https://static01.graylady3jvrrxbe.onion/newsgraphics/2019/02/26/music/2654e69ed27287529e314df4aede0297716f8c07/icons/spotify-white.svg}Full
  Track}
\item
  \protect\hyperlink{ux2fmariah-carey}{3 `A No No' Mariah Carey }
  \href{https://open.spotify.com/track/7qg349cIMHC1syfpyQZtJ4}{\includegraphics{https://static01.graylady3jvrrxbe.onion/newsgraphics/2019/02/26/music/2654e69ed27287529e314df4aede0297716f8c07/icons/spotify-white.svg}Full
  Track}
\item
  \protect\hyperlink{ux2fmeek-mill}{4 `Trauma' Meek Mill }
  \href{https://open.spotify.com/track/45CCe4gu08OYG1I4MH8TU6}{\includegraphics{https://static01.graylady3jvrrxbe.onion/newsgraphics/2019/02/26/music/2654e69ed27287529e314df4aede0297716f8c07/icons/spotify-white.svg}Full
  Track}
\item
  \protect\hyperlink{ux2ftravis-scott}{5 `Sicko Mode' Travis Scott }
  \href{https://open.spotify.com/track/2xLMifQCjDGFmkHkpNLD9h}{\includegraphics{https://static01.graylady3jvrrxbe.onion/newsgraphics/2019/02/26/music/2654e69ed27287529e314df4aede0297716f8c07/icons/spotify-white.svg}Full
  Track}
\item
  \protect\hyperlink{ux2flady-gaga}{6 `Why Did You Do That?' Lady Gaga }
  \href{https://open.spotify.com/track/1sv0BhPPnrUgQ6dziZlmuM}{\includegraphics{https://static01.graylady3jvrrxbe.onion/newsgraphics/2019/02/26/music/2654e69ed27287529e314df4aede0297716f8c07/icons/spotify-white.svg}Full
  Track}
\item
  \protect\hyperlink{ux2fpinkfong}{7 `Baby Shark' Pinkfong }
  \href{https://open.spotify.com/track/5ygDXis42ncn6kYG14lEVG}{\includegraphics{https://static01.graylady3jvrrxbe.onion/newsgraphics/2019/02/26/music/2654e69ed27287529e314df4aede0297716f8c07/icons/spotify-white.svg}Full
  Track}
\item
  \protect\hyperlink{ux2fthe-carters}{8 `Apeshit' The Carters }
  \href{https://open.spotify.com/track/0E6PsO3ymCfUh7pJQjBgkj}{\includegraphics{https://static01.graylady3jvrrxbe.onion/newsgraphics/2019/02/26/music/2654e69ed27287529e314df4aede0297716f8c07/icons/spotify-white.svg}Full
  Track}
\item
  \protect\hyperlink{ux2fjuice-wrld}{9 `Lucid Dreams' Juice Wrld }
  \href{https://open.spotify.com/track/285pBltuF7vW8TeWk8hdRR}{\includegraphics{https://static01.graylady3jvrrxbe.onion/newsgraphics/2019/02/26/music/2654e69ed27287529e314df4aede0297716f8c07/icons/spotify-white.svg}Full
  Track}
\item
  \protect\hyperlink{ux2fkacey-musgraves}{10 `Slow Burn' Kacey Musgraves
  }
  \href{https://open.spotify.com/track/6ET9kf9riLETWs9lePUEAI}{\includegraphics{https://static01.graylady3jvrrxbe.onion/newsgraphics/2019/02/26/music/2654e69ed27287529e314df4aede0297716f8c07/icons/spotify-white.svg}Full
  Track}
\item
  \protect\hyperlink{ux2fpost-malone}{11 `Sunflower' Post Malone }
  \href{https://open.spotify.com/track/3KkXRkHbMCARz0aVfEt68P}{\includegraphics{https://static01.graylady3jvrrxbe.onion/newsgraphics/2019/02/26/music/2654e69ed27287529e314df4aede0297716f8c07/icons/spotify-white.svg}Full
  Track}
\item
  \protect\hyperlink{ux2ftierra-whack}{12 `Bugs Life' Tierra Whack }
  \href{https://open.spotify.com/track/4lLdrM9fpLA1Pcng9WNMLp}{\includegraphics{https://static01.graylady3jvrrxbe.onion/newsgraphics/2019/02/26/music/2654e69ed27287529e314df4aede0297716f8c07/icons/spotify-white.svg}Full
  Track}
\item
  \protect\hyperlink{ux2fparquet-courts}{13 `Before the Water ...
  Parquet Courts }
  \href{https://open.spotify.com/track/5wpBEy8gYVQmb5VArtayfp}{\includegraphics{https://static01.graylady3jvrrxbe.onion/newsgraphics/2019/02/26/music/2654e69ed27287529e314df4aede0297716f8c07/icons/spotify-white.svg}Full
  Track}
\item
  \protect\hyperlink{ux2fgrimes-feat-hana}{14 `We Appreciate Power'
  Grimes feat.... }
  \href{https://open.spotify.com/track/08ZCgkLKpcgrcSPCHN2v8J}{\includegraphics{https://static01.graylady3jvrrxbe.onion/newsgraphics/2019/02/26/music/2654e69ed27287529e314df4aede0297716f8c07/icons/spotify-white.svg}Full
  Track}
\item
  \protect\hyperlink{ux2fjulia-holter}{15 `I Shall Love 2' Julia Holter
  }
  \href{https://open.spotify.com/track/4VxGlr9PawFDaThx20USnS}{\includegraphics{https://static01.graylady3jvrrxbe.onion/newsgraphics/2019/02/26/music/2654e69ed27287529e314df4aede0297716f8c07/icons/spotify-white.svg}Full
  Track}
\item
  \protect\hyperlink{ux2fmaroon-5}{16 `Girls Like You' Maroon 5 }
  \href{https://open.spotify.com/track/6FRLCMO5TUHTexlWo8ym1W}{\includegraphics{https://static01.graylady3jvrrxbe.onion/newsgraphics/2019/02/26/music/2654e69ed27287529e314df4aede0297716f8c07/icons/spotify-white.svg}Full
  Track}
\item
  \protect\hyperlink{ux2fx6ix9ine}{17 `Stoopid' 6ix9ine }
  \href{https://open.spotify.com/track/6k2iq1yULhXILe0i8xQqPF}{\includegraphics{https://static01.graylady3jvrrxbe.onion/newsgraphics/2019/02/26/music/2654e69ed27287529e314df4aede0297716f8c07/icons/spotify-white.svg}Full
  Track}
\item
  \protect\hyperlink{ux2frobyn}{18 `Honey' Robyn }
  \href{https://open.spotify.com/track/3olYZQTYOLYNImERBnZZqm}{\includegraphics{https://static01.graylady3jvrrxbe.onion/newsgraphics/2019/02/26/music/2654e69ed27287529e314df4aede0297716f8c07/icons/spotify-white.svg}Full
  Track}
\item
  \protect\hyperlink{ux2frosala}{19 `Malamente' Rosalía }
  \href{https://open.spotify.com/track/1B0BQaSRHxhI0AUlItY3LK}{\includegraphics{https://static01.graylady3jvrrxbe.onion/newsgraphics/2019/02/26/music/2654e69ed27287529e314df4aede0297716f8c07/icons/spotify-white.svg}Full
  Track}
\item
  \protect\hyperlink{ux2fmarie-davidson}{20 `Work It' Marie Davidson }
  \href{https://open.spotify.com/track/5QnzAZ2W3Ap3gSiO2vsGQG}{\includegraphics{https://static01.graylady3jvrrxbe.onion/newsgraphics/2019/02/26/music/2654e69ed27287529e314df4aede0297716f8c07/icons/spotify-white.svg}Full
  Track}
\item
  \protect\hyperlink{ux2fjames-blake}{21 `Assume Form' James Blake }
  \href{https://open.spotify.com/track/6cJrQc34txUzXd90xWEAOh}{\includegraphics{https://static01.graylady3jvrrxbe.onion/newsgraphics/2019/02/26/music/2654e69ed27287529e314df4aede0297716f8c07/icons/spotify-white.svg}Full
  Track}
\item
  \protect\hyperlink{ux2fthe-1975}{22 `Love It If We Made It' The 1975 }
  \href{https://open.spotify.com/track/6WmIyn2fx1PKQ0XDpYj4VR}{\includegraphics{https://static01.graylady3jvrrxbe.onion/newsgraphics/2019/02/26/music/2654e69ed27287529e314df4aede0297716f8c07/icons/spotify-white.svg}Full
  Track}
\item
  \protect\hyperlink{ux2fweezer}{23 `Can't Knock the Hustle' Weezer }
  \href{https://open.spotify.com/track/3OSQcSWOOpBUGBPgCMKDdK}{\includegraphics{https://static01.graylady3jvrrxbe.onion/newsgraphics/2019/02/26/music/2654e69ed27287529e314df4aede0297716f8c07/icons/spotify-white.svg}Full
  Track}
\item
  \protect\hyperlink{ux2fsharon-van-etten}{24 `Comeback Kid' Sharon Van
  E... }
  \href{https://open.spotify.com/track/5nUVtL9EgUDCsjZT2Hcy9H}{\includegraphics{https://static01.graylady3jvrrxbe.onion/newsgraphics/2019/02/26/music/2654e69ed27287529e314df4aede0297716f8c07/icons/spotify-white.svg}Full
  Track}
\item
  \protect\hyperlink{ux2fsons-of-kemet}{25 `My Queen Is Harriet Tubman'
  Sons of Kemet }
  \href{https://open.spotify.com/track/4dSjQyjl3aHWJpk3swB0ig}{\includegraphics{https://static01.graylady3jvrrxbe.onion/newsgraphics/2019/02/26/music/2654e69ed27287529e314df4aede0297716f8c07/icons/spotify-white.svg}Full
  Track}
\end{itemize}

\href{https://open.spotify.com/user/nytmag/playlist/2IffJZke1mcKi7GttLi7JH?si=PWMrqjyTS-aUPFHW6AaarA}{\includegraphics{https://static01.graylady3jvrrxbe.onion/newsgraphics/2019/02/26/music/2654e69ed27287529e314df4aede0297716f8c07/icons/spotify_playlist-white.svg}}

\hypertarget{the}{%
\section{The}\label{the}}

Music\\
Issue\\
2019

\hypertarget{the-25}{%
\subsection{The 25}\label{the-25}}

songs that matter\\
right now

\includegraphics{https://static01.graylady3jvrrxbe.onion/newsgraphics/2019/02/26/music/2654e69ed27287529e314df4aede0297716f8c07/10mag-cover-desktop.png}
\includegraphics{https://static01.graylady3jvrrxbe.onion/newsgraphics/2019/02/26/music/2654e69ed27287529e314df4aede0297716f8c07/10mag-cover-mobile.png}

Photo illustration by Cristiana Couceiro

\hypertarget{introduction}{%
\subsubsection{Introduction}\label{introduction}}

By Nitsuh Abebe

It usually takes a while --- a decade or two --- before we can look back
at a particular era of American life and see it as something coherent,
something whose every aspect is marked by one overarching mood. It takes
a certain amount of hindsight to notice how all the wildly different
reactions people had to the moment were still, in the end, reactions to
the same thing; all the different poses they adopted were still being
struck against the same backdrop.

But \emph{this} era --- this year, and the last one, and one or two
before that --- might be an exception. There's an oddly strong
in-the-moment consensus on how everyone is feeling these days, and it is
not good. At some point it became a routine conversational tic for all
sorts of people, of all sorts of persuasions, to express, with an
incredulous gesture, that things feel a bit grueling and frantic lately,
don't they? Musicians are no exception. ``Life is pretty tumultuous
right now for all of us,'' said the crossover country star Kacey
Musgraves, while accepting a Grammy for the Album of the Year. The
Swedish singer Robyn acknowledges that ``pop at the moment is
depressing'' in an interview midway through this issue. ``The music kids
are listening to is heavy! Maybe it's hard to be positive and optimistic
at the moment.''

What's amazing is that the musical expression of all this isn't always
some big swing toward darkness, or anger, or anxiety. (Though there is,
in certain genres, plenty of all that as well.) Read through this list,
and what you'll often see instead is a very earnest, very serious desire
to find the right reaction to a world that feels tense and high-stakes
--- an ambient conviction that music should be looking for ways to cope,
ways to protect ourselves, moments of escape, hard reckonings with our
collective responsibilities, ideas for how to make the world feel less
brutal. The 25 songs and artists below include blockbuster hits,
critical darlings and inescapable conversation pieces, but few of them
take a direct route to the usual joys of pop --- the songs about dancing
and boasting and sex and love, the ones about what a fantastic night
everyone's about to have or what ecstasies they intend to find by the
end of it. No, a lot of these songs seem focused on deeper challenges:
How do we get to those joys in the first place? Who gets to have them,
and who deserves them? And in one case: Which of them are worth the
corresponding rise in sea levels?

The artists do not always sound thrilled about the circumstances. (``The
society we live in at the moment,'' Robyn says --- ``we didn't really
make it very good, you know?'') But they're a lot more motivated ---
whether it's to articulate something bleak or find their way toward
something better --- than you might expect.

\emph{Nitsuh Abebe is a story editor for the magazine.}

\emph{Photo illustration source photographs: Rosalía: Christian
Bertrand/Alamy. The 1975: Dafydd Owen/Newscom. Davidson: John
Londono/Ninja Tune. All other source photographs: Getty Images.}

Read more

Read with\\
Audio?

Yes

No

\includegraphics{https://static01.graylady3jvrrxbe.onion/images/2019/03/10/magazine/10mag-bruce-springsteen/10mag-bruce-springsteen-master180-v3.jpg}

\hypertarget{--born-in-the-usa-2018}{%
\subsection{\texorpdfstring{ `Born in the U.S.A.'
(2018)}{  `Born in the U.S.A.' (2018)}}\label{--born-in-the-usa-2018}}

\hypertarget{bruce-springsteen}{%
\paragraph{Bruce Springsteen}\label{bruce-springsteen}}

\hypertarget{01}{%
\subsection{01}\label{01}}

\includegraphics{https://static01.graylady3jvrrxbe.onion/images/2019/03/10/magazine/10mag-bruce-springsteen/10mag-bruce-springsteen-master180-v3.jpg}

How aging --- and the age --- can change a song's meaning.

By HANIF ABDURRAQIB

\href{https://open.spotify.com/track/6TGhSdj9CaXcxTHLjbNNYL}{\includegraphics{https://static01.graylady3jvrrxbe.onion/newsgraphics/2019/02/26/music/2654e69ed27287529e314df4aede0297716f8c07/icons/spotify.png}}

I don't know if Bruce Springsteen thinks about death as much as I think
about the inevitability of his dying. I've lived an entire life as a fan
of Bruce Springsteen, which means I have already imagined the world
without him in it, and I have mourned that world. If you're lucky enough
to age gracefully beyond a certain point, with that aging will come an
acceptance of finality or of the idea that there is going to be a
darkness from which you can't return. If you're lucky enough to have
made a life writing songs or stories or something at the intersection of
songs and stories, this could mean that there comes a point where you
make sure people hear you clearly, one last time, before you go.

I don't know if Springsteen himself thinks about his life and death in
this way, but the silences in
\href{https://www.nytimes3xbfgragh.onion/2017/10/12/theater/bruce-springsteen-on-broadway-review.html}{``Springsteen
on Broadway''} --- which ran on Broadway from Oct. 3, 2017, to Dec. 15,
2018, culminating in a soundtrack and a Netflix special --- suggest he
might. The spaces he built in between the songs allow the artist to
explain and give context not to just the music but also to the life
built around the music. If the project of Springsteen's Broadway show
was to attach histories and legacies to the individual songs long adored
by the public, there is also something to be said about what time does
to natural storytellers. They can become more tactile with age, drawing
out stories that have been told several times before the most current
retelling --- leaving a listener with even more touchable moments than
otherwise might have been asked for or sought, so that when the
storyteller is long gone, there might still be fragments of his or her
stories that span generations.

Of the many gripping examples of this in ``Springsteen on Broadway,''
the one that stands out most memorably is the sprawling story he tells
before playing his iconic (and often misconstrued song) ``Born in the
U.S.A.'' The story centers on Walter Cichon, who was the frontman for
the Motifs, a band Springsteen still considers one of the best rock 'n'
roll bands from the Jersey Shore. In the '60s, the Motifs played weekend
shows to rooms packed with teenage admirers. Cichon wore his hair long
and sported pointy black boots. When he performed, he would shake out
his hair and send beads of sweat flying past the stage lights. For
anyone who has ever lived in any town where a band was on the verge of
``making it,'' you know the epiphany: This band is too good to be here,
in this place, in this moment. That was the Motifs, with their frantic
and warbly guitars laid below Cichon's howling vocals.

Walter Cichon was drafted when he was 21 and didn't come back from
Vietnam. He went missing in action in 1968.

On Broadway, Bruce Springsteen performs ``Born in the U.S.A.'' largely
in silence. The song is half-spoken, half-sung, Springsteen's voice
rough and breaking beneath the decades of labor it has done --- labor
rendered romantic through writing and performance. What has always been
true about the career of Bruce Springsteen is that he's most
entertaining when backed by his pals, but he's most earnest when he's
alone. To hear ``Born in the U.S.A.'' presented without an instrument is
to hear the strain that pushes toward the edge of anger, that hovering
sentiment that was lost in the original's bombastic wall of sound and
perhaps camouflaged by its imagery. At the time of the song's release,
Springsteen was a young, attractive, muscular man who appeared midjump
in front of an American flag on the single's cover. From a zoomed-out
perspective --- a white musician writing about the intricacies of labor
--- it could seem as if he represented everything that a particular
America would be proud of. The misreading of the original song was not
purely accidental: Its volume and fanfare meant that it sounded (and
still sounds) good bursting out of speakers while fireworks explode in
the sky, and its loudest words in the chorus are about land and
birthright. But with the drums and bursts of keyboards gone, the
relentlessly hollow hope of the song is gone, too. On the isolated stage
of a theater, all that's left is knowing that the singer has loved and
dreamed and lost in a country sometimes not worth loving and dreaming
and losing in.

In his long monologue introducing ``Born in the U.S.A.'' on Broadway,
Springsteen talks about ``the blood and the confusion and the pride and
the shame and the grace that comes with birthplace,'' and I get it.
There are some of us who didn't ask to be born in our particular here,
and there are some of us who didn't ask to come to this particular here,
but to be in wherever your here is means that you might be compelled to
both fight for it and forgive it. On Broadway, Springsteen mentions
something else: He tells the story of him and two of his friends being
summoned to the selective-service office, as a prelude to being sent to
Vietnam --- for what, he says, ``we were sure was going to be our
funeral.'' They did everything they could to get out of being drafted,
and succeeded. He ends the story by exhaling softly and pausing before
telling the audience: ``I do sometimes wonder who went in my place.
Because somebody did.''

I imagine that's it. To live a long enough life in a place founded, in
part, on violence and volatility is to know that long life may depend on
someone else walking through a door you wanted no part of. Or to know
that the heroes from your hometown never made it out because war got to
them first. Stripped to its barest bones, ``Born in the U.S.A.'' asks a
listener to recognize that human survival is not something we can count
on. The song matters now in a different way than it did in 1984, largely
because of the artist behind it: Springsteen, trying to wrestle not only
with the song's current legacy but also with how it might be co-opted
decades from now, when he won't be around to make sure people understand
the ache behind the song's fury.

\emph{Hanif Abdurraqib is a poet, an essayist and a cultural critic in
Columbus, Ohio.}

Read more

\includegraphics{https://static01.graylady3jvrrxbe.onion/images/2019/03/10/magazine/10mag-ariana-grande/10mag-ariana-grande-master180-v4.jpg}

\hypertarget{--thank-u-next}{%
\subsection{\texorpdfstring{ `Thank u,
Next'}{  `Thank u, Next'}}\label{--thank-u-next}}

\hypertarget{ariana-grande}{%
\paragraph{Ariana Grande}\label{ariana-grande}}

\hypertarget{02}{%
\subsection{02}\label{02}}

\includegraphics{https://static01.graylady3jvrrxbe.onion/images/2019/03/10/magazine/10mag-ariana-grande/10mag-ariana-grande-master180-v4.jpg}

The Ambiguous kiss-off.

By AMANDA HESS

\href{https://open.spotify.com/track/3e9HZxeyfWwjeyPAMmWSSQ}{\includegraphics{https://static01.graylady3jvrrxbe.onion/newsgraphics/2019/02/26/music/2654e69ed27287529e314df4aede0297716f8c07/icons/spotify.png}}

`Thank u, next'' is styled like a tweet, which is how the phrase first
appeared --- as a cryptic rebuke that Ariana Grande thumbed off in the
comedian Pete Davidson's general direction a few days after their
breakup. Since then, the phrase has pulsed through the culture,
undulating between naughty and nice. When the song debuted, it was
revealed to be not a takedown but instead a sincere tribute to
relationships past, lifting Davidson (``for Pete, I'm so thankful'') to
the same plane as her ex Mac Miller, the rapper who died of a drug
overdose last year (``he was an angel''). But
\href{https://www.youtube.com/watch?v=gl1aHhXnN1k}{then the video hit},
and in it, Grande cosplayed as Regina George, the demon Barbie of ``Mean
Girls,'' scrawling notes about her exes in a burn book. But then all the
things she wrote about them were really nice! A quick shot in the video
shows the apology she scrawled on Davidson's page --- ``sry i dipped''
--- which was secretly the most savage note of all: For the record,
\emph{she} dumped \emph{him}.

Online, the phrase has bloomed into a deliciously ambiguous kiss-off, a
usage modeled by Grande herself, directed toward anyone from a no-name
rapper who covered the song to Piers Morgan, who criticized pop stars
for appearing in revealing photo shoots. Like the Southern ``bless your
heart,'' the passive-aggressive niceties that sustain the entire Midwest
or the chill of the British stiff upper lip, the internet has found a
discreet slight of its own in ``thank u, next.''

The phrase is breezily transactional. It renders news anchors and
ex-boyfriends into strangers, reducing them to the base level of
politeness required by the social contract. It sharpens respect into a
shiv. And yet it is vexing in its restraint, pre-emptively silencing any
retaliatory efforts.

We are living in a time of great pettiness. A big star can grow two
sizes by doing something very small. Bhad Bhabie chucked a drink at Iggy
Azalea and cemented her status as a memetic folk hero. Pusha T lobbed a
literal baby into the middle of his rap war with Drake. Grande's grace
could, in one reading, signal a rejection of all of this calculated
cruelty, a classy subversion of the fan-and-industry expectation that a
celebrity may only rise by tearing another down.

But it's also a savvy strategy for advancing the game. The next level of
beef is always the high road --- ascending to that rarefied realm of
conflict where put-downs are joined seamlessly with self-respect. In
``thank u, next,'' Grande casts her exes as steppingstones on her path
to greatness, men who taught her ``patience,'' ``pain'' and, ultimately,
how little she needed them. There's nothing harsher than having your
relationship converted into a \emph{learning experience}. And the worst
part is, you know she's right. The secret message of ``thank u, next''
is that women can dismiss men who sap their energies and undermine their
success, and this is not an act of cruelty or a symptom of bitchiness.
It's a simple social nicety.

\emph{Amanda Hess is a critic at large for The Times. She writes about
internet culture for the Arts section.}

Read more

\includegraphics{https://static01.graylady3jvrrxbe.onion/images/2019/03/10/magazine/10mag-mariah-carey/10mag-mariah-carey-master180-v5.jpg}

\hypertarget{--a-no-no}{%
\subsection{\texorpdfstring{ `A No No'}{  `A No No'}}\label{--a-no-no}}

\hypertarget{mariah-carey}{%
\paragraph{Mariah Carey}\label{mariah-carey}}

\hypertarget{03}{%
\subsection{03}\label{03}}

\includegraphics{https://static01.graylady3jvrrxbe.onion/images/2019/03/10/magazine/10mag-mariah-carey/10mag-mariah-carey-master180-v5.jpg}

The new R.\&B. litmus test requires artists to be accessible, enigmatic
and maybe a little bit flawed. Sanging is optional.

By ANGELA FLOURNOY

\href{https://open.spotify.com/track/7qg349cIMHC1syfpyQZtJ4}{\includegraphics{https://static01.graylady3jvrrxbe.onion/newsgraphics/2019/02/26/music/2654e69ed27287529e314df4aede0297716f8c07/icons/spotify.png}}

From the vocal singularity embodied in Aretha Franklin to the
otherworldly dance moves of Michael Jackson, black folks have long
expected rigor from our R.\&B. entertainers. (We institutionalized this
expectation at the Apollo Theater, where, on Amateur Night, an
``executioner'' used to chase away mediocre performers midact with a
broom or a pitchfork; now he just dances them off the stage.) Being the
best in R.\&B. meant that you had a honed, real-deal ability to
entertain, that you could stand on a stage and perform a remarkable act
that separated you from the rest of us.

I grew up hearing debates about the worthiness of this or that singer
devolve into shouting matches, the assertion that a favored artist could
sing but not really \emph{sang} being an affront to your system of taste
and judgment. Mariah Carey was always an easy win. In a single verse,
her melismatic contralto might argue with her teasing falsetto,
alternating between lower and higher notes until she sounded more bird
than human. In the '90s, girl-next-door Mariah belted out uptempo love
songs on one station; a few turns of the dial over, she sang the same
record with the addition of a hip-hop verse, thereby sliding from
mainstream to ``urban,'' which is to say from white to black. She
straddled two coded worlds as a biracial person, a sometimes-fraught
experience that she addressed on her 1997 song ``Outside.''

Her desire for cross-genre acceptance is part of what pushed her to
write and arrange songs for herself that few other human beings could
cover. Through the mid-2000s, to listen to a Mariah album, from lead
single to deep cut, was to marvel at a maximalist pulling off her
excesses, every run more dazzling than the last. And yet by the end of
the aughts, she had begun receding behind her production, talk-singing
and whispering where she used to overaccentuate each phrase. The rumored
loss of her voice seemed to mark the end of an era altogether.

Few people argue over the voice of a singer the way they used to, but
R.\&B. is back in vogue after having spent several years in the
background as E.D.M. dictated pop's music imperatives. Younger artists
are pushing the genre forward in many respects: intriguing, pared-down
personal style (SZA); forceful, nuanced messaging (Solange); swaggering
vocals that don't feel yoked to the rap feature (H.E.R.); and a
much-needed expansion of whom a woman might be singing love songs to in
the first place (Syd). But the terms by which we expect rigor from these
artists have changed, too. A voice that sounds as though it were gifted
from the heavens is no longer a likely predictor of critical success,
though it may garner you a stint on ``The Voice.'' More important is
appearing unvarnished, or idiosyncratic. Contemporary R.\&B. seems to
prefer the D.I.Y. ethos of indie rock or riffing on the earthy,
unadorned feel of neo-soul. The preferred feel is that of a raw
outpouring of emotion alone in a bedroom with a laptop. To see R.\&B.'s
newest names perform at events like the Grammys is a bit like seeing
your high school gym teacher on a date --- who even knew they owned
fancy clothes?

Mariah Carey doesn't seem built for this new R.\&B. moment, but ``A No
No,'' from her 2018 album ``Caution,'' works for all the reasons we
would once expect it not to. It's a straightforward sample of an
extremely familiar song (a remix of Lil' Kim's 1997 ``Crush on You'').
Vocally, it's uncomplicated. The track has a few elastic moments at the
top of verses, but for the most part, Carey maintains a syncopated,
crooning sing-speak. She comes down from the vocal stratosphere to some
place closer to the younger R.\&B. chanteuses, but it never feels like a
cop-out. ``Caution'' as a whole forgoes Carey's hallmark vocal
pyrotechnics, save for a few whistle tones that creep into the final
notes of several tracks. What makes it different from her previous
attempts at less ornate vocal arrangement is the confidence Carey
exudes. She isn't hiding; she's recalibrating.

This new phase of R.\&B. is one for which Mariah the songwriter is well
suited. She has always been a quick study of current trends, and as a
writer on 17 of her 18 No.1 singles, she proved that she knew how to
make her voice fit within them. Carey possesses a mischievous sense of
humor (best employed on Eminem diss tracks) that is fit for our current
age of trolling and lyrics made for memes. In ``A No No'' she drawls out
the line ``Irregardless of what transpired,'' daring the listener to
think too hard about whether she knows that she has deployed a fake word
(of course she does). Over the past three decades, Mariah the vocalist
has been so singular that other Mariahs went overlooked --- the canny
recognizer of trends, the pop star who pushed her label to make unlikely
hip-hop collaborations happen and the songwriter who was funnier than
people understood. Mariah, queen of glitter and lover of glamour, might
never pull off a down-to-earth visual aesthetic, but she still possesses
the tools to make music that embodies that feeling --- and she has had
these tools for years.

\emph{Angela Flournoy is the author of ``The Turner House.''}

Read more

\includegraphics{https://static01.graylady3jvrrxbe.onion/images/2019/03/10/magazine/10mag-music-slideshow-slide-99OH/10mag-music-slideshow-slide-99OH-master180-v2.png}

\hypertarget{--trauma}{%
\subsection{\texorpdfstring{ `Trauma'}{  `Trauma'}}\label{--trauma}}

\hypertarget{meek-mill}{%
\paragraph{Meek Mill}\label{meek-mill}}

\hypertarget{04}{%
\subsection{04}\label{04}}

\includegraphics{https://static01.graylady3jvrrxbe.onion/images/2019/03/10/magazine/10mag-music-slideshow-slide-99OH/10mag-music-slideshow-slide-99OH-master180-v2.png}

He's been rapping about it for years. Now a lot more people are
listening.

Interview by NIKOLE HANNAH-JONES

\href{https://open.spotify.com/track/45CCe4gu08OYG1I4MH8TU6}{\includegraphics{https://static01.graylady3jvrrxbe.onion/newsgraphics/2019/02/26/music/2654e69ed27287529e314df4aede0297716f8c07/icons/spotify.png}}

Last year, in a span of months, Meek Mill went from solitary confinement
in a Pennsylvania prison to releasing an album that debuted at No.1 on
the Billboard chart. It's a paradoxical narrative that has defined the
Philadelphia rapper since his teenage years, when he landed both a
record deal and criminal convictions on drug and gun charges. Sentenced
at 21 to prison and years of probation, the 31-year-old rapper has spent
his entire adult life in and out of courts and prison, often for
noncriminal violations like not adequately reporting his travel plans.

But 2018 proved transformative. New scrutiny of the judge overseeing his
probation, and of the initial case --- he has always maintained that the
police made up charges --- roused public support from powerful people,
including Jay-Z, Philadelphia's district attorney and the Philadelphia
76ers co-owner Michael Rubin. With tour dates for his ``Championships''
album selling out, Mill has become a leading voice on the failings of
the criminal-justice system. Along with
\href{https://www.nytimes3xbfgragh.onion/2018/11/26/opinion/meek-mill-criminal-justice-reform.html}{writing
an op-ed for The New York Times} and appearing on national news shows,
the rapper helped start the Reform Alliance, an organization dedicated
to getting one million Americans out of the prison system.

\emph{How long did you work on the album?} Probably eight months. Since
I came out.

\emph{I usually take about eight months to produce a piece as well. Then
there's panic --- you've gone over it so many times, you can't even tell
anymore if it's good or not.} I feel the same way. Going through the
music so much, not even knowing if it's good anymore --- I done heard it
at least a thousand times, trying to analyze and look at it so many
ways.

\emph{At the end of ``Championships,'' you say we had to beat the
streets, beat racism, beat poverty. So much of what you write is
sociological, a study of the neighborhood. Your being from Philadelphia
made me think about W.E.B. Du Bois in 1899 --- he did a study called
``The Philadelphia Negro.''} Yeah, I read pieces of it in prison. One of
the older guys probably gave it to me.

\emph{Then you know what I'm talking about. This was known as the first
sociological study of black Americans in the country. He was trying to
understand why black folks lived the way they lived. And the social
problems he identified --- poverty, crime, illiteracy, white
discrimination --- are the exact same things you talk about 120 years
later.} That was my life coming up, so it was normal. I always tell
people, I've been living the life I'm living now for the last eight
years, but I've been living in the ghetto for 23. Things we've been
traumatized by our whole lives --- we have a right to talk about it.

\includegraphics{https://static01.graylady3jvrrxbe.onion/images/2019/03/10/magazine/10mag-music-slideshow-slide-HTUR/10mag-music-slideshow-slide-HTUR-master180-v2.png}

\includegraphics{https://static01.graylady3jvrrxbe.onion/images/2019/03/10/magazine/10mag-music-slideshow-slide-RLVZ/10mag-music-slideshow-slide-RLVZ-master180-v2.png}

\emph{You refer to school a couple of times on this album --- your mom
praying you'd go to Yale, or going to a school with bullet holes in the
lockers. You said you were on the honor roll?} I used to be, until,
like, third grade. There was another book I got in prison, about black
kids --- the fourth grade, things go wrong and grades start to decline.
That was my life. Mine was because I moved to a different neighborhood,
rougher than the one I came from.

\emph{One of my favorite quotes from James Baldwin is ``The kids had
been told that they weren't worth {[}expletive{]}, and everything they
saw around them proved it.'' What did your schools say about what
society thought of you?} We used to have teachers say, ``Oh, you're
probably going to be dead or in jail, or you're probably going to be a
failure.'' I had some good teachers, too, but it was rough in our
schools.

\emph{I feel like that's a message we start telling black boys very
early: You're going to be dead or in jail. I know we're trying to warn
them, but it also seems as if you're dictating their future.} Yeah, but
I never believed that. I used to say I was going to be a normal story of
the ghetto. But if you have your mom telling you you're going to be a
great person and then your teachers talk to you like that --- you
wouldn't really want to listen to that person anymore.

\includegraphics{https://static01.graylady3jvrrxbe.onion/images/2019/03/10/magazine/10mag-music-slideshow-slide-TANQ/10mag-music-slideshow-slide-TANQ-master180-v2.png}

\emph{Was school easy for you?} Pretty easy. I didn't have to study to
pass my tests. Even if I ran the hallways, I would still be fairly good.
Later on, when I really stopped trying, I was put in disciplinary
schools. It was like a jail. You get strip-searched before you go in,
fingerprinted every day. One day I just climbed over the gate and left.

\emph{It was a public school organized like a jail? In other words, it
was early conditioning for what everybody assumed your future was going
to be.} When I finally went to jail, I already knew everybody. Everybody
I went to school with was in the jail.

\emph{What were you put in that school for?} Fighting and acting up. I
said in one of my raps, I was acting up in school because I thought it
was cool, but really I was hurt. Your mom's at work, your dad's in the
graveyard, you're not really getting nurtured, and I guess you just
resort to acting up.

\emph{You have a song called ``Trauma,'' which talks about that, the
PTSD young people in these neighborhoods are dealing with. But that's
why I love that Baldwin quote --- everything about the environment tells
these kids they're nothing, right? But when they react in what's
actually a very normal way, we want to break them down.} You don't value
yourself because you grow up not being valued. What type of motivation
do you get if your mom is on drugs? Your self-esteem is automatically
just low. Some people have the determination to shoot to the top. But,
you know, that's not normal --- for just a regular kid to have the
determination to do that.

\includegraphics{https://static01.graylady3jvrrxbe.onion/images/2019/03/10/magazine/10mag-music-slideshow-slide-3ICQ/10mag-music-slideshow-slide-3ICQ-master180.png}

\emph{I always say anger is an easier emotion to deal with than pain.}
Yeah. That's why I say, you come through neighborhoods and you see kids
with their faces all balled up or looking angry? They're probably really
angry. His mom is probably really on drugs, and he probably really don't
have food in the house, and his dad is probably really dead. They hurt,
they torn, they scarred. It's nothing that words can really fix. ``Why
is this kid always angry?'' Well, he's been through something.

\emph{Who's the first black writer you ever read?} I'd never be able to
determine who was black and white when I was reading back then. I don't
know the first. ``The New Jim Crow'' --- that's a black writer?

\emph{Yes. Michelle Alexander. When you were in high school, you weren't
exposed to black literature, black writers?} I went to public school.
The books were falling apart. They probably still got the same books
from when I was in school. We didn't get no black literature in public
school.

\emph{I read a lot as a child, mostly because I was grounded all the
time. Then we had a black-studies course in high school, and I became
obsessed with black history because it felt like, for the first time,
the world made sense. You would see your community and how people lived,
and they would tell you we just did not want better. But I could see how
hard people worked, and they still could not get ahead. Studying history
calmed me.} The most I ever read was in prison. There's nothing to do,
so you turn to books. Reading made me process the system. Sometimes I
thought it was just millions of black people, and Spanish --- when you
come to prison, it's black and Spanish. Looking deeper, and seeing the
way some of those things are broken down in
\href{https://www.nytimes3xbfgragh.onion/2018/01/18/us/new-jim-crow-book-ban-prison.html}{``The
New Jim Crow,''} it made me really wonder. Because I am already a
conspiracy theorist.

\emph{I don't understand how you can be black and not be a conspiracy
theorist.} I was doing a show on CNN, and a guy was like, ``Why are you
saying the system is similar to slavery?'' It kind of caught me off
guard, and I couldn't really answer. But you've got people working for
eight cents in here. You're feeding people the {[}expletive{]} you would
feed slaves. You've got people cuffed up, shackled from top to bottom.
People locked in the basement for 23 hours a day, being beat by the
officers. It's basically the same. And one of the amendments says that
when you're under custody of the government, you can be treated like a
slave.

\emph{Yes, the 13th. So you were first arrested --- for the original
charge --- at 19?} My first arrest was actually going to school. In
sixth or seventh grade. I was suspended, and I didn't want to tell my
mom, so I tried to hang out in the hallways. I got caught and went to
jail for trespassing. My mom had to come get me.

\includegraphics{https://static01.graylady3jvrrxbe.onion/images/2019/03/10/magazine/10mag-music-slideshow-slide-FA7X/10mag-music-slideshow-slide-FA7X-master180-v2.png}

\emph{What's the charge you're still on probation for?} Selling crack.
This is to my dad's soul: I wasn't selling crack when they locked me up
for it. When I got back, I had to get back in the street and start
really selling weed to get me a lawyer, because everyone who had a
public defender got crucified. My mug shot has my face swollen, both
sides of my face beat up. I was charged with fracturing the cop's hand.
You know how his hand got fractured?

\emph{Yeah, punching you}. Yeah. He charged me for him punching my face.
They said I pointed a gun at them. If somebody can sit there and tell
you the story of how they didn't point a gun at cops, nine times out of
10, it's true. It's not too many people who can make it out of pointing
a gun at a couple cops.

I remember one time, this judge said, ``I don't give people three to six
months; I give people three to six years'' --- for something like a
first-time weed charge. That always stuck with me. That's not O.K. I
mean, you can't shoot nobody and expect to be getting chances. But if
you were on probation and began smoking weed? People in the 'hood are
going through real {[}expletive{]}. I barely sleep from so much trauma.
Sometimes you just want to smoke and go to sleep.

And you've got to think about it --- you've got 18-, 19-year-old kids,
sending them to a whole building full of rapists, killers. They put you
in a cell with a 45-year-old man who got a life sentence, who's a
killer. This is your dad right here. He's going to raise you. He's going
to show you how to clean the toilet, how to carry yourself. They're not
thinking about that. They're just giving this guy three years to get
taught this mentality.

\emph{My mom was a probation officer. She would tell me how certain
officers would wait outside the person's house, trying to catch them.
And it's for noncriminal things, right? The average person, I think,
believes that people are being violated because they're doing criminal
activity. But I've had family and friends incarcerated, and part of
their probation would be that they couldn't drive. In a place with no
public transit, they would drive to work and get violated. You can't
associate with known felons, but that means you can't be around your
family members or go to the barbershop.} You can't associate with felons
when you just came from prison, with a thousand felons in your face
every day. That makes no sense. One time the judge was like, ``This is
lenient,'' and in my head --- I couldn't say this --- I was like, Who
are you to even say this is lenient? If you gave me three months, that
is not lenient. I'm going to lose my job, lose the lease on my house.
She made it a condition that I couldn't even rap.

\emph{In prison, you were 23-and-1, right? Isolated all but one hour a
day? I can't even imagine what that's like.} Nobody can. I was just
talking to 21 Savage in prison, and I was like, This is the closest
you've ever been to God in a room like this. The last time I was on
23-and-1, my lawyer came to see me, and I was like, ``Did I go crazy and
just don't know I went crazy?'' I started writing everything I was
feeling, but when I went back and looked at it, I was spelling
everything wrong, things I know how to spell. I kept blacking out in the
middle of the day --- not passing out, but like falling asleep. I was
counting the birds on the wire: This bird's gonna fly off in 10, 9, 8,
7. The bird don't fly off, start over. Twenty-three and a half hours a
day. Come out to take a shower, back to your cell. And I wasn't in there
for punishment --- they had me on a mental block because I'm a
celebrity, and they didn't have anywhere to put me. Every time I got
out, I'm like, yeah, I'm not the same no more.

\includegraphics{https://static01.graylady3jvrrxbe.onion/images/2019/03/10/magazine/10mag-music-slideshow-slide-5NIP/10mag-music-slideshow-slide-5NIP-master180-v2.png}

\emph{You've become the face of criminal-justice problems. When did you
realize that you had a platform, and that you should use it to advocate
for more than yourself?} When I saw the support people gave me.
Everybody was saying ``Free Meek,'' but it was really like saying ``Free
everybody who goes through these conditions.'' It surprised me, because
I've been in and out of prison for these types of things for the past 11
years, and people were just like: ``Aw, he's stupid. He keeps going to
jail.'' I am not stupid. It's just normal mistakes. I caught one case at
the age of 19. I am 31. I have never been back to prison for a crime.

\emph{You're working with some powerful people in the Reform Alliance.
Jay-Z, Michael Rubin, Robert Kraft, Clara Wu Tsai.} Basically all
billionaires except me. Robert Kraft saw me in prison, and he was like,
How are you still smiling? He was like, If that was me, I would be
depressed, mad, angry.

\emph{You're not? You can't sleep.} Yeah, but that comes from my
environment. Coming from seeing violence, people robbed, people
murdered, you heard gunshots every night --- couldn't sleep.

\emph{So how do you deal with the trauma?} I just override it. I don't
know. Rapping is one of my therapies. I've never been a dweller to sit
back and be sad about something. The saddest thing I can think about is
Lil Snupe, an artist I had signed, got killed at 18 by a grown man. That
bothered me a lot for two years, but I suppressed it and never really
addressed it. Then one day, I started realizing that had damaged me, and
I thought about it a lot.

\emph{Do you actually think Reform Alliance can change the system?} I
think it's a possibility we can make a change. In Pennsylvania, they're
talking about changing the laws of probation already, to where the cap
is five years. That will be a big win for a lot of kids who will enter
the system and probably would have gotten 10 to 20.

\emph{After everything, you're still hopeful.} Yeah. Hell, yeah. I got a
mean team with me. I don't think none of us lose in anything we do.

\emph{Nikole Hannah-Jones is a staff writer for the magazine.}

\emph{Devin Yalkin is a photographer from New York who has covered a
range of subjects for the magazine, including
\href{https://www.nytimes3xbfgragh.onion/2018/10/18/magazine/dirt-track-racing-nascar-tony-stewart.html}{dirt-track
racing},
\href{https://www.nytimes3xbfgragh.onion/2016/04/17/magazine/donald-trump-american-preacher.html}{Trump
rallies} and
\href{https://www.nytimes3xbfgragh.onion/2018/04/04/magazine/arlee-warriors-montana-basketball-flathead-indian-reservation.html}{basketball
on Montana's Flathead Indian Reservation}.}

\emph{This interview has been condensed and edited.}

Read more

\includegraphics{https://static01.graylady3jvrrxbe.onion/images/2019/03/10/magazine/10mag-travis-scott/10mag-travis-scott-master180-v8.jpg}

\hypertarget{--sicko-mode}{%
\subsection{\texorpdfstring{ `Sicko
Mode'}{  `Sicko Mode'}}\label{--sicko-mode}}

\hypertarget{travis-scott}{%
\paragraph{Travis Scott}\label{travis-scott}}

\hypertarget{05}{%
\subsection{05}\label{05}}

\includegraphics{https://static01.graylady3jvrrxbe.onion/images/2019/03/10/magazine/10mag-travis-scott/10mag-travis-scott-master180-v8.jpg}

How dozens of people own a slice of a hit.

By JONAH WEINER

\href{https://open.spotify.com/track/2xLMifQCjDGFmkHkpNLD9h}{\includegraphics{https://static01.graylady3jvrrxbe.onion/newsgraphics/2019/02/26/music/2654e69ed27287529e314df4aede0297716f8c07/icons/spotify.png}}

How many people does it take to write a No. 1 hit? In the case of Travis
Scott's smash ``Sicko Mode,'' which came out Aug. 3, 2018, and has been
on Billboard's Hot 100 chart every week since, 30 different songwriters
are credited. That's a preposterously high number that speaks, in part,
to the shift in top-tier rap toward a collaboration-heavy, auteurist
mode of creation. It also speaks to the song's unorthodox structure.
``Sicko Mode,'' which Scott performed at the Super Bowl halftime show,
switches between three different beats created by six different
producers, with additional work, it turns out, from a Switzerland-based
washing-machine salesman. In addition to three guest vocalists --- one
of whom is the Canadian superstar Drake, one of whom utters only three
words and one of whom has been dead for 13 years --- it includes two
vocals sampled from landmark rap songs that are themselves dense with
further samples from '70s-era funk bands and, for good measure, a
handful of other landmark rap songs. The DNA of ``Sicko Mode,'' that is,
carries with it strands of ancient genetic material, and even if much of
it is audible nowhere in the song itself, those strands show up in the
credits. Here's where each one leads.

\includegraphics{https://static01.graylady3jvrrxbe.onion/images/2019/03/10/magazine/10mag-scott-slideshow-slide-TJN5/10mag-scott-slideshow-slide-TJN5-master180.png}

1.

\includegraphics{https://static01.graylady3jvrrxbe.onion/images/2019/03/10/magazine/10mag-scott-slideshow-slide-8IRP/10mag-scott-slideshow-slide-8IRP-master180.png}

1.

\emph{00:00 --- 01:00}

\emph{00:00 --- 01:00}

1. The woozy organ riff in the song's first section was created by the
producer and classically trained pianist \textbf{Rogét Chahayed,} whose
breakthrough track was D.R.A.M.'s 2016 single ``Broccoli.'' The bass
line and percussion come courtesy of \textbf{Chauncey Hollis,} better
known as Hit-Boy, whose most famous co-production is Jay-Z and Kanye
West's ``Niggas in Paris.''

\includegraphics{https://static01.graylady3jvrrxbe.onion/images/2019/03/10/magazine/10mag-scott-slideshow-slide-Y7EU/10mag-scott-slideshow-slide-Y7EU-master180.png}

2.

\emph{00:28 --- 01:00}

\emph{00:28 --- 01:00}

2. \textbf{Aubrey Graham,} a.k.a. Drake, is the first voice we hear,
though his verse will be abruptly cut off. Chahayed has explained that
Drake and Travis Scott recorded a complete song over his and Hit-Boy's
beat, but in this Frankenstein version, we hear only about a minute of
it.

\includegraphics{https://static01.graylady3jvrrxbe.onion/images/2019/03/10/magazine/10mag-scott-slideshow-slide-U66M/10mag-scott-slideshow-slide-U66M-master180.png}

3.

\emph{01:00} \emph{--- 02:48}

\emph{01:00} \emph{--- 02:48}

3. Travis Scott, born \textbf{Jacques Webster,} doesn't appear on his
own song until a minute in, when the music changes entirely --- one of
the track's many idiosyncrasies. Scott, asked if he could hear the
song's chart-topping potential before its release, says: ``Yeah, but
other people weren't hearing it. They're looking at me like I'm crazy.
I'm like, `O.K. Just wait till it drops.'''

\includegraphics{https://static01.graylady3jvrrxbe.onion/images/2019/03/10/magazine/10mag-scott-slideshow-slide-HXPA/10mag-scott-slideshow-slide-HXPA-master180.png}

4.

\includegraphics{https://static01.graylady3jvrrxbe.onion/images/2019/03/10/magazine/10mag-scott-slideshow-slide-XX9K/10mag-scott-slideshow-slide-XX9K-master180.png}

4A.

\includegraphics{https://static01.graylady3jvrrxbe.onion/images/2019/03/10/magazine/10mag-scott-slideshow-slide-71M0/10mag-scott-slideshow-slide-71M0-master180.png}

4B.

\emph{01:00} \emph{--- 02:45}

\emph{01:00} \emph{--- 02:45}

4. In 2015, the Swiss producer \textbf{Ozan Yildirim,} a.k.a. Oz, was
given an email address that supposedly belonged to Travis Scott. He
emailed beats for ``a year and a half,'' he says, with no reply.
Finally, a response arrived: ``You have dope {[}expletive{]}. Keep
sending.'' On Jan. 11, 2018, Oz sent what would become the second
section of ``Sicko Mode.''

4A. Oz got help with a synthesizer sound from his friend \textbf{Mirsad
Dervic,} a.k.a. M-Dee, an appliance salesman who makes music on his days
off.

4B. Oz also used a sound from a pack of samples created by the German
producing duo \textbf{Tim and Kevin Gomringer,} a.k.a. Cubeatz.

\includegraphics{https://static01.graylady3jvrrxbe.onion/images/2019/03/10/magazine/10mag-scott-slideshow-slide-O4MZ/10mag-scott-slideshow-slide-O4MZ-master180.png}

5.

\includegraphics{https://static01.graylady3jvrrxbe.onion/images/2019/03/10/magazine/10mag-scott-slideshow-slide-16B2/10mag-scott-slideshow-slide-16B2-master180.png}

5A.

\includegraphics{https://static01.graylady3jvrrxbe.onion/images/2019/03/10/magazine/10mag-scott-slideshow-slide-33EX/10mag-scott-slideshow-slide-33EX-master180.png}

5B.

\includegraphics{https://static01.graylady3jvrrxbe.onion/images/2019/03/10/magazine/10mag-scott-slideshow-slide-KZJ7/10mag-scott-slideshow-slide-KZJ7-master180.png}

5C.

\emph{01:15}

\emph{01:15}

5. Scott raps three words --- ``Gimme the Loot'' --- and as a result, 14
different people earn credits. Scott is quoting 1994's ``Gimme the
Loot'' by the Notorious B.I.G. and includes a sample from it, so Biggie
\textbf{(Christopher Wallace)} and the producer Easy Mo Bee
\textbf{(Osten Harvey)} get credit. Things telescope from there \ldots{}

5A. ``Gimme the Loot'' samples a vocal from Gang Starr's ``Just to Get a
Rep,'' which means that song's authors, Guru \textbf{(Keith Elam)} and
DJ Premier \textbf{(Christopher Martin),} are credited.

5B. ``Gimme the Loot'' also samples Sticky Fingaz's verse on Onyx's rap
classic ``Throw Ya Gunz,'' so even though Fingaz \textbf{(Kirk Jones)}
plays no part on ``Sicko Mode,'' he's credited, along with his Onyx
partners Chyskillz \textbf{(Chylow Parker),} Sonny Seeza \textbf{(Tyrone
Taylor)} and Fredro Starr \textbf{(Fred Scruggs).}

5C. ``Gimme the Loot'' samples another vocal, from A Tribe Called
Quest's ``Scenario (Remix)'' --- a line from the late Troy Anthony Hall,
a.k.a. Kid Hood. Strangely, Hall isn't a credited writer on Biggie's
song or Scott's, though Tribe's Q-Tip \textbf{(Kamaal Fareed),} Phife
Dawg \textbf{(Malik Taylor)} and \textbf{Ali Shaheed Jones-Muhammad}
are, along with the collaborators Busta Rhymes \textbf{(Trevor Smith),}
\textbf{Bryan Higgins} and \textbf{James Jackson.}

\includegraphics{https://static01.graylady3jvrrxbe.onion/images/2019/03/10/magazine/10mag-scott-slideshow-slide-QV0A/10mag-scott-slideshow-slide-QV0A-master180.png}

6.

\emph{01:36}

\emph{01:36}

6. Swae Lee, born \textbf{Khalif Brown} and half of the pop-rap duo Rae
Sremmurd, turns a fragmentary sung phrase --- ``Someone said'' --- into
one of ``Sicko Mode'''s unlikeliest hooks.

\includegraphics{https://static01.graylady3jvrrxbe.onion/images/2019/03/10/magazine/10mag-scott-slideshow-slide-ZKYA/10mag-scott-slideshow-slide-ZKYA-master180.png}

7.

\emph{01:40}

\emph{01:40}

7. In one of Scott's many tributes to Houston screw music --- a narcotic
local subgenre named after the late DJ Screw --- Scott includes two
pitched-down a cappella bars from a Screw compatriot, \textbf{John
Edward Hawkins,} a.k.a. Big Hawk, who was killed in 2006.

\includegraphics{https://static01.graylady3jvrrxbe.onion/images/2019/03/10/magazine/10mag-scott-slideshow-slide-9KL4/10mag-scott-slideshow-slide-9KL4-master180.png}

8.

\includegraphics{https://static01.graylady3jvrrxbe.onion/images/2019/03/10/magazine/10mag-scott-slideshow-slide-JX48/10mag-scott-slideshow-slide-JX48-master180.png}

8A.

\emph{02:17}

\emph{02:17}

8. Scott shouts out the Miami rap icon Uncle Luke --- born
\textbf{Luther Campbell,} notorious for his work in 2 Live Crew --- and
drops in a sample from his 1992 song ``I Wanna Rock.''

8A. ``I Wanna Rock'' samples K.C. and the Sunshine Band's ``That's the
Way (I Like It),'' and so that group's \textbf{Harry Wayne Casey} and
\textbf{Richard Finch} get credits, despite appearing nowhere on ``Sicko
Mode.''

\includegraphics{https://static01.graylady3jvrrxbe.onion/images/2019/03/10/magazine/10mag-scott-slideshow-slide-DXZD/10mag-scott-slideshow-slide-DXZD-master180.png}

9.

\emph{02:48} \emph{--- 02:56}

\emph{02:48} \emph{--- 02:56}

9. The beat grinds to a halt with a series of distorted kick drums
before moving to its final section. This transition contains production
from the Houston multi-instrumentalist \textbf{Mike Dean,} an executive
producer on ``Astroworld.''

\includegraphics{https://static01.graylady3jvrrxbe.onion/images/2019/03/10/magazine/10mag-scott-slideshow-slide-P935/10mag-scott-slideshow-slide-P935-master180.png}

10.

\emph{02:56} \emph{--- 05:12}

\emph{02:56} \emph{--- 05:12}

10. Tay Keith, born \textbf{BryTavious Chambers,} just graduated from
college in Tennessee. He was producing for local M.C.s when Drake got in
touch via Instagram to discuss collaborating. He's responsible for the
final section, including a drum pattern much like the one he built for
Drake's ``Nonstop.''

\includegraphics{https://static01.graylady3jvrrxbe.onion/images/2019/03/10/magazine/10mag-scott-slideshow-slide-3N88/10mag-scott-slideshow-slide-3N88-master180.png}

11.

XXX

XXX

11. Scott has known the Chicago rapper CyHi the Prynce, born
\textbf{Cydel Young,} since Scott's early days with Kanye West's
G.O.O.D. Music crew, of which Young is a member. Young helped Scott in
crafting lyrics.

\emph{Jonah Weiner is a contributing writer for the magazine. His last
feature for the magazine was about
\href{https://www.nytimes3xbfgragh.onion/2018/11/29/magazine/adam-mckay-dick-cheney-vice.html}{the
director Adam McKay}.}

\emph{Source photographs: Oz and Cubeatz: Alexandros Tiakas/Views of
Views Media. Big Hawk: Shawn Brauch/Pen \& Pixel Graphics, from the
University of Houston Libraries. Tay Keith: Zach Boisjoly. Mirsad Dervic
by Ozan Yildirim. All other source photographs: Getty Images.}

\includegraphics{https://static01.graylady3jvrrxbe.onion/images/2019/03/10/magazine/10mag-gaga/10mag-gaga-master180-v4.jpg}

\hypertarget{--why-did-you-do-that}{%
\subsection{\texorpdfstring{ `Why Did You Do
That?'}{  `Why Did You Do That?'}}\label{--why-did-you-do-that}}

\hypertarget{lady-gaga}{%
\paragraph{Lady Gaga}\label{lady-gaga}}

\hypertarget{06}{%
\subsection{06}\label{06}}

\includegraphics{https://static01.graylady3jvrrxbe.onion/images/2019/03/10/magazine/10mag-gaga/10mag-gaga-master180-v4.jpg}

The real star of `a star is born' is a song that few can agree on. is it
any good? yes. Yes it is.

By WESLEY MORRIS

\href{https://open.spotify.com/track/1sv0BhPPnrUgQ6dziZlmuM}{\includegraphics{https://static01.graylady3jvrrxbe.onion/newsgraphics/2019/02/26/music/2654e69ed27287529e314df4aede0297716f8c07/icons/spotify.png}}

Bradley Cooper's ``A Star Is Born'' is a movie about an unknown singer
named Ally, who's afraid to perform her own songs until a famous
musician takes one of her ballads, rearranges it, then drags her onstage
to sing it with him for a stadium full of people who have no idea who
she is. That song --- ``Shallow'' --- won an Oscar last month. It's a
soaring, Airbus of a ballad that's satisfying to sing whether you're
Kelly Clarkson or pitchy old me.

But ``Shallow'' isn't the number that epitomizes the movie. \emph{That}
comes past the halfway point, and your response to it sums up how far
under Cooper's spell you've fallen. I was under pretty deep. Ally's on
the verge of superstardom, but nearly all of the singing we've seen her
do has been with Cooper's character, Jackson Maine --- this impossible
fusion of grunge, roots country, pills, booze, pain, encroaching
deafness and the Whole Damn American Truth.

We can sense that Jackson's artistic attraction to Ally arises not just
from her industrial-strength voice but from her way with an abstractly
honest lyric and her knack for melody. She, too, seems ``authentic'' and
virtuous (she scribbles down ideas in a notebook; she's struggled and
suffered to ``make it''). His aversion to the artifice of showbiz would
seem to be hers: \emph{It's all about the craft for us, baby.} The spell
being cast is a matter of taste and prerogative, essentially that Ally
is made from Jackson's rib of purity.

I, at least, assumed that Ally would turn into somebody like Brandi
Carlile, a songwriter whose singing regularly reaches the stratosphere
but who we can tell is grounded and real because she holds a guitar the
way, for some of us, a lawyer holds a degree from Yale --- and because
... she ... isn't ... a dancer. Ally, on the other hand, does dance
(perhaps because it's what her craft-neutral wisp of a manager wants).
And the first time we see America seeing her dance is after she's
introduced as the musical guest on ``Saturday Night Live.'' The song ---
``Why Did You Do That?'' --- opens with the plinky simulations of a
steel drum or a music box, and then the question ``Why?'' ``\emph{Why}
do you look so good in those jeans?'' she sings, as if she were all of
Destiny's Children. ``Why'd you come around me with an ass like that?''
Ally's onstage with new, orange hair. (Thanks, craft-neutral manager!)
She's wearing spangling athleisure and a pink hooded jacket made of a
fabric I can describe only as ``gift bag.'' She's with a male dancer
who's dressed complementarily in blue, and when she says, ``ass like
that,'' he grinds himself into her.

\href{https://www.nytimes3xbfgragh.onion/interactive/2018/10/03/magazine/lady-gaga-movie-star-is-born.html}{{[}Read
Rachel Syme's profile of Lady Gaga.{]}}

It's here that I should say that I love this song. But according to the
movie's competing authenticities (Jackson's idea of it versus her
manager's; rock versus pop), love is a luxury. Gaga wrote ``Why Did You
Do That?'' with, among other people, Diane Warren, a master of
lugubrious balladry, a cheese whiz. But these women are grilling that
cheese. Gaga's enthusiasm for repetition in a chorus slips an earworm on
the hook. \emph{Why did you do that --- do that, do that, do that, do
that --- to ME?} Next comes a sugary helping of ``Hoooh-oooh-oooh,''
while a sloppy, synthy bass line slithers around it all. This song is
sung with such umbrage and so much alarm that you don't know whether to
sing along or call Gloria Allred.

But I watched Ally perform it with my hand to my mouth. I might have
said aloud: ``Oh, Ally. Jackson's going to \emph{hate} this.'' And no
sooner had I said it than the movie cut to Jackson watching with a face
of stone. This song is confection and sex and feel-copping. It's
showbiz. Basically, it's everything Jackson would seem to hate about
whatever music is right now.

When the movie cuts to Jackson's face, the Ike Turner of ``What's Love
Got to Do With It'' crossed my mind. And a few scenes later, a version
of Ike comes over him while Ally is trying to have herself a ``Pretty
Woman'' bath and Jackson enters. She's sudsy. He's soused. ``Why you
come around me with an ass like that,'' he says in disgust. ``Maybe I
{[}expletive{]} failed you,'' he goes on. ``You're embarrassing,'' he
slurs. ``You're just {[}expletive{]} ugly,'' he slurs some more. Jackson
doesn't want to see ``Why Did You Do That?'' as the hit of an ingénue,
something anonymous-seeming that a new pop star tries before a truer
identity bubbles up: Pink doing ``Most Girls,'' Katy Perry and ``I
Kissed a Girl,'' Rihanna's ``Pon De Replay,'' Gaga's ``Just Dance,''
whatever the perma-ingénue Ariana Grande's currently up to.

Last fall, this one song, from this megahit movie, provoked perplexed
essays and inspired The Times's Kyle Buchanan to track down Warren and
ask, Is it \emph{supposed} to be bad? Jackson thinks so. As much as I
wanted to save this sexy, damaged, doomed man, on this, we disagree.
``Maybe it's time to let the old ways die,'' goes his most mournful
lyric. But maybe it's also time to admit nothing's wrong with an ass
like that.

\emph{Wesley Morris is a staff writer for the magazine, a critic at
large for The New York Times and co-host of the podcast
\href{https://www.nytimes3xbfgragh.onion/column/still-processing-podcast}{``Still
Processing.''}}

Read more

\includegraphics{https://static01.graylady3jvrrxbe.onion/images/2019/03/10/magazine/10mag-pinkfong/10mag-pinkfong-master180-v4.jpg}

\hypertarget{--baby-shark}{%
\subsection{\texorpdfstring{ `Baby
Shark'}{  `Baby Shark'}}\label{--baby-shark}}

\hypertarget{pinkfong}{%
\paragraph{Pinkfong}\label{pinkfong}}

\hypertarget{07}{%
\subsection{07}\label{07}}

\includegraphics{https://static01.graylady3jvrrxbe.onion/images/2019/03/10/magazine/10mag-pinkfong/10mag-pinkfong-master180-v4.jpg}

The origins and meaning of a meaningless viral hit.

By JODY ROSEN

\href{https://open.spotify.com/track/5ygDXis42ncn6kYG14lEVG}{\includegraphics{https://static01.graylady3jvrrxbe.onion/newsgraphics/2019/02/26/music/2654e69ed27287529e314df4aede0297716f8c07/icons/spotify.png}}

The children's song ``Baby Shark'' is a global smash, a hit that has
captivated millions, conquering charts from Asia to Australia to the
United States, where, this January, it reached No. 32 on the Billboard
Hot 100. It's also a folk song, with origins as obscure as ``The Dowie
Dens o' Yarrow'' or any of those other ancient airs whose authorship
long ago vanished in the mists. ``Baby Shark'' is thought to have been
born in American summer camps, perhaps several decades ago. It is a
staple of singalongs, the sort of song that gets belted out by groups
gathered at marshmallow roasts and swimming pools. One theory holds that
the song was born in the summer of 1975, when Steven Spielberg's
``Jaws'' was in theaters and great white sharks were gliding through the
murky waters of collective consciousness, to the strains of John
Williams's brooding score.

The precise provenance of ``Baby Shark'' may never be pinned down. Its
recent history is clearer. In November 2015, Pinkfong, a South Korean
educational brand, released a hopped-up rendition with an accompanying
animated video. The following June, Pinkfong put out a second video,
\href{https://www.youtube.com/watch?v=XqZsoesa55w}{``Baby Shark
Dance,''} featuring two cute kids performing goofy dance moves. It was
this clip that inspired the hashtag \#BabySharkChallenge, instigating a
viral craze that has racked up more than two billion YouTube views and
spawned unnumbered spinoffs starring everyone from Indonesian
farmworkers to Filipino marines to Cardi B to, undoubtedly, your
friends, your family, your baby trussed in a shark costume. In the
United States, the \#BabySharkChallenge has received an extra viral
boost, cross-pollinating with a dance craze linked to Drake's anthem
``In My Feelings,'' in which people exit moving motor vehicles to dance
and lip-sync as the car putters alongside.

In other words, ``Baby Shark'' has completed two full transmigrations
between folk and pop. It's a folk song that became a pop song that
filtered into social media to become a folk song again --- a grass-roots
phenomenon that propelled the pop recording to improbable heights of
ubiquity. It exemplifies several features of 21st-century culture: the
porous boundaries between the pop industrial complex and the amateur
homespun; a globalized circulation of songs based as much on memes as on
music; the popularity of unchallenging dance ``challenges''; the
hegemony of Drake.

``Baby Shark'' also exemplifies the timeless appeal of really stupid
songs. Musically, ``Baby Shark'' is efficient, with a foursquare beat
driving an unvarying melody that returns repeatedly to the wordless
chorus ``Doo doo doo doo doo doo.'' This refrain repeats fully 27 times
--- a lot, in a song that runs 1 minute 36 seconds. The first minute is
devoted to introducing the dramatis personae: ``Baby Shark/Doo doo doo
doo doo doo ... Mommy Shark/Doo doo doo doo doo doo,'' etc. The
narrative, such as it is, is compressed into the song's final third. The
shark family goes hunting, and its prey elude capture: ``Safe at
last/Doo doo doo doo doo doo.''

Even the most rabid ``Baby Shark'' fan will concede that it is
infantile. Which is not to say that it's a children's song. In fact,
it's an adult song masquerading as a kids' song. ``Baby Shark'' has been
sanitized: Traditional versions sung by campers are mischievous and
macabre, telling tales of sharks attacking swimmers who lose limbs and,
usually, lives. (``Call 911/Doo doo doo ... It's too late/Doo doo
doo.'') In one variation, a ``surfer dude'' is mauled, dies and is
reincarnated as a baby shark: an artful cycle of gore, death, rebirth
and more gore.

``Baby Shark,'' by contrast, seems engineered to please the parents.
It's a garishly wholesome affirmation of the nuclear family and ends
happily; its dance can be mastered by the most rhythm-impaired. Is it a
stretch to suggest that ``Baby Shark'' is music for grown-ups who are,
as it were, in their feelings, stressed by the turmoil and dislocations
of 21st-century life? Compare ``Baby Shark'' with music popular with
millennials: relentlessly dour rap and hip-hop-inflected pop, full of
menace and foreboding, which face the bummer of 2019 head-on. ``Baby
Shark'' offers an antidote, an escape: a song that delivers us from
danger --- safe at last! --- leaving nary a trace of blood in the water.

\emph{Jody Rosen is a contributing writer for the magazine and the
author of a forthcoming book about bicycles.}

Read more

\includegraphics{https://static01.graylady3jvrrxbe.onion/images/2019/03/10/magazine/10mag-the-carters/10mag-the-carters-master180-v3.jpg}

\hypertarget{--apeshit}{%
\subsection{\texorpdfstring{ `Apeshit'}{  `Apeshit'}}\label{--apeshit}}

\hypertarget{the-carters}{%
\paragraph{The Carters}\label{the-carters}}

\hypertarget{08}{%
\subsection{08}\label{08}}

\includegraphics{https://static01.graylady3jvrrxbe.onion/images/2019/03/10/magazine/10mag-the-carters/10mag-the-carters-master180-v3.jpg}

Beyoncé And Jay-Z know whose work has long been left out of the canon,
and they won't let it happen to theirs.

By JENNA WORTHAM

\href{https://open.spotify.com/track/0E6PsO3ymCfUh7pJQjBgkj}{\includegraphics{https://static01.graylady3jvrrxbe.onion/newsgraphics/2019/02/26/music/2654e69ed27287529e314df4aede0297716f8c07/icons/spotify.png}}

Last year, on one of the first stops of Beyoncé and Jay-Z's ``On the Run
II'' tour, the lights of London Stadium dimmed, and
\href{https://www.youtube.com/watch?v=kbMqWXnpXcA}{the video for
``Apeshit''} began playing on enormous screens. The video opens with the
Carters dressed in gorgeous suits (hers a Peter Pilotto in pink and red;
his, sea-foam green Dries Van Noten) standing --- alone --- in front of
the Mona Lisa at the Louvre. ``Have you ever seen a crowd going
apeshit?'' the song asks, and the answer lay in the reaction of the
British concertgoers, who screamed, cried and danced.

Whose history belongs in our museums? The video and song meditate on
this question. The history of black people has too often been presented
as little more than a curiosity. The 1889 world's fair in Paris, for
example, included a ``human zoo'' of indigenous people taken from French
colonies and placed in ``natural'' habitats for onlookers. During the
16th century, Africans were exhibited in the Vatican, and in 1906 a
young Congolese man called Ota Benga was forcibly kept at the Bronx Zoo.
Even now, landmark museums like the Louvre tend to exhibit artwork that
depicts Africans and their descendants as household servants and
domestic workers. The Carters know this painful past, and in ``Apeshit''
they confront it.

Lyrics like ``We livin' lavish, lavish/I got expensive fabrics/I got
expensive habits'' could give the impression that the song is merely
reveling in the luxuries that extreme wealth can buy. One great
complexity regarding the couple is their overt embrace of capitalism.
Are they disrupting the status quo or reinforcing it? But just beneath
all that spending seethes an abject rage. ``I said no to the Super Bowl,
you need me, I don't need you,'' Jay-Z raps. ``Every night we in the end
zone, tell the N.F.L. we in stadiums too.'' The Carters are Colin
Kaepernick-level fed up with institutions that barely recognize them,
whether it be the Grammys, which nominated Jay-Z eight times last year
but gave him zero wins, or Coachella, which featured Beyoncé as its
first black female headliner only last year. ``Gimme my check, put some
respect on my check. Or pay me in equity, pay me in equity'' is about
money, yes, but it's also a cry to be acknowledged for what they've
earned. Their best revenge is their paper, but it's also their own
music-streaming service (Tidal) and a clothing line (Ivy Park).

\href{https://www.nytimes3xbfgragh.onion/2019/03/07/reader-center/jenna-wortham-new-york-times-magazine-still-processing.html}{{[}\emph{Get
to know the author of this article, Jenna Wortham, in a new Behind the
Byline interview.}{]}}

The Carters have collaborated for almost two decades on songs that
rotate around their love of money (``'03 Bonnie \& Clyde'') and of
infatuation (``Crazy in Love''). But ``Apeshit'' represents a new era
for their economic philosophies and value systems. Wealth isn't just for
flashy living; it's about creating an empire for themselves and their
offspring. Love is hard, unflattering work that sometimes requires
setting aside ego and reputation. When Beyoncé sings of plans to get her
girls and ``put 'em all on a spaceship,'' you feel that it's not
lyrically convenient but that she really means it. What would a world
created entirely by and for black people look like?

Even as Beyoncé and Jay-Z have an estimated net worth of more than a
billion dollars between them, they have managed to remain largely
unscathed by a cultural tide that disdains the obscenity of late-stage
capitalism. Perhaps that's because they seem to be trying to undo a
larger project of disinheritance. The video continues its tour of the
Louvre, showing the Carters and dancers posing among some of the world's
most famous art. They are asserting that they belong. ``All of my
people, I free 'em all,'' Beyoncé sings. It's a boast, but it's also
their mission.

\emph{Jenna Wortham is a staff writer for the magazine and co-host of
the podcast
\href{https://www.nytimes3xbfgragh.onion/column/still-processing-podcast}{``Still
Processing.''}}

Read more

\includegraphics{https://static01.graylady3jvrrxbe.onion/images/2019/03/10/magazine/10mag-juice-world/10mag-juice-world-master180-v5.jpg}

\hypertarget{--lucid-dreams}{%
\subsection{\texorpdfstring{ `Lucid
Dreams'}{  `Lucid Dreams'}}\label{--lucid-dreams}}

\hypertarget{juice-wrld}{%
\paragraph{Juice Wrld}\label{juice-wrld}}

\hypertarget{09}{%
\subsection{09}\label{09}}

\includegraphics{https://static01.graylady3jvrrxbe.onion/images/2019/03/10/magazine/10mag-juice-world/10mag-juice-world-master180-v5.jpg}

The same affected adolescent pain, updated for modern times.

By JAIME LAUREN KEILES

\href{https://open.spotify.com/track/285pBltuF7vW8TeWk8hdRR}{\includegraphics{https://static01.graylady3jvrrxbe.onion/newsgraphics/2019/02/26/music/2654e69ed27287529e314df4aede0297716f8c07/icons/spotify.png}}

If you grew up listening to pop punk, as I did, then it's easy to feel
as though the mid-to-late aughts were lost to a fold in adolescent
space-time. Are we even sure that the genre ever happened? Pop punk
married punk power chords with the singable hook of a radio hit. The
aesthetic was embarrassing, even in its time --- circuses, graveyards,
men in eyeliner. Want to fantasize about murdering your ex? For a brief,
fun lapse in those dubious years, such thoughts were best expressed in a
high, clear whine, interspersed with bouts of indiscriminate screaming.
Today we might call pop punk ``problematic'' (or maybe we'd call it
musical theater). To me, at 14, it was more than visceral, a soundtrack
for a time of hormonal disarray.

Flash-forward nearly 15 years to the present, and somehow, improbably,
pop punk is back in the form of the rap song ``Lucid Dreams'' by Juice
WRLD. The track first appeared on the streaming platform SoundCloud and
rose through the ranks of the Billboard Hot 100 on the wings of
brooding, sung-rapped pain: ``I still see your shadows in my room/can't
take back the love that I gave you/it's to the point where I love and I
hate you.'' ``Lucid Dreams'' is pop punk recapitulated --- the same
themes, the same whine, the same singable hook, with the power chords
swapped out for insular drums and the plaintive guitar of Sting's
``Shape of My Heart.'' It echoes the pangs of a Fall Out Boy track,
throbbing with urgent teenage despair.

Juice WRLD is 20, from the suburbs of Chicago. He describes his own
music as ``a therapy session'' and cites influences like Fall Out Boy,
Bullet for My Valentine, Senses Fail and Panic! at the Disco. Like most
rappers of this latest generation, these influences evolved in a
post-streaming world, where albums existed as free-floating tracks,
somewhat detached from imposed genre labels.

Hit pop songs that exploit algorithms are sometimes described as
``Spotifycore'' or ``streambait.'' If ``Lucid Dreams'' was not produced
to game the numbers, then at least this new system of musical incentives
might help explain its unlikely rise. Rap music turns on its
habit-forming beats, and pop punk thrives on earwormish hooks.
Accounting for the keen melodrama of both genres, it makes perfect sense
that a hybridized form would triumph in this new streaming ecosystem.

Juice WRLD is not the first or only artist to work in the emo-rap
subgenre. The rapper LiL Peep, who died from an overdose of fentanyl and
Xanax in 2017, was extolled as ``the future of emo.'' The same might be
said of XXXTentacion, a rapper who made violent, confessional music
before he was fatally shot last June. The troubles of this music scene
have been well covered; in brief, they reflect the real perils of our
time --- gun violence, a crisis of masculinity, dual drug and
mental-health epidemics. If the pop-punk songs of decades past were
grandiose enough to be written off as camp, then the latest wave of
emo-rap seems somehow right-sized for the terrors of our moment. With
this in mind, ``Lucid Dreams'' sounds less catchy, or maybe it just
sounds less catchy to adults.

\emph{Jamie Lauren Keiles is a writer in Queens working on a novel about
smoking.}

Read more

\includegraphics{https://static01.graylady3jvrrxbe.onion/images/2019/03/10/magazine/10mag-music-slideshow-slide-6BZ4/10mag-music-slideshow-slide-6BZ4-master180-v3.png}

\hypertarget{--slow-burn}{%
\subsection{\texorpdfstring{ `Slow
Burn'}{  `Slow Burn'}}\label{--slow-burn}}

\hypertarget{kacey-musgraves}{%
\paragraph{Kacey Musgraves}\label{kacey-musgraves}}

\hypertarget{10}{%
\subsection{10}\label{10}}

\includegraphics{https://static01.graylady3jvrrxbe.onion/images/2019/03/10/magazine/10mag-music-slideshow-slide-6BZ4/10mag-music-slideshow-slide-6BZ4-master180-v3.png}

Our freshest pop and country superstar is not the usual kind of outlaw.

By LIZZY GOODMAN

\href{https://open.spotify.com/track/6ET9kf9riLETWs9lePUEAI}{\includegraphics{https://static01.graylady3jvrrxbe.onion/newsgraphics/2019/02/26/music/2654e69ed27287529e314df4aede0297716f8c07/icons/spotify.png}}

A couple of weeks before she would step onstage to accept the Grammy for
Album of the Year, Kacey Musgraves was under the covers in the bedroom
at the back of her tour bus, pondering the nature of the universe. She
had a little unexpected time on her hands. A show in Chicago had been
canceled, thanks to the polar freeze that had descended over the
Midwest, leaving her stuck in the middle of a vast tundra with a buildup
of tour adrenaline and nowhere to put it. She watched some ``The
Marvelous Mrs. Maisel,'' read a little ``The Catcher in the Rye'' and
then lost herself down her favorite YouTube rabbit hole, a video genre
in which someone mashes up tweedy old-school physics lectures with wonky
electro beats. ``It's like if Daft Punk went totally science,'' she
says, ``and I'm \emph{here} for that.''

Later, she would stand in a diaphanous scarlet Valentino dress at the
Grammys, giving a speech that could, given her tone and reputation, be
read as subtly anti-authoritarian. ``Life is pretty tumultuous right now
for all of us,'' she told the crowd. ``I feel like, because of that, art
is really thriving.'' Musgraves is well known for her support of the
L.G.B.T. community, her unabashed love of weed and her ability to turn a
cutting phrase in her perfect Texas twang --- particularly when she's
writing about the shackles of social convention. She started writing her
winning album, ``Golden Hour,'' early in 2016, when Donald Trump was
still assigning stinging nicknames to his Republican primary opponents,
and began recording it just after Trump's inauguration. She has tweeted
in support of the Women's March and in disdain of Eric Trump's
family-branded Christmas ornaments. She speaks with pride about the
stack of detention slips her mother still keeps from her daughter's time
in the Mineola, Tex., school system: ``I was always getting in trouble
for, like, insubordination.'' Add all that to the nose piercing that, as
she famously sings on the rapturous single ``Slow Burn,'' made her
grandmother cry, and you might expect her album to be a bit of a call to
arms, a middle finger to a broken world.

Not so much. ``I just got tripped out one day,'' she says, musing on her
inspiration for the album. ``Thinking, Whoa, wait, we live in this world
that seems so mundane, but at the same time that I'm sitting here,
there's things that are glowing in the ocean and eating each other ---
and there's also northern lights and shooting stars and plants that grow
and literally heal people.'' She paused for breath. ``And it's all
happening around us, you know?'' Falling for the man who is now her
husband --- the singer and songwriter Ruston Kelly --- was another part
of the album's genesis. ``It's sort of a love song to him,'' she says,
``but also to nature, the human race, Earth and why we're here. We don't
know, and I kind of love it.''

In other words, this wry firebrand's big statement on the state of the
world --- at a time when so many of the issues she has become famous
writing about, like feminism and gay rights, are making daily headlines
--- is a metaphysical country-pop record more inspired by Carl Sagan
than Willie Nelson. And just so we're clear, Musgraves was on acid only
\emph{part} of the time. ``It's not like I was tripping my face off
every day,'' she clarifies. (After she told reporters that psychedelics
influenced a couple of songs, including ``Slow Burn,'' it was all anyone
wanted to talk about.) ``It has only been a couple times. And very
responsibly! Enough to be able to get outside of yourself and see a
different perspective or point of view.''

\includegraphics{https://static01.graylady3jvrrxbe.onion/images/2019/03/10/magazine/10mag-music-slideshow-slide-GT2I/10mag-music-slideshow-slide-GT2I-master180.png}

What makes Musgraves such a resonant figure right now, in fact, is the
way her response to a dark, anxious moment in human history is to move
willfully closer to lightness, to stillness, toward the possibility of a
world that comes in more colors than red or blue. When she talks about
art thriving in this climate, she means it --- just not in the same
sense as, say, angry punks railing against the Reagan administration.
What she means is that right now, the best rebellion involves turning
off the hate and making space for hope. Or, as she puts it: ``The
{[}expletive{]} storm won't last forever, and I want to make music that
does.''

{[}\href{https://www.nytimes3xbfgragh.onion/2018/10/24/arts/music/kacey-musgraves-slow-burn.html}{Watch
Kacey Musgraves turn country music psychadelic.}{]}

I caught up with Musgraves in Wisconsin, on the tail end of January's
alarming deep freeze, which had temperatures in the upper Midwest
dropping as low as minus 40. (I missed her in Chicago, where everyone
was trapped inside, the streets vacant apart from the odd
extreme-weather junkie taking photographs of ice floes.) Far from
Valentino, she was, for the moment, in sweaty Victoria's Secret workout
tights and a fluorescent-green beanie, sitting straddle-style on the
floor of the bar at a Madison venue called the Sylvee, having just
finished a workout via Skype with Erin Oprea, a trainer to many of
Nashville's stars.

\includegraphics{https://static01.graylady3jvrrxbe.onion/images/2019/03/10/magazine/10mag-music-slideshow-slide-D2X2/10mag-music-slideshow-slide-D2X2-master180.png}

``O.K., so this is the one I put on my story yesterday,'' she said,
finding a clip she had posted to Instagram and showing me her phone. It
was something called Symphony of Science's ``Quantum World,'' a favorite
among those space-disco physics videos. ``Featuring Neil DeGrasse
Tyson,'' she chuckled, reading from the chyron at the bottom of a
related clip. I had indeed seen her Instagramming this kind of
mysterious, late-night Discovery Channel-type stuff --- the sort of
thing teenagers once saw at the IMAX theater on a field trip after
getting stoned. How did she get into it? ``Oh, who knows, it was years
ago,'' she replied, then sang happily along to a remix that showed
Morgan Freeman superimposed on a colorful tunnel of celestial light. One
of the scientists' 1970s professorial look, she pointed out, was
``literally like what the band wears'' in her stage show.

\includegraphics{https://static01.graylady3jvrrxbe.onion/images/2019/03/10/magazine/10mag-music-slideshow-slide-4JUC/10mag-music-slideshow-slide-4JUC-master180.png}

When Musgraves was 18 and a contestant on the reality show ``Nashville
Star'' --- she placed seventh --- she had to fill out a
getting-to-know-the-contenders C.V. Under the category of ``dream
vacation,'' she listed ``staying in a huge log cabin in the mountains,
riding horses, hunting and four-wheeling with my friends.'' The
``craziest'' thing she'd ever done? ``Hunting for Bigfoot deep in the
woods of East Texas. ... We didn't get him.'' That version of Musgraves
--- the one who cited Jack in the Box egg rolls and beer as her favorite
foods --- still appears at every show, even when she's dressed like
Bianca Jagger heading to Studio 54. She's the one leading the ritual
preshow group shot of tequila, taken from tiny cactus-shaped glasses she
and her band have long been toting from show to show. And she's the one
hanging with me on the floor of a bar in Wisconsin, looking at videos by
a user with the handle ``melodysheep.''

And yet even in her early years, when Musgraves looked more the part of
your average Nashville aspirant, in cowboy boots and blond highlights,
there was always a kind of poise, an innate regality that set her apart.
This, perhaps, is the other side of her East Texas grit --- the one that
manifests less as yee-haw joy and more as D.I.Y. conviction. ``When it
comes to art, I will not bend,'' she says. ``I won't.''

\includegraphics{https://static01.graylady3jvrrxbe.onion/images/2019/03/10/magazine/10mag-music-slideshow-slide-ZS2R/10mag-music-slideshow-slide-ZS2R-master180-v2.png}

\includegraphics{https://static01.graylady3jvrrxbe.onion/images/2019/03/10/magazine/10mag-music-slideshow-slide-SE3Q/10mag-music-slideshow-slide-SE3Q-master180-v2.png}

Musgraves grew up in Golden, Tex., a town so small it doesn't even have
an elementary school. ``A few hundred people,'' she guesses, is the
total population. It's about 90 minutes from Dallas, and about six or
seven miles outside Mineola, where Musgraves and her sister, Kelly
Christine Sutton --- a photographer, who shot the ``Golden Hour'' cover
--- went to school, and where their parents had a small printing shop.
``Even at a young age,'' Sutton says, ``I always knew my sister would be
known for her music. And not really on anyone else's timeline. She would
make it happen on her own terms.'' It's their parents' model of
small-town independence, Musgraves figures, that gave her a
tend-your-own-garden will. ``A large part of who I am comes from the
fact that I never saw my parents have bosses,'' she says. ``They've
never answered to anyone but themselves. And not in a baller way ---
like very small-business, check-to-check kind of a thing. But they made
all their own decisions.''

Growing up, she had a Spice Girls poster in her room --- Ginger, with
her wild tattoo, made a strong impression --- and listened to emo rock
bands like the Used and Dashboard Confessional. But that wasn't the sort
of music she played. ``I was part of this kids' group called the
Buckaroos that would meet every month in the Fort Worth Stockyards and
would dress up in cowboy clothing and stroll the stockyards and learn
instruments,'' she recalls. ``There were mentors there who kind of
encourage kids to learn this old stuff.'' By 9, she was writing her own
songs and playing guitar; by 12, she was singing Western swing and
yodeling at festivals on the weekends. Bookings and press kits were
handled by her grandmother --- the same one who later cried when
Musgraves pierced her nose, and who referred to ``It Is What It Is,''
the singer's melancholic ode to casual sex, as ``the slut song.''
``She's a hoot,'' Musgraves says. ``She was wheeling and dealing.''

There was, of course, the requisite period in which a teenage Musgraves
turned her back on the whole cowgirl thing. ``I was like, Dude, none of
my friends think this is cool. If they saw me I'd be superembarrassed.
I'm yodeling, you know what I mean?'' She rebelled --- for a second. ``I
chopped off all my hair and was like: Suck on that! Now I can't wear a
cowboy hat, Mom!'' (``You would not believe how upset my family was,''
Sutton remembers.) But this rebellion turned out to be short-lived. By
the time Musgraves moved full time to Nashville, at 19, she had realized
``nobody really in the country world was embracing the hard-core roots
of the genre.'' But she had come to worship John Prine and Loretta Lynn
--- big-hearted, sharp-tongued, storytelling pillars of country's outlaw
roots who had risen up in the '70s by staging their own insurgency
against the bubble-gumming-up of the genre. ``I randomly already had
this superknowledge about all the old songs that came before me, and
that style, the Western fashion,'' Musgraves says. ``I was like: I'm
going to bring it back. I want to mix that in with something modern.''
Prine himself is now one of her many admirers; he compares her to ``a
goofy Cinderella.'' ``She has a certain honesty to her voice,'' he says.
``She's breaking down barriers.''

There's a famous tale about Kacey Musgraves's first big showcase
performance for the Country Radio Seminar. This is a big-deal event in
the business; its attendees are queen-makers in an industry in which
success is still determined by access to radio airwaves. The story of
her 2013 debut there smacks of a plotline on ``Nashville'' ---
appropriate, given that she co-wrote ``Undermine,'' one of the hits to
emerge from the juggernaut TV series. You can easily picture the kind of
episode Musgraves's performance might have inspired. A young woman takes
the stage at the legendary Ryman Auditorium, the so-called Mother Church
of country, about to play the song that could make or break her career.
We learn that she's being hyped --- thanks to her preternatural
songwriting skills, good looks and the already-feverish crossover
response to her first single, ``Merry Go Round'' --- as the Veronica to
Taylor Swift's Betty: a sassier but potentially just as marketable
product. As is the industry's way, the corporate powers would like this
rising phenom to be herself, but only within reason. They'd prefer she
refrain from playing what will become her third single --- the one with
the lines about smoking joints and kissing girls --- until \emph{after}
she can already be heard in every Walgreens in America. But the woman
steps up to the microphone, leans into it a little and speaks: ``I'm
kind of a big believer in people doing whatever the hell they want to
do, because I feel like society is probably going to have an opinion
either way.'' Then she starts strumming ``Follow Your Arrow.'' As she
delivers its opening line --- ``If you save yourself for marriage,
you're a bore/If you don't save yourself for marriage you're a horr-''
--- the crowd gasps. Then she finishes --- ``-ible person'' --- and the
audience laughs. A star is born.

\includegraphics{https://static01.graylady3jvrrxbe.onion/images/2019/03/10/magazine/10mag-music-slideshow-slide-BLUO/10mag-music-slideshow-slide-BLUO-master180.png}

It's easy to wonder where Musgraves got the nerve. It's one thing to
admire your parents' entrepreneurial spirit, and another to hold your
own amid the chaos and pressure of the starmaking machine. For
Musgraves, performing alongside Dolly Parton at the Grammys, winning
Album of the Year, presenting an award at the Oscars --- all of this is
unequivocally her dream. But it's also something she believes she could
live without, and remembering this has become a kind of daily
meditation, especially as the scope of her fame has increased. ``I don't
get high off my own supply, you know?'' she says, mentioning, by way of
inspiration, Willie Nelson's ability to welcome ``Republicans, rappers,
presidents, my grandpa, your grandpa, our hipster friends, me'' without
treating anyone as superior to anyone else. ``You can be proud of
yourself and excited for what you're doing, and you could even really
have a high level of confidence, without being a D-bag.''

In Musgraves's mind, she made it the minute she signed her first
songwriting deal, back in 2009, penning tunes that would be sold around
Nashville to other performers. That was the day she realized she would
never again have to work a job in which she dressed up as Disney
characters for children's birthday parties, one of many day gigs she had
endured. ``For the next few years,'' she says, ``I was like: Really?
Wait, I can use my brain, sit on my ass and make a living?'' When her
current label first made her an offer to record as an artist, Musgraves
turned it down; she was having a perfectly good time as a writer. She
also knew she didn't have real access to her own voice yet. ``Those
songs were fine for other artists,'' she says. ``Maybe they could be
popular on the radio or something, but they're not very me.''

\includegraphics{https://static01.graylady3jvrrxbe.onion/images/2019/03/10/magazine/10mag-music-slideshow-slide-NLF7/10mag-music-slideshow-slide-NLF7-master180.png}

By the time Musgraves eventually located her particular voice, it was
already honed to a sharp edge. Her first hit, ``Merry Go Round,'' from
2012, is packed with the kind of mordant wordplay she'd be known for,
conjuring a ``same trailer, different park'' world where people marry
out of boredom and settle ``just like dust'' into small-town lives:
``Mama's hooked on Mary Kay, brother's hooked on Mary Jane, Daddy's
hooked on Mary two doors down.'' But after two albums and multiple world
tours, Musgraves felt worn out by her own verbal cleverness. ``Everyone
hopefully knows I can flip a phrase by now, and I like that,'' she says.
``But I don't want bumper-sticker songs.'' It also concerned Musgraves
that the refreshing directness with which she had addressed social
issues might start to feel heavy-handed, even ideologically gimmicky.
She is, as she puts it diplomatically, ``noticing things about the world
that I'm not happy with.'' But when she started working on ``Golden
Hour,'' it no longer felt right to address them directly. ``Everyone
that's listened to any of my music knows exactly how I feel,'' she says.
``This record does kind of nod to some of the social and political
things that are going on, but it's just doing it in a different way.
It's not as linear.''

Back on her bus, in Wisconsin, after playing to a couple thousand
freezing fans who arrived lit and ready to party, Musgraves decompressed
again. Gone was the collection of products the singer uses to transform
herself from the kind of girl her sister remembers Musgraves sometimes
presented as in high school --- ``Converse, Dickies and black eyeliner''
--- to the flamboyantly feminine star who shares the stage with
``RuPaul's Drag Race'' contestants. (``I have thought, Am I just doing
this because it's expected, or do I actually enjoy it?'' Musgraves said
to me earlier, while spraying her face with a mist of foundation. ``And
it's like: No! I enjoy it!'') Now she was in slippers and a kind of
housecoat, which I mentioned made her look approximately like my
grandmother. ``Let's just say I also have a towel warmer,'' she replied.
She puttered around her kitchen, making mugs of ginger tea. Then she
pointed out where the bus's temperature controls were, and the cabinet
where she keeps the melatonin gummies, and said good night.

\includegraphics{https://static01.graylady3jvrrxbe.onion/images/2019/03/10/magazine/10mag-music-slideshow-slide-2261/10mag-music-slideshow-slide-2261-master180.png}

Tucked away with her tasteful crystal collection in the earth-toned
bedroom in the back of the bus --- the first she has ever had, after
touring for years alongside ``15 other people'' --- there were any
number of things might have turned her attention to. She could have
looked at specs for the adult coloring book she's designing with her
mother, or maybe FaceTimed with her husband, Ruston Kelly, also on tour
somewhere in the frozen Midwest. She might have scrolled through the
looks her stylist had just sent through for the Grammys; she was still
searching for something just right to match Dolly Parton. Or there was
Musgrave's common insomnia treatment, shopping on eBay: ``I get down a
little rabbit hole,'' she told me, ``ordering old toys that they don't
make anymore. Like dolls from the '80s or '90s. If I ever have a girl,
it could be cute to give her P.J. Sparkles, or Makeup Beauty, or
whatever, you know?'' They would all be shipped to Nashville, where
they'd be waiting for her when she got home --- whenever that was.

``I mean, my 2020 is planned out,'' she told me earlier. ``I'm going to
be putting another record out. I haven't made that yet, so I need to
write that and make that. Lots to do.'' It is imposing, you'd guess, to
have an album scheduled for release that you haven't begun to record.
``It's scary,'' she allowed. ``You worry that the muse is not going to
visit you again.'' She reached for her lip gloss. ``There's no banking
on it. So it's odd that there's an entire industry banking on it.''
She's not worried, though. Her latest idea showed up a few weeks ago ---
a low-key, ``kind of like Bill Withers meets Sade'' track she's calling,
for the moment, ``Good Wife.'' She likes its ease, she says: ``It's not
trying too hard.''

\emph{Lizzy Goodman is a journalist and the author of ``Meet Me in the
Bathroom,'' an oral history of music in New York City from 2001 to
2011.}

\emph{Devin Yalkin is a photographer from New York.}

Read more

\includegraphics{https://static01.graylady3jvrrxbe.onion/images/2019/03/10/magazine/10mag-post-malone/10mag-post-malone-master180-v2.jpg}

\hypertarget{---sunflower-}{%
\subsection{\texorpdfstring{ `Sunflower'
}{   `Sunflower' }}\label{---sunflower-}}

\hypertarget{post-malone}{%
\paragraph{Post Malone}\label{post-malone}}

\hypertarget{11}{%
\subsection{11}\label{11}}

\includegraphics{https://static01.graylady3jvrrxbe.onion/images/2019/03/10/magazine/10mag-post-malone/10mag-post-malone-master180-v2.jpg}

Post Malone, Spider-Man and the unexpected powers of adolescent
awkwardness.

By SAM ANDERSON

\href{https://open.spotify.com/track/3KkXRkHbMCARz0aVfEt68P}{\includegraphics{https://static01.graylady3jvrrxbe.onion/newsgraphics/2019/02/26/music/2654e69ed27287529e314df4aede0297716f8c07/icons/spotify.png}}

Like most ancient people in their declining years, I keep up with
today's hit music entirely through my teenage daughter. She carries her
Bluetooth speaker from room to room with the tender devotion of a mother
cat ferrying kittens across a flooded stream. Thanks to her, we scramble
eggs and make toast and fold laundry in an ambient cloud of Chance the
Rapper, Drake, Nicki Minaj, Nicki Minaj featuring Drake, Drake featuring
Nicki Minaj, Lorde, Rihanna, ``country'' Taylor Swift and ``dubstep''
Taylor Swift.

Over the last year, an increasingly dominant voice in this mix has been
Post Malone, a 23-year-old sort-of-rapper from suburban Dallas.
``Posty,'' as my daughter and other fans call him, first went viral on
SoundCloud in 2015. By 2018, he trailed only the omnipotent Drake on
Spotify's list of most-streamed artists. For a long time I had trouble
distinguishing Post Malone's superhits from everyone else's. Like most
other post-Drake stars, he is an amphibious rap-singer who likes to brag
about his vast wealth and sexual conquests --- except when he is
spending long soulful interludes lamenting exactly those things.

But Post Malone, my daughter helped me understand, is popular as much
for his persona as for his music. He is a superhero of silly, sloppy,
irresponsible ease --- a hard-living, cheerful goofball whose happiness
makes everyone else happy. He seems to smile with extra teeth. Post
Malone is slightly chubby and unkempt, with scuzzy facial hair and
infinite tattoos, including two under his eyes that say, in fancy
script, ``Always Tired.'' (He meant to choose two words of equal length,
he has said, but got it wrong because --- of course --- he was tired.)
Everything he does seems half-accidental. He first learned to play
guitar because he was extremely good at the video game Guitar Hero. He
chose his stage name using an online rap-name generator. (His real name
is Austin Post.) He once tweeted a photo of himself in Australia petting
a kangaroo, with the caption ``Met a koala today.''

This sort of giddy misidentification is, in fact, the key to Post
Malone. He is not exactly a rapper but is also not \emph{not} a rapper.
His musical roots reach down to country, metal, folk and rock ---
online, you can watch him play loving covers of Bob Dylan and Nirvana.
What finally distinguishes Posty from the mainstream crowd is his voice,
which has a touch of rock 'n' roll grit that wrestles through the
Auto-Tune. And yet his megasuccess has mainly come under the umbrella of
hip-hop. He performed his first hit single, ``White Iverson,'' in gold
teeth and cornrows, raising inevitable questions of cultural
appropriation. He says he prefers to think of himself as beyond genre,
which is convenient, because he has sometimes been head-slappingly
inarticulate on the subject. ``If you're looking for lyrics, if you're
looking to cry, if you're looking to think about life, don't listen to
hip-hop,'' he once said, to near-universal disapproval.

Post Malone, in other words, is a big roiling mess of contradictions. No
wonder he is so popular with teenagers. He is the perfect avatar of
adolescence: the excruciating ridiculousness of being a person caught
between worlds, in transition, half-young and half-old, in possession of
powers you don't fully understand, blasting off into inscrutable realms
in which mysterious things will be expected of you. This also makes Post
Malone a perfect fit for Spider-Man, the canonical story of awkward
adolescent empowerment. Posty's latest No. 1 hit, ``Sunflower,'' is not
merely featured on the soundtrack of the franchise's newest iteration,
``Spider-Man: Into the Spider-Verse'': It is actually used, in the film,
to introduce the hero. We meet the teenage Miles Morales in his bedroom,
alone, doodling and bobbing his head to the bouncy hit about a
dysfunctional relationship. (``Callin' it quits now, baby, I'm a
wreck,'' his collaborator Swae Lee sings. ``Crash at my place, baby,
you're a wreck.'') Miles sings along, straying off key, only to be
interrupted by his father, who yells at him to get ready for school. The
awkward teenager is called, awkwardly, out into the world. Amid all the
cringiness, his unexpected superpowers will bloom. Adolescence, despite
its obvious flaws, can still save the world.

\emph{Sam Anderson is a staff writer for the magazine.}

Read more

\includegraphics{https://static01.graylady3jvrrxbe.onion/images/2019/03/10/magazine/10mag-music-slideshow-slide-PQXE/10mag-music-slideshow-slide-PQXE-master180-v3.png}

\hypertarget{--bugs-life}{%
\subsection{\texorpdfstring{ `Bugs
Life'}{  `Bugs Life'}}\label{--bugs-life}}

\hypertarget{tierra-whack}{%
\paragraph{Tierra Whack}\label{tierra-whack}}

\hypertarget{12}{%
\subsection{12}\label{12}}

\includegraphics{https://static01.graylady3jvrrxbe.onion/images/2019/03/10/magazine/10mag-music-slideshow-slide-PQXE/10mag-music-slideshow-slide-PQXE-master180-v3.png}

A 15-Track Fun house album that adds up to a single song.

By JOHN WRAY

\href{https://open.spotify.com/track/4lLdrM9fpLA1Pcng9WNMLp}{\includegraphics{https://static01.graylady3jvrrxbe.onion/newsgraphics/2019/02/26/music/2654e69ed27287529e314df4aede0297716f8c07/icons/spotify.png}}

``Whack World,'' Tierra Whack's debut collection of 15 tracks --- each
clocks in at 60 seconds --- was one of the riskiest gambles of the past
year in pop: It would have been written off as a gimmick if it weren't
so goofily, trippily, intoxicatingly good.

The 15-minute album, which dropped in May with an accompanying
\href{https://youtu.be/EOTebhPy04g}{``Visual and Auditory Project''} on
various video platforms, offers a guided tour of one precociously gifted
23-year-old American's worldview. It is both a brazen bid for the big
time and a disquietingly intimate glimpse inside a wildly idiosyncratic
mind --- in tantalizing, and occasionally maddening, chunks of tightly
rationed time. The experience is a bit like being invited up to the
treehouse of the artsy, slightly spooky girl next door, to discover that
it's vastly larger and more colorful on the inside --- only to be booted
out again before you've done much more than glimpse the décor.

Recent years may have seen the rise of what Pitchfork dubbed the
``super-short rap song,'' with (mostly male) acts like Lil Pump and
Trippie Redd releasing tracks as short as 1 minute 32 seconds; but
``Whack World'' is a different beast entirely. Each track ends after no
more than one minute: some segue seamlessly into the next musical idea,
some cut off in what feels like midverse. (A minute, not coincidentally,
was the maximum length supported by Instagram when ``Whack World'' had
its candy-cane-colored debut.) Repetition is the lifeblood of any pop
song, but most of the album's lyrical fragments barely last long enough
to revisit the first hook, let alone the chorus --- if there is one ---
before we're off to the next room in the fun house.

\includegraphics{https://static01.graylady3jvrrxbe.onion/images/2019/03/10/magazine/10mag-music-slideshow-slide-KL8E/10mag-music-slideshow-slide-KL8E-master180-v2.png}

\includegraphics{https://static01.graylady3jvrrxbe.onion/images/2019/03/10/magazine/10mag-music-slideshow-slide-HQAY/10mag-music-slideshow-slide-HQAY-master180-v2.png}

The difference between ``Whack World'' and an aggressively repetitive
earworm like Lil Pump's hit, ``Gucci Gang,'' can be summed up very
simply: Tierra Whack abhors boredom, and her first line of defense is,
for want of a better word, wackiness. Where Pump mumbles vaguely about
forgetting girls' names while walking down a high school hallway with a
tiger, holding football-size bags of weed, Whack rides a rickety
exercise bike, in a fat suit, singing in a helium-tweaked voice about
lowering her cholesterol. She has the weird kid's ability (and
willingness) to laugh at herself, a quality that her self-professed role
models Missy Elliott and Andre 3000 have always possessed but that is
sorely lacking in a great deal of contemporary hip-hop --- and pop music
in general. It says more than a little about Whack's sensibility, not to
mention her take on relationships, that the album's one and only love
song is about a dead dog.

\includegraphics{https://static01.graylady3jvrrxbe.onion/images/2019/03/10/magazine/10mag-music-slideshow-slide-WCN1/10mag-music-slideshow-slide-WCN1-master180-v2.png}

``Whack World'' the video, with its
Mister-Rogers'-Neighborhood-with-LSD-in-the-water-supply visuals,
invokes Elliott's spirit in another sense too. Tierra Whack is
beautiful, in a normal, human way, but unlike many of her contemporaries
--- Ariana Grande, Chris Brown, Ava Max, Drake --- she's far too
intoxicated by her own hypercreativity to trade on her looks. In the
video's second vignette, which accompanies the playfully morose ``Bugs
Life,'' Whack sits in a nail salon with the right half of her face
grotesquely swollen from an insect bite, singing in a voice dripping
with deadpan irony: ``Probably would've blowed up overnight ... if I was
white.'' Sex may sell out where the rest of us live, but it has
marvelously little currency in Whack's treehouse. Whack --- as opposed
to, say, Frank Ocean --- is by no means a piner. Past romance is
referenced from time to time, but largely in passing, as if the
interesting stuff lay elsewhere. In fact, the album's speed and brevity
feel driven by nothing so much as precocious impatience: You can't
escape the impression that its creator would gladly have contributed
another six, or a dozen, or even 100 additional tracks --- however many
it took to roll up literally everything she'd ever experienced into a
single beautiful, lopsided, sequin-spangled ball.

\includegraphics{https://static01.graylady3jvrrxbe.onion/images/2019/03/10/magazine/10mag-music-slideshow-slide-MNJU/10mag-music-slideshow-slide-MNJU-master180-v2.png}

``Whack World'' subverts accepted notions of ``album'' and ``single'' in
equal measure. In spite of its undeniable of-the-moment-ness, this is
not a collection of music best served by Spotify (or any other
randomized and algorithm-driven playlist). Almost no one, I'm willing to
bet, would be satisfied playing only one of the album's 15 tracks.
Whack's debut comes closest to a distinctly old-fashioned (and
distinctly uncool) staple of popular music: the medley. In a
quarter-hour, she takes us on an attention-deficit-disorder safari
through seemingly every genre and subgenre that has ever turned her
head, from R\&B to trap to doo-wop to reggaeton to --- well, to
something that can best be described as psychedelic anti-country as sung
by Weird Al's manic-depressive little sister. By the time you've finally
begun to acclimate to the sheer magpie-inventiveness of ``Whack World,''
the magical mystery tour is over. And what a short, strange trip it was.

\emph{John Wray is the author, most recently, of ``Godsend: A Novel.''
He last wrote for the magazine
\href{https://www.nytimes3xbfgragh.onion/interactive/2016/03/10/magazine/25-songs-that-tell-us-where-music-is-going.html\#/mac-demarco-blue-boy}{about
Mac DeMarco}.}

\emph{Devin Yalkin is a photographer from New York.}

Read more

\includegraphics{https://static01.graylady3jvrrxbe.onion/images/2019/03/10/magazine/10mag-parquet-courts/10mag-parquet-courts-master180-v2.jpg}

\hypertarget{--before-the-water-gets-too-high}{%
\subsection{\texorpdfstring{ `Before the Water Gets Too
High'}{  `Before the Water Gets Too High'}}\label{--before-the-water-gets-too-high}}

\hypertarget{parquet-courts}{%
\paragraph{Parquet Courts}\label{parquet-courts}}

\hypertarget{13}{%
\subsection{13}\label{13}}

\includegraphics{https://static01.graylady3jvrrxbe.onion/images/2019/03/10/magazine/10mag-parquet-courts/10mag-parquet-courts-master180-v2.jpg}

Cutting through the (bong) smoke to address the \emph{whens} of climate
change.

By LARRY FITZMAURICE

\href{https://open.spotify.com/track/5wpBEy8gYVQmb5VArtayfp}{\includegraphics{https://static01.graylady3jvrrxbe.onion/newsgraphics/2019/02/26/music/2654e69ed27287529e314df4aede0297716f8c07/icons/spotify.png}}

Music has mourned the death of our planet for decades. ``How much more
abuse from man can she stand?'' Marvin Gaye asked in 1971 on ``Mercy
Mercy Me (The Ecology).'' The college-rock astronaut Black Francis sang
of holes in the sky and rising temperatures on the Pixies' ``Monkey Gone
to Heaven'' in 1989, grimly concluding that ``Everything is gonna
burn.'' Four years later, the dance duo Orbital used warning klaxons on
``Impact (The Earth Is Burning)'' to conjure urgency about our impending
global doom. Melissa Etheridge asked ``Have I been careless?'' on a song
called ``I Need to Wake Up,'' from the soundtrack to ``An Inconvenient
Truth.'' Others, perhaps, are resigned to watching the world burn: ``I
wanna see the animals die in the trees,'' Anohni proclaimed in 2016 on
her acerbic indictment ``4 Degrees.''

If you've lost sleep over gigantic holes in Antarctic glaciers or the
drastic decline of insect populations, the last several years have felt
like the final third of Lars von Trier's ``Melancholia,'' in which
humanity awaits Earth's catastrophic collision with another planet:
watching our fate snap back like a boomerang, coming at us faster than
ever with little in the way of prevention or defense. Until recently,
there have been so many perceived wrongs to address on any given day
that climate change has frequently found itself low on the list of
to-do's in our general consciousness; now, not even Demi Lovato and Joe
Jonas --- whose 2010 climate-change anthem, ``Make a Wave,'' claimed
that ``We hold the key that turns the tide'' --- can save us.

Which brings us to the Brooklyn indie rockers Parquet Courts: ``Which
hands get to turn the final page?'' Andrew Savage dryly intones on
``Before the Water Gets Too High,'' a dread-drenched meditation that
skips the \emph{if}s of climate change and heads straight to the
\emph{when}s. How do we prepare for devastation, and can we reckon with
how useless our efforts to stop it have been?

Such questions have largely gone unasked in the indie sphere, especially
as the genre signifier has transitioned over the last decade from ethos
to marketing term. They're new to the oeuvre of Parquet Courts as well;
before the political party-punk of last year's ``Wide Awake!'' the band
spent more time musing about stoned bodega trips and the literal
gathering of dust. But ``Wide Awake!'' found them addressing
sociopolitical concerns including gentrification and groupthink,
structural violence and the aftereffects of apathy, all with the
freaked-out clarity of a wasted reveler realizing that the globe's last
keg is about to be kicked.

The dubby fragrance of ``Before the Water Gets Too High'' suggests that
their bong-session days aren't behind them yet, but it also situates the
song within dub reggae's history of cutting through the smoke to address
the fear involved in confronting societal calamities. ``Glass barely
bends before it cracks,'' Savage lilts on the song's bridge, underlining
the starkest truth of our certain demise: It's coming, and we're all out
of warning signs.

\emph{Larry Fitzmaurice is a writer and an editor in Brooklyn.}

Read more

\hypertarget{--we-appreciate-power}{%
\subsection{\texorpdfstring{ `We Appreciate
Power'}{  `We Appreciate Power'}}\label{--we-appreciate-power}}

\hypertarget{grimes-feat-hana}{%
\paragraph{Grimes feat. HANA}\label{grimes-feat-hana}}

\hypertarget{14}{%
\subsection{14}\label{14}}

An annotated lyrical guide.

Interview by REGGIE UGWU

\href{https://open.spotify.com/track/08ZCgkLKpcgrcSPCHN2v8J}{\includegraphics{https://static01.graylady3jvrrxbe.onion/newsgraphics/2019/02/26/music/2654e69ed27287529e314df4aede0297716f8c07/icons/spotify.png}}

\href{http://nytimes3xbfgragh.onion\#tooltip-1}{We appreciate power
(x2)} When Grimes released ``We Appreciate Power'' in late 2018, a press
statement explained the song's conceit: Inspired by North Korean pop, it
was written from the perspective of a girl group working to advance the
aims of artificial intelligence. ``Simply by listening to this song,''
the statement said, ``the future General A.I. overlords will see that
you've supported their message and be less likely to delete your
offspring.'' (According to Page Six, it was a shared joke about this
type of scenario --- a thought experiment called ``Roko's Basilisk'' ---
that led the singer into a much publicized relationship with the tech
entrepreneur Elon Musk.) We asked Grimes to elaborate.

Elevate the human race

Putting makeup on my face

We appreciate power (x2)

Simulation, give me something good

God's creation, so misunderstood

\href{http://nytimes3xbfgragh.onion\#tooltip-2}{Pray to the divinity}
\textbf{SO, WHAT'S GOING ON WITH THE PRO-A.I. GIRL GROUP?} The lyrics
are so worshipful. There's a subtext that they're kind of scared.

The keeper of the key

One day everyone will believe

What will it take to make you capitulate

We appreciate power (x2)

When will the state agree to cooperate

We appreciate power (x2)

People like to say that we're insane

\href{http://nytimes3xbfgragh.onion\#tooltip-3}{But A.I. will reward us
when it reigns} \textbf{HOW WILL HUMANS AND A.I. COEXIST?} I don't know
if A.I. has as much motivation to wipe out humanity as people think. I'd
be like: ``Oh, I kind of like these humans. They made me.'' I would
probably be like Thanos and kill half the people. Just at random. But
I'd still be interested in humans and maybe want to engage with them in
some nonviolent way.

Pledge allegiance to the world's most powerful computer

Simulation, it's the future

What will it take to make you capitulate

We appreciate power (x2)

Elevate the human race

Putting makeup on my face

We appreciate power (x2)

And if you long to never die

Baby plug in upload your mind

Come on you're not even alive

If you're not backed up on a drive

\href{http://nytimes3xbfgragh.onion\#tooltip-4}{What will it take to
make you capitulate} \textbf{WOULD AN A.I. REALLY WANT TO DELETE US OR
OUR OFFSPRING?} It's a thought experiment that says the A.I. will be
angry at people who do not try their hardest to bring it into existence.
It'll have access to all our texts. It'll have access to the whole
internet and probably absorb that within a few minutes. And it will know
everything about everybody. So it will be angry and punish people who
try to inhibit it. A lot of the development of this type of technology
is in a military context --- we're mostly not developing it for love and
happiness.

We appreciate power (x2)

Elevate the human race

Putting makeup on my face

We appreciate power (x2)

What will it take to make you capitulate

We appreciate power (x2)

\href{http://nytimes3xbfgragh.onion\#tooltip-5}{When will the state
agree to cooperate} \textbf{WOULD AN A.I. MAKE A GOOD LEADER?} I'm not
necessarily positive that A.I. would not be a good form of government.
Like with corruption in government, it's potentially worth taking the
chance of having an A.I. leadership. Because at least it's objective and
probably doesn't care about money. It can just get whatever it wants.
Maybe the A.I. would get Trump's taxes

We appreciate power (x10)

Neanderthal to human being

Evolution, kill the gene

Biology is superficial

Intelligence is artificial

\href{http://nytimes3xbfgragh.onion\#tooltip-6}{Submit} \textbf{HOW CAN
PEOPLE PROVE THEIR FEALTY TO A.I.?} I feel like if you engage with
anything that involves machine learning, even if you're using Alexa,
you're technically giving it knowledge and helping it learn. But the
main people who are going to be saved are the people working to bring it
to fruition. It'll probably spare Google. (x8)

*Reggie Ugwu is a pop-culture reporter for The Times.\\
*

\emph{This interview has been condensed and edited.}

Read more

\includegraphics{https://static01.graylady3jvrrxbe.onion/images/2019/03/10/magazine/10mag-holter/10mag-holter-master180-v8.jpg}

\hypertarget{--i-shall-love-2}{%
\subsection{\texorpdfstring{ `I Shall Love
2'}{  `I Shall Love 2'}}\label{--i-shall-love-2}}

\hypertarget{julia-holter}{%
\paragraph{Julia Holter}\label{julia-holter}}

\hypertarget{15}{%
\subsection{15}\label{15}}

\includegraphics{https://static01.graylady3jvrrxbe.onion/images/2019/03/10/magazine/10mag-holter/10mag-holter-master180-v8.jpg}

Music not made for this world.

By ALEXANDRA KLEEMAN

\href{https://open.spotify.com/track/4VxGlr9PawFDaThx20USnS}{\includegraphics{https://static01.graylady3jvrrxbe.onion/newsgraphics/2019/02/26/music/2654e69ed27287529e314df4aede0297716f8c07/icons/spotify.png}}

``Cerebral pop'' is an uneasy category, like going on a ``working
vacation'' or dressing ``business casual.'' The term betrays a certain
amount of internal struggle: What exactly is this music asking you to do
with your body? Sigh, stare up at the ceiling fan and ponder the song as
if it were a text? Or do what you do when some other tune catches you
--- flail your limbs, move your hips in weird little circles, bob your
head rhythmically up and down? The world was built for pop songs: Public
spaces pump the voices of stars through speakers the way air flows
through ventilation ducts, and that sweet, consistent flavor --- like
Diet Coke or pamplemousse LaCroix --- pairs easily enough with any
modern pastime. But if the territory of pop music is everywhere, how and
where does a piece of art pop --- something equal parts challenging and
engaging --- make its home?

Julia Holter, a Los Angeles-based artist with a background in
composition, answers this question by creating otherworldly spaces in
her own work. On her albums, medieval harmonies and insectlike synths
enjamb; there's a sense that what you're hearing is very ancient and
very modern at the same time, or even that those words no longer matter
at all. This is the sort of world-building on display in Holter's sixth
studio L.P., ``Aviary,'' which takes its title from a line by the
Lebanese-American writer Etel Adnan: ``I found myself in an aviary full
of shrieking birds.''

From its opening --- a cacophony of cymbals and anxiously pacing strings
--- the album is a study in creating a private dwelling place amid the
chaos and uncertainty of the world. On one track it's a place of
emergency, with crisscrossing bagpipes and voice evoking a landscape
pierced by unending sirens; on another it's a dream, with a wintry,
synth-heavy gloom giving way to hazy tenderness, like a Kate Bush song
peered at through depth-obliterating fog. The worlds glimpsed here are
varied, sometimes wildly so, but what they share is the sense that they
are not so much depicting reality as taking inspiration from it,
channeling familiar features into new forms.

Out of this thicket of sound and confusion comes ``I Shall Love 2,'' a
five-minute track that feels like stepping out of a dark forest that
you've been wandering in a daze, into a clearing where you can feel the
full strength of the sun. You may still be lost, but now there's a real
sense of liberation to it. Holter begins with a bare-voiced reading of a
fragment from an old troubadour song --- ``That is all, that is
all/there is nothing else'' --- followed by soft organ and a cantering,
loping beat. There's a lull that lasts far longer than you'd expect
before Holter begins to murmur, low and private, as if musing to
herself. The lyrics are a patchwork --- lines borrowed from troubadour
song and Dante's ``Inferno,'' Holter's own words --- all unified by her
agile delivery, which glides from that soothing murmur to a yelp to a
torch song. One by one, new patterns and phrases are thrust into the
space of the song, lingering and dissipating and overlapping, until
they've become a joyous squall, a raucous flock of melody that's
ecstatic and ethereal and raw all at once.

Holter, in other words, takes the garden path to catharsis, allowing
something uplifting to emerge from the tumult, making chaos resolve
itself into something humane and beautiful and full of intention. And
she has found, even at music festivals and rock clubs, hushed and
attentive audiences for this. (Much like her contemporaries Julianna
Barwick or the Norwegian musician Jenny Hval, Holter's music occupies
that narrow margin of overlap between the stuff you listen to for pure
enjoyment and the stuff you listen to for mental nourishment.) Her
performances are absorbing: They highlight the organic beauty and
authority of her voice, the way the meanings of words can be a sort of
veneer over their untamed musicality. The music rewards more than just
\emph{hearing} it. It rewards some other kind of listening, asking you
to let yourself become porous.

And lately it can fill an appetite that seems both modern and primal at
once: to make whole a fractured attention span, to find a ritual that
works. Our days are full of tiny slivers of time that we offhandedly
cram with music, filling the gaps between tasks and places like someone
idly coloring in a picture. It's satisfying to engage with sounds that
won't fit easily into those pre-existing slots, that dwell strangely in
the everyday, transforming it, opening it up, urging you to conjure the
imaginary, not-yet-existent space in which they might fit. Holter's
music is the opposite of filler: It's a trap door leading to someplace
that's not your own.

*Alexandra Kleeman lives in Staten Island and is the author of the novel
``You Too Can Have a Body Like Mine.''\\
*

Read more

\includegraphics{https://static01.graylady3jvrrxbe.onion/images/2019/03/10/magazine/10mag-maroon5/10mag-maroon5-master180-v8.jpg}

\hypertarget{--girls-like-you}{%
\subsection{\texorpdfstring{ `Girls Like
You'}{  `Girls Like You'}}\label{--girls-like-you}}

\hypertarget{maroon-5}{%
\paragraph{Maroon 5}\label{maroon-5}}

\hypertarget{16}{%
\subsection{16}\label{16}}

\includegraphics{https://static01.graylady3jvrrxbe.onion/images/2019/03/10/magazine/10mag-maroon5/10mag-maroon5-master180-v8.jpg}

The pop star cedes the stage.

By CHARLES AARON

\href{https://open.spotify.com/track/6FRLCMO5TUHTexlWo8ym1W}{\includegraphics{https://static01.graylady3jvrrxbe.onion/newsgraphics/2019/02/26/music/2654e69ed27287529e314df4aede0297716f8c07/icons/spotify.png}}

Maroon 5's success contains multitudes. There may be no better example
of this than the group's pop-soul exhalation ``Girls Like You,'' which
spent seven weeks last year atop Billboard's Hot 100 and 33 weeks in the
Top 10. Though the song began as a demo by the L.G.B.T. artist Starrah,
about her own feelings for a female friend, it was then punched up and
polished by a trio of gnomic writers and producers --- Henry (Cirkut)
Walter, Jason Evigan and Gian Stone --- before the heavily compressed
and reverbed track was spiced up with the most distinctive voice of the
past year, that of Cardi B.

These figures are not the ones who most make ``Girls Like You'' matter,
though. Neither does Adam Levine (who gets a writing credit) or his
happy-to-be-here sidemen who constitute the Maroon 5 touring entity. Nor
do the boisterous Voice of Atlanta gospel choir and Equinox Percussion
drum corps who enlivened the band's performance of the song during this
year's Super Bowl halftime show.

These contributors all gave shape and solidity to Maroon 5's vaporous
cultural presence. But at a time when women and nonwhite men have been
facing down the White House and law enforcement's relentless disrespect
with civic, public and artistic resolve, the most crucial multitude were
the 26 female entertainers, athletes, politicians, activists and
survivors who seized control of
\href{https://www.youtube.com/watch?v=aJOTlE1K90k}{the song's video},
which was the most-watched clip of 2018, with 1.6 billion views.

As the camera circles, Levine stands in the center of a soundstage, arms
by his side, his voice skipping nimbly over the melody. He's the
ultimate embodiment of a stubbly, privileged white bro who belatedly
realizes that he needs his partner. Coming from Levine, the song's
colloquial lyric seems to distill toxic masculinity and the exploitation
of women's emotional labor: ``Maybe it's 6:45/Maybe I'm barely
alive/Maybe you've taken my shit for the last time, yeah/Maybe I know
that I'm drunk/Maybe I know you're the one/Maybe you're thinking it's
better if you drive.''

As the verse-chorus unfolds, Levine is joined one at a time, their backs
to his back, by the 26 women. Dancing and exuding playful charisma, some
acknowledge him; some don't. Then, less than two minutes in, he suddenly
disappears, as if ceding the spotlight. When Cardi B delivers her final
flourish, he returns briefly, but by the end of the video, the
soundstage is occupied by only the women.

Adam Levine is to a rock star as a 2019 rock star is to a 2019 rapper.
In other words, he ain't it. Rather, he's the genial tattooed tycoon who
craves cooler and edgier friends but just can't seem to make it happen.
At 39, he's the shredded-abs dad bod that everybody has seen one too
many times. But in the ``Girls Like You'' video, Levine owns all that.
At least in this moment, he leaves the pocket T-shirt on, keeps the
guitar in the closet and hands the mic to the long-suffering women who
have chosen to support him. For the first time, maybe ever, he flashes
some legit star-power potency.

\emph{Charles Aaron is a writer in Durham, N.C.}

Read more

\includegraphics{https://static01.graylady3jvrrxbe.onion/images/2019/03/10/magazine/10mag-six-nine/10mag-six-nine-master180-v5.jpg}

\hypertarget{--stoopid}{%
\subsection{\texorpdfstring{ `Stoopid'}{  `Stoopid'}}\label{--stoopid}}

\hypertarget{6ix9ine}{%
\paragraph{6ix9ine}\label{6ix9ine}}

\hypertarget{17}{%
\subsection{17}\label{17}}

\includegraphics{https://static01.graylady3jvrrxbe.onion/images/2019/03/10/magazine/10mag-six-nine/10mag-six-nine-master180-v5.jpg}

The new face of gangsta rap broadcasts his crimes and winds up in
handcuffs.

By WILLY STALEY

\href{https://open.spotify.com/track/6k2iq1yULhXILe0i8xQqPF}{\includegraphics{https://static01.graylady3jvrrxbe.onion/newsgraphics/2019/02/26/music/2654e69ed27287529e314df4aede0297716f8c07/icons/spotify.png}}

The first time I saw
\href{https://www.youtube.com/watch?v=gAs9HZC9c7Y}{the video for Tekashi
6ix9ine's ``Gummo,''} I felt a little bit like Rip Van Winkle wandering
down from the Catskills. \emph{What in the world happened here? I was
only gone for an hour!} Some elements were familiar (a crew of guys in
front of a brownstone, drinking and mugging for the camera), and some
were menacing (the number of red bandannas and guns on display), but it
was the man at the center of the video who startled me most; he seemed
almost precision-engineered to make people feel old.

He was screaming himself hoarse about robbing and shooting people, over
a haunting, spare beat, but I'd heard that before --- it was his
appearance that shocked me. In an era when most young rappers have a
couple of face tattoos, 6ix9ine had the number 69 inked above his right
eye in 72-point type. He had the same number spelled out in cursive over
his left eye. It was everywhere on his body. His scraggly hair was dyed
ROYGBIV, and he had a grill to match. He looked like one of those frogs
you're not supposed to touch.

Within about a year, he would be in federal custody, a 22-year-old
facing life in prison for a number of charges, including racketeering
and attempted murder. Law enforcement accused 6ix9ine and others in his
circle of being members of the Nine Trey Gangsta Bloods and, under the
RICO Act, charged them with participating in shootings and robberies all
over New York City. Normally this sort of arrest leads to an outcry
about literal-minded police overreach. Not this time. People generally
seemed pleased to see the rapper in cuffs.

This was partly because 6ix9ine was universally reviled by music critics
and journalists, on account of a crime he committed before he became
famous: In 2015, he pleaded guilty to the use of a minor in a sexual
performance, for having filmed and shared on social media a video of a
girl performing oral sex on his friend. But it was also because he had
spent the past year living the life of a Looney Tunes character:
courting danger, narrowly escaping it, then taunting his foes. This
genuinely incredible run netted him more than 150 stories on TMZ: gang
members in San Antonio threatening his life; a shootout at the Barclays
Center; shots fired at a video shoot in Brooklyn; more shots fired at a
Beverly Hills video set. Through it all, he posted on Instagram, usually
wearing red, often handling bricks of cash, sometimes clutching
extremely illegal-looking guns, but never betraying an ounce of concern
for his well-being.

{[}\href{https://www.nytimes3xbfgragh.onion/2018/11/29/nyregion/tekashi6ix9ine-jail-treyway.html}{More
on
the}\href{https://www.nytimes3xbfgragh.onion/2018/11/29/nyregion/tekashi6ix9ine-jail-treyway.html}{rapid
rise and sudden fall of Tekashi 6ix9ine.}{]}

Cultivating this sort of personal mythology is not at all new; it dates
back to the earliest days of gangsta rap. Ever since Eazy-E bankrolled
NWA with drug money, a certain proximity to criminality has been
expected of certain rappers. If your music purports to document life on
the other side of the law, then you better know what you're talking
about. Not long ago, rappers had just a few limited channels through
which to prove that they did: lyrics, album art and, if they were famous
enough, music videos. Like Old Testament gods, they willed whole
universes into being through their words.

Now they have social media. Case in point: On ``Stoopid,'' which came
out last fall, 6ix9ine insults two rappers he was feuding with at the
time, but unless you were listening for it, you probably wouldn't have
noticed. Much more memorable were the Instagram videos he made to
antagonize the two; in one he took his rival's former girlfriend and
mother of his child out shopping.

This sort of online mythmaking is second nature to SoundCloud rappers,
so called for the streaming service that birthed the scene. SoundCloud
rap is not characterized by a particular sound so much as its anarchic
energy --- the face tattoos, the prescription drugs, the
orthographically complex handles. This scene has been alarmingly fecund
and seems to accelerate evolution in disturbing directions, a little
like Lake Springfield on ``The Simpsons.'' Even within this ecosystem,
6ix9ine stood out like a chumbox thumbnail --- is he ... \emph{real}?

The problem, for 6ix9ine, was that a big part of his adopted persona,
both on Instagram and in his music, involved being a member of the Nine
Trey Gangsta Bloods. According to a
\href{https://www.rollingstone.com/music/music-features/tekashi-69-rise-and-fall-feature-777971/}{Rolling
Stone profile} that came out after his arrest in November, this was
essentially an act: Danny Hernandez, in the years leading up to his
fame, had been a trollish and goofy Bushwick deli employee; his industry
blacklisting had pushed him into the hands of an apparently
gang-affiliated manager, who also provided him with a new edge. 6ix9ine
is far from the only rapper to have ever made dubious claims to being a
Blood, but Instagram has a way of making even the most absurd feints
real. Maybe the whole thing really \emph{was} a put-on, but also, he
really did it. The Rolling Stone article recounts how, at his
arraignment, the presiding judge asked the prosecution how it knew
Hernandez was at real-life crime scenes. ``The answer was often simple:
Hernandez had posted about it on Instagram.''

A liminal space has always existed between rappers and their personas.
Sometimes it's as thin as a fig leaf, but even then it's essential, both
legally and morally. The gap between 6ix9ine and Danny Hernandez was
considerably wider, but he snapped it shut with his phone, merging
fantasy with reality through a front-facing camera. Once intertwined on
a criminal indictment, the two aren't easily unwound. It was reported in
February that 6ix9ine, who pleaded guilty, agreed to help prosecutors in
their case against his co-defendants, hoping for leniency: a reduced
sentence and possibly witness protection. But helping 6ix9ine disappear
into some corner of America might prove difficult, and not just because
of the tattoos.

*Willy Staley is a story editor for the magazine.\\
*

Read more

\includegraphics{https://static01.graylady3jvrrxbe.onion/images/2019/03/10/magazine/10mag-music-slideshow-slide-9FNS/10mag-music-slideshow-slide-9FNS-master180.png}

\hypertarget{--honey}{%
\subsection{\texorpdfstring{ `Honey'}{  `Honey'}}\label{--honey}}

\hypertarget{robyn}{%
\paragraph{Robyn}\label{robyn}}

\hypertarget{18}{%
\subsection{18}\label{18}}

\includegraphics{https://static01.graylady3jvrrxbe.onion/images/2019/03/10/magazine/10mag-music-slideshow-slide-9FNS/10mag-music-slideshow-slide-9FNS-master180.png}

Talking to the pop star about approaching life as if you're seducing it.

Interview by LAURA SNAPES

\href{https://open.spotify.com/track/3olYZQTYOLYNImERBnZZqm}{\includegraphics{https://static01.graylady3jvrrxbe.onion/newsgraphics/2019/02/26/music/2654e69ed27287529e314df4aede0297716f8c07/icons/spotify.png}}

In 1993, the Swedish singer-songwriter Robyn turned 14 and finished
middle school; then she signed a record deal. She has been making music
ever since, starting with her 1995 debut album, ``Robyn Is Here,'' which
netted her two Billboard Hot 100 singles. Her 2005 album, ``Robyn,''
cemented her as a beloved, idiosyncratic pop star and earned her a
Grammy nomination. She released a trilogy of mini-albums in 2010, which
included some of her biggest hits --- ``Dancing on My Own'' and ``Call
Your Girlfriend.'' Now, after a long gestation period, she has returned
with a new album and single, ``Honey.''

\emph{You've described the making of this song as searching for a
feeling.} A feeling of healing from sadness and wanting to share that
with the world and with myself --- a sense of self-love, excitement,
some kind of peace of mind. Like when your strength is coming back. I
wanted to describe this feeling like when you're super close to someone
else's skin, or when you're falling asleep, or when you've taken a drug
that's starting to kick in --- sensations that are pleasurable, where
your body is reacting to something outside of you.

\emph{I felt naïve because I thought it was about self-discovery, but a
friend said, ``No, it's about oral sex!''} It could be about oral sex,
sure! Intimacy, definitely, but it could be with yourself. It doesn't
have to do with another person. In a way I think your analysis is
probably the best one, because isn't oral sex self-discovery? Any
experience you have that will give you a new point in your scale of
emotions will make any other experience richer because you have a new
point of reference.

\emph{Maybe it's about elevating self-care to the level of eroticism?
Not reserving that deep pleasure for a sexual sensation, but something
you could experience day to day.} Exactly! That's exactly it. It's like
approaching life as if you're seducing it. As if you're seducing
yourself, or the other person, or your work, or whatever. Intimacy in
every little thing.

\includegraphics{https://static01.graylady3jvrrxbe.onion/images/2019/03/10/magazine/10mag-music-slideshow-slide-2RI5/10mag-music-slideshow-slide-2RI5-master180-v2.png}

\includegraphics{https://static01.graylady3jvrrxbe.onion/images/2019/03/10/magazine/10mag-music-slideshow-slide-UX3W/10mag-music-slideshow-slide-UX3W-master180-v2.png}

\emph{How do you approach life as if you're seducing it?} I feel like I
have to work for it every day. It's like starting a little fire: You
have to be so careful with it, not force it. You get it going and then
you can use it and tend to it and start it back up again.

\emph{Is your fire well tended?} Not at all. I'm not in a great place at
the moment, to be honest. It's a good conversation for me to have, a
good reminder. I maybe need to go back and listen to some of my songs
myself to figure this out.

{[}\href{https://www.nytimes3xbfgragh.onion/2018/09/21/arts/music/robyn-honey-interview.html}{How
Robyn,}\href{https://www.nytimes3xbfgragh.onion/2018/09/21/arts/music/robyn-honey-interview.html}{pop's
glittery rebel, danced her way back from darkness.}{]}

\emph{Your songs are known for intermingling sadness and euphoria. But
``Honey'' is less triumphant than that.} I don't think life is that easy
anymore. I used to believe it would all make sense if you just powered
through. I really don't feel that way anymore at all. I think it's
pragmatic --- I really don't feel pessimistic.

\includegraphics{https://static01.graylady3jvrrxbe.onion/images/2019/03/10/magazine/10mag-music-slideshow-slide-URXR/10mag-music-slideshow-slide-URXR-master180-v2.png}

\emph{The idea of ``self-care'' can seem like a challenge.
Post-recession capitalism has glorified the hustle so much.} The society
we live in at the moment --- we didn't really make it very good, you
know? It's really tough to take care of yourself. Self-care is difficult
when you have a 9-to-5 job, and people are on Instagram projecting these
images of themselves --- we're in a time where everyone's engaged in
storytelling in such an aggressive way. Stories are amazing --- they're
part of what it is to be human --- but you have to be aware they're
stories. That's the problem with capitalism: It's always trying to trick
you into thinking there's a purpose for all of this. But you can
actually use a story that relates to something more real than buying
yourself out of anxiety.

\emph{It's jarring to see people turning themselves into personal brands
as part of that storytelling.} That's the strange thing about being an
artist. There's always a gap between how I see myself and how other
people see me. When I was younger, I felt like it was easier for me to
absorb other people's thoughts about me, and more complex to understand
what was me and what was other people, and how to respond to those
expectations. I'm still learning that. I've been more specific with my
expression on this album than I've been able to be before, but I'm aware
that that doesn't mean that people understand me. I don't even know if
that's the point.

\includegraphics{https://static01.graylady3jvrrxbe.onion/images/2019/03/10/magazine/10mag-music-slideshow-slide-5U1W/10mag-music-slideshow-slide-5U1W-master180.png}

\emph{A lot of today's pop is about nihilism, depression, anxiety.
There's real sadness behind your album, too. It struck me as something
that's easy for the industry to exploit.} Definitely: Pop at the moment
is depressing. Hip-hop is really dark. The music kids are listening to
is heavy! Maybe it's hard to be positive and optimistic at the moment,
for a lot of young people --- the number of voices that are trying to
share this space in their brains is crazy. But maybe it's cool that they
get to explore their darkness as a collective earlier in life, because
maybe it will also get them to the point where they're letting go of it
early as well.

\emph{Is the industry set up for artists to be able to share their pain
but protect themselves?} It's really difficult for anyone to not be
exploited in our society as a whole. As an artist, you're exposed,
because you deal in emotions all the time. People want you to be
vulnerable. If you think about sports --- you look at it because
someone's pushing themselves to their limit. If you don't risk anything,
why would people want to see it happen?

\emph{You turn 40 this June. You could listen to ``Honey'' as a
celebration of the richness that comes with age, and maybe aging as a
woman.} I think it can be that, for sure. I'm not looking forward to
being 40, to getting older. I know that's not a very mature thing to
say, but it is what it is.

\includegraphics{https://static01.graylady3jvrrxbe.onion/images/2019/03/10/magazine/10mag-music-slideshow-slide-0A5B/10mag-music-slideshow-slide-0A5B-master180-v2.png}

\emph{Laura Snapes is the deputy music editor of The Guardian.}

\emph{Devin Yalkin is a photographer from New York.}

\emph{This interview has been condensed and edited.}

Read more

\includegraphics{https://static01.graylady3jvrrxbe.onion/images/2019/03/10/magazine/10mag-rosalia/10mag-rosalia-master180-v5.jpg}

\hypertarget{--malamente}{%
\subsection{\texorpdfstring{
`Malamente'}{  `Malamente'}}\label{--malamente}}

\hypertarget{rosaluxeda}{%
\paragraph{Rosalía}\label{rosaluxeda}}

\hypertarget{19}{%
\subsection{19}\label{19}}

\includegraphics{https://static01.graylady3jvrrxbe.onion/images/2019/03/10/magazine/10mag-rosalia/10mag-rosalia-master180-v5.jpg}

Who knew that flamenco --- and a medieval love-gone-wrong tale --- could
be so relevant?

By MARTA BAUSELLS

\href{https://open.spotify.com/track/1B0BQaSRHxhI0AUlItY3LK}{\includegraphics{https://static01.graylady3jvrrxbe.onion/newsgraphics/2019/02/26/music/2654e69ed27287529e314df4aede0297716f8c07/icons/spotify.png}}

It was hard to tell how many people in the club liked flamenco, an art
form not much associated with young people anymore. But on New Year's
Eve in Barcelona, when the \emph{palmas} hand claps that open Rosalía's
``Malamente'' started pulsing from the speakers, looks of recognition
flashed from face to face. And by the time Rosalía --- an artist from
the nearby town of Sant Esteve Sesrovires --- started rapping, everyone
was clapping along. Some of the younger girls even twerked.

It's not unheard-of for flamenco to make incursions into mainstream
Spanish pop culture, but Rosalía is different. She sounds and feels
cosmopolitan, cool in a sophisticated and almost \emph{foreign} way. She
doesn't sound as if she's playing just for us. And indeed, she isn't:
This two-and-a-half-minute song, with its balmy synth bass and
irresistible syncopations and enticing whispers peppered throughout, has
racked up more than 59 million views on YouTube. Her success, she has
argued, is partly because of listeners' weariness with the
homogenization --- which is to say, the Anglicization --- of pop. Her
own aesthetic is polished, globally recognizable, informed by hip-hop
and trap music. But it's also saturated with bullfighting and windmills
and fire. Her sophomore album, ``El Mal Querer'' (which roughly
translates as ``loving badly''), is a conceptual project patterned on
the subversive medieval text ``Flamenca.'' The last frames of the
\href{https://www.youtube.com/watch?v=Rht7rBHuXW8}{``Malamente'' video}
feature a figure wearing the conical \emph{capirote} hat worn by
penitents during Holy Week --- while also riding a skateboard.

``Malamente'''s appearance in the club also seemed to inspire a few eye
rolls. Maybe this is the price of success in a culture that looks
askance at overt displays of ambition or self-actualization, especially
by women. By New Year's Eve, Rosalía had come up in every conversation
I'd had for days --- she was as interesting to my grandmother as she was
to my 12-year-old cousin. The local fascination tended to focus less on
her art and more on her as a phenomenon, on the extraordinary speed of
her rise to stardom. It would spark arguments too, about cultural
appropriation and the Romany community, who have always been closely
associated with flamenco.

``Malamente'' means ``badly,'' and the story of ``Flamenca'' is not a
happy one. A woman gets married to a man who later grows jealous and
imprisons her. It's not surprising, these days, that a global star would
be singing about such suffering. But Rosalía reminds us that pain and
possession are topics that have interested humans for centuries.
Millions of her listeners must be enjoying this music without
understanding the language, but right now, for me, it feels cathartic
--- soothing, even --- to be able to stand in a crowd singing ``badly,
badly, so badly'' while clapping my hands.

*Marta Bausells is a writer in London and Barcelona.\\
*

Read more

\includegraphics{https://static01.graylady3jvrrxbe.onion/images/2019/03/10/magazine/10mag-marie-davidson/10mag-marie-davidson-master180-v3.jpg}

\hypertarget{--work-it}{%
\subsection{\texorpdfstring{ `Work It'}{  `Work It'}}\label{--work-it}}

\hypertarget{marie-davidson}{%
\paragraph{Marie Davidson}\label{marie-davidson}}

\hypertarget{20}{%
\subsection{20}\label{20}}

\includegraphics{https://static01.graylady3jvrrxbe.onion/images/2019/03/10/magazine/10mag-marie-davidson/10mag-marie-davidson-master180-v3.jpg}

The singer talks about what it costs us, having to do all that work.

Interview by HAZEL CILLS

\href{https://open.spotify.com/track/5QnzAZ2W3Ap3gSiO2vsGQG}{\includegraphics{https://static01.graylady3jvrrxbe.onion/newsgraphics/2019/02/26/music/2654e69ed27287529e314df4aede0297716f8c07/icons/spotify.png}}

Last fall, Marie Davidson, a French Canadian producer of electronic
music, released ``Work It,'' an aggressively rattling track on which she
serves ``kind of like an M.C./personal trainer,'' exhorting listeners
not to dance but to work, constantly and sweatily, seven days a week.
The song became a minor anthem, and the album it came from, ``Working
Class Woman,'' was embraced as one of the best dance records of the
year.

\emph{What sort of place were you at in your life when you wrote this
song?} Obviously I was working a lot. I had already toured Europe and
the U.S., and I was going back to Europe to play festivals. I wanted to
make a banger to play live --- I just picked up my microphone and
started talking. The song came out in a funny way, but the undertone is
serious.

I don't think all people should work all the time. It's sarcastic, it's
humorous, but in the joke there's also truth. It's why I say, toward the
end, that when I say ``work,'' you have to work for yourself, love
yourself, feed yourself. Whatever you do, whatever amount of energy you
put into something, you have to do it for yourself and not to please
others. Not to build this facade or this persona or achievement.

\emph{Do you think people base too much of their self-worth on their
work?} Absolutely. We live in a society that is based on work --- goals,
achievement, money. Of course! It's a real problem --- workaholism,
exhaustion, burnout. We're not taught in this society --- it's not
valued --- to take time and just learn to love yourself. You're taught
to be strong, to look smart, to be useful. But I think you become a much
more useful person if you learn how to love yourself.

\emph{Do you think people who aren't musicians are aware of how
physically taxing it can be?} It would be hard to know. When you look at
pictures or videos, it doesn't look like that. It looks really fun and
glamorous. And it is, sometimes, for a few hours. But it's not a
glamorous job being a touring musician. I still have people come up to
me after a gig or somewhere, saying: ``You're so lucky. I wish I had
your life.'' And I'm like: What do you mean by that? Do you think I woke
up one morning and became who I am? It's years and years of work and
trial and error. It's a real job, like being a carpenter or being a
designer.

\emph{People think of the dance floor as this freeing space. But you're
kind of pointing out that this is just another workplace.} For me, at
least, it is. It used to be different. When I was 16 and I started going
out in Montreal, going to underground parties and raves and clubs, it
was magical. But it was different: I wasn't working. It wasn't my job. I
was going there for fun. Even if I was playing, it was special. It
wasn't in the schedule, where you just do it over and over again. That
space is now a work space for me. There's no more leisure. I've
sacrificed that freedom for other things. Now if I want to feel
something mind-blowing or magical, I have to look for it outside of club
culture. The music never loses its magic, but the social thing happening
at a party or something like that? No.

\emph{It sounds as though the song stemmed from your personal
experience, but it feels universal.} When I made it, I knew anyone could
relate. Because this is the time we live in. Everything goes really fast
now. People are expected to produce and achieve. It's kind of sad, I
think. There's a rigidity in that mind-set about what you should be
doing with your life. We're ruled by capitalism, so that's what we get.
We are living comfortably, and people feel that they are free, but I
don't think that individuals are that free in society.

\emph{So how do you make art under capitalism?} I don't care. When I'm
in my studio composing music, I really don't think about that --- about
what is the potential of that song, where is it going to take me, what
does it mean to the world. I don't care. I never did. I don't think
about my label or my manager or anybody. My insecurities come after,
when it's time to be in the world.

\emph{Interview has been condensed and edited.}

*Hazel Cills is the pop-culture reporter at Jezebel.\\
*

Read more

\includegraphics{https://static01.graylady3jvrrxbe.onion/images/2019/03/10/magazine/10mag-music-slideshow-slide-42XF/10mag-music-slideshow-slide-42XF-master180.png}

\hypertarget{--assume-form}{%
\subsection{\texorpdfstring{ `Assume
Form'}{  `Assume Form'}}\label{--assume-form}}

\hypertarget{james-blake}{%
\paragraph{James Blake}\label{james-blake}}

\hypertarget{21}{%
\subsection{21}\label{21}}

\includegraphics{https://static01.graylady3jvrrxbe.onion/images/2019/03/10/magazine/10mag-music-slideshow-slide-42XF/10mag-music-slideshow-slide-42XF-master180.png}

The british singer-songwriter who turned homage into a signature style.

By JONAH WEINER

\href{https://open.spotify.com/track/6cJrQc34txUzXd90xWEAOh}{\includegraphics{https://static01.graylady3jvrrxbe.onion/newsgraphics/2019/02/26/music/2654e69ed27287529e314df4aede0297716f8c07/icons/spotify.png}}

James Blake was talking about \emph{shanzhai}, a Chinese phrase that
means, roughly, ``knockoff,'' and that stems from the time-honored
tradition in Chinese art of apprentices' mimicking their masters ---
``copying something until you can better it,'' he said, ``or if not
better it, compete with it.''

Blake, a Grammy-winning avant-gardist with an ear for pop, who has been
playing the piano since he was about 6, has a long list of heroes whom
he has studiously copied in pursuit of his own sound. Copying the
virtuoso jazz-pianist Art Tatum, the protominimalist French composer
Erik Satie and the midcentury gospel maestro the Rev. James Cleveland
taught Blake novel ways of opening up complex chord structures and
fitting them --- to gorgeous, aching effect --- around deceptively
simple melodies. Copying singer-songwriters like Joni Mitchell and
Stevie Wonder emboldened him to write and sing pop songs with increasing
emotional candor. Copying the hip-hop producer Timbaland and the South
London dubstep craftsman Mala informed Blake's love of distorted timbres
--- electronic rhythms that skitter and twitch as often as they groove
--- and of sampling, a technique of literal copying (and cutting and
pasting) so central to Blake's practice today that, even when he isn't
sampling anyone else, he routinely samples himself. ``I'll sit at a
piano, improvising'' into a microphone, he explained, then search the
tape for building blocks that he can pluck out and, with editing
software, arrange into new compositions.

Blake stands at an imposing 6-foot-6 and carries himself with the
deliberateness of a man at risk of scraping his head on doorways. Early
in February, in a wood-paneled rehearsal space in the Larchmont
neighborhood of Los Angeles, he sat at a Prophet '08 synthesizer
opposite Rob McAndrews, a multi-instrumentalist, and Ben Assiter, a
drummer, his longtime friends and touring bandmates. At their feet,
black cables snaked and cloverleafed among clusters of red-, blue-,
silver- and cream-colored effects pedals, like tracks connecting
villages in a model-train set.

Blake is from London, but three and a half years ago he began dating the
British comedic actress Jameela Jamil, and when she booked a lead role
on the NBC sitcom ``The Good Place,'' he tagged along with her to
California. Now Blake led McAndrews and Assiter through a practice run
of ``I'll Come Too,'' a swooning new song about joining Jamil out west
--- about that moment when you want to follow a crush anywhere. The song
comes from his fourth LP, ``Assume Form,'' which was released in
January. ``That's a fun one to play,'' he said when they finished. ``I
really like it. But it's hard --- I'm singing all the way through, in
all these different ranges, and I don't get a chance to breathe. When I
recorded it, I broke the vocal up.''

\includegraphics{https://static01.graylady3jvrrxbe.onion/images/2019/03/10/magazine/10mag-music-slideshow-slide-TSJB/10mag-music-slideshow-slide-TSJB-master180-v2.png}

The extent to which Blake has digested the lessons of his musical heroes
is illustrated not only by his decade-spanning run of singles, EPs and
albums but also by the number of pop auteurs who have collaborated with
him. Frank Ocean, Beyoncé, Kendrick Lamar, Jay-Z, André 3000 and Travis
Scott are among those artists who have enlisted Blake's assistance. As
an influence and a collaborator, Blake has helped shape two of the more
striking trends in contemporary pop: beats that mutate over the course
of a song, resisting any traditionally identifiable center, and an
emotional atmosphere in which the line between hedonism and melancholy,
bliss and despair comes undone.

His influence extends even to artists he hasn't worked with directly. In
2011, I visited Drake --- a pop giant whose entire musical project has
been about smudging the line between hedonism and melancholy --- at a
converted Toronto warehouse, where he was working on his second album
with his musical right hand, the producer known as 40. The room was free
of decoration save for 40's copy of Blake's debut LP, perched on a
windowsill like a talisman. Drake later sampled unreleased music of
Blake's on a single called ``0 to 100/The Catch Up.'' Blake, who hadn't
approved this usage, got his part removed. At the start of his career,
Blake told me, he'd been a solitary, ``solipsistic'' bedroom producer.
``I was so protective of my ideas --- so militantly and obsessively
protective,'' he explained, that it wasn't until years later, when ``I
found personal openness, for maybe the first time in my life, that I
found collaborative openness, too.''

\includegraphics{https://static01.graylady3jvrrxbe.onion/images/2019/03/10/magazine/10mag-music-slideshow-slide-IVSB/10mag-music-slideshow-slide-IVSB-master180-v2.png}

\includegraphics{https://static01.graylady3jvrrxbe.onion/images/2019/03/10/magazine/10mag-music-slideshow-slide-HM33/10mag-music-slideshow-slide-HM33-master180-v2.png}

``Assume Form'' is an album expressly about getting out of one's skull.
Five-odd years ago, Blake suffered from a depression so severe that he
considered suicide. ``Assume Form'' chronicles his escape from that
depression, and its title track, which leads the album, plays like a
statement of purpose. Blake described it to me as ``a song about
intimacy --- about feeling like you don't deserve intimacy and feeling
like there's no way somebody could really want it with you.'' The song's
themes are sex, shame and --- by the end --- happiness. ``It's by far
the most vulnerable, exposed song on the entire record,'' Blake said.
``It's saying, If you're comfortable with this, then the bar has been
set.''

Blake was two and a half weeks into rehearsals for a tour that would
take him around the country and then around the world. He and his band
manned a variety of instruments, mostly electronic: the Prophet '08; a
large, modular synthesizer, studded with knobs, plugs and blinking
lights, which McAndrews operated; and a rubber-covered pad, nestled
between cymbals and high-hats, that Assiter thwacked with drumsticks to
trigger pulses and prerecorded percussive sounds. Translating and
arranging Blake's intricate studio creations for live concerts is a
tricky task made trickier still by his refusal to use a laptop. ``With
live electronic music, I've seen the man-checking-emails situation so
many times,'' Blake said, referring to that category of lackluster
performance in which artists stand unbudgingly at MacBooks, pecking at
keys. ``I want an audience member to see what's happening. So if Ben is
playing the kick and Rob is over here fiddling with knobs and it's clear
that what he's doing is changing the sound, then it makes sense.''

\includegraphics{https://static01.graylady3jvrrxbe.onion/images/2019/03/10/magazine/10mag-music-slideshow-slide-H7B2/10mag-music-slideshow-slide-H7B2-master180-v2.png}

The trio consulted a whiteboard upon which the new album's dozen songs
were listed. ``Should we try `Assume Form?''' Assiter asked. Blake
furrowed his brow. The new album's title track is, structurally
speaking, strange. As its lyrics switch between optimistic vows of
commitment and confessions of insecurity, this duality is echoed in the
music, which consists of two alternating piano motifs --- one
shimmering, the other overcast. The track began as a long, meandering
improvisation from which Blake eventually sampled two disparate chunks,
putting them into jarring conversation. When I asked Blake to explain
the harmonic unease between these two parts, he replied: ``The song
essentially modulates from D major in the first section to B flat in the
second, which are unrelated keys. The first section has the tonic as the
bass note, which gives it this firmly rooted presence, whereas the other
section has the third in the bass, which makes it feel suspended ---
which is when the lyrics turn to self-doubt.'' He thought for a moment.
``It's quite an odd pairing. It doesn't work, musically. But there's a
harmony there. Joni Mitchell does something similar on `My Old Man,'
where it goes into a minor key when she sings, `But when he's gone.
...''' Complicating things further, Blake's piano playing on the track
was ``loose --- like, microseconds'' offbeat, he said, which would add
yet another layer of difficulty when it came to replicating ``Assume
Form'' live. ``Let's try something else,'' he told Assiter.

Blake was raised by his father, James Litherland, a singer-songwriter
and guitarist with a prog-rock pedigree, and his mother, a graphic
designer and cycling instructor, in Enfield, a North London suburb. He
described his life from adolescence on as largely unhappy, warm and
supportive parents notwithstanding. ``I had anger,'' he said. ``A lot of
it.'' As a teenager, ``I had terrible acne, and now I think it was a
physical presentation of how angry I was.'' What angered him? ``My
treatment by other people,'' he said. ``People I'd been open with.
Romantic and personal betrayals. And just a feeling of persecution. I
didn't fit in, and I resented not fitting in. I had a couple friends
outside of school, so I wasn't totally alone, but I remember years of
sitting by myself at lunch. Or just going home because I didn't want to
suffer in public.'' Rather than getting out this anger, he repressed it.
``As a child I was very open and emotionally vulnerable and said how I
felt,'' he recalled. ``And I sort of had that beaten out of me --- not
physically but emotionally. At school, and in British society, there was
this sense of, if you don't behave in the right way --- if you're not
coolheaded --- then people punish you for that. So that was my
childhood, that reflex being stamped out of me. And it stayed with me
well into my 20s.''

\includegraphics{https://static01.graylady3jvrrxbe.onion/images/2019/03/10/magazine/10mag-music-slideshow-slide-C3CB/10mag-music-slideshow-slide-C3CB-master180-v2.png}

\includegraphics{https://static01.graylady3jvrrxbe.onion/images/2019/03/10/magazine/10mag-music-slideshow-slide-FC9P/10mag-music-slideshow-slide-FC9P-master180-v2.png}

Instead, Blake said, music ``started to be my first language'' and
``talking about how I felt was second or third.'' In 2007 he enrolled in
the Popular Music program at Goldsmiths University of London. As
important as his classes were the nighttime excursions he took to clubs
like Plastic People and Mass. There, Blake discovered a community of
producers and D.J.s --- black, white, ``from all parts of London,'' he
said --- who were busy hashing out the oblong dimensions of dubstep, a
new kind of dance music defined by cavernous reverb, sinister drones and
deceptively complex rhythms in which growlingly deep bass lines were the
primary propellant. Whereas an amped-up version of dubstep soon grew
into a global phenomenon, throbbing in GoPro commercials and glitzy Las
Vegas clubs, it was more subtle in its dynamics at first. Its architects
assumed gnomic pseudonyms like Coki, Skream and Loefah and tended to
direct attention away from themselves and toward the dance floor. ``You
can imagine how that appealed to me at the time,'' Blake said.

On small but influential labels, he began releasing his own
dubstep-inspired songs marked by his sophisticated harmonic sense.
Initially he sampled other people's vocals, but later he began singing
himself. Blake eventually became comfortable enough to let his voice ---
a lovely, quaking baritone --- ring out clearly, but at first he
distorted it self-protectively: ``I intentionally disembodied my
voice,'' he said, explaining that ``there's a safety in alteration,
because it no longer has to represent the natural resonance that came
out of your chest.''

\includegraphics{https://static01.graylady3jvrrxbe.onion/images/2019/03/10/magazine/10mag-music-slideshow-slide-8GT9/10mag-music-slideshow-slide-8GT9-master180.png}

In 2011, his debut album, ``James Blake,'' was shortlisted alongside
releases by Adele and P.J. Harvey for Britain's prestigious Mercury
Prize. The album's fans included Bjork, who attended Blake's concerts;
Thom Yorke, who dropped Blake's songs into D.J. mixes; and Brian Eno and
RZA, who collaborated with Blake on ``Overgrown'' (2013), which was also
shortlisted for a Mercury (this time he won). Success didn't cure
Blake's long-seething unhappiness, however. Around this time, he told
me: ``I was at home, playing PS4 and smoking weed, which I'd been doing
a lot, and I started to have these hallucinations of black. The screen
stopped being the game and started being the void. I had physical
tremors and panic attacks and had to go to my room and just lie there. I
started saying things like, What's the point? What's the point of me?''

He was having trouble writing new music, which inspired an existential
dread in him. He remembered thinking, That's my only expression as a
human being, so I may as well not be here. He never took concrete steps
to kill himself, he said, ``but I was thinking about it a lot. Thinking
about nothingness. I was just despondent. It didn't matter to me whether
I was here or not. So I was at that point. And I was caught just in
time.''

It was Jamil who caught him --- she, more than anyone else in his life,
Blake said, helped him to break free of his self-destructive tendencies,
prodding him to speak up when he grew sullen and requiring complete
emotional transparency. ``Nothing else will do,'' he said. ``In a
relationship, something rots unless there's full disclosure.'' He also
saw a therapist specializing in a technique known as eye-movement
desensitization reprocessing, or E.M.D.R., in which a patient recalls
unhappy memories while moving the eyes back and forth rapidly and
tapping the hands in time --- reconfiguring a relationship to past
traumas, in effect, by setting them to a beat.

Blake's default mode on ``Assume Form,'' which he calls ``essentially a
pop album,'' is one you might call cautious euphoria. It is also the
most collaborative thing he has put out, featuring guest production from
the Atlanta-based hip-hop luminary Metro Boomin and a series of duets
with stars like Travis Scott, André 3000 and the breakout Catalan singer
Rosalía. It would be important, Blake said, when playing these songs
live, to carve out room for improvisatory runs. ``After seven shows, you
almost feel like you're on rails, and I hate that,'' he said. ``Feeling
like I don't have a choice. With tours you're already so tightly
scheduled in all other regards, that's when I start to get anxious,
feeling like I'm locked in.''

\includegraphics{https://static01.graylady3jvrrxbe.onion/images/2019/03/10/magazine/10mag-music-slideshow-slide-VX68/10mag-music-slideshow-slide-VX68-master180.png}

The band moved next to a song called ``Where's the Catch,'' on which
Blake and André 3000 sing and rap, respectively, about what Blake
characterized as ``a distrust of really great things: `Everything's
seemingly nice --- for today. But what's around the corner?''' Rather
than play André's verse all the way through at concerts, the band
intended to chop up his vocals and distribute them across the track in
stuttering bursts. As they rehearsed the song, Blake's vocals, too, were
chopped up and fed through McAndrews's synthesizer, where he warped them
even further. The trio rode out the song with a jam session, adding
layer after layer of noise on their way to a squalling crescendo.
Blake's vocals grew increasingly refracted and spectral, and there was
something both frightening and exhilarating about the sound of the
electronics swallowing him up whole.

\emph{Jonah Weiner is a contributing writer for the magazine. His last
feature for the magazine was about
\href{https://www.nytimes3xbfgragh.onion/2018/11/29/magazine/adam-mckay-dick-cheney-vice.html}{the
director Adam McKay}.}

\emph{Devin Yalkin is a photographer from New York.}

Read more

\includegraphics{https://static01.graylady3jvrrxbe.onion/images/2019/03/10/magazine/10mag-the-1975/10mag-the-1975-master180-v4.jpg}

\hypertarget{--love-it-if-we-made-it}{%
\subsection{\texorpdfstring{ `Love It If We Made
It'}{  `Love It If We Made It'}}\label{--love-it-if-we-made-it}}

\hypertarget{the-1975}{%
\paragraph{The 1975}\label{the-1975}}

\hypertarget{22}{%
\subsection{22}\label{22}}

\includegraphics{https://static01.graylady3jvrrxbe.onion/images/2019/03/10/magazine/10mag-the-1975/10mag-the-1975-master180-v4.jpg}

A shameless polymath pop act delivers the
cascading-contemporary-references track we didn't know we needed.

By STEVEN HYDEN

\href{https://open.spotify.com/track/6WmIyn2fx1PKQ0XDpYj4VR}{\includegraphics{https://static01.graylady3jvrrxbe.onion/newsgraphics/2019/02/26/music/2654e69ed27287529e314df4aede0297716f8c07/icons/spotify.png}}

`Love It if We Made It'' is a Jedi mind trick. \emph{This} is \emph{the
track with contemporary relevance that you're looking for. You} will
\emph{consider it a statement that mimics the nonstop rattle of social
media and the slow drip of Trump-era anxiety.} The song, though, is
decidedly less subtle than the wave of Obi-Wan Kenobi's hand: Matty
Healy, the gregarious lead singer of the English pop-rock band the 1975,
delivers his litany of contemporary pop-culture references in an
impassioned emo-dude howl, over an unchanging and relentless synth-pop
stomp that sounds like New Order paying homage to the Fall. Donald
Trump, Kanye West, Eric Garner, Lil Peep, the failures of modernity, the
ubiquity of fake news, Healy's own heroin addiction --- it's all
accounted for with the maniacal thoroughness of Billy Joel straining to
rhyme ``Zhou Enlai'' with ``Bridge on the River Kwai.''

But ``Love It if We Made It'' no longer makes me think about the
\emph{Sturm und Drang} of right now, or about millennials, or about
being extremely online. It makes me think of one of the most popular
rock bands of the '70s and '80s.

I blame ``Bohemian Rhapsody.'' It's tempting to view the runaway
box-office success of the Freddie Mercury biopic --- with a worldwide
gross of \$860 million and counting, it's the most popular film about a
real-life musician ever made --- as an indication that the public yearns
for the return of bands that will (stomp) will (stomp) rock you (clap).
But overt revival acts --- like the popular Michigan act Greta Van
Fleet, whose members all dress exactly as Brian May did in 1973 --- tend
not to sonically resemble Queen beyond its first single, the quicksilver
Led Zeppelin nod ``Keep Yourself Alive.'' No, the contemporary band that
comes closest to emulating Queen's whole legacy --- as a shameless
polymath pop act, willing to try anything, including but not remotely
limited to metal, disco, rockabilly, theatrical English music-hall,
folk, synth-pop and power ballads that turn unexpectedly into mock-opera
freakouts --- is the 1975. (Perhaps Greta Van Fleet should have called
themselves the 1973.)

Here's a scene that's not in ``Bohemian Rhapsody'': In 1978, while
making Queen's seventh album, ``Jazz,'' Freddie Mercury was inspired to
write his own litany-of-pop-culture-references song. He decided that
``Jaws'' wasn't his scene, and he didn't like ``Star Wars.'' He was also
skeptical of God, Peter Pan, Frankenstein and Superman, but staunchly
pro-drugs. There was no such thing as logging off back then, so his
symbol of freedom and release was an old-fashioned one: bicycling. ``I
want to ride it where I like,'' Mercury sang, as always committed to
nothing else above his own liberation. ``Bicycle Race,'' a worldwide
hit, is a perfect example of how Queen routinely violated one of the
central truisms of rock, later articulated by David St. Hubbins in the
1984 film ``This Is Spinal Tap'' --- ``It's such a fine line between
stupid and clever.'' The implication is that striving for transcendence
is admirable, but it can easily embarrass you.

{[}\href{https://www.nytimes3xbfgragh.onion/2018/10/11/arts/music/the-1975-matty-healy-interview.html}{Can
a}\href{https://www.nytimes3xbfgragh.onion/2018/10/11/arts/music/the-1975-matty-healy-interview.html}{quote-unquote
band drag rock into the future? The 1975 is trying its hardest.}{]}

For Mercury and his bandmates, there was no line between stupid and
clever; in many of the best Queen songs, stupid \emph{is} clever. It
takes a special kind of fearlessness to write songs about killer queens
who are dynamite with laser beams, or fat-bottomed girls who make the
rocking world go 'round, or really fast cars that you might want to have
sex with. Part of the thrill of listening to Queen is hearing them get
away with this sublime silliness, again and again.

The 1975 have a similar inclination, in the words of Spin magazine, ``to
look and sound ridiculous.'' When the band made its debut on ``Saturday
Night Live'' in 2016, Healy pulled out every conceivable louche
rock-singer move --- he wore leather pants with no shirt, he tousled his
curly black Sideshow Bob mane, he stuck out his tongue, he thrust his
crotch like Tom Jones serenading an audience of divorcées. And yet
there's an ongoing debate as to whether the 1975 should even be
considered a ``real'' rock band at all. Its third album, ``A Brief
Inquiry Into Online Relationships,'' is its most willfully diffuse yet,
touching on bedroom R.\&B., jazzy torch songs, SoundCloud rap, ambient
music and lots of other genres that a ``smarter,'' more conservative
band would never dream of dabbling in. Last year, when pressed by
Pitchfork to explain ``Love It if We Made It,'' Healy posed a delectable
stupid-clever rhetorical: ``How weird is reality?''

The 1975 polarizes indie-rock fans --- a bloc never comfortable with
ostentatious displays of self-valorization --- but detractors and
admirers alike agree on this band's utter absurdity. There may be no
other way for a proper rock band to act. At a time when our lives are on
display like never before, putting us all on the razor's edge of
life-changing public shame, how exhilarating is it to see people insist
on riding where they like?

*Steven Hyden is the cultural critic for Uproxx and the author of
``Twilight of the Gods: A Journey to the End of Classic Rock.''\\
*

Read more

\includegraphics{https://static01.graylady3jvrrxbe.onion/images/2019/03/10/magazine/10mag-weezer3/10mag-weezer3-master180.jpg}

\hypertarget{--cant-knock-the-hustle}{%
\subsection{\texorpdfstring{ `Can't Knock the
Hustle'}{  `Can't Knock the Hustle'}}\label{--cant-knock-the-hustle}}

\hypertarget{weezer}{%
\paragraph{Weezer}\label{weezer}}

\hypertarget{23}{%
\subsection{23}\label{23}}

\includegraphics{https://static01.graylady3jvrrxbe.onion/images/2019/03/10/magazine/10mag-weezer3/10mag-weezer3-master180.jpg}

Growing up means seeing heroes age, rudely, at the same pace as you.

By LYDIA KIESLING

\href{https://open.spotify.com/track/3OSQcSWOOpBUGBPgCMKDdK}{\includegraphics{https://static01.graylady3jvrrxbe.onion/newsgraphics/2019/02/26/music/2654e69ed27287529e314df4aede0297716f8c07/icons/spotify.png}}

In one episode of ``The Office,'' Michael Scott (played by Steve Carell)
puts together a video called ``Lazy Scranton'' to welcome new employees
to his branch. In keeping with his character's genius for mangling
pop-culture references, the video is Michael's misguided stab at
parodying the viral ``S.N.L.'' digital short ``Lazy Sunday,'' wherein
Andy Samberg and Chris Parnell rap about getting cupcakes and going to
see ``The Chronicles of Narnia.'' Michael's version, which includes
lines like ``Call poison control if you're bit by a spider/But check
that it's covered by your health care provider!'' leaves the new
employees stone-faced. When one of them quits, Michael pleads with him.
``Didn't you think `Lazy Scranton' was funny?'' ``No,'' the employee
says. ``Was it supposed to be funny?''

I thought about this moment when I heard ``Can't Knock the Hustle,'' a
song Weezer released late last year. The track, a reimagining of Jay-Z's
song by the same title, is the marquee single from the band's ``Black
Album.'' I first heard Weezer's version while driving a car, and I
almost had to pull over. ``What is happening?'' I asked.

The Jay-Z original is about selling drugs, with lines like ``At my
arraignment screaming/All us blacks got is sports and entertainment,
until we even/Thieving, as long as I'm breathing,'' with a chorus sung
by Mary J. Blige. The Weezer interpretation is about the gig economy and
unspools in painfully dorky lines like ``I'm an ugly {[}expletive{]} but
I work hella harder/And you can write a blog about it'' and ``Leave a
five-star review and I'll leave you one too.'' The song is unnervingly,
relentlessly bouncy. In the video, Pete Wentz of Fall Out Boy plays a
beleaguered ride-share driver. ``What is happening?'' his face seems to
say.

Earlier this year, the band released an album of actual covers (``No
Scrubs,'' ``Paranoid'' and ``Africa,'' among others), none of which had
the overwhelmingly depressing effect of ``Can't Knock the Hustle.'' What
is happening with this song? There's the peculiar comic engagement with
the desperate race to the bottom for American employers. There's the
white appropriation of black art, a refraction of hip-hop tropes that is
about one thousand times less successful than ``Lazy Sunday'' (possibly
even ``Lazy Scranton''). And there's the growing-old phenomenon of
seeing former heroes make choices you don't understand, in a way that
makes you grapple with all your own previous choices.

I have to acknowledge that I may be taking ``Can't Knock the Hustle''
personally to an unreasonable degree. Weezer is a band with a fanatical
but divided fan base, much of which believes that the band went downhill
after its second album, ``Pinkerton'' (there's even an ``S.N.L.'' skit
about this schism). I'm a Weezer fan who considers her soul to have been
irrevocably altered by ``The Blue Album,'' the band's first, which was
big guitars, big harmonization, big falsettos, big emotions. It can
still conjure sense memories of decades past --- windows down, crooning
out into the forgiving dusk. Then there was ``Pinkerton,'' which had, in
addition to its big guitars and emotions, big weirdness about gender and
race. I look back on my 18-year-old self belting out lines like ``You
are 18-year-old girl who live in small city of Japan,'' and have, if not
regrets, questions. ``Pinkerton'' now seems like a raw, risky, sometimes
repellent work of art, but one that earned its place in the age-old
conversation about what is permitted by the artist.

``Can't Knock the Hustle,'' on the other hand, sounds like an insincere
effort to start a conversation that nobody wants to have, or at least
that nobody wants to have with Weezer. The band's recent spate of
covers, even the ones that rise to the level of ``fun curiosity,'' feel
like another race to the bottom in a cultural economy where attention
matters more than anything else. (``Was it supposed to be funny?'' I
hear the new employee ask.)

It's not Weezer's fault that I can't let go of the past, but getting
older is hard enough without Weezer making you feel it so acutely.
``Can't Knock the Hustle'' fills me with a vague, general regret and an
apprehension for what's to come. As the song itself portends: ``The
future's so bright I gotta poke my eyes out.''

*Lydia Kiesling is the author of the novel ``The Golden State.''\\
*

Read more

\includegraphics{https://static01.graylady3jvrrxbe.onion/images/2019/03/10/magazine/10mag-sharon-von-etten/10mag-sharon-von-etten-master180-v5.jpg}

\hypertarget{--comeback-kid}{%
\subsection{\texorpdfstring{ `Comeback
Kid'}{  `Comeback Kid'}}\label{--comeback-kid}}

\hypertarget{sharon-van-etten}{%
\paragraph{Sharon Van Etten}\label{sharon-van-etten}}

\hypertarget{24}{%
\subsection{24}\label{24}}

\includegraphics{https://static01.graylady3jvrrxbe.onion/images/2019/03/10/magazine/10mag-sharon-von-etten/10mag-sharon-von-etten-master180-v5.jpg}

What do you do when you can't put off the future any more?

By RYAN BRADLEY

\href{https://open.spotify.com/track/5nUVtL9EgUDCsjZT2Hcy9H}{\includegraphics{https://static01.graylady3jvrrxbe.onion/newsgraphics/2019/02/26/music/2654e69ed27287529e314df4aede0297716f8c07/icons/spotify.png}}

A few years ago, Sharon Van Etten was interviewing another musician ---
Mimi Parker, from the band Low --- on the NPR program ``All Songs
Considered.'' Van Etten, then in her mid-30s, was thinking about having
a kid and wanted to ask Parker, a mother of two, about what it would be
like to be a mom and a touring musician at the same time.

``My music is the center of my world,'' Van Etten said while setting up
a very big question: How does someone --- anyone, but especially an
artist --- make room in her life for her work, when her work might no
longer be the center of her world?

``You just figure it out,'' Parker answered.

In the years that followed, Van Etten fell in love, pulled back from
touring, enrolled in school to study psychology, was cast for an acting
part in the Netflix show ``The OA,'' wrote a movie score, gave birth (a
boy, now 2) --- and in January released a new album, her first since
2014.

The album's title --- ``Remind Me Tomorrow'' --- comes from an Apple
computer software-update alert option that Van Etten found herself
selecting again and again, a perpetual prompt for something she needed
to do and never quite got done. It's a tidy summation of motherhood and
artistry and the process of figuring them out, often at the same time,
because, well, a lot of the figuring out seems to be figuring out what
can wait until tomorrow.

{[}\href{https://www.nytimes3xbfgragh.onion/2019/01/03/arts/music/sharon-van-etten-remind-me-tomorrow-interview.html}{Read
about the many lives of Sharon Van Etten.}{]}

The album's cover is another reflection of (or on) this conflict: a
photo of a room with two toddlers in it, their stuff --- toys, clothing,
everything --- spilled out around them. The image comes from the
filmmaker Katherine Dieckmann. (It was Dieckmann's film ``Strange
Weather'' that Van Etten scored.) Those are Dieckmann's kids amid the
chaos. Van Etten recalled in a Vanity Fair interview that when she told
Dieckmann she was pregnant and worried about how she was going to make
motherhood work, Dieckmann pulled out her phone and pulled up the photo.
``You'll figure it out,'' she said.

``Remind Me Tomorrow'' is about all the stuff that can't be KonMari'd
out of existence: life, love, the figuring, the delaying, and the fact
that eventually, one day, but probably not today, the time will come to
finally do the all those things the ``remind me tomorrow'' prompts had
so imperfectly kept at bay.

The standout track, the song you want to raise up your fists and loosen
your hips to, is ``Comeback Kid.'' It's driving, then sweet, and it's
very stripped down. Every minor variation of the refrain seems to offer
a new perspective. Sometimes when she sings it, Van Etten seems to be
talking to an earlier version of herself, ``Come back, kid!'' Other
times, she's talking about someone who is definitely not her: the
comeback kid. \href{https://www.youtube.com/watch?v=h4mewwymxbI}{In the
video}, Van Etten stands singing as old photos are projected onto her
face and body and the wall behind her. The photos are from Van Etten's
past, but there's very little that's specific about them. They just look
like the past in general. Which is probably the point: The past can be
both particular and general, as her decision to use cover art featuring
someone else's kids might indicate.

``Comeback Kid'' is a song about glancing at our past selves while
dealing with the whole jumble of conflicting emotions that looking too
closely at our past might dredge up. But mostly, it's about the looking
itself. It's the looking that's important. The world shifts; you look at
the past; you look at the future. And then what do you do? You figure it
out.

*Ryan Bradley is a writer in Los Angeles.\\
*

Read more

\includegraphics{https://static01.graylady3jvrrxbe.onion/images/2019/03/10/magazine/10mag-sons/10mag-sons-master180-v6.jpg}

\hypertarget{--my-queen-is-harriet-tubman}{%
\subsection{\texorpdfstring{ `My Queen Is Harriet
Tubman'}{  `My Queen Is Harriet Tubman'}}\label{--my-queen-is-harriet-tubman}}

\hypertarget{sons-of-kemet}{%
\paragraph{Sons of Kemet}\label{sons-of-kemet}}

\hypertarget{25}{%
\subsection{25}\label{25}}

\includegraphics{https://static01.graylady3jvrrxbe.onion/images/2019/03/10/magazine/10mag-sons/10mag-sons-master180-v6.jpg}

A saxophone, a tuba and two drum kits combine to create a piece of music
that somehow captures the whole history of jazz.

By MARK RICHARDSON

\href{https://open.spotify.com/track/4dSjQyjl3aHWJpk3swB0ig}{\includegraphics{https://static01.graylady3jvrrxbe.onion/newsgraphics/2019/02/26/music/2654e69ed27287529e314df4aede0297716f8c07/icons/spotify.png}}

Listen closely to the third track from the 2018 studio album ``Your
Queen Is a Reptile,'' by the London jazz outfit Sons of Kemet, and you
will hear a tenor saxophone, a tuba and two drummers, all moving at
great speed. The whole operation sounds like four people piled into a
wagon tumbling down a hill, just barely in control. Each element
contributes equally. Because the bass comes from an instrument powered
by breath, the darting low end is less of a woofer-pumping presence and
more of a song-within-a-song, a melody that you can hum on its own. The
saxophone shouts back, offering growling rhythmic lines with just a
pinch of melody. And the dueling drummers build one intensely syncopated
beat from parts of several --- the foundational Caribbean rhythm of the
Cuban \emph{tresillo}, martial snare rolls, pinging metallic percussion
reminiscent of the roaring Afrobeat of Fela Kuti. Listen without knowing
another thing about it, and this is a viscerally overwhelming piece of
music.

Widen the focus, though, and you will learn that the song is called ``My
Queen Is Harriet Tubman,'' and that each song title on the album
references a heroic black woman who has inspired the band's leader, the
London-born, Barbados-raised saxophone player Shabaka Hutchings. Maybe
that knowledge gives the burning intensity of the song --- its feeling
of joy streaked with struggle --- a new dimension. Pull back further and
take in Hutchings's place as a central figure of one of this era's most
remarkable music scenes, sometimes called New London Jazz --- an
approach that incorporates sounds and styles from all over the African
diaspora, along with locally sourced ingredients from the British
electronic underground. Let even more into the frame --- say, that
Hutchings has performed with the Sun Ra Arkestra and is now signed to
Impulse!, the imprint that issued major works from artists including
John Coltrane, Alice Coltrane and Pharoah Sanders --- and then listen
again to situate this music within the entire sweep of recorded jazz
over the last 100 years.

All this history is carried inside the song and transmitted by these
master musicians thorough their instruments. It's a sax, a tuba and two
drum kits. It's a tune and an unforgettable beat. It's a call to action
and a command to find the dance floor. It's a deeply political piece of
music that says a great deal without uttering a word. It's music of this
exact moment, built from sounds that have bounced back and forth across
the Atlantic over centuries. Sounds that landed here, in a song honoring
a woman whose face will, perhaps, on some future Sons of Kemet tour
across the United States, be found on a \$20 bill in Hutchings's pocket,
a queen's visage replacing that of the dead president so admired by the
current one.

*Mark Richardson is the former executive editor of Pitchfork and a
writer and an editor in Brooklyn.\\
*

Read more

\subsection{}

\begin{itemize}
\item
  \href{https://www.nytimes3xbfgragh.onion/interactive/2018/10/03/magazine/lady-gaga-movie-star-is-born.html}{}

  \includegraphics{https://static01.graylady3jvrrxbe.onion/images/2018/10/07/magazine/07LadyGaga/07LadyGaga-mediumThreeByTwo225-v2.jpg}

  \hypertarget{lady-gaga-isnt-done-shape-shifting-yet}{%
  \subsection{Lady Gaga Isn't Done Shape-Shifting
  Yet}\label{lady-gaga-isnt-done-shape-shifting-yet}}

  Oct. 8, 2018
\item
  \href{https://www.nytimes3xbfgragh.onion/2018/04/19/magazine/how-janelle-monae-found-her-voice.html}{}

  \includegraphics{https://static01.graylady3jvrrxbe.onion/images/2018/04/19/magazine/19mag-janelle-image1/19mag-janelle-image1-mediumThreeByTwo225-v10.jpg}

  \hypertarget{how-janelle-monuxe1e-found-her-voice}{%
  \subsection{How Janelle Monáe Found Her
  Voice}\label{how-janelle-monuxe1e-found-her-voice}}

  June 9, 2018
\item
  \href{https://www.nytimes3xbfgragh.onion/interactive/2018/03/08/magazine/25-songs-future-of-music.html}{}

  \includegraphics{https://static01.graylady3jvrrxbe.onion/images/2018/03/08/magazine/25-songs-future-of-music-1520378467104/25-songs-future-of-music-1520378467104-mediumThreeByTwo225.png}

  \hypertarget{25-songs-that-tell-us-where-music-is-going}{%
  \subsection{25 Songs That Tell Us Where Music Is
  Going}\label{25-songs-that-tell-us-where-music-is-going}}

  March 15, 2018
\item
  \href{https://www.nytimes3xbfgragh.onion/interactive/2017/03/09/magazine/25-songs-that-tell-us-where-music-is-going.html}{}

  \includegraphics{https://static01.graylady3jvrrxbe.onion/images/2017/03/12/magazine/12musicissue-MusicIssue/12musicissue-MusicIssue-mediumThreeByTwo225.jpg}

  \hypertarget{25-songs-that-tell-us-where-music-is-going-1}{%
  \subsection{25 Songs That Tell Us Where Music Is
  Going}\label{25-songs-that-tell-us-where-music-is-going-1}}

  March 10, 2017
\item
  \href{https://www.nytimes3xbfgragh.onion/interactive/2016/03/10/magazine/25-songs-that-tell-us-where-music-is-going.html}{}

  \includegraphics{https://static01.graylady3jvrrxbe.onion/images/2016/03/13/magazine/13mag-music-promo/13mag-music-promo-mediumThreeByTwo225.png}

  \hypertarget{25-songs-that-tell-us-where-music-is-going-2}{%
  \subsection{25 Songs That Tell Us Where Music Is
  Going}\label{25-songs-that-tell-us-where-music-is-going-2}}

  June 5, 2017
\end{itemize}

Advertisement

\hypertarget{site-information-navigation}{%
\subsection{Site Information
Navigation}\label{site-information-navigation}}

\begin{itemize}
\tightlist
\item
  \href{https://help.nytimes3xbfgragh.onion/hc/en-us/articles/115014792127-Copyright-notice}{©
  2020 The New York Times Company}
\item
  \href{https://www.nytimes3xbfgragh.onion}{Home}
\item
  \href{https://www.nytimes3xbfgragh.onion/search/}{Search}
\item
  Accessibility concerns? Email us at
  \href{mailto:accessibility@NYTimes.com}{\nolinkurl{accessibility@NYTimes.com}}.
  We would love to hear from you.
\item
  \href{https://help.nytimes3xbfgragh.onion/hc/en-us/articles/115015385887-Contact-Us}{Contact
  Us}
\item
  \href{https://www.nytco.com/careers/}{Work with us}
\item
  \href{https://nytmediakit.com/}{Advertise}
\item
  \href{https://help.nytimes3xbfgragh.onion/hc/en-us/articles/115014892108-Privacy-policy\#pp}{Your
  Ad Choices}
\item
  \href{https://help.nytimes3xbfgragh.onion/hc/en-us/articles/115014892108-Privacy-policy}{Privacy}
\item
  \href{https://help.nytimes3xbfgragh.onion/hc/en-us/articles/115014893428-Terms-of-service}{Terms
  of Service}
\item
  \href{https://help.nytimes3xbfgragh.onion/hc/en-us/articles/115014893968-Terms-of-sale}{Terms
  of Sale}
\end{itemize}

\hypertarget{site-information-navigation-1}{%
\subsection{Site Information
Navigation}\label{site-information-navigation-1}}

\begin{itemize}
\tightlist
\item
  \href{https://spiderbites.nytimes3xbfgragh.onion}{Site Map}
\item
  \href{https://help.nytimes3xbfgragh.onion/hc/en-us}{Help}
\item
  \href{https://help.nytimes3xbfgragh.onion/hc/en-us/articles/115015385887-Contact-Us?redir=myacc}{Site
  Feedback}
\item
  \href{https://www.nytimes3xbfgragh.onion/subscription?campaignId=37WXW}{Subscriptions}
\end{itemize}
