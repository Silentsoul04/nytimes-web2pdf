 **NYTimes.com no longer supports Internet Explorer 9 or earlier. Please
upgrade your browser.
\href{http://www.nytimes3xbfgragh.onion/content/help/site/ie9-support.html}{LEARN
MORE »}

**Sections

**Home

**Search

\hypertarget{the-new-york-times}{%
\subsection{\texorpdfstring{\href{http://www.nytimes3xbfgragh.onion/}{The
New York Times}}{The New York Times}}\label{the-new-york-times}}

\hypertarget{-business-}{%
\subsubsection{\texorpdfstring{ \href{/section/business}{Business}
}{ Business }}\label{-business-}}

 \href{/section/business/energy-environment}{Energy \& Environment}
\textbar{}California Wildfires: How PG\&E Ignored Risks in Favor of
Profits

**Close search

\hypertarget{site-search-navigation}{%
\subsection{Site Search Navigation}\label{site-search-navigation}}

Search NYTimes.com

**Clear this text input

Go

\url{https://nyti.ms/2FdDEWm}

\hypertarget{site-navigation}{%
\subsection{Site Navigation}\label{site-navigation}}

\hypertarget{site-mobile-navigation}{%
\subsection{Site Mobile Navigation}\label{site-mobile-navigation}}

\hypertarget{california-wildfires-how-pge-ignored-risks-in-favor-of-profits}{%
\section{California Wildfires: How PG\&E Ignored Risks in Favor of
Profits}\label{california-wildfires-how-pge-ignored-risks-in-favor-of-profits}}

buttex300

10 days 2 days 1 day 12 hrs. PINEGROVE WESTPOINT MOUNTAINRANCH
SanFrancisco LosAngeles 2 miles Tree hits PG\&E power line

buttex720

10 days 2 days 1 day 12 hrs. SANANDREAS JACKSON MOKELUMNEHILL SHEEPRANCH
MOUNTAINRANCH SanFrancisco LosAngeles 2 miles Tree hits PG\&E power line

buttex1050

10 days 2 days 1 days 12 hrs. SANANDREAS JACKSON MOKELUMNEHILL
SHEEPRANCH MOUNTAINRANCH SanFrancisco LosAngeles 2 miles Tree hits PG\&E
power line

napax300

10 days 3 days 2 days 2 days 1 day 1 day 1 day 1 hour Other fires start
the same day after trees fall into PG\&E power lines 1 hour 1 hour Tree
hits PG\&E power line SANTA ROSA GLENELLEN SONOMA NAPA ST. HELENA
PETALUMA SanFrancisco LosAngeles 2 miles N

napax720

10 days 3 days 2 days 2 days 1 day 1 day 1 day Other fires start the
same day after trees fall into PG\&E power lines 1 hour 1 hour 1 hour
Tree hits PG\&E power line GLENELLEN SONOMA NAPA ST. HELENA SanFrancisco
LosAngeles 2 miles

napax1050

10 days 3 days 2 days 2 days 1 day 1 day 1 day Other fires start the
same day after trees fall into PG\&E power lines 1 hour 1 hour 1 hour
Tree hits PG\&E power line GLENELLEN SONOMA NAPA ST. HELENA PETALUMA 2
miles SanFrancisco LosAngeles

campx300

13 days 4 days 2 days 11 hrs. Live wire breaks from 99-year-old PG\&E
electric tower CHICO DURHAM PARADISE MAGALIA PULGA SanFrancisco
LosAngeles 2 miles N

campx720

13 days 4 days 2 days 11 hrs. Live wire breaks from 99-year-old PG\&E
electric tower CHICO DURHAM PARADISE MAGALIA PULGA SanFrancisco
LosAngeles 2 miles

campx1050

4 days 2 days 11 hrs. Live wire breaks from 99-year-old PG\&E electric
tower CHICO DURHAM PARADISE MAGALIA PULGA SanFrancisco LosAngeles 2
miles

Pacific Gas \& Electric, California's largest utility, has been
responsible for wildfires in recent years that destroyed hundreds of
thousands of acres. Several proved deadly.

One of the biggest fires started near Sacramento in 2015, when a tree
that PG\&E had failed to maintain hit one of its power lines.

The fire covered more than 70,000 acres. Two people died.

In 2017, four fires erupted in the Napa area when trees hit PG\&E power
lines in several locations.

Some of those fires fed into others. In total, more than 100,000 acres
and 1,475 structures burned.

Last year, near Paradise, a live wire broke free of a tower that was a
quarter-century past what PG\&E considers its ``useful life.''

What followed destroyed nearly 14,000 homes and killed 85 people. It was
the most destructive wildfire in California history.

Tower 27/222 looms almost 100 feet tall in the Sierra Nevada foothills,
a hunk of steel that has endured through 18 United States presidents.
The transmission lines that it supports keep electricity flowing to much
of California.

On the morning of Nov. 8, a live wire broke free of its grip. A power
failure occurred on the line, affecting a single customer. But 15
minutes later, a fire was observed nearby. Within hours, flames engulfed
the region, ultimately killing 85 and destroying the town of Paradise.

The equipment belonged to the state's biggest utility, Pacific Gas and
Electric. To the company's critics, the tower and its vulnerability
reflect a broken safety culture.

Five of the 10
\href{http://www.fire.ca.gov/communications/downloads/fact_sheets/Top20_Destruction.pdf}{most
destructive fires} in California since 2015 have been linked to PG\&E's
electrical network. Regulators have found that in many fires, PG\&E
violated state law or could have done more to make its equipment safer.

Long before the failure suspected in the Paradise fire, a company email
had noted that some of PG\&E's structures in the area, known for fierce
winds, were at risk of collapse. It reported corrosion of one tower so
severe that it endangered crews trying to repair the tower. The
company's own guidelines put Tower 27/222 a quarter-century beyond its
useful life --- but the tower remained.

\includegraphics{https://static01.graylady3jvrrxbe.onion/packages/flash/multimedia/ICONS/transparent.png}

Tower 27/222. Utility experts have been incredulous that Pacific Gas \&
Electric let it stand for so long. Max Whittaker for The New York Times

In January, the company sought bankruptcy protection, saying it might
face more than \$30 billion in wildfire liabilities. Its financial
straits could hamper its preparations for the next wildfire season, and
those beyond, even as weather patterns increase the fire risk.

``There is a climate change component to this,'' said Michael W. Wara,
director of the climate and energy policy program at Stanford University
and a member of a state commission examining the cost of wildfires.
``But there's also a failure of management and a failure of vision.''

Another major utility in the state, San Diego Gas \& Electric, has added
hundreds of weather stations, cameras and satellite technology in recent
years to reduce fire risk. PG\&E is now trying to catch up.

Beyond wildfires, PG\&E has a broader history of safety problems. A 2010
explosion of a PG\&E gas pipeline killed eight people and destroyed a
suburban neighborhood, prompting state and federal officials to
investigate PG\&E's safety practices. Regulators ultimately fined the
utility \$1.6 billion, and a federal jury convicted it of violating a
pipeline safety law and obstructing an investigation. The company is
still under court-supervised probation.

PG\&E executives acknowledge that the company has made mistakes. ``We
have heard the calls for change and are committed to taking action by
focusing our resources on reducing risk and improving safety throughout
our system,'' John Simon, PG\&E's interim chief executive, said in a
recent statement.

But Gov. Gavin Newsom said the company's record made it hard to take its
promises seriously.

``They have simply been caught red-handed over and over again, lying,
manipulating or misleading the public,'' Mr. Newsom said in an
interview. ``They cannot be trusted.''

\hypertarget{finding-flaws-but-choosing-not-to-fix-them}{%
\subsection{Finding flaws, but choosing not to fix
them}\label{finding-flaws-but-choosing-not-to-fix-them}}

\includegraphics{https://static01.graylady3jvrrxbe.onion/packages/flash/multimedia/ICONS/transparent.png}

A broken insulator underneath a PG\&E transmission line outside the town
of Pulga, Calif. Max Whittaker for The New York Times

Iris Natividad and her partner of 28 years, Andrew Downer, had spent a
decade in a Paradise home, where they had an antiques business and Mr.
Downer's 80,000-piece antique marble collection. Ms. Natividad was away
when Mr. Downer told her by phone that a fire had been sighted in the
town of Pulga and was headed his way.

Pulga seemed distant enough --- about a 20-minute drive --- that there
would be time to escape if necessary. But 30 minutes later, the flames
were a mile away, and there was no one to evacuate Mr. Downer, a
54-year-old amputee.

``He told me, `Today might be a good day to die,''' Ms. Natividad
recalled. Mr. Downer and his service dog, a Labrador, were killed.

The state has not finished its investigation of the blaze, known as the
Camp Fire, which became California's deadliest ever. But the company
said recently that its equipment was probably the cause. And utility
experts were incredulous that PG\&E had let Tower 27/222 stand for so
long. The company's critics say the utility could have easily obtained
approval from state regulators to replace the tower and recover the cost
from ratepayers.

``Some people believe that you run equipment to failure,'' Catherine
Sandoval, a former California regulator who has been pushing for
improved maintenance of electrical poles and towers. ``They believe `run
to failure' to save money. This is the danger of run to failure.''

In December 2012, five other aging towers on the same stretch, the
Caribou-Palermo line, collapsed in a storm. In July 2013, Brian Cherry,
PG\&E's vice president for regulatory affairs at the time, notified
state regulators that the company would replace the five fallen towers
and one more, but not 27/222.

A 2014 company email that has come to light in the bankruptcy
proceedings said that ``the likelihood of failed structures happening is
high.'' But PG\&E determined that if the structures failed, the cause
would probably be heavy rain, precluding a wildfire risk. PG\&E said
this week that the structures in question were temporary wooden poles
that had since been replaced.

In April 2016, PG\&E made another request to regulators: to install
fresh wires on the Caribou-Palermo line. But the company said it would
not replace any of the line's remaining nearly century-old towers.

That October, during painting work on a lattice tower on the line, a
piece of hardware called a J hook broke when a contract worker grabbed
it while repositioning himself. A PG\&E report said workers had
determined that corrosion --- the reason for the painting --- was enough
of a problem that ``crews working on these towers need to use caution.''

The company said that tower had a different design from Tower 27/222's.
But it would not comment on why it didn't replace 27/222 given its age.
It said it considered many factors when making decisions on maintenance
and repairs.

\hypertarget{a-focus-on-the-bottom-line-over-everything}{%
\subsection{`A focus on the bottom line over
everything'}\label{a-focus-on-the-bottom-line-over-everything}}

\includegraphics{https://static01.graylady3jvrrxbe.onion/packages/flash/multimedia/ICONS/transparent.png}

A 2010 pipeline explosion in a San Francisco suburb killed eight people
and destroyed a neighborhood. Paul Sakuma/Associated Press

The deadly 2010 gas pipeline explosion in San Bruno, a San Francisco
suburb, was PG\&E's second in a two-year period. The ensuing
investigations and litigation produced an alarming picture of the
company's practices and priorities.

In court depositions, employees said supervisors routinely ignored their
concerns about the company's use of faulty analysis and outdated
equipment. The state's Public Utilities Commission, which regulates
PG\&E, concluded that the company was more concerned with profit than
with safety.

The commission's
\href{http://www.cpuc.ca.gov/uploadedFiles/CPUC_Public_Website/Content/Safety/Natural_Gas_Pipeline/News/AgendaStaffReportreOIIPGESanBrunoExplosion.pdf}{safety
and enforcement division found} in 2012 that PG\&E's gas and
transmission revenues exceeded what it was authorized to collect by
\$224 million in the decade leading up to the explosion. But capital
spending fell \$93 million short of its authorized budget between 1997
and 2000. PG\&E also spent millions less on operations and maintenance
than it was supposed to.

``There was very much a focus on the bottom line over everything: `What
are the earnings we can report this quarter?''' said Mike Florio, a
utilities commissioner from 2011 through 2016. ``And things really got
squeezed on the maintenance side.''

Five years after the explosion, a PG\&E line started the Butte Fire,
which scorched more than 70,000 acres, killing two people and destroying
nearly a thousand homes and other buildings.

State investigators said workers should have known that when they had
cleared a stand of trees for PG\&E, they had exposed a gray pine weak
enough to be blown into a power line. On Sept. 9, 2015, strong winds
knocked that tree into the line, igniting the fire.

State officials also blamed PG\&E equipment for starting 17 of 21 major
fires in 2017 that ripped through Northern California, including
wine-growing Napa and Sonoma Counties.

A
\href{http://docs.cpuc.ca.gov/PublishedDocs/Efile/G000/M235/K399/235399881.PDF}{2017
report} commissioned by state regulators determined that PG\&E often
made improvements only after a disaster. The report, which was produced
by NorthStar Consulting, also found that the transmission and
distribution side of the company had less robust safety policies than
its gas and power generation divisions.

\hypertarget{bonuses-were-linked-to-reducing-fallen-wires-but-the-policy-didnt-last}{%
\subsection{Bonuses were linked to reducing fallen wires, but the policy
didn't
last}\label{bonuses-were-linked-to-reducing-fallen-wires-but-the-policy-didnt-last}}

\includegraphics{https://static01.graylady3jvrrxbe.onion/packages/flash/multimedia/ICONS/transparent.png}

The remains from a fire that damaged homes and cars in Santa Rosa,
Calif., in October 2017. Justin Sullivan/Getty Images

Under regulatory and legal pressure after the San Bruno gas explosion,
PG\&E took steps to elevate safety as a priority. In 2012, the company
started linking a portion of executives' annual bonuses to minimizing
the number of downed power lines.

But after the company reported that the number of wires down had jumped
nearly 30 percent in 2016, it stopped using that measure to help
calculate bonuses.

PG\&E started using a benchmark that tracked vegetation-management
efforts as well as inspections and upgrades of electrical lines. The
company said the new statistic helped reduce the number of fallen wires.

The change came as PG\&E installed a new chief executive, Geisha J.
Williams, formerly head of the company's electric operations. Weeks
later, in May 2017, the company approved a dividend increase, the second
in just over a year, that helped win investors' favor.

After the devastating wildfires of October 2017, Ms. Williams detailed
steps the company was taking to prevent more fires, like disabling power
lines when the risk of fires was high and improving how it dealt with
foliage near its lines.

But Ms. Williams also said the company was often being blamed for fires
when it had done nothing wrong. She took aim at a California legal
principle that held companies liable for the damage caused by their
equipment even when the businesses had not acted negligently, and said
PG\&E was pushing regulators, lawmakers and courts to change the
provision. At an industry conference last March, she said doing so was
``the most important thing for us right now.''

While the company did not get everything it wanted, the California
Legislature passed a law in September that allowed utilities to recoup
some of the cost of wildfire liabilities by raising electricity rates.
But that law did not address 2018 wildfires, leaving the company
responsible for tens of billions of dollars in damage.

In January, with a bankruptcy filing imminent, Ms. Williams left the
company. She did not respond to requests for comment for this article.

\hypertarget{after-its-own-fire-disaster-another-utility-became-a-safety-innovator}{%
\subsection{After its own fire disaster, another utility became a safety
innovator}\label{after-its-own-fire-disaster-another-utility-became-a-safety-innovator}}

\includegraphics{https://static01.graylady3jvrrxbe.onion/packages/flash/multimedia/ICONS/transparent.png}

Chris Arends, the meteorology program manager for San Diego Gas \&
Electric, in front of a wildfire activity model at the company's
headquarters. John Francis Peters for The New York Times

State officials say there is a good template elsewhere in California for
what PG\&E should be aiming for: the practices of San Diego Gas \&
Electric.

The San Diego utility keeps data on every utility pole and transmission
tower in its service territory, which is smaller than PG\&E's but has a
higher proportion of overhead lines in areas at high fire risk. It uses
nearly 177 stations to monitor temperature, humidity and wind speeds in
an area roughly the size of Connecticut and records video from 100
high-definition cameras. It uses satellites to track how green or dry
the grass is and employs the state's largest water-dropping helicopter
to douse fires quickly. When data indicates a high wildfire threat, the
utility cuts off power to some areas.

San Diego Gas \& Electric upgraded its fire-prevention efforts after
residents sued it for causing a devastating wildfire in 2007. In recent
years, it has been responsible for far fewer fires than PG\&E. ``We want
to make sure that we're doing everything we can to mitigate ignition,''
said Scott Drury, the utility's president.

Sumeet Singh, PG\&E's vice president for community wildfire safety, said
in an interview that the utility was putting new safety policies in
place, often using San Diego's approach. They include ground and
helicopter inspections of poles and towers, and weather monitoring
systems.

But he suggested that PG\&E had an uphill task given climate change. He
said equipment failures that would have caused little or no damage a few
years ago now set off fires that burn thousands of acres because
California forests had become much more combustible.

The risk of wildfires in California has clearly gone up, but Robert
McCullough, a longtime consultant to energy companies and state and
federal officials investigating them, said there would have been far
fewer destructive fires had PG\&E followed San Diego's lead years ago.

``Their culture of a lack of safety is unique, in my opinion,'' he said.

Note: In the map showing the fires that took place in the Napa area in
2017, not all points of ignition are shown, nor fires for which Pacific
Gas \& Electric was not found to have violated state law. \textbar{}
Graphic sources: United States Geological Survey and the California
Department of Forestry and Fire Protection.

Graphics by KEITH COLLINS and JUGAL K. PATEL

\hypertarget{more-on-nytimescom}{%
\subsection{More on NYTimes.com}\label{more-on-nytimescom}}

Advertisement

\hypertarget{site-information-navigation}{%
\subsection{Site Information
Navigation}\label{site-information-navigation}}

\begin{itemize}
\tightlist
\item
  \href{https://help.nytimes3xbfgragh.onion/hc/en-us/articles/115014792127-Copyright-notice}{©
  2020 The New York Times Company}
\item
  \href{https://www.nytimes3xbfgragh.onion}{Home}
\item
  \href{https://www.nytimes3xbfgragh.onion/search/}{Search}
\item
  Accessibility concerns? Email us at
  \href{mailto:accessibility@NYTimes.com}{\nolinkurl{accessibility@NYTimes.com}}.
  We would love to hear from you.
\item
  \href{https://help.nytimes3xbfgragh.onion/hc/en-us/articles/115015385887-Contact-Us}{Contact
  Us}
\item
  \href{https://www.nytco.com/careers/}{Work with us}
\item
  \href{https://nytmediakit.com/}{Advertise}
\item
  \href{https://help.nytimes3xbfgragh.onion/hc/en-us/articles/115014892108-Privacy-policy\#pp}{Your
  Ad Choices}
\item
  \href{https://help.nytimes3xbfgragh.onion/hc/en-us/articles/115014892108-Privacy-policy}{Privacy}
\item
  \href{https://help.nytimes3xbfgragh.onion/hc/en-us/articles/115014893428-Terms-of-service}{Terms
  of Service}
\item
  \href{https://help.nytimes3xbfgragh.onion/hc/en-us/articles/115014893968-Terms-of-sale}{Terms
  of Sale}
\end{itemize}

\hypertarget{site-information-navigation-1}{%
\subsection{Site Information
Navigation}\label{site-information-navigation-1}}

\begin{itemize}
\tightlist
\item
  \href{https://spiderbites.nytimes3xbfgragh.onion}{Site Map}
\item
  \href{https://help.nytimes3xbfgragh.onion/hc/en-us}{Help}
\item
  \href{https://help.nytimes3xbfgragh.onion/hc/en-us/articles/115015385887-Contact-Us?redir=myacc}{Site
  Feedback}
\item
  \href{https://www.nytimes3xbfgragh.onion/subscription?campaignId=37WXW}{Subscriptions}
\end{itemize}
