 **NYTimes.com no longer supports Internet Explorer 9 or earlier. Please
upgrade your browser.
\href{http://www.nytimes3xbfgragh.onion/content/help/site/ie9-support.html}{LEARN
MORE »}

**Sections

**Home

**Search

\hypertarget{the-new-york-times}{%
\subsection{\texorpdfstring{\href{http://www.nytimes3xbfgragh.onion/}{The
New York Times}}{The New York Times}}\label{the-new-york-times}}

\hypertarget{-magazine-}{%
\subsubsection{\texorpdfstring{
\href{https://www.nytimes3xbfgragh.onion/section/magazine}{Magazine}
}{ Magazine }}\label{-magazine-}}

 \href{https://www.nytimes3xbfgragh.onion/section/magazine}{Magazine}
\textbar{}It Was Just a Kayaking Trip. Until It Upended Our Lives.

**Close search

\hypertarget{site-search-navigation}{%
\subsection{Site Search Navigation}\label{site-search-navigation}}

Search NYTimes.com

**Clear this text input

Go

\url{https://nyti.ms/2UL31W1}

\hypertarget{site-navigation}{%
\subsection{Site Navigation}\label{site-navigation}}

\hypertarget{site-mobile-navigation}{%
\subsection{Site Mobile Navigation}\label{site-mobile-navigation}}

\hypertarget{it-was-just-a-kayaking-trip-until-it-upended-our-lives}{%
\section{It Was Just a Kayaking Trip. Until It Upended Our
Lives.}\label{it-was-just-a-kayaking-trip-until-it-upended-our-lives}}

A weeklong trip in Alaska was supposed to be an adventure. In an
instant, it became the experience that defined us all.

\begin{itemize}
\tightlist
\item
  The Voyages Issue
\item
  \href{https://www.nytimes3xbfgragh.onion/interactive/2019/03/20/magazine/rick-steves-travel-world.html}{Rick
  Steves Wants to Save the World, One Vacation at a Time}
\item
  \href{https://www.nytimes3xbfgragh.onion/interactive/2019/03/21/magazine/mushrooms-sweden-island.html}{The
  Mystery Gift at the End of the World}
\item
  \href{https://www.nytimes3xbfgragh.onion/interactive/2019/03/20/magazine/kayaking-trip-alaska.html}{It
  Was Just a Kayaking Trip. Until It Upended Our Lives.}
\item
  \href{https://www.nytimes3xbfgragh.onion/interactive/2019/03/20/magazine/train-across-america-amtrak.html}{There
  Is No Reason to Cross the U.S. by Train. But I Did It Anyway.}
\end{itemize}

The Voyages Issue

\section{}

The Senseless

Logic of

the Wild

 A weeklong trip in Alaska was supposed to be an adventure. In an
instant, it became the experience that defined us all.

By JON MOOALLEM

photo illustration by:

DAVID BENJAMIN SHERRY

\textbf{The whale sighting} \textbf{happened} right away, minutes into
Day 1. Jon, Dave and I had just been dropped off on a remote Alaskan
shoreline, an hour and a half by boat from the closest speck of a town.
Jon was working as a sea-kayaking guide that summer in Glacier Bay
National Park, and he had invited us up for a seven-day excursion during
his week off. As the boat that delivered us vanished, the drone of its
engine dampening into a murmur and then finally trailing off, it became
unthinkably quiet on the beach, and the largeness and strangeness of our
surroundings were suddenly apparent. It was a familiar phenomenon for
Jon from the start of all his trips: a moment that people instinctually
paused to soak in. To me, it felt like those scenes of astronauts who,
having finally rattled free of the earth's atmosphere, slip into the
stillness of space. Except we weren't in space. We were on earth ---
finally, really on earth.

We were only starting to move around again, packing our gear into the
kayaks, when we heard the first huff of a blowhole, not far offshore.

Jon was ecstatic. It seemed to him as if the animal were putting on a
show, swimming playfully in the kelp, diving, resurfacing, then plowing
its open mouth across the surface to feed. He took it as a good omen.
Though I had no idea at the time, he was anxious that Dave and I might
feel intimidated about making the trip; such a big payoff, so quickly,
would get us excited and defuse any apprehensions.

For Dave, the whale-sighting had exactly the opposite effect. Once, when
he was a kid, his dad took him scuba diving with dolphins. They were
friendly, awe-inspiring creatures, purportedly, but they terrified Dave
instead. He could still conjure the feeling of hanging defenselessly in
that water while the animals deftly swirled around him, less like solid
objects than flashes of reflected light, while he could move only in
comparative slow-motion. Ever since, he had harbored a fear of large sea
creatures --- a niche phobia, particularly for a young man who lived in
the Bronx, but a genuine one still. And so, even as Dave understood that
a chance to see whales up close like this was a major draw of a kayaking
trip in Alaska, and though he feigned being thrilled, some second
thoughts were kicking in: We were going out there, he realized.

The whale left me exhilarated and gleeful, like Jon; but deeper down, I
also remember feeling shaken, like Dave. Nothing about the animal
registered to me as playful or welcoming. It just appeared in the
distance, then transited quickly past us, from left to right. My
uneasiness had something to do with the whale's great size and
indifference --- its obliviousness --- as it passed. Watching it made me
feel profoundly out of place and register how large that wilderness was,
relative to me.

At the time, I was working at a literary magazine in New York City
called The Hudson Review, picking poems out of the slush pile and
mailing them to an outside panel of editorial advisers. I was trying
hard in my letters to impress one of them: Hayden Carruth, a gruff and
irreverent 81-year-old poet who lived far upstate. I loved Carruth's
work but was more enamored with his persona: his yeoman life in the
woods, his intolerance for phoniness and, most of all, the precision
with which he articulated common suffering, including one strain of his
own suffering that I related to, particularly in those years, but
wouldn't have had the courage, or clarity, to examine.

``I had always been aware,'' Carruth once wrote of his youth, ``that the
Universe is sad; everything in it, animate or inanimate, the wild
creatures, the stones, the stars, was enveloped in the great sadness,
pervaded by it. ... Never then or now have I been able to look at a
cloudless sky at night and see beauty there. A kind of grandeur, yes ---
but not beauty. The profusion and variety of celestial lights have
always frightened me. Why are they there? Why these instead of others?
Why these instead of nothing?''

That was how I felt, watching the whale from the beach: afraid that
everything was accidents. Then again, maybe it's just hard to picture
the start of the trip in retrospect without amplifying some feeling of
foreboding. Something else Carruth wrote that has always stuck with me:
``The wilderness begins at the edge of my body, at the edge of my
consciousness, and extends to the edge of the universe, and it is filled
with menace.''

\href{https://www.nytimes3xbfgragh.onion/interactive/2018/03/22/magazine/voyages-kayaking-across-ocean-at-70.html}{{[}Read
about Aleksander Doba, the 71-year-old who kayaked across the Atlantic 3
times.{]}}

\textbf{It was mid-August} 2002, and we were 23, 24 and 25. We had
graduated from college together two years earlier. Dave, whom I also
grew up with, shot out of undergrad knowing he wanted to be a doctor and
had just finished his first year of medical school. Any similar momentum
I had after graduation was instantly sapped. Three nights after I
returned to my parents' house from school, I found myself driving my
father to the emergency room. Three weeks after that, he died. My grief
was disorienting and total; at a moment in life when everything is
supposed to feel possible, making any single decision became impossible.
I gave into that sadness for the better part of a year, resettling at
home in New Jersey with my widowed mother and sliding back to the summer
job I worked during school, glumly breaking down beef at a butcher shop
two towns over.

I coped with my fatherlessness and confusion in ways I'm not proud of
and still don't understand. I read a lot of books about Ronald Reagan,
for example, even the collection of his love letters to Nancy. I also
lashed out at Dave, who was living at home that summer, too, studying
for the MCAT. He withdrew awkwardly after the funeral, and I suppose I
was happy to hold that against him. It triggered some longstanding
jealousy. A part of me always resented how he seemed unfairly exempt
from the self-doubt and heaviness that I was prone to.

Jon, meanwhile, was teaching at a rustic little boarding school in
Switzerland, where his mother was from. The summer after graduation,
before starting the job, he set out for Alaska with a friend, sleeping
in the bed of their old pickup. In the minuscule town of Gustavus, the
gateway to Glacier Bay, he picked up seasonal work in the warehouse of a
kayak-tour company. Jon had little actual experience of sea kayaking but
had always felt drawn to the ocean in the abstract. In college, he and
another friend plotted out a paddling expedition near Glacier Bay,
across the border in Canada and applied for a grant from our school to
fund it. The grant was set up in memory of an alumnus who died in an
avalanche while mountaineering. It was meant to encourage the
``responsible and conscientious pursuit of wilderness expeditions.''
Safety was key. But the committee rejected Jon and his partner's
application. They seemed insufficiently prepared.

That wasn't surprising. Jon grew up doing a lot of backcountry camping
and was a competent outdoorsman, but putting together a grant
application required a kind of administrative fastidiousness he didn't
always possess. He was bright but scatterbrained, forever picking up
things and putting them down, both figuratively (music projects,
conversations) but also literally. I can still picture him hustling
around the house we shared in college, hunting for his keys or his
soldering iron, having gotten in over his head rewiring some device. He
was an artist; one piece I remember consisted of a half-peeled banana,
implanted with circuitry and suspended in a jar of formaldehyde. Once,
he grew grass in our upstairs bathroom --- a living bathmat, he said ---
until the turf became muddy and flooded the downstairs.

This was Jon's third summer in Alaska, and he'd worked his way up to
leading expeditions, taking out vacationers for days at a time. Our
trip, however, would venture beyond the typical circuit, into a remote
corner of the park that he'd never been to. Jon had no serious concerns
about our safety, but he felt he bore responsibility for our emotional
well-being. To enjoy ourselves, we would need to feel comfortable, not
just in the wilderness but also with him as a leader.

He suspected we wouldn't trust him entirely. We didn't. We knew him
before he became a professional guide, and our perception of his
expertise lagged behind the reality. ``With Jon,'' Dave told me, ``it
was always unclear to what extent he'd thought everything through.''
Dave remembered landing in Gustavus the night before we got underway and
casually asking Jon a lot of questions: Where are we going, exactly? Do
we have everything we need? Jon seemed to have solid answers for all of
them. As we headed back to his place for a good night's sleep, he told
us to wait in the yard. He was living alone for the summer in a house
that an acquaintance was building in the woods. The structure was
framed-up but largely wall-less, and Jon, to be safe, needed to check
that no moose had wandered in.

\textbf{After a spectacular} first day of paddling, we came ashore on a
rocky tidal flat about two miles from where we were dropped. Jon gave us
his detailed tutorial about bear safety while we set up our campsite. He
taught us, for example, to holler ``Hey, bear!'' if we heard any
rustling but also preventively, ahead of us, when we walked through the
woods. The last thing you wanted was to come across a brown bear
unannounced.

``Hey, bear!'' Jon kept hollering, by way of demonstration. He said it
goofily, like a children's TV host greeting some down-on-his-luck ursine
neighbor at the doorway to their clubhouse. This was intentional. Jon
had noticed that the people on his trips often resisted bellowing ``Hey,
bear!'' into the wilderness. It was essential for their safety, but it
felt silly or vulnerable somehow, like singing in public. So he learned
to turn it into a shtick, spinning it into a stream-of-consciousness
narration: Hey, bear, I'm coming into the trees now. Hope you're having
a fantastic evening, Mr. Bear! It loosened everyone up. They were
performing for their friends now; the whole group was in on the joke.

I had never seen a wild bear, though I have backpacked in bear country a
handful of times. I felt comfortable with the animals in the abstract.
But here, the bears weren't abstract; they breached the material plane.
There were bear trails everywhere, leading from the tree line to the
water, and disquietingly close, I felt, to where we were pitching our
tent. We found heaps of their scat. We saw trees where the animals had
slashed off the bark to eat the inner layer, tufts of fur from their
paws still plastered in the sap.

I pretended I was having fun. But that evening I grew increasingly
petrified, almost delirious. My eyes tightened, scanning for bears. The
sound of the wind became bears, and so did the mossy sticks cracking
under our feet. I gave myself a migraine, then phased in and out of
sleep.

At sunrise, I woke feeling foolish. While Jon cooked pancakes, I
reasoned with myself, privately, in a notebook I brought on the trip. I
tried to conceive of the situation as a geometry problem. Yes, some
number of bears roved this landscape, I wrote: relatively tiny,
independent blips, going about their business randomly, just like us. In
all that empty space and confusion, a lethal collision of their moving
blips and our moving blips would be an improbable coincidence. I'd been
distorting those odds, mistaking myself for ``the absolute focus of all
bears' attention,'' I wrote. It was embarrassing, really. ``To be afraid
of bears,'' I concluded, ``is to be narcissistic.''

I was reminding myself that freakishly horrible things are, by
definition, unlikely to happen. Even now, my reasoning feels sound.

\href{https://www.nytimes3xbfgragh.onion/interactive/2018/03/22/magazine/voyages-joshua-tree-lost-hiker.html}{{[}Read
about the search for a missing hiker in Joshua Tree's wild interior.{]}}

\textbf{Day 2 was} a slog. We paddled through a spitting drizzle in an
endless straight line, along the high granite walls of the coast. We
talked less and less, just pushed through the emerald chop. Then
eventually we gave up, hauling in our boats and making camp in a wide,
crescent-shaped cove, short of the site that Jon originally picked out
on his map.

We had entered Dundas Bay, a rarely visited pocket of the national park
that, I've since learned, has a storied history as a hide-out for
solitary misanthropes. In the 1930s, one prospector built a cabin not
far from our campsite and brandished a gun at the Alaska Natives who
passed through.

We intuited that the scenery was beautiful, but we could see very little
of it through the fog. Our guidebook explained that ``the east side of
the bay'' --- where we were --- ``can get extremely rough during foul
weather, since large waves roll in ... and batter this shoreline.'' That
was happening now: The weather that plinked at us all afternoon was
roiling into a storm. Soon, the big rain started. We rushed through
dinner, then loafed in our tent until, eventually, the loafing turned to
sleep.

A local newspaper would later describe the storm as ``short but
intense.'' In Gustavus, a creek swelled to about a foot higher than its
previous record. Gale winds, with gusts up to 59 miles per hour, turned
back two cruise ships in Skagway, about 85 miles north. Around 2 a.m.,
we woke to discover the wind had shorn the rain fly off our tent. Jon's
sleeping bag and mine were soaked, while Dave was snug and dry between
us. We heard torrents of water lashing down and the waves crashing in
the cove.

We got up three or four hours later. The rain and wind no longer felt
ferocious but were still too gnarly to paddle through; there was no
question, Jon said, that we were staying put. We cooked breakfast and
took turns playing chess in the tent. By late morning, the storm seemed
to have passed. We were antsy. We figured we would take a look around.

The terrain was crammed with thickets of alder and spruce, underlain by
ferns and a furor of prickly things. Jon pointed out devil's club: three
or four feet tall and leafy, armored up and down with spines. The plant
pierced fleece and hurt like fire.

There were no trails. We'd been trudging for some time when we reached a
fast-moving stream, maybe 10 feet wide. Jon was surprised; it wasn't on
his map, most likely just a drainage bloated by the storm. We followed
it downstream, looking for a way across, and eventually found it bridged
by a hefty tree trunk. It seemed like an easy crossing. Jon stepped up
and led the way, and Dave and I waited in a single-file line on the
stream bank behind him. The creek was loud, like a factory with all its
gears and rollers churning. Looking down, Jon realized there was more
water than he'd thought.

That's when I heard the snap in the woods behind me. After all my
paranoia, I instantly understood that the many bears I'd thought I heard
before were absolutely not bears --- were nothing --- because this sound
was so unmistakable and crisp, so explicitly something. I turned and
hollered, ``Hey, bear!'' then waited a beat. Maybe I said ``Hey, bear!''
again; I'm not sure. But I must have scanned those trees long enough to
feel satisfied and safe, because I know I was turning my head, to go
back to my friends, when I saw the dark shape rushing forward in my
peripheral vision.

\textbf{What I heard} must have been roots popping. If a tree is large
enough, you can apparently hear them cracking underground like gunfire.

The thud was seismic. The trunk crashed down right next to me. Mapping
out bits of evidence later, we concluded that the tree must have been
about 80 feet tall and perhaps two feet in diameter. It was some kind of
conifer --- a spruce or cedar. I screamed, involuntarily, ``Look out!''
then watched Dave, a few steps directly in front of me, dive sideways
and hit the ground. When I got to him, he was crouching, stunned but
O.K. He looked up and said, ``Go get Jon.''

It hadn't clicked back in for me: There were three of us. The sight of
Dave going down had canceled out everything else. I scrambled out over
the creek, running across the tree that had just fallen, shouting Jon's
name, then spotted him in the water, tangled in a snarl of sheared-off
branches near the bank behind me --- a cage, which kept him from
hurtling downstream.

He did not know he'd been hit by a falling tree. It had narrowly missed
his head, struck his left shoulder, shearing it from his collarbone and
breaking many of his ribs. Later, a doctor would explain that the
downward force had been so powerful that it had probably squashed Jon's
entire upper body, and all the organs inside, down toward his waist,
momentarily compressing him like a bellows; for a split second, his
shoulders headed in the direction of his bellybutton, before his torso
sprang up again.

Jon had heard nothing, seen nothing. He was turning around to help Dave
onto the log --- again, feeling responsible for our safety --- and the
next thing he knew, he was in the water. He tried to reach out his left
arm but could not make it move. He could not move his legs. He felt a
bolt of pain down his spine.

Jon later described flashing through an idiosyncratic sequence of
thoughts, all in a few milliseconds, as if watching a deck of cards
fanning across a table. One was an image of himself in a wheelchair,
sitting behind a mixing console in a fancy recording studio. ``I guess I
can become a recording engineer in a wheelchair,'' he remembered
thinking. He had never worked in a recording studio and, though he
played music, he had no particular plans to. Still, this vision
apparently felt like an acceptable future and freed him to resurface in
the present. That was when he registered me, screaming his name.

Jon told himself he shouldn't move. He knew from his many wilderness
first-responder trainings that moving a person with spinal injuries
risks paralysis. Then again, he also knew that most of his body was
submerged in cold water, and he recognized that he risked dying of
hypothermia if he didn't move. ``If I'm already paralyzed,'' he
concluded, ``I may as well move.''

He somehow hoisted himself out of the stream before Dave or I got to
him, using his right arm and his chin and biting into something loamy
with his teeth, for additional leverage. He reassessed the situation:
better. Also: worse. He now realized that we were at least a mile inland
from our camp.

Suddenly, his body was walking; his legs just started working. Dave and
I put him between us, supporting his frame. He was moving faster than we
expected, but uncoordinatedly. Then he crumpled between us. We tried
again; Jon was dead weight. Dave noticed that his breathing was shallow
and his voice was low --- signs, Dave knew from med school, of a
collapsed lung. He began battering Jon with a pep talk, telling him,
firmly, that he had to get up, that we had to get out of here. Jon
didn't need that explained to him; he was cogent and still trying to
plot our next steps in his mind. He looked down to see why this log he
was resting on was so lumpy and realized that he was, in fact, sitting
on his left arm. The arm was slack, obviously broken; his sleeve,
pierced up and down with devil's club. Jon had zero feeling in it. He
found it amusing, this sensation of complete estrangement from one of
his limbs.

Jon had been stressing that it was important to stay together. But this
was another theory of wilderness survival that appeared to be breaking
down in practice. Someone would have to get on the radio back at our
camp. By chance, while marooned in our tent during the rainstorm the
night before, Jon showed us how to use the device, though he did it
almost as a formality; the hand-held VHF unit was merely a line-of-sight
radio, he told us, meaning its range was small, its signal too weak to
pass through most obstacles. You were unlikely to reach anyone you
couldn't see, and we hadn't seen anyone since a faraway fishing boat,
early on Day 1.

There was a moment of discussion, or maybe just an exchange of looks
between me and Dave. I told Dave he should go. I didn't trust myself to
find my way back. I also knew that I lacked the courage to try; whether
I was being sensible or cowardly, I still don't know. Besides, I took
for granted that Dave would make it. He was more capable in my mind,
less likely to cinch himself in indecisive knots.

Recently, though, Dave told me: ``You probably had no idea how much in
my own head I was. I know that you, growing up, definitely felt insecure
about things, and I think you looked at me and thought, Dave has
everything figured out. But I had so much anxiety.'' He brought up the
tremor he used to have in his hands. I knew about it; in high school, we
waited tables together, and I occasionally had to carry out Dave's soup
orders, so he wouldn't spill. But I guess I thought of the tremor as
strictly physiological. I couldn't see the vulnerability causing it.

Now, as Dave sprinted away from me and Jon, swatting devil's club from
his path with the rubberized sleeve of his rain jacket, his nerves rose
up and rattled him. He worried he wouldn't be able to find the radio
once he got back or know how to turn it on. What if he broke the radio,
foreclosing whatever marginal chance we had of getting help? There were
lots of ways to screw this up, Dave realized. More occurred to him as he
ran.

He found the radio. He turned it on. Then, having solved these problems,
he encountered another he hadn't anticipated: ``What is the appropriate
thing you're supposed to say?'' he remembered thinking. On TV, you see a
lot of people saying ``Mayday.'' And so, Dave faced the open water and
started broadcasting into the fog: ``Mayday, Mayday.'' Even in that
moment, though, alone on a beach in the middle of nowhere, he felt
slightly self-conscious about it. This is so goddamn cliché, he thought.

\textbf{Back in the woods,} kneeling over Jon, I was having the same
problem: I didn't know what to say. He was lying near a log on his
injured side, his beard and glasses flecked with dirt and tendrils of
moss. He seemed to be on the brink of losing consciousness. At no time
would the possibility of Jon's dying surface concretely in any of our
minds. Still, I knew I was supposed to keep talking to him, to tether
him to the world with my voice somehow.

I started vamping platitudes: We were going to get out of here soon, and
so forth. But I could feel myself treading water, even blundering, at
one point, into a long-winded apology, worried I overstayed my welcome
that one Christmas with his family. I was afraid that the helplessness
in my voice might be counterproductive, unsettling Jon instead of
steadying him. It was a tremendous silence to fill.

What can a person say? I had two literature professors in college who
made us memorize poems. You never knew when some lines of verse would
come in handy, they claimed. One liked to brag that, while traveling
through Ireland, he found that if he spat out some Yeats at a pub, he
could drink free. This is how I wound up reciting a love poem to Jon.

It was ``The Shampoo,'' by Elizabeth Bishop, a lyric poem about the
enormity of time, which turns startlingly intimate at the end, when
Bishop offers to shampoo her lover's silvering hair: ``Come, let me wash
it in this big tin basin,/battered and shiny like the moon.''

After that, I imagine I also did some W.H. Auden; I knew a fair amount
of Auden back then. The stuff in rhyme and meter was always easiest to
memorize --- ``Looking up at the stars, I know quite well/That, for all
they care, I can go to hell'' --- which is why I had a lot of Robert
Frost at my disposal as well:
\href{https://www.poetryfoundation.org/poems/42891/stopping-by-woods-on-a-snowy-evening}{``Stopping
by Woods on a Snowy Evening,''}
\href{https://www.poetryfoundation.org/poems/44272/the-road-not-taken}{``The
Road Not Taken.''} For the most part, I trafficked in hits.

Jon and I would spend about an hour and a half together alone on the
forest floor. I ran through everything in my quiver --- Kay Ryan, A.R.
Ammons, Michael Donaghy --- padding each poem with little prefatory
remarks, while Jon said nothing, just signaled with his eyes or produced
a sound whenever I checked in. I felt like a radio D.J. playing records
in the middle of the night, unsure if anyone was listening. And here's
one about owls by Richard Wilbur, I would tell Jon, and off we would go.

I must have also done at least one by Hayden Carruth, my curmudgeonly
pen pal at the literary magazine. Carruth's poems didn't lend themselves
to memorization, but I'd worked hard to nail one of my favorites, in
which he describes stopping to notice a deer standing in an apple
thicket, then realizing the northern lights are flaring overhead. Hayden
and the animal pass a moment in stillness together. ``We are proud to be
afraid,'' he writes, ``proud to share/the silent magnetic storm that
destroys the stars.'' Relative to that boundless violence above them, he
and the deer are momentarily allied, though still not entirely
connected: ``a glimpse, an acknowledgment/it is enough and never
enough.''

That's what I said to my friend, powerlessly, tenting my jacket over his
face when it started to rain. The title of the poem is:
\href{https://www.poetryfoundation.org/poems/42586/i-know-i-remember-but-how-can-i-help-you}{``I
Know, I Remember, But How Can I Help You.''}

\textbf{The Coast Guard} cutter Mustang wasn't where it was supposed to
be. The 110-foot patrol boat normally spent its time coursing through
the Gulf of Alaska, inspecting halibut-fishing vessels, or circulating,
as a terrorist deterrent, near the oil terminals at Valdez. It was
home-ported in Seward, hundreds of miles from Glacier Bay. But the crew
was transiting to Juneau for a training when, a few days earlier, they
were smacked by the same storm that later poured inland, over us. ``We
had gotten absolutely pummeled,'' John Roberts, a petty officer on the
Mustang, told me recently. For two days, the boat swished around in
15-foot-plus seas. Many on the crew had been hunkered in the mess deck,
vomiting, while Roberts and a couple of his shipmates did their best to
cover everyone's watches. Finally, the Mustang slipped into Glacier Bay
to find some protection. The weather started to ease. That afternoon, as
Roberts piloted the Mustang east, toward Dundas Bay, his pallid
crewmates were finally staggering back up to the bridge, asking where
the hell they were.

That was when Dave's Mayday call came through. The signal on the
Mustang's radio was thin and faint, barely edging into range. Another of
the ship's petty officers, Eamon McCormack, explained to me that in
retrospect the connection feels ``mind-boggling.'' Glacier Bay National
Park extends over more than 5,000 square miles. Our signal would have
covered two or three miles at most. And yet, a boat --- a Coast Guard
boat, no less --- happened to be passing through that exceedingly small
window at precisely the right time. ``I don't know if, nine times out of
10, you play that over again and the outcome would be the same,''
McCormack said. A moment earlier or later --- seconds, potentially ---
and we might have slipped out of alignment. The moving boat would have
cruised out of range, uncoupling from us forever.

\hypertarget{-1}{%
\subsection{}\label{-1}}

\textbf{It was 1:25 p.m}. when the Mustang received Dave's call,
according to one of the subsequent Coast Guard reports. Roberts couldn't
believe it. ``Come on, man, I'm tired,'' he said aloud, wearily, to the
receiver in front of him. Roberts waited for a moment, per protocol, on
the off chance that the Coast Guard's central communications center in
Juneau would pick up the call instead. Then he turned and asked his
watch commander to pull out all the standardized search-and-rescue
paperwork. He was steeling himself, resummoning his professionalism. ``I
guess we're doing this,'' he said.

Roberts was the crew member on the Mustang with the most current medical
training; he would complete his E.M.T. certification the following
month. As he started firing questions at Dave on the radio, he didn't
like the answers that he heard coming back: the shallowness of Jon's
breathing, the likelihood of a punctured lung. More fundamental, Roberts
remembered: ``Any time a tree falls on somebody, it's not good.'' He was
also unsettled to learn that Dave and I both lived in New York City ---
a red flag, he had found, when someone winds up in trouble in the
wilderness.

We were 100 nautical miles from the nearest hospital; a half-day trip,
even in ideal conditions. The Mustang requested that the Coast Guard Air
Station in Sitka send a helicopter, but the immediate plan was for
Roberts and three crewmates to peel toward shore in the ship's Zodiac
and track us down. Dave had found the flare in Jon's emergency kit and
now, at 2:20, with the Zodiac underway, the Coast Guard asked him to
fire it. He was still in front of our campsite, facing the water. He'd
never shot off a flare before. He aimed straight up, then watched as the
bright tracer rose and arced somewhere far behind him, deep in the
woods. He was uncertain whether this counted as a success. He started
scanning the fog in front of him, but the Zodiac never appeared.

Someone on the Mustang caught sight of the flare near the end of its arc
and immediately directed the crew on the Zodiac toward it, steering them
far away from Dave to the opposite side of the little peninsula we'd
camped on. And yet, this was lucky: they wound up coming ashore much
closer to where I was waiting in the woods with Jon. Soon, whatever poem
I was reciting was interrupted by whistles blowing and voices calling,
and eventually three shapes, wearing hard hats and heavy orange rain
gear, rushed toward us out of the trees.

Roberts was especially impressive, a reassuringly large Boston-area
native with a booming voice. He knelt and took Jon's vitals. The
information was troubling: his pulse was 60 beats per minute; his
breathing, fast and shallow. They put his neck in a brace and eased him
onto a kind of truncated backboard, called a Miller board, to move him
out to the beach. Dave had returned by then. He and I crouched at one
end of the board, near Jon's feet, as someone --- presumably Roberts ---
bellowed a count of three to lift.

Later that night, lying down to sleep in a bed-and-breakfast in Gustavus
--- stunned and depleted, but dry and warm --- Dave and I would talk and
talk, reviewing the entire ordeal. We had drooped into a long silence,
coasting toward sleep, when Dave spoke up with one last observation.
When we were getting ready to lift Jon on the backboard, he said, it
occurred to him that this was one of those crisis moments you hear
about, like when mothers are suddenly able to lift a car off their baby.
Dave expected we were going to have superhuman strength.

We did not have superhuman strength. On Roberts's command, the men
raised Jon to waist height, swiftly and seemingly perfectly level, as
though their arms and deltoids were hydraulic. Then, in one motion, they
took off downhill, with negligible help from us. This can't be accurate,
but I remember the sensation of being almost dragged, like children in a
sled.

\textbf{A National Geographic} television crew was embedded at the Coast
Guard's air station in Sitka, filming an installment of a thrill-ride
reality series. The network had sent crews to other Coast Guard stations
around the country too, though this assignment appeared to hold the most
dramatic potential. Air Station Sitka was unique: Its pilots were
responsible for 12,000 miles of coastline, a sprawling, treacherous
wilderness riven with fjords, inlets and glaciers, often buffeted by
implacably horrible weather. People who went into the backcountry in
Alaska had a way of getting themselves into a different magnitude of
trouble, too; as Roberts put it, ``When stuff happens in Alaska, it's
big.'' Still, this was the television crew's eighth day in Sitka, and as
the show's producer, Annabelle Hester, explained: ``I was having calls
with my bosses at headquarters saying, `Nothing is happening!' We were
scrambling to come up with Plan B.'' Then, the Mustang's call came in at
1:42.

``What type of injuries are we looking at?'' asked the dispatcher. She
was taking the call from behind a semicircular counter, like the
reception desk at a midlevel corporate branch office. She had a framed
snapshot of a parakeet to brighten her work space, and a photograph of a
dog with a heart that said, ``I Woof You.'' A cameraman stood
conspicuously beside her, holding a tense, tight shot.

``Probable broken ribs, a definite broken arm,'' said the man on the
other end. Then his voice faltered, seemed to give up: ``And whatever
else would happen to you if a tree fell on you,'' he added.

The dispatcher retrieved the appropriate paperwork and scribbled ``Tree
fell on person'' on one line. She read the current weather aloud: ``30
knots wind, 300 ceiling, heavy rain and one-mile vis.'' That would soon
be revised: the ceiling had dropped to 100 feet. Entering the weather
conditions on one of the Coast Guard incident reports, someone would
write, in a kind of nihilistic catchall: ``Extremely terrible.''

The Coast Guard's policy was to deploy a helicopter within 30 minutes of
the initial request, but the Air Station's operations officer, Cmdr.
Karl Baldessari, informed everyone that this mission would take longer
to plan. Baldessari was a 25-year veteran of the Coast Guard, a
fast-moving, sinewy man in a blousy flight suit, with a tidy mustache
and spiky hair. His role at the air station was that of a firehouse
chief. He was responsible for the safety of everyone working there,
which meant making judicious decisions about what warranted sending them
hurtling through the sky.

That calculus got knotty in conditions like these, though there was a
baseline volatility to flying in Alaska at all. The Coast Guard didn't
let its helicopter pilots fly lead out of Sitka, no matter how much
experience they had at other air stations, until they practiced
difficult landings at specific locations in the region and got their
egos battered a little by logging a full winter in the state. Visibility
in Alaska was frequently poor; conditions changed quickly. One pilot
told me about blindly tunneling through fog in the dark when his
co-pilot got ``caged'': The man lifted his eyes momentarily from his
instruments and, without any visual references or a horizon to latch
onto, found it impossible to reorient himself, lost all sense of
direction and was felled by vertigo.

During much of the year it was also cold enough, with sufficient
moisture in the air, that ascending to clear the region's many minor
mountains or even just flying through a cloud risked the aircraft's
icing up. To mitigate this, the Coast Guard had laid out virtual ``track
lines'' across the entirety of their range: a grid of GPS points and a
network of paths connecting them, along which pilots could chart a
course and fly at a relatively low altitude, confident they weren't
going to smash into a mountain. The system wasn't comprehensive; the
track lines got the pilots close to their destination, but ultimately
they had to diverge from this GPS superhighway and fly the remaining
distance the old-fashioned way, with their radar and eyes. It was like
taking an exit off the interstate, except there might be a granite wall
in front of you wherever you chose to get off. It was possible the
pilots would travel very far --- a half-mile away from whoever needed
their help --- only to discover that the last leg was too risky and be
forced to turn back.

Baldessari gathered the two pilots on duty that afternoon and the Air
Station's flight surgeon, then unrolled a large paper map. He pointed to
our location, explaining: ``That's probably one of the lousiest places
we fly in and out of. This Inian Pass, right here, is the worst place we
could possibly go.''

Inian Pass is a slim channel near the center of the Icy Strait, the
long, interconnected system of waterways stretching through Glacier Bay.
Conditions in the Icy Strait can be bad 300 days of the year, Baldessari
recently told me; wind, rain and storm surges all push through it fast
from the open ocean. But Inian Pass is a narrow keyhole at the center of
the strait --- a mile-wide opening between a few uninhabited islands and
a rocky point --- where all that weather speeds up. The only way for the
pilots to reach us would be to fly straight through it.

Nothing in the National Geographic footage, at this point, feels
reassuring. The flight surgeon holds his hand over his mouth and bites
his lip. The co-pilot, Chris Ferguson, only a few months into his
posting in Alaska, mills around and fidgets with his ear. It's obvious
Baldessari needs convincing. He wasn't eager to send his men up if he
didn't have to and wasn't certain they would make it all the way there
if he did.

``It's kind of funny,'' he told the pilots, pointing at the map.
``You've got a boat right here.''

\textbf{Lying on his} backboard like a burl of driftwood, Jon was
conscious and cognizant of his pain, but he had started to feel somehow
buffered from his body, uninterested in connecting with the world beyond
it. He would later describe himself as a ``thinking blob. It was a very
passive experience.'' He didn't know what was happening but could tell
our momentum had stalled. He was confused and felt impatient. In his
mind, the three of us had solved the impossible problem: We'd managed to
get help. This was supposed to be the simple part, when everyone rushed
him to the hospital. Instead, his condition deteriorated. Within 10
minutes of reaching the beach, Jon threw up. I'd never seen anything
like it, a kind of dark purple gristle. I took out my wool cap to wipe
his face, and he retched a second time, straight into my hat.

``I got that all over me,'' John Roberts told me recently. He'd seen
vomit like that before; it meant Jon had ingested a fair amount of blood
and signaled internal injuries. It made Roberts anxious. He had been on
the Mustang for two and a half years at that point but had spent the
previous four years in Palm Beach, a busy but less extreme posting that
often involved rescuing weekend boaters from relatively close to shore
--- and where, Roberts pointed out, the water is warm and won't
necessarily kill you if you go in. Moreover, the bulk of the Coast
Guard's training is for maritime rescues, not rescues on land.
Counterintuitive as it sounds, Roberts's comfort level and confidence
had dropped significantly once he hopped off the Zodiac and set foot on
the beach.

He reported back to the Mustang that Jon had thrown up, then soon
radioed again, explaining that Jon was going into shock. He kept giving
and requesting updates, trying to gauge how long this might take, and
eventually started erecting a makeshift shelter out of plastic sheeting
and medical tape, hoping to keep Jon out of the rain. Out of earshot of
us, Roberts explained to his crew mate Eamon McCormack what the vomit
meant: The possibility of Jon dying, here under their care, was real. At
one point in the National Geographic footage, as Roberts's calls are
relayed to the air station in Sitka, you can see where the dispatcher
clearly writes on her form: ``E.M.T. does not feel comfortable.''

By this time, the air station's flight surgeon had received enough
information to be alarmed. ``It sounds like he's got a pretty
significant chest injury,'' he told Baldessari. Baldessari understood
they would need to launch a helicopter but warned the Mustang that the
aircraft might not make it through the weather; ultimately it would be
the pilots' call, once they veered off their last track line and tried
to shoot through Inian Pass.

They would go and give it a look, Baldessari explained over the radio,
but the outlook was iffy. The guys on the beach, he said, must be
prepared to get Jon back on their cutter and haul him to a hospital
themselves, as fast as they could.

\textbf{One evening this} winter, my phone rang, and it was Karl
Baldessari. Long retired from the Coast Guard, he was teaching aviation
at a community college in Oregon, where I left a voice mail message
earlier that day. I meanwhile had metamorphosed into a 40-year-old
father of two and fumbled to explain to Baldessari that, as thrilled as
I was to have tracked him down, I was, at the moment, racing to finish a
risotto for my daughters before gymnastics practice and would have to
call him back. Without missing a beat, Baldessari blared orders at me,
joking, but still sounding as instinctually in charge as he did in the
National Geographic footage: ``O.K.,'' he said, ``you want to stir it
constantly, but slowly!''

I didn't expect any of the Coast Guardsmen I was cold-calling to
remember that day. However dramatic it remained for me, I assumed it
would have been obscured in a yearslong wash of more sensational
incidents. But everyone I spoke to did remember it, immediately and in
detail. Baldessari had been involved in hundreds of rescue operations
during his 30-year career, and yet, as I stood at the stove on the phone
that evening, he told me: ``The moment I listened to your voice mail, I
knew exactly the case! It was almost like it was yesterday.''

There was something about the supreme freakishness of the accident that
left a lasting impression. For those who came ashore, the experience was
also marked by a feeling of subtly escalating chaos and the pressure to
surmount it. McCormack told me that ours was a story he retold
endlessly, often to the younger Coast Guardsmen he was eventually tasked
with training. In it was a lesson about ``not taking situations that
look impossible at face value,'' he said. ``When things start to go
wrong, don't panic or lose sight of what resources you've got.'' Keep
working the problem until its absolute end --- even, McCormack added, if
it means deviating from official policy.

\hypertarget{-2}{%
\subsection{}\label{-2}}

McCormack was not supposed to be landing an inflatable boat on an
unforgivably rocky Alaskan shoreline, for example. But there he was,
anyway, beaching the Zodiac as gingerly as he could, so that Roberts and
the other men could load Jon aboard. They slid him in on his side ``like
a folder into a filing cabinet,'' as Jon put it, and started motoring
through the chop, very cautiously, back to the Mustang, about a mile
away.

As relieved as Jon had been when the Coast Guard first arrived, he also
felt instantaneously more vulnerable. Strapped to the back board, his
neck in the collar, he surrendered control of his body, however
imperfect that control had been. He was being hauled around as an object
now, with no ability to wriggle or shift positions, to manage his pain
or even to turn his head and see what was happening. He was helpless,
entirely dependent on the upright people operating around him, those
voices he could hear discussing him on the far side of some gauzy
divide. About 10 minutes into the trip on the Zodiac, Jon heard one of
those voices say, ``Oh, shit, we're losing air.''

A section of the Zodiac's sponson --- the inflatable fender that wraps
around the boat --- had punctured. One side was completely deflated.
``It's a big deal,'' McCormack recently explained to me, sounding
surprised that I had to ask. The sponson increases the boat's buoyancy
and stability, as well as keeping water from cresting over the side;
under normal conditions, a Zodiac with a broken sponson would have been
taken out of service automatically. Instead, McCormack found the
puncture and wedged the nozzle of a small pump inside. Then --- steering
the boat with one hand, operating the throttle with the other --- he
started working the pump with his foot, essentially doing leg presses,
to keep the fender partly inflated. The ride was already bumpy in
four-foot seas. Now McCormack began tracing a slow, zigzagging course,
doing what he could to tamp down the turbulence and the violence to
Jon's spine, as well as to guard against the possibility of the injured
man's suddenly bounding over the side on his backboard.

Roberts and the other Coast Guardsmen on the Zodiac leaned over Jon to
shield him from the splash. The pain was heinous; Jon seemed to be
passing out. Roberts talked to him, held his hand. Roberts felt crushed,
he told me; he was torturing this guy in order to save him. When they
finally reached the Mustang, rather than hoist Jon off the Zodiac, they
swung the ship's crane around and simply lifted the entire boat out of
the water, level with the deck, and then carried him aboard, to keep
from joggling him any more.

McCormack eventually returned for me and Dave, and a half-hour later we
were reunited with Jon in the Mustang's athwartship passageway, a
cramped, steel hallway, like the space between two cars of a train. Jon
was still battened to the backboard, wedged up to keep the weight of his
body on his less-painful side. They had cut off his clothes, though he'd
murmured a plea not to --- he was wearing a brand-new Patagonia jacket
that he had borrowed from a friend --- then swaddled him in a
hypothermia blanket. Dave and I knelt and rubbed his feet.

The helicopter was going to make it. I don't remember there being a
grand announcement. I'm not sure we were ever made aware of the
possibility that it wouldn't. Now the crew got busy below: tying down
anything that could be blown off by the rotor wash or stashing it in the
mess. I also don't remember hearing the helicopter when it finally
arrived. Instead, I remember only a heavy door to our left swinging open
to reveal, like a scene from an action movie, the silhouette of a man in
a blue flight suit, feet planted shoulder-width apart to steady himself
as the ship rocked sideways. The cable he'd been lowered on drew back
into the ocean spray and fog behind him. ``I'm flight surgeon Russ
Bowman,'' he said and stepped inside.

Bowman took Jon's vitals and gave him several, successive shots of
morphine. Soon, everyone was working to squeeze him back through the
narrow doorway and onto the deck where the helicopter, an MH-60 Jayhawk,
was idling overhead.

Until recently, the story I told about the accident unfolded in two
basic acts: the tree fell, instantaneously unleashing a kind of
unfathomable chaos; then the Coast Guard appeared and, just as swiftly,
regathered that chaos into order. It was like watching footage of an
exploding object, then watching it run in reverse. The maneuver the
Coast Guard was readying to execute now, on the deck of the Mustang,
would be the climax of that progression.

The helicopter hovered 30 or 40 feet over the boat, mirroring its speed
and trajectory, while both vehicles moved slowly forward. ``Looks like
you're heading for a rain squall,'' the co-pilot, Chris Ferguson,
radioed the Mustang at one point, and asked the ship to adjust its
course, to keep them in as forgiving weather as possible. Soon the
flight mechanic was calling out instructions to tuck the aircraft into
alignment: ``Forward and right 30. Forward and right 20. Forward and
right 10.'' Then, finally --- speaking, in the flight recordings, with
an almost galling air of imperturbability --- the lead helicopter pilot,
Rich McIntyre, radioed the flight mechanic to begin the hoist.

The whole procedure, from our vantage point, seemed seamless and
routine. In a way, it was: After the agonized deliberation at the air
station, the pilots exited off their GPS route into fairly manageable
conditions around Inian Pass. The winds were workable; the water wasn't
excessively choppy. Ultimately, scooping Jon off the deck of the Mustang
would resemble a standard exercise that the pilots drilled in their
trainings. ``Not to dumb it down,'' the co-pilot, Chris Ferguson, told
me --- plucking someone with a spinal injury off a moving boat and
hoisting them into a moving helicopter is a pretty insane thing to do.
``But we normalize what isn't normal.''

A few moments earlier, as the men scurried around Jon on his backboard,
packaging and fastening him for the hoist, Jon worried that the second
he got airborne he would start twirling uncontrollably, like the
feathery end of a cat toy, and potentially thwack his head on the
equipment on deck. But now, he was levitating smoothly --- a solitary,
swaddled bale of a man, perfectly perpendicular to the ground. Dave and
I watched it happen: our friend rising steadily away from us,
improbably, to safety. As Jon floated higher, he could hear the Coast
Guardsmen on the Mustang beneath him begin to cheer. He felt it was safe
to open his eyes. When he did, he saw someone, hunched in the open cargo
door of the helicopter, pointing a television camera at him.

 Video Clip From National Geographic Image Collection

\textbf{Jon was rushed} into surgery at the hospital in Sitka that
evening. He'd punctured both lungs, one to the point of collapse,
sustained multiple fractures on eight of his ribs, broken several
vertebrae, shattered his left shoulder blade and snapped his brachial
plexus nerves. His spleen had been macerated into countless flecks.
After awakening from surgery, Jon was disappointed that the doctors had
swept those shards into a bag and thrown his spleen in the trash; he
wanted to get a look at it, maybe even keep it preserved in a jar,
alongside his cyborg-banana.

Once back in Gustavus, Dave and I realized that we would need to call
Jon's parents in Switzerland. I didn't have to push the job on Dave this
time; he was adamant. He felt he would need to face conversations like
these if he was going to be a doctor. It was Jon's father who picked up,
and after absorbing the news, he paused and caught Dave off guard.
``Thank you,'' he said solemnly. ``You guys saved my son's life.''

Dave's stomach dropped. ``I remember thinking about it,'' he told me
recently, ``and realizing, Yeah. I guess, logistically, we did.'' I had
the same reaction when Dave hung up the phone and, clearly shaken,
relayed his conversation to me. Until that moment, the idea that we
saved Jon's life had never occurred to us, possibly because the idea
that Jon might have died still hadn't occurred to us. We had zero sense
of accomplishment, or even agency. In our minds, all we did was avoid
screwing up until the real help could arrive and save him.

But Jon hadn't absorbed the story that way. From the instant he willed
himself out of the water, he felt all of us locking into that same
seamless flow of order steadily displacing chaos that Dave and I only
experienced once the Coast Guard arrived. It was amazing to him how the
three of us managed to generate solutions for each successive problem.
Even my reciting those poems, which to me had always felt like a moment
of utter helplessness, became, in Jon's telling, a perfect emblem of
that streak of serendipitous problem-solving. ``You conveyed a
calmness,'' he told me recently. ``I remember it being this nice
moment.'' He added that if he ever has to spend two hours dying on a
remote forest floor again, having me there to recite poetry would be one
of his top ways to do it.

The feeling of inevitability that day became only more pronounced for
Jon as time passed and the entire story of our rescue receded into a
prologue to the rest of his life. The surgery in Sitka was only the
first of half a dozen, and it would take several years for him to regain
60 percent of the use of his arm, wrist and hand, as the nerves
gradually regrew along his injured side. He was in good enough shape to
go back to Alaska the summer after the accident --- repairing boats in
the company's warehouse and occasionally helping out at the
bed-and-breakfast --- but he struggled. He could repair kayaks but
needed help lifting them. He was unable to wrestle the mattress corners
into the fitted sheets when he made the beds. After that, he started
working at a recording studio in Portland, just as he envisioned while
stuck in the water, and he now runs his own audio-mastering company:
Spleenless Mastering.

Eventually Jon seemed to have recovered from the accident without any
conspicuous disabilities. But his life has been quietly corroded by
chronic pain and, almost equally, by the stresses of navigating the
doctors, medications (and their side effects) to manage it. About two
years after the accident, he learned he had PTSD. The trauma wasn't the
falling tree, but his experience of powerlessness as a perpetual patient
in the American medical system. It manifested as a kind of unbearable
empathy for anyone who was suffering. Jon found himself shouting at
doctors, on his own behalf but also on behalf of strangers in waiting
rooms who weren't being seen. He would hear interviews with
natural-disaster victims or the homeless on NPR and have to pull his car
over. There continued to be other tribulations, too --- more mundane
ones. A few times a year, he still rebreaks a rib out of nowhere; once
or twice, Jon told me, all it has taken is an especially affectionate
hug from his wife.

Jon found early on that he could cordon off this suffering, both in his
own mind and in conversation, by making jokes about the accident itself
and sticking to the happy ending of our rescue, a trick that got much
easier after the National Geographic show aired later that year.
``Mission Rescue: Final Frontier,'' the program was called. The
soundtrack was all heart-thwacking synth drums and shredding guitar. A
foreboding, Ken Burns-effected snapshot of Dave and Jon looking joyful
before the trip gave way to a whirring re-enactment of someone else's
legs --- cast in the role of Dave's legs --- sprinting through the
blurry woods for our radio. A melodramatic narrator pondered the fate of
``Kayaker Jon Cohrs.''

Initially, the schlockiness of the production felt like a blessing. The
show depersonalized the accident, giving us all a shorthand to convey
how dramatic that day had been, without confronting how destabilizing
and senseless it might have felt. At a party, you could lay out the
basics --- a tree fell on Jon --- then say, ``National Geographic even
made a TV special about it,'' and everyone would go wide-eyed but then
move on, figuring you would unspool the real story some other time.

But we never realized the degree to which that kitschy shorthand started
to obscure the real story --- then, gradually, to replace it. I'm
embarrassed to admit that, though Jon and I have remained close, I did
not know the extent to which he has continued to suffer for the last 17
years until talking to him for several hours in order to write this
account.

\textbf{The morning after} the accident, Dave and I traveled back to
Dundas Bay to pack up our campsite and collect the kayaks we abandoned
the previous evening. We were shuttled there from Gustavus by the same
boat captain who dropped us off three days earlier, a forbiddingly
taciturn commercial fisherman named Doug Ogilvy.

The tide in the cove was way out when we arrived; it was, as Ogilvy put
it, ``a suck-ass beach.'' The approach was so shallow that he had to
drop anchor a hundred yards or more from shore. He asked if we had
waders. We did not. So Ogilvy put on his, climbed down the ladder and
told Dave to get on his back. Then stoically, like an ox or an old-timey
strongman hauling a safe, he trudged through the thigh-high water,
dropped Dave on the gravel beach, then lurched back and hauled me the
same way, as if I were a man-size infant in a papoose.

Dave told me he'd had a strange feeling on the ride out, as if we would
discover that an even more massive tree had fallen on our tent since we
last slept there and that all three of us would have been crushed and
killed if we'd spent another night in Dundas Bay, as planned. That is,
he half-expected to find evidence that the accident had been fortuitous
somehow, that there was a reason, or redemptive value, behind it. My
mother had the same instinct when I called her the night before. On the
phone I strained to emphasize for her --- she was only two years into
her cruelly premature widowhood, and I was new at being the
overprotective son of a widow --- that Jon was going to be all right,
and that Dave and I were safe. She told me that my dad must have been up
there looking out for us somehow.

I resented all the supernatural thinking. If it comforted other people,
fine, but I'd somehow known right away that I didn't need a reason for
the accident. It was senseless, but straightforward, as unequivocal a
fact as my father's death had been. A tree fell in the woods. It might
not have, but it did. Jon could have died, but he didn't. Other
possibilities spiraled infinitely outward from there, though apparently
I wasn't too interested in contemplating them. As strange as it sounds,
it was years before I realized that the tree could have hit me --- and
only after a friend pointed this out, as I told the story around a fire
one night. And it was only a few weeks ago, while on the phone with Jon,
that it occurred to me that the tree could have hit all three of us ---
we were standing in a single-file line, after all, waiting to cross the
creek --- and that we all might have wound up clobbered and scattered in
that river, dying slowly and watching each other die.

It's also probably true that I helped preclude these possibilities by
being so feverishly paranoid about bears, wheeling around at the sound
of the snapping roots. That's what allowed me to see the tree coming,
just barely, and scream that infinitesimal heads-up for Dave. And so,
the real meaning of the accident, if I felt compelled to find one, might
be that it validated my most exaggerated fears. But instead, it somehow
helped cleanse me of them. There was comfort for me in accepting the
arbitrariness of what happened, in regarding it as a spasm of random
damage in time and space that, just as randomly, a small number of human
beings got the opportunity to repair. We were more capable than I had
understood. We were also far more helpless.

On the ride back to Gustavus with our gear, I pictured myself, again, as
a small blip in empty space. The ride was rough and jumpy as Ogilvy
impatiently pounded his boat through the last vestigial wave energy of
the storm; Dave and I had to hold on, to plant ourselves on the bench
behind him. But there was a moment when I felt so safe that I loosened
my grip, leaned slightly into the motion of the boat, and, closing my
eyes, felt myself lift off the seat.

\emph{Jon Mooallem is a writer at large for the magazine who is working
on a book about the great Alaska earthquake of 1964.
\href{https://www.nytimes3xbfgragh.onion/2017/04/19/magazine/our-climate-future-is-actually-our-climate-present.html}{His
last feature for the magazine was about our climatological future.}}

The Voyages Issue

\hypertarget{-3}{%
\section{}\label{-3}}

Rick Steves

Wants to

Set You Free

 The travel guru believes the tiniest exposure to other cultures will
change Americans' entire lives.

By \href{https://www.nytimes3xbfgragh.onion/by/sam-anderson}{SAM
ANDERSON}

photo illustrations by:

Zachary Scott

\textbf{Rick Steves can} \textbf{tell you} how to avoid having your
pocket picked on the subway in Istanbul. He can tell you where to buy
cookies from cloistered Spanish nuns on a hilltop in Andalusia. He can
tell you approximately what percentage of Russia's gross domestic
product comes from bribery. He can teach you the magic idiom that
unlocks perfectly complementary gelato flavors in Florence (``What
marries well?'').

But Rick Steves does not know his way around New York City.

``In the Western Hemisphere,'' Steves told me one afternoon last March,
``I am a terrible traveler.''

We were, at that moment, very much inside the Western Hemisphere, 4,000
miles west of Rome, inching through Manhattan in a hired black car.
Steves was in the middle of a grueling speaking tour of the United
States: 21 cities in 34 days. New York was stop No.17. He had just flown
in from Pittsburgh, where he had spent less than 24 hours, and he would
soon be off to Los Angeles, Denver and Dallas. In his brief windows of
down time, Steves did not go out searching for quaint restaurants or
architectural treasures. He sat alone in his hotel rooms, clacking away
on his laptop, working on new projects. His whole world, for the time
being, had been reduced to a concrete blur of airports, hotels, lecture
halls and media appearances.

In this town car, however, rolling through Midtown, Steves was brimming
with delight. He was between a TV interview at the New York Stock
Exchange and a podcast at CBS, and he seemed as enchanted by all the
big-city bustle as the most wide-eyed tourist.

``Look at all the buildings!'' he exclaimed. ``There's so much energy!
Man, oh, man!''

A woman crossed the street pushing two Yorkies in a stroller.

\hypertarget{-4}{%
\subsection{}\label{-4}}

``How cute!'' Steves shouted.

The town car crawled toward a shabby metal hulk spanning the East River.

``Wow!'' Steves said. ``Is that the Brooklyn Bridge?''

It was almost the opposite of the Brooklyn Bridge. The Brooklyn Bridge
is one of the most recognizable structures in the world: a stretched
stone cathedral. This was its unloved upriver cousin, a tangle of
discolored metal, vibrating with cars, perpetually under construction.
The driver told Steves that it was the Ed Koch Queensboro Bridge --- or,
as most New Yorkers still thought of it, the 59th Street Bridge.

This revelation only increased Steves's wonder.

``The 59th Street Bridge!'' he said. ``That's one of my favorite
songs!''

With buoyant enthusiasm, Steves started to sing Simon and Garfunkel's
classic 1966 tune ``The 59th Street Bridge Song (Feelin' Groovy).''

``Slow down, you move too fast,'' he sang. ``You got to make the mornin'
last --- just --- kickin' down the cobblestones. ... ''

The car hit traffic and lurched to a stop. Steves paused to scan the
street outside. ``Where are the cobblestones?'' he asked. Then he
refocused. He finished the song with a flourish: ``Lookin' for fun and
feelin' --- GROOOVYYYYYY!''

There was a silence in the car.

``Can you imagine those two guys walking around right here?'' Steves
said. ``Just feeling groovy? Gosh, that's cool.''

Steves pulled out his phone and, for his online fans, recorded a video
of himself singing ``The 59th Street Bridge Song (Feelin' Groovy).''

``It's fun to be in New York City,'' he signed off. ``Happy travels!''

There was another silence in the car, this one longer.

``You know,'' the driver said finally, ``you're not very different than
you are on your show.''

This was correct. The driver was referring to Steves's long-running,
widely syndicated, family-friendly public-television travel series,
``Rick Steves' Europe,'' on which Steves is a joyful and jaunty host,
all eager-beaver smiles and expressive head tilts. With a backpack over
one shoulder and a hand tucked into his pocket, Steves gushes poetically
about England's Lake District (``a lush land steeped in a rich brew of
history, culture and nature'') and Erfurt, Germany (``this half-timbered
medieval town with a shallow river gurgling through its center'') and
Istanbul (``this sprawling metropolis on the Bosporus'') and Lisbon
(``like San Francisco, but older and grittier and less expensive''). He
reclines jauntily atop the cliffs of Dover and is vigorously scrubbed in
a Turkish bath. The show has aired now for nearly 20 years, and in that
time, among travelers, Steves has established himself as one of the
legendary PBS superdorks --- right there in the pantheon with Mr.
Rogers, Bob Ross and Big Bird. Like them, Steves is a gentle soul who
wants to help you feel at home in the world. Like them, he seems
miraculously untouched by the need to look cool, which of course makes
him sneakily cool. To the aspiring traveler, Steves is as inspirational
as Julia Child once was to the aspiring home chef.

\href{https://www.nytimes3xbfgragh.onion/2015/10/25/magazine/terry-gross-and-the-art-of-opening-up.html}{{[}Read
a profile of Terry Gross, the host of ``Fresh Air.''{]}}

Eventually, Steves's busy New York day ended on the Upper East Side,
where he was scheduled to give a talk at a Barnes \& Noble. As we drove
to the event, Steves confessed that he wasn't sure what kind of crowd he
would get. You never knew exactly where his Rickniks (as the hard-core
fans call themselves) would materialize en masse. Some Steves
appearances were mobbed; others were sparse. His appeal is slightly
cultish. For every Ricknik out in the world, a large contingent of
average people have no idea who he is.

I was mildly skeptical about Steves's drawing power in New York. It was
hard to imagine a bunch of cynical, worldly, urban, polyglot,
multicultural East Coast sophisticates --- people who probably
vacationed at deconsecrated eco-hostels in Oman or Madagascar ---
getting excited about public television's reigning expert on Europe.

We arrived, however, to find the bookstore overflowing. A solid wave of
applause met Steves at the door. Fans had been pouring in, the organizer
told us, for two solid hours. People sat in the aisles and stood in the
back. Some wore T-shirts and hats bearing the Rick Steves slogan: ``Keep
on Travelin'.'' The crowd's body heat overwhelmed the building's climate
control.

I noticed a group of hipster 20-somethings standing near the back, and
at first I assumed they had all come sarcastically. But as Steves began
to speak, they grinned and laughed with absolute earnestness. Everyone
here was, apparently, a superfan. At one point, Steves showed a slide of
tourists swimming in a sunny French river underneath a Roman aqueduct,
and the whole crowd gasped. When he mentioned that his website featured
a special video devoted to packing light for women, a woman in the crowd
actually pumped her fist.

At the end of his talk, Steves offered to sign books --- but not in the
traditional way. There were too many people for a signing table, he
said, and anyway, single-file lines were always inefficient. (This is
one of his travel credos: avoid waiting in line.) Instead of sitting
down, Steves walked out into the center of the room and invited everyone
to open their books and surround him. He pulled out a Sharpie. And then
he started to spin. Steves held out his pen and signed book after book
after book, fluidly, on the move, smiling as the crowd pressed in. ``We
went to Portugal on our honeymoon,'' a man shouted. ``How romantic!''
Steves answered, still spinning. A woman asked him where to celebrate
Christmas in Europe. Steves, in midrotation, still signing furiously,
told her that he had made a whole special about precisely that question
and that it was available free on his website. ``Keep on travelin',
Rick!'' someone shouted. ``Keep on travelin'!'' Steves shouted back. As
he spun, Steves thanked everyone and gave quick, off-the-cuff advice. In
an astonishingly short time, he had signed every book. The people were
satisfied. The crowd thinned. Steves finally came to a stop.

\textbf{Rick Steves is absolutely American.} He wears jeans every single
day. He drinks frozen orange juice from a can. He likes his hash browns
burned, his coffee extra hot. He dislikes most fancy restaurants; when
he's on the road, he prefers to buy a foot-long Subway sandwich and
split it between lunch and dinner. He has a great spontaneous honk of a
laugh --- it bursts out of him, when he is truly delighted, with the
sharpness of a firecracker on the Fourth of July. Steves is so
completely American that when you stop to really look at his name, you
realize it's just the name Rick followed by the plural of Steve --- that
he is a one-man crowd of absolutely regular everyday American guys: one
Rick, many Steves. Although Steves spends nearly half his life
traveling, he insists, passionately, that he would never live anywhere
but the United States --- and you know when he says it that this is
absolutely true. In fact, Steves still lives in the small Seattle suburb
where he grew up, and every morning he walks to work on the same block,
downtown, where his parents owned a piano store 50 years ago. On
Sundays, Steves wears his jeans to church, where he plays the congas,
with great arm-pumping spirit, in the inspirational soft-rock band that
serenades the congregation before the service starts, and then he sits
down and sings classic Lutheran hymns without even needing to refer to
the hymnal. Although Steves has published many foreign-language phrase
books, the only language he speaks fluently is English. He built his
business in America, raised his kids in America and gives frequent
loving paeans to the glories of American life.

And yet: Rick Steves desperately wants you to leave America. The tiniest
exposure to the outside world, he believes, will change your entire
life. Travel, Steves likes to say, ``wallops your ethnocentricity'' and
``carbonates your experience'' and ``rearranges your cultural
furniture.'' Like sealed windows on a hot day, a nation's borders can be
stultifying. Steves wants to crack them open, to let humanity's breezes
circulate. The more rootedly American you are, the more Rick Steves
wants this for you. If you have never had a passport, if you are afraid
of the world, if your family would prefer to vacation exclusively at
Walt Disney World, if you worry that foreigners are rude and predatory
and prone to violence or at least that their food will give you
diarrhea, then Steves wants you --- especially you --- to go to Europe.
Then he wants you to go beyond. (For a majority of his audience, Steves
says, ``Europe is the wading pool for world exploration.'') Perhaps,
like him, you will need large headphones and half a tab of Ambien to
properly relax on the flight, but Steves wants you to know that it will
be worth it. He wants you to stand and make little moaning sounds on a
cobblestone street the first time you taste authentic Italian gelato ---
flavors so pure they seem like the primordial essence of peach or melon
or pistachio or rice distilled into molecules and stirred directly into
your own molecules. He wants you to hike on a dirt path along a cliff
over the almost-too-blue Mediterranean, with villages and vineyards
spilling down the rugged mountains above you. He wants you to arrive at
the Parthenon at dusk, just before it closes, when all the tour groups
are loading back onto their cruise ships, so that you have the whole
place to yourself and can stand there feeling like a private witness to
the birth, and then the ruination, of Western civilization.

Steves wants you to go to Europe for as long as you can afford to, and
he also wants to help you afford it. (Much of his guru energy is focused
on cutting costs.) He wants you to go as many times as possible, and
while you're there, he wants you to get way down deep into the culture,
to eat with locals in the teeming markets, to make a sympathetic fool of
yourself, to get entirely lost in your lack of America.

Out of this paradoxical desire --- the enlightenment of Americans
through their extraction from America --- Steves has built his quirky
travel empire. His guidebooks, which started as hand-typed and
photocopied information packets for his scraggly 1970s tour groups, now
dominate the American market; their distinctive blue-and-yellow spines
brighten the travel sections of bookstores everywhere. Steves is less
interested in reaching sophisticated travelers than he is in converting
the uninitiated. (``There will be more rejoicing in heaven over one
sinner who repents,'' the Bible tells us, ``than over 99 righteous
persons who do not need to repent.'') Last year, his company led close
to 30,000 paying customers on dozens of elaborate European itineraries.
Steves teaches his followers everything from how to pack a toiletries
kit to how to make themselves at home in a small hotel room to how to
appreciate a religious tradition they may have been raised to despise.
(In order to enjoy St. Peter's Basilica, Steves admits, he had to learn
to ``park my Protestant sword at the door.'') He is a sort of spiritual
travel agent for America's curious but hesitant middle classes. He is
simultaneously goofy and dead serious; he can ping, in an instant, from
golly-gee Pollyanna cheerfulness to deep critiques of the modern world.
In a series of long, affectionate, candid conversations, Steves's
colleagues described him to me using the words ``sophomoric,''
``knucklehead'' and ``Santa Claus'' --- but also ``juggernaut,''
``evangelical'' and ``revolutionary.'' Rick Steves wants us to travel
because it's fun, yes, but also because he believes it might actually
save the world.

\textbf{I can testify,} firsthand, to the power of Rick Steves. In 1998,
he spoke at my college. Nothing about the encounter seemed promising.
Our campus was a tiny outpost in a tiny town, and Steves delivered his
talk not in some grand lecture hall but in a drab room in the basement
of the student union. I was poor, shy, anxious, sheltered, repressed and
extremely pale. I was a particular kind of Pacific Northwest white guy
--- blind to myself and my place in the world. I had never really
traveled; I was more comfortable on Greyhound buses than on airplanes.
Going to Europe seemed like something aristocrats did, like fox hunting
or debutante balls.

My girlfriend dragged me to the talk. I had never even heard of Steves.
He entered looking like the kind of guy who would bring an acoustic
guitar to every single church picnic within a two-hour radius of his
favorite Applebee's: large glasses, floppy hair, bluejeans, wholesome
grin. But what he said over the next hour or so changed the rest of my
life.

It's hard to describe how thoroughly energized Steves becomes in front
of a crowd. He paces, gesticulates and speaks very fast. He tells his
favorite old jokes as if they were eternally new. (``Eet smells like zee
feet of angels,'' the French cheesemonger always exclaims.) Onstage, he
is a combination of preacher, comedian, salesman, life-hacker, professor
and inspirational speaker. Steves told us, that day, how to pack our
entire lives into a single bag measuring 9 by 22 by 14 inches. (``It's
enlightened to pack light,'' Steves insists. ``It's a blessing to pack
light.'') He told us how to find excellent cheap hotels, how to survive
on minimalist picnics in public parks, how to wash clothing in bathroom
sinks and how to make friends without sharing a language. Steves's
signature book, ``Europe Through the Back Door,'' seemed less like a
travel philosophy than a whole mode of being: scrappy, prepared,
independent, extroverted. Europe's front door, he told us, was
positioned to feed travelers directly into exploitation: overpriced
cafes, trinket shops, long lines, corporate high-rise hotels. The back
door, by contrast, led to revelations. He showed us impossibly enticing
photos: cobblestone piazzas teeming with fruit stalls, quirky wooden
hotels among wildflowers in the Alps, vast arsenals of multicolored
cheese. He made travel seem less like a luxury than a necessary
exploration of the self, a civic responsibility, a basic courtesy to
your fellow humans. It seemed almost unreasonable \emph{not} to go.
Above all, Steves told us, do not be afraid. The people of the world are
wonderful, and the planet we share is spectacular. But the only way to
really understand that is to go and see it for yourself. So go.

My girlfriend and I left the room converts to the gospel of Rick Steves.
We bought his book and highlighted it to near-meaninglessness. We
started mapping itineraries, squirreling away money, asking relatives
for donations. (In probably the worst phone call of my life, my rancher
grandfather expressed shock and dismay that I would ask him to support
this meaningless overseas lark.) Eventually, over many months, we
scraped together just enough to buy plane tickets and order minimalist
Steves-approved supplies, including a travel towel so thin and
nonabsorbent that it seemed to just push the moisture around your skin
until you forgot you were wet. We packed exactly as Steves taught us:
T-shirts rolled into space-saving noodles, just enough clothes to get us
from one hotel laundry session to the next. Then, for the first time in
our lives, we left North America.

One of Steves's strongest recommendations is to keep a journal. Mine was
so corny that its cover actually said ``A TRAVELER'S NOTEBOOK'' over a
picture of the Eiffel Tower. When I opened it recently, the reality of
that long-ago trip hissed out with fresh urgency. My 20-year-old self
recorded everything. On our first day in Europe, we bought imported
Austrian apples with fat, heavy English coins and saw a woman stumble on
a staircase, breaking an entire bag of newly bought china. We arrived at
our first hostel, the Y.M.C.A. in Bath, to find a man urinating in the
stairwell --- so we kept walking until we happened into a nearby
churchyard, where the gravestones were so old and thin they were almost
translucent. As we tried to make out the names of the dead, songbirds
sang strenuously in the trees all around us. This juxtaposition --- old
death, new life --- blew my jet-lagged American mind. ``Already, after
just one day in Bath,'' I wrote in my journal, ``the world has grown
firmer. Reality fills its gaps.''

That, more or less, was the theme of the trip. For six weeks, we
followed the Steves game plan. We shared squalid bunks with other young
travelers from Denmark, Australia, Canada and Japan. In the stately
public parks of Paris, we ate rotisserie chickens with our bare hands.
One stifling afternoon at the Colosseum in Rome, we watched a worker
slam his ladder against the edge of an arch and break off some ancient
bricks. (He looked over at us, looked down at the bricks, kicked dirt
over them and kept working.) We were moved by Van Gogh, Picasso and
Gaudí, but unmoved by Versailles (``more vain than beautiful,'' I
wrote), bullfighting (``more brutal than artful'') and Goya (``vague and
blurry''). Once, I left my underwear on a Mediterranean beach overnight
and, since I could not afford to lose a pair, had to go back and pick it
up the next day, in full view of all the sunbathers.

Wherever we went, Rick Steves was with us. In my journal, I referred to
him half-jokingly as our ``worldly uncle and guiding light,'' and as we
walked around, I annoyed my girlfriend by doing impressions of him. We
seemed to have entered the world of his slides: the fruit markets and
overnight trains, the sunny French river under the ancient Roman
aqueduct. Sometimes our European hosts, with the quiet pride of someone
who once met Elvis, told us stories about Steves. He was a gentleman,
they said, a truly good man, and he always came in person to check out
their hotels, and he never failed to ask them how their children were
doing.

By the end of our trip, we were completely broke. We couldn't afford
even a baguette on our last day in Paris. We flew home looking ragged,
shaggy, weather-beaten and exhausted.

\hypertarget{-5}{%
\subsection{}\label{-5}}

But of course Steves was right: Our lives were never the same. We were
still young Americans, but we felt liberated and empowered, like true
citizens of the world. The most important things we learned all had to
do with home. As the English writer G.K. Chesterton once put it, in a
quote I found printed in my corny old travel journal: ``The whole object
of travel is not to set foot on foreign land; it is at last to set foot
on one's own country as a foreign land.'' After looking at a Roman stone
wall topped by a Saxon stone wall topped by a medieval English wall next
to a modern paved street, I began to see what a thin crust of national
history the United States actually stands on. I began to realize how
silly and narrow our notion of exceptionalism is --- this impulse to
consider ourselves somehow immune to the forces that shape the rest of
the world. The environment I grew up in, with its malls and freeways,
its fantasies of heroic individualism, began to seem unnatural. I
started to sense how much reality exists elsewhere in the world --- not
just in a theoretical sense, in books and movies, but with the full
urgent weight of the real. And not just in Europe but on every other
continent, all the time, forever. I began to realize how much I still
had to learn before I could pretend to understand anything. Not everyone
needs Steves's help to get to this point. Some people get there
themselves, or their communities help them. But I needed him, and I am
eternally glad I was dragged that day to see him talk.

\href{https://www.nytimes3xbfgragh.onion/2019/02/27/travel/venice-cicchetti-small-plates.html}{{[}How
to eat in Venice like a Venetian.{]}}

\textbf{Steves answered} his front door slightly distracted. I had come
in the middle of his breakfast preparations. He was stirring a block of
frozen orange juice into a pitcher of water. ``Freshly squeezed from the
can!'' he quipped. This was April 2018, exactly 20 years after my first
trip to Europe. I had come to see Steves in the most exotic place
possible: his home. He lives just north of Seattle, in a town so rainy
it has a free umbrella-share program. There is nothing particularly
exotic about the house itself. It has beige carpeting, professionally
trimmed shrubs and a back deck with a hot tub. What was exotic was
simply that Steves was there. He had just returned from his frenetic
speaking tour of the United States and would be leaving almost
immediately on his annual trip to Europe. For now, he was making
breakfast: frozen blueberries, Kashi cereal, O.J. ``I would eat this
every day for the rest of my life if I could,'' he said.

But of course, he could not. Steves is gone too much, yo-yoing between
the misty forests of the Pacific Northwest and the sun-baked cathedrals
of Europe. Every year, no matter what else is going on, Steves spends at
least four months practicing the kind of travel he has preached for
40-odd years: hauling his backpack up narrow staircases in cheap hotels,
washing his clothes in sinks, improvising picnics.

He is now 63, and he could afford to retire many times over. But he
doesn't have the metabolism for sitting around. Among his colleagues,
Steves is a notorious workaholic. After grueling days of filming in
Europe, he has been known to slip script revisions under the crew's
doors at 2 a.m., and then to ask them, at breakfast, for their feedback.
On long car rides, he sits in the back seat and types op-eds on his
laptop. His relentless hands-on control of every aspect of his business
is what has distinguished the Rick Steves brand.

It is also, obviously, exhausting --- if not for Steves, then at least
for the people around him. He has two children, now grown, and for much
of their childhoods, Steves was gone. He was building his company,
changing the world. For very long stretches, his wife was forced to be a
single mother. (She and Steves divorced in 2010 after 25 years of
marriage.) Every summer, when the family joined Steves in Europe, his
pace hardly slackened: They would cover major cities in 48 hours,
blitzing through huge museums back to back. The kids complained so much,
on one trip, that Steves finally snapped --- if they were so miserable,
he said, they could just go sit in the hotel room all day and play video
games. They remember this day as heaven. One year, while Steves was
away, the children converted to Catholicism. His son, Andy Steves,
eventually went into the family business: He now works as a tour guide
and even published a European guidebook.

Steves is fully aware that his obsessive work ethic is unusual. He
admits that he has regrets. But he cannot make himself stop. He has the
fervor of the true evangelist: The more people he meets, the more cities
he visits, the more lives he might change. At one point, as we talked,
he pulled out the itinerary for his coming trip --- from Sicily to
Iceland, with no down time whatsoever. Just looking at it made him
giddy. I asked why he couldn't ease up slightly --- maybe just spend two
months in Europe, maybe just speak in 10 American cities.

``It's a strange thing,'' he said. ``I get energy from it. It's like I'm
breathing straight oxygen. What would I do if I stayed home? Not much.
Nothing I would remember.''

In his house, Steves offered up a little show and tell. He pointed out
an antique silver cigarette lighter shaped like the Space Needle. He sat
down at his baby grand piano and lost himself, for a few happy minutes,
playing Scarlatti. He took me to a room filled with books and reached up
to a very high shelf. ``I don't show this to too many people,'' he said,
``because they'll think I'm nuts.'' Steves pulled down a thick red
binder, the contents of which were, indeed, pretty nutty. When Steves
was 13, he decided, for no apparent reason, to conduct a deep
statistical analysis of the 1968 Billboard pop charts. Every week, he
would clip the rankings out of his local newspaper and, using a point
system of his own devising, graph the top bands' success on sheets of
gridded paper. The lines were multicolored and interwoven --- it looked
like the subway map of some fantastical foreign city. You could see, at
a glance, the rising and falling fortunes of the Beatles (red) and
Creedence Clearwater Revival (black) and Elvis Presley (dots and
dashes). Steves kept this up for three years, taping together many
pieces of graph paper, and in the end he summarized the data in an
authoritative-looking table that he typed on the family typewriter. This
is what was in that binder: a systematic breakdown of the most
successful bands from 1968 to 1970, as determined by the objective
statistics of an analytical adolescent weirdo. (The winners, of course,
were the Beatles --- 1,739 points --- followed by Creedence, Simon and
Garfunkel, Neil Diamond.)

Steves laughed. It was ridiculous. But it was also a perfect window into
his mind. Even at 13, a powerful energy was coiled inside him --- an
unusual combination of obsession and precision, just waiting for some
worthwhile project to burst out in.

And that, coincidentally, was exactly when he found it: the project of
his life. In the summer of 1969, when Steves was 14, his parents took
him to Europe. They owned a business tuning and importing pianos, and
they wanted to see factories firsthand. Steves approached this first
trip abroad with the same meticulous energy he brought to his Billboard
graphs. As he traveled around the continent, he recorded the essential
data of his journey on the backs of postcards: locations, activities,
weather, expenses. One day, Steves spent 40 cents on fishing gear.
Another, he met a 79-year-old man who had witnessed the assassination of
Archduke Franz Ferdinand. To keep everything in order, Steves numbered
the postcards sequentially. He still has them all packed lovingly into
an old wooden box.

On that same formative trip, the Steves family visited relatives in
Norway. They happened to be there in July 1969, when Neil Armstrong
walked on the moon. ``\emph{Ett lite skritt for et menneske},'' the
television said, ``\emph{ett stort sprang for menneskeheten}.'' In that
moment, in that strange place, young Rick Steves felt the concept of
``\emph{menneskeheten}'' --- ``mankind'' --- at a depth he never would
have been able to access back home. Europe was a crash course in
cultural relativity. In a park in Oslo, he had an epiphany: The foreign
humans around him, he realized, were leading existences every bit as
rich and full as his own. ``Right there,'' he would write later, ``my
14-year-old egocentric worldview took a huge hit.'' A life-changing
realization clicked into place. ``This planet must be home to billions
of equally lovable children of God.''

That first trip set the course for everything that followed. When Steves
was 18, he went back to Europe without his parents. Soon, life in
America became a series of interludes between travel. He taught piano to
earn money, then stretched that money as far as he possibly could,
sleeping on church pews and park benches, in empty barns and
construction zones, from Western Europe to Afghanistan. He turned his
cheapness into a science. Instead of paying for a hotel room in a city,
Steves would use his Railpass and sleep on a train for the night ---
four hours out, four hours back. He would stuff himself on free
breakfast bread, then try to eat as little as possible for the rest of
the day. Naturally, he recorded all this, and today he has an impressive
archive of old travel journals. Their pages preserve, in tiny
handwriting, shadowy young dissidents in Moscow, diarrhea in Bulgaria,
revolution in Nicaragua.

In his 20s, Steves brought his wide-roaming wisdom back to the United
States. He started to supplement his piano teaching with travel
seminars. His signature class, European Travel Cheap, ran for six hours.
Steves could have talked longer than that, but it struck him as
impractical for his students. In Europe, he rented a nine-seat minibus
and started to lead small tours. Eventually, his seminars and tour notes
morphed into his books. The first edition of ``Europe Through the Back
Door,'' published in 1980, was typed on a rented IBM Selectric. It had
no ISBN and looked so amateurish that bookstores assumed it was an early
review copy. ``Anyone caught reprinting any material herein for any
purpose whatsoever will be thanked profusely,'' it said. This was the
birth of the Rick Steves empire.

\textbf{Rick Steves both} is and is not his TV persona. Offscreen, he
allows himself to be much more explicitly political. He has the passion
of the autodidact. Growing up, Steves led a relatively sheltered
existence: He was a white, comfortable, middle-class baby boomer in a
white, comfortable, middle-class pocket of America. Travel did for him
what he promises it will do for everyone else: It put him in contact
with other realities. He saw desperate poverty in Iran and became
obsessed with economic injustice. He started searching for answers in
books, scribbling notes in the margins of ``Bread for the World,'' by
Arthur Simon, and ``The Origins of Totalitarianism,'' by Hannah Arendt.
He studied the war industry and colonial exploitation. The first time
Steves traveled to Central America, he came back so outraged that he
wrote a fiery tract called ``There's Blood on Your Banana,'' then flew
to Washington and hand-delivered a copy to the office of every member of
Congress.

In the early days, Steves injected political lessons into his European
tours. Sometimes he would arrive in a city with no hotel reservations,
just to make his privileged customers feel the anxiety of homelessness.
In Munich, he would set up camp in an infamous hippie circus tent, among
all the countercultural wanderers of Europe.

Today, Steves is more strategic. His most powerful tool, he realizes, is
his broad appeal. He has an uncanny knack for making serious criticism
feel gentle and friendly. Often he disguises critiques of America with a
rhetorical move that I like to think of as ``U.S.A.! U.S.A.! U.S.A.!
(But. ... )'' ``I'm unapologetically proud to be an American,'' he
writes in the introduction to his book ``Travel as a Political Act.''
``The happiest day of any trip is the day I come home. ... But other
nations have some pretty good ideas too.''

That's when he hits his audience with legal prostitution, high tax rates
and universal health care.

When I asked Steves about this strategy, he chuckled.

``It's not America-bashing,'' he said. ``It's America-loving. I think
it's loving America to look at it critically. But you've got to set it
up. You've got to allay people's concerns that you're a communist. So
you explain to them: I'm a capitalist, I make a lot of money, I employ a
lot of people, I love the laws of supply and demand. It seems kind of
silly, but you've got to say that. Then, especially the husbands who are
dragged there by their wives, they go, `I thought he was a commie, but
he's O.K.' And then you don't need to be too gentle. You can confront
people with a different perspective, and you'll get through.''

Steves learned this strategy, he said, from his early days running
tours, living with the same people for weeks at a time. Survival
required being pleasant. People didn't want grating lectures about
America's shortcomings --- even if that was sometimes his instinct.
Instead, he pointed out different perspectives with a smile. He became
fluent in the needs of American tourists. ``I know what their buttons
are,'' he said. ``I know what their attention span is. I don't want to
just preach to the choir. I want to preach to organizations that need to
hear this, so I need to compromise a little bit so the gatekeepers let
it through to their world.''

This balancing act has become increasingly difficult over the past two
decades, in a world of terrorism, war, nationalism and metastasizing
partisanship. After the Sept. 11 attacks, most travel companies
anticipated that the bottom was about to fall out of the market. They
canceled tours and cut back budgets. Steves, however, remained defiantly
optimistic. He promised his staff that there would be no cuts, no
layoffs and no shift in message. He insisted that a world in crisis
needed travel more, not less. Soon the shock of Sept. 11 turned into the
Iraq war, which strained the relationship between the United States and
even its closest European allies, sending the travel industry deeper
into its trough. In his hometown, Steves caused a controversy when he
walked around removing rows of American flags that had been set up in
support of the war. It was, he argued, an act of patriotism: The flag is
meant to represent all Americans, not just war supporters. ``I was
shark-bait on Seattle's right-wing radio talk shows for several days,''
he wrote.

Lately, Steves concedes, his political message has begun to take over
his teaching. In ``Travel as a Political Act,'' the familiar elements of
his guidebooks --- walking tours, museum guides, hotel reviews --- are
replaced by rabble-rousing cultural critique. Steves expresses deep
admiration for Scandinavian-style social democracy and calls out many of
America's faults: our addiction to cars and guns and mass incarceration;
our deference to corporations; our long history of cultural imperialism
(``one of the ugliest things one nation can do is write another nation's
textbooks''). Some moments in the book verge on un-American.
``Sometimes, when I'm frustrated with the impact of American foreign
policy on the developing world,'' Steves writes, ``I have this feeling
that an impotent America is better for the world than an America whose
power isn't always used for good.''

Occasionally, despite his best efforts, Steves still ruffles feathers.
Recent TV specials have covered Iran --- ``I believe if you're going to
bomb a place,'' Steves has written, ``you should know its people first''
--- and the rise of fascism in Europe. In a special about the Holy Land,
Steves refers unapologetically to ``Palestine'' instead of ``the West
Bank'' or ``Palestinian territories''; some viewers were so outraged
that they told Steves they were removing PBS from their wills. After one
recent speech in the Deep South, event organizers refused to pay Steves
--- their conservative sponsors, he learned, considered his message a
form of liberal propaganda.

In recent years, Steves has become a happy warrior for an unlikely
cause: \href{https://www.youtube.com/watch?v=tvCxGQCeWc4}{the
legalization of marijuana.} He first tried the drug in Afghanistan, in
the 1970s, in the name of cultural immersion, and he was fascinated by
its effect on his mind. Today, he is a board member of Norml, the
National Organization for the Reform of Marijuana Laws, and a regular
speaker at Hempfest. In his headquarters you will find a poster of the
Mona Lisa holding a gargantuan spliff. In 2012, Steves campaigned hard
for Washington State's successful legalization initiative, and since
then he has barnstormed other states (Oregon, Maine, Vermont and more)
to make sure the civil liberties are properly passed around. On a shelf
in his living room, right there among all the European knickknacks,
Steves displays a sizable bong.

Sometimes, fans urge Steves to run for office. When I asked him if he
would ever get into politics, he had an answer ready: ``I already am.''
Good travel teaching, in his eyes, is inherently political. To stay in a
family-owned hotel in Bulgaria is to strengthen global democracy; to
pack light is to break the iron logic of consumerism; to ride a train
across Europe is to challenge the fossil-fuel industry. Travel, to
Steves, is not some frivolous luxury --- it is an engine for improving
humankind, for connecting people and removing their prejudices, for
knocking distant cultures together to make unlikely sparks of joy and
insight. Given that millions of people have encountered the work of
Steves over the last 40 years, on TV or online or in his guidebooks, and
that they have carried those lessons to untold other millions of people,
it is fair to say that his life's work has had a real effect on the
collective life of our planet. When people tell Steves to stay out of
politics, to stick to travel, he can only laugh.

``It's flattering to think I could run for office,'' he admitted. ``And
it would be exciting. But I think I'm accomplishing more right now than
I would in office, and I'm having more fun. I'm skiing with beautiful
wax on my skis. When I want to do something, I can do it.''

Steves is deeply indifferent to creature comforts. When I visited him,
the back seat of his car was covered with a greenish slime, practically
disintegrating, because of a mysterious leak. He just cracked the
windows to try to dry it out. Steves prefers to spend his money on his
favorite causes. His activism can be quirky and impulsive. In 2011,
after hearing that his local symphony orchestra was struggling, he
stepped in with a gift of \$1 million, spread over 10 years, to help
keep it operating. (This, pointedly, was how much money he would get
back from President George W. Bush's tax cut over a decade.) Last year,
during a chat with one of the national leaders of the Lutheran Church,
Steves wondered how much it would cost to send every single Lutheran
congregation in the United States a DVD of his recent TV special about
Martin Luther. It was something like \$30,000; Steves happily wrote the
check. In the 1990s, working in partnership with the Y.W.C.A., he
started investing his retirement savings in local real estate in order
to house homeless mothers and their children. The plan was to take that
money out of the banking system and let it do a few decades of social
good, at which point Steves could sell the buildings to fund his
retirement. Eventually he worked his way up to buying a whole 24-unit
apartment complex --- and then he donated it outright to the Y.W.C.A.
The mothers, he said, needed it more than he would.

Steves is obsessed with the problem of poverty and amazed at our
perpetual misunderstanding of it. ``It's not just: You screwed up, so
you're poor,'' he said. ``There's a \emph{structure} that keeps half of
humanity poor. This needs to be talked about. I can do it, and I can get
away with it.'' His next TV special, in production now, will investigate
extreme poverty and hunger through two very different non-European
countries: Guatemala and Ethiopia. In the meantime, all the royalties of
his latest book --- an updated edition of ``Travel as a Political Act''
--- are being donated to Bread for the World, an organization that
lobbies on behalf of hungry people. He is working on making his
company's tours completely carbon-neutral.

``If I was trying to build a career on the speaking circuit --- if I was
struggling, and I needed these gigs --- I would not talk about that
stuff,'' he said. ``I could just talk about light stuff, and everybody
would love it. But I'm not working right now to do that. I'm not trying
to get anywhere that I'm not already. I don't need to be anything I'm
not. I'm 63 years old. I could retire now. But I'm ramping up.''

Indeed, Steves's business has been booming. Once the travel market
finally recovered, some years after Sept. 11, Steves occupied a
disproportionately big share of it --- precisely because he had refused
to scale back. By taking a principled stand, Steves flourished. Today,
his chipper voice is reaching more Americans than ever. ``Fear,'' as
Steves likes to say, ``is for people who don't get out very much.''

\href{https://www.nytimes3xbfgragh.onion/2019/03/13/travel/women-activism-seoul-korea.html}{{[}4
stops for a woman-focused trip to Seoul.{]}}

\textbf{One night,} in his living room, Steves pulled out a plain black
notebook. ``Here's something you might find interesting,'' he said with
his trademark cheer, and he flipped open to a random page and prepared
to read aloud. I was familiar, by then, with Steves's deep archive of
old travel journals, and so I settled in to listen to further adventures
from 1975 Moscow or 1997 Paris. This, however, was something else
entirely --- a record of a very different kind of journey.

``Getting high,'' Steves read, ``releases the human in me.''

``Intelligence is a rubber band,'' he continued. ``Getting high is
stretching it.''

I was sitting in the beige living room of America's foremost travel
guru, underneath framed reproductions of popular European masterworks,
and my mind was about to be well and truly blown. For the next 20
minutes, Steves would read me koans about the glories of being stoned.

``High is the present,'' he read.

``When you're high, you debate long and hard over whether to put on your
sweater or turn up the heat.''

This journal, Steves explained, contained what he called his ``High
Notes.'' For nearly 40 years, he had been writing in it exclusively
after smoking marijuana. He would get baked, open up to somewhere in the
middle and jot down whatever he happened to be thinking --- deep or
shallow, silly or angry. There is no chronology; on every page, axioms
from many different decades commingle. It is a lifelong treasury of
Steves's stoner thoughts.

He continued to read.

``As soon as I stop mattering so much, I'll be happier.''

``A baby doesn't know if the hanging is on the wall or if the wall is on
the hanging.''

``Make a rug with vacuum marks, so it always looks freshly vacuumed.''

The entries covered an impressively wide territory. Some were little
shreds of oracular poetry (``We all have a divine harness''), while
others were dashed-off semi-witticisms (``Wolfgang von Bewildered'') or
bitter social critiques (``The spiritual cesspool of America --- our
shopping malls''). ``They don't let you into heaven without calluses on
your soul,'' he read at one point. ``Suffer or weep.'' There were scraps
of humorous dialogue (`` `Nothing is wrong with an ego!' he bellowed'')
and sentences that would have made great bumper stickers on rusting VW
buses: ``I'd like to be quarantined from reality.''

I found myself wondering, for the thousandth time: Who \emph{does} this?
What kind of mind not only thinks of such a project but actually follows
through with it, decade after decade after decade? Who, for God's sake,
is this disciplined when they're high?

As Steves read, he interrupted himself again and again with great
shouting honks of laughter, and I cackled right along with him. Then,
suddenly, with almost no transition, we would find ourselves deep in
earnest conversation about the nature of true happiness or the dangers
of ambition. And then we would suddenly be cackling again. We were, in
other words, getting high on Steves's ``High Notes.''

He kept reading.

``I've been craning my mind to see you,'' he said.

``I feel like a hungry bird, but I won't eat any worm I don't like.''

Steves showed me complex analytical graphs about true love and divorce
rates, about the way music sounds when you're high versus sober, about
the degrees of honesty possible with the various people in your life.
(``Scale of Unconditional Regard,'' this last one was called.) One page
of the journal had a strand of hair taped to it, labeled ``split end.''
There was a drawing of a woman's breasts. And of course there were many,
many more descriptions of getting high itself. ``Getting high is like
roasting an English muffin,'' Steves read. ``You start out cold and
doughy, and you toast it to a crisp brown, and just a little more and
you get all black and burned.''

At some point, he looked up from the journal. ``To me,'' he said, ``this
is a precious thing. Because this is \emph{me}.''

He kept reading.

``Time spent socially is time spent at the expense of personal
betterment,'' he said.

He shook his head. That's how he had thought when he was young. ``That's
my problem,'' he added ruefully. ``I work all the time.''

Then he kept reading.

``When I die,'' he read, ``scatter me all over the budget hotels of
Europe.''

\emph{\textbf{Sam Anderson} is a staff writer at the magazine and the
author of ``Boom Town,'' a book about Oklahoma City. In 2017, he won a
National Magazine Award
\href{https://www.nytimes3xbfgragh.onion/2016/08/21/magazine/davids-ankles-how-imperfections-could-bring-down-the-worlds-most-perfect-statue.html}{for
his article about Michelangelo's ``David.''} \textbf{Zachary Scott} is a
photographer known for his humorous and highly stylized work, which has
been featured in the magazine's Year in Ideas and Comedy Issues.}

\hypertarget{correction-march-22-2019}{%
\subparagraph{\texorpdfstring{\textbf{Correction} March 22,
2019}{Correction March 22, 2019}}\label{correction-march-22-2019}}

An earlier version of this article misstated the size of a bus Steves
used in his early tours through Europe. It was a nine-seat minibus, not
a nine-foot minibus.

The Voyages Issue

\hypertarget{-6}{%
\section{}\label{-6}}

The Gift at the

Edge of the World

By BRIAN REA

\textbf{When my wife} and I were married, my mother-in-law told us she
had a special gift for us. I wasn't sure what to expect. She didn't tell
us what the gift was, but she did tell us where it was. In Sweden, on an
island, in the forest.

\textbf{As with all} magical places, getting to the island in Sweden
requires some effort (particularly as my wife, son and I live in Los
Angeles). After the plane, the train and a car ride to the countryside,
a boat ferries us across the lake from the mainland. There are only a
handful of cottages --- with no electricity or running water --- on the
island. It hasn't been developed (and, I hope, never will be). The
forest service long ago gave over the island's forest management to
nature, so the place is thick with trees: gangly evergreens and white
birch, as well as wildflowers, blueberry plants and many others. The
island is only about 50 acres, but it's quite easy to get lost.
Distances walking in the forest are hard to determine. You spend so much
time walking over, under and around branches, brush and fallen trees
that a simple hike can quickly become a disorienting journey. There are
no straight lines in a forest.

\textbf{In Sweden,} mushrooms are like gold. Specifically chanterelle
mushrooms. Aside from their high cost and their subtle earthy flavor
(cooked in butter and served on toast), their value is enhanced by how
late in the season they grow. So Swedes are extremely protective of
their chanterelle patches. And though the custom of
\emph{allemansratten} (``freedom to roam'') allows everyone access to
walk and forage on all lands, when a patch is discovered deep in the
forest, people sometimes record the GPS coordinates and pass them down
like a closely guarded secret.

\textbf{When my wife} and I were married, my mother-in-law told us she
had a special gift for us. I wasn't sure what to expect. She didn't tell
us what the gift was, but she did tell us where it was. In Sweden, on an
island, in the forest.

\textbf{As with all} magical places, getting to the island in Sweden
requires some effort (particularly as my wife, son and I live in Los
Angeles). After the plane, the train and a car ride to the countryside,
a boat ferries us across the lake from the mainland. There are only a
handful of cottages --- with no electricity or running water --- on the
island. It hasn't been developed (and, I hope, never will be). The
forest service long ago gave over the island's forest management to
nature, so the place is thick with trees: gangly evergreens and white
birch, as well as wildflowers, blueberry plants and many others. The
island is only about 50 acres, but it's quite easy to get lost.
Distances walking in the forest are hard to determine. You spend so much
time walking over, under and around branches, brush and fallen trees
that a simple hike can quickly become a disorienting journey. There are
no straight lines in a forest.

\textbf{In Sweden,} mushrooms are like gold. Specifically chanterelle
mushrooms. Aside from their high cost and their subtle earthy flavor
(cooked in butter and served on toast), their value is enhanced by how
late in the season they grow. So Swedes are extremely protective of
their chanterelle patches. And though the custom of
\emph{allemansratten} (``freedom to roam'') allows everyone access to
walk and forage on all lands, when a patch is discovered deep in the
forest, people sometimes record the GPS coordinates and pass them down
like a closely guarded secret.

\textbf{The day my} mother-in-law took us for our first walk, everything
seemed slow and quiet (besides the buzz of the mosquitoes). I listened
to her tell stories of playing here as a child; exploring it made me
feel young, and nostalgic for a past I had never lived. I marched behind
my wife and was careful when stepping over fallen trees or catching
branches she bent back to allow me to pass. I noticed too that colors
were brighter --- particularly the striking golden-yellow chanterelles,
whose unusual billowing shapes sometimes reminded me of linens blowing
on a clothesline or tiny versions of Marilyn Monroe's dress in ``The
Seven Year Itch.'' Once I reoriented myself to searching like this, it
became sport to find them. Some mushrooms you can eat, and some can make
you very sick. Animals know this, and people who spend lots of time in
the forest know this. My mother-in-law knows.

I wasn't sure how far we had walked, and I would not even have seen the
path if it weren't for my mother-in-law's pointing it out. She took us
to a clearing among some trees, looked around a bit, then stopped and
bent down. She carefully pulled back some leaves and brush and said,
``This is for you.'' It was a small patch of chanterelle mushrooms: a
cluster of about five or six undulating golden nuggets jutting out of
the dark, wet ground. She said she had given each of her children a
patch in the forest where she found that mushrooms consistently grew
each year. ``Some years the animals get to them first,'' she said, ``but
we're lucky this year.''

\textbf{It's winter now,} and we are back in Los Angeles. But I find
myself thinking a lot about my mother-in-law's \emph{gava,} her gift ---
our tiny plot of mushrooms. The quiet of the forest, maybe it's snowing,
maybe it's dark. It's strangely comforting to think of something so
small and delicate and so far away --- our small space alone on its
island where we will return year after year.

\emph{Brian Rea is an artist in Los Angeles, where he has an exhibit on
display at the CMay Gallery. His book ``Death Wins a Goldfish'' was
published earlier this year.}

\emph{Illustrations by Brian Rea. Animations by Pablo Delcan.}

The Voyages Issue

\hypertarget{-7}{%
\section{}\label{-7}}

We're All

in This

Together

 The particular sheen of America by Amtrak.

By CAITY WEAVER

photographs by:

Holly Andres

\textbf{Tell your fellow} \textbf{americans} that you plan to cross the
United States by train, and their reactions will range from amusement at
your spellbinding eccentricity to naked horror that they, through some
fatal social miscalculation, have become acquainted with a person who
would plan to cross the United States by train. Depending how you slice
it --- time or money --- there are either 61 or 960 immediate reasons
not to travel by Amtrak trains from New York City to Los Angeles. Those
are the extra hours and dollars, respectively, that you might reasonably
expect to forfeit if you forgo a six-hour \$129 nonstop flight and opt
instead for an Amtrak sleeper car. Covering the interjacent 2,448.8
miles can easily consume some 67 hours for a mind-boggling \$1,089.

Of course, you might remind your quote-unquote fellows, any form of
modern engine-based transport, even Amtrak, is preposterously fast
compared with the method that Homo sapiens employed to move ourselves
and, more important, our tchotchkes for most of our species'
300,000-year history, which is walking. Crossing the stretch of land
where roughly half the Donner party starved, froze or, in the case of
the group's two Miwok guides, were shot to death for food --- an
overland journey that took the party about five months to complete in
1847 --- could be done in under two hours by a Honda Accord today,
assuming normal traffic, while a plane from Springfield, Ill., their
starting point, to Sacramento would zoom over their whole route in half
a day, including layover. Because of this ability to effectively
teleport between locations, 21st-century Americans have become flippant
about transcontinental voyaging. To truly appreciate the size of the
landmass (the third-largest country in the world by land area) and the
variety of its terrain (rain forests, deserts, prairies, Margaritaville,
etc.), you have to see it from the ground.

Amtrak clings to the hope that someday people will view its service not
as something that sucks and that they hate, but as something that is
actually nice and that they don't hate. There's a whole separate Amtrak
website dedicated to this dream (AmtrakVacations.com), where Amtrak does
things like describe Los Angeles to people who have never heard of it.
\href{https://www.amtrakvacations.com/destination/los-angeles-ca/}{``The
`City of Angels' is one of the premier attractions in sunny Southern
California.''} But the other selling point of a cross-country train trip
is a chance to look behind the American scrim: to learn where the nation
makes and stores the hidden parts that run it, to find new places you
wish you had been born, to spy on backyards and high school football
fields whose possible existence had never occurred to you. Or me. Why
not me? My boyfriend and I were planning a short vacation out West
anyway. I could just leave a few days before him and get there after he
arrived.

As I quickly learned, there are no passenger rail routes that cross the
entire United States in a single trip, nor are there likely to be any
soon. Even proponents of the high-speed railway systems much lauded in
Asia and Europe (and tentatively proposed in Congress's Green New Deal
resolutions) generally give the competitive edge to planes for travel
across distances greater than 600 miles. At present, reaching California
by rail from New York requires at least two trains, one of which will
depart from New Orleans or Chicago, all of which, like most lines
operated by Amtrak, have names so sumptuously picturesque (Maple Leaf,
Coast Starlight, Sunset Limited) they make the storybook ``Polar
Express'' sound as sterile as ``Amtrak'' by comparison. To book tickets,
a person must first complete a battery of tests measuring her patience,
hand-eye coordination and aptitude for deductive mathematical reasoning,
in the guise of Amtrak's impossible-to-use online trip planner. (While
the trip planner cannot identify the train station nearest to an
address, or even a city, it can tell you the name of the city you have
already typed into its search bar, provided there is an Amtrak train
station there.) The fastest way to complete this slow journey is to take
the Lake Shore Limited to Chicago's Union Station, then board the
Southwest Chief to Los Angeles, one of sunny Southern California's
much-hyped premier attractions.

\hypertarget{for-your-safety-do-not-walk-or-play-on-the-tracks}{%
\paragraph{For your safety, do not walk or play on the
tracks}\label{for-your-safety-do-not-walk-or-play-on-the-tracks}}

Contrary to multiple acquaintances' declarations that I would encounter
``some real weirdos'' on the train, the first person I met on board my
first sleeper car after boarding the train in Penn Station was a man in
a sparkly cardigan and leather pants who breezily identified himself as
``a prophet,'' which is perhaps the world's second-oldest profession.
And forgive me if I find nothing ``weird'' about being gainfully
employed under a supervisor with the kind of multinational name
recognition God has.

As he doubtless expected, the prophet and I were in opposite Viewliner
roomettes --- private compartments Amtrak describes as ``designed for
one or two passengers,'' although a roomette is both narrower and
shorter than a standard porta potty. What Amtrak has managed to cram
into this minuscule space is impressive: a fold-down sink, two cushioned
benches that convert to a bed, a second premade bed that lowers from the
ceiling, a tiny foldout table with an inset of alternating colored
squares for checkers or chess, a coat hook, a luggage cubby, a large
picture window and the largest variety of not-quite-matching shades of
dark blue upholstery fabrics ever assembled. There is even a small metal
toilet covered with a puce-colored lid, which invites the brainteaser:
Is it more luxurious to have a private toilet inches away from your
sleeping area, or a shared toilet elsewhere?

The prophet sat silently in his compartment with the curtains open.

\href{https://www.nytimes3xbfgragh.onion/interactive/2015/09/23/magazine/the-voyages-issue.html\#Nigeria}{{[}See
a voyage by train through Nigeria.{]}}

The car's friendly attendant advised me that the recommended way to
enter the upper berth was to step first atop the toilet seat (a little
over a foot off the ground), then, using a wall-mounted handle for
balance, climb onto the narrow built-in ledge above the toilet, rotate
my body 90 degrees and, fueled by a cocktail of optimism and derring-do,
launch myself into the bed suspended in midair. To prevent occupants
from rolling off their 28-inch-wide mattresses (the same width as a
standard casket) and falling several feet to the floor, stowed beneath
the mattress of every upper bunk is a kind of net of seatbelts that
hooks with grim determination into the ceiling. Once on the bed, I
subjected my body to a series of Cirque du Soleil-inspired experiments
to confirm that this safety web would indeed hold my weight, were I to
roll unconsciously into it at 2 a.m. I tested the strength of the straps
with one leg. I rolled from the wall into the net, flopping my limbs. I
placed each hand on a segment of net and pushed against it with the full
force of my upper body, something that I had never done in my sleep but
that now seemed possible or even probable. It seemed secure.

It also seemed representative of Amtrak's casual, makeshift approach to
passengers --- a slightly refreshing, slightly unnerving attitude to
encounter after a lifetime of air travel. The freedom to move about in a
train evokes an illicit, almost danger-courting autonomy.
(\href{https://injuryfacts.nsc.org/all-injuries/preventable-death-overview/odds-of-dying/}{The
nonprofit National Safety Council reports} that a person in the United
States is several times more likely to die of ``sharp objects'' than a
plane or train crash, though the events that preceded the recent
emergency Boeing groundings make such statistics cold comfort.)

The instructions given by conductors and attendants were not so much
formulaic as they were desperately obvious --- a black comic litany of
bare-minimum survival tips. ``Just for your safety please do not walk or
play on these tracks,'' went one announcement. Another asked parents to
ensure young children did not ``wander around the train alone.''
Although there was no whiff of a T.S.A. screening in place (it would
presumably be possible for someone to arrive one minute before departure
carrying a duffel bag of uranium and swords and hop right on, although
hopefully no one will), pantomimes of security distributed
responsibility among everyone aboard. ``WE'RE ALL IN THIS TOGETHER ...
LITERALLY'' read the text on a safety brochure promoting Amtrak's ``If
You See Something, Say Something'' campaign.

Even on short plane trips, every passenger is offered the
kindergartner's communion of juice and cookies, as if a majority of
adults are incapable of going 90 minutes without such provisions. On
trains, passengers are treated as individuals even more powerful than
adults: independent teenagers who just want to smoke. Amtrak knows you
want to smoke. Amtrak knows you love to smoke. But while you're living
under Amtrak's roof, you have to follow the rules, of which there is
only one, and that is: Don't smoke inside.

``Albany is going to be a smoke break,'' a young male voice declared
over the intercom as the train hurtled northward. ``Just a reminder,
ladies and gentlemen,'' a voice like that of a female jazz radio D.J.
warned on a westbound train. ``This is a completely nonsmoking train.''
She added: ``Your first official stop for a smoke break is Kansas City,
Mo.''

In winter, the 3:40 Lake Shore Limited experiences just 90 minutes of
daylight before darkness descends for a majority of its journey west to
Chicago. The first leg of the trip follows the Hudson River, revealing
glimpses of hidden islands and idyllic ruins --- like the crumbling
remains of a fanciful 20th-century castle built by an arms dealer in
need of an out-of-the-way place to stash his stores of live ammunition,
some of which eventually exploded, creating the crumbling remains. At
sunset, when all that was left of the day was a tangelo slash along the
horizon, that same color flashed up from partly melted ice craters that
caught the light as the train chugged past. Suddenly, the air outside
the train became crows --- thousands of crows, rushing in from all
angles and alighting on the blue-white frozen river, as if deposited
there by an unseen hand.

Sleep the first night came easily and, as it was interrupted several
times, frequently. After performing the traditional nighttime rituals of
climbing atop the toilet and carefully catapulting into bed, I was
rewarded with the gentle rocking of a hammock experiencing a constant
minor earthquake tremor. The atmosphere on board was librarylike; even
the periodic train whistle sounded very far away, as if in someone
else's dream.

\href{https://www.nytimes3xbfgragh.onion/2013/03/03/magazine/47-hour-train-ride.html}{{[}How
to spend 47 hours on a train and not go crazy.{]}}

\hypertarget{among-the-train-people}{%
\paragraph{\texorpdfstring{\textbf{Among the train
people}}{Among the train people}}\label{among-the-train-people}}

The most unifying characteristic of my fellow passengers was not age
(although, as a rule, the sleeping cars skewed retired), race (very
mixed), income (while sleepers are astronomically priced, coach seats
can be downright economical for shorter segments) or even fear of flying
(no one I spoke to had it); it was their relaxed, easygoing,
train-lulled contentment. To opt to travel long distance via Amtrak ---
a method deemed ``on time'' just 71.2 percent of the time by its own
generous metric --- is to say: As long as I get there eventually, I'm
satisfied.

Train people are content to stare out the window for hours, like indoor
cats. The trouble with the Lake Shore Limited is that the amount of
enjoyment it is possible to derive from staring out the window of a
train is inversely proportional to the population density of the land
you are traversing. People need things, and unfortunately most of those
things are ugly to look at. Many of them are gray. Views picked up
considerably when, after a five-hour layover in Chicago, I transferred
onto the Southwest Chief, a double-decker ``Superliner'' with many of
its coach seats, sleeping quarters and lounges on the top level.
Sightseer Lounges are the crown jewels of Amtrak's long-distance trains:
entire cars of retro-futuristic curved floor-to-ceiling windows where
passengers can sit at tables or outward-facing upholstered chairs and
watch the scenery streak by. Shortly into its route, the Chief passes
the single best thing in the United States: a silo in Mendota, Ill.,
with an 80-by-20-foot ear of corn painted on one side.

Train people are also individuals for whom small talk is as invigorating
as a rail of cocaine. For them, every meal on board Amtrak (communal
seating like a Benihana, reservations only, included with the price of a
sleeping-car ticket, check in with the dining-car attendant) is a rager.
A white middle-aged man in motorcycle gear discussed leukemia treatment
with a swish black grandmother. Another man, while gathering up armfuls
of research books from a table, bid farewell to a farmer and suggested
that he might run into him on the same train next year. I was seated at
dinner with an Amish couple traveling to Arizona for a construction job,
and by the time our Amtrak Signature Steaks with optional Béarnaise
sauce arrived (the food is on a par with the fourth-best airplane meal
you could ever imagine), we were deep in a conversation about one of my
favorite topics, which is myself. I offered a tip I'd learned about
cleaning up glitter using dryer sheets, and they laughed as they tried
to envision a situation in which this information could ever be useful.

`` `Who told you that?' '' the husband asked himself, anticipating
companions' questions. `` `Some girl that writes in The New York
Times!' ''

``They'd never believe us,'' mused his wife, who had ordered cheesecake
for dinner.

At another meal, my table mates were a Missouri-based retired physician
and her husband, a retired special-ed teacher, plus a retired architect
from Arizona who was traveling alone. In the middle of a conversation
about how they met their spouses, the architect suddenly seemed
preoccupied with his iPhone. ``I read one where it said,'' he muttered
into his chest, `` `Keep your photo of your wife when you met her.' ''
He lifted the phone and showed the table his lock screen: a
black-and-white photo of a beautiful young woman in 1960s dress. I
barely managed not to cry into my Land \& Sea entree (Amtrak Signature
Steak with optional Béarnaise sauce, plus additional crab, shrimp and
scallop cake).

Back in my warm little room, there was something I couldn't put my
finger on that made it subtly nicer than my Lake Shore Limited
accommodations, and that was the in-room toilet, because this roomette
did not have one. I had been given a stationary pointing tour of the
compartment by the Chief's sleeping-car attendant --- a middle-aged
woman from a small town in Mexico, who, like every Amtrak attendant with
whom I interacted over the course of three days, hummed along with the
unflustered friendliness of a benevolent spirit continuing to go about
its business in a hotel decades after the property has been converted
into luxury condos. Her soothing voice made everything she said sound
like the hurried recitation of a familiar recipe. Her assessment of me
--- ``You are on vacation, you probably want to close the curtains and
sleep and sleep, wake up and eat and then go and take another nap, it's
O.K., that's why you are on vacation'' --- was delivered all in one
breath.

\hypertarget{ecstasy-in-the-sightseer-lounge}{%
\paragraph{\texorpdfstring{\textbf{Ecstasy in the Sightseer
Lounge}}{Ecstasy in the Sightseer Lounge}}\label{ecstasy-in-the-sightseer-lounge}}

Kansas shares a border with Colorado. I never could have imagined that I
would one day say this, and I know many people will be disconcerted by
the statement. They will wonder if, this whole time, they have been
reading an avant-garde work of science fiction, or perhaps a Mad Lib.
``Is magical realism always this scary?'' they will ask themselves. Some
will claim I am lying. Many will assume I am wrong, demented or a clumsy
typist.

To all of whom I respond: The truth of our nation's internal
demarcations is stranger than fiction --- stranger than even the kind of
brilliant avant-garde science fiction I am most likely capable of
producing yet choose not to. But the unvarnished fact is Colorado has to
start somewhere, and for whatever reason, that's inside Kansas.

I woke in Colorado to a weather phenomenon called the pogonip ---
freezing fog that condensed on tree limbs and sagebrush until they
looked dusted with powdered sugar. The terrain of the Colorado
tablelands is so flat that it seemed possible to detect the exact
location where the pogonip ended and blue skies began, the margins of
the changing landscape revealing themselves as definitively as gutters
between panels of a newspaper comic.

A childlike compulsion to identify distant cows rippled through the
observation car as we hurried along. So fast did we fly past baby deer
that the ``aw!''s caught in our throats. Whichever way you face, you are
privy to an all-day show, although there is a nagging sensation that by
being focused in one direction, you are missing something spectacular
unfolding in another. Sometimes you are. Sometimes other people will
even tell you you are, like when a grizzled stranger sat down next to
me, close enough to be way too close, jerked his head behind us,
growled, ``That's Pikes Peak'' and walked away. Unknown to me, on the
north side of the train, the Rockies had just begun to loom up out of
the prairie.

Azure and golden orange were the colors of the afternoon. Action-movie
posters are dominated by this color combination, famous for its
vibrancy, and indeed, a horizon filled with just these hues seemed to
draw the Sightseer Lounge into a kind of trance. For a long while there
was nothing but sky and earth to observe --- I saw actual tumbleweeds
somersault by --- yet everyone, me included, remained riveted to the
windows. It was possible, in the Sightseer Lounge, to watch weather roll
in from a great distance, even from one side of the car to the other. As
we ascended hills covered in pinyon and juniper, flakes began to fall,
and soon we were in a winter forest. As quickly as we had entered the
snowscape, however, we were back in dusty New Mexican grasslands,
rolling through a hailstorm of white birds.

Sunset pushed the denizens of the Sightseer Lounge to the brink of
insanity, as all but the Amish frantically tried to capture the
flame-colored sky on our cellphone cameras. A companionable mother I met
earlier in the day, accompanying her own parents on a casino trip to
Nevada, dashed from another car to make sure I was facing out of the
best side of the lounge to photograph the heavens. When the sun dipped
below the horizon, the sky turned the color of wet slate, then dark
denim blue with a pale apricot smear that we chased west for several
miles.

Scale on a rail trip is what's most arresting. We live so much of our
lives close-up --- scrolling through phones, watching our type appear on
computer screens, scrutinizing papers, preparing meals, cleaning our
homes room by room. Very few elements of our day-to-day tasks remain out
of arms' reach. An extended train ride affords a chance not just to see
a horizon but also to soak it up. To luxuriate in the far-off for
uninterrupted hours. To exist, briefly, in the uncharted sections of the
cellphone-coverage map.

And it feels as if you're getting away with something --- seeing more
than you deserve. The best part of the trip wasn't spying on the
backyards of houses; it was out here, in the open. The bright hues of
the nation's choropleth population-density maps fade to white in these
areas, yet many of the most beautiful habitable parts of the United
States, no offense to Boston, are contained within those colorless
expanses. Amtrak takes advantage of this circumstance. It is fortunate
that its routes were laid during a period of industrious optimism, when
everyone assumed the West would soon be made as unbearable as the East;
if they had known it would remain beautiful, it would have been
difficult to justify the financial investment.

Lying in my berth, I felt as happy as an egg in an incubator with no
plans to hatch. My mood was so upbeat that when I spotted a vitamin on
the carpet, I optimistically assumed it was the one I'd been keeping in
my pocket for weeks but forgetting to take, and I popped it in my mouth,
reminding myself to look up the writing stamped on it later. It turned
out to have been a supplement for adults 50 and over. I had become
train-lulled.

When I awoke on the third day, we were about an hour behind schedule. It
had happened, our attendant explained, when assistance for a handicapped
passenger was slow to arrive at an overnight stop. ``We can't rush
them!'' she chided (referring, presumably, to the passenger rather than
the assistance), though delay appeared to have dampened no moods; it
meant that the sun rose over the San Bernardino Mountains at breakfast.
As we approached our final destination, the scenery deteriorated, the
red rock vistas replaced by heaps of wooden pallets stacked in
strip-mall parking lots. When we pulled into the last stop on the line,
the train was almost empty. I had surveyed thousands of miles of
panoramic splendor, and I couldn't believe I had come all that way just
to get to Los Angeles.

\emph{\textbf{Caity Weaver} is a writer at large for the magazine and a
writer for The Times's Styles section. She last wrote for the magazine
\href{https://www.nytimes3xbfgragh.onion/2018/09/14/magazine/maya-rudolph-snl-amazon-forever.html}{about
the actress and comedian Maya Rudolph.} \textbf{Holly Andres} is a
photographer known for her cinematic style. She last photographed the
figure skater Jason Brown for the magazine.}

\hypertarget{-8}{%
\subsection{}\label{-8}}

\begin{itemize}
\item
  \href{https://www.nytimes3xbfgragh.onion/interactive/2019/03/20/magazine/rick-steves-travel-world.html}{}

  \includegraphics{https://static01.graylady3jvrrxbe.onion/images/2019/03/24/magazine/24mag-steves1/24mag-steves1-mediumThreeByTwo225-v3.jpg}

  \hypertarget{rick-steves-wants-to-save-the-world-one-vacation-at-a-time}{%
  \subsection{Rick Steves Wants to Save the World, One Vacation at a
  Time}\label{rick-steves-wants-to-save-the-world-one-vacation-at-a-time}}

  May 13, 2019
\item
  \href{https://www.nytimes3xbfgragh.onion/interactive/2019/03/20/magazine/train-across-america-amtrak.html}{}

  \includegraphics{https://static01.graylady3jvrrxbe.onion/images/2019/03/24/magazine/24mag-train-slideshow-slide-A3W0/24mag-train-slideshow-slide-A3W0-mediumThreeByTwo225-v2.jpg}

  \hypertarget{there-is-no-reason-to-cross-the-us-by-train-but-i-did-it-anyway}{%
  \subsection{There Is No Reason to Cross the U.S. by Train. But I Did
  It
  Anyway.}\label{there-is-no-reason-to-cross-the-us-by-train-but-i-did-it-anyway}}

  March 24, 2019
\item
  \href{https://www.nytimes3xbfgragh.onion/interactive/2019/03/21/magazine/mushrooms-sweden-island.html}{}

  \includegraphics{https://static01.graylady3jvrrxbe.onion/images/2019/03/24/magazine/24mag-mushrooms-slide-OT0K/24mag-mushrooms-slide-OT0K-mediumThreeByTwo225-v2.jpg}

  \hypertarget{the-mystery-gift-at-the-end-of-the-world}{%
  \subsection{The Mystery Gift at the End of the
  World}\label{the-mystery-gift-at-the-end-of-the-world}}

  March 24, 2019
\item
  \href{https://www.nytimes3xbfgragh.onion/interactive/2018/03/22/magazine/voyages-kayaking-across-ocean-at-70.html}{}

  \includegraphics{https://static01.graylady3jvrrxbe.onion/images/2018/03/25/magazine/25mag-kayaker1/25mag-kayaker1-mediumThreeByTwo225.jpg}

  \hypertarget{why-he-kayaked-across-the-atlantic-at-70-for-the-third-time}{%
  \subsection{Why He Kayaked Across the Atlantic at 70 (for the Third
  Time)}\label{why-he-kayaked-across-the-atlantic-at-70-for-the-third-time}}

  Aug. 24, 2018
\item
  \href{https://www.nytimes3xbfgragh.onion/2018/03/22/magazine/our-10-most-remarkable-voyages-to-far-flung-destinations.html}{}

  \includegraphics{https://static01.graylady3jvrrxbe.onion/images/2016/09/25/magazine/25danakil-ss-slide-CNK4/25danakil-ss-slide-CNK4-mediumThreeByTwo225.jpg}

  \hypertarget{our-10-most-remarkable-voyages-to-far-flung-destinations}{%
  \subsection{Our 10 Most Remarkable Voyages to Far-Flung
  Destinations}\label{our-10-most-remarkable-voyages-to-far-flung-destinations}}

  June 8, 2018
\end{itemize}

Advertisement

\hypertarget{site-information-navigation}{%
\subsection{Site Information
Navigation}\label{site-information-navigation}}

\begin{itemize}
\tightlist
\item
  \href{https://help.nytimes3xbfgragh.onion/hc/en-us/articles/115014792127-Copyright-notice}{©
  2020 The New York Times Company}
\item
  \href{https://www.nytimes3xbfgragh.onion}{Home}
\item
  \href{https://www.nytimes3xbfgragh.onion/search/}{Search}
\item
  Accessibility concerns? Email us at
  \href{mailto:accessibility@NYTimes.com}{\nolinkurl{accessibility@NYTimes.com}}.
  We would love to hear from you.
\item
  \href{https://help.nytimes3xbfgragh.onion/hc/en-us/articles/115015385887-Contact-Us}{Contact
  Us}
\item
  \href{https://www.nytco.com/careers/}{Work with us}
\item
  \href{https://nytmediakit.com/}{Advertise}
\item
  \href{https://help.nytimes3xbfgragh.onion/hc/en-us/articles/115014892108-Privacy-policy\#pp}{Your
  Ad Choices}
\item
  \href{https://help.nytimes3xbfgragh.onion/hc/en-us/articles/115014892108-Privacy-policy}{Privacy}
\item
  \href{https://help.nytimes3xbfgragh.onion/hc/en-us/articles/115014893428-Terms-of-service}{Terms
  of Service}
\item
  \href{https://help.nytimes3xbfgragh.onion/hc/en-us/articles/115014893968-Terms-of-sale}{Terms
  of Sale}
\end{itemize}

\hypertarget{site-information-navigation-1}{%
\subsection{Site Information
Navigation}\label{site-information-navigation-1}}

\begin{itemize}
\tightlist
\item
  \href{https://spiderbites.nytimes3xbfgragh.onion}{Site Map}
\item
  \href{https://help.nytimes3xbfgragh.onion/hc/en-us}{Help}
\item
  \href{https://help.nytimes3xbfgragh.onion/hc/en-us/articles/115015385887-Contact-Us?redir=myacc}{Site
  Feedback}
\item
  \href{https://www.nytimes3xbfgragh.onion/subscription?campaignId=37WXW}{Subscriptions}
\end{itemize}
