 **NYTimes.com no longer supports Internet Explorer 9 or earlier. Please
upgrade your browser.
\href{http://www.nytimes3xbfgragh.onion/content/help/site/ie9-support.html}{LEARN
MORE »}

**Sections

**Home

**Search

\hypertarget{the-new-york-times}{%
\subsection{\texorpdfstring{\href{http://www.nytimes3xbfgragh.onion/}{The
New York Times}}{The New York Times}}\label{the-new-york-times}}

 \href{https://www.nytimes3xbfgragh.onion/section/magazine}{Magazine}
\textbar{}I'm 20. I Have 32 Half Siblings. This Is My Family Portrait.

**Close search

\hypertarget{site-search-navigation}{%
\subsection{Site Search Navigation}\label{site-search-navigation}}

Search NYTimes.com

**Clear this text input

Go

\url{https://nyti.ms/2KCaq8I}

\hypertarget{site-navigation}{%
\subsection{Site Navigation}\label{site-navigation}}

\hypertarget{site-mobile-navigation}{%
\subsection{Site Mobile Navigation}\label{site-mobile-navigation}}

\hypertarget{im-20-i-have-32-half-siblings-this-is-my-family-portrait}{%
\section{I'm 20. I Have 32 Half Siblings. This Is My Family
Portrait.}\label{im-20-i-have-32-half-siblings-this-is-my-family-portrait}}

I always knew I was conceived using a sperm donor.

\includegraphics{https://static01.graylady3jvrrxbe.onion/packages/flash/multimedia/ICONS/transparent.png}

\includegraphics{https://static01.graylady3jvrrxbe.onion/packages/flash/multimedia/ICONS/transparent.png}

\includegraphics{https://static01.graylady3jvrrxbe.onion/packages/flash/multimedia/ICONS/transparent.png}

\includegraphics{https://static01.graylady3jvrrxbe.onion/packages/flash/multimedia/ICONS/transparent.png}

\includegraphics{https://static01.graylady3jvrrxbe.onion/packages/flash/multimedia/ICONS/transparent.png}

But I was 19 before I discovered I had half siblings.

\includegraphics{https://static01.graylady3jvrrxbe.onion/packages/flash/multimedia/ICONS/transparent.png}

\includegraphics{https://static01.graylady3jvrrxbe.onion/packages/flash/multimedia/ICONS/transparent.png}

\includegraphics{https://static01.graylady3jvrrxbe.onion/packages/flash/multimedia/ICONS/transparent.png}

\includegraphics{https://static01.graylady3jvrrxbe.onion/packages/flash/multimedia/ICONS/transparent.png}

\includegraphics{https://static01.graylady3jvrrxbe.onion/packages/flash/multimedia/ICONS/transparent.png}

\includegraphics{https://static01.graylady3jvrrxbe.onion/packages/flash/multimedia/ICONS/transparent.png}

\includegraphics{https://static01.graylady3jvrrxbe.onion/packages/flash/multimedia/ICONS/transparent.png}

\includegraphics{https://static01.graylady3jvrrxbe.onion/packages/flash/multimedia/ICONS/transparent.png}

Then I went searching --- for all 32 of them.

\includegraphics{https://static01.graylady3jvrrxbe.onion/packages/flash/multimedia/ICONS/transparent.png}

\includegraphics{https://static01.graylady3jvrrxbe.onion/packages/flash/multimedia/ICONS/transparent.png}

\includegraphics{https://static01.graylady3jvrrxbe.onion/packages/flash/multimedia/ICONS/transparent.png}

\includegraphics{https://static01.graylady3jvrrxbe.onion/packages/flash/multimedia/ICONS/transparent.png}

\includegraphics{https://static01.graylady3jvrrxbe.onion/packages/flash/multimedia/ICONS/transparent.png}

\includegraphics{https://static01.graylady3jvrrxbe.onion/packages/flash/multimedia/ICONS/transparent.png}

\includegraphics{https://static01.graylady3jvrrxbe.onion/packages/flash/multimedia/ICONS/transparent.png}

\includegraphics{https://static01.graylady3jvrrxbe.onion/packages/flash/multimedia/ICONS/transparent.png}

\includegraphics{https://static01.graylady3jvrrxbe.onion/packages/flash/multimedia/ICONS/transparent.png}

\includegraphics{https://static01.graylady3jvrrxbe.onion/packages/flash/multimedia/ICONS/transparent.png}

\includegraphics{https://static01.graylady3jvrrxbe.onion/packages/flash/multimedia/ICONS/transparent.png}

\includegraphics{https://static01.graylady3jvrrxbe.onion/packages/flash/multimedia/ICONS/transparent.png}

\includegraphics{https://static01.graylady3jvrrxbe.onion/packages/flash/multimedia/ICONS/transparent.png}

\includegraphics{https://static01.graylady3jvrrxbe.onion/packages/flash/multimedia/ICONS/transparent.png}

\includegraphics{https://static01.graylady3jvrrxbe.onion/packages/flash/multimedia/ICONS/transparent.png}

\includegraphics{https://static01.graylady3jvrrxbe.onion/packages/flash/multimedia/ICONS/transparent.png}

\includegraphics{https://static01.graylady3jvrrxbe.onion/packages/flash/multimedia/ICONS/transparent.png}

\includegraphics{https://static01.graylady3jvrrxbe.onion/packages/flash/multimedia/ICONS/transparent.png}

\includegraphics{https://static01.graylady3jvrrxbe.onion/packages/flash/multimedia/ICONS/transparent.png}

\hypertarget{a-family-portrait-brothers-sisters-strangers}{%
\section{A Family Portrait: Brothers, Sisters,
Strangers}\label{a-family-portrait-brothers-sisters-strangers}}

Photo essay by Eli Baden-Lasar\\
Text by Eli Baden-Lasar as told to Susan Dominus

June 26, 2019

\textbf{It was never a secret} in my house that I was conceived with the
help of an anonymous sperm donor. For a majority of my childhood, I
never really thought about him. But when I was around 11, I went through
a period of having questions. My parents --- I have two mothers --- gave
me a photo copy of a questionnaire that was sent to them from the sperm
bank they used, California Cryobank. The donor filled it out in 1996,
two years before I was born.

I remember carrying the form with me in my backpack, taking it to school
and studying it occasionally when I remembered I had it. There was this
sense of touch --- this person had used his hand to answer these
questions; I could see where he had crossed things out. It wasn't that I
was so desperate to imagine who he was; it was enough to have proof that
he was real, entangled with who I am and yet, as that document showed,
totally separate. The form made him concrete, if inscrutable. It also
gave me the sense that there was this larger world, this process and
this bureaucracy that my existence was built upon. It was a way to help
me understand myself.

\includegraphics{https://static01.graylady3jvrrxbe.onion/packages/flash/multimedia/ICONS/transparent.png}

\textbf{Eli Baden-Lasar.} The photographer, sitting on his mother's bed
in the home where he grew up in Oakland.

I knew a lot of other children whose parents had used donors to conceive
because every summer we went to a camp for same-sex families. Last
summer, news traveled through the community that two kids from two
families who attended the camp for years had independently gone on to a
registry for family members trying to connect with donors or donor
siblings. The two discovered that they shared a donor --- that they were
half siblings.

Until that moment, it had not really occurred to me --- or my mothers,
even though one is an ObGyn --- that I might have half siblings out
there. It makes no sense that we didn't think about that, because my
parents deliberately chose a donor whose sperm had successfully produced
at least one live birth, whose sperm had, in a sense, ``worked.'' I
think they were just so focused on thinking about the new family they
were creating that they never stopped to think about the implications of
the huge, inadvertent social experiment they were joining.

The news about the two kids at camp made me curious to find out if
\emph{I} had half siblings that I did not know about. So that same
month, last August, when I was 19, I dug up the questionnaire, went to
the sibling registry for California Cryobank, the largest sperm bank in
the nation, and typed in the donor's number. I landed on a message board
for children of my particular donor and saw about a dozen cryptic user
names of various mothers or children who were perhaps hesitant to reveal
themselves completely. One jumped out at me --- it said jplamb.

\emph{{[}}\href{https://www.nytimes3xbfgragh.onion/2019/06/26/magazine/sperm-donor-questions.html}{Sperm
donors can't stay secret anymore. Here's what that means.}\emph{{]}}

\textbf{I grew up} in Oakland, but I spent a semester in high school at
a program in New York for kids interested in experiential learning, and
one friend I made there, I knew, had two mothers who used a sperm donor
to conceive him. His name was Gus Lamb. Right away, I texted him to ask
if he had registered on the California Cryobank. He said he had. We
exchanged donor numbers, and then we knew: We were half siblings.

It was a moment of glee but also of horror. I knew that as a story it
was mind-blowing, but it was also disturbing --- to have the script
switched, to go from friends to brothers. In our experiential-learning
program, we were constantly being asked to write personal essays to try
to understand our lives. For four months, we were doing that and reading
each other's work and sleeping on the same floor of a dorm, all the
while not knowing that we were half brothers --- the perversity of that
was not lost on either of us.

Sheepishly, we both wondered how that was possible: How could we not
have somehow known? But at the same time, we both recognized that it
didn't seem so obvious. I had this suspicious feeling that scientists
were conducting an experiment, had taken a lunch break and then
forgotten to check back. But no one was watching through the two-way
mirror, and instead we were stuck looking at each other, reflected and
refracted, different people, but the same, mouths agape. If it was an
experiment, the variables had not yielded some thrilling result. There
had been no instant connection or unbreakable bond, and we easily lost
touch when the program ended.

We got on the phone, me in California, Gus in Massachusetts. Gus told me
that he had never been especially drawn to learning more about the donor
siblings. His sister Izzy, however, who had the same donor, had done
research for medical reasons after having her appendix out. ``There's
tons of siblings,'' Gus told me. That was another shock. Many of them,
he said, had been in touch for years. Gus and Izzy even had
video-chatted with a few.

When we hung up, I told my parents what I'd learned, and they were
equally stunned. I felt both curious and anxious about these people and
what they exactly meant to me. The sheer quantity of them gave me a
feeling of having been mass-produced.

Even as I was trying to take this information in, I was realizing that
one way I could maybe make sense of all of this was through photography,
a medium I've been interested in from a young age. I could use the
camera as an excuse to meet each sibling and maybe the process of making
pictures would help me find some sort of stability, even as I also
recognized that conflict, discomfort and maybe even a kind of love would
be part of the experience.

\textbf{The first people} I planned to shoot were Gus and Izzy. My
younger sister, Ruby, who was conceived using a different donor,
traveled with me to their home outside Boston. Hanging out with Gus felt
familiar and alien at the same time. Our time at that school together
was a prologue; now we were beginning again, and this time I was
learning about him in a different way. There are some things about a
person you can't understand without seeing the place where they grew up.
It's a type of access and point of view that allows you to see someone
in a very vulnerable state: This was their given life, messiness and
all, not necessarily the life they want to build for themselves.

\includegraphics{https://static01.graylady3jvrrxbe.onion/packages/flash/multimedia/ICONS/transparent.png}

\textbf{Isadora, 16, and Gus Lamb, 20,} in the yard of their mother's
home in Jamaica Plain, Mass. She is in high school; he will attend
college in the fall. ``I used to be very uncomfortable having so many
half siblings. l never knew how much time to put in. Was I supposed to
choose the coolest on the internet? Meeting some of them in person has
completely turned around my mind-set. I'm much more intrigued and
excited by it than I used to be. Whenever I meet half siblings, we're
very gentle with each other.'' --- Gus

\includegraphics{https://static01.graylady3jvrrxbe.onion/packages/flash/multimedia/ICONS/transparent.png}

\textbf{Sydney Hall, 18,} in her bedroom in Upper Chichester, Pa. ``I
got in touch with the group about a year ago. I learned that there are
so many of them it's hard to feel included. I'm an only child and was
expecting a sibling relationship, not just like, ``Hey, cool, we have
the same blood, whatever.'' I told myself that it wasn't a big deal that
I had siblings, just to numb the pain.''

\includegraphics{https://static01.graylady3jvrrxbe.onion/packages/flash/multimedia/ICONS/transparent.png}

\textbf{Alexis Clay, 20,} in a common room at her college in Vermont.
``As a kid, I had this burning curiosity to find out who my donor was.
When I first found I had half siblings, it was a source of comfort. But
as more and more half siblings were introduced into my life, it made me
feel like a statistic rather than an actual person. I feel drowned out
with the numbers.''

\includegraphics{https://static01.graylady3jvrrxbe.onion/packages/flash/multimedia/ICONS/transparent.png}

\textbf{English Jackson, 21,} in Grottoes, Va., where she grew up; she
attends college nearby.

\textbf{I knew I} wanted to try to photograph all the siblings in the
environments in which they were raised, and I knew I wanted the images
to convey a sense of drama even when depicting quotidian scenes. I
decided to learn how to use a view camera, which is a large-format,
old-fashioned-looking film camera with bellows. It requires a lot of
technical fiddling, focusing and refocusing and finding the right angle,
which makes taking pictures incredibly, if not painfully, slow ---
usually at least an hour. For the siblings, I think taking that kind of
photograph was strange, but it also allowed them to sit still and
concentrate on the picture as much as I was. The camera makes images
that are rich and detailed. I wanted something that was going to feel
like the opposite of mass production, that would have none of the
slickness that I was starting to associate with the sperm bank. It has a
clean, simple, commercial message about helping families and ads that
present donors as superheroes, their future babies as geniuses. I wanted
to produce something that would be exhaustive and overwhelming, that
would complicate the industry's message --- that would refute any simple
narratives.

In the picture of Gus and Izzy, they are posed in red plastic lawn
chairs that look like blown-up versions of toddler's chairs. I was after
this combination of both a formality --- in their almost regal posture
--- and a whimsy or childishness, as if the chairs had grown up with
them.

Gus included me in a group chat that about half the siblings use. From
there one led me to another and another until I was in contact with all
of them. I kept those exchanges brief, because I wanted to feel the
potency of our first encounter.

I took Gus and Izzy to the next shoot. It was the first time I met a
sibling that I hadn't already known, and I was suddenly more nervous
than I expected. When we all got out of the car, my hand began to
tremble so much that I dropped my keys. The physiological betrayal
rattled me, because I knew I was going to have to do this about 30 more
times. As a way of managing my nerves in the early meetings with
siblings, I was immediately focused on the work, on figuring out where
we would take the picture and what kind of image would be powerful. We
would walk together through various rooms in the house, contemplate our
options, before finally deciding on the right place.

When I met Sadie, a college student in Portland, Ore., she was living in
a single room in a small guesthouse, so there was just one place we
could take the picture. We spent most of our time talking and listening
to ``Best of Motown'' from a massive speaker she found on Craigslist.
She started hunting through things she had bought in thrift stores that
we could use in the picture, like the half-moon visible above her head
in the photograph. But she also showed me things about her life that we
knew were not going to be in the image: her own photos, an album she
made of other people's abandoned shopping lists.

By then, it had become clear to me that 90 percent of the time that I
spent with each sibling needed to be unrelated to the photograph itself.
It needed to be about our getting to know each other, about my trying to
understand the other person's life. It couldn't be rushed. The emotional
labor of the project was intended to be almost reparative --- a response
to the transactional nature of the sperm bank and the financial exchange
our parents made in order to create us.

\includegraphics{https://static01.graylady3jvrrxbe.onion/packages/flash/multimedia/ICONS/transparent.png}

\textbf{Ben, 20, and Julia Berman, 17,} in Julia's bedroom in San
Anselmo, Calif. He attends college in Louisiana; she is in high school.
``We were on our way to a cousin's wedding when I was little, and my mom
said to my brother and me, ``When you're dating, you'll have to be
careful and take a DNA test to make sure it's not your half sibling.'' I
was like: ``Mom, what are you talking about? I'm 7!'' But it was
something that was definitely on her mind.''

\includegraphics{https://static01.graylady3jvrrxbe.onion/packages/flash/multimedia/ICONS/transparent.png}

\textbf{Zachary Osborne-Schaefer, 20,} in his mother's tropical-fish
store in Santa Rosa, Calif.

\includegraphics{https://static01.graylady3jvrrxbe.onion/packages/flash/multimedia/ICONS/transparent.png}

\textbf{Nick Leonard, 20,} a college student, in Eugene, Ore.

\includegraphics{https://static01.graylady3jvrrxbe.onion/packages/flash/multimedia/ICONS/transparent.png}

\textbf{Neylan Griffy, 19,} a college student, in the home where her
mother grew up in Fowler, Colo.

\includegraphics{https://static01.graylady3jvrrxbe.onion/packages/flash/multimedia/ICONS/transparent.png}

\textbf{Sadie Pearson, 20,} in an apartment in Portland, Ore., where she
attends college. ``When I first met everyone, it was more magical
because we were younger. In the beginning there were seven main siblings
who talked. Now, when we find new ones, I'm kind of numb to the fact
that there are more siblings. How is it going to be now? How will I be
close to everybody?''

\includegraphics{https://static01.graylady3jvrrxbe.onion/packages/flash/multimedia/ICONS/transparent.png}

\textbf{Lily Supovitz, 21,} in the living room of an apartment in Bend,
Ore., where she attends college. ``Since meeting my siblings, I've
become more confident of my identity. I'm no longer wondering, Who am I?
And being connected to that side of my genes really helped me feel less
alone, because a lot of the siblings, when I first met them, were going
through similar struggles. And honestly, more excited for life ---
because you just never know what's going to happen.''

\textbf{So many of} these kinds of half sibling relationships that I've
heard about are hard to sustain because they're built over text and
social media --- the geographical separations become too great, which
can make it easier for people to distance themselves. You can't really
get to know someone online, this space where we make our lives more
consumable for one another. By meeting in person, there was no hiding.

Over 10 months, I traveled to 16 states to meet and shoot the 32
siblings. (One did not participate.) Sometimes I spent an afternoon,
sometimes a few days. I decided not to bring an assistant to help with
the light or make the process run more smoothly --- even if that would
have helped produce the best images possible, technically.

Looking through the camera, I had a feeling I couldn't shake: that these
people were all versions of me, just formed in different parts of the
country --- but were also strangers who might as well have been picked
out of a hat. The camera gave me an excuse to study each person --- to
look deeply at them in a way that without a camera would have been
uncomfortable and socially unacceptable.

Every once in a while, I would see something eerie about myself in one
of the other siblings that could completely scramble my sense of self
--- the way that one's neck became splotchy when she was uncomfortable
or the way another one bit his lip. Once, I heard a sibling laugh, and
it was so much my own laugh that it made the hair on my neck stand up.

In December, I made a trip to Honolulu, where I visited Kelsi Ikeda at
the home in which she grew up. It was the first place I traveled by
plane, and I remember waking up in her house that first morning feeling
disoriented. It was hot, I could feel the breeze of a fan, it smelled
different, and I was in a bunk bed. It took me a while to realize where
I was. And I remember thinking: Why am I here? Whom can I hold
accountable for this feeling? The bank? My parents? The donor? Myself?
What am I doing exactly? And what am I trying to accomplish?

At times, committing to a project like this has felt masochistic. I'm
generally an introverted person, and it was hard to feel as if I
constantly had to be on, performing the most appealing version of
myself. Though the feeling of performance quickly dissolved, I still had
a recurring sensation of being in a confused state of just-waking-up, of
trying to find my place in all these different parts of America as well
as in this strange social landscape.

I spent four days with Kelsi and her family, enough time that by the end
I felt a real affection for them all. In her picture, she's wearing her
prom dress from junior year of high school. It feels funny, even tinged
with a hint of embarrassment, to try on the clothes of our past selves.
But it helped us get acquainted.

During the time I spent with my half siblings, we exchanged secrets.
People get very confessional around a stranger who has no stake in their
life on a day-to-day basis. We had a connection, which meant they could
trust me, but I wasn't a potential future friend they needed to impress.
I was something else --- some third thing.

\includegraphics{https://static01.graylady3jvrrxbe.onion/packages/flash/multimedia/ICONS/transparent.png}

\textbf{Kelsi Ikeda, 20,} a college student in Southern California, at
her childhood home in Honolulu. ``I am not super close with any of the
siblings. Even though we are related, we are sort of just strangers. I'm
the only Asian one I'm aware of. I feel like if there is another
partially Asian person, I might connect more with them.''

\includegraphics{https://static01.graylady3jvrrxbe.onion/packages/flash/multimedia/ICONS/transparent.png}

\textbf{Bradley Holland, 20,} at his home in Lake Forest, Calif.

\includegraphics{https://static01.graylady3jvrrxbe.onion/packages/flash/multimedia/ICONS/transparent.png}

\textbf{Mickey Mann, 21,} a recent college graduate, in a home she
sublets in Bellingham, Wash.

\includegraphics{https://static01.graylady3jvrrxbe.onion/packages/flash/multimedia/ICONS/transparent.png}

\textbf{Matt Holland, 20,} twin brother of Bradley, **** in the yard of
his family's home in Lake Forest, Calif.; he attends college nearby.

\textbf{The following month,} I met Daniel Claypoole, who could be
described as the great connector: He seeks siblings out and sort of
holds the group together. He's social and extroverted and rallies people
around the idea of this being a group.

He lives in Savannah, Ga., where he had been going to art school, but I
met him in Albuquerque. His two younger brothers, Zeke, 14, and Grayson,
4, who both share our donor, live there. His sister, who is 9, and who
does not share our donor, was there, too, and she was trying to explain
to Grayson who I was. I don't know if he understood.

I knew I wanted to make Grayson's picture on that beanbag chair in their
living room. You can't see it in the image, but the entire wall is
covered in a pattern of crosses. I set up one continuous light and
instructed Grayson, a typical frenetic little kid, to stay still.
Amazingly, he did. Many people find the hot light uncomfortable, but he
seemed warmed by it. When we left, he cried, and I felt a pang of guilt.
I'm sure he was crying over Daniel, but I also wondered what he made of
the word ``brother,'' which was thrown around when we were introduced,
because I knew I would most likely not see him again for a very long
time, if ever.

Daniel and I drove to Clovis, N.M., to visit the house he grew up in
with his grandmother, grandfather and great-grandmother. I wanted to
incorporate biographical details in the photograph, like the painting
hanging on the wall of his great-great-grandfather, a man with piercing
eyes who, he had been told, was the chief firefighter in Clovis. On the
bottom left of the photo you can see a dictionary in which his
great-grandmother stores the family photos --- the wedding and baby
pictures are loosely tucked into random pages.

Even though Daniel has been in touch with other siblings for many years,
I was the first sibling from outside the family that his grandparents
had ever met. We felt very close by the end of the visit. I felt so
grateful for the way that his grandparents welcomed me, just as so many
other families had, giving me a place to sleep, a seat at family dinner.
It's strange to think that I have been in all of their homes, but none
of them have ever been in mine or met my parents.

\includegraphics{https://static01.graylady3jvrrxbe.onion/packages/flash/multimedia/ICONS/transparent.png}

\textbf{Daniel Claypoole, 19,} older brother of Zeke and Grayson
Barrett, in the living room of the house where he grew up in Clovis,
N.M. ``Mattie, one of my donor half sisters, and I have been inseparable
since we first met at a donor sibling meet-up when I was in eighth
grade. We both enrolled in SCAD, a top school for design, and we live in
the same building. She's still in school, but I had to drop out for
financial reasons. It can be hard to see Mattie doing what I want to be
doing, but I know I'm going to find a way to keep being creative. I'm
working as an assistant wedding planner. It's strange how many of the
siblings are artists or musicians.''

\includegraphics{https://static01.graylady3jvrrxbe.onion/packages/flash/multimedia/ICONS/transparent.png}

\textbf{Zeke Barrett, 14,} at his home in Albuquerque.

\includegraphics{https://static01.graylady3jvrrxbe.onion/packages/flash/multimedia/ICONS/transparent.png}

\textbf{Grayson Barrett, 4,} at his home in Albuquerque.

\includegraphics{https://static01.graylady3jvrrxbe.onion/packages/flash/multimedia/ICONS/transparent.png}

\textbf{Anna Grace Bond, 19,} in a field between her mother's and
grandparents' home in Wiggins, Miss.; she attends college nearby.
\textbf{Fletcher Bond, 19,} is currently in the Air Force. ``My granny
had been encouraging my mom to tell my twin brother, Fletcher, and me
that we were conceived by sperm donor. Finally, my mom told us at a
McDonald's drive-through. We were 17. My brother's reaction was, It
doesn't change anything. He's close to my dad. For me, it was like ---
life crisis. The very next day, I made my mom help me find the donor
number, and that was how I found the siblings. They were so welcoming
--- like, ``Hey, sister!'' I was like, ``This is so cool.'' To me, it's
not healthy to have to keep how you're made a secret.''

I'm always hesitant to call anybody a brother or sister. But many of the
other siblings use that language very loosely. I don't, probably because
I already have a sister, and she will always be most important to me.

But I have been struck by the closeness that comes from the intensity of
the time that we spent together or, who knows, maybe something more than
that. I spent about 12 hours with one of the siblings, Neylan Griffy.
She drove me from Denver almost all the way to Kansas to show me where
she is from. It was pitch dark, and we talked the whole time. When I was
leaving Colorado, and we were saying goodbye, she said, almost with
trepidation: ``I don't know if it's too early to say this, but I heart
you.''

She didn't want to push me or expect anything; she was just expressing
her feelings. She was one of the first siblings to connect with others,
so she may be more comfortable with that. I laughed and said, ``I heart
you, too.''

One of the last siblings I photographed was Dawson Johnson. One of the
others found him on 23andMe in January and connected me with him. He'd
never communicated with a sibling before me. I took Gus on my trip to
visit him, and the three of us met at an IHOP. He pulled up in a massive
black truck with oversize wheels. He's muscular, a taekwondo instructor
who was a serious high school wrestler. I wanted to portray that side of
him in the picture but also to capture something about his manner, which
was gentle. In the photograph we made, I placed him shirtless and
beneath the truck, in this vulnerable state.

\includegraphics{https://static01.graylady3jvrrxbe.onion/packages/flash/multimedia/ICONS/transparent.png}

\textbf{Dawson Johnson, 20,} near the home where he grew up in Memphis;
he works as a taekwondo instructor in Jackson, Tenn. ``I always knew my
sperm donor was white; I really don't know why my mom chose the way she
did. For the longest time, I wanted a dad, but I didn't get one. Meeting
all the siblings helped me see a little bit more clearly what my mom did
and why she did it . I used to see it as, My mom didn't give me a dad.
Now I see it as, Wow, she wanted me so badly.''

\includegraphics{https://static01.graylady3jvrrxbe.onion/packages/flash/multimedia/ICONS/transparent.png}

\textbf{Josh Lambert, 19,} outside his family's apartment in Long Beach,
Calif., where he attends college.

\includegraphics{https://static01.graylady3jvrrxbe.onion/packages/flash/multimedia/ICONS/transparent.png}

\textbf{Mattie Overmyer, 21,} an art-school student, in Savannah, Ga.

\includegraphics{https://static01.graylady3jvrrxbe.onion/packages/flash/multimedia/ICONS/transparent.png}

\textbf{Kyle Luzzi-Dundon, 20,} in his parents' home in Haydenville,
Mass. He lives in Avon, Colo.

\textbf{At some point,} in each sibling encounter, we would inevitably
end up talking about the donor. He represented this absence we all had
in common, almost a spectral figure hovering above our lives. Some
siblings, once they turned 18, had written to the donor and received
long letters back. A different sibling told me that although he wasn't
interested in actually contacting the donor, he wished he had the
ability to be invisible, to watch over him for one day as he went about
his life, a sort of inversion of the dynamic.

At one point, Izzy got her hands on an audio interview of the donor that
the bank made and that another sibling's mother had. (You can get more
information about the donor from the bank --- more extensive
questionnaire forms or an audio recording --- if you pay extra for it.)
She, Gus, Kyle Luzzi-Dundon (another sibling) and I listened to the
recording one night, huddled in a circle in a sort of séance.

The bank asks the donor at the end of the audio interview whether he has
anything he would like to tell any children conceived with his sperm.
Our donor's response: ``I wish them all the luck.'' One sibling
scribbled that on his bedroom wall during high school in colorful chalk
as if it were an inspirational quote. I heard it more as an irreverent
provocation: \emph{My job here is done. May the odds be ever in your
favor.}

Trying to understand what the donor means to me has been complicated. I
never planned on trying to contact him, but I ultimately did to let him
know about this project. He declined to be a part of it at this stage.
To me, it is more interesting for him to remain the missing and
invisible figure he has always been. I don't think he had any idea, at
the time he donated his sperm, that he was creating a kind of time
capsule that could potentially explode.

For me, there is a strange pleasure in being able to collapse space and
time by putting all these people from all these different locations next
to one another. For the viewer, there might be intrigue in searching for
the similarities and differences among each of us or even just knowing
that we are all connected on this deeper, genetic level.

\includegraphics{https://static01.graylady3jvrrxbe.onion/packages/flash/multimedia/ICONS/transparent.png}

\textbf{Nik Morgan, 21,} a college student in Boulder, Colo.; he grew up
in Aspen.

\includegraphics{https://static01.graylady3jvrrxbe.onion/packages/flash/multimedia/ICONS/transparent.png}

\textbf{Jack Shiley, 20,} in a garage where his band practices in New
Lenox, Ill.; he works at Guitar Center.

\includegraphics{https://static01.graylady3jvrrxbe.onion/packages/flash/multimedia/ICONS/transparent.png}

\textbf{Hannah, 17, and Jonny Schulman, 19,} in the backyard of their
family's home in Troy, N.Y. She's a high school student; he's in college
in Binghamton.

\includegraphics{https://static01.graylady3jvrrxbe.onion/packages/flash/multimedia/ICONS/transparent.png}

\textbf{Grayson Klages, 19,} a college student in Portland, Ore., in the
bedroom of his childhood home in Boulder, Colo.

\includegraphics{https://static01.graylady3jvrrxbe.onion/packages/flash/multimedia/ICONS/transparent.png}

\textbf{Nate Savinar, 20,} in his apartment in Denver, where he was born
and raised, with his training dummy; he teaches Krav Maga.

These pictures also capture a transitional stage in most of our lives
--- we are at the close of adolescence, on the brink of becoming our
adult selves. \emph{The basketball hoop has fallen in the front yard;
the prom dress has been tucked away in the back of the closet; the
bicycle with training wheels will soon be thrown out or given away.}

The project has no determined end, because other siblings may emerge in
the next weeks, months and years. Once, two siblings who hadn't met yet
but who'd seen photos of each other discovered that they were in an
airport at the same time. This incident seemed to confirm our paranoia
that we might be walking by siblings all the time without knowing it: in
the streets, on the subway, at our liberal-arts colleges.

Since finishing the project, or at least this phase of it, I sometimes
feel this haze state fall over me, in which other people start to look
like me. One day recently, on the subway, a young man about my age sat
down across from me. Medium build, dark auburn hair, full lips, one of
the most consistent features in all the siblings. I looked at his hands
--- they were knuckly and slender. They looked so much like mine. I
continued to stare and found myself on the brink of asking him an
uncomfortable question. But I didn't, and instead I thought about what
it means to be able to see yourself in strangers --- if, in the course
of this project, my capacity for empathy has grown, has opened me up, or
if the whole thing has been secretly rooted in self-interest, a fixation
with understanding who I am.

The photographs have been developed, selected, printed; I stare at them
now, see them side by side, I think about the work that made them ---
and still I'm not sure.

\emph{{[}}\href{https://www.nytimes3xbfgragh.onion/2019/06/26/magazine/sperm-donor-questions.html}{Sperm
donors can't stay secret anymore. Here's what that means.}\emph{{]}}

Tap anywhere to close

Please rotate your device.

\textbf{Eli Baden-Lasar} is a photographer from Oakland, Calif., who
will start his sophomore year at Wesleyan University in the fall.
\textbf{Susan Dominus} is a staff writer for the magazine. She last
wrote a feature about
\href{https://www.nytimes3xbfgragh.onion/interactive/2019/02/21/magazine/autism-office-design.html}{an
office designed for people with autism}.

Interviews of siblings by Susan Dominus. They have been edited and
condensed.

\subsection{}

\begin{itemize}
\item
  \href{https://www.nytimes3xbfgragh.onion/2019/06/29/reader-center/sperm-donor-siblings.html}{}

  \includegraphics{https://static01.graylady3jvrrxbe.onion/images/2019/06/29/insider/28-insider-pia/28-insider-pia-mediumThreeByTwo225.jpg}

  \hypertarget{he-found-out-he-had-32-siblings-for-the-times-magazine-he-took-their-pictures}{%
  \subsection{He Found Out He Had 32 Siblings. For The Times Magazine,
  He Took Their
  Pictures.}\label{he-found-out-he-had-32-siblings-for-the-times-magazine-he-took-their-pictures}}

  June 29, 2019
\item
  \href{https://www.nytimes3xbfgragh.onion/2019/06/26/magazine/sperm-donor-questions.html}{}

  \includegraphics{https://static01.graylady3jvrrxbe.onion/images/2019/06/30/magazine/30mag-DOMINUS/058a4aa76a8f49ffa9b0e06f3992cf0a-mediumThreeByTwo225.jpg}

  \hypertarget{sperm-donors-cant-stay-secret-anymore-heres-what-that-means}{%
  \subsection{Sperm Donors Can't Stay Secret Anymore. Here's What That
  Means.}\label{sperm-donors-cant-stay-secret-anymore-heres-what-that-means}}

  June 29, 2019
\item
  \href{https://www.nytimes3xbfgragh.onion/2019/06/29/reader-center/sperm-donor-siblings.html}{}

  \includegraphics{https://static01.graylady3jvrrxbe.onion/images/2019/06/29/insider/28-insider-pia/28-insider-pia-mediumThreeByTwo225.jpg}

  \hypertarget{he-found-out-he-had-32-siblings-for-the-times-magazine-he-took-their-pictures-1}{%
  \subsection{He Found Out He Had 32 Siblings. For The Times Magazine,
  He Took Their
  Pictures.}\label{he-found-out-he-had-32-siblings-for-the-times-magazine-he-took-their-pictures-1}}

  June 29, 2019
\item
  \href{https://www.nytimes3xbfgragh.onion/2015/07/12/magazine/the-mixed-up-brothers-of-bogota.html}{}

  \includegraphics{https://static01.graylady3jvrrxbe.onion/images/2015/07/12/magazine/12twins11/12twins11-mediumThreeByTwo225.jpg}

  \hypertarget{the-mixed-up-brothers-of-bogotuxe1}{%
  \subsection{The Mixed-Up Brothers of
  Bogotá}\label{the-mixed-up-brothers-of-bogotuxe1}}

  March 22, 2020
\item
  \href{https://www.nytimes3xbfgragh.onion/2015/04/19/magazine/the-cost-of-sally-manns-exposure.html}{}

  \includegraphics{https://static01.graylady3jvrrxbe.onion/images/2015/04/16/magazine/19mag-mann-cover-portrait/19mag-mann-cover-portrait-mediumThreeByTwo225.jpg}

  \hypertarget{sally-manns-exposure}{%
  \subsection{Sally Mann's Exposure}\label{sally-manns-exposure}}

  Jan. 19, 2018
\item
  \href{https://www.nytimes3xbfgragh.onion/2019/06/15/style/sperm-donation-fathers-day.html}{}

  \includegraphics{https://static01.graylady3jvrrxbe.onion/images/2019/06/15/fashion/15FATHERSDAY-ESSAY2/15FATHERDAY-ESSAY-mediumThreeByTwo225.jpg}

  \hypertarget{could-donor-2065-be-my-father}{%
  \subsection{Could Donor \#2065 Be My
  Father?}\label{could-donor-2065-be-my-father}}

  Oct. 30, 2019
\end{itemize}

Advertisement

\hypertarget{site-information-navigation}{%
\subsection{Site Information
Navigation}\label{site-information-navigation}}

\begin{itemize}
\tightlist
\item
  \href{https://help.nytimes3xbfgragh.onion/hc/en-us/articles/115014792127-Copyright-notice}{©
  2020 The New York Times Company}
\item
  \href{https://www.nytimes3xbfgragh.onion}{Home}
\item
  \href{https://www.nytimes3xbfgragh.onion/search/}{Search}
\item
  Accessibility concerns? Email us at
  \href{mailto:accessibility@NYTimes.com}{\nolinkurl{accessibility@NYTimes.com}}.
  We would love to hear from you.
\item
  \href{https://help.nytimes3xbfgragh.onion/hc/en-us/articles/115015385887-Contact-Us}{Contact
  Us}
\item
  \href{https://www.nytco.com/careers/}{Work with us}
\item
  \href{https://nytmediakit.com/}{Advertise}
\item
  \href{https://help.nytimes3xbfgragh.onion/hc/en-us/articles/115014892108-Privacy-policy\#pp}{Your
  Ad Choices}
\item
  \href{https://help.nytimes3xbfgragh.onion/hc/en-us/articles/115014892108-Privacy-policy}{Privacy}
\item
  \href{https://help.nytimes3xbfgragh.onion/hc/en-us/articles/115014893428-Terms-of-service}{Terms
  of Service}
\item
  \href{https://help.nytimes3xbfgragh.onion/hc/en-us/articles/115014893968-Terms-of-sale}{Terms
  of Sale}
\end{itemize}

\hypertarget{site-information-navigation-1}{%
\subsection{Site Information
Navigation}\label{site-information-navigation-1}}

\begin{itemize}
\tightlist
\item
  \href{https://spiderbites.nytimes3xbfgragh.onion}{Site Map}
\item
  \href{https://help.nytimes3xbfgragh.onion/hc/en-us}{Help}
\item
  \href{https://help.nytimes3xbfgragh.onion/hc/en-us/articles/115015385887-Contact-Us?redir=myacc}{Site
  Feedback}
\item
  \href{https://www.nytimes3xbfgragh.onion/subscription?campaignId=37WXW}{Subscriptions}
\end{itemize}
