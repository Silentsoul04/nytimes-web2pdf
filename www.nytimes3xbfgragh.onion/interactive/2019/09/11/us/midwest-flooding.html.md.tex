Sections

SEARCH

\protect\hyperlink{site-content}{Skip to
content}\protect\hyperlink{site-index}{Skip to site index}

\hypertarget{comments}{%
\subsection{\texorpdfstring{\protect\hyperlink{commentsContainer}{Comments}}{Comments}}\label{comments}}

\href{}{The Great Flood of 2019: A Complete Picture of a Slow-Motion
Disaster}\href{}{Skip to Comments}

The comments section is closed. To submit a letter to the editor for
publication, write to
\href{mailto:letters@NYTimes.com}{\nolinkurl{letters@NYTimes.com}}.

\hypertarget{the-great-flood-of-2019-a-complete-picture-of-a-slow-motion-disaster}{%
\section{The Great Flood of 2019: A Complete Picture of a Slow-Motion
Disaster}\label{the-great-flood-of-2019-a-complete-picture-of-a-slow-motion-disaster}}

By \href{https://www.nytimes3xbfgragh.onion/by/sarah-almukhtar}{Sarah
Almukhtar},
\href{https://www.nytimes3xbfgragh.onion/by/blacki-migliozzi}{Blacki
Migliozzi},
\href{https://www.nytimes3xbfgragh.onion/by/john-schwartz}{John
Schwartz} and
\href{https://www.nytimes3xbfgragh.onion/by/josh-williams}{Josh
Williams}Sept. 11, 2019

\begin{itemize}
\item
\item
\item
\item
\item
  \emph{56}
\end{itemize}

\begin{itemize}
\item
\item
\item
\item
\item
\item
\item
\item
\item
\end{itemize}

This year's flooding across the Midwest and the South affected nearly 14
million people, yet the full scale of the slowly unfolding disaster has
been difficult to fathom. To visualize just how extensive it was, The
New York Times created this composite map showing all the areas that
were inundated at some point from January through June.

The blue on the map, generated from satellite data used to detect
flooding, shows the estimated extent and intensity of \textbf{flooding}.
The usual extents of \textbf{rivers} are shown in white.
\textbf{Farmland}, much of which was so drenched that farmers had to
delay planting, is shown in green.

Minneapolis

Minneapolis

Eau Claire

Eau Claire

Mankato

Mankato

Rochester

Rochester

La Crosse

La Crosse

Sioux Falls

Sioux Falls

SOUTH DAKOTA

SOUTH DAKOTA

*

IOWA

Sioux City

Sioux City

Dubuque

Dubuque

Waterloo

Waterloo

Rockford

Rockford

Missouri

River

Missouri

River

IOWA

Norfolk

Ames

Cedar Rapids

Cedar Rapids

Chicago

Chicago

NEBRASKA

NEBRASKA

Des Moines

Des Moines

Columbus

Columbus

Fremont

Fremont

Ottawa

Ottawa

Omaha

Omaha

Kankakee

Kankakee

IOWA

Creston

Grand Island

Grand Island

York

Lincoln

Lincoln

Burlington

Burlington

Shenandoah

6 MILES

6 MILES

Peoria

Peoria

Nebraska City

*

Bloomington

Bloomington

Craig

Champaign

Champaign

MISSOURI

Beardstown

Indianapolis

Indianapolis

Quincy

Decatur

Quincy

Decatur

Springfield

Springfield

ILLINOIS

Chillicothe

Chillicothe

KANSAS

St. Joseph

St. Joseph

Jacksonville

Atchison

MISSOURI

Atchison

ILLINOIS

Brunswick

Louisiana

Brunswick

Louisiana

KANSAS

Norborne

Norborne

Manhattan

5 MILES

5 MILES

Glasgow

9 MILES

9 MILES

Kansas City

Kansas City

MISSOURI

Topeka

Topeka

*

Lawrence

Lawrence

Columbia

Columbia

Missouri River

Missouri River

Sedalia

St. Louis

St. Louis

Jefferson City

Jefferson City

Louisville

Louisville

Mt. Vernon

Grayville

Mississippi

River

Evansville

Evansville

Mt. Vernon

Mt. Vernon

Chester

Chester

MISSOURI

Owensboro

Owensboro

ILLINOIS

Wichita

Wichita

Mississippi

River

ILLINOIS

Livermore

Livermore

Cape

Girardeau

Cape Girardeau

KENTUCKY

Springfield

Springfield

Joplin

Joplin

*

Clarksville

Union City

Mississippi

River

Mississippi

River

Pocahontas

Nashville

Pocahontas

Nashville

Tulsa

Tulsa

ARKANSAS

Dyersburg

Dyersburg

TENNESSEE

Jonesboro

Jonesboro

12 MILES

12 MILES

Muskogee

Muskogee

Jackson

Jackson

Oklahoma City

Oklahoma City

Newport

Payneway

*

TENNESSEE

Savannah

Savannah

Russellville

Memphis

Memphis

OKLAHOMA

OKLAHOMA

Conway

Huntsville

Marianna

Little Rock

Little Rock

Mississippi

River

Mississippi

River

Hot Springs

Hot Springs

ARKANSAS

Tupelo

Tupelo

Mellwood

Pine Bluff

Pine Bluff

Arkansas

River

MISSISSIPPI

Oxberry

Oxberry

ARKANSAS

Birmingham

MISSISSIPPI

Greenville

Texarkana

Greenville

Texarkana

Tuscaloosa

Anguilla

ALABAMA

ALABAMA

Valley Park

Valley Park

Montgomery

Longview

Monroe

Longview

Monroe

20 MILES

20 MILES

Shreveport

Shreveport

LOUISIANA

Jackson

Tyler

LOUISIANA

Tyler

Jackson

*

Winnsboro

TEXAS

TEXAS

MISSISSIPPI

Natchez

Natchez

Sibley

Hattiesburg

Hattiesburg

MISSISSIPPI

Lufkin

Lufkin

Alexandria

Alexandria

LOUISIANA

Mobile

Morganza

Mississippi

River

Mississippi

River

Baton Rouge

LOUISIANA

Baton Rouge

Lake Charles

Lake Charles

Lafayette

Lafayette

Beaumont

*

New Orleans

New Orleans

Houston

Houston

Public interest in natural disasters tends to focus on big, discrete
weather events like hurricanes. But flooding that unfolds over months
across a broad area has a harder time breaking through. It is only when
seen as a single, connected event that the stunning scale of the 2019
flood season becomes clear.

To measure the scope of the spring floods, The New York Times analyzed
satellite data from the Joint Polar Satellite System using software,
developed by government and academic researchers for flood detection,
that is frequently used in disaster response.

The data covers the period from Jan. 15 to June 30 and shows an
interconnected catastrophe along the Missouri, Mississippi and Arkansas
Rivers, a system that drains more than 40 percent of the landmass of the
continental United States.

Areas that flooded

NORTH

DAKOTA

Missouri

River

MINNESOTA

WISCONSIN

SOUTH

DAKOTA

IOWA

NEBRASKA

ILLINOIS

Arkansas

River

KANSAS

MISSOURI

TENNESSEE

ARKANSAS

OKLAHOMA

Mississippi

River

LOUISIANA

TEXAS

MISSISSIPPI

N.D.

Missouri

River

MINN.

WISC.

S.D.

IA.

NEB.

ILL.

Arkansas

River

KS.

MO.

TENN.

ARK.

OKLA.

Mississippi

River

LA.

TEX.

MISS.

The New York Times

The flooding started in earnest in March as heavy rains fell on frozen
ground already bearing a deep load of winter snow. As the water spread
across parts of South Dakota, Nebraska and Iowa, traditional defenses
that communities had in place to protect against rising waters were
quickly overwhelmed.

As the high water from the many tributaries of the Mississippi came
together, the river became a record-breaking monster.

Major flood stage sites, Jan. 15 to June 30

Federal assistance-approved counties

NORTH

DAKOTA

Missouri

River

MINNESOTA

WISCONSIN

SOUTH

DAKOTA

IOWA

NEBRASKA

ILLINOIS

Arkansas

River

KANSAS

MISSOURI

TENNESSEE

ARKANSAS

OKLAHOMA

Mississippi

River

LOUISIANA

TEXAS

MISSISSIPPI

N.D.

Missouri

River

MINN.

WISC.

S.D.

IA.

NEB.

ILL.

Arkansas

River

KS.

MO.

TENN.

ARK.

OKLA.

Mississippi

River

LA.

TEX.

MISS.

Source: United States Geological Survey \textbar{} The New York Times

By the end of June, the flooding was so intense and widespread that at
least 11 states had sought federal disaster funds for more than 400
counties. Forty-nine United States Geological Survey gauges measured
more water this year than at any time in at least 20 years.

Bryan Tuma, assistant director of Nebraska's Emergency Management
Agency, said, simply, ``I would describe it as biblical.''

\includegraphics{https://static01.graylady3jvrrxbe.onion/packages/flash/multimedia/ICONS/transparent.png}

\includegraphics{https://static01.graylady3jvrrxbe.onion/newsgraphics/2019/08/07/midwest-flooding-scroll/assets/images/ackermangruber_cli-missouri-nebraska_066-2000.jpg}

Floodwaters spread across the town of Hamburg, Iowa, in March. Tim
Gruber for The New York Times

The causes of flooding are complicated, but climate change is
increasingly an exacerbating factor. Warmer air can hold more moisture,
and that moisture can fall back out of the sky, whether as rain or snow,
in greater amounts.

The year through May 2019 was the
\href{https://www.noaa.gov/news/rain-soaked-us-had-its-2nd-wettest-month-on-record-in-may}{wettest}
12-month period on record in the United States, according to the
National Oceanic and Atmospheric Administration. Nearly 38 inches of
water fell, almost eight inches above average.

\includegraphics{https://static01.graylady3jvrrxbe.onion/packages/flash/multimedia/ICONS/transparent.png}

\includegraphics{https://static01.graylady3jvrrxbe.onion/newsgraphics/2019/08/07/midwest-flooding-scroll/assets/images/gruber_missouririver_0008-2000.jpg}

Flooding outside of Fortescue, Mo., damaged the levee designed to hold
back water. Tim Gruber for The New York Times

A Mississippi River \href{https://www.mrcti.org/}{mayors council}
estimated that the cost of infrastructure damage and emergency response
was at least \$2 billion. That number is likely to rise as the water
recedes and officials can check the extent of the damage. The full cost
to repair homes and businesses has yet to be calculated.

David Alexander, a professor of risk and disaster reduction at
University College London, said that typical recovery times from major
disasters are ``in the range of 10 to 25 years.''

\includegraphics{https://static01.graylady3jvrrxbe.onion/packages/flash/multimedia/ICONS/transparent.png}

\includegraphics{https://static01.graylady3jvrrxbe.onion/newsgraphics/2019/08/07/midwest-flooding-scroll/assets/images/190506_grafton_flood_01-2000.jpg}

A boat made its way through a flooded street in Grafton, Ill., in May.
Whitney Curtis for The New York Times

Ultimately, the volume of water is only one source of damage: What's in
that water also plays a role.

The waters of the Mississippi, carrying chemical fertilizers from
heartland farms, lawns and other sources, have contributed to a
``\href{https://www.noaa.gov/media-release/noaa-forecasts-very-large-dead-zone-for-gulf-of-mexico}{dead
zone}'' in the Gulf of Mexico, an area with too little oxygen to support
fish and other marine life.

NOAA predicts that this year's dead zone will be 8,717 square miles ---
about the size of New Hampshire.

\includegraphics{https://static01.graylady3jvrrxbe.onion/packages/flash/multimedia/ICONS/transparent.png}

\includegraphics{https://static01.graylady3jvrrxbe.onion/newsgraphics/2019/08/07/midwest-flooding-scroll/assets/images/ms_flood_2019-41-2000.jpg}

Farms in the lower Mississippi Delta remained underwater in June. Rory
Doyle for The New York Times

Back in March, Edward Clark, director of NOAA's National Water Center,
said that what lay ahead was a ``potentially unprecedented'' flood
season, even worse than
the\href{https://www.nytimes3xbfgragh.onion/2019/06/03/us/midwest-floods.html}{}\href{https://www.nytimes3xbfgragh.onion/2019/06/03/us/midwest-floods.html}{Great
Flood of 1993}.

In a recent interview, Mr. Clark said, with no satisfaction, ``Yes, we
got it right.''

Sixty-three percent of comparable United States Geological Survey gauges
in the region recorded higher peaks this year than in the same period of
1993.

``This is a year that will remain in our cultural memory, in our
history,'' he said.

Methodology

To produce a single image of this year's flooding, The Times analyzed
six months of satellite imagery from VIIRS sensors provided by the
National Oceanic and Atmospheric Administration covering January to June
2019. The extent of flooding in each image was estimated by using a
\href{https://cimss.ssec.wisc.edu/cspp/}{model} produced by the
University of Wisconsin and described in an academic
\href{https://www.sciencedirect.com/science/article/abs/pii/S0034425717304431}{paper},
and checked against accounts of local officials in affected areas.

The Times analyzed multiple images to create daily and then weekly
composites. These were then aggregated into a single image that shows
the extent of flooding over the six-month period. Darker blue areas show
more intense flooding.

The model's prediction takes into account cloud cover and terrain
shadows, but has some limitations when estimating flooding in swamps,
marshland and coastal areas.

\textbf{Correction:}~Sept. 12, 2019

An earlier version of the methodology for this article described
incorrectly the source of the data. The article uses data from VIIRS
sensors on two satellites, not a VIIRS satellite. The methodology also
described incorrectly the model used to analyze the data. The model was
made available to the public, but parts of the underlying code were not
published; it is not the case that the model was open source.

Read 56 Comments

\begin{itemize}
\item
\item
\item
\item
\end{itemize}

Advertisement

\protect\hyperlink{after-bottom}{Continue reading the main story}

\hypertarget{site-index}{%
\subsection{Site Index}\label{site-index}}

\hypertarget{site-information-navigation}{%
\subsection{Site Information
Navigation}\label{site-information-navigation}}

\begin{itemize}
\tightlist
\item
  \href{https://help.nytimes3xbfgragh.onion/hc/en-us/articles/115014792127-Copyright-notice}{©~2020~The
  New York Times Company}
\end{itemize}

\begin{itemize}
\tightlist
\item
  \href{https://www.nytco.com/}{NYTCo}
\item
  \href{https://help.nytimes3xbfgragh.onion/hc/en-us/articles/115015385887-Contact-Us}{Contact
  Us}
\item
  \href{https://www.nytco.com/careers/}{Work with us}
\item
  \href{https://nytmediakit.com/}{Advertise}
\item
  \href{http://www.tbrandstudio.com/}{T Brand Studio}
\item
  \href{https://www.nytimes3xbfgragh.onion/privacy/cookie-policy\#how-do-i-manage-trackers}{Your
  Ad Choices}
\item
  \href{https://www.nytimes3xbfgragh.onion/privacy}{Privacy}
\item
  \href{https://help.nytimes3xbfgragh.onion/hc/en-us/articles/115014893428-Terms-of-service}{Terms
  of Service}
\item
  \href{https://help.nytimes3xbfgragh.onion/hc/en-us/articles/115014893968-Terms-of-sale}{Terms
  of Sale}
\item
  \href{https://spiderbites.nytimes3xbfgragh.onion}{Site Map}
\item
  \href{https://help.nytimes3xbfgragh.onion/hc/en-us}{Help}
\item
  \href{https://www.nytimes3xbfgragh.onion/subscription?campaignId=37WXW}{Subscriptions}
\end{itemize}
