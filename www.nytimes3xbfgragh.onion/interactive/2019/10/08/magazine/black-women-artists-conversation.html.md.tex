 **NYTimes.com no longer supports Internet Explorer 9 or earlier. Please
upgrade your browser.
\href{http://www.nytimes3xbfgragh.onion/content/help/site/ie9-support.html}{LEARN
MORE »}

**Sections

**Home

**Search

\hypertarget{the-new-york-times}{%
\subsection{\texorpdfstring{\href{http://www.nytimes3xbfgragh.onion/}{The
New York Times}}{The New York Times}}\label{the-new-york-times}}

\hypertarget{-magazine-}{%
\subsubsection{\texorpdfstring{
\href{https://www.nytimes3xbfgragh.onion/section/magazine}{Magazine}
}{ Magazine }}\label{-magazine-}}

 \href{https://www.nytimes3xbfgragh.onion/section/magazine}{Magazine}
\textbar{}`I Want to Explore the Wonder of What It Is to Be a Black
American'

**Close search

\hypertarget{site-search-navigation}{%
\subsection{Site Search Navigation}\label{site-search-navigation}}

Search NYTimes.com

**Clear this text input

Go

\url{https://nyti.ms/2ogB7GH}

\hypertarget{site-navigation}{%
\subsection{Site Navigation}\label{site-navigation}}

\hypertarget{site-mobile-navigation}{%
\subsection{Site Mobile Navigation}\label{site-mobile-navigation}}

\hypertarget{i-want-to-explore-the-wonder-of-what-it-is-to-be-a-black-american}{%
\section{`I Want to Explore the Wonder of What It Is to Be a Black
American'}\label{i-want-to-explore-the-wonder-of-what-it-is-to-be-a-black-american}}

Simone Leigh, Amy Sherald and Lorna Simpson talk about the expectations
faced by black women in an art world obsessed with identity.

\includegraphics{https://static01.graylady3jvrrxbe.onion/newsgraphics/2019/10/13/culture/1a9865dae9d1a0d21273eb2bff896ea8752f0273/close.svg}

\hypertarget{i-want-to-explore-the-wonder-of-what-it-is-to-be-a-black-american-1}{%
\section{`I Want to Explore the Wonder of What It Is to Be a Black
American'}\label{i-want-to-explore-the-wonder-of-what-it-is-to-be-a-black-american-1}}

Simone Leigh, Amy Sherald and Lorna Simpson talk about the expectations
faced by black women in an art world obsessed with identity.

A conversation with JENNA WORTHAM OCT. 8, 2019

\textbf{H}istorically speaking, work by black American artists has long
been overlooked or excluded from major American museums and galleries.
But recently, that position has started to shift, as institutions
reconsider their longstanding biases. A wave of black scholars and
academics are being appointed to curatorial roles, shaping (and
reshaping) exhibitions and landmark solo shows and revising collections
to honor the contributions of black artists. The heightened visibility
has also led to a commercial frenzy, including record auction prices.
Last year, for example, Sean Combs, the music producer,
\href{https://www.nytimes3xbfgragh.onion/2018/05/18/arts/sean-combs-kerry-james-marshall.html}{bought
the painting ``Past Times,''} by Kerry James Marshall, for \$21.1
million, which was the highest price ever paid for an artwork by a
living African-American artist.

On a warm September afternoon, three accomplished artists met at the bar
Ode to Babel in Brooklyn. With incense burning in the background, the
women gathered on soft leather couches with tea and snacks to discuss
their observations and reflections on this moment in time.

Simone Leigh, 51, is a sculptor whose work has been acquired by the
Solomon R. Guggenheim Museum in New York, among others. Her 16-foot
bronze figurative statue,
\href{https://www.nytimes3xbfgragh.onion/2019/04/04/arts/design/high-line-simone-leigh-brick-house.html}{``Brick
House,''} was the inaugural commission for the High Line Plinth in
Manhattan. She is currently in the running to make a memorial to replace
the Central Park statue of J. Marion Sims, a 19-century physician who
performed gynecological experiments on enslaved black women.

Amy Sherald, 46, a painter, was
\href{https://www.newyorker.com/culture/annals-of-appearances/the-mystery-of-amy-sheralds-portrait-of-michelle-obama}{selected
in 2017 by Michelle Obama to create her portrait} for the Smithsonian
Institution's National Portrait Gallery. In 2018 she was awarded the
David C. Driskell Prize for her contributions to art within the African
diaspora and was a recipient of the Pollock Prize for Creativity, which
honors outstanding artists whose work has an impact on society. In
September, she mounted her first New York solo show at Hauser \& Wirth.

\href{https://www.nytimes3xbfgragh.onion/2019/06/13/arts/design/lorna-simpson-paintings-hauser-wirth.html}{Lorna
Simpson}, 59, is an artist who has been showing her work since the
1980s. In 1990 she became one of the first African-American women to
exhibit at the Venice Biennale, and in September, she received a J. Paul
Getty Medal for her contributions to the art world.

\textbf{Jenna Wortham:} We're having this conversation right now because
it feels as if we're in a particular moment for black artists, and even
more specifically black female artists. I would love to talk about what
that moment is, exactly --- and how you see your position in it.

\textbf{Simone Leigh:} I do think the idea of who needs to be
represented and how is being reassessed in many different fields right
now, including art. The changes I see are across culture --- there was a
black photographer today, for example, and our moderator is black.
That's more normal than it would have been even a year ago in my
experience --- and it wasn't even a surprise to me.

\href{https://www.nytimes3xbfgragh.onion/interactive/2018/10/03/magazine/morality-social-justice-art-entertainment.html}{\emph{{[}Should
art be a battleground for social justice?{]}}}

\textbf{Amy Sherald:} Yes, it does feel like a moment. I think it'll be
easier to understand it when looking back than it is when you're in the
midst of it. My niece came up with her father to New York to see Serena
Williams and Coco Gauff and Naomi Osaka play in the U.S. Open, and I
realized that she's going to grow up never knowing there were no black
tennis players. She was born when Barack Obama was president, so
blackness is just a norm for her. It's different than it was for our
generation. We still walk into spaces, and we see black people, and we
high-five --- ``Yeah, we're here.'' We treat it as if it's an
achievement, and they will treat it as if it's a norm.

\textbf{Leigh:} There have been other recent landmark moments too: Faith
Ringgold, the 89-year-old, Harlem-born African-American artist, recently
had her first-ever European survey at the Serpentine Galleries in
London, and Lola Flash, the 60-year-old activist and photographer who
documented queer life during the AIDS crisis, was at Autograph, a London
art institution. Elizabeth Alexander, a black poet, was appointed
president of the Andrew W. Mellon Foundation. Rashida Bumbray, who is an
amazing black curator and choreographer, was appointed the first
director of culture and art at the Open Society Foundations, a global
philanthropic organization based in New York. Her reach and the projects
she'll be able to grant and fund are beyond what a museum curator can
do, and it's stunning.

The focus is also now turning to black feminist intellectuals, like
Saidiya Hartman, who writes about the afterlife of the trans-Atlantic
slave trade, and who was just awarded a MacArthur ``genius grant.'' So
many contemporary artists and scholars have been changed by the book
``In the Wake: On Blackness and Being,'' by Christina Sharpe, which
examines the legacy of black death in the context of chattel slavery.
Before, you would hear people talk about and quote Audre Lorde as if
black feminist thought were something in the past, and not happening
now. There has been some focus on black radical thought, but there has
never been a focus on black feminist theory, which seems to be starting
to happen. And I'm really happy about it.

\textbf{Lorna Simpson:} This is a moment, but I think this is also a
moment that has been in the making. In the late '80s and '90s,
commercial galleries did not focus on works by artists who were not
European or white. The Studio Museum in Harlem and Just Above Midtown
(JAM) were interested in cultivating and presenting work by artists that
you could not see in other venues. JAM showed the work of the black
conceptual artist David Hammons, and for a young artist like me, seeing
his work was an amazing, emboldening experience. The Studio Museum gave
so many young curators and artists opportunities to write and exhibit
work. The curators Kellie Jones and Okwui Enwezor provided the same kind
of support for exhibitions that we see happening right now. In order to
think about the artists working today, you also have to think about the
work of all these others who came before. Yes, this is an important
moment, but it reflects the previous changes that were made within
institutions. We have to also see this not just as a moment of
visibility for black artists but also one of historically white
institutions finally dragging themselves into the 21st century.

\textbf{Wortham:} I can see and feel that we're being fed in a different
way, and it's so nourishing. At the same time, visibility is really
complicated, because it's not just black people who are hungering for
these works and shows and images. There has always been a deep desire
for those outside the black American experience to consume it and to
rely on it for a larger understanding of the American narrative. Do you
feel that heightened interest and attention from white audiences, and
how does it impact how you think about your work and practice --- if at
all?

\textbf{Sherald:} Because I was a late bloomer, and started to think
about the work that I wanted to make when I was in my mid-30s, I was
looking at the conversations that we as black artists were having in the
2000s, and thinking about how I could participate in those
conversations. To make myself relevant, I had to think about what
conversation I was going to bring to the table. I didn't see any work
that was about stepping away from the public identity of blackness. When
I watched the movie ``Big Fish,'' an American fantasy drama, I realized
I'd never seen black narratives that had frivolity, that weren't heavy
and serious, that didn't show us running for our lives, struggling for
our lives. And I realized that I didn't want to make work that was
necessarily didactic, but instead was just about everyday blackness.

I think my work represents and connects to the journey that I'm on as a
woman and as a black woman, and as a Southern black woman, because
that's a separate identity as well, and trying to shed all these
external directives --- from social norms, from expectations from my
family, from my all-white school, from religion and popular culture ---
that influenced how I think about myself. My work was a way of
processing an identity that was given to me. Being told whom to worship,
what to wear, how to act in front of white people. All those were things
that were projected on me. A lot of my identity was performed. I
realized that I wanted to do away with all that, and then I tried to
figure out who I really was --- who am I outside of all of these things?

I think that's essentially what my paintings are about. A reflection of
something other than what's projected out in the world. I'm focused on
the way that we experience ourselves, our interiority. It's that private
journey, the interior space, and stepping away from the public journey
of blackness, and how people consume who we are, what we make, food,
culture and all of that.

Elizabeth Alexander just wrote a book called ``The Black Interior,'' and
there's a line that says, ``Tapping into this black imaginary helps us
envision what we are not meant to envision: complex black selves, real
and enactable black power, rampant and unfetishized black beauty.''
Kevin Quashie, a professor of black cultural and literary studies at
Brown, published a book about the power of stillness called ``The
Sovereignty of Quiet: Beyond Resistance in Black Culture.'' They gave a
vocabulary to my work --- I knew what I was making, but my language is
visual. They helped explain me to myself.

When it comes to audience, I don't think about the public generally, but
I do think a little bit about the black public, because I want my work
to be a gift to them. I was given such a gift in the sixth grade, when I
walked into a museum in Columbus, Ga., and saw the Bo Bartlett painting
``Object Permanence,'' which is a self-portrait of himself as a black
man, and experienced the power of seeing a painting of a person who
looked like me. I essentially built my career around that moment and
creating an inflection point for someone else.

\textbf{Simpson:} There is the business of art, and then there's the
making of it. I never had an agenda directed at an audience that I
needed to address in a particular way. Viewers don't always see what I
see in the work; therefore, I never felt that I could proceed in a way
that would anticipate what the viewers would need or not need. I have no
idea what people are going to feel or get from my work. For me, it has
been really important throughout my career to realize that I just make
the work that I make, and that's it. If the things that I experience in
life come into the making of the work, then that is enough.

\textbf{Wortham:} Simone, I'm curious to hear you speak on this. With
your work in particular, there are so many references embedded in the
show titles that seem to be speaking directly to an audience, like ``The
Waiting Room,'' a reference to a black woman named Esmin Elizabeth
Green, who died while waiting for treatment at Kings County Hospital
Center in New York in 2008. Or ``Loophole of Retreat,'' which refers to
a chapter in the book ``Incidents in the Life of a Slave Girl',' written
by Harriet Jacobs, an African-American woman who escaped enslavement by
spending seven years hiding in a tiny attic. To me, it feels like an
encoded system of links designed to draw viewers from point to point,
like a hidden map embedded within the work for black people to feel seen
and acknowledged, even if they are encountering the work in a space
filled with mostly nonblack people.

\textbf{Leigh:} When I would give talks or have speaking engagements,
like at the Creative Time Summit in Venice in 2015, I would announce
that I was making my work primarily for black women. And people thought
I was out of my gourd. Even my mentor, the civil rights activist and art
collector Peggy Cooper Cafritz, was like, ``You can't say that, what's
wrong with you?'' She was wanting to protect me. I didn't feel like I
was taking a big risk at that point. Being largely ignored for so many
years of my career gave me freedom to grow and be bold, to build
confidence around my thoughts. I told Peggy it was too late. I had
already been saying it for years. What I didn't expect was that it would
start to function as a call. It brought a lot of people into my life who
could see that I was trying to do something different. It helped me
build a community around myself. It's not an original idea. Toni
Morrison set a model for this mode of working, prioritizing black
audiences, and black women in particular. But in art, which is so
dependent on spectatorship, calling for a black audience seemed
outrageous.

\textbf{Wortham:} Did it deter commercial interest or turn white
collectors away from buying your work?

\textbf{Leigh:} It didn't impact their interest, but I was called out by
a collector during a talk once. Now I'm celebrated for it.

When we talk about black radical movements in history, women usually
fail to appear. I chose the title ``Loophole of Retreat'' for my
exhibition and also for a conference that I organized earlier this year
with Saidiya Hartman and Tina Campt, a feminist theorist of visual
culture and contemporary art, because it's a description of a point of
view of a woman who was incarcerated, while hiding in a cramped crawl
space above her master's house. Yet she recognized that she held a
certain kind of power, because she was able to observe so much. I was
thinking about the ways women have stepped out of line and resisted,
despite our de facto incarcerated states of being, and the way that has
not been widely recognized. The conference did not have a specific topic
--- it was an intellectual free-for-all. The point was to focus on our
agency. Our point of view from a position of power. I am not interested
at all in whiteness studies or describing racism and why it happens. I
was trying really hard to have the audience focus on these women as
authors, their scholarship, their contributions.

\textbf{Sherald:} That's making me think of when I do talks, and a white
person asks me why I don't paint white people, and I have to say ---

\textbf{Simpson:} That's a Toni Morrison question! Charlie Rose asked
her in the late '90s about her narratives and why she didn't include
white characters. Other interviewers asked her that question, too.
Watching that broadcast, I can see she positions herself brilliantly in
terms of understanding that you do not have to compromise your subject
because it pertains to African-Americans. There need not be an apology
or an explanation. I can't do it as brilliantly as Toni Morrison, but
she said you would never ask a male white writer that question about his
work. It's a question that was posed in the past and is being posed
again, and it's a question that has nothing to do with you or your work
as an artist. The question should be: Why are we still being asked this?

\textbf{Sherald:} I guess I shouldn't be surprised, but I am surprised
that some people don't understand the structures that are set in place
for them, the reflections of themselves that they take for granted.

\textbf{Leigh:} Yes, when people talk about statuary, and they say,
``There's too much of it, and we don't need any more of this kind of
art,'' they're not really considering who was represented in that
statuary.

\textbf{Sherald:} Like when you're walking up the street, and there are
sculptures of white men on horses. For others, that emboldens their
identity. But there's always a little part of me that doesn't feel good
enough. And I think it's subconsciously because of that lack of imagery.
So my response to those white people asking why I don't paint white
people is always: ``Because you guys have done a great job of
perpetuating your own image of yourself. If I can't pick my own ideal,
then who am I? Who am I to offer up a figure into art history that
doesn't look like myself?''

I'm beginning to search for iconic American photographs that compel us
to look at our country's biggest moments from a historical perspective
that felt incomplete to me. Like the figure of the man on the
construction beam in my painting ``If you surrendered to the air, you
could ride it'' --- a line from Toni Morrison's book ``Song of
Solomon.'' I was thinking about the Charles Ebbets photograph of a group
of men sitting on a beam in midair eating lunch taken during the
construction of Rockefeller Center. That building represents American
productivity and industry. And just adding his presence alone speaks
volumes.

\textbf{Leigh:} I've never heard a white artist be accused of their work
being too white. Not one time.

\textbf{Sherald:} For me, it comes down to painting the work that I want
to see exist in the world. I understand my power as a figurative painter
and someone who makes images. Because images are how we experience one
another. My paintings are also an expression of humanity. At the Paris
Exposition in 1900, W.E.B. Du Bois contributed hundreds of photographs
featuring everyday African-Americans to the American Negro exhibit, in
addition to charts, graphs and books intended to demonstrate the
development and progress of blacks at that time. He showed images of
black people in the South, to represent who we really are. The idea was
to combat racist propaganda.

It's important, because there's no neutrality to our existence. Being
black is still very political despite our highly nuanced existence. I
want to show that there isn't one kind of black identity.

When I went to college --- I went to Clark Atlanta University, a
historically black university --- I shaved my head. I started dressing
like a goth chick. I was having an experience that's outside of what
people would consider the black experience. When I found my first model
--- she embodied her own identity. She was wearing vintage clothes, had
an Afro and was six feet tall. She was not controlled by what people
thought a black person should be. Those were the experiences of black
identity that I was trying to expound upon. We are all different kinds
of people, and I'm trying to capture that in my work. I want to explore
the wonder of what it is to be a black American. And the complexity of
that identity. I want the everydayness of the people to psychologically
stay with you and change how you interact with others.

\textbf{Wortham:} And how do you imagine the work transforming those who
encounter it?

\textbf{Sherald:} I think about those seventh-grade black students on a
field trip to the Museum of Fine Arts in Boston who were harassed by
visitors and museum staff. People have ideas about what blackness is,
and they were projecting it onto these kids. I would like to think that
because my images will hang in places like that, they might effectuate
something positive by their presence alone. We as black people still
have the power to shift spaces.

\textbf{Simpson:} Can I make a comment? This is kind of the elephant in
the room. Amy, you painted Michelle Obama, royalty in America. That is a
seminal work of an iconic person. The two portraits of the Obamas ---
the president was painted by Kehinde Wiley --- are the first paintings
of their sort that will exist at the National Portrait Gallery in
Washington. There was a lot of weight on that painting: exterior
expectations of what a portrait of Michelle Obama should be, the
historical importance of the presidency and of her as a first lady. With
those pressures in mind, it takes a certain amount of bravery to make
what you want to make and not just go the route of people's
expectations. On every freaking level, that is unusual. You're giving
people the opportunity to see differently than is expected, both in the
color palette you use and the context of the other paintings in the
hall. And it is quite wonderful.

\textbf{Sherald:} Thank you. And maybe that's why in my portraits I
paint the skin gray. Maybe that's so that viewers don't necessarily
immediately think about race. I have had conversations with black people
who say that they experience themselves differently when they look at
the work. There was a sense of ownership that comes from having someone
like Michelle Obama in a space like the National Portrait Gallery. I'm
thinking about Parker Curry, the little girl who was photographed
looking at the painting. For Michelle, yeah, I wanted to represent her
interiority. It's really about filling these historical gaps, and
offering a corrective narrative.

\textbf{Simpson:} See, that's kind of cool though.

\textbf{Sherald:} I make these images of things that we normally do but
we don't get to see in spaces like museums. Like black people going to
the beach in the painting ``Precious Jewels by the Sea,'' which shows a
sunny day with two couples in their late teens having a picnic and
enjoying themselves against a broad blue sky. It makes me think about my
mother. She didn't know how to swim, and she didn't like going to the
segregated black beaches because she wanted to go to the prettier white
beaches. It makes me think about how much things have changed
generationally. My friends' kids go to Martha's Vineyard every year.
They see that, and it's totally normal for them. So it's really just
about creating American narratives about American people --- while
critiquing it at the same time. This painting, and all of them, are
about expressions of freedom.

\textbf{Wortham:} Did any of that resonate for you, Simone? Are you
working out questions of selfhood and identity in the construction of
your sculptures?

\textbf{Leigh:} I make figurative sculpture, but I do think of it in
more formal terms, in that I'm not doing portraiture or representing
anyone in particular --- but maybe many people, maybe a state of being.
In 2016, for the ``Anatomy of Architecture'' series at the Hammer Museum
in Los Angeles, I was working in a tradition of thinking about the
status of women. By associating the body with the idea of a dwelling,
refuge, container, tool, even loophole of retreat. I'm working in a long
tradition of associating women's bodies with architecture or vessels
like Louise Bourgeois's ``Femme Maison,'' Judy Chicago and Miriam
Schapiro's ``Womanhouse'' and Magdalene Odundo's pots. I'm also drawing
on traditions of building in Africa, specifically Mousgoum architecture
in Cameroon.

\textbf{Wortham:} Something that comes up frequently in your work,
Simone, is networks of care, the well-being of black women and the
histories of activism around wellness in black communities.

\textbf{Leigh:} When I did the ``Free People's Medical Clinic,'' a 2014
exhibition in Weeksville --- a Brooklyn community established by free
blacks in 1838 --- that offered workshops in herbalism and free H.I.V.
and blood-pressure screenings, I discovered this group called the United
Order of Tents. It was a secret society started in 1867 by formerly
enslaved women, and I started to see a real value in not making
everything visible. Most artists are trying to get as much visibility
for their work as possible. I wanted to think about no visibility as a
strategy. There are so many examples in history where people have had to
go underground, because it was unsafe to be public. After Nat Turner's
rebellion, churches were banned in Charleston, S.C. --- so for more than
30 years, Mother Emanuel Church, one of the oldest black churches in
this country, met underground until after the Civil War. I like the
strategy of working in that way, so I did a New Museum project called
``Home Economics'' with Aimee Meredith Cox and Karen Rose. It involved
15 girls. No one saw this work; it happened on the days the museum was
closed.

\textbf{Wortham:} Lorna, how have themes of identity and expression
shifted in your work over the years?

\textbf{Simpson:} As Amy and Simone are saying, there is this kind of
sociopolitical side of the lives that we lead every day, in terms of the
way people interact and speak and expectations and lack of expectations.
But at the same time, there's this whole interior world to explore. My
recent paintings that were shown at Hauser \& Wirth earlier this year
come out of the experience of living in America. I was weaving figures
and characters into a harsh and unforgiving landscape that is blue and
icy, and it looks vacant and beautiful but somewhat terrifying. I feel
like we're all doing that in different ways, mapping our experience of
being American and living in America and figuring out what identity is.

The project of my earlier work, from the '80s and the '90s, was really a
project about interrogating identity. I liked to present the backs of
figures, because it represented a refusal of engagement with a
quote-unquote audience. You were confronted with this image that seemed
kind of like a portrait, although you couldn't see who it was, which
creates a certain amount of distance. In one image from 1986, a woman
has on a shift dress, and she's holding a silver jug in one hand and a
plastic container in the other, and water is pouring out of them, and
underneath there's a brief proselike narrative about disappearances and
being believed and telling your account: ``She saw him disappear by the
river, they asked her to tell what happened, only to discount her
memory.'' The text was meant to confuse or be in contradiction with the
image, in a way. That's actually what we're currently struggling with in
this country: that there is no truth, or that the truth can be
dismissed.

There's another piece called ``Wigs'' that is talking about gender and
sexuality. I went out and bought as many different wigs as I could to
talk about the construction of identity, race, sexuality and gender,
hung them and photographed them. It was 1994, and there was so much
societal pressure toward conformity around language and lifestyle. There
are texts that refer to how the state controls us, with dress codes and
registering our ``true'' gender at the time of death. The use of ``he''
and ``she'' and the societal appetite for black and white confirmations.
``Wigs'' questions all of that, and shows how language is a trap, and
how the imposition of societal language can upend the lives of people.
``Wigs'' was a deconstruction of identity, a way to make it more open
--- no one and nothing should have to fit into one way of speaking or
identifying.

But the way ``Wigs'' was reviewed was like, ``Oh, it's about the way
black women wear their hair.'' But there was so much more there than
this general narrative about blackness --- there are also very specific
things about \emph{humanness.} I am making the assumption, much like the
way Simone and Amy do in their work, that the presentation of a black
subject is not inhibiting in any way regardless of the audience. Very
early on I understood that. That I had to take a stance, that if Kiki
Smith can have a flayed-body sculpture whose skin is Caucasian, and I am
supposed to resonate with that work and understand why she is doing
that, then I am expecting the same thing for me and my work. That just
as the Caucasian figure in contemporary art is seen as universal, the
black figure of African descent should be, too. It should not be seen
only as something related to the black experience or be described as an
``urban'' piece.

\textbf{Wortham:} We started off this conversation talking about this
particular moment in the art world. But how are you each thinking about
longevity and the legacy you'd like to leave in the world?

\textbf{Simpson:} I have never really thought of my career as something
that I'm planning out in decades. It's a far more interesting life to
not know what the next thing is going to be, exactly. I love changing my
mind. And I love doing things that may seem unexpected from the outside.
So I think the legacy of the work is just to continue making work the
way that I choose. And now I have the great privilege of being able to
do whatever I want.

\textbf{Sherald:} I've always felt that a legacy was sealed as soon as
you have work being acquired by museum institutions. That's the
beginning of the work living its life throughout art history. Also, I'm
inspired by artists who have taken their success and utilized it in a
way that creates legacy outside of their work. Mark Bradford opened Art
\& Practice, an art and community space in Los Angeles. Kehinde Wiley
established an artist residency in Senegal. I'd like to do something
along those lines.

\textbf{Leigh:} I think that my most important legacy right now is the
long-term ongoing relationships I have with mentors, mentees and other
artists and intellectuals. For example without the scholarship of
Sharifa Rhodes-Pitts, who writes about black utopias, I would cease to
exist. Also, the Smithsonian's Archives of American Art is collecting my
letters. And so I've had to think, What are my letters? I'm like, Text
messages? Email?

But what I'm hoping is that in the course of art history, I made a
contribution, and that people recognize it. Mostly, I've appreciated the
opportunity to make my work. That's really all I ever wanted.

\emph{This conversation has been edited and condensed.}

The Culture Issue

\begin{itemize}
\tightlist
\item
  \href{https://www.nytimes3xbfgragh.onion/interactive/2019/10/09/magazine/tyler-perry-black-theater.html}{Black
  Theater Is Having a Moment. Thank Tyler Perry. (Seriously.)}
\item
  \href{https://www.nytimes3xbfgragh.onion/interactive/2019/10/09/magazine/PewDiePie-interview.html}{What
  Does PewDiePie Really Believe?}
\item
  \href{https://www.nytimes3xbfgragh.onion/interactive/2019/10/09/magazine/kathryn-hahn-mrs-fletcher.html}{Kathryn
  Hahn's Funny, Sensual Portrayals of Female Desire}
\item
  \href{https://www.nytimes3xbfgragh.onion/interactive/2019/10/08/magazine/black-women-artists-conversation.html}{`I
  Want to Explore the Wonder of What It Is to Be a Black American'}
\item
  \href{https://www.nytimes3xbfgragh.onion/interactive/2019/10/08/magazine/rosalia-flamenco.html}{Rosalía's
  Incredible Journey From Flamenco to Megastardom}
\item
  \href{https://www.nytimes3xbfgragh.onion/interactive/2019/10/08/magazine/ben-lerner-topeka-school.html}{To
  Decode White Male Rage, First He Had to Write in His Mother's Voice}
\item
  \href{https://www.nytimes3xbfgragh.onion/interactive/2019/10/08/magazine/susan-sontag.html}{How
  Susan Sontag Taught Me to Think}
\item
  \href{https://www.nytimes3xbfgragh.onion/interactive/2019/10/09/magazine/moma-reopening.html}{Backstage
  at the Modern}
\end{itemize}

\protect\hyperlink{}{} \protect\hyperlink{}{}

\includegraphics{https://static01.graylady3jvrrxbe.onion/newsgraphics/2019/10/13/culture/1a9865dae9d1a0d21273eb2bff896ea8752f0273/caret.svg}

\hypertarget{more-culture}{%
\subsection{More Culture}\label{more-culture}}

\begin{itemize}
\tightlist
\item
  \href{https://www.nytimes3xbfgragh.onion/interactive/2019/10/09/magazine/tyler-perry-black-theater.html}{}
\item
  \href{https://www.nytimes3xbfgragh.onion/interactive/2019/10/09/magazine/PewDiePie-interview.html}{}
\item
  \href{https://www.nytimes3xbfgragh.onion/interactive/2019/10/09/magazine/kathryn-hahn-mrs-fletcher.html}{}
\item
  \href{https://www.nytimes3xbfgragh.onion/interactive/2019/10/08/magazine/rosalia-flamenco.html}{}
\item
  \href{https://www.nytimes3xbfgragh.onion/interactive/2019/10/08/magazine/ben-lerner-topeka-school.html}{}
\item
  \href{https://www.nytimes3xbfgragh.onion/interactive/2019/10/08/magazine/susan-sontag.html}{}
\item
  \href{https://www.nytimes3xbfgragh.onion/interactive/2019/10/09/magazine/moma-reopening.html}{}
\end{itemize}

SOLOMON

\hypertarget{more-on-nytimescom}{%
\subsection{More on NYTimes.com}\label{more-on-nytimescom}}

Advertisement

\hypertarget{site-information-navigation}{%
\subsection{Site Information
Navigation}\label{site-information-navigation}}

\begin{itemize}
\tightlist
\item
  \href{https://help.nytimes3xbfgragh.onion/hc/en-us/articles/115014792127-Copyright-notice}{©
  2020 The New York Times Company}
\item
  \href{https://www.nytimes3xbfgragh.onion}{Home}
\item
  \href{https://www.nytimes3xbfgragh.onion/search/}{Search}
\item
  Accessibility concerns? Email us at
  \href{mailto:accessibility@NYTimes.com}{\nolinkurl{accessibility@NYTimes.com}}.
  We would love to hear from you.
\item
  \href{https://help.nytimes3xbfgragh.onion/hc/en-us/articles/115015385887-Contact-Us}{Contact
  Us}
\item
  \href{https://www.nytco.com/careers/}{Work with us}
\item
  \href{https://nytmediakit.com/}{Advertise}
\item
  \href{https://help.nytimes3xbfgragh.onion/hc/en-us/articles/115014892108-Privacy-policy\#pp}{Your
  Ad Choices}
\item
  \href{https://help.nytimes3xbfgragh.onion/hc/en-us/articles/115014892108-Privacy-policy}{Privacy}
\item
  \href{https://help.nytimes3xbfgragh.onion/hc/en-us/articles/115014893428-Terms-of-service}{Terms
  of Service}
\item
  \href{https://help.nytimes3xbfgragh.onion/hc/en-us/articles/115014893968-Terms-of-sale}{Terms
  of Sale}
\end{itemize}

\hypertarget{site-information-navigation-1}{%
\subsection{Site Information
Navigation}\label{site-information-navigation-1}}

\begin{itemize}
\tightlist
\item
  \href{https://spiderbites.nytimes3xbfgragh.onion}{Site Map}
\item
  \href{https://help.nytimes3xbfgragh.onion/hc/en-us}{Help}
\item
  \href{https://help.nytimes3xbfgragh.onion/hc/en-us/articles/115015385887-Contact-Us?redir=myacc}{Site
  Feedback}
\item
  \href{https://www.nytimes3xbfgragh.onion/subscription?campaignId=37WXW}{Subscriptions}
\end{itemize}
