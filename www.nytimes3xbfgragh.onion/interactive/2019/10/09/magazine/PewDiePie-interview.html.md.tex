 **NYTimes.com no longer supports Internet Explorer 9 or earlier. Please
upgrade your browser.
\href{http://www.nytimes3xbfgragh.onion/content/help/site/ie9-support.html}{LEARN
MORE »}

**Sections

**Home

**Search

\hypertarget{the-new-york-times}{%
\subsection{\texorpdfstring{\href{http://www.nytimes3xbfgragh.onion/}{The
New York Times}}{The New York Times}}\label{the-new-york-times}}

\hypertarget{-magazine-}{%
\subsubsection{\texorpdfstring{
\href{https://www.nytimes3xbfgragh.onion/section/magazine}{Magazine}
}{ Magazine }}\label{-magazine-}}

 \href{https://www.nytimes3xbfgragh.onion/section/magazine}{Magazine}
\textbar{}What Does PewDiePie Really Believe?

**Close search

\hypertarget{site-search-navigation}{%
\subsection{Site Search Navigation}\label{site-search-navigation}}

Search NYTimes.com

**Clear this text input

Go

\url{https://nyti.ms/312Im2e}

\hypertarget{site-navigation}{%
\subsection{Site Navigation}\label{site-navigation}}

\hypertarget{site-mobile-navigation}{%
\subsection{Site Mobile Navigation}\label{site-mobile-navigation}}

\hypertarget{what-does-pewdiepie-really-believe}{%
\section{What Does PewDiePie Really
Believe?}\label{what-does-pewdiepie-really-believe}}

The biggest YouTuber in the world has been accused of being a closet
white nationalist and even inspiring mass shootings. He says it's all a
misunderstanding.

\includegraphics{https://static01.graylady3jvrrxbe.onion/newsgraphics/2019/10/13/culture/1a9865dae9d1a0d21273eb2bff896ea8752f0273/close.svg}

\hypertarget{what-does-pewdiepie-really-believe-1}{%
\section{What Does PewDiePie Really
Believe?}\label{what-does-pewdiepie-really-believe-1}}

The biggest YouTuber in the world has been accused of being a closet
white nationalist and even inspiring mass shootings. He says it's all a
misunderstanding.

By \href{https://www.nytimes3xbfgragh.onion/by/kevin-roose}{KEVIN ROOSE}
OCT. 9, 2019

\textbf{O}ne day last November, Mitch Japczyk, an administrator at an
Illinois staffing agency, was called upon to help solve an office
mystery. A handful of his co-workers were huddled around the office
printer, where a one-page document had just printed itself out,
unprompted. Nobody in the office knew what it meant.

``ATTENTION!'' the document began. ``PewDiePie is in trouble and he
needs your help to defeat T-Series!'' It went on to explain that
PewDiePie, a 29-year-old Swedish YouTube personality, was in danger of
being overtaken as the platform's most popular channel by T-Series, an
Indian music label and Bollywood production studio. In order to prevent
this from happening, the author recommended five steps:

1. Unsubscribe from T-Series.

2. Subscribe to PewDiePie.

3. Share awareness to this issue. \#SavePewDiePie.

4. Tell everyone you know. Seriously.

5. BROFIST!

Japczyk, now 33, spends more time on YouTube than many of his colleagues
and was able to explain the broad strokes: PewDiePie (rhymes with
``cutie pie''; real name: Felix Kjellberg) was a YouTube megacelebrity
with a group of hard-core fans, known as the Bro Army, who often went to
extreme lengths to show their support. But it was only when Japczyk got
back to his computer that he realized the scale of the operation. A
hacker was
\href{https://www.theverge.com/2018/11/30/18119576/pewdiepie-printer-hack-t-series-youtube}{claiming
responsibility} for finding a set of 50,000 printers all over the world
with unsecured network connections, taking them over and using them to
print these fliers.

The printer hack wasn't an isolated incident. For months, tributes to
PewDiePie had been popping up all over the world. Billboards and fliers
appeared in India, Bangladesh and Times Square. Early this year,
hundreds of Estonian fans held a parade in the nation's capital city,
chanting his name and holding signs that said ``Sub 2 Pewds.'' Someone
chalked ``Subscribe to PewDiePie'' on a World War II memorial in
Brooklyn.

Like Japczyk's colleagues at the staffing agency, many people who
witnessed these events most likely wrote them off as dumb stunts or
random emanations from some Gen Z media universe they weren't plugged
into --- which, of course, they were. But as the ``Subscribe to
PewDiePie'' movement grew, its meaning got blurrier. Some people were
cheering for PewDiePie because they liked his videos. Other people saw
him as the flag-bearer of an older, weirder internet culture that was
being steamrollered by bland corporate interests and needed to be
defended. And a few had much darker motives.

On March 15, a white nationalist in Christchurch, New Zealand, said,
``Remember, lads, subscribe to PewDiePie'' on his Facebook live stream,
just before going on a
\href{https://www.nytimes3xbfgragh.onion/2019/03/15/technology/pewdiepie-new-zealand-shooting.html}{shooting
rampage}, killing dozens of worshipers inside two mosques. A few weeks
later, another violent white nationalist cited PewDiePie, this time in
writing. In his manifesto, the suspect --- accused of killing one woman
and injuring three others at a synagogue in Poway, Calif. --- even
claimed that the YouTuber aided in the plot. ``I had the help of a man
named Felix Arvid Ulf Kjellberg,'' wrote the shooting suspect, who, like
the Christchurch suspect, has pleaded not guilty. ``He was kind enough
to plan and fund this whole operation --- the sly bastard.''

Just to be abundantly clear: As far as we know, these were both sick
jokes. Kjellberg has no connection to either man, and there is zero
evidence that their deeds were actually inspired by him. But he was not
a randomly chosen target either. For years, Kjellberg has been trying,
fairly unsuccessfully, to shed a reputation as a far-right sympathizer.
Partly, that's because a few years ago, he made some Holocaust jokes on
\href{https://www.youtube.com/channel/UC-lHJZR3Gqxm24_Vd_AJ5Yw}{his
channel}, which led to a nuclear-grade backlash. And partly, it's
because Kjellberg responded to being called out for those jokes, and
other offensive statements, the way a reactionary might: by mocking his
critics, casting himself as the victim of a media smear campaign and
refusing to back down.

\subsection{}

Modern right-wing extremism is wrapped in so much irony that it's hard
to know what motivates any particular adherent. The suspects in the
shootings in Christchurch and Poway might have been casually trolling
PewDiePie by mentioning him in the context of mass murder --- knowing
that it would set off a media frenzy and stall his campaign to
rehabilitate his image. They might have been trying to bait reporters
into wrongly blaming PewDiePie for the killings. (Which would make those
reporters look gullible and out of touch, thereby proving that PewDiePie
was right about the media.) They might have just been chaos-loving
nihilists. A lot of internet culture exists in this frustrating quantum
state --- things are either total jokes or total nonjokes, depending on
their context and your vantage point.

A few weeks after the Poway shooting, Kjellberg's publicist called me.
Kjellberg hadn't given an interview in years --- a guy with millions of
YouTube subscribers has little need for reporters --- but in the wake of
the shootings, I had asked, and he agreed to talk. People were accusing
him of supporting white nationalism again, and he wanted to explain why
they were mistaken. I flew to Brighton, the seaside town in Britain
where Kjellberg has lived since 2013, and we met at an Airbnb near his
house. He showed up a few minutes late, popping out his earbuds to greet
me.

``Here we are,'' he said.

``Here we are,'' I agreed.

In his YouTube videos, Kjellberg is a spring-loaded ball of manic energy
--- he screams, he curses, he cracks himself up. But in person, he was
withdrawn and polite, with the stiff body language of a job applicant.
He seemed eager to make a good impression, or at least to appear
nonthreatening.

``I'm happy to have this opportunity,'' he said unconvincingly.

\textbf{One crucial thing} to understand about YouTube is that there are
really two of them. The first YouTube is the YouTube that everyone knows
--- the vast reference library filled with sports highlights, music
videos and old Comedy Central roasts. But there's a second YouTube
inside that one. It is a self-contained universe with its own values and
customs, its own incentive structures and market dynamics and its own
fully developed celebrity culture that includes gamers, beauty vloggers,
musicians, D.I.Y.ers, political commentators, artists and pranksters.
The biggest of these personalities have millions of subscribers and
Oprah-level influence over their fandoms. Many Inner YouTubers never
watch TV and develop elaborate parasocial bonds with their favorite
creators. For people who frequent Inner YouTube --- generally people
under 25, along with some older people with abundant free time --- the
site is not just a video platform but a prism through which all culture
and information is refracted.

I started hanging out on Inner YouTube in earnest a few years ago, and
its scale and insularity was jarring at first. Imagine a genetic
mutation that gave everyone born after 1995 the ability to see
ultraviolet light. Imagine that these people developed an identity
around UV light, started calling themselves ``UVers'' and became
suspicious of any media product made exclusively on the visible
spectrum. As an old person with normal eyes, you would experience this
change as a kind of slow cognitive decline. Every day, as more and more
of the world played out in UV, you would struggle to catch glimpses of
it. All of a sudden, people would be talking about Area 51 or eating
Tide Pods, and you'd have no idea why. This deep chasm of understanding
between Inner YouTube and the rest of the world has proved to be the
defining problem of Kjellberg's career.

When Kjellberg started his channel in 2010, YouTube culture hardly
existed. He was a 21-year-old college student in Gothenburg, Sweden, who
liked playing video games in his apartment. Eventually he took his game
footage, superimposed some running commentary in the corner and started
uploading it to YouTube --- early examples of a genre that became known
as ``let's play.'' Thanks to some combination of goofy charisma and
algorithmic luck, Kjellberg's channel blew up, in a way no YouTube
channel ever had. In 2012, he hit a million subscribers. The following
year, PewDiePie became the biggest channel on the site, and Kjellberg
started making serious money --- a reported \$4 million from ads on his
channel --- and was involved in various lucrative offshoot projects,
including a deal with the Disney-owned Maker Studios. In 2014,
\href{https://variety.com/2014/digital/news/survey-youtube-stars-more-popular-than-mainstream-celebs-among-u-s-teens-1201275245/}{a
survey} found that PewDiePie was more popular with American teenagers
than Katy Perry, Johnny Depp or Leonardo DiCaprio.

Despite the fact that PewDiePie's audience had grown larger than that of
any late-night talk show, many mainstream outlets still treated him like
an exotic animal. Journalists wrote breathless stories about his
earnings (sample 2014 headline: ``This Guy Makes Millions Playing Video
Games on YouTube. What?!''), and late-night hosts puzzled over his mass
appeal (``Why do you think people like it so much, watching you play
games?'' Stephen Colbert asked him on ``The Late Show''). He appeared on
several episodes of ``South Park,'' where the running gag was that
nobody could understand what he did for a living.

Inside YouTube, PewDiePie was a prized talent. The company put up
PewDiePie posters in its headquarters and gave him a starring role in
YouTube Rewind, the company's year-in-review video. (Rewind has become a
kind of annual report card for Inner YouTube's top creators, who analyze
it frame by frame, looking for hints about their place in the platform's
pecking order.) During these years, PewDiePie was not just the YouTuber
with the biggest channel. To many Inner YouTubers, he represented the
values of the platform --- lo-fi, authentic, defiantly weird. In 2016,
when his channel became the first in history to hit 50 million
subscribers, YouTube commissioned
\href{https://www.youtube.com/watch?v=7Vj5M0qKh8g}{an enormous
ruby-colored statue of the ``brofist''} --- he signs off most videos by
raising a clenched hand to the webcam --- and mailed it to him.

Kjellberg unboxed it in his kitchen with his camera rolling and seemed
genuinely moved.

``YouTube really has given me everything,'' he said.

\textbf{Years ago, YouTube} embarked on a radical experiment in
self-governance. In 2012, the company
\href{https://www.theverge.com/2012/4/13/2945243/youtube-partner-program-monetization}{vastly
expanded} the number of creators it allowed to make money from ads on
their channels, provided they stayed within some loose boundaries of
taste. This made YouTube unique among social platforms, in that it was
possible for popular creators to earn a full-time living directly from
the platform. YouTube built an algorithm that recommended those
creators' videos based on how engaging they were. Then it stepped back,
let the machines run and let a thousand media moguls bloom.

So Kjellberg's relationship with YouTube has always been a two-way
street. His videos brought the company lucrative advertising revenue and
a steady stream of loyal users at a time when it was fending off
competition from Facebook, Netflix and other video platforms. In
exchange, the company promoted his channel and turned a blind eye to
some of his more erratic behavior --- as in 2012, when he was criticized
for his habit of referring to beating his video-game opponents as
``raping'' them, or in 2016, when he was temporarily banned by Twitter
after joking that he was joining ISIS. Kjellberg called these mistakes
his ``oopsies,'' and he knew they were easily fixed with an apology. At
the time, YouTube took pride in being a creator's utopia, and PewDiePie
was hardly the only one pushing the limits. Gamers with David Duke
vocabularies were everywhere, and a nascent group of right-wing
reactionaries was beginning to learn that skewering political
correctness was a ticket to YouTube virality. But so was a more
garden-variety brand of shock humor, carefully calibrated to the
sensibilities of teenage boys.

Edgelords --- people who post offensive things online for attention ---
had always existed on message boards like 4chan. But YouTube brought
them out of the shadows and turned provocation into a viable career
path. On YouTube, there were few rules and no lawyers looking over
creators' shoulders --- which is precisely why millions of young people
went there, to find the kind of stuff they couldn't get on TV. The
platform's algorithms promoted engaging videos, with little regard for
what made them engaging, and showered ad revenue on the most successful
channels. And as all kinds of boundary-pushers raced to fill this void,
it became harder to tell who had an actual ideology and who was just
feeding the machines what they wanted.

Kjellberg knew plenty of edgelords --- he was a gamer, after all --- but
he never considered himself one of them. Sure, he cursed and shouted
while playing video games, but that was normal behavior. Of the hundreds
of videos he posted every year, most were solidly PG-13. Around 2015,
though, he began to take more risks. He continued playing video games,
but he started experimenting. He did viral challenges, made fun of other
YouTubers and reviewed meme submissions from his fans. His video titles
from that period sound like a jumbled set of X-rated refrigerator-poetry
magnets:

``SEX WITH YOUR CAR SIMULATOR''

``FIST ME DADDY''

``FINGERING SHREK''

``TOILET EXPLOSION DEATH???''

Today Kjellberg attributes this period to a combination of immaturity,
boredom and YouTube's platform incentives --- which encouraged creators
to increase their watch time by doing outrageous things. He says that he
grew sick of playing video games and that his channel's growth had
plateaued, which gave him the urge to let loose. ``Looking back, it was
a bubble waiting to burst --- this bubble of, how far can we push
this?'' Kjellberg told me. ``I think YouTube at that time was at a place
where no one really knew where the limit was.''

But Kjellberg did know where the limit was, because he had already
started breaching it. In early 2015, he nearly blew up his Maker Studios
deal over a video in which he read erotic fan fiction starring
characters from the hit Disney movie ``Frozen.'' According to a person
with knowledge of the incident, Bob Iger, Disney's chief executive, was
very upset when he found out that a YouTuber employed by one of his
subsidiaries was cackling over pornographic depictions of Elsa and Olaf
the Snowman. Iger calmed down after a Maker Studios staff member
explained that erotic fan fiction was a popular internet genre. (A
spokeswoman for Disney said that Iger found the video to be ``in poor
taste.'') Another former colleague of Kjellberg's told me about an
incident that happened in November 2016. Kjellberg arrived at a taping
of ``Scare PewDiePie'' --- a show he was making for YouTube's premium
video service --- in a T-shirt with a swastika drawn on it. Kjellberg
explained that the shirt was part of a running joke on his vlog, in
which he and his friends, at least one of whom was Jewish, pranked one
another by drawing things on one another's shirts. The colleague
intervened, telling Kjellberg that it would be irresponsible to wear a
Nazi symbol on a show that millions of young people would see. (Videos
posted by Kjellberg during this period show him and a friend wearing
shirts with swastikas drawn on them, but Kjellberg denied wearing one on
the show's set.)

I talked to roughly a dozen people who knew or collaborated with
Kjellberg during this period, and none thought he was a genuine
anti-Semite. Some theorized that Nazi jokes were just the most offensive
type of humor he and his friends could imagine, and therefore the
funniest. They said he didn't understand the power he had as YouTube's
most popular creator, or what millions of impressionable kids might make
of a blond, blue-eyed European joking about Hitler. Even today,
Kjellberg --- \href{https://www.youtube.com/watch?v=DYlesHOaPkY}{whose
100 million subscribers} would collectively form the 14th-most-populous
nation on earth --- has a hard time making sense of his influence. In
his mind, he's just a normal guy who happened to get famous.

``My job is just: I go to my office; I record a video in front of a
camera,'' he told me. He then glanced down at the microphones sitting on
the table. ``It's weird for me to be in this position, because I don't
really want to be in this position.''

\textbf{In February 2017,} Kjellberg learned that The Wall Street
Journal was preparing to run an article about nine videos, posted on his
channel over the course of six months, that contained anti-Semitic jokes
or Nazi imagery. Kjellberg knew that the videos were controversial, and
he did some quick damage control. In a
\href{https://pewdie.tumblr.com/post/157160889655/just-to-clear-some-things-up}{post
on his Tumblr}, he explained, ``I think of the content that I create as
entertainment, and not a place for any serious political commentary.''
His jokes, he said, were ``in no way supporting any kind of hateful
attitudes.''

Some of the videos
\href{https://www.wsj.com/articles/disney-severs-ties-with-youtube-star-pewdiepie-after-anti-semitic-posts-1487034533}{The
Journal featured} were clearly less offensive in context, like the one
in which he told his fans to stop building swastikas in a video game.
But other jokes --- like a video in which he tested the limits of what
people on the gig website Fiverr would do for money, including paying
two Indian men to hold up a sign that read ``Death to All Jews'' ---
were harder to explain. A few of these videos captured the attention of
established anti-Semitic groups, who speculated that Kjellberg wasn't
totally kidding and might be converting to their cause. The Daily
Stormer, a neo-Nazi hate site, was so hopeful that it changed its
tagline to ``the world's \#1 PewDiePie fan site.'' Today, Kjellberg
blames himself for stepping over the line, but he also characterizes
what happened as a kind of category error --- a group of outsiders,
blind to UV light, who mistook his trolling for genuine hate. Of course,
even people who understood that he was trolling found it irresponsible.
As a writer on the video-game website
\href{https://www.polygon.com/2017/2/15/14610652/pewdiepie-versus-the-media-disney-youtube-google}{Polygon
put it}, ``Intent only gets you so far when it comes to toying with hate
speech in front of an audience of tens of millions, many of whom are
younger children.''

The fallout from The Journal's article
\href{https://www.nytimes3xbfgragh.onion/2017/02/14/business/pewdiepie-youtube-disney.html}{was
swift and brutal}. Maker Studios ended its partnership with Kjellberg,
and YouTube canceled his ``Scare PewDiePie'' series and dropped him from
a V.I.P. advertising program. By the end of the day, Kjellberg --- who
was on a Valentine's Day getaway with his longtime girlfriend, Marzia
Bisognin, when the article went online --- was in danger of losing his
whole empire. ``I spent the day being in this little cottage with no
internet,'' he said. ``And then I go on Twitter, and there's J.K.
Rowling calling me a fascist, and I'm like: `How is this happening? This
is crazy.' ''

A few days later, Kjellberg posted an angry, defensive video in which he
vacillated between self-deprecation and grandiosity, accusing The
Journal of punching down at a ``rookie comedian'' seconds before saying
things like, ``Old-school media does not like internet personalities
because they're scared of us.'' He recast The Journal's article as a
conspiracy, implying that the jealous, dishonest mainstream media had
ginned up a fake controversy in order to take him (and by extension, all
of YouTube) down a peg. By turning attention to the news media,
Kjellberg found a fight he could win. Mainstream outlets had, in fact,
botched some of their early coverage of YouTube culture, and YouTubers
were inherently suspicious that the media establishment saw them as
competition.

PewDiePie felt besieged, even though his audience dwarfed that of any
mainstream media outlet --- and the Bro Army agreed. For weeks, they
barraged The Journal's reporters with harassment and threats, and even
dug up off-color jokes one reporter made years earlier on Twitter.
Kjellberg promptly featured the tweets in a video; afterward, the
reporter received so many death threats that The Journal offered to
briefly move him out of his house. The Bro Army also took aim at
YouTube, which fans believed was siding with the media and punishing
PewDiePie in response to outside pressure. When the 2017 version of
YouTube Rewind came out, he was nowhere to be found.

After the Journal article, Kjellberg seemed to lose whatever was left of
his inhibitions. He started his own parody news series, ``Pew News,'' in
which he mocked the media and dissected negative articles about him. He
took on the kinds of culture-war topics he once avoided, like
microaggressions and the wage gap. Kjellberg never identified himself as
a conservative, but his new, more politicized views attracted the
attention of right-wing personalities like Carl Benjamin, a British
YouTuber known as Sargon of Akkad, and Alex Jones, who offered Kjellberg
a guest slot on Infowars. (Kjellberg politely declined.)

Kjellberg also kept making oopsies --- bad ones, the kind that would
instantly end careers anywhere outside YouTube. In a September 2017
video, he yelled ``nigger'' on a gaming live stream. (He apologized,
saying in a video, ``I know I can't keep messing up like this.'') In
2018, he recommended an anime review channel that, upon closer
inspection, turned out to be rife with anti-Semitic and hateful content.
These scandals made Kjellberg toxic outside YouTube. But they only
strengthened his fan base's
\href{https://www.nytimes3xbfgragh.onion/2017/02/16/magazine/youtubes-monster-pewdiepie-and-his-populist-revolt.html}{perception
of him as a martyr}, a symbol of YouTube's freewheeling culture
sacrificed at the altar of corporate profits. And they endeared him to
the alt-right, which saw in him a potentially valuable ally. ``In an age
of ruthless cultural conformity, someone like PewDiePie isn't supposed
to exist,'' Paul Joseph Watson, a far-right YouTube commentator, said in
a 2019 video defending Kjellberg from his critics. ``Which is why
they've tried to tear him down, over and over again.''

\textbf{Kjellberg and I} sat in the Airbnb together for an entire
afternoon, and I spent at least an hour trying to pin down his actual
political beliefs or get him to talk about how this period of intense
criticism shaped his views. He spoke freely about his feud with the news
media --- he now regrets going after the Journal reporters, he said,
especially given what happened to them afterward --- but when the
conversation approached partisan politics, he clammed up. He told me
that he was ``more apolitical than anything,'' and when I asked him
whether he was more right-wing or left-wing, he said he was ``somewhere
in between.'' We had a lot of long, futile exchanges like this:

``Are there any politicians who excite you?''

``No.''

``Like, anywhere in the world?''

``I couldn't name one, no.''

``What did you think about UKIP endorsing you?'' I asked. On Twitter,
the far-right British party had recently told its followers to subscribe
to his channel to stop T-Series from overtaking him.

``It's kind of funny how a political party would post about a meme,'' he
said. ``But it's also kind of like, Ehh, don't drag me into your
politics.''

To Kjellberg, the past few years have proved that there is no reward for
engaging in politics, even if it would be good fodder for his channel.
He is especially annoyed that he continues to be linked to the far
right, despite his insistence that he doesn't support its causes. When
Kjellberg is accused of being far right, he doesn't respond by
protesting that he's actually a ``classical liberal'' or a ``heterodox
Stoic'' or whatever YouTube reactionaries are calling themselves these
days. Instead, he insists that he's not interested in politics at all.
It's a clever strategic position for a guy whose audience straddles the
partisan divide.

But it's also plausible, if you watch more than a few of the roughly
4,000 videos on his channel. PewDiePie doesn't endorse candidates,
debate podcast hosts or make fun of transgender-bathroom bills. He plays
video games, reads memes off Reddit and sometimes jokes about stuff in
the news. His political preferences, to the extent they exist, seem
almost entirely predicated on entertainment value. President Trump, he
told me, ``became a meme for a while,'' until it stopped being funny.
And while recent history has taught us that extremists often use stupid
memes to smuggle their views into the mainstream, in Kjellberg's case,
the memes themselves seem to be the point.

Kjellberg described the New Zealand shooting as a major turning point in
his life. The morning after, he was in bed, struggling to get back to
sleep, when his phone began buzzing. He saw, on Twitter and in a flurry
of texts, that the shooting suspect had said, ``Subscribe to PewDiePie''
just before he began his slaughter. After Kjellberg absorbed the shock,
a sequence began unspooling in his mind. First, the news media --- which
wouldn't understand that all that ``Subscribe to PewDiePie'' stuff was
as much a joke as an actual endorsement --- would accuse him of
inspiring a mass murderer. All the Nazi stuff would come back up. He
would have to issue a statement, and doing so would make it seem as
though he were drawing attention to himself and away from the victims.

It had been about a month since the shooting when I asked Kjellberg
about it, and he looked genuinely pained. He repeated his concern for
the victims and their families, and he reiterated what he said about
himself a hundred times before: He wasn't a white nationalist, and he
didn't condone violence. He told me that he was planning to release a
video calling for an end to the ``Subscribe to PewDiePie'' campaign and
was hoping to return to his roots as a goofy, uncontroversial gaming
vlogger.

We parted ways. And then, for whatever it's worth, everything he said
would happen did. A few days after our interview,
\href{https://www.youtube.com/watch?v=Ah5MYGQBYRo}{he persuaded his fans
to end} ``Subscribe to PewDiePie.'' He stopped grinding axes with the
news media and started playing Minecraft again.
\href{https://www.cnn.com/2019/08/20/entertainment/pewdiepie-and-marzia-bisognin-marry-trnd/index.html}{He
and Bisognin married} and went on their honeymoon. He got new
advertising deals, and he returned to YouTube's good graces. YouTube
sent him a ``red diamond'' play button on the occasion of his 100
millionth subscriber, along with a congratulatory letter from Susan
Wojcicki, the chief executive --- a sign that, in the company's eyes, he
has turned back from the edge.

The charitable way to interpret PewDiePie's new attitude is that he has
actually grown up, developed a thicker skin and taken stock of his own
power and responsibility. The more pragmatic way is to point out that
Kjellberg is a professional YouTuber and that --- for now, at least ---
his career prospects are partly dependent on his ability to stay out of
trouble. Either way, if Kjellberg ever achieves something like
redemption, it will always be complicated by the world-historic nature
of his YouTube stardom and the daily high-wire act it requires of him.
Stand too close to the edge, and he risks jeopardizing his standing with
the people who sign his checks, host his videos and write about him on
gaming websites. Stand too far from the edge, and he risks looking like
a sellout in front of 100 million fans, many of whom follow him
expressly for the unfiltered straight talk --- or worse --- and could
turn on him if they sense that he is taking directions from above.

\textbf{In September, after} reaching 100 million subscribers, PewDiePie
uploaded a new video. In it,
\href{https://www.theverge.com/2019/9/11/20860553/pewdiepie-felix-kjellberg-adl-donation-anti-defamation-league-conspiracy-theories-50000}{he
pledged to give \$50,000} to the Anti-Defamation League, one
organization that criticized him for joking about the Holocaust. It was
a symbolic gesture of closure, a signal that he had matured and was
ready to apologize and move on. ``I've finally come to terms,'' he said,
``with the responsibility I have as a creator.''

Some of his fans, though, weren't ready to bury the hatchet. They
barraged the video's comment section with anti-Semitic slurs and
conspiracy theories and accused the Anti-Defamation League of
blackmailing him into making a donation. Some of them were actual
anti-Semites, but others were just suspicious. Why would PewDiePie
donate to an organization that criticized him? Was he being bullied?
They had been conditioned to think that any time a YouTuber apologized
for stepping over the line, it was because someone from the outside ---
a media organization, an advertiser, YouTube corporate --- was forcing
his or her hand. ``As his second family,'' one commenter on a
PewDiePie-themed subreddit wrote, ``I feel like we're obligated to voice
our concerns when there's something clearly not right here.''

The next day, Kjellberg
\href{https://www.youtube.com/watch?v=PbfX3ZyHLJg}{posted another video}
and announced that he wasn't giving the \$50,000 to the Anti-Defamation
League after all. He said that it ``doesn't feel genuine for me to
proceed with a donation at this point'' and that he ``didn't know a lot
of things'' about the group's activities when he made the pledge. He
would find another charity, he said, one that better represented his
values. Some people interpreted Kjellberg's about-face as a sign that he
wasn't actually sorry for the Nazi jokes. Others surmised, probably more
accurately, that he feared offending his most vocal fans more than he
feared the criticism he would face for retracting a donation to a Jewish
anti-hate group. No one was particularly happy, least of all Kjellberg,
who took to his YouTube channel to confess that he just wanted to make
the drama go away.

Watching all this play out, I thought about something he told me back in
Brighton. We had been talking about the likelihood that T-Series would
soon dethrone him after six years as YouTube's top creator. Kjellberg
surprised me by saying that it didn't really bother him --- that, in
fact, he had been getting nostalgic for the days when he had fewer
subscribers. He confessed that he periodically thought about giving up
on YouTube altogether. His wife had given up her popular YouTube channel
last year. I suggested that he could do the same thing --- shutter his
channel and spend his days lounging on a hammock somewhere with her.

``Don't tempt me,'' he said with a smile. ``I kind of question if the
positive outweighs the negative,'' he went on. ``It's a lot more than I
think I signed up for.'' He seemed to catch himself mid-daydream and
clarified: Deleting his channel is not something he would really go
through with. Like many other extreme ideas, it's just something he
plays with from time to time.

Kevin Roose is a technology columnist for The Times and a contributing
writer for the magazine.
\href{https://www.nytimes3xbfgragh.onion/interactive/2017/11/09/magazine/tech-design-autonomous-future-cars-detroit-ford.html}{He
last wrote about Ford and self-driving cars.}

The Culture Issue

\begin{itemize}
\tightlist
\item
  \href{https://www.nytimes3xbfgragh.onion/interactive/2019/10/09/magazine/tyler-perry-black-theater.html}{Black
  Theater Is Having a Moment. Thank Tyler Perry. (Seriously.)}
\item
  \href{https://www.nytimes3xbfgragh.onion/interactive/2019/10/09/magazine/PewDiePie-interview.html}{What
  Does PewDiePie Really Believe?}
\item
  \href{https://www.nytimes3xbfgragh.onion/interactive/2019/10/09/magazine/kathryn-hahn-mrs-fletcher.html}{Kathryn
  Hahn's Funny, Sensual Portrayals of Female Desire}
\item
  \href{https://www.nytimes3xbfgragh.onion/interactive/2019/10/08/magazine/black-women-artists-conversation.html}{`I
  Want to Explore the Wonder of What It Is to Be a Black American'}
\item
  \href{https://www.nytimes3xbfgragh.onion/interactive/2019/10/08/magazine/rosalia-flamenco.html}{Rosalía's
  Incredible Journey From Flamenco to Megastardom}
\item
  \href{https://www.nytimes3xbfgragh.onion/interactive/2019/10/08/magazine/ben-lerner-topeka-school.html}{To
  Decode White Male Rage, First He Had to Write in His Mother's Voice}
\item
  \href{https://www.nytimes3xbfgragh.onion/interactive/2019/10/08/magazine/susan-sontag.html}{How
  Susan Sontag Taught Me to Think}
\item
  \href{https://www.nytimes3xbfgragh.onion/interactive/2019/10/09/magazine/moma-reopening.html}{Backstage
  at the Modern}
\end{itemize}

\protect\hyperlink{}{} \protect\hyperlink{}{}

\includegraphics{https://static01.graylady3jvrrxbe.onion/newsgraphics/2019/10/13/culture/1a9865dae9d1a0d21273eb2bff896ea8752f0273/caret.svg}

\hypertarget{more-culture}{%
\subsection{More Culture}\label{more-culture}}

\begin{itemize}
\tightlist
\item
  \href{https://www.nytimes3xbfgragh.onion/interactive/2019/10/09/magazine/tyler-perry-black-theater.html}{}
\item
  \href{https://www.nytimes3xbfgragh.onion/interactive/2019/10/09/magazine/kathryn-hahn-mrs-fletcher.html}{}
\item
  \href{https://www.nytimes3xbfgragh.onion/interactive/2019/10/08/magazine/black-women-artists-conversation.html}{}
\item
  \href{https://www.nytimes3xbfgragh.onion/interactive/2019/10/08/magazine/rosalia-flamenco.html}{}
\item
  \href{https://www.nytimes3xbfgragh.onion/interactive/2019/10/08/magazine/ben-lerner-topeka-school.html}{}
\item
  \href{https://www.nytimes3xbfgragh.onion/interactive/2019/10/08/magazine/susan-sontag.html}{}
\item
  \href{https://www.nytimes3xbfgragh.onion/interactive/2019/10/09/magazine/moma-reopening.html}{}
\end{itemize}

SOLOMON

\hypertarget{more-on-nytimescom}{%
\subsection{More on NYTimes.com}\label{more-on-nytimescom}}

Advertisement

\hypertarget{site-information-navigation}{%
\subsection{Site Information
Navigation}\label{site-information-navigation}}

\begin{itemize}
\tightlist
\item
  \href{https://help.nytimes3xbfgragh.onion/hc/en-us/articles/115014792127-Copyright-notice}{©
  2020 The New York Times Company}
\item
  \href{https://www.nytimes3xbfgragh.onion}{Home}
\item
  \href{https://www.nytimes3xbfgragh.onion/search/}{Search}
\item
  Accessibility concerns? Email us at
  \href{mailto:accessibility@NYTimes.com}{\nolinkurl{accessibility@NYTimes.com}}.
  We would love to hear from you.
\item
  \href{https://help.nytimes3xbfgragh.onion/hc/en-us/articles/115015385887-Contact-Us}{Contact
  Us}
\item
  \href{https://www.nytco.com/careers/}{Work with us}
\item
  \href{https://nytmediakit.com/}{Advertise}
\item
  \href{https://help.nytimes3xbfgragh.onion/hc/en-us/articles/115014892108-Privacy-policy\#pp}{Your
  Ad Choices}
\item
  \href{https://help.nytimes3xbfgragh.onion/hc/en-us/articles/115014892108-Privacy-policy}{Privacy}
\item
  \href{https://help.nytimes3xbfgragh.onion/hc/en-us/articles/115014893428-Terms-of-service}{Terms
  of Service}
\item
  \href{https://help.nytimes3xbfgragh.onion/hc/en-us/articles/115014893968-Terms-of-sale}{Terms
  of Sale}
\end{itemize}

\hypertarget{site-information-navigation-1}{%
\subsection{Site Information
Navigation}\label{site-information-navigation-1}}

\begin{itemize}
\tightlist
\item
  \href{https://spiderbites.nytimes3xbfgragh.onion}{Site Map}
\item
  \href{https://help.nytimes3xbfgragh.onion/hc/en-us}{Help}
\item
  \href{https://help.nytimes3xbfgragh.onion/hc/en-us/articles/115015385887-Contact-Us?redir=myacc}{Site
  Feedback}
\item
  \href{https://www.nytimes3xbfgragh.onion/subscription?campaignId=37WXW}{Subscriptions}
\end{itemize}
