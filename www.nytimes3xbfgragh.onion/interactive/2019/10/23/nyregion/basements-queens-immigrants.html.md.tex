 **NYTimes.com no longer supports Internet Explorer 9 or earlier. Please
upgrade your browser.
\href{http://www.nytimes3xbfgragh.onion/content/help/site/ie9-support.html}{LEARN
MORE »}

**Sections

**Home

**Search

\hypertarget{the-new-york-times}{%
\subsection{\texorpdfstring{\href{http://www.nytimes3xbfgragh.onion/}{The
New York Times}}{The New York Times}}\label{the-new-york-times}}

\hypertarget{-new-york-}{%
\subsubsection{\texorpdfstring{ \href{/section/nyregion}{New York}
}{ New York }}\label{-new-york-}}

 \href{/section/nyregion}{New York} \textbar{}Underground Lives: The
Sunless World of Immigrants in Queens

**Close search

\hypertarget{site-search-navigation}{%
\subsection{Site Search Navigation}\label{site-search-navigation}}

Search NYTimes.com

**Clear this text input

Go

\url{https://nyti.ms/2MGL6hL}

\hypertarget{site-navigation}{%
\subsection{Site Navigation}\label{site-navigation}}

\hypertarget{site-mobile-navigation}{%
\subsection{Site Mobile Navigation}\label{site-mobile-navigation}}

\hypertarget{underground-lives-the-sunless-world-of-immigrants-in-queens}{%
\section{Underground Lives: The Sunless World of Immigrants in
Queens}\label{underground-lives-the-sunless-world-of-immigrants-in-queens}}

OCT. 23, 2019

TEMPLATE NOT FOUND: blank

\hypertarget{underground-lives-the-sunless-world-of-immigrants-in-queens-1}{%
\section{Underground Lives: The Sunless World of Immigrants in
Queens}\label{underground-lives-the-sunless-world-of-immigrants-in-queens-1}}

By \textbf{Nikita Stewart, Ryan Christopher Jones, Sergio Peçanha,
Jeffrey Furticella} and \textbf{Josh Williams}

\href{https://www.nytimes3xbfgragh.onion/interactive/2019/10/28/nyregion/migrantes-queens-sotanos.html}{Leer
en español}

In New York, hope sometimes comes at the price of the sun.

The city welcomes poor immigrants, but its housing does not. Most rents
are far beyond the means of people like Amado, who arrive looking for a
better life or to make money to send back home.

So they turn to the basements of Queens.

Underneath the borough lives a shadow city of illegal apartments,
shielded from the light.

Owners of one- and two-family homes have carved up their basements into
makeshift dorms, illicitly constructed with narrow hallways, windowless
bedrooms, shaky walls and electrical wiring strung together like knotted
shoelaces. There is no accurate count of how many exist, but estimates
are in the tens of thousands.

An open secret, the basements are a haven for thousands of people who
work in restaurant kitchens, on delivery bikes, in small factories or on
construction sites.

They live in tiny rooms alone, or share tight spaces with strangers, or
even sleep in shifts.

The apartments are unquestionably dangerous. But there is also a kind of
safety there. Immigrants from all over the world, undocumented and
documented, have found sanctuary in Queens among people who share their
language, food and culture.

The small, dark room where Amado stays costs \$650 a month. His share is
\$325. His roommate is undocumented, and so is his roommate's brother,
who earlier this year frequently stayed over and shared a twin bed. The
brothers slept head-to-toe, back-to-back, while Amado lay on his twin
bed, quietly reading the Old Testament.

In his corner of the room, the glow of a small television set played the
news and telenovelas, offering some brightness.

He squinted, even in glasses.

\hypertarget{the-basements-of-queens}{%
\subsection{The basements of Queens}\label{the-basements-of-queens}}

Queens is the city's most diverse borough, with 800 languages spoken
among its residents. Vendors work long into the night because people are
always on their way to or from work.

All over, ``FOR RENT'' signs dot windows in homes and cover electrical
poles. There is no explanation of the rules, and no warning of danger.
The basement apartments differ considerably in size, quality and
legality.

The threat of raids by immigration authorities had residents on edge
during the summer. But immigrants said they still feel a great deal of
protection in Queens, where there is more access to free legal help to
assist them with their cases.

They can also find housing that is less expensive than in other places,
even if it does not meet city building standards. Eight areas in Queens
are consistently among the top 10 places in the city with the most
complaints for illegal home conversions.

Those conversions can lead to tragedies, including cases where residents
could not escape fast-moving fires that enveloped basements where they
lived.

Pushed by advocacy groups like Chhaya, which focuses on the housing
needs of South Asian residents, the city recently started a pilot
program that will provide homeowners with loans to install windows,
carbon monoxide and smoke detectors, and separate exits. But it will
benefit just 40 homeowners in Brooklyn.

City law says a basement's ceilings must be at least 7 feet 6 inches
high. Living spaces must have a window. The city must approve apartments
with a certificate of occupancy before they can be rented.

Such rules make practically all of the basement housing stock in Queens
illegal, including Amado's home. The ceilings where he lives are just
under seven feet high and the space has no visible smoke detectors. More
than half of the apartment sits underground, which technically makes the
home a cellar. In New York, all cellars are illegal dwellings.

Amado's room shares a window with the room next to his. He cannot use it
as an exit in case of an emergency. It is too small. And if he opens or
closes the window, it affects the other room, like a window that sits
between two rows on an airplane.

The home had been cited years earlier for illegal conversion and illegal
occupancy. But according to city records, an inspector had not visited
since November 2003, when he could not get inside. When he returned
three weeks later, he did not find any boarders.

But the basement apartment was eventually filled again. Amado moved in a
few years ago after spotting a sign in a window. Most of the time he and
his roommates stay quiet. The only visible sign of life is the light
coming through the slots under the men's closed bedroom doors.

\hypertarget{life-underground}{%
\subsection{Life underground}\label{life-underground}}

At the entrance of the home of Sosimo and Dalilia, there is a tangle of
wires, like a messy plate of spaghetti that stuck after being thrown on
the wall.

The married couple work in dry cleaning and live with two other men in a
wood-paneled basement deep underground.

They moved there when their daughter, Jessica, now 27, went to college
so they could help with tuition. They had previously paid \$1,600 a
month for an apartment. They now pay \$500 a month for their room.

The narrow kitchen is taken up by a stove attached to a propane tank
that was designed for outdoor cooking. But they cannot use the setup in
the summer because it heats up the apartment to uncomfortable
temperatures.

In a common room, Sosimo records his amateur radio show that draws
listeners from New York, Los Angeles and his native San Andrés Solaga in
Oaxaca, Mexico. Sosimo took up radio hosting about two years ago.

Some residents have been able to find basements that are structurally
sound, including Itamar Bispo, a 76-year-old guitarist, who lives in an
apartment that has windows in every room and two easy exits.

But he must walk through the landlord's office to get to his apartment,
which is labeled ``OFFICE'' on the door.

His bathroom is outside his apartment in the real office, and he has no
stove. His refrigerator sits next to a chest of drawers.

In a corner of the room that serves as both the kitchen and living room,
he makes music, setting up his microphone and singing tunes. ``You only
need one song,'' he says, still hoping to make the hit record that
eluded him in his youth.

For now, his retirement benefits and an occasional recording gig are
just enough to afford the \$850-a-month apartment.

There are exceptions to the darkness and loneliness that can come with
living underground. Oksoon Son, 65, moved about two years ago into the
basement of the house her daughter and son-in-law rent, wanting to help
raise her grandchildren.

Ms. Son, a retired nail technician, first arrived in Queens from South
Korea in 1980. After living in a homeless shelter, Ms. Son raised her
daughter and son in Queensbridge public housing.

``I like this place,'' she said in Korean one afternoon as light from
the windows filled it up. ``The sun.''

\hypertarget{light}{%
\subsection{Light}\label{light}}

That's what Amado wants. A little sun, which he gets briefly when he
walks to work, six days a week, at the restaurant where he's worked for
11 years.

Amado, who is in his 50s, has lived in the United States for nearly 29
years, coming by way of a smuggler through California, on a bus, first
to Brooklyn with a brother, then to Queens. He is now in the country
legally and able to travel back and forth to Mexico.

``I was the first one in the family that arrived here,'' he said in
Spanish. ``It was very difficult for me to be in this city. I was very
afraid of, like, the escalators.'' Amado laughed at himself.

He doesn't have much of a social life in New York. On Wednesdays, he
heads to Make the Road New York, a nonprofit organization that operates
out of a storefront on Roosevelt Avenue. He owes them, he says, for
legal assistance they provided two years ago when he had problems with
his green card. He shows up to nearly every protest to lend his body and
voice to whatever cause.

Otherwise he's in his basement, where the glow from his phone, his
lifeline to his family in Mexico, provides a little extra light.

A wall in Amado's room reflects the sacrifice he's made: Receipts tacked
to the wall show bank deposits and money orders sent back to his wife in
Coatzingo, Mexico.

Amado longs for Coatzingo, the village where he's from, and where his
family lives. He's able to make the trip there only twice a year.

Life is different there. In Coatzingo, Amado switches his baseball cap
for a cowboy hat, and an entirely different man emerges.

He is a well-regarded man in his village, a man with a wife, two
stepdaughters, an extended family, friends. Some might even call him
rich. He has a piece of land to call his own.

He basks in the sun. He swims in a cool, natural pool. He tends to the
family cornfields and a few lime trees.

He caresses his wife, Justina, able to touch her and not just hear her
voice in the dark in the basement.

He runs errands, he pays bills, takes care of his family, mundane tasks
that bring him pride.

With the money he saves living frugally in the basement in New York, he
is patron to both his family and the town, doling out his earnings to
pay for small luxuries, like a large flat-screen television.

This is why he's in New York. If Amado had stayed in Coatzingo, he most
likely would not have a job that could support his family. The crops of
the family farm bear fruit but earn little money.

He plans to stay in New York for at least another five years, when he
thinks he will have saved enough money to live a secure life in Mexico
and can give his daughters the option of living in the United States to
pursue whatever dreams they might have.

Until then, he will stay in the dark.

Source: New York City Department of Buildings; Note: Complaint data from
January 2014 through September 2019.

Additional reporting by Seungjae Seo. Additional research by Susan
Beachy. Diagram by Evan Grothjan. Map by Troy Griggs. Additional editing
and production by Meghan Louttit.

\subsection{}

\begin{itemize}
\item
  \href{https://www.nytimes3xbfgragh.onion/2019/10/23/reader-center/immigrants-queens-photography.html}{}

  \includegraphics{https://static01.graylady3jvrrxbe.onion/images/2019/10/24/insider/24insider-amado5/24insider-amado5-mediumThreeByTwo225.jpg}

  \hypertarget{finding-amado-in-queens-and-going-with-him-to-mexico}{%
  \subsection{Finding Amado in Queens, and Going With Him to
  Mexico}\label{finding-amado-in-queens-and-going-with-him-to-mexico}}

  Oct. 24, 2019
\item
  \href{https://www.nytimes3xbfgragh.onion/2019/05/19/nyregion/nyc-taxis-medallions-suicides.html}{}

  \includegraphics{https://static01.graylady3jvrrxbe.onion/images/2019/02/21/nyregion/00medallionsnewFINALpromo/00medallionsnewFINALpromo-mediumThreeByTwo225-v4.jpg}

  \hypertarget{part-1-reckless-loans-devastated-a-generation-of-taxi-drivers}{%
  \subsection{Part 1: Reckless Loans Devastated a Generation of Taxi
  Drivers}\label{part-1-reckless-loans-devastated-a-generation-of-taxi-drivers}}

  July 10, 2020
\item
  \href{https://www.nytimes3xbfgragh.onion/interactive/2019/06/14/nyregion/new-york-skyline-inequality.html}{}

  \includegraphics{https://static01.graylady3jvrrxbe.onion/images/2019/06/14/autossell/intro-horizontal/intro-horizontal-mediumThreeByTwo225.jpg}

  \hypertarget{new-yorks-skyline-is-changing-to-give-the-wealthy-a-better-view}{%
  \subsection{New York's Skyline Is Changing to Give the Wealthy a
  Better
  View}\label{new-yorks-skyline-is-changing-to-give-the-wealthy-a-better-view}}

  June 27, 2019
\item
  \href{https://www.nytimes3xbfgragh.onion/interactive/2018/10/30/nyregion/homeless-children.html}{}

  \includegraphics{https://static01.graylady3jvrrxbe.onion/images/2018/11/04/nyregion/04bornhomelessPROMO2/merlin_144659847_9d21f227-30a6-422e-9cc4-beba517281fd-mediumThreeByTwo225.jpg}

  \hypertarget{baby-antonio-5-pounds-12-ounces-and-homeless-from-birth}{%
  \subsection{Baby Antonio: 5 Pounds, 12 Ounces and Homeless From
  Birth}\label{baby-antonio-5-pounds-12-ounces-and-homeless-from-birth}}

  June 11, 2020
\item
  \href{https://www.nytimes3xbfgragh.onion/interactive/2018/10/18/nyregion/new-york-city-inequality-gentrification.html}{}

  \includegraphics{https://static01.graylady3jvrrxbe.onion/images/2018/10/20/nyregion/20-inequality-diptych/20-inequality-diptych-mediumThreeByTwo225-v2.jpg}

  \hypertarget{how-a-garden-for-the-poor-became-a-playground-for-the-rich}{%
  \subsection{How a Garden for the Poor Became a Playground for the
  Rich}\label{how-a-garden-for-the-poor-became-a-playground-for-the-rich}}

  Oct. 20, 2018
\end{itemize}

Advertisement

\hypertarget{site-information-navigation}{%
\subsection{Site Information
Navigation}\label{site-information-navigation}}

\begin{itemize}
\tightlist
\item
  \href{https://help.nytimes3xbfgragh.onion/hc/en-us/articles/115014792127-Copyright-notice}{©
  2020 The New York Times Company}
\item
  \href{https://www.nytimes3xbfgragh.onion}{Home}
\item
  \href{https://www.nytimes3xbfgragh.onion/search/}{Search}
\item
  Accessibility concerns? Email us at
  \href{mailto:accessibility@NYTimes.com}{\nolinkurl{accessibility@NYTimes.com}}.
  We would love to hear from you.
\item
  \href{https://help.nytimes3xbfgragh.onion/hc/en-us/articles/115015385887-Contact-Us}{Contact
  Us}
\item
  \href{https://www.nytco.com/careers/}{Work with us}
\item
  \href{https://nytmediakit.com/}{Advertise}
\item
  \href{https://help.nytimes3xbfgragh.onion/hc/en-us/articles/115014892108-Privacy-policy\#pp}{Your
  Ad Choices}
\item
  \href{https://help.nytimes3xbfgragh.onion/hc/en-us/articles/115014892108-Privacy-policy}{Privacy}
\item
  \href{https://help.nytimes3xbfgragh.onion/hc/en-us/articles/115014893428-Terms-of-service}{Terms
  of Service}
\item
  \href{https://help.nytimes3xbfgragh.onion/hc/en-us/articles/115014893968-Terms-of-sale}{Terms
  of Sale}
\end{itemize}

\hypertarget{site-information-navigation-1}{%
\subsection{Site Information
Navigation}\label{site-information-navigation-1}}

\begin{itemize}
\tightlist
\item
  \href{https://spiderbites.nytimes3xbfgragh.onion}{Site Map}
\item
  \href{https://help.nytimes3xbfgragh.onion/hc/en-us}{Help}
\item
  \href{https://help.nytimes3xbfgragh.onion/hc/en-us/articles/115015385887-Contact-Us?redir=myacc}{Site
  Feedback}
\item
  \href{https://www.nytimes3xbfgragh.onion/subscription?campaignId=37WXW}{Subscriptions}
\end{itemize}
