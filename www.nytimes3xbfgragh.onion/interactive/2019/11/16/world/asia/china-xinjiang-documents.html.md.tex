Sections

SEARCH

\protect\hyperlink{site-content}{Skip to
content}\protect\hyperlink{site-index}{Skip to site index}

\hypertarget{comments}{%
\subsection{\texorpdfstring{\protect\hyperlink{commentsContainer}{Comments}}{Comments}}\label{comments}}

\href{}{`Absolutely No Mercy': Leaked Files Expose How China Organized
Mass Detentions of Muslims}\href{}{Skip to Comments}

The comments section is closed. To submit a letter to the editor for
publication, write to
\href{mailto:letters@NYTimes.com}{\nolinkurl{letters@NYTimes.com}}.

\hypertarget{absolutely-no-mercy-leaked-files-expose-how-china-organized-mass-detentions-of-muslims}{%
\section{`Absolutely No Mercy': Leaked Files Expose How China Organized
Mass Detentions of
Muslims}\label{absolutely-no-mercy-leaked-files-expose-how-china-organized-mass-detentions-of-muslims}}

By \href{https://www.nytimes3xbfgragh.onion/by/austin-ramzy}{Austin
Ramzy} and
\href{https://www.nytimes3xbfgragh.onion/by/chris-buckley}{Chris
Buckley}Nov. 16, 2019

\begin{itemize}
\item
\item
\item
\item
\item
  \emph{371}
\end{itemize}

More than 400 pages of internal Chinese documents provide an
unprecedented inside look at the crackdown on ethnic minorities in the
Xinjiang region.

\hypertarget{the-xinjiang-papers}{%
\subsubsection{The Xinjiang Papers}\label{the-xinjiang-papers}}

\hypertarget{absolutely-no-mercy-leaked-files-expose-how-china-organized-mass-detentions-of-muslims-1}{%
\section{`Absolutely No Mercy': Leaked Files Expose How China Organized
Mass Detentions of
Muslims}\label{absolutely-no-mercy-leaked-files-expose-how-china-organized-mass-detentions-of-muslims-1}}

More than 400 pages of internal Chinese documents provide an
unprecedented inside look at the crackdown on ethnic minorities in the
Xinjiang region.

By \href{https://www.nytimes3xbfgragh.onion/by/austin-ramzy}{Austin
Ramzy} and
\href{https://www.nytimes3xbfgragh.onion/by/chris-buckley}{Chris
Buckley}

November 16, 2019

\href{https://www.nytimes3xbfgragh.onion/zh/2019/11/16/world/asia/xinjiang-documents-chinese.html}{阅读简体中文版}

\includegraphics{https://static01.graylady3jvrrxbe.onion/newsgraphics/2019/10/21/xinjiang-article/13c819960fa0d589a50c0179792cdbf8c4b72c5b/top-page-02.jpg}
\includegraphics{https://static01.graylady3jvrrxbe.onion/newsgraphics/2019/10/21/xinjiang-article/13c819960fa0d589a50c0179792cdbf8c4b72c5b/top-page-06.jpg}
\includegraphics{https://static01.graylady3jvrrxbe.onion/newsgraphics/2019/10/21/xinjiang-article/13c819960fa0d589a50c0179792cdbf8c4b72c5b/top-page-03.jpg}
\includegraphics{https://static01.graylady3jvrrxbe.onion/newsgraphics/2019/10/21/xinjiang-article/13c819960fa0d589a50c0179792cdbf8c4b72c5b/top-page-11.jpg}
\includegraphics{https://static01.graylady3jvrrxbe.onion/newsgraphics/2019/10/21/xinjiang-article/13c819960fa0d589a50c0179792cdbf8c4b72c5b/top-page-10.jpg}
\includegraphics{https://static01.graylady3jvrrxbe.onion/newsgraphics/2019/10/21/xinjiang-article/13c819960fa0d589a50c0179792cdbf8c4b72c5b/top-page-08.jpg}
\includegraphics{https://static01.graylady3jvrrxbe.onion/newsgraphics/2019/10/21/xinjiang-article/13c819960fa0d589a50c0179792cdbf8c4b72c5b/top-page-export-2-1.jpg}
\includegraphics{https://static01.graylady3jvrrxbe.onion/newsgraphics/2019/10/21/xinjiang-article/13c819960fa0d589a50c0179792cdbf8c4b72c5b/top-page-04.jpg}
\includegraphics{https://static01.graylady3jvrrxbe.onion/newsgraphics/2019/10/21/xinjiang-article/13c819960fa0d589a50c0179792cdbf8c4b72c5b/top-page-05.jpg}
\includegraphics{https://static01.graylady3jvrrxbe.onion/newsgraphics/2019/10/21/xinjiang-article/13c819960fa0d589a50c0179792cdbf8c4b72c5b/top-page-09.jpg}
\includegraphics{https://static01.graylady3jvrrxbe.onion/newsgraphics/2019/10/21/xinjiang-article/13c819960fa0d589a50c0179792cdbf8c4b72c5b/top-page-12.jpg}
\includegraphics{https://static01.graylady3jvrrxbe.onion/newsgraphics/2019/10/21/xinjiang-article/13c819960fa0d589a50c0179792cdbf8c4b72c5b/top-page-07.jpg}

\textbf{HONG KONG ---} The students booked their tickets home at the end
of the semester, hoping for a relaxing break after exams and a summer of
happy reunions with family in China's far west.

Instead, they would soon be told that their parents were gone, relatives
had vanished and neighbors were missing --- all of them locked up in an
expanding network of detention camps built to hold Muslim ethnic
minorities.

The authorities in the Xinjiang region worried the situation was a
powder keg. And so they prepared.

The leadership distributed a classified directive advising local
officials to corner returning students as soon as they arrived and keep
them quiet. It included a chillingly bureaucratic guide for how to
handle their anguished questions, beginning with the most obvious: Where
is my family?

\includegraphics{https://static01.graylady3jvrrxbe.onion/newsgraphics/2019/10/21/xinjiang-article/13c819960fa0d589a50c0179792cdbf8c4b72c5b/doc1-highlight-01.jpg}

``\textbf{They're in a training school set up by the government,}'' the
prescribed answer began. If pressed, officials were to tell students
that their relatives were not criminals --- yet could not leave these
``schools.''

\includegraphics{https://static01.graylady3jvrrxbe.onion/newsgraphics/2019/10/21/xinjiang-article/13c819960fa0d589a50c0179792cdbf8c4b72c5b/doc1-highlight-02.jpg}

The question-and-answer script also included a \textbf{barely concealed
threat}: Students were to be told that their behavior could either
shorten or extend the detention of their relatives.

\includegraphics{https://static01.graylady3jvrrxbe.onion/newsgraphics/2019/10/21/xinjiang-article/13c819960fa0d589a50c0179792cdbf8c4b72c5b/doc1-highlight-03.jpg}

 **** ``\textbf{I'm sure that you will support them, because this is for
their own good,}'' officials were advised to say, ``\textbf{and also for
your own good.}''

​

The directive was among 403 pages of internal documents that have been
shared with The New York Times in one of the most significant leaks of
government papers from inside China's ruling Communist Party in decades.
They provide an unprecedented inside view of the
\href{https://www.nytimes3xbfgragh.onion/2018/09/08/world/asia/china-uighur-muslim-detention-camp.html}{continuing
clampdown in Xinjiang, in which the authorities have corralled as many
as a million ethnic Uighurs, Kazakhs and others into internment camps
and prisons} over the past three years.

\href{https://www.nytimes3xbfgragh.onion/interactive/2019/11/16/world/asia/china-detention-directive.html}{Read
the Full Document: What Chinese Officials Told Children Whose Families
Were Put in Camps}

The party has
\href{https://www.nytimes3xbfgragh.onion/2019/03/12/world/asia/china-xinjiang.html}{rejected
international criticism of the camps and described them as job-training
centers} that use mild methods to fight Islamic extremism. But the
documents confirm the coercive nature of the crackdown in the words and
orders of the very officials who conceived and orchestrated it.

Even as the government presented its efforts in Xinjiang to the public
as benevolent and unexceptional, it discussed and organized a ruthless
and extraordinary campaign in these internal communications. Senior
party leaders are recorded ordering drastic and urgent action against
extremist violence, including the mass detentions, and discussing the
consequences with cool detachment.

Children saw their parents taken away, students wondered who would pay
their tuition and crops could not be planted or harvested for lack of
manpower, the reports noted. Yet officials were directed to tell people
who complained to be grateful for the Communist Party's help and stay
quiet.

The leaked papers offer a striking picture of how the hidden machinery
of the Chinese state carried out the country's most far-reaching
internment campaign since the Mao era. The key disclosures in the
documents include:

• President Xi Jinping, the party chief, laid the groundwork for the
crackdown in a series of speeches delivered in private to officials
during and after a visit to Xinjiang in April 2014, just weeks after
Uighur militants
\href{https://www.nytimes3xbfgragh.onion/2015/03/25/world/asia/china-executes-3-over-deadly-knife-attack-at-train-station-in-2014.html}{stabbed
more than 150 people at a train station}, killing 31. Mr. Xi called for
an all-out ``struggle against terrorism, infiltration and separatism''
using the ``organs of dictatorship,'' and showing ``absolutely no
mercy.''

• Terrorist attacks abroad and the drawdown of American troops in
Afghanistan heightened the leadership's fears and helped shape the
crackdown. Officials argued that attacks in Britain resulted from
policies that put ``human rights above security,'' and Mr. Xi urged the
party to emulate aspects of America's ``war on terror'' after the Sept.
11 attacks.

• The internment camps in Xinjiang expanded rapidly after the
appointment in August 2016 of Chen Quanguo, a zealous new party boss for
the region. He distributed Mr. Xi's speeches to justify the campaign and
exhorted officials to ``round up everyone who should be rounded up.''

• The crackdown encountered doubts and resistance from local officials
who feared it would exacerbate ethnic tensions and stifle economic
growth. Mr. Chen responded by purging officials suspected of standing in
his way, including one county leader who was jailed after quietly
releasing thousands of inmates from the camps.

The leaked papers consist of 24 documents, some of which contain
duplicated material. They include nearly 200 pages of internal speeches
by Mr. Xi and other leaders, and more than 150 pages of directives and
reports on the surveillance and control of the Uighur population in
Xinjiang. There are also references to plans to extend restrictions on
Islam to other parts of China.

The documents include 96 pages of internal speeches by

Mr. Xi,

102 pages of internal speeches by other officials,

161 pages of directives and reports on the surveillance and

control of the Uighur population in Xinjiang

and 44 pages of material from internal investigations

into local officials.

The documents include 96 pages of internal speeches by

Mr. Xi,

102 pages of internal speeches by other officials,

161 pages of directives and reports on the surveillance

and control of the Uighur population in Xinjiang

and 44 pages of material from internal investigations

into local officials.

The documents include 96 pages of

internal speeches by Mr. Xi,

102 pages of internal speeches by

other officials,

161 pages of directives and reports

on the surveillance and control of

the Uighur population in Xinjiang

and 44 pages of material from internal

investigations into local officials.

The documents include 96 pages

of internal speeches by Mr. Xi,

102 pages of internal speeches by

other officials,

161 pages of directives and reports

on the surveillance and control of

the Uighur population in Xinjiang

and 44 pages of material from

internal investigations into local

officials.

The documents include 96 pages of internal speeches by

Mr. Xi,

102 pages of internal speeches by other officials,

161 pages of directives and reports on the surveillance and

control of the Uighur population in Xinjiang

and 44 pages of material from internal investigations

into local officials.

Though it is unclear how the documents were gathered and selected, the
leak suggests greater discontent inside the party apparatus over the
crackdown than previously known. The papers were brought to light by a
member of the Chinese political establishment who requested anonymity
and expressed hope that their disclosure would prevent party leaders,
including Mr. Xi, from escaping culpability for the mass detentions.

The Chinese leadership wraps policymaking in secrecy, especially when it
comes to Xinjiang, a resource-rich territory located on the sensitive
frontier with Pakistan, Afghanistan and Central Asia. Predominantly
Muslim ethnic minority groups make up more than half the region's
population of 25 million. The largest of these groups are the Uighurs,
who speak a Turkic language and have long faced discrimination and
restrictions on cultural and religious activities.

Beijing has sought for decades to suppress Uighur resistance to Chinese
rule in Xinjiang. The current crackdown began after a surge of
antigovernment and anti-Chinese violence, including
\href{https://www.nytimes3xbfgragh.onion/2009/07/18/world/asia/18xinjiang.html}{ethnic
riots in 2009 in Urumqi}, the regional capital, and a May 2014
\href{https://www.nytimes3xbfgragh.onion/2014/12/09/world/asia/8-sentenced-to-death-for-terrorist-attacks-in-western-china.html}{attack
on an outdoor market that killed 39 people} just days before Mr. Xi
convened a leadership conference in Beijing to set a new policy course
for Xinjiang.

Since 2017, the authorities in Xinjiang have detained many hundreds of
thousands of Uighurs, Kazakhs and other Muslims in internment camps.
Inmates undergo months or years of indoctrination and interrogation
aimed at transforming them into secular and loyal supporters of the
party.

Of the 24 documents, the directive on how to handle minority students
returning home to Xinjiang in the summer of 2017 offers the most
detailed discussion of the indoctrination camps --- and the clearest
illustration of the regimented way the party told the public one story
while mobilizing around a much harsher narrative internally.

Even as the document advises officials to inform students that their
relatives are receiving ``treatment'' for exposure to radical Islam, its
title refers to family members who are being ``dealt with,'' or chuzhi,
a euphemism used in party documents to mean punishment.

Officials in Turpan, a city in eastern Xinjiang, drafted the
question-and-answer script after the regional government warned local
officials to prepare for the returning students. The agency coordinating
efforts to ``maintain stability'' across Xinjiang then distributed the
guide across the region and urged officials to use it as a model.

The government sends Xinjiang's brightest young Uighurs to universities
across China, with the goal of training a new generation of Uighur civil
servants and teachers loyal to the party.

The crackdown has been so extensive that it affected even these elite
students, the directive shows. And that made the authorities nervous.

``Returning students from other parts of China have widespread social
ties across the entire country,'' the directive noted. ``The moment they
issue incorrect opinions on WeChat, Weibo and other social media
platforms, the impact is widespread and difficult to eradicate.''

\includegraphics{https://static01.graylady3jvrrxbe.onion/newsgraphics/2019/10/21/xinjiang-article/13c819960fa0d589a50c0179792cdbf8c4b72c5b/doc2-highlight-01.jpg}

The document warned that there was a ``\textbf{serious possibility}''
students might sink into ``\textbf{turmoil}'' after learning what had
happened to their relatives. It recommended that police officers in
plain clothes and experienced local officials meet them as soon as they
returned ``to show humane concern and stress the rules.''

\includegraphics{https://static01.graylady3jvrrxbe.onion/newsgraphics/2019/10/21/xinjiang-article/13c819960fa0d589a50c0179792cdbf8c4b72c5b/doc2-highlight-02.jpg}

The directive's question-and-answer guide begins gently, with officials
advised to tell the students that they have ``\textbf{absolutely no need
to worry}'' about relatives who have disappeared.

\includegraphics{https://static01.graylady3jvrrxbe.onion/newsgraphics/2019/10/21/xinjiang-article/13c819960fa0d589a50c0179792cdbf8c4b72c5b/doc2-highlight-03.jpg}

``\textbf{Tuition for their period of study is free and so are food and
living costs, and the standards are quite high,}'' officials were told
to say, before adding that the authorities were spending more than \$3
per day on meals for each detainee, ``\textbf{even better than the
living standards that some students have back home.}''

\includegraphics{https://static01.graylady3jvrrxbe.onion/newsgraphics/2019/10/21/xinjiang-article/13c819960fa0d589a50c0179792cdbf8c4b72c5b/doc2-highlight-04.jpg}

``\textbf{If you want to see them,}'' the answer concluded, ``\textbf{we
can arrange for you to have a video meeting.}''

​

The authorities anticipated, however, that this was unlikely to mollify
students and provided replies to a series of other questions: When will
my relatives be released? If this is for training, why can't they come
home? Can they request a leave? How will I afford school if my parents
are studying and there is no one to work on the farm?

The guide recommended increasingly firm replies telling the students
that their relatives had been ``infected'' by the ``virus'' of Islamic
radicalism and must be quarantined and cured. Even grandparents and
family members who seemed too old to carry out violence could not be
spared, officials were directed to say.

``If they don't undergo study and training, they'll never thoroughly and
fully understand the dangers of religious extremism,'' one answer said,
citing the civil war in Syria and the rise of the Islamic State. ``No
matter what age, anyone who has been infected by religious extremism
must undergo study.''

Students should be grateful that the authorities had taken their
relatives away, the document said.

``Treasure this chance for free education that the party and government
has provided to thoroughly eradicate erroneous thinking, and also learn
Chinese and job skills,'' one answer said. ``This offers a great
foundation for a happy life for your family.''

The authorities appear to be using a scoring system to determine who can
be released from the camps: The document instructed officials to tell
the students that their behavior could hurt their relatives' scores, and
to assess the daily behavior of the students and record their attendance
at training sessions, meetings and other activities.

\includegraphics{https://static01.graylady3jvrrxbe.onion/newsgraphics/2019/10/21/xinjiang-article/13c819960fa0d589a50c0179792cdbf8c4b72c5b/doc3-highlight-01.jpg}

``\textbf{Family members, including you, must abide by the state's laws
and rules, and not believe or spread rumors,}'' officials were told to
say. ``\textbf{Only then can you add points for your family member, and
after a period of assessment they can leave the school if they meet
course completion standards.}''

\includegraphics{https://static01.graylady3jvrrxbe.onion/newsgraphics/2019/10/21/xinjiang-article/13c819960fa0d589a50c0179792cdbf8c4b72c5b/doc3-highlight-02.jpg}

If asked about the impact of the detentions on family finances,
officials were advised to assure students that ``\textbf{the party and
the government will do everything possible to ease your hardships.}''

\includegraphics{https://static01.graylady3jvrrxbe.onion/newsgraphics/2019/10/21/xinjiang-article/13c819960fa0d589a50c0179792cdbf8c4b72c5b/doc3-highlight-03.jpg}

 **** The line that stands out most in the script, however, may be the
model answer for how to respond to students who ask of their detained
relatives, ``\textbf{Did they commit a crime?}''

\includegraphics{https://static01.graylady3jvrrxbe.onion/newsgraphics/2019/10/21/xinjiang-article/13c819960fa0d589a50c0179792cdbf8c4b72c5b/doc3-highlight-04.jpg}

 **** The document instructed officials to acknowledge that they had
not. ``\textbf{It is just that their thinking has been infected by
unhealthy thoughts,}'' the script said.

\includegraphics{https://static01.graylady3jvrrxbe.onion/newsgraphics/2019/10/21/xinjiang-article/13c819960fa0d589a50c0179792cdbf8c4b72c5b/doc3-highlight-05.jpg}

 **** ``\textbf{Freedom is only possible when this `virus' in their
thinking is eradicated and they are in good health.}''

​

\hypertarget{-secret-speeches-}{%
\paragraph{ Secret Speeches }\label{-secret-speeches-}}

The ideas driving the mass detentions can be traced back to Xi Jinping's
first and only visit to Xinjiang as China's leader, a tour shadowed by
violence.

In 2014, little more than a year after becoming president, he spent four
days in the region, and on the last day of the trip, two Uighur
militants staged a suicide bombing outside a train station in Urumqi
that injured nearly 80 people, one fatally.

Weeks earlier, militants with knives had gone on a rampage at another
railway station, in southwest China, killing 31 people and injuring more
than 140. And less than a month after Mr. Xi's visit, assailants tossed
explosives into a vegetable market in Urumqi, wounding 94 people and
killing at least 39.

Against this backdrop of bloodshed, Mr. Xi delivered a series of secret
speeches setting the hard-line course that culminated in the security
offensive now underway in Xinjiang. While state media have alluded to
these speeches, none were made public.

The text of four of them, though, were among the leaked documents ---
and they provide a rare, unfiltered look at the origins of the crackdown
and the beliefs of the man who set it in motion.

``The methods that our comrades have at hand are too primitive,'' Mr. Xi
said in one talk, after inspecting a counterterrorism police squad in
Urumqi. ``None of these weapons is any answer for their big machete
blades, ax heads and cold steel weapons.''

``We must be as harsh as them,'' he added, ``and show absolutely no
mercy.''

In free-flowing monologues in Xinjiang and at a subsequent leadership
conference on Xinjiang policy in Beijing, Mr. Xi is recorded thinking
through what he called a crucial national security issue and laying out
his ideas for a ``people's war'' in the region.

Although he did not order mass detentions in these speeches, he called
on the party to unleash the tools of ``dictatorship'' to eradicate
radical Islam in Xinjiang.

Mr. Xi displayed a fixation with the issue that seemed to go well beyond
his public remarks on the subject. He likened Islamic extremism
alternately to a virus-like contagion and a dangerously addictive drug,
and declared that addressing it would require ``a period of painful,
interventionary treatment.''

``The psychological impact of extremist religious thought on people must
never be underestimated,'' Mr. Xi told officials in Urumqi on April 30,
2014, the final day of his trip to Xinjiang. ``People who are captured
by religious extremism --- male or female, old or young --- have their
consciences destroyed, lose their humanity and murder without blinking
an eye.''

In another speech, at the leadership conclave in Beijing a month later,
he warned of ``the toxicity of religious extremism.''

``As soon as you believe in it,'' he said, ``it's like taking a drug,
and you lose your sense, go crazy and will do anything.''

In several surprising passages, given the crackdown that followed, Mr.
Xi also told officials to not discriminate against Uighurs and to
respect their right to worship. He warned against overreacting to
natural friction between Uighurs and Han Chinese, the nation's dominant
ethnic group, and rejected proposals to try to eliminate Islam entirely
in China.

``In light of separatist and terrorist forces under the banner of Islam,
some people have argued that Islam should be restricted or even
eradicated,'' he said during the Beijing conference. He called that view
``biased, even wrong.''

But Mr. Xi's main point was unmistakable: He was leading the party in a
sharp turn toward greater repression in Xinjiang.

Before Mr. Xi, the party had often described attacks in Xinjiang as the
work of a few fanatics inspired and orchestrated by shadowy separatist
groups abroad. But Mr. Xi argued that Islamic extremism had taken root
across swaths of Uighur society.

In fact, the vast majority of Uighurs adhere to moderate traditions,
though some began embracing more conservative and more public religious
practices in the 1990s, despite state controls on Islam. Mr. Xi's
remarks suggest he was alarmed by the revival of public piety. He blamed
lax controls on religion, suggesting that his predecessors had let down
their guard.

While previous Chinese leaders emphasized economic development to stifle
unrest in Xinjiang, Mr. Xi said that was not enough. He demanded an
ideological cure, an effort to rewire the thinking of the region's
Muslim minorities.

``The weapons of the people's democratic dictatorship must be wielded
without any hesitation or wavering,'' Mr. Xi told the leadership
conference on Xinjiang policy, which convened six days after the deadly
attack on the vegetable market.

\hypertarget{-the-soviet-prism-}{%
\paragraph{ The Soviet Prism }\label{-the-soviet-prism-}}

Mr. Xi is the
\href{https://sinosphere.blogs.nytimes3xbfgragh.onion/2013/10/15/china-venerates-a-revolutionary-the-father-of-its-new-leader/}{son
of an early Communist Party leader} who in the 1980s supported more
relaxed policies toward ethnic minority groups, and some analysts had
expected he might follow his father's milder ways when he
\href{https://www.nytimes3xbfgragh.onion/2012/11/15/world/asia/communists-conclude-party-congress-in-china.html}{assumed
leadership of the party in November 2012}.

But the speeches underscore how Mr. Xi sees risks to China through the
prism of the collapse of the Soviet Union, which he blamed on
ideological laxity and spineless leadership.

Across China, he set about eliminating challenges to party rule;
dissidents and
\href{https://www.nytimes3xbfgragh.onion/2019/01/28/world/asia/china-wang-quanzhang-human-rights.html}{human
rights lawyers disappeared in waves of arrests}. In Xinjiang, he pointed
to examples from the former Soviet bloc to argue that economic growth
would not immunize a society against ethnic separatism.

The Baltic republics were among the most developed in the Soviet Union
but also the first to leave when the country broke up, he told the
leadership conference. Yugoslavia's relative prosperity did not prevent
its disintegration either, he added.

``We say that development is the top priority and the basis for
achieving lasting security, and that's right,'' Mr. Xi said. ``But it
would be wrong to believe that with development every problem solves
itself.''

In the speeches, Mr. Xi showed a deep familiarity with the history of
Uighur resistance to Chinese rule, or at least Beijing's official
version of it, and discussed episodes rarely if ever mentioned by
Chinese leaders in public, including brief periods of Uighur self-rule
in the first half of the 20th century.

Violence by Uighur militants has never threatened Communist control of
the region. Though attacks grew deadlier after 2009, when nearly 200
people died in ethnic riots in Urumqi, they remained relatively small,
scattered and unsophisticated.

Even so, Mr. Xi warned that the violence was spilling from Xinjiang into
other parts of China and could taint the party's image of strength.
Unless the threat was extinguished, Mr. Xi told the leadership
conference, ``social stability will suffer shocks, the general unity of
people of every ethnicity will be damaged, and the broad outlook for
reform, development and stability will be affected.''

Setting aside diplomatic niceties, he traced the origins of Islamic
extremism in Xinjiang to the Middle East, and warned that turmoil in
Syria and Afghanistan would magnify the risks for China. Uighurs had
traveled to both countries, he said, and could return to China as
seasoned fighters seeking an independent homeland, which they called
East Turkestan.

``After the United States pulls troops out of Afghanistan, terrorist
organizations positioned on the frontiers of Afghanistan and Pakistan
may quickly infiltrate into Central Asia,'' Mr. Xi said. ``East
Turkestan's terrorists who have received real-war training in Syria and
Afghanistan could at any time launch terrorist attacks in Xinjiang.''

Mr. Xi's predecessor, Hu Jintao, responded to the 2009 riots in Urumqi
with a clampdown but he also stressed economic development as a cure for
ethnic discontent --- longstanding party policy. But Mr. Xi signaled a
break with Mr. Hu's approach in the speeches.

``In recent years, Xinjiang has grown very quickly and the standard of
living has consistently risen, but even so ethnic separatism and
terrorist violence have still been on the rise,'' he said. ``This goes
to show that economic development does not automatically bring lasting
order and security.''

Ensuring stability in Xinjiang would require a sweeping campaign of
surveillance and intelligence gathering to root out resistance in Uighur
society, Mr. Xi argued.

He said new technology must be part of the solution, foreshadowing the
party's deployment of facial recognition, genetic testing and big data
in Xinjiang. But he also emphasized old-fashioned methods, such as
neighborhood informants, and urged officials to study how Americans
responded to the Sept. 11 attacks.

Like the United States, he said, China ``must make the public an
important resource in protecting national security.''

``We Communists should be naturals at fighting a people's war,'' he
said. ``We're the best at organizing for a task.''

The only suggestion in these speeches that Mr. Xi envisioned the
internment camps now at the heart of the crackdown was an endorsement of
more intense indoctrination programs in Xinjiang's prisons.

``There must be effective educational remolding and transformation of
criminals,'' he told officials in southern Xinjiang on the second day of
his trip. ``And even after these people are released, their education
and transformation must continue.''

Within months, indoctrination sites began opening across Xinjiang ---
mostly small facilities at first, which held dozens or hundreds of
Uighurs at a time for sessions intended to pressure them into disavowing
devotion to Islam and professing gratitude for the party.

Then in August 2016, a hard-liner named Chen Quanguo was transferred
from Tibet to govern Xinjiang. Within weeks, he called on local
officials to ``remobilize'' around Mr. Xi's goals and declared that Mr.
Xi's speeches ``set the direction for making a success of Xinjiang.''

New security controls and a drastic expansion of the indoctrination
camps followed.

\includegraphics{https://static01.graylady3jvrrxbe.onion/newsgraphics/2019/10/21/xinjiang-article/13c819960fa0d589a50c0179792cdbf8c4b72c5b/doc4-highlight-01.jpg}

``\textbf{The struggle against terror and to safeguard stability is a
protracted war, and also a war of offense,}'' Mr. Chen said in a speech
to the regional leadership in October 2017 that was among the leaked
papers.

\includegraphics{https://static01.graylady3jvrrxbe.onion/newsgraphics/2019/10/21/xinjiang-article/13c819960fa0d589a50c0179792cdbf8c4b72c5b/doc4-highlight-02.jpg}

In another document, a record of his remarks in a video conference in
August 2017, he cited ``\textbf{vocational skills, education training
and transformation centers}'' as an example of ``\textbf{good
practices}'' for achieving Mr. Xi's goals for Xinjiang.

​

The crackdown appears to have smothered violent unrest in Xinjiang, but
many experts have warned that the extreme security measures and mass
detentions are likely to breed resentment that could eventually inspire
worse ethnic clashes.

The camps have been condemned in Washington and other foreign capitals.
As early as the May 2014 leadership conference, though, Mr. Xi
anticipated international criticism and urged officials behind closed
doors to ignore it.

``Don't be afraid if hostile forces whine, or if hostile forces malign
the image of Xinjiang,'' he said.

\hypertarget{-round-up-everyone-}{%
\paragraph{ `Round Up Everyone' }\label{-round-up-everyone-}}

The documents show there was more resistance to the crackdown inside the
party than previously known --- and highlight the key role that the new
party boss in Xinjiang played in overcoming it.

Mr. Chen led a campaign akin to one of Mao's turbulent political
crusades, in which top-down pressure on local officials encouraged
overreach and any expression of doubt was treated as a crime.

In February 2017, he told thousands of police officers and troops
standing at attention in a vast square in Urumqi to prepare for a
``smashing, obliterating offensive.'' In the following weeks, the
documents indicate, the leadership settled on plans to detain Uighurs in
large numbers.

Mr. Chen issued a sweeping order: ``Round up everyone who should be
rounded up.'' The vague phrase appears repeatedly in internal documents
from 2017.

The party had previously used the phrase --- ``ying shou jin shou'' in
Chinese --- when demanding that officials be vigilant and comprehensive
in collecting taxes or measuring harvests. Now it was being applied to
humans in directives that ordered, with no mention of judicial
procedures, the detention of anyone who displayed ``symptoms'' of
religious radicalism or antigovernment views.

The authorities laid out dozens of such signs, including common behavior
among devout Uighurs such as wearing long beards, giving up smoking or
drinking, studying Arabic and praying outside mosques.

Party leaders reinforced the orders with warnings about terrorism abroad
and potential copycat attacks in China.

\includegraphics{https://static01.graylady3jvrrxbe.onion/newsgraphics/2019/10/21/xinjiang-article/13c819960fa0d589a50c0179792cdbf8c4b72c5b/doc5-highlight-01.jpg}

For example, a 10-page directive in June 2017 signed by Zhu Hailun, then
Xinjiang's top security official, called recent terrorist attacks in
Britain ``\textbf{a warning and a lesson for us.}'' It blamed the
British government's ``excessive emphasis on `human rights above
security,' and inadequate controls on the propagation of extremism on
the internet and in society.''

\includegraphics{https://static01.graylady3jvrrxbe.onion/newsgraphics/2019/10/21/xinjiang-article/13c819960fa0d589a50c0179792cdbf8c4b72c5b/doc5-highlight-02.jpg}

It also complained of security lapses in Xinjiang, including sloppy
investigations, malfunctions in surveillance equipment and the failure
to hold people accused of suspicious behavior.

\includegraphics{https://static01.graylady3jvrrxbe.onion/newsgraphics/2019/10/21/xinjiang-article/13c819960fa0d589a50c0179792cdbf8c4b72c5b/doc5-highlight-03.jpg}

 **** Keep up the detentions, it ordered. ``\textbf{Stick to rounding up
everyone who should be rounded up,}'' it said. ``\textbf{If they're
there, round them up.}''

​

The number of people swept into the camps remains a closely guarded
secret. But one of the leaked documents offers a hint of the scale of
the campaign: It instructed officials to prevent the spread of
infectious diseases in crowded facilities.

\hypertarget{-i-broke-the-rules-}{%
\paragraph{ `I Broke the Rules' }\label{-i-broke-the-rules-}}

The orders were especially urgent and contentious in Yarkand County, a
collection of rural towns and villages in southern Xinjiang where nearly
all of the 900,000 residents are Uighur.

In the 2014 speeches, Mr. Xi had singled out southern Xinjiang as the
front line in his fight against religious extremism. Uighurs make up
close to 90 percent of the population in the south, compared to just
under half in Xinjiang over all, and Mr. Xi set a long-term goal of
attracting more Han Chinese settlers.

He and other party leaders ordered a quasi-military organization, the
Xinjiang Production and Construction Corps, to accelerate efforts to
settle the area with more Han Chinese, the documents show.

A few months later, more than 100 Uighur militants armed with axes and
knives
\href{https://www.nytimes3xbfgragh.onion/2014/08/04/world/asia/china-says-nearly-100-are-killed-in-week-of-unrest-in-xinjiang.html}{attacked
a government office and police station in Yarkand}, killing 37 people,
according to government reports. In the battle, the security forces shot
dead 59 assailants, the reports said.

An official named Wang Yongzhi was appointed to run Yarkand soon
afterward. With his glasses and crew cut, he looked the picture of a
party technocrat. He had grown up and spent his career in southern
Xinjiang and was seen as a deft, seasoned official who could deliver on
the party's top priorities in the area: economic development and firm
control of the Uighurs.

But among the most revealing documents in the leaked papers are two that
describe Mr. Wang's downfall --- an 11-page report summarizing the
party's internal investigation into his actions, and the text of a
15-page confession that he may have given under duress. Both were
distributed inside the party as a warning to officials to fall in line
behind the crackdown.

Han officials like Mr. Wang serve as the party's anchors in southern
Xinjiang, watching over Uighur officials in more junior positions, and
he seemed to enjoy the blessing of top leaders, including Yu Zhengsheng,
then China's most senior official for ethnic issues, who visited the
county in 2015.

Mr. Wang set about beefing up security in Yarkand but he also pushed
economic development to address ethnic discontent. And he sought to
soften the party's religious policies, declaring that there was nothing
wrong with having a Quran at home and encouraging party officials to
read it to better understand Uighur traditions.

When the mass detentions began, Mr. Wang did as he was told at first and
appeared to embrace the task with zeal.

He built two sprawling new detention facilities, including one as big as
50 basketball courts, and herded 20,000 people into them.

He sharply increased funding for the security forces in 2017, more than
doubling spending on outlays such as checkpoints and surveillance to
1.37 billion renminbi, or about \$180 million.

And he lined up party members for a rally in a public square and urged
them to press the fight against terrorists. ``Wipe them out
completely,'' he said. ``Destroy them root and branch.''

But privately, Mr. Wang had misgivings, according to the confession that
he later signed, which would have been carefully vetted by the party.

He was under intense pressure to prevent an outburst of violence in
Yarkand, and worried the crackdown would provoke a backlash.

The authorities set numeric targets for Uighur detentions in parts of
Xinjiang, and while it is unclear if they did so in Yarkand, Mr. Wang
felt the orders left no room for moderation and would poison ethnic
relations in the county.

He also worried that the mass detentions would make it impossible to
record the economic progress he needed to earn a promotion.

The leadership had set goals to reduce poverty in Xinjiang. But with so
many working-age residents being sent to the camps, Mr. Wang was afraid
the targets would be out of reach, along with his hopes for a better
job.

His superiors, he wrote, were ``overly ambitious and unrealistic.''

``The policies and measures taken by higher levels were at gaping odds
with realities on the ground and could not be implemented in full,'' he
added.

To help enforce the crackdown in southern Xinjiang, Mr. Chen transferred
in hundreds of officials from the north. Publicly, Mr. Wang welcomed the
62 assigned to Yarkand. Privately, he seethed that they did not
understand how to work with local officials and residents.

The pressure on officials in Xinjiang to detain Uighurs and prevent
fresh violence was relentless, and Mr. Wang said in the confession ---
presumably signed under pressure --- that he drank on the job. He
described one episode in which he collapsed drunk during a meeting on
security.

``While reporting on my work in the afternoon meeting, I rambled
incoherently,'' he said. ``I'd just spoken a few sentences and my head
collapsed on the table. It became the biggest joke across the whole
prefecture.''

Thousands of officials in Xinjiang were punished for resisting or
failing to carry out the crackdown with sufficient zeal. Uighur
officials were accused of protecting fellow Uighurs, and Gu Wensheng,
the Han leader of another southern county, was jailed for trying to slow
the detentions and shield Uighur officials, according to the documents.

Secret teams of investigators traveled across the region identifying
those who were not doing enough. In 2017, the party opened more than
12,000 investigations into party members in Xinjiang for infractions in
the ``fight against separatism,'' more than 20 times the figure in the
previous year, according to official statistics.

Mr. Wang may have gone further than any other official.

Quietly, he ordered the release of more than 7,000 camp inmates --- an
act of defiance for which he would be detained, stripped of power and
prosecuted.

\includegraphics{https://static01.graylady3jvrrxbe.onion/newsgraphics/2019/10/21/xinjiang-article/13c819960fa0d589a50c0179792cdbf8c4b72c5b/doc6-highlight-01.jpg}

``\textbf{I undercut, acted selectively and made my own adjustments,
believing that rounding up so many people would knowingly fan conflict
and deepen resentment},'' Mr. Wang wrote.

\includegraphics{https://static01.graylady3jvrrxbe.onion/newsgraphics/2019/10/21/xinjiang-article/13c819960fa0d589a50c0179792cdbf8c4b72c5b/doc6-highlight-02.jpg}

``\textbf{Without approval and on my own initiative},'' he added,
``\textbf{I broke the rules.}''

​

\hypertarget{-brazen-defiance-}{%
\paragraph{ Brazen Defiance }\label{-brazen-defiance-}}

Mr. Wang quietly disappeared from public view after September 2017.

About six months later, the party made an example of him, announcing
that he was being investigated for ``gravely disobeying the party
central leadership's strategy for governing Xinjiang.''

The internal report on the investigation was more direct. ``He should
have given his all to serving the party,'' it said. ``Instead, he
ignored the party central leadership's strategy for Xinjiang, and he
went as far as brazen defiance.''

Both the report and Mr. Wang's confession were read aloud to officials
across Xinjiang. The message was plain: The party would not tolerate any
hesitation in carrying out the mass detentions.

Propaganda outlets described Mr. Wang as irredeemably corrupt, and the
internal report accused him of taking bribes on construction and mining
deals and paying off superiors to win promotions.

The authorities also emphasized he was no friend of Uighurs. To hit
poverty-reduction targets, he was said to have forced 1,500 families to
move into unheated apartments in the middle of the winter. Some
villagers burned wood indoors to keep warm, leading to injuries and
deaths, his confession said.

But Mr. Wang's greatest political sin was not revealed to the public.
Instead, the authorities hid it in the internal report.

``He refused,'' it said, ``to round up everyone who should be rounded
up.''

Design and development by Rebecca Lieberman. Additional production by
Jessica White.

To omit identifying markings, these documents have been retyped to
resemble the originals.

Read 371 Comments

\begin{itemize}
\item
\item
\item
\item
\end{itemize}

Advertisement

\protect\hyperlink{after-bottom}{Continue reading the main story}

\hypertarget{site-index}{%
\subsection{Site Index}\label{site-index}}

\hypertarget{site-information-navigation}{%
\subsection{Site Information
Navigation}\label{site-information-navigation}}

\begin{itemize}
\tightlist
\item
  \href{https://help.nytimes3xbfgragh.onion/hc/en-us/articles/115014792127-Copyright-notice}{©~2020~The
  New York Times Company}
\end{itemize}

\begin{itemize}
\tightlist
\item
  \href{https://www.nytco.com/}{NYTCo}
\item
  \href{https://help.nytimes3xbfgragh.onion/hc/en-us/articles/115015385887-Contact-Us}{Contact
  Us}
\item
  \href{https://www.nytco.com/careers/}{Work with us}
\item
  \href{https://nytmediakit.com/}{Advertise}
\item
  \href{http://www.tbrandstudio.com/}{T Brand Studio}
\item
  \href{https://www.nytimes3xbfgragh.onion/privacy/cookie-policy\#how-do-i-manage-trackers}{Your
  Ad Choices}
\item
  \href{https://www.nytimes3xbfgragh.onion/privacy}{Privacy}
\item
  \href{https://help.nytimes3xbfgragh.onion/hc/en-us/articles/115014893428-Terms-of-service}{Terms
  of Service}
\item
  \href{https://help.nytimes3xbfgragh.onion/hc/en-us/articles/115014893968-Terms-of-sale}{Terms
  of Sale}
\item
  \href{https://spiderbites.nytimes3xbfgragh.onion}{Site Map}
\item
  \href{https://help.nytimes3xbfgragh.onion/hc/en-us}{Help}
\item
  \href{https://www.nytimes3xbfgragh.onion/subscription?campaignId=37WXW}{Subscriptions}
\end{itemize}
