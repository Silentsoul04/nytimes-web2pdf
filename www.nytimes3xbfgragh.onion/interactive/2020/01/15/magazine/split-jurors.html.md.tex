Sections

SEARCH

\protect\hyperlink{site-content}{Skip to
content}\protect\hyperlink{site-index}{Skip to site index}

\href{https://www.nytimes3xbfgragh.onion/section/magazine}{Magazine}

\href{https://myaccount.nytimes3xbfgragh.onion/auth/login?response_type=cookie\&client_id=vi}{}

\href{https://www.nytimes3xbfgragh.onion/section/todayspaper}{Today's
Paper}

\hypertarget{comments}{%
\subsection{\texorpdfstring{\protect\hyperlink{commentsContainer}{Comments}}{Comments}}\label{comments}}

\href{}{Two Jurors Voted to Acquit. He Was Convicted of Murder
Anyway.}\href{}{Skip to Comments}

The comments section is closed. To submit a letter to the editor for
publication, write to
\href{mailto:letters@NYTimes.com}{\nolinkurl{letters@NYTimes.com}}.

\hypertarget{two-jurors-voted-to-acquit-he-was-convicted-of-murder-anyway}{%
\section{Two Jurors Voted to Acquit. He Was Convicted of Murder
Anyway.}\label{two-jurors-voted-to-acquit-he-was-convicted-of-murder-anyway}}

By \href{https://www.nytimes3xbfgragh.onion/by/emily-bazelon}{Emily
Bazelon}Jan. 15, 2020

\begin{itemize}
\item
\item
\item
\item
\item
  \emph{53}
\end{itemize}

\hypertarget{in-2011-michael-shannon-was-wrongly-convicted-of-murder-even-though-two-jurors-voted-to-acquit-him--a-result-of-a-louisiana-law-rooted-in-discrimination}{%
\subsection{In 2011, Michael Shannon was wrongly convicted of murder,
even though two jurors voted to acquit him --- a result of a Louisiana
law rooted in
discrimination.}\label{in-2011-michael-shannon-was-wrongly-convicted-of-murder-even-though-two-jurors-voted-to-acquit-him--a-result-of-a-louisiana-law-rooted-in-discrimination}}

\includegraphics{https://static01.graylady3jvrrxbe.onion/packages/flash/multimedia/ICONS/transparent.png}

\includegraphics{https://static01.graylady3jvrrxbe.onion/images/2020/01/19/magazine/19mag-splitjurors/19mag-splitjurors-master1050-v2.jpg}

Michael Shannon in November 2019.

\hypertarget{but-for-defendants-like-shannon--and-the-holdout-jurors-who-believed-in-their-innocence--it-has-left-a-bitter-legacy}{%
\subsection{But for defendants like Shannon --- and the holdout jurors
who believed in their innocence --- it has left a bitter
legacy.}\label{but-for-defendants-like-shannon--and-the-holdout-jurors-who-believed-in-their-innocence--it-has-left-a-bitter-legacy}}

Photo portfolio by Larry Fink

\hypertarget{shadow-of-a-doubt}{%
\subsection{Shadow of a Doubt}\label{shadow-of-a-doubt}}

\hypertarget{i}{%
\subsection{I.}\label{i}}

When Mi Wha Morrison sat on the jury of a murder trial, in February
2011, the testimony that mattered most came from the prosecution's
single eyewitness, Emma Bourgoyne. The jurors listened to Bourgoyne say
she was sitting in the passenger seat of a car her husband was driving,
on a November day nearly seven years earlier, when the couple stopped at
a red light near a highway exit in New Orleans. Bourgoyne noticed three
men chatting by a curb outside an Exxon Station, their motorcycles
parked nearby. All of a sudden, a fourth man came up behind them,
Bourgoyne said. He raised his hand and fired a gun. ``The man in the
middle fell over,'' she told the jury.

In the courtroom, the prosecutor asked Bourgoyne if she saw the shooter,
who ran off after the killing. ``He's the gentleman sitting in the
white-and-black plaid shirt over at that table,'' she said, shaking with
emotion and pointing to the defendant, Michael Shannon. Bourgoyne said
she picked Shannon's photo out of a lineup nearly six months after the
murder. She was ``a hundred percent sure'' she had the right man.

But Morrison, who worked as a teacher, was surprised to hear Bourgoyne,
who was in her 60s, say she'd seen the shooter for only 15 to 20 seconds
--- and to hear that she told a prosecutor after the murder that she
hadn't been wearing her glasses. Morrison thought about whether she'd be
able to identify a person she'd seen briefly, from inside a car, six
months after the fact. She was 34 and had perfect vision. It seemed like
a stretch.

On cross-examination, Bourgoyne admitted that she told the police the
shooter was about six feet tall and stocky, with a neck like a football
player. When Shannon stood up, Morrison saw that he was at least six
inches shorter, at 5 foot 6, and slight of build. It mattered to her too
that Shannon was black and Bourgoyne was white.
(\href{https://www.ncbi.nlm.nih.gov/pmc/articles/PMC3496291/}{Research
shows that it's harder for witnesses to identify people of a different
race.})

Morrison wanted to hear from other eyewitnesses to the killing. Earlier
in the trial, the homicide detective who investigated the shooting
testified that he'd interviewed the two men standing next to the victim,
a 46-year-old named Ralph Cole, and four additional people who were
nearby when the shooting took place. But when Bourgoyne finished
testifying, the prosecution rested. Shannon's lawyer called no
witnesses. The trial ended after only about five hours.

\includegraphics{https://static01.graylady3jvrrxbe.onion/packages/flash/multimedia/ICONS/transparent.png}

\includegraphics{https://static01.graylady3jvrrxbe.onion/images/2020/01/19/magazine/19mag-splitjurors-02/19mag-splitjurors-02-master1050-v2.jpg}

Mi Wha Morrison, one of the two dissenting jurors, with Shannon.

On her way to deliberate with the other jurors, Morrison thought they
would easily agree: There wasn't enough proof to send Michael Shannon to
prison. But in the room, Morrison says, the foreman, who was white, took
charge and argued strongly for a guilty verdict. As Morrison remembers
it, the conversation veered from the facts of the case to the spike in
crime in New Orleans after Hurricane Katrina. The storm hit the city in
August 2005, nine months after Cole's murder, and its only bearing on
the trial was the yearslong delay the storm caused. But as the jurors
­talked about Ralph Cole's killing --- his mother had described him
working three jobs and raising a 6-year-old son with his wife --- they
began bringing up other crime victims they knew, as if they were somehow
relevant to the guilt or innocence of Michael Shannon. (Other jurors
couldn't remember specific details or couldn't be reached.)

When Morrison expressed her doubts about Bourgoyne's identification of
Shannon, an African-American woman on the jury agreed that the evidence
was not strong enough, according to Morrison. But a second black woman
was ``adamant about Shannon's guilt,'' Morrison told me, remembering her
saying, ``If the prosecutor said he should be off the streets, then we
needed to put him away.'' Morrison continued, ``I felt like they were
bringing in these outside frustrations without looking much at the facts
of this particular case.''

After about half an hour of deliberation, Morrison recalled, the foreman
went around the table. Ten members of the jury agreed to find Shannon
guilty of second-degree murder. Morrison and her ally stood their
ground. In almost any other state, the jurors would have continued to
deliberate. If they truly couldn't agree, the case would have ended in a
mistrial, because only a unanimous jury would have the power to convict.
But in Louisiana in 2011, a vote of 10 to 2 (or 11 to 1) was enough.

When the judge praised the two prosecutors for ``one of the quickest
homicide cases I've had,'' Morrison felt a surge of anger. Afterward,
whenever she drove by the corner where Ralph Cole was killed, she was
haunted by the photographs she'd seen of his body lying on the ground.
From time to time, she also thought about how thwarted she felt in the
jury room.

Last October,
\href{https://www.supremecourt.gov/oral_arguments/argument_transcripts/2019/18-5924_4gcj.pdf}{the
Supreme Court heard arguments} about whether split verdicts like the one
in Shannon's case violate the Constitution, and its decision is expected
in the coming weeks or months. If the justices end the practice, they
will finally close a chapter in American jurisprudence, in which two
states --- because of laws based in discrimination --- have for decades
been allowed to disregard a fundamental premise of our legal system. A
result, for some defendants like Shannon, has been years behind bars for
crimes they didn't commit. And for jurors like Morrison, forced to stand
by as convictions take place over their serious objections, a result has
been a feeling of powerlessness --- as if justice had been denied not
merely to the defendants but, in a sense, to them too.

\hypertarget{shannons-case-is-far-from-the-only-one-we-asked-the-photographer-larry-fink-to-make-portraits-of-others-who-were-convicted-by-non-unanimous-juries-and-later-exonerated--along-with-jurors-from-their-cases-here-are-their-stories}{%
\subsection{Shannon's case is far from the only one. We asked the
photographer Larry Fink to make portraits of others who were convicted
by non-unanimous juries and later exonerated --- along with jurors from
their cases. Here are their
stories.}\label{shannons-case-is-far-from-the-only-one-we-asked-the-photographer-larry-fink-to-make-portraits-of-others-who-were-convicted-by-non-unanimous-juries-and-later-exonerated--along-with-jurors-from-their-cases-here-are-their-stories}}

\hypertarget{kia-stewart-and-chris-steinmetz}{%
\subsection{Kia Stewart and Chris
Steinmetz}\label{kia-stewart-and-chris-steinmetz}}

\includegraphics{https://static01.graylady3jvrrxbe.onion/packages/flash/multimedia/ICONS/transparent.png}

\includegraphics{https://static01.graylady3jvrrxbe.onion/images/2020/01/19/magazine/19mag-splitjurors-06/19mag-splitjurors-06-master1050-v2.jpg}

Kia Stewart in November 2019.

In August 2005, when Stewart was 17, he turned himself in after learning
he was wanted on a murder charge. A witness picked his photo out of a
lineup. Stewart thought it would be easy to show that was a mistake.
Other witnesses named a different suspect. But New Orleans was evacuated
for Katrina a month after the killing, and the police didn't
meaningfully pursue those leads. Stewart was shuttled around the state
because of the hurricane; at one facility, an inmate attacked him with a
machete, leaving him with 16 staples in his scalp. When Stewart's case
finally went to trial in 2009, the jury found him guilty by a vote of 10
to 2.

After a decade behind bars, he was released in 2015, when the Innocence
Project New Orleans, led by Emily Maw and with the help of Stewart's
family,
\href{https://ip-no.org/what-we-do/free-innocent-prisoners/client-profiles/kia-stewart/}{presented
18 witnesses} with direct evidence that Stewart was not the shooter. In
2018, in an essay for The Marshall Project, a juror who voted to convict
said she regretted her decision; she'd interpreted Stewart's body
language during the trial as indifference, which she also regretted.
When I asked Stewart about this, he said that he avoided eye contact
with the jury because ``I knew they got my life in their hands.'' He
added in frustration: ``What am I supposed to do? Be sitting at the
bench constantly crying?''

Steinmetz was one of the two jurors who voted to acquit Stewart. He said
that initially four or five other jurors opposed conviction. But another
juror argued for the reliability of the prosecution's sole eyewitness.
``He said, `` `You have to be willing to risk your own safety to
testify,' '' Steinmetz said. ``That flipped people.''

\includegraphics{https://static01.graylady3jvrrxbe.onion/packages/flash/multimedia/ICONS/transparent.png}

\includegraphics{https://static01.graylady3jvrrxbe.onion/images/2020/01/19/magazine/19mag-splitjurors-05/19mag-splitjurors-05-master1050-v2.jpg}

Chris Steinmetz, one of the dissenting jurors.

\hypertarget{reginald-adams-and-yvette-bowens}{%
\subsection{Reginald Adams and Yvette
Bowens}\label{reginald-adams-and-yvette-bowens}}

\includegraphics{https://static01.graylady3jvrrxbe.onion/packages/flash/multimedia/ICONS/transparent.png}

\includegraphics{https://static01.graylady3jvrrxbe.onion/images/2020/01/19/magazine/19mag-splitjurors-09/19mag-splitjurors-09-master1050-v2.jpg}

Reginald Adams in November 2019.

In 1980, Adams falsely confessed to killing the wife of a police
officer, after other New Orleans police officers gave him Valium and
alcohol during his interrogation. A jury convicted Adams of murder, 10
to 2. In 2006, the police officer whose wife was killed was found guilty
of murdering his second wife. A few years later,
\href{https://ip-no.org/what-we-do/free-innocent-prisoners/client-profiles/reginald-adams/}{the
Innocence Project New Orleans investigated} and found a serious
violation at Adams's trial: The jury never learned that before the
police questioned Adams, they had already found the murder weapon and
traced it to two other people, one of whom was in possession of jewelry
that belonged to the victim. Adams was exonerated and released from
Angola prison in 2014 after nearly 34 years.

In November, Adams and Bowens, one of the 10 jurors who voted to convict
him, agreed to meet and be photographed. The first thing Adams said to
Bowens was, ``In light of everything, taken into consideration, I'm not
mad at you.''

If the police and the lawyers ``had done what they was supposed to do, I
think it would have had a different outcome,'' Bowens said. When Adams
told her he'd been at Angola, she said she'd visited the prison a couple
of times for a crafts fair where prisoners sell their wares. ``I used to
make pocketbooks,'' Adams said. ``You might have seen some of my stuff
out there.''

\includegraphics{https://static01.graylady3jvrrxbe.onion/packages/flash/multimedia/ICONS/transparent.png}

\includegraphics{https://static01.graylady3jvrrxbe.onion/images/2020/01/19/magazine/19mag-splitjurors-11/19mag-splitjurors-11-master1050-v2.jpg}

Yvette Bowens, a juror, and Adams.

\hypertarget{joann-rockett-miller-and-gerald-burge}{%
\subsection{Joann Rockett Miller and Gerald
Burge}\label{joann-rockett-miller-and-gerald-burge}}

\includegraphics{https://static01.graylady3jvrrxbe.onion/packages/flash/multimedia/ICONS/transparent.png}

\includegraphics{https://static01.graylady3jvrrxbe.onion/images/2020/01/19/magazine/19mag-splitjurors-08/19mag-splitjurors-08-master1050-v2.jpg}

Joann Rockett Miller, one of the dissenting jurors.

Serving on a jury in Covington, La., in 1986, Miller held out, along
with a second female juror, against convicting Burge on a murder charge.
As she remembers it, the foreman bullied a third woman on the jury into
switching sides. ``He was going at us, and she just burst into tears,''
Miller said. ``In the jury room, it is not that easy to stand up to a
group of peers and especially to a group of men.'' She added, ``You're
going to have to know what you're doing and why you're doing it.''

In 1992 Burge was found not guilty in a new trial, after it came to
light that
\href{https://caselaw.findlaw.com/la-supreme-court/1555524.html}{a
police officer had hidden exonerating evidence}. Burge won a settlement
for his wrongful conviction, but some people in his family remain
uncertain of his innocence, he said. ``Everyone around here thinks I got
out on a technicality. The ones who really know me, they say we know you
had nothing to do with it. But other people form an opinion, and it's
something you can't live down.'' In 2016, the district attorney who
prosecuted Burge, Walter Reed, was
\href{https://www.nola.com/news/courts/article_08cb561f-eadc-579f-9cb2-3f023cbe3575.html}{convicted
on 18 counts of corruption} and fraud and later sentenced to four years
in prison.

\includegraphics{https://static01.graylady3jvrrxbe.onion/packages/flash/multimedia/ICONS/transparent.png}

\includegraphics{https://static01.graylady3jvrrxbe.onion/images/2020/01/19/magazine/19mag-splitjurors-07/19mag-splitjurors-07-master1050-v2.jpg}

Gerald Burge in November 2019.

\hypertarget{ii}{%
\subsection{II.}\label{ii}}

In England, the birthplace of the modern jury, the need for unanimous
verdicts was established in 1367,
\href{https://www.supremecourt.gov/DocketPDF/18/18-5924/103406/20190618182217707_No\%2018-5924\%20Ramos\%20v\%20\%20Louisiana\%20tsac\%20amicus\%20brief.pdf}{when
a court refused to accept an 11-to-1 guilty vote}, after one juror said
he'd rather die in prison than consent to convict. The practice spread
to the American colonies along with many other elements of British
common law. The Constitution provided for the right to trial by jury
\href{https://constitutioncenter.org/interactive-constitution/amendment/amendment-vi}{in
the Sixth Amendment}. The text uses the word ``impartial,'' not
``unanimous,'' but judges of the founding era commonly issued
instructions saying that unanimity was part of the jury right, and
commentators agreed. ``It is the unanimity of the jury that preserves
the rights of mankind,'' John Adams, the future president, wrote in
1786. In a line of cases
\href{https://supreme.justia.com/cases/federal/us/170/343/}{from 1898}
\href{https://supreme.justia.com/cases/federal/us/333/740/}{to 1948},
the Supreme Court said that the Sixth Amendment required unanimous
verdicts.

In many countries, judges, not juries, decide a defendant's guilt.
Others ---
\href{https://ir.lawnet.fordham.edu/cgi/viewcontent.cgi?article=1084\&context=faculty_scholarship}{France,
Italy}
\href{https://academic.oup.com/ssjj/article-abstract/19/1/116/2451767?redirectedFrom=fulltext}{and
Japan}, for example --- rely on a mixed group of laypeople and judges to
deliver a verdict, without requiring consensus. But these are countries
in which prosecutors play a more neutral fact-finding role. In the
United States and Canada, the unanimous jury, like the high bar of proof
beyond a reasonable doubt, remains a bulwark against state power and the
risk of convicting the innocent. It also helps ensure that the jury
represents a cross-section of the community: If every vote counts,
\href{https://www.law.cornell.edu/supremecourt/text/406/356\#writing-USSC_CR_0406_0356_ZD1}{then
the majority can't ignore the minority.}

Louisiana and Oregon are the only states (along with Puerto Rico) that
have deviated from unanimous jury convictions. In Louisiana, black
people began to serve on juries during Reconstruction, when they won the
right to vote. The first introduction of split verdicts in the state was
in 1880, after Reconstruction ended, as
\href{https://www.washingtonpost.com/opinions/these-jury-systems-are-vestiges-of-white-supremacy/2017/09/22/d7f1897a-9f13-11e7-9c8d-cf053ff30921_story.html}{part
of a movement of white Southerners} to re-establish their supremacy. A
trial of a black person with ``Negro jurors would be a farce,'' a
Louisiana newspaper called The Weekly Messenger editorialized in 1893.
``Must we permit our women and even female children to live in constant
peril of outrage?''

In 1898, Louisiana
\href{https://babel.hathitrust.org/cgi/pt?id=njp.32101065310607\&view=1up\&seq=7}{held
a state constitutional convention}. The delegates asked for a racial
breakdown of the voting rolls and learned that about 15 percent of
eligible voters, and thus jurors, were black. The delegates then voted
to allow convictions in 12-juror trials even if three jurors dissented
--- permitting the majority to override the opinion of a likely number
of black jurors at any trial. (In 1973, at another state constitutional
convention, Louisiana increased, from nine to 10, the number of jurors
required for conviction for crimes that carry a penalty of hard labor
and imprisonment.)

In Oregon, nonunanimous juries stemmed from another prejudice:
\href{https://scholarsbank.uoregon.edu/xmlui/bitstream/handle/1794/22247/Kaplan.pdf?sequence=1\&isAllowed=y}{anti-Semitism,
along with xenophobia}. In 1933, a Jewish hotel owner was
\href{https://www.oregonlive.com/pacific-northwest-news/2017/09/inside_the_1933_murder_trial_t.html}{charged
with the murder} of two white Protestants in a gangland-style execution.
In a sensationalized trial, the jury returned a verdict of manslaughter
rather than second-degree murder. The Morning Oregonian published
editorials calling for nonunanimous juries in order to cope with ``vast
immigration into America from Southern and Eastern Europe'' --- the
latter region was where many Jewish newcomers originated --- ``of people
untrained in the jury system.'' In a special election in May 1934, on a
state constitutional amendment, voters approved a 10-out-of-12 rule for
jury convictions for all crimes except first-degree murder.

In keeping with their origins, laws permitting nonunanimous juries tend
to disadvantage minority jurors and defendants most. Prosecutors strike
black people from juries at more than twice the rate that they strike
white people, according to
\href{https://www.nola.com/news/courts/article_64f67fc8-9ab4-56b6-bb45-598b6795cffa.html}{a
2018 Pulitzer Prize-winning investigation by The Advocate}, Louisiana's
largest newspaper.
\href{https://www.nola.com/news/courts/article_8e284de1-9c5c-5d77-bcc5-6e22a3053aa0.html}{In
one of its reports}, the newspaper reviewed 46 cases, and it found that
when black jurors were seated, they were nearly three times as likely to
vote not guilty as white jurors if the verdict was not unanimous. And in
a set of nearly 1,000 cases, a greater share of black defendants (more
than four out of 10) were convicted by split juries than white
defendants (more than three out of 10).

The quality of a jury's deliberation also suffers.
\href{https://www.ucl.ac.uk/human-resources/sites/human-resources/files/sommers_2006_-_identifying_multiple_effects_of_racial_composition_on_jury_deliberations.pdf}{Studies
show} that on racially mixed juries, white as well as black jurors
describe the facts of the case more accurately and are more systematic
about going through the evidence. The benefit may be lost, however, when
the majority on a jury can simply bypass the minority.
\href{https://global.oup.com/academic/product/the-jury-under-fire-9780190201340?cc=us\&lang=en\&}{Research}
\href{https://espace.library.uq.edu.au/view/UQ:335118}{shows} that
requiring juries to be unanimous tends to ensure more reliable
decisions. Reaching consensus is a burden, but it also offers
``opportunities for jurors to correct memory errors and mistaken
interpretations,'' Valerie Hans, a Cornell law professor who studies
juries, wrote in
\href{https://scholarship.law.cornell.edu/cgi/viewcontent.cgi?article=1381\&context=facpub}{a
2007 review} of the research in The Chicago-Kent Law Review. ``The group
discussion helps jurors individually and collectively to clarify their
positions and conclusions and increases their certainty that they are
reaching the right verdict.''

The lower threshold for a jury conviction affects cases in other ways.
In Louisiana, as in most states, more than 95 percent of prison
sentences result from plea deals, not jury trials. Prosecutors know that
they can afford to lose the votes of two jurors and still win. That
changes the calculus for assessing the quality of the police work they
bring to trial --- and, by extension, how tough prosecutors can be in
striking plea bargains. Prosecutors have more leverage to insist on
murder charges rather than manslaughter, for example, jacking up prison
sentences.

Louisiana's lower standard for jury convictions operates alongside other
relatively weak protections for the innocent. Some district attorney's
offices,
\href{https://www.washingtonpost.com/news/the-watch/wp/2015/10/27/new-orleanss-persistent-prosecutor-problem/}{including
the one in New Orleans}, have a history of failing to disclose
exonerating evidence, including in death-penalty cases. The Louisiana
Supreme Court
\href{https://www.courtlistener.com/opinion/1677614/state-v-peart/}{warned
in 1993} of ``widely disparate, nonuniform and totally inadequate
funding'' for defense lawyers for people who couldn't afford to pay for
them. After Katrina, a public defender's office opened in New Orleans,
but caseloads have been
\href{https://www.nytimes3xbfgragh.onion/2016/02/19/opinion/when-the-public-defender-says-i-cant-help.html}{far
above the national recommended standard}.

All told, Louisiana's system has produced the highest incarceration rate
in the country and
\href{https://www.theadvocate.com/baton_rouge/news/article_f6309822-17ac-11ea-8750-f7d212aa28f8.html}{one
of the highest rates of life sentences}. Life is the automatic
punishment for second-degree murder (a charge that applies to
accomplices who did not pull the trigger) and an available punishment
for kidnapping or rape. Defendants can also be sentenced to virtual life
--- 50 years or more --- for armed robbery, even if it's a first
offense.

Six thousand people in Louisiana are serving sentences of either life or
50 years or more, a higher number than in Texas, Arkansas, Mississippi,
Alabama and Tennessee combined,
\href{https://www.theadvocate.com/baton_rouge/news/article_f6309822-17ac-11ea-8750-f7d212aa28f8.html}{The
Advocate pointed out} in its 2018 investigation. There's no mercy on the
back end: The Louisiana Legislature abolished parole for all life
sentences in 1979.

And over the last generation, Louisiana has been among the top five
states for one more alarming indicator: the rate of wrongful
convictions. A total of 43 people in the state have been exonerated
since 2000, according to the Innocence Project New Orleans --- and split
juries are one reason.

\hypertarget{iii}{%
\subsection{III.}\label{iii}}

Calvin Duncan began to see a connection between split juries and bad
verdicts not long after becoming an inmate at Louisiana's Angola prison.
In 1985, when he was 24, he was convicted of murder and sentenced to
life. At Angola,
\href{https://www.nytimes3xbfgragh.onion/2019/08/05/us/politics/supreme-court-nonunanimous-juries.html}{Duncan
trained to be a jailhouse lawyer}, or ``inmate counsel substitute,'' as
the state job, which paid 20 cents an hour, was called. In a small booth
in the prison's law library, where he worked until midnight on behalf of
fellow inmates who were trying to appeal their convictions, Duncan found
that juries often voted 10 to 2 or 11 to 1 when he had reason to think,
based on evidence the jury might not have seen, that the guilty verdict
was mistaken. ``I'd see there were one or two jurors who had it right in
cases when there was a great possibility that someone was innocent,'' he
said when we talked on the phone in December.

Duncan did some research to determine whether the constitutionality of a
state's split-jury law had ever been disputed. He found a Supreme Court
case from 1972, \href{https://www.oyez.org/cases/1971/69-5046}{Apodaca
v. Oregon}, in which a defendant challenged his 10-to-2 guilty verdict.
The justices divided in a distinctive way. Four of them said the Sixth
Amendment right to trial by jury did not require unanimous verdicts.
Five justices said it did, sticking with the previous Supreme Court
statements to that effect. But one of the five, Lewis Powell, said that
the Sixth Amendment did not apply to proceedings in state court.
Powell's fifth vote allowed nonunanimous jury laws to remain in force.
It also created an exception to the general rule of applying the Bill of
Rights to the states as well as the federal government.

Duncan knew that because of Apodaca, lower courts were bound to reject
future challenges to split verdicts. Still, he started including such
claims at the end of the appeals he helped file, in case something
changed.

In 2000, something did. In the case
\href{https://www.oyez.org/cases/1999/99-478}{Apprendi v. New Jersey},
the Supreme Court implicitly rejected Justice Powell's logic in Apodaca.
The justices ruled that juries, not judges, have to determine any fact
that increases the penalty for a crime beyond the maximum prescribed by
statute. And they said that defendants had this right in state court as
well as federal because the Sixth Amendment right to trial by jury
applied to the states via the Fourteenth Amendment's right to due
process.

When Duncan met a lawyer named Ben Cohen, who represents people on
Louisiana's death row, they started talking about the weakness of
split-jury verdicts and revisiting the Supreme Court's 1972 ruling in
Apodaca. ``Calvin made me understand this as a civil rights issue
affecting many, many people,'' Cohen told me. ``I said, any case you
bring me, I'll take it to the Supreme Court.'' He started filing
petitions in 2004 asking the court to decide anew whether nonunanimous
juries were constitutional. But without explaining why, the Supreme
Court rejected case after case.

In 2011, Duncan --- who always maintained his innocence --- was released
on a plea deal negotiated by the Innocence Project New Orleans. In the
library at the Louisiana Supreme Court, studying for an undergraduate
degree at Tulane in preparation for law school, he read an article about
the racist 19th-century origins of Louisiana's nonunanimous juries. He
started traveling the state to talk about changing the jury law wherever
he could: at community gatherings, at law schools, at meetings of the
Louisiana bar.

Soon Duncan was working with Cohen at the Promise of Justice Initiative,
a nonprofit organization focused on criminal-justice reform. With the
civil rights group VOTE, led by formerly incarcerated people, they
started a coalition to change Louisiana's jury law. Americans for
Prosperity, the political advocacy group funded by the Koch brothers,
signed on in 2018 in support of a November ballot measure to amend the
state constitution to eliminate split juries. At first judges and
\href{https://www.nola.com/news/courts/article_8e284de1-9c5c-5d77-bcc5-6e22a3053aa0.html}{some
prosecutors defended the law} in the interest of ``efficiency,'' as a
check on mistrials. But then the Louisiana District Attorneys
Association dropped its opposition to the measure, and the ballot
initiative
\href{https://www.nola.com/news/politics/article_e36481c9-b8be-5f80-a11e-46daee2d3974.html}{passed
with 64 percent of the vote}.

The law went into effect in January 2019. But it applies only to
defendants who are charged with crimes going forward. Thousands of
people remain in prison, convicted in nonunanimous verdicts of the past.
Estimates suggest that about 2,400 of them are serving life in
Louisiana.

Duncan and Cohen continued to petition the Supreme Court to open the
door to new trials for past nonunanimous verdicts. Last March, the court
agreed to hear the 24th petition Cohen brought, in the case
\href{https://www.scotusblog.com/case-files/cases/ramos-v-louisiana/}{Ramos
v. Louisiana}, in which a man was convicted by a 10-to-2 jury of killing
a woman in New Orleans in 2014, even though there was
\href{https://www.supremecourt.gov/DocketPDF/18/18-5924/73345/20181127123743863_18.5924.Ramos.Louisiana.Reply.pdf}{no
eyewitness testimony or direct physical evidence} tying him to the
crime.

The court's decision to grant review followed the arrival of two new
justices, Neil M. Gorsuch and Brett Kavanaugh, who have shown an
interest in applying the Bill of Rights uniformly to the states --- a
cause on the right since 2010, when
\href{https://www.oyez.org/cases/2009/08-1521}{a conservative majority
on the court ruled} that states are bound by the right to bear arms in
the Second Amendment. Last term,
\href{https://www.supremecourt.gov/opinions/18pdf/17-1091_5536.pdf}{the
court said unanimously} that states must comply with the Eighth
Amendment's protection against excessive fees and fines, and noted that
Louisiana and Oregon's anomalous jury rules were now the sole exceptions
to incorporating the Bill of Rights into state law.

During argument in the Ramos case in October, the court's
liberal-moderate justices pointed out the problems with split-jury
verdicts. Gorsuch asked whether the court should ``forever ensconce an
incorrect view of the United States Constitution for perpetuity, for all
states and all people, denying them a right that we believe was
originally given to them'' for the sake of Louisiana's criminal
convictions --- in other words, whether the court should misread the
Sixth Amendment to spare a state from revisiting its past verdicts.
``Can I pick up on Justice Gorsuch's question?'' Kavanaugh continued.
``There are defendants who have been convicted and sentenced to life, 10
to 2 or 11 to 1, who otherwise would have not been convicted. So that
seems like a serious issue for us to think about.''

Kavanaugh also brought up the origins of Louisiana's split-jury law.
``The rule in question here is rooted in a --- in racism, you know,''
Kavanaugh said, ``rooted in a desire, apparently, to diminish the voices
of black jurors in the late 1890s.'' In Ramos, only about a few dozen
convictions are directly at stake, because the appeals process is over
in all the other cases of past split-jury verdicts. The court could find
nonunanimous juries unconstitutional and leave for another day the
broader scope of its ruling, and the prospects of a new trial for
thousands of prisoners in Louisiana.

\hypertarget{iv}{%
\subsection{IV.}\label{iv}}

Michael Shannon was 47 when he was sentenced to life in April 2011; he
arrived at Angola prison three months after Duncan left. As the deadline
for challenging his conviction bore down on him --- he had only months
to file his first appeal --- he began attending a Saturday prison law
class. His teachers were jailhouse lawyers whom Duncan helped train
before his release. They took note of Shannon's split-jury verdict ---
``The guys in the law class were pushing real hard to change the 10-to-2
rule,'' he remembered --- and they helped him identify the flaws that
riddled his trial.

In a police report in his case file, Shannon found notes on interviews
with six eyewitnesses to Ralph Cole's murder who said either that
Shannon was not the shooter or that they could not identify anyone. All
six were black. Shannon's family say they paid thousands of dollars to a
series of lawyers; none of them contacted any of the witnesses.

As Shannon worked on his own appeal with the jailhouse lawyers, he wrote
to lawyers all over the country, anyone he could think of who might take
a case like his. He heard back from Paul Casteleiro, legal director of
Centurion Ministries in New Jersey, which investigates wrongful
convictions. Casteleiro hired an investigator, who tracked down five of
the six African-American eyewitnesses. (The remaining witness had died.)
Casteleiro flew to New Orleans to talk to them, and they said that the
man they saw shoot Ralph Cole was about six feet tall, and as far as
they could tell, not Shannon.

In December 2016, the eyewitnesses testified at a hearing ordered by the
Louisiana Supreme Court. Asked why he hadn't contacted them before the
trial, Shannon's lawyer at the time, Paul Fleming, answered, ``I've been
racking my brain trying to remember.'' The judge threw out Shannon's
conviction in July 2017, and he came home to his sister and the rest of
his family. He had to pay about \$300 a month for an ankle monitor while
the New Orleans district attorney's office appealed and decided whether
to retry him; the D.A. finally dismissed the murder charge in February
2018.

Last fall, I started looking for the two jurors who tried to stop
Michael Shannon's wrongful conviction. When I found Mi Wha Morrison, she
told me she'd sometimes wondered about Shannon; she didn't know he'd
been exonerated. Shannon said he wanted to talk to her, and Morrison was
willing. We set up a meeting in November at the New Orleans courthouse
where Shannon's trial took place.

Shannon and Morrison sat down on a courtroom bench, and they went over
the day of the trial --- Bourgoyne's testimony, the prosecutor's claims,
the defense lawyer's errors --- like two survivors of a shipwreck,
piecing together what went wrong. Shannon is still slender, with
close-cropped hair and glasses. As he talked, his teeth flashed gold,
repaired after being knocked out by a guard after his arrest. Morrison
listened, her hands clasped on her lap, and asked Shannon about his
family. He told her about missing his daughter's graduation from high
school and about his mother's death. ``When my mama passed in 2014, that
was the hardest part of just being locked up, period,'' he said. He was
allowed to go to the funeral, but he had to wear handcuffs and chains.
``You in front of everybody shackled up on a murder that you don't
having nothing to do with.''

He told Morrison that his daughter had just become a registered nurse.
He also had a grandson who had gone to college and called Shannon Paw
Paw, though they met for the first time when his grandson was 21.
Shannon had a job as a prep cook, and he received \$25,000 a year from
the state in compensation. He'd married a former guard he met in prison.
``Found love in Angola,'' he marveled as they laughed together. ``Ain't
that something?''

The afternoon sun was dimming in the courtroom. Shannon, his head bent
close to Morrison's, said, ``I had to live through the jungle to come
through right here to have this conversation.'' Morrison told him that
she was sorry. But Shannon had already made it clear he didn't find her
at fault.

``You're the one that saved me,'' he said in the moment they greeted
each other.

``I tried to,'' she told him. ``I wanted to.''

Additional design and development by Jacqueline Myint

\textbf{Correction:}~Jan. 17, 2020

An earlier version of
this~\href{https://www.nytimes3xbfgragh.onion/interactive/2020/01/15/magazine/split-jurors.html}{article}~misstated
the number of people~the Innocence Project New Orleans says have been
exonerated in Louisiana since 2000. It is 43, not 56.

Read 53 Comments

\begin{itemize}
\item
\item
\item
\item
\end{itemize}

Advertisement

\protect\hyperlink{after-bottom}{Continue reading the main story}

\hypertarget{site-index}{%
\subsection{Site Index}\label{site-index}}

\hypertarget{site-information-navigation}{%
\subsection{Site Information
Navigation}\label{site-information-navigation}}

\begin{itemize}
\tightlist
\item
  \href{https://help.nytimes3xbfgragh.onion/hc/en-us/articles/115014792127-Copyright-notice}{©~2020~The
  New York Times Company}
\end{itemize}

\begin{itemize}
\tightlist
\item
  \href{https://www.nytco.com/}{NYTCo}
\item
  \href{https://help.nytimes3xbfgragh.onion/hc/en-us/articles/115015385887-Contact-Us}{Contact
  Us}
\item
  \href{https://www.nytco.com/careers/}{Work with us}
\item
  \href{https://nytmediakit.com/}{Advertise}
\item
  \href{http://www.tbrandstudio.com/}{T Brand Studio}
\item
  \href{https://www.nytimes3xbfgragh.onion/privacy/cookie-policy\#how-do-i-manage-trackers}{Your
  Ad Choices}
\item
  \href{https://www.nytimes3xbfgragh.onion/privacy}{Privacy}
\item
  \href{https://help.nytimes3xbfgragh.onion/hc/en-us/articles/115014893428-Terms-of-service}{Terms
  of Service}
\item
  \href{https://help.nytimes3xbfgragh.onion/hc/en-us/articles/115014893968-Terms-of-sale}{Terms
  of Sale}
\item
  \href{https://spiderbites.nytimes3xbfgragh.onion}{Site Map}
\item
  \href{https://help.nytimes3xbfgragh.onion/hc/en-us}{Help}
\item
  \href{https://www.nytimes3xbfgragh.onion/subscription?campaignId=37WXW}{Subscriptions}
\end{itemize}
