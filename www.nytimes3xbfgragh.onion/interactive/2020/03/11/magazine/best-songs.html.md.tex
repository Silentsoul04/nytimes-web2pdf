Sections

SEARCH

\protect\hyperlink{site-content}{Skip to
content}\protect\hyperlink{site-index}{Skip to site index}

\hypertarget{comments}{%
\subsection{\texorpdfstring{\protect\hyperlink{commentsContainer}{Comments}}{Comments}}\label{comments}}

\href{}{25 Songs That Matter Now}\href{}{Skip to Comments}

The comments section is closed. To submit a letter to the editor for
publication, write to
\href{mailto:letters@NYTimes.com}{\nolinkurl{letters@NYTimes.com}}.

The Music Issue

\hypertarget{25-songs-that-matter-now}{%
\section{25 Songs That Matter Now}\label{25-songs-that-matter-now}}

March 12, 2020

\begin{itemize}
\item
\item
\item
\item
\item
  \emph{+}
\end{itemize}

Sound On

Taylor Swift ``The Man''

Billie Eilish ``Everything I Know''

Lil Nas X ``Old Town Road''

Romeo Santos ``El Beso''

Megan Thee Stallion ``Hot Girl Summer''

\hypertarget{25}{%
\subsection{25}\label{25}}

Songs That Matter NowSongs from Spotify

SwipeScroll

Swipe\\
/Tap to go through the list

Or jump to an artist:

\begin{itemize}
\tightlist
\item
  1 King Princess
\item
  2 Tyler the Creator
\item
  3 Harry Styles
\item
  4 Brittany Howard
\item
  5 Romeo Santos feat. Kiko Rodriguez
\item
  6 Taylor Swift
\item
  7 Billie Eilish
\item
  8 Burna Boy
\item
  9 black midi
\item
  10 Lil Nas X
\item
  11 `Cats'
\item
  12 Denzel Curry
\item
  13 Normani
\item
  14 Lana Del Rey
\item
  15 Red Hearse
\item
  16 Lizzo
\item
  17 Richard Dawson
\item
  18 Kanye West
\item
  19 Vampire Weekend
\item
  20 100 gecs
\item
  21 Summer Walker
\item
  22 Megan Thee Stallion
\item
  23 The Highwomen
\item
  24 Kaytranada
\item
  25 Sleater-Kinney
\end{itemize}

\hypertarget{1}{%
\subparagraph{1}\label{1}}

\hypertarget{king-princess}{%
\subparagraph{King Princess}\label{king-princess}}

`Prophet'

\href{https://open.spotify.com/track/4eA4z7FCy5Fp6mh8acV3GY?si=3CrPDAWMROObx6wcpKjOmw}{
Full Track}

It's rare for a musician to shock anymore. Mikaela Straus says
{[}expletive{]} that. By Lizzy Goodman

\href{https://open.spotify.com/track/4eA4z7FCy5Fp6mh8acV3GY?si=3CrPDAWMROObx6wcpKjOmw}{Full
Track }

Next

It's rare for a musician to shock anymore. Mikaela Straus says
{[}expletive{]} that. By Lizzy Goodman Hanging out with King Princess
can feel like entering a time warp. At 21, she has the lush, broken
voice of a hard-living lounge singer in a David Lynch film, and her
music is similarly timeless: guitar-driven torch songs with lyrics
sharpened by what sounds like a thousand years of love gone wrong.
Listening to the yearning R\&B swoon of ``Prophet'' --- from King
Princess' debut album, ``Cheap Queen'' --- you may imagine its maker as
a sad-eyed recluse, one who wears her sorrow like armor. But in person
--- her birth name is Mikaela Straus --- the singer comes off as a
different kind of throwback: a bawdy, trash-talking caricature of
old-school rock 'n' roll excess.
\href{https://www.nytimes3xbfgragh.onion/interactive/2020/03/11/magazine/king-princess-profile.html}{Read
the profile}

Hanging out with King Princess can feel like entering a time warp. At
21, she has the lush, broken voice of a hard-living lounge singer in a
David Lynch film, and her music is similarly timeless: guitar-driven
torch songs with lyrics sharpened by what sounds like a thousand years
of love gone wrong. Listening to the yearning R\&B swoon of ``Prophet''
--- from King Princess' debut album, ``Cheap Queen'' --- you may imagine
its maker as a sad-eyed recluse, one who wears her sorrow like armor.
But in person --- her birth name is Mikaela Straus --- the singer comes
off as a different kind of throwback: a bawdy, trash-talking caricature
of old-school rock 'n' roll excess.

In the time we spent together, I saw her mime masturbation after talking
about how hot she thinks the singer Rosalía is (``She gave me a hug, and
I was like, `You smell good.''); declare that if she were a man, she
would ``have a small {[}expletive{]}, but it would work good''; and
announce that she wants to give her girlfriend a cast of her vagina for
their anniversary. In between takes at the video shoot for ``Ohio'' ---
a slow-burn ballad that descends into an unhinged rock jam --- she asked
several members of her team to smell her armpits (noting that the left
one was noticeably more rank than the right) and talked colorful smack
about other artists, esteemed music-industry institutions and an
ex-girlfriend's new girlfriend. ``I want you to come to my apartment
after this,'' she said, staring me down during a moment of tenuous calm
as her makeup artist sprayed a fine mist of glittering fuchsia across
her cheekbone, ``because I can tell it's inconvenient for you.''
\href{https://www.nytimes3xbfgragh.onion/interactive/2020/03/11/magazine/king-princess-profile.html}{Continue
Reading}

\textbf{Lizzy Goodman} is a journalist and the author of ``Meet Me in
the Bathroom,'' an oral history of music in New York City from 2001-11.
\href{https://www.nytimes3xbfgragh.onion/2019/06/10/magazine/womens-soccer-inequality-pay.html}{She
last wrote for the magazine about the U.S. women's national soccer team
and pay equality.} \textbf{Arielle Bobb-Willis} is a photographer from
New York who was recently featured in Aperture's ``The New Black
Vanguard.'' This is her first assignment for the magazine.

`Earfquake'

\includegraphics{https://static01.graylady3jvrrxbe.onion/packages/flash/multimedia/ICONS/transparent.png}

\includegraphics{https://static01.graylady3jvrrxbe.onion/images/2020/03/15/magazine/15mag-illustration-11/15mag-illustration-11-master675-v2.jpg}

\hypertarget{2}{%
\subparagraph{2}\label{2}}

\hypertarget{tyler-the-creator}{%
\subparagraph{Tyler the Creator}\label{tyler-the-creator}}

`Earfquake'

\href{https://open.spotify.com/track/5hVghJ4KaYES3BFUATCYn0}{ Full
Track}

There's ``growing up,'' and then there's what happened to Tyler Okonma.
By Carvell Wallace

\href{https://open.spotify.com/track/5hVghJ4KaYES3BFUATCYn0}{Full Track
}

Next

There's ``growing up,'' and then there's what happened to Tyler Okonma.
By Carvell Wallace How thrilling must it have been to be a young
teenager in 2008, witnessing the birth of Odd Future, a Los Angeles
collective of skater kids. They must have made it feel possible to do
everything and be every way you wanted to but weren't sure was allowed.
I was too old to be that sort of fan when Odd Future dropped its first
mix tape, but when I saw Tyler Okonma --- the rapper Tyler, the Creator
­--- one of the collective's breakout stars, reclining on the couch of
the reborn ``Arsenio Hall Show'' a few years later, feet up, explaining
a youthful worldview that was as harebrained as it was shrewd, as
unmannerly as it was undeniable, my first thought was that I wish he had
existed when I was teenager. I needed him. Read more

How thrilling must it have been to be a young teenager in 2008,
witnessing the birth of Odd Future, a Los Angeles collective of skater
kids. They must have made it feel possible to do everything and be every
way you wanted to but weren't sure was allowed. I was too old to be that
sort of fan when Odd Future dropped its first mix tape, but when I saw
Tyler Okonma --- the rapper Tyler, the Creator ­--- one of the
collective's breakout stars, reclining on the couch of the reborn
``Arsenio Hall Show'' a few years later, feet up, explaining a youthful
worldview that was as harebrained as it was shrewd, as unmannerly as it
was undeniable, my first thought was that I wish he had existed when I
was teenager. I needed him.

He was a force, eating cockroaches in music videos, fantasizing about
murder and suicide, delighting in hiding behind a veneer of edgelord
homophobic, misogynist lyrics. His was a youthful, playful, nearly
theoretical form of destruction. He reveled in blurring the line between
character and artist and troll, between ego and id. Yet he was so
sure-handed in his artistry, so unfettered, that he was precisely what
I, and probably you, could have used as a kid: someone who does not care
about anyone's rules but seems such a genius that no one opposing him
could ever be taken seriously. Your parents can ground you; they can't
do a thing to Tyler. Thus: He represented freedom. A grisly and dark
one, for sure, but a freedom nonetheless.

His growing up came in stages. On earlier albums like ``Bastard'' (2009)
and ``Goblin'' (2011), he is still thrilled by the unassailable powers
of potty mouth, enchanted by the possibility of scandalizing people in a
way that rap used to but hadn't in years. On ``Cherry Bomb'' (2015) and
``Flower Boy'' (2017), he's thrilled by the power of music itself, and
perhaps by the power of success, too. On ``Flower Boy,'' in particular,
he is writing about a male lover --- plot twist! --- and is now brimming
with questions about life and love, beauty and bees. The
annoying-little-brother skater Tyler is certainly still in the building,
but quite often his inclusion feels like an afterthought, a decision
made in post­production, a conciliation to the fans that stop him at Six
Flags and ask for pictures in hopes of validating their own flagging
youth.

Then comes
\href{https://www.nytimes3xbfgragh.onion/2019/06/07/arts/music/tyler-the-creator-igor-kevin-abstract-arizona-baby-review.html}{last
year's ``IGOR,''} and it's an entirely different game. For many longtime
Tyler fans, the album felt like an insult: Where are the bars, the
bangers, the middle fingers? But as anyone doing a close reading of
Tyler's discography knows, the real ``IGOR'' was with us all along. Here
the production wizardry and pure musicality he explored on ``Flower
Boy'' meets the earnestness of ``Cherry Bomb.'' And the fact that the
man who once rhymed ``Aretha'' with ``urethra'' had now made an entire
album about one single breakup? Well, that's the biggest middle finger
to anyone missing the old Tyler.

All this was brought into vivid relief on the album's first and only
single, ``Earf­quake,'' anchored by a pop-friendly three-chord
progression, vocals by the Gap Band legend Charlie Wilson and a verse
from the rapper Playboi Carti delivered so breezily as to almost be a
parody of a Playboi Carti verse. Tyler later admitted that he originally
wrote it for Justin Bieber or Rihanna, both of whom passed before he
took it himself. The Gap Band influence is strong: You can easily
superimpose Tyler's chorus over the verse chords from one of its biggest
hits, ``Outstanding'' (1982), and Wilson's influential voice technique
--- later imitated by the likes of Keith Sweat and Aaron Hall of Guy ---
is present here, in a telescopic callback to the late '80s New Jack
Swing he helped inspire.

For me though, what hits hardest about ``Earf­quake'' is that Tyler
presents a version of himself with no room to spare. The song, like much
of the album, finds him at a loss. He has ditched the safety of youthful
disaffection to seek love and some version of earnestness --- but,
surprise, it hurts. Such is the cost of genuinely trying to care.
``Don't leave,'' he begs, ``it's my fault.'' This is a Tyler without an
answer. The certainty is gone, replaced with pleading. He is
refreshingly, if painfully, not in control. We've already heard angry
Tyler, swaggering Tyler, depressive and violent Tyler,
double-middle-finger Tyler and I'm-too-smart-for-all-this Tyler.
``Earf­quake'' is the first time we hear a Tyler in need of someone
else. The ground beneath his feet has indeed been shaken.

This pop-fueled exploration into genuine vulnerability --- not in the
sense of baring all, but in the sense of being open to being harmed ---
has reached a broader audience than any of Tyler's other work ever has.
``IGOR'' was the first No.1 album of his decade-plus career; it netted
him a Grammy. Musically, production-wise, it's his most skilled
creation, and literarily it's his most completely thought-through. It
can't be everyone's favorite. But since his first solo effort in 2009,
Tyler has gone from hating himself to loving himself to loving someone
else --- with all the destruction and emotional detritus that brings. I
can't think of a better definition of growing up than that.

Carvell Wallace is a contributing writer for the magazine and a
podcaster based in Oakland, Calif.
\href{https://www.nytimes3xbfgragh.onion/2019/11/21/magazine/queen-slim-movie.html}{He
last wrote a feature article about the film ``Queen \& Slim.''}
Illustration by Denise Nestor. Source photograph: Stephane
Cardinale/Corbis, via Getty Images.

`Adore You'

\includegraphics{https://static01.graylady3jvrrxbe.onion/packages/flash/multimedia/ICONS/transparent.png}

\includegraphics{https://static01.graylady3jvrrxbe.onion/images/2020/03/15/magazine/15mag-illustration-02/15mag-illustration-02-master675-v2.jpg}

\hypertarget{3}{%
\subparagraph{3}\label{3}}

\hypertarget{harry-styles}{%
\subparagraph{Harry Styles}\label{harry-styles}}

`Adore You'

\href{https://open.spotify.com/track/1M4qEo4HE3PRaCOM7EXNJq?si=RynkgC_oSaCIc7mojYTzhA}{
Full Track}

The solo star evolves teen pop --- into the 1970s. By Lindsay Zoladz

\href{https://open.spotify.com/track/1M4qEo4HE3PRaCOM7EXNJq?si=RynkgC_oSaCIc7mojYTzhA}{Full
Track }

Next

The solo star evolves teen pop --- into the 1970s. By Lindsay Zoladz
It's almost unprecedented to see all the former members of a boy band
vie for a solo career at once, but One Direction was always different.
Niall Horan, Louis Tomlinson and Liam Payne have tuned their respective
antennas to what they believe makes a pop hit these days: aping Ed
Sheeran, a duet with the hitmaker Bebe Rexha or a bad-boy-by-numbers
approach. But their former bandmate Harry Styles, the one people over 30
would recognize on the street, has elected to make what you'd almost
have to call baby-boomer music. He cites Van Morrison, Steely Dan and
Joni Mitchell as his influences; he dresses like Mick Jagger circa
``Exile on Main Street.'' And he made last year's
\href{https://www.nytimes3xbfgragh.onion/2019/12/23/arts/music/harry-styles-fine-line-billboard.html}{``Fine
Line,''} a throwback to album-rock days that has somehow become a
streaming-era hit. Read more

It's almost unprecedented to see all the former members of a boy band
vie for a solo career at once, but One Direction was always different.
Niall Horan, Louis Tomlinson and Liam Payne have tuned their respective
antennas to what they believe makes a pop hit these days: aping Ed
Sheeran, a duet with the hitmaker Bebe Rexha or a bad-boy-by-numbers
approach. But their former bandmate Harry Styles, the one people over 30
would recognize on the street, has elected to make what you'd almost
have to call baby-boomer music. He cites Van Morrison, Steely Dan and
Joni Mitchell as his influences; he dresses like Mick Jagger circa
``Exile on Main Street.'' And he made last year's
\href{https://www.nytimes3xbfgragh.onion/2019/12/23/arts/music/harry-styles-fine-line-billboard.html}{``Fine
Line,''} a throwback to album-rock days that has somehow become a
streaming-era hit.

You can see why ``Adore You'' became the album's defining single: It's
the one in which past and present coalesce most seamlessly, its playful
psychedelia streamlined into sleek modern hit-making. It lets Styles
show off the full, androgynous range of his voice: The ethereal high
note he hits in the chorus is the equivalent of the sheer Gucci blouse
he wore to last year's Met Gala. Styles has the cross-platform
popularity to take some risks in his music --- and in a world
hyperfocused on the next big thing, he's managed to repurpose the past
to become a new kind of star.

Lindsay Zoladz is a critic, a reporter and an essayist in Brooklyn. She
was previously a staff writer for The Ringer and before that the pop
critic for New York magazine. It is her first time writing for the
magazine. Illustration by Denise Nestor. Source photograph: Kevin
Mazur/Getty Images.

`Goat Head'

\includegraphics{https://static01.graylady3jvrrxbe.onion/packages/flash/multimedia/ICONS/transparent.png}

\includegraphics{https://static01.graylady3jvrrxbe.onion/images/2020/03/15/magazine/15mag-brittany-02/15mag-brittany-02-master675-v2.jpg}

\hypertarget{4}{%
\subparagraph{4}\label{4}}

\hypertarget{brittany-howard}{%
\subparagraph{Brittany Howard}\label{brittany-howard}}

`Goat Head'

\href{https://open.spotify.com/track/4rGUfotRdDyEmen7Hs83Jg?si=eSfqicVxSNOPZcBGmQiW4Q}{
Full Track}

A ghastly moment in the family history. By Zandria F. Robinson

\href{https://open.spotify.com/track/4rGUfotRdDyEmen7Hs83Jg?si=eSfqicVxSNOPZcBGmQiW4Q}{Full
Track }

Next

A ghastly moment in the family history. By Zandria F. Robinson Brittany
Howard, progeny of Sister Rosetta Tharpe and Prince and owner of all
genres, took herself to the proverbial woodshed (in this case a
greenhouse in Topanga, Calif.) and made a black-queer-Southern-feminist
debut solo album that rings up Meshell Ndegeocello's ``Bitter'' from the
future. Named after her big sister, who, before she died of a rare eye
cancer at 13, gave Howard the best tools a big sister might --- poetry
and sound --- ``Jaime'' feels like that elusive and mystical other side
of grief, where tears, no longer happy or sad but something much greater
and far more satisfying, surprise and amuse.
\href{https://www.nytimes3xbfgragh.onion/interactive/2020/03/11/magazine/brittany-howard.html}{Read
more}

Brittany Howard, progeny of Sister Rosetta Tharpe and Prince and owner
of all genres, took herself to the proverbial woodshed (in this case a
greenhouse in Topanga, Calif.) and made a black-queer-Southern-feminist
debut solo album that rings up Meshell Ndegeocello's ``Bitter'' from the
future. Named after her big sister, who, before she died of a rare eye
cancer at 13, gave Howard the best tools a big sister might --- poetry
and sound --- ``Jaime'' feels like that elusive and mystical other side
of grief, where tears, no longer happy or sad but something much greater
and far more satisfying, surprise and amuse.

Heretofore known as the luminous frontwoman of
\href{https://www.nytimes3xbfgragh.onion/2015/03/22/magazine/alabama-shakess-soul-stirring-shape-shifting-new-sound.html}{the
blues-rock band Alabama Shakes}, Howard has been lionized by Southerners
and non-Southerners alike as the region's radically reconstructed blues
woman, its weird, funky rock savior and an emblem of a South that might
finally see promiscuity and difference --- racial, sexual, sonic --- as
a liberal good. Howard herself has palpably resisted this role and its
box, in part by genre-hopping from garage rock to Americana in the
groups Thunderbitch and Bermuda Triangle. On
\href{https://www.nytimes3xbfgragh.onion/2019/09/05/arts/music/brittany-howard-jaime.html}{``Jaime,''}
this resistance appears effortless, intimate, self-evident and interior.

Here, Howard models luscious generosity toward herself as a woman,
lover, Southerner, daughter and sister, modes of being that spread to
every crevice of the album. She deftly guides her voice --- a divine,
soaring, hand-to-hand, declarative, love-soaked instrument only made
more expansive in service of herself and her sound alone --- through the
varying registers of the blues.
\href{https://www.nytimes3xbfgragh.onion/interactive/2020/03/11/magazine/brittany-howard.html}{Continue
Reading}

\textbf{Zandria F. Robinson} is a writer, professor and cultural critic
based in Washington. This is her first article for the magazine.
\textbf{Arielle Bobb-Willis} is a photographer from New York who was
recently featured in Aperture's ``The New Black Vanguard.'' This is her
first assignment for the magazine.

`El Beso Que No Le Di'

\includegraphics{https://static01.graylady3jvrrxbe.onion/packages/flash/multimedia/ICONS/transparent.png}

\includegraphics{https://static01.graylady3jvrrxbe.onion/images/2020/03/15/magazine/15mag-illustration-13/15mag-illustration-13-master675-v2.jpg}

\hypertarget{5}{%
\subparagraph{5}\label{5}}

\hypertarget{romeo-santos-feat-kiko-rodriguez}{%
\subparagraph{Romeo Santos feat. Kiko
Rodriguez}\label{romeo-santos-feat-kiko-rodriguez}}

`El Beso Que No Le Di'

\href{https://open.spotify.com/track/2kcc5humLS6PKOI9rwvXex?si=5BkZZdxsT_WRJZdWOzyqeg}{
Full Track}

A bachata singer is keeping old-fashioned romance on the charts. By Jody
Rosen

\href{https://open.spotify.com/track/2kcc5humLS6PKOI9rwvXex?si=5BkZZdxsT_WRJZdWOzyqeg}{Full
Track }

Next

A bachata singer is keeping old-fashioned romance on the charts. By Jody
Rosen Is romance dead? If you listen to the radio or scroll through a
streaming playlist, you may find yourself wondering where the love has
gone. Today's pop often focuses dispassionately, at times clinically, on
carnal matters --- a vision of romance as a not-very-sexy zero-sum
contest for sex that, perhaps, captures the embattled mood of 2020. But
there are exceptions. Romeo Santos, the bachata superstar from the
Bronx, is a balladeer of the old school: Courtly and theatrical, he
lavishes the women in his songs with entreaties and worshipful words,
venerating love itself as a kind of holy force --- a deity that beams
golden light down through the skylight and into the boudoir. Read more

Is romance dead? If you listen to the radio or scroll through a
streaming playlist, you may find yourself wondering where the love has
gone. Today's pop often focuses dispassionately, at times clinically, on
carnal matters --- a vision of romance as a not-very-sexy zero-sum
contest for sex that, perhaps, captures the embattled mood of 2020. But
there are exceptions. Romeo Santos, the bachata superstar from the
Bronx, is a balladeer of the old school: Courtly and theatrical, he
lavishes the women in his songs with entreaties and worshipful words,
venerating love itself as a kind of holy force --- a deity that beams
golden light down through the skylight and into the boudoir.

Santos's old-fashioned passions are displayed to greatest effect in the
songs in which he pines for lost loves. ``El Beso Que No Le Di,'' a duet
with Kiko Rodriguez, is one of those: a melodrama about a jilted lover
who attends the wedding of his ex. Over a typically sensuous bachata
arrangement --- floating melody, jittery guitar line, chattering bongos
and guiro --- the singers take turns, but it is Santos who commands
center stage: His voice, a falsetto pitched somewhere between Aaron
Neville and a chirruping bluebird, is one of pop's great sounds. The
song is a lament about ``the kiss that I didn't give,'' but Santos, a
child of hip-hop, has a way of slipping in boasts alongside his plaints.
``She won't forget Romeo,'' he coos. ``Unh-unh.''

Jody Rosen is a contributing writer for the magazine and the author of
``Two Wheels Good: The Bicycle on Planet Earth and Elsewhere,'' to be
published next year.
\href{https://www.nytimes3xbfgragh.onion/2019/12/12/magazine/colin-kaepernick-nfl-video.html}{He
last wrote about Colin Kaepernick and image management for a Screenland
column.} Illustration by Denise Nestor. Source photograph: Johnny
Nunez/Getty Images.

`The Man'

\includegraphics{https://static01.graylady3jvrrxbe.onion/packages/flash/multimedia/ICONS/transparent.png}

\includegraphics{https://static01.graylady3jvrrxbe.onion/images/2020/03/15/magazine/15mag-illustration-06/15mag-illustration-06-master675-v2.jpg}

\hypertarget{6}{%
\subparagraph{6}\label{6}}

\hypertarget{taylor-swift}{%
\subparagraph{Taylor Swift}\label{taylor-swift}}

`The Man'

\href{https://open.spotify.com/track/3RauEVgRgj1IuWdJ9fDs70}{ Full
Track}

Pop's best humblebraggart lands a sly protest song. By Jody Rosen

\href{https://open.spotify.com/track/3RauEVgRgj1IuWdJ9fDs70}{Full Track
}

Next

Pop's best humblebraggart lands a sly protest song. By Jody Rosen For
more than a decade, Taylor Swift has been pop's top bellyacher, turning
a now-familiar set of grievances into
\href{https://www.nytimes3xbfgragh.onion/2019/12/24/arts/music/taylor-swift-lover.html}{great
songs}. In ``The Man,'' she channels that indignation into a broader
protest against the sexism and skepticism that all women face.

For more than a decade, Taylor Swift has been pop's top bellyacher,
turning a now-familiar set of grievances into
\href{https://www.nytimes3xbfgragh.onion/2019/12/24/arts/music/taylor-swift-lover.html}{great
songs}. In ``The Man,'' she channels that indignation into a broader
protest against the sexism and skepticism that all women face.

{[}Verse{]}

I would be complex -\/-br-\/- I would be cool

The lyrics may prompt a chuckle --- Taylor Swift, cool? --- but the
aggravation behind them is palpable: ``Complex'' is a descriptor
journalists toss hither-and-yon at male musicians but apply sparingly to
the likes of Swift.

{[}Verse{]}

When everyone believes ya -\/-br-\/- What′s that like

The song's most hard-hitting line is a plaintive rhetorical question
that calls to mind a \#MeToo movement slogan: \#BelieveWomen.

{[}Chorus{]}

′Cause if I was a man -\/-br-\/- Then I'd be the man

As a lyricist, Swift's great strength is storytelling. But ``The Man''
turns on a punchline that calls to mind the pun-happy pros of
Nashville's Music Row, where Swift cut her teeth as a songwriter.

{[}Verse{]}

Could all be separated from my good ideas and power moves

In recent years, Swift's ``power moves'' have extended to the eternal
music-industry struggle between labor and management: She has squared
off with Spotify about royalties and clashed with record executives over
control of her masters.

{[}Verse{]}

So it's okay that I'm mad

Maybe ``The Man'' is best seen as an update of Swift's earlier smash
``Shake It Off.'' Where that hit deflected sexist slights with cheeky
humor and passive-aggressive musical pep, Swift has --- like so many
women in this era --- embraced something more robust: the power of anger
without apology.

Jody Rosen is a contributing writer for the magazine and the author of
``Two Wheels Good: The Bicycle on Planet Earth and Elsewhere,'' to be
published next year.
\href{https://www.nytimes3xbfgragh.onion/2019/12/12/magazine/colin-kaepernick-nfl-video.html}{He
last wrote about Colin Kaepernick and image management for a Screenland
column.} Illustration by Denise Nestor. Source photograph: Mat
Hayward/GC Images, via Getty Images.

`Everything I Wanted'

\includegraphics{https://static01.graylady3jvrrxbe.onion/packages/flash/multimedia/ICONS/transparent.png}

\includegraphics{https://static01.graylady3jvrrxbe.onion/images/2020/03/15/magazine/15mag-billie-08/15mag-billie-08-master675-v3.jpg}

\includegraphics{https://static01.graylady3jvrrxbe.onion/packages/flash/multimedia/ICONS/transparent.png}

\includegraphics{https://static01.graylady3jvrrxbe.onion/images/2020/03/15/magazine/15mag-billie-03/15mag-billie-03-master675-v3.jpg}

\hypertarget{7}{%
\subparagraph{7}\label{7}}

\hypertarget{billie-eilish}{%
\subparagraph{Billie Eilish}\label{billie-eilish}}

`Everything I Wanted'

\href{https://open.spotify.com/track/3ZCTVFBt2Brf31RLEnCkWJ}{ Full
Track}

Her transgressive, melancholy pop led to mainstream stardom. By Jonah
Weiner

\href{https://open.spotify.com/track/3ZCTVFBt2Brf31RLEnCkWJ}{Full Track
}

Next

Her transgressive, melancholy pop led to mainstream stardom. By Jonah
Weiner Sometime during the night of Sept. 4, 2018, Billie Eilish took
her own life --- in a dream. ``I jumped off a building,'' she recalled
recently. What was most alarming about it, as she looked back, was how
little it alarmed her. ``I was in a really bad place mentally,'' Eilish
said; the dream struck her less as a nightmare than as a grimly alluring
fantasy. The next day, she approached her older brother, Finneas
O'Connell, a songwriter and producer, and told him about it. They have
collaborated on every piece of music she has put out, and she presented
the dream to him as possible inspiration for a new song.
\href{https://www.nytimes3xbfgragh.onion/interactive/2020/03/11/magazine/billie-eilish-profile.html}{Read
the profile}

Sometime during the night of Sept. 4, 2018, Billie Eilish took her own
life --- in a dream. ``I jumped off a building,'' she recalled recently.
What was most alarming about it, as she looked back, was how little it
alarmed her. ``I was in a really bad place mentally,'' Eilish said; the
dream struck her less as a nightmare than as a grimly alluring fantasy.
The next day, she approached her older brother, Finneas O'Connell, a
songwriter and producer, and told him about it. They have collaborated
on every piece of music she has put out, and she presented the dream to
him as possible inspiration for a new song.
\href{https://www.nytimes3xbfgragh.onion/interactive/2020/03/11/magazine/billie-eilish-profile.html}{Continue
Reading}

\textbf{Jonah Weiner} is a contributing writer for the magazine.
\href{https://www.nytimes3xbfgragh.onion/2020/01/08/magazine/is-the-viral-non-ad-ad-the-future-of-advertising.html}{He
last wrote about viral narrative ads.} \textbf{Arielle Bobb-Willis} is a
photographer from New York who was recently featured in Aperture's ``The
New Black Vanguard.'' This is her first assignment for the magazine.

`Dangote'

\includegraphics{https://static01.graylady3jvrrxbe.onion/packages/flash/multimedia/ICONS/transparent.png}

\includegraphics{https://static01.graylady3jvrrxbe.onion/images/2020/03/15/magazine/15mag-illustration-10/15mag-illustration-10-master675-v2.jpg}

\hypertarget{8}{%
\subparagraph{8}\label{8}}

\hypertarget{burna-boy}{%
\subparagraph{Burna Boy}\label{burna-boy}}

`Dangote'

\href{https://open.spotify.com/track/07XZZTucIfMyOSnkF0GPWJ?si=8pXt430uSiuCk1Drjl8eGQ}{
Full Track}

You can imitate the 1\% or resent them --- or, in this case, both. By
Jody Rosen

\href{https://open.spotify.com/track/07XZZTucIfMyOSnkF0GPWJ?si=8pXt430uSiuCk1Drjl8eGQ}{Full
Track }

Next

You can imitate the 1\% or resent them --- or, in this case, both. By
Jody Rosen Damini Ebunoluwa Ogulu, the singer-songwriter known as Burna
Boy, is arguably the biggest star in Afrobeats, a catchall term for the
plush, funky genre centered in Nigeria and Ghana. Afrobeats draws on
American hip-hop, Jamaican dance hall and London club sounds, but in
recent years the flow of influence has reversed, with superstars like
Beyoncé and Drake borrowing from, and collaborating with, Afrobeats
singers and producers. Burna Boy's 2019 album, ``African Giant,''
epitomizes the appeal: smooth vocals and slinky rhythms powering songs
whose lyrics range from come-ons and braggadocio to forays into politics
and protest. Read more

Damini Ebunoluwa Ogulu, the singer-songwriter known as Burna Boy, is
arguably the biggest star in Afrobeats, a catchall term for the plush,
funky genre centered in Nigeria and Ghana. Afrobeats draws on American
hip-hop, Jamaican dance hall and London club sounds, but in recent years
the flow of influence has reversed, with superstars like Beyoncé and
Drake borrowing from, and collaborating with, Afrobeats singers and
producers. Burna Boy's 2019 album, ``African Giant,'' epitomizes the
appeal: smooth vocals and slinky rhythms powering songs whose lyrics
range from come-ons and braggadocio to forays into politics and protest.

This big hit, named for the billionaire industrialist Aliko Dangote, may
at first appear to be a straightforward boast: Burna Boy touting his
industriousness and indefatigable drive to ``find money.'' But the
return, again and again, to the figure of Dangote --- and the catalog of
other Nigerian tycoons and politicians invoked in the song's coda ---
suggests a more ambivalent message. Burna Boy is describing a system
that elevates a lucky few to positions of unimaginable wealth and leaves
millions scraping and scrambling. The video, with its images of
working-class Lagos, sides with the masses. But ultimately the song is
reportorial: a dispatch from a pitiless world in which that refrain ---
``Dangote, Dangote'' --- functions as both a striver's mantra and a yelp
of despair.

Jody Rosen is a contributing writer for the magazine and the author of
``Two Wheels Good: The Bicycle on Planet Earth and Elsewhere,'' to be
published next year.
\href{https://www.nytimes3xbfgragh.onion/2019/12/12/magazine/colin-kaepernick-nfl-video.html}{He
last wrote about Colin Kaepernick and image management for a Screenland
column.} Illustration by Denise Nestor. Source photograph: Maury
Phillips/Getty Images.

`953'

\includegraphics{https://static01.graylady3jvrrxbe.onion/packages/flash/multimedia/ICONS/transparent.png}

\includegraphics{https://static01.graylady3jvrrxbe.onion/images/2020/03/15/magazine/15mag-illustration-04/15mag-illustration-04-master675-v2.jpg}

\hypertarget{9}{%
\subparagraph{9}\label{9}}

\hypertarget{black-midi}{%
\subparagraph{black midi}\label{black-midi}}

`953'

\href{https://open.spotify.com/track/1KFB35I79oxWR54uAewV2x?si=WnKVU5gJQqOPQpgJQL7xhA}{
Full Track}

A young band leaps off the musical grid --- and into something
refreshingly human. By Jeremy D. Larson

\href{https://open.spotify.com/track/1KFB35I79oxWR54uAewV2x?si=WnKVU5gJQqOPQpgJQL7xhA}{Full
Track }

Next

A young band leaps off the musical grid --- and into something
refreshingly human. By Jeremy D. Larson One-two-three \emph{four}
\ldots{} one-two-three \emph{four} \ldots{} --- at least that's how I
count out the opening riff of ``953,'' a brazen statement of purpose
from the guitarist and singer Geordie Greep, pictured, off black midi's
2019 debut album, ``Schlagenheim.'' Or I tap it out on the desk: thumb,
index, pinkie, ring \ldots{} thumb, index, pinkie, ring. Four big notes
and two little ghost notes, making up a riff that clambers over the
beginning and end of the song. A minute in, they do something similar,
except now they make that last beat stretch and hang in the air as if
time has stopped, and a whole song's worth of drumming seems to crowd
in. Read more

One-two-three \emph{four} \ldots{} one-two-three \emph{four} \ldots{}
--- at least that's how I count out the opening riff of ``953,'' a
brazen statement of purpose from the guitarist and singer Geordie Greep,
pictured, off black midi's 2019 debut album, ``Schlagenheim.'' Or I tap
it out on the desk: thumb, index, pinkie, ring \ldots{} thumb, index,
pinkie, ring. Four big notes and two little ghost notes, making up a
riff that clambers over the beginning and end of the song. A minute in,
they do something similar, except now they make that last beat stretch
and hang in the air as if time has stopped, and a whole song's worth of
drumming seems to crowd in.

It sounds like proof of life in our digitized and automated world,
orchestrated by four curious whelps from Britain trying to break music
off its perfect mechanical grid and into unpredictable, imperfect
spaces.

The band's name comes from a niche genre of music that exists almost
exclusively on the internet. A ``black MIDI'' song --- the genre, not
the band --- is one with so many notes, crammed so tightly into
composing software, that the staff becomes one thick black line,
darkened with
\href{https://www.youtube.com/watch?v=mUy8vnZLwhk}{millions or billions
of notes}. The musicians and programmers who create this are called
blackers, and their work sounds like a legion of grand pianos at war
inside an unusually crowded Dave \& Busters. This is part avant-garde
process music, part one-upmanship, part stress test on what a song can
contain and what will happen when a computer can't handle it anymore.
The song ``Armageddon v3,'' by Gingeas, consists of a single note
repeated 93 trillion times and is definitely unplayable on your laptop,
unless you have 256 terabytes of hard-drive space lying around.

The band black midi doesn't share many aesthetics with this stuff, but
there's something apt to the name, a puckish appetite for disruption.
The four musicians came together as teenagers at the BRIT School for
Performing Arts and Technology in London, making wiry, odd songs born
out of a boundless Gen Z curiosity; they're filled with long
improvisations and impressionist lyrics that suggest there's more life
in the underbelly of a moneyed city than there could ever be on its
gilded surface. ``953'' is particularly experimental: Its time signature
shifts constantly, from a light-speed onslaught to a relaxed, cavernous
crawl and then back. By the end, the band repeats its opening riff while
steadily slowing the tempo with impossible precision, like a clock
winding down.

Between Morgan Simpson's
\href{https://www.youtube.com/watch?v=TMn1UuEIVvA}{cephalopodic
drumming} and Greep's ancient croon, the band does have a mathy,
prog-rock air --- but in place of the pretensions normally associated
with hyper­technical music, black midi has a kind of pathos. Here they
are in this computational age, building whole songs around human
uncertainty, songs that keep fidgeting and breathing, even when they're
only repeating a few handfuls of notes. The music's code is so restless,
so elastic, that it resists classification; it hides from the algorithm.
Music like this can be a respite from our increasingly mechanized
soundscape, smoothed out by Zamboni­like digital tools that ensure
tempos never vary and nothing is out of place.

The way music reaches its listeners is becoming increasingly opaque,
too, policed by a computational process that has redefined the spaces in
which we listen to and share songs. There are now, literally, hidden
proprietary digital codes that shape the way music is distributed and
ranked, and we are being conditioned to view their effects as an
accurate measurement of the world --- to assume that the most-streamed
song must be the most popular one, even when we are clueless about the
source of the underlying data.

But that grand rubato at the end of ``953,'' the riff becoming slower
and slower by pure feel --- that's something that feels untouched by a
machine. These four guys pushed this dented jalopy of a song to its
breaking point, and now it's sputtering out on those four notes,
one-two-three \emph{four} \ldots{} one-two-three \emph{four}. \ldots{}
This small and simple line of code programs into the song a high
possibility of being omitted from big popular playlists because it is
neither clean nor harmonious. It suggests something both obvious and
inevitable: In a genre­less future, alternative music won't be a
response to what is popular --- but an attack on the computational
nature of music itself.

Jeremy D. Larson is the reviews editor for Pitchfork.
\href{https://www.nytimes3xbfgragh.onion/2019/01/08/magazine/the-scourge-of-relatable-in-art-and-politics.html}{He
previously wrote for the magazine about the word ``relatable.''}
Illustration by Denise Nestor. Source photograph: Yis Kid.

`Old Town Road'

\hypertarget{10}{%
\subparagraph{10}\label{10}}

\hypertarget{lil-nas-x}{%
\subparagraph{Lil Nas X}\label{lil-nas-x}}

`Old Town Road'

\href{https://open.spotify.com/track/0F7FA14euOIX8KcbEturGH?si=YS-PEjT0R9KWB9wJWxpYoA}{
Full Track}

Last year's big song just keeps on giving. By Wesley Morris

\href{https://open.spotify.com/track/0F7FA14euOIX8KcbEturGH?si=YS-PEjT0R9KWB9wJWxpYoA}{Full
Track }

Next

Last year's big song just keeps on giving. By Wesley Morris Is there any
getting over this song? Isn't there some new town road we should be
strolling down? The charts say we've moved on to ``The Box'' and
``Roxanne'' and ``Hot Girl Bummer.'' I heard a different story at a
Christone (Kingfish) Ingram concert last month. Ingram is a strong
rockin' blues musician --- a traditionalist, basically, who can set
fires with his guitar --- and I expected him to stick to the traditional
script. But Ingram is also 21, just a few months older than Lil Nas X.
What does he care about scripts?
\href{https://www.nytimes3xbfgragh.onion/interactive/2020/03/11/magazine/lil-nas-x-old-town-road.html}{Read
more}

Is there any getting over this song? Isn't there some new town road we
should be strolling down? The charts say we've moved on to ``The Box''
and ``Roxanne'' and ``Hot Girl Bummer.'' I heard a different story at a
Christone (Kingfish) Ingram concert last month. Ingram is a strong
rockin' blues musician --- a traditionalist, basically, who can set
fires with his guitar --- and I expected him to stick to the traditional
script. But Ingram is also 21, just a few months older than Lil Nas X.
What does he care about scripts?

He just started connecting a little bit of then to little bits of now.
The connection made that night in his ferocious, flabbergasting
instrumental dismount was between Prince and ``Old Town Road.'' He
didn't give it the novelty treatment, two bars then onto the next; he
stayed there and made love to that song as if it had been around for
half a century, as if he were conceived to it. Suddenly, it was ``Hotel
California,'' ``Stairway to Heaven.'' Meaty, serious, canonical.

\href{https://www.nytimes3xbfgragh.onion/2019/05/10/arts/music/old-town-road-lil-nas-x.html}{``Old
Town Road''} is the longest-running chart-topper ever. It spent 19 weeks
up there. A whole summer!
\href{https://www.nytimes3xbfgragh.onion/interactive/2020/03/11/magazine/lil-nas-x-old-town-road.html}{Continue
Reading}

\textbf{Wesley Morris} is a staff writer for the magazine, a critic at
large for The New York Times and co-host of the podcast
\href{https://www.nytimes3xbfgragh.onion/column/still-processing-podcast}{``Still
Processing.''} In 2019, he won the Roger Ebert Award from the
African-American Film Critics Association. \textbf{Arielle Bobb-Willis}
is a photographer from New York who was recently featured in Aperture's
``The New Black Vanguard.'' This is her first assignment for the
magazine.

`Jellicle Songs for Jellicle Cats'

\includegraphics{https://static01.graylady3jvrrxbe.onion/packages/flash/multimedia/ICONS/transparent.png}

\includegraphics{https://static01.graylady3jvrrxbe.onion/images/2020/03/15/magazine/15mag-music-cats/15mag-music-cats-master675.jpg}

\hypertarget{11}{%
\subparagraph{11}\label{11}}

\hypertarget{cats}{%
\subparagraph{`Cats'}\label{cats}}

`Jellicle Songs for Jellicle Cats'

\href{https://open.spotify.com/track/1jCDF7sWqYmOr2Ll5WkNuT?si=wvWpEVF3QSm-cSiNh90NSg}{
Full Track}

How Hollywood spayed and neutered everything extraordinary about
``Cats.'' By Jamie Lauren Keiles

\href{https://open.spotify.com/track/1jCDF7sWqYmOr2Ll5WkNuT?si=wvWpEVF3QSm-cSiNh90NSg}{Full
Track }

Next

How Hollywood spayed and neutered everything extraordinary about
``Cats.'' By Jamie Lauren Keiles ``Cats'' is a vast and expanding
ecosystem, comprising a stage show, a cast album, a fandom and now a
quasi flop of a blockbuster film. This latest reworking, which came out
last December, repeats the original plot of the show: A group of urbane
and world-weary cats compete in an annual talent show for a chance to
ascend to the Heaviside Layer, a vaguely defined yet enticing
afterworld. This premise does little to justify itself, which is but one
of the many charms of ``Cats.'' Read more

``Cats'' is a vast and expanding ecosystem, comprising a stage show, a
cast album, a fandom and now a quasi flop of a blockbuster film. This
latest reworking, which came out last December, repeats the original
plot of the show: A group of urbane and world-weary cats compete in an
annual talent show for a chance to ascend to the Heaviside Layer, a
vaguely defined yet enticing afterworld. This premise does little to
justify itself, which is but one of the many charms of ``Cats.''

Even in a time of so much cultural recursion --- of news about memes
based on tweets about news --- the show amounts to something uniquely
dense and strange. Last year's film neglected this fact, to its great
detriment. ``Jellicle Songs for Jellicle Cats,'' the overture from the
movie and the show, tells the whole of this tale in microcosm.

It begins as many music stories do, with a white person imitating a
white person imitating a black person. In the 1930s, the poets Ezra
Pound and T.S. Eliot wrote letters, channeling the voice of Joel
Chandler Harris, who was channeling the fake black dialect of Uncle
Remus. In those days, such banter was called ``modernism.'' Eliot
repackaged the bit as a collection of light verse, which was published
in 1939 as ``Old Possum's Book of Practical Cats.''

Sir Andrew Lloyd Webber, the English composer, first came to know this
book as a child. He returned to the text in 1977, translating the
lighthearted verses into song as a diversion during rehearsals for
``Evita.'' What emerged from that whim was a rarefied ``song cycle,''
scored word for word to Eliot's poems. This work birthed the stage
musical, which ran on Broadway for 18 years straight --- finding a
mainstream audience despite, or perhaps because of, its unlikely
premise.

``Jellicle Songs for Jellicle Cats'' is the first and strangest song in
this very strange show. One of just a few not scored to Eliot's verse,
it turns on a tautological refrain: ``Jellicles can and jellicles do!
Jellicles do and jellicles can!'' These Jellicle cats are the show's
only stars, but we come to know them only through oblique interrogation:
``Are you blind when you're born? Can you see in the dark?'' Jellicles
are, and jellicles can. They run up walls and walk on wires. They were
present for the construction of the Sphinx, but beyond this,
``jellicle'' is not clearly defined. Part of the pleasure of falling
into ``Cats'' is letting Webber guide you imperceptibly toward
knowledge. Like many great avant-garde works, ``Jellicle'' teaches you
to understand itself.

In the \href{https://www.youtube.com/watch?v=GbpP3Sxp-1U}{original cast
recording}, the music does as much educating as the lyrics. The song
begins in a dark alleyway, where textured horns seem to steal between
shadows. The blink of a xylophone's dissonant chords suggests a pair of
yellow cat eyes in the night --- and further suggests the show's classic
poster. This combination of music and sensual, not-quite-camp feline
dance made ``Cats'' the longest-running show of its time, shattering
records on both Broadway and the West End. The dancing cats formed their
suicide cult more than 16,000 times throughout the era, as busloads of
tourists arrived to watch them die. The show's T-shirt was a best seller
of the '80s, second only to merch from the Hard Rock Cafe. That a
cat-themed stage extravaganza --- written by a dense modernist and
channeled through song and dance without any narration --- could find
the same appeal as rock 'n' roll-themed chain restaurants is among the
most unlikely art stories ever told.

With this in mind, it makes perfect sense that infinitely regurgitative
Hollywood would think to adapt
\href{https://www.nytimes3xbfgragh.onion/2019/12/18/movies/cats-review.html}{``Cats''
into a movie}. Even as reboots go, the intellectual-property behemoth
seems uniquely suited to our time: absurdist, nostalgic, a pastiche of
different genres starring the internet's lead protagonist. In a time of
much debate on what a blockbuster should be, the success of ``Cats''
onstage provides compelling evidence that the middlebrow public will
show up for things that challenge the least common denominator. The
movie could have done as much --- but it failed.

The latest version of ``Jellicle'' seems afraid of everything that made
the old version work. In place of spacious trumpets, the alleyway floods
with synthesized, digitally echoed pianos. Instead of the allusion to
the blinking yellow eyes, the song opts for what sound like actual
recordings of cats' paws. The cats in the original Broadway cast album
channeled the voices of stereotypical 19th-century eccentrics: the
faded, screeching hag; the foppish, mincing lush; the booming, bombastic
man about town. In the movie, the singers sound as smooth and
radio-ready as Jason Derulo. (Indeed, one character is Jason Derulo.)
This flat trepidation carries over to the visuals, in which felinity was
so overexpressed that the cast looked disturbing, somehow both fully
human and fully cat. The film, in other words, does not believe that its
mass audience can synthesize strangeness. It struggles to explain the
weird world it conjures, thus not ascending to the Heaviside Layer but
landing with a thud in the litter box of culture, kicking up a cloud
that quickly settles into nothing.

Jamie Lauren Keiles is a contributing writer for the magazine.
\href{https://www.nytimes3xbfgragh.onion/interactive/2019/12/23/magazine/etika-death.html}{Their
last article was about the death of the YouTube star Etika.}
Illustration by Denise Nestor

`Ricky'

\includegraphics{https://static01.graylady3jvrrxbe.onion/packages/flash/multimedia/ICONS/transparent.png}

\includegraphics{https://static01.graylady3jvrrxbe.onion/images/2020/03/15/magazine/15mag-illustration/15mag-illustration-master675-v2.jpg}

\hypertarget{12}{%
\subparagraph{12}\label{12}}

\hypertarget{denzel-curry}{%
\subparagraph{Denzel Curry}\label{denzel-curry}}

`Ricky'

\href{https://open.spotify.com/track/2QbGvQssb0VLLS4x5NOmyJ?si=NQhTBAl7TUOVgCelHRw72Q}{
Full Track}

Unsentimental rap about a father's unsentimental lessons. By Jody Rosen

\href{https://open.spotify.com/track/2QbGvQssb0VLLS4x5NOmyJ?si=NQhTBAl7TUOVgCelHRw72Q}{Full
Track }

Next

Unsentimental rap about a father's unsentimental lessons. By Jody Rosen
Denzel Curry, who released his first mixtape at 16, has always projected
wisdom beyond his years. Now 25, the rapper from Carol City, Fla., is a
veteran, and on ``Ricky'' he sounds the part: exulting over his success,
repping his home turf and delivering life lessons in a voice of
authority. Read more

Denzel Curry, who released his first mixtape at 16, has always projected
wisdom beyond his years. Now 25, the rapper from Carol City, Fla., is a
veteran, and on ``Ricky'' he sounds the part: exulting over his success,
repping his home turf and delivering life lessons in a voice of
authority.

The titular Ricky is Curry's father. Many songwriters pay tribute to
family and formative years, but few gaze backward, as Curry does, with a
fondness devoid of sentimentality:

\begin{quote}
My daddy said: ``Trust no man but your brothers''\\
``And never leave your day ones in the gutter''\\
My daddy said: ``Treat young girls like your mother''\\
My momma said: ``Trust no ho, use a rubber.''
\end{quote}

Curry's life has been marked by death --- he was a classmate of Trayvon
Martin's, his brother died after being tasered by the police and his old
roommate, the rapper XXXTentacion, was killed in 2018.

Curry is one of the inventors of woozy-sounding SoundCloud rap, but
``Ricky'' is a classicist move: a hard, unfussy beat; tidy rhymes;
brisk, vivid storytelling reminiscent of old-school heroes such as Tupac
and Ice Cube. Like many songs on the superb 2019 album ``ZUU,''
``Ricky'' is short, just 2 minutes 27 seconds --- but Curry has the wit,
and the soul, to pull off brevity.

Jody Rosen is a contributing writer for the magazine and the author of
``Two Wheels Good: The Bicycle on Planet Earth and Elsewhere,'' to be
published next year.
\href{https://www.nytimes3xbfgragh.onion/2019/12/12/magazine/colin-kaepernick-nfl-video.html}{He
last wrote about Colin Kaepernick and image management for a Screenland
column.} llustration by Denise Nestor. Source photograph: Matthew
Eisman/Getty Images.

`Motivation'

\includegraphics{https://static01.graylady3jvrrxbe.onion/packages/flash/multimedia/ICONS/transparent.png}

\includegraphics{https://static01.graylady3jvrrxbe.onion/images/2020/03/15/magazine/15mag-illustration-03/15mag-illustration-03-master675-v2.jpg}

\hypertarget{13}{%
\subparagraph{13}\label{13}}

\hypertarget{normani}{%
\subparagraph{Normani}\label{normani}}

`Motivation'

\href{https://open.spotify.com/track/0rIAC4PXANcKmitJfoqmVm?si=5VXVERD_QOaWBacH-5cgXQ}{
Full Track}

R\&B finds nostalgia's newest frontier. By Lindsay Zoladz

\href{https://open.spotify.com/track/0rIAC4PXANcKmitJfoqmVm?si=5VXVERD_QOaWBacH-5cgXQ}{Full
Track }

Next

R\&B finds nostalgia's newest frontier. By Lindsay Zoladz At the
beginning of the video for ``Motivation,'' a girl settles in front of a
television tuned to BET's defunct show ``106 \& Park.'' The hosts
introduce ``the No. 1 video in the world right now: Normani!'' The scene
is full of nostalgia for the recent past, from the bulky TV to the white
text in the corner of the screen listing the video's directors, Dave
Meyers and Daniel Russell. That spirit inhabits the
video-within-the-video too. Read more

At the beginning of the video for ``Motivation,'' a girl settles in
front of a television tuned to BET's defunct show ``106 \& Park.'' The
hosts introduce ``the No. 1 video in the world right now: Normani!'' The
scene is full of nostalgia for the recent past, from the bulky TV to the
white text in the corner of the screen listing the video's directors,
Dave Meyers and Daniel Russell. That spirit inhabits the
video-within-the-video too.

The schoolyard where Normani dances evokes Britney Spears's
``\ldots{}Baby One More Time'' (1998). Her tank top and commanding strut
recall Beyoncé's ``Crazy in Love'' (2003). And if the basketball court
you see reminds you of Jennifer Lopez and Ja Rule's ``I'm Real (Remix)''
(2001), that's no coincidence. It's the same court and the same
director: Dave Meyers. All three of those videos were released before
Normani, who was raised in New Orleans and Houston, turned 7.

The song itself is a sunny early-aughts reverie, breaking through the
depressive haze of contemporary pop radio. Recently there has been a
surge in '90s-born stars nodding to the musical touchstones of their
youth. A new single from Ariana Grande incorporated a verse from NSYNC;
the R\&B artist Summer Walker samples an early Usher hit on her current
single --- and also got him to perform a duet with her. Like
\href{https://www.youtube.com/watch?v=FKXSh14svlQ}{``Motivation,''}
these songs belong to a new generation of female idols, fast on their
way to becoming the inspiration for the next wave of nostalgic pop
stars.

Lindsay Zoladz is a critic, a reporter and an essayist in Brooklyn. She
was previously a staff writer for The Ringer and before that the pop
critic for New York magazine. It is her first time writing for the
magazine. Illustration by Denise Nestor. Source photograph: Taylor
Hill/WireImage, via Getty Images.

`Doin' Time'

\includegraphics{https://static01.graylady3jvrrxbe.onion/packages/flash/multimedia/ICONS/transparent.png}

\includegraphics{https://static01.graylady3jvrrxbe.onion/images/2020/03/15/magazine/15mag-illustration-12/15mag-illustration-12-master675-v2.jpg}

\hypertarget{14}{%
\subparagraph{14}\label{14}}

\hypertarget{lana-del-rey}{%
\subparagraph{Lana Del Rey}\label{lana-del-rey}}

`Doin' Time'

\href{https://open.spotify.com/track/0Oqc0kKFsQ6MhFOLBNZIGX?si=6QMxgD2LSVCiWTrGx5r1SQ}{
Full Track}

The least culturally sensitive singer still loved by all. By Jonah E.
Bromwich

\href{https://open.spotify.com/track/0Oqc0kKFsQ6MhFOLBNZIGX?si=6QMxgD2LSVCiWTrGx5r1SQ}{Full
Track }

Next

The least culturally sensitive singer still loved by all. By Jonah E.
Bromwich Do people have less tolerance these days for moral ambiguity in
art? That's certainly the sense you get from the going conversations
about it --- not just those parts where reactionaries rail against the
death of artistic freedom at the hands of puritanical wokesters, but
also those parts where people welcome moral engagement with the art they
consume. ``We're talking less about whether a work is good art but
simply whether it's good,'' the New York Times critic Wesley Morris
\href{https://www.nytimes3xbfgragh.onion/interactive/2018/10/03/magazine/morality-social-justice-art-entertainment.html}{wrote}
in 2018. ``Good for us, good for the culture, good for the world.'' Read
more

Do people have less tolerance these days for moral ambiguity in art?
That's certainly the sense you get from the going conversations about it
--- not just those parts where reactionaries rail against the death of
artistic freedom at the hands of puritanical wokesters, but also those
parts where people welcome moral engagement with the art they consume.
``We're talking less about whether a work is good art but simply whether
it's good,'' the New York Times critic Wesley Morris
\href{https://www.nytimes3xbfgragh.onion/interactive/2018/10/03/magazine/morality-social-justice-art-entertainment.html}{wrote}
in 2018. ``Good for us, good for the culture, good for the world.''

Artists plainly sense that they need to be more careful about what they
say and depict, as comedians are constantly complaining; even fans
exercise caution, acknowledging which of their enthusiasms are
``problematic faves,'' ethical splotches marring the purity of their
taste.

We talk about this new mode of cultural consumption constantly, but it's
unclear exactly how real it actually is. Certainly, there's no risk that
all art will be bowdlerized into nice stories about people saving
puppies, but it's not wrong to note a fading appetite for antiheroes and
bad behavior. What is clear is that there is at least one exception, an
artist who breathes easily in this atmosphere: Lana Del Rey. And the
greatest proof that she isn't subject to the anxieties governing her
peers came last year, when her cover of Sublime's 1997 single
\href{https://www.youtube.com/watch?v=H3ypeugNzYA}{``Doin' Time''}
became inescapable, ubiquitous, the alternative song of the summer.

This made some obvious sense: ``Doin' Time'' is a breezy seasonal
anthem, built on a sample of Herbie Mann's version of DuBose Heyward and
George Gershwin's ``Summertime.'' But it also made no sense, because the
song tells a story we're supposedly loath to hear. Our narrator's
girlfriend ``spreads her lovin' all over'' but declines to sleep with
him, so he masturbates grimly and then confesses to murderous desire:
``I'd like to hold her head underwater.'' It's a portrait of male
discontent, an incel ballad. It was also a widely lauded single on the
most critically acclaimed album of Del Rey's career. And vanishingly few
people bothered to cluck about that, or to mention the lyrics at all.

The selective silence was similar when Del Rey's album arrived in late
summer and listeners began quoting their favorite lyrics, especially
lines from the opening track. ``Goddamn man-child,'' they tweeted, or
``Your poetry's bad and you blame the news.'' These were revealing
choices: Del Rey isn't using the title track to humiliate a sad-sack
partner. She's bemoaning the way she remains in thrall to him. It's just
that the lines in which she acknowledges her powerlessness --- ``Why
wait for the best when I could have you?'' --- tended to produce less
intense reactions. Even on websites full of young women joking about the
sometimes-humiliating condition of being attracted to uninspiring young
men, there seemed to be something less appealing about publicly
embracing those lines.

Del Rey has portrayed herself in thrall to shadowy, unpleasant men since
the start of her career. At the beginning of the decade, when Beyoncé
was singing about girls running the world and pop stars were obsessed
with empowerment, Del Rey was crooning that ``My old man is a bad man.
... He loves me with every beat of his cocaine heart'' and that ``Heaven
is a place on Earth where you tell me all the things you want to do.''
Her 2012 debut ended with the song ``This Is What Makes Us Girls'' ---
the definition being ``We don't stick together, cause we put love
first.'' By 2014's album
\href{https://www.nytimes3xbfgragh.onion/2014/06/15/arts/music/lana-del-rey-still-stirs-things-up-with-ultraviolence.html}{``Ultraviolence,''}
she added a darker layer: Along with songs like ``Pretty When You Cry''
and ``{[}Expletive{]} My Way Up to the Top,'' the album quoted the
infamous Crystals song ``He Hit Me (And It Felt Like a Kiss).''

Yet Del Rey is popular in precisely the young-and-righteous demographic
that would supposedly scold an artist into oblivion for this sort of
thing. It helps, surely, that we know Del Rey can't possibly be the
person she is in her songs: That person lives in an alternate dimension
in which the 1950s, '60s and '70s, Hollywood and Vegas, pinups and
``Lolita,'' are all crushed together into a context-free mood board of
picturesque references and familiar phrases. Last year, Del Rey
responded to an essay by the critic Ann Powers with a tweet: ``Never had
a persona. Never needed one. Never will.'' But this is like a magician's
insisting the magic tricks are real. It's also exactly what Del Rey's
persona would say.

We can't have a ``real'' relationship with this fantasy, this construct.
And that has left Del Rey with the freedom to dramatize all the
less-photogenic emotions that have enthralled her fans: self-abasement,
self-destructiveness, moony romanticism, the appeal of someone else's
power. A friend told me that, for her, Del Rey captures how confused
sexuality and longing are, partly because her songs are so melodramatic
that they relieve the tension inherent in craving things like submission
or pain. And in doing so, she has created space for a new generation of
'90s-inflected pop stars --- like, say, Billie Eilish --- to plumb
complicated feelings. The path Del Rey has walked isn't necessarily that
hard to follow: Just reject the modish turn toward authenticity, toward
a carefully curated ``openness'' with fans. It's far more difficult to
condemn someone when nothing --- not her songs, not her profiles, not
even her Instagram feed --- gives you a clue as to who that someone is.

Jonah E. Bromwich is a reporter for The Times's Styles section. He
writes about cultural change --- shifts in the way we date, eat, think
and use language and technology. Illustration by Denise Nestor. Source
photograph: Jon Kopaloff/Getty Images.

`Everybody Wants You'

\includegraphics{https://static01.graylady3jvrrxbe.onion/packages/flash/multimedia/ICONS/transparent.png}

\includegraphics{https://static01.graylady3jvrrxbe.onion/images/2020/03/15/magazine/15mag-antonoff-03/15mag-antonoff-03-master675-v2.jpg}

\hypertarget{15}{%
\subparagraph{15}\label{15}}

\hypertarget{red-hearse}{%
\subparagraph{Red Hearse}\label{red-hearse}}

`Everybody Wants You'

\href{https://open.spotify.com/track/4ab0gHfPvYGeF8qNycLNkE}{ Full
Track}

Jack Antonoff is making good, old-fashioned, cutting-edge 21st-century
albums. By Jody Rosen

\href{https://open.spotify.com/track/4ab0gHfPvYGeF8qNycLNkE}{Full Track
}

Next

Jack Antonoff is making good, old-fashioned, cutting-edge 21st-century
albums. By Jody Rosen ``Everybody Wants You,'' by the trio Red Hearse,
is sneaky --- a song that creeps up on you, working its magic by
insinuation. The song begins in a dulcet mode, with the singer Sam Dew
crooning over synthesizer chords. Then other sounds surface and spread
across the stereo spectrum: a beat that begins as an almost inaudible
pulse; harmony vocals that swell and swoop; a distorted keyboard, panned
hard to the right channel, where it buzzes in your ear like a housefly
or a disquieting thought you can't push from your mind.
\href{https://www.nytimes3xbfgragh.onion/interactive/2020/03/11/magazine/jack-antonoff-profile.html}{Read
the profile}

``Everybody Wants You,'' by the trio Red Hearse, is sneaky --- a song
that creeps up on you, working its magic by insinuation. The song begins
in a dulcet mode, with the singer Sam Dew crooning over synthesizer
chords. Then other sounds surface and spread across the stereo spectrum:
a beat that begins as an almost inaudible pulse; harmony vocals that
swell and swoop; a distorted keyboard, panned hard to the right channel,
where it buzzes in your ear like a housefly or a disquieting thought you
can't push from your mind.
\href{https://www.nytimes3xbfgragh.onion/interactive/2020/03/11/magazine/jack-antonoff-profile.html}{Continue
Reading}

\textbf{Jody Rosen} is a contributing writer for the magazine and the
author of ``Two Wheels Good: The Bicycle on Planet Earth and
Elsewhere,'' to be published next year.
\href{https://www.nytimes3xbfgragh.onion/2019/12/12/magazine/colin-kaepernick-nfl-video.html}{He
last wrote about Colin Kaepernick and image management for a Screenland
column.} \textbf{Arielle Bobb-Willis} is a photographer from New York
who was recently featured in Aperture's ``The New Black Vanguard.'' This
is her first assignment for the magazine.

`Truth Hurts'

\includegraphics{https://static01.graylady3jvrrxbe.onion/packages/flash/multimedia/ICONS/transparent.png}

\includegraphics{https://static01.graylady3jvrrxbe.onion/images/2020/03/15/magazine/15mag-illustration-08/15mag-illustration-08-master675-v2.jpg}

\hypertarget{16}{%
\subparagraph{16}\label{16}}

\hypertarget{lizzo}{%
\subparagraph{Lizzo}\label{lizzo}}

`Truth Hurts'

\href{https://open.spotify.com/track/3HWzoMvoF3TQfYg4UPszDq?si=RPQGRXusQpmcP9Yld2uDyA}{
Full Track}

A song that glows with the fury of the big black women in pop's past. By
Danyel Smith

\href{https://open.spotify.com/track/3HWzoMvoF3TQfYg4UPszDq?si=RPQGRXusQpmcP9Yld2uDyA}{Full
Track }

Next

A song that glows with the fury of the big black women in pop's past. By
Danyel Smith To have a No. 1 pop song, even in these
more-algorithmic-than-rhythmic times, is still a huge deal. The No. 1
spot determines cultural coverage, appearance fees, venue sizes. No. 1
signals to talk-show and festival bookers that an artist is enjoyed by
fans across race and gender borders. No. 1 is a knee in the gut of lax
awards juries. And for black artists? No. 1 means the ``mainstream''
\emph{really} enjoys your music. Read more

To have a No. 1 pop song, even in these more-algorithmic-than-rhythmic
times, is still a huge deal. The No. 1 spot determines cultural
coverage, appearance fees, venue sizes. No. 1 signals to talk-show and
festival bookers that an artist is enjoyed by fans across race and
gender borders. No. 1 is a knee in the gut of lax awards juries. And for
black artists? No. 1 means the ``mainstream'' \emph{really} enjoys your
music.

For black women, it means you have joined a superheroic clique that
includes the Shirelles, Little Eva, the Dixie Cups, Diana Ross (with and
without the Supremes), Dionne Warwick, Aretha Franklin and Gladys
Knight. Gloria Gaynor, Deniece Williams and Tina Turner. Brandy Norwood,
Monica Arnold, Toni Braxton, Ciara Harris, TLC, Lauryn Hill and Alicia
Keys. Mariah Carey and Janet Jackson. Donna Summer and Whitney Houston.
Robyn Rihanna Fenty. Beyoncé Knowles-Carter (with and without Destiny's
Child). Cardi B. And now Melissa Viviane Jefferson --- better known as
Lizzo.

Her 2017 song ``Truth Hurts'' became a viral hit in 2019 via a Netflix
soundtrack, TikTok and tidal waves of radio play, eventually spending
seven weeks at the top of Billboard's Hot 100. ``Truth'' is a deeply
cute bop, a taunt, a million middle fingers atwirl at fear and
heartbreak. Lizzo has given late credit to the songwriter Mina Lioness
for using a Lioness tweet in the song, but ``Truth'' is about more than
being ``100 percent that bitch.'' When Lizzo singsongs, ``You coulda had
a bad bitch/Noncommittal/Help you with your career/Just a little,'' I
hear Lauryn Hill chanting, ``You might win some/But you just lost one.''
That line, from Hill's 1998 ``Lost Ones,'' is one of the most satisfying
chin checks in pop history --- and ``Truth Hurts'' is right in its ring.

\href{https://www.nytimes3xbfgragh.onion/video/arts/music/100000006417770/lizzo-juice.html}{{[}Diary
of a Song: Watch Lizzo Make 'Juice.'{]}}

Lizzo's song glows with the fury of the big black women in pop's past.
Black pop subtexts --- that tension you feel in even the happiest of big
black records --- often function as a counterbalance for historical
undervaluation, and ``Truth'' can be heard as payback for the erasure
and exploitation of singers like Martha Wash. In the mid-1970s, Wash
began her career in a backup duo with Izora Armstead called Two Tons o'
Fun. Renamed the Weather Girls, they were Grammy-nominated for their
1982 classic ``It's Raining Men.'' When Wash re-emerged as an in-demand
session singer, her demo and background vocals were used as lead vocals
--- without permission or proper payment.

At one point Wash was suing three different labels for, variously,
fraud, deceptive packaging, unauthorized use of her voice and commercial
appropriation, including on Seduction's
\href{https://www.youtube.com/watch?v=0IBJlUWDy9c}{``(You're My One and
Only) True Love''}; Black Box's
\href{https://www.youtube.com/watch?v=18N2k1TBBRE}{``Everybody
Everybody''}; and C \& C Music Factory's
\href{https://www.youtube.com/watch?v=LaTGrV58wec}{``Gonna Make You
Sweat (Everybody Dance Now).''} That's Wash singing the words
``everybody,'' ``dance'' and ``now.''

``Sweat'' was on the pop charts for six months, and for two weeks in
1991 it was the No. 1 song in America. In the video, the lithe singer
Zelma Davis lip-synced Wash's vocals. In the video for ``Everybody
Everybody,'' the model Katrin Quinol mouthed Wash's work. Wash's cases
were settled out of court, even as her battle set in motion federal
legislation mandating proper vocal credits. She never got her Lizzo
moment --- her own No. 1.

``The vocals that I did are my vocals,'' Wash said on the news show ``A
Current Affair'' in 1991. ``This is my career. This is my life.'' Insult
was added to injury when C \& C Music Factory's Freedom Williams said in
the same episode, ``I don't mean to be rude, harsh, callous or maligning
or vilifying,'' and then just said it out loud: ``but I'd rather look at
Zelma onstage.''

Got it. So when Lizzo raps, ``We don't {[}expletive{]} with lies, we
don't do goodbyes/We just keep it pushing like/Aye yi yi,'' the message
feels like ``Hi, I'm Wash's lawyer/We demand she be credited and paid/So
please govern yourself accordingly.'' My body mass index quivers to
``Aye yi yi'' joyously, every single time.

The line from Wash to Lizzo connects through the singer-songwriter Kelly
Price, who helped lift the Notorious B.I.G.'s 1997 No. 1
\href{https://www.youtube.com/watch?v=gUhRKVIjJtw}{``Mo Money Mo
Problems''} to immortality with her chorus: ``I don't know/What they
want from me/It's like the more money we come across/The more problems
we see.'' Price did appear in the video --- in a small box, visible only
from the shoulders up.

Sixteen years later, Price, on the red carpet for the ``R\&B Divas: Los
Angeles'' premiere, said she was pitching a reality-TV show called ``Too
Fat for Fame.'' ``How much good singing, dancing, entertaining, acting
do you think the world has missed out on,'' Price asked, ``because
somebody was told, `No, because you're too big'?''

And then here comes Lizzo. Body too big. Personality too big. Talent too
much. Audacity so consistent. Talking about living out of her car. About
not feeling bad about feeling good. Setting the social streets afire by
wearing her ass literally out, courtside at Staples Center. In the music
video for ``Truth Hurts,'' Lizzo's lush bits nearly fall from ivory
lingerie. She rocks her heft like a safety pin through her lip.

Who's gon' stop a black woman from rap-singing from a place where pain
hardens to growth, and tears rise steaming? We tend to uplift a black
woman who works out her blues publicly, to a melody. When black girls
perform, we dance. We karaoke. Everybody screams, ``Goals!'' God bless
Lizzo for pulling up to pop culture big, loud and stomping. God forbid
the black woman in the next cubicle speak a hurting truth in her indoor
voice, though. On behalf of herself. To power. Or about anything else
that matters.

Danyel Smith is a writer based in Los Angeles and the author of the
novels ``More Like Wrestling'' and ``Bliss.'' Her first book of
nonfiction, a history of black women in popular music, will be published
by One World, an imprint of Random House, in February 2021. This is her
first essay for the magazine. Illustration by Denise Nestor. Source
photograph: Karwai Tang/WireImage, via Getty Images.

`Jogging'

\includegraphics{https://static01.graylady3jvrrxbe.onion/packages/flash/multimedia/ICONS/transparent.png}

\includegraphics{https://static01.graylady3jvrrxbe.onion/images/2020/03/15/magazine/15mag-illustration-05/15mag-illustration-05-master675-v2.jpg}

\hypertarget{17}{%
\subparagraph{17}\label{17}}

\hypertarget{richard-dawson}{%
\subparagraph{Richard Dawson}\label{richard-dawson}}

`Jogging'

\href{https://open.spotify.com/track/3TYTZkdgqwkNIA94nKGt1B?si=NHCtjvYDRp6OOddx1ew7VA}{
Full Track}

It's not easy to write the exceedingly normal. By Nitsuh Abebe

\href{https://open.spotify.com/track/3TYTZkdgqwkNIA94nKGt1B?si=NHCtjvYDRp6OOddx1ew7VA}{Full
Track }

Next

It's not easy to write the exceedingly normal. By Nitsuh Abebe Even for
an obscure independent artist, Richard Dawson does not make things easy
on himself. You could almost describe him as a tender English folk
singer, but he's far too steeped in experimental music for that to work.
The poignancy in his records tends to emerge from songs that can be raw,
even slightly grotesque, laced with earthy rattles and croaks and
dissonances. The music he writes is precisely the sort of searching,
ambitious stuff we all say artists should attempt, then make awkward
faces about when one tries. One running joke about his 2017 masterpiece,
``Peasant,'' was how hard it was to recommend to friends: ``Uh, yeah,
it's an entire album about ... characters living in the north of England
... in the Dark Ages?'' Read more

Even for an obscure independent artist, Richard Dawson does not make
things easy on himself. You could almost describe him as a tender
English folk singer, but he's far too steeped in experimental music for
that to work. The poignancy in his records tends to emerge from songs
that can be raw, even slightly grotesque, laced with earthy rattles and
croaks and dissonances. The music he writes is precisely the sort of
searching, ambitious stuff we all say artists should attempt, then make
awkward faces about when one tries. One running joke about his 2017
masterpiece, ``Peasant,'' was how hard it was to recommend to friends:
``Uh, yeah, it's an entire album about ... characters living in the
north of England ... in the Dark Ages?''

So imagine when he returned with ``2020,'' a successor on which he
applies the same premise --- songs about characters living in the north
of England --- to the present day. Singing about weavers and beggars in
pre-Norman Britain at least fit the odd shapes of Dawson's arrangements
and let him revel in poetic language, as when he described a settlement
as \href{https://www.youtube.com/watch?v=jfaW-yD7DoA}{``a dice of houses
cast with clay and sheep dung,''} emerging through ``a soup of starlit
peat smoke.'' The present won't accept any of that stuff. On ``2020''
he's writing in plain language about shipping-center managers and U.F.O.
sightings and ``the guy from the vape shop.'' All of this risks terrific
awkwardness. But if you believe in music's ability to speak frankly
about how we live together, it's exactly the right challenge --- and if
``2020'' doesn't sound as good as ``Peasant,'' maybe it will 1,500 years
from now.

Throughout ``2020,'' Dawson manages to wring huge emotion from the
extremely ordinary. A
\href{https://www.youtube.com/watch?v=NkSvn23fbAg}{red-faced soccer
dad}, driving his child home from a poor performance, swallows his
disappointment and lets the kid choose dinner; pub owners are surprised
to find neighbors helping to save the place from flooding; a homeless
man listens to children's laughter as he nods off at a chicken joint.
It's difficult to pin down what it is about Dawson's performances that
makes these things so richly affecting --- my money's either on the slow
winding of his mournful melodies or the bellowing mournful-bear quality
of his voice --- but the ultimate effect is that he feels more humane
than almost anything else on offer.

``Recently I've been struggling with anxiety,'' begins the album's first
single, ``Jogging,'' after a flurry of lumbering power chords. The
characters in these songs manage to be studiously gender-neutral, but
Dawson has let slip in interviews that he pictures this one as a woman.
Cut loose from her job as a school counselor, she does freelance
graphic-design work and fears leaving her flat; she senses judging eyes
on the bus, imagines buskers slipping cuss words into ``Wonderwall'' as
she passes. ``I know I must be paranoid,'' she says, but as the song
proceeds you consider that she may not be. She lives against a backdrop
of austerity, social-service cuts, gig-economy torpor and post-Brexit
nativism. The Kurdish family downstairs had a brick thrown through their
window, and the police don't seem to care. ``I feel the atmosphere round
here is growing nastier,'' she says. A doctor suggests beta blockers and
exercise, and she's surprised to find herself running hundreds of miles,
the song lumbering and trudging along with her.

The great gut punches in Dawson's songs are the moments where people
seem to care about one another on levels more visceral than any love
song ever claims. Many involve parents and children, but there's an odd
turn at the end of ``Jogging'' that has the same raw emotional pull. As
the song soars up to a climax, the narrator suddenly switches to direct
address: ``Would you like to sponsor me for running the London
Marathon?'' she asks. ``Though it's really daunting, we're aiming to
raise a thousand pounds for the British Red Cross.'' Running, it seems,
hasn't just helped this person physically; it has become an avenue for
her to do one small, plodding thing to cut through the bad atmosphere
and help someone else. And the real gut punch is what this means for the
rest of the story we've just heard. Is it possible our anxious, shut-in
jogger has been pouring out all of this --- the work history, the
paranoia, the doctor's appointments, the Kurdish family --- to some
stranger, somewhere on a doorstep or a park bench?

Nitsuh Abebe is a story editor for the magazine. He was formerly the pop
critic for New York magazine and a columnist for Pitchfork. Illustration
by Denise Nestor. Source photograph: Gaelle Beri.

`Follow God'

\includegraphics{https://static01.graylady3jvrrxbe.onion/packages/flash/multimedia/ICONS/transparent.png}

\includegraphics{https://static01.graylady3jvrrxbe.onion/images/2020/03/15/magazine/15mag-illustration-09/15mag-illustration-09-master675-v3.jpg}

\hypertarget{18}{%
\subparagraph{18}\label{18}}

\hypertarget{kanye-west}{%
\subparagraph{Kanye West}\label{kanye-west}}

`Follow God'

\href{https://open.spotify.com/track/2QpGZOhTCHHiKmpSO9FW4h?si=ETIXCm3WQC-MpAZ2BiTyIw}{
Full Track}

He joins a long line of secular artists to take up the gospel. By Jody
Rosen

\href{https://open.spotify.com/track/2QpGZOhTCHHiKmpSO9FW4h?si=ETIXCm3WQC-MpAZ2BiTyIw}{Full
Track }

Next

He joins a long line of secular artists to take up the gospel. By Jody
Rosen The archetypal pathway for American popular music runs from the
sacred to the profane, from gospel roots to secular pop. But some of
music's biggest stars have traveled in the opposite direction. In 2019,
Kanye West --- like Bob Dylan and Prince and many others before him ---
sought refuge from fame (and infamy) in old-time religion, founding a
gospel group and releasing ``Jesus Is King,'' an album-length
testimonial.

The archetypal pathway for American popular music runs from the sacred
to the profane, from gospel roots to secular pop. But some of music's
biggest stars have traveled in the opposite direction. In 2019, Kanye
West --- like Bob Dylan and Prince and many others before him --- sought
refuge from fame (and infamy) in old-time religion, founding a gospel
group and releasing ``Jesus Is King,'' an album-length testimonial.

{[}Intro{]}

Father, I stretch -\/-br-\/- Stretch my hands to You

In ``Follow God,'' he is by turns a proselytizer and a penitent,
fulminating over a sample whose beseeching lyric echoes two earlier West
songs from 2016, ``Father Stretch My Hands'' Parts 1 and 2.

{[}Verse{]}

Not another word, letter, picture or a decimal -\/-br-\/- Wrestlin' with
God, I don't really want to wrestle

Beginning with early hits like ``Jesus Walks'' (2004), West's songs have
cast him as a man caught in a tussle between God and mammon, a spasming
seeker prone to embracing false prophets --- Donald Trump, yes, but also
the man in the mirror. It was less than a decade ago that West nicknamed
himself Yeezus and cheekily declared, ``I am a god.''

{[}Outro{]}

Give him some advice, he starts spazzin' on me -\/-br-\/- I start
spazzin' back, he said, ``That ain't Christ-like'' -\/-br-\/- I said,
``Aah!''

Kanye has never been a great pure M.C. His true form is the rant, those
moments --- on record, onstage or online --- when he stops rapping and
gives voice to his ego, superego and id. In the rant that closes
``Follow God,'' he describes a conversation with his father. (Or is it
God the Father?) The song ends with Kanye letting out a shriek: a cry of
pain, a howl of spiritual ecstasy or --- maybe, probably --- both.

Jody Rosen is a contributing writer for the magazine and the author of
``Two Wheels Good: The Bicycle on Planet Earth and Elsewhere,'' to be
published next year.
\href{https://www.nytimes3xbfgragh.onion/2019/12/12/magazine/colin-kaepernick-nfl-video.html}{He
last wrote about Colin Kaepernick and image management for a Screenland
column.} Illustration by Denise Nestor. Source photographs: Kwaku
Alston/Corbis, via Getty Images.

`Harmony Hall'

\includegraphics{https://static01.graylady3jvrrxbe.onion/packages/flash/multimedia/ICONS/transparent.png}

\includegraphics{https://static01.graylady3jvrrxbe.onion/images/2020/03/15/magazine/15mag-ezra-01/15mag-ezra-01-master675.jpg}

\includegraphics{https://static01.graylady3jvrrxbe.onion/packages/flash/multimedia/ICONS/transparent.png}

\includegraphics{https://static01.graylady3jvrrxbe.onion/images/2020/03/15/magazine/15mag-ezra/15mag-ezra-master675-v5.jpg}

\hypertarget{19}{%
\subparagraph{19}\label{19}}

\hypertarget{vampire-weekend}{%
\subparagraph{Vampire Weekend}\label{vampire-weekend}}

`Harmony Hall'

\href{https://open.spotify.com/track/39exKIvycQDgs4T6uXdyu0?si=7Wg7njivRLKJrGsRAYT7fg}{
Full Track}

Ezra Koenig on the beauty of the jam band. By Steven Hyden

\href{https://open.spotify.com/track/39exKIvycQDgs4T6uXdyu0?si=7Wg7njivRLKJrGsRAYT7fg}{Full
Track }

Next

Ezra Koenig on the beauty of the jam band. By Steven Hyden Ezra Koenig
is the singer and primary songwriter for Vampire Weekend, whose 2019 LP,
``Father of the Bride,'' debuted at No. 1 on the Billboard chart and won
the Grammy for best alternative music album. I asked him about how the
album was frequently likened by critics --- sometimes favorably,
sometimes not --- to jam bands like the Grateful Dead and Phish.
\href{https://www.nytimes3xbfgragh.onion/interactive/2020/03/11/magazine/vampire-weekend-interview.html}{Read
the interview}

Ezra Koenig is the singer and primary songwriter for Vampire Weekend,
whose 2019 LP, ``Father of the Bride,'' debuted at No. 1 on the
Billboard chart and won the Grammy for best alternative music album. I
asked him about how the album was frequently likened by critics ---
sometimes favorably, sometimes not --- to jam bands like the Grateful
Dead and Phish.

\textbf{In coverage of a song like ``Harmony Hall,'' there was a
recurring theme of ``Vampire Weekend is being provocative by having
these jam-band signifiers. They're being contrarian or they're
challenging the audience.'' Was that your intention?} To a large extent
it was just looking for what we thought was cool in the unfashionable
pile. I legitimately think that Phish are more inspiring,
forward-thinking, exciting and talented than a lot of what was higher up
in the cool hierarchy. There might have been stuff when I was 14 that I
thought was cooler than jam bands, for whatever reason. And then I just
flipped. It's not my gut anymore. I'm familiar with a certain type of
hierarchy of cool. I know it. The artistic part of me just doesn't buy
it anymore.
\href{https://www.nytimes3xbfgragh.onion/interactive/2020/03/11/magazine/vampire-weekend-interview.html}{Continue
Reading}

\textbf{Steven Hyden} is the author of four books, including ``This
Isn't Happening: Radiohead's `Kid A' and the Beginning of the 21st
Century,'' out in September.
\href{https://www.nytimes3xbfgragh.onion/2018/07/02/magazine/national-mood-emotions-polls-online.html}{He
last wrote for the magazine about the national mood.} \textbf{Arielle
Bobb-Willis} is a photographer from New York who was recently featured
in Aperture's ``The New Black Vanguard.'' This is her first assignment
for the magazine.

`Stupid Horse'

\includegraphics{https://static01.graylady3jvrrxbe.onion/packages/flash/multimedia/ICONS/transparent.png}

\includegraphics{https://static01.graylady3jvrrxbe.onion/images/2020/03/15/magazine/15mag-music-100-gecs/15mag-music-100-gecs-master675-v2.jpg}

\hypertarget{20}{%
\subparagraph{20}\label{20}}

\hypertarget{100-gecs}{%
\subparagraph{100 gecs}\label{100-gecs}}

`Stupid Horse'

\href{https://open.spotify.com/track/23Dapn8107GgxvXDIsPwWm?si=xX4CMd6EQsC7Zst_wa121Q}{
Full Track}

There's a long history to music that assaults the senses. By Mark
Richardson

\href{https://open.spotify.com/track/23Dapn8107GgxvXDIsPwWm?si=xX4CMd6EQsC7Zst_wa121Q}{Full
Track }

Next

There's a long history to music that assaults the senses. By Mark
Richardson Dylan Brady and Laura Les met in the St. Louis suburbs in
2012 and now make manic music together as 100 gecs. They take the most
flimsy and tasteless signifiers of contemporary pop --- processed
vocals, dubstep bass drops, trap high-hats, cheap synths, mall-punk
yelps, ska's overeager bounce --- and channel them into a brute-force
electro-bubblegum attack. Read more

Dylan Brady and Laura Les met in the St. Louis suburbs in 2012 and now
make manic music together as 100 gecs. They take the most flimsy and
tasteless signifiers of contemporary pop --- processed vocals, dubstep
bass drops, trap high-hats, cheap synths, mall-punk yelps, ska's
overeager bounce --- and channel them into a brute-force
electro-bubblegum attack.

``stupid horse,'' from their 2019 album
\href{https://www.nytimes3xbfgragh.onion/2019/09/04/arts/music/100-gecs-review.html}{``1000
gecs,''} is pop intensified, a 220-volt current coursing through our
110-volt brains. It is, in a word, extreme: The tempo is extremely fast,
and the melody is extremely catchy. The lyrics, about an equine-related
caper involving a sports car, are extremely absurd. They pitch the
vocals extremely high, and the ripping guitar solo --- along with the
uncanny twirling ``Whoo!'' that signals its launch --- is extremely
weird.

Those not on 100 gecs' wavelength find them extremely annoying. ``So
this is what music has come to?'' is a common, and possibly intended,
response, leading to discussions about the warped tastes of digital
natives, about information overload and diminishing attention spans. But
this sort of provocation has a history. Those who are tuned into their
squiggly frequency, this writer included, hear in ``stupid horse'' a
connection to songs past where the joyful too-muchness of it all is the
whole point.

Here are five more in that ignoble lineage:

\textbf{Sleigh Bells} ******
\emph{\href{https://www.youtube.com/watch?v=c3bSCZXKvUA}{``Tell 'Em''
(2010)}}

As Sleigh Bells, Alexis Krauss and the guitarist Derek Edward Miller
pushed the limits of gnarly guitar noise and perky singsong tunefulness.
It was an mp3 world, and hearing the impossibly distorted ``Tell 'Em''
for the first time made you wonder if you had downloaded the wrong
version.

\textbf{Party Animals}
\emph{\href{https://www.youtube.com/watch?v=2er9d5hb_Fc}{``Have You Ever
Been Mellow'' (1995)}}

Happy hardcore was a zippy subgenre of rave that bounced around the
world at 170 beats per minute in the mid-1990s. The economy was good and
the drugs were better, which meant that a sped-up version of an Olivia
Newton-John hit reworked by Dutch producers could sell tens of thousands
of copies.

\textbf{Naked City}
\emph{\href{https://www.youtube.com/watch?v=D12QW4sh1-k}{``Hellraiser''
(1990)}}

100 gecs cite John Zorn's band Naked City as an influence. On the
inhumanly fast and startlingly brief ``Hellraiser,'' they cycle between
grindcore (that ferocious triple-time variant of punk) and tasteful
supper-club jazz and back again in less than 45 seconds.

\textbf{Hot Butter}
\emph{\href{https://www.youtube.com/watch?v=YK3ZP6frAMc}{``Popcorn''
(1972)}}

An oddball instrumental that hit the Top 10 in the U.S., its Moog-driven
arrangement sounded like the future. But what most people remember about
it is the weapons-grade infectiousness of its melody, which you're
probably hearing in your head right now. (Sorry.)

\textbf{The Trashmen}
\emph{\href{https://www.youtube.com/watch?v=9Gc4QTqslN4}{``Surfin'
Bird'' (1963)}}

Bizarro voice? Mind-numbing repetition? All hopped-up and too fast to
dance to? Perfectly engineered to drive grown-ups mad? Check, check,
check and check --- or better yet, gec, gec, gec and gec.

Mark Richardson is the rock and pop music critic at The Wall Street
Journal and was previously the editor in chief of Pitchfork.
\href{https://www.nytimes3xbfgragh.onion/interactive/2019/03/07/magazine/top-songs.html\#/sons-of-kemet}{He
wrote about Sons of Kemet for the 2019 Music Issue.} Illustration by
Denise Nestor

`Playing Games'

\includegraphics{https://static01.graylady3jvrrxbe.onion/packages/flash/multimedia/ICONS/transparent.png}

\includegraphics{https://static01.graylady3jvrrxbe.onion/images/2020/03/15/magazine/15mag-music-summer-walker/15mag-music-summer-walker-master675.jpg}

\hypertarget{21}{%
\subparagraph{21}\label{21}}

\hypertarget{summer-walker}{%
\subparagraph{Summer Walker}\label{summer-walker}}

`Playing Games'

\href{https://open.spotify.com/track/4hWgC5BV5M7EtTleKVXNxw?si=zZFBfWIpS4-Rk1b_bloNPA}{
Full Track}

When digital life and real life collide. By Lindsay Zoladz

\href{https://open.spotify.com/track/4hWgC5BV5M7EtTleKVXNxw?si=zZFBfWIpS4-Rk1b_bloNPA}{Full
Track }

Next

When digital life and real life collide. By Lindsay Zoladz Destiny's
Child's \href{https://www.youtube.com/watch?v=sQgd6MccwZc}{``Say My
Name''} --- in which a phone call leads to suspicions of infidelity ---
may not have played out the same way in the age of the text message. But
in ``Playing Games,'' Summer Walker elegantly sketches out an updated
version of the same feeling. ``Swear that you doing the most, but we
take a picture, can't post it,'' she sings to someone. With its slow
guitar lick and slurry production, it's a late-night outpouring of
bottled-up grievances. Read more

Destiny's Child's
\href{https://www.youtube.com/watch?v=sQgd6MccwZc}{``Say My Name''} ---
in which a phone call leads to suspicions of infidelity --- may not have
played out the same way in the age of the text message. But in ``Playing
Games,'' Summer Walker elegantly sketches out an updated version of the
same feeling. ``Swear that you doing the most, but we take a picture,
can't post it,'' she sings to someone. With its slow guitar lick and
slurry production, it's a late-night outpouring of bottled-up
grievances.

There's a remix in which Bryson Tiller voices the other side of the
conversation, but something about the solo version feels more in line
with what Walker is trying to capture: a fed-up monologue that may well
go unanswered.

Walker knows something about the smoke-and-mirrors game of
self-expression in the digital era. Last November, she canceled 20
concert dates because, as she put it, touring ``doesn't really coexist
with my social anxiety and introverted personality.'' On social media,
she can seem confessional and spontaneous, leading some to accuse her of
faking her anxiety. Both her music and her persona remind us, finally,
just how much digital communication allows us to control narratives, for
better and for worse. The person on the other end may never know why you
didn't post a picture or return a text --- or how many selfies were
deleted before getting one that told just the right story.

Lindsay Zoladz is a critic, a reporter and an essayist in Brooklyn. She
was previously a staff writer for The Ringer and before that the pop
critic for New York magazine. It is her first time writing for the
magazine. Illustration by Denise Nestor. Source photograph: Prince
Williams/WireImage, via Getty Images.

`Hot Girl Summer'

\hypertarget{22}{%
\subparagraph{22}\label{22}}

\hypertarget{megan-thee-stallion}{%
\subparagraph{Megan Thee Stallion}\label{megan-thee-stallion}}

`Hot Girl Summer'

\href{https://open.spotify.com/track/5N1o6d8zGcSZSeMFkOUQOk}{ Full
Track}

She turned ``hot'' into a state of mind. By Jenna Wortham

\href{https://open.spotify.com/track/5N1o6d8zGcSZSeMFkOUQOk}{Full Track
}

Next

She turned ``hot'' into a state of mind. By Jenna Wortham The summer of
2019 had a sound, a meme, a hashtag and an entire mood set by a then
24-year-old rapper from Houston named Megan Thee Stallion.
\#HotGirlSummer started out as a tweet that morphed into a meme that
became a chart-topping track featuring Nicki Minaj that catapulted the
artist into a national spotlight, with legions of fans she has nicknamed
her ``Hotties.'' The genius of ``Hot Girl Summer'' is that it was much
more than a song --- it was a feeling, propagated by social media,
particularly Instagram and TikTok, of freedom and abandon that could
contain everything from a performance of Megan twerking while wearing a
particularly bright pair of lime green chaps to a photograph of Tom
Hanks smiling beatifically while wearing a white dress shirt tied in a
knot.
\href{https://www.nytimes3xbfgragh.onion/interactive/2020/03/11/magazine/megan-thee-stallion-interview.html}{Read
the interview}

The summer of 2019 had a sound, a meme, a hashtag and an entire mood set
by a then 24-year-old rapper from Houston named Megan Thee Stallion.
\#HotGirlSummer started out as a tweet that morphed into a meme that
became a chart-topping track featuring Nicki Minaj that catapulted the
artist into a national spotlight, with legions of fans she has nicknamed
her ``Hotties.'' The genius of ``Hot Girl Summer'' is that it was much
more than a song --- it was a feeling, propagated by social media,
particularly Instagram and TikTok, of freedom and abandon that could
contain everything from a performance of Megan twerking while wearing a
particularly bright pair of lime green chaps to a photograph of Tom
Hanks smiling beatifically while wearing a white dress shirt tied in a
knot.
\href{https://www.nytimes3xbfgragh.onion/interactive/2020/03/11/magazine/megan-thee-stallion-interview.html}{Continue
Reading}

\textbf{Jenna Wortham} is a staff writer for the magazine and co-host of
the podcast ``Still
Processing.''\href{https://www.nytimes3xbfgragh.onion/2019/08/28/magazine/how-queer-people-brought-some-actual-reality-to-dating-reality-tv.html}{}\href{https://www.nytimes3xbfgragh.onion/2019/08/28/magazine/how-queer-people-brought-some-actual-reality-to-dating-reality-tv.html}{She
previously wrote about a reboot of the show ``The L Word'' for a
Screenland column.} \textbf{Arielle Bobb-Willis} is a photographer from
New York who was recently featured in Aperture's ``The New Black
Vanguard.'' This is her first assignment for the magazine.

`Redesigning Women'

\includegraphics{https://static01.graylady3jvrrxbe.onion/packages/flash/multimedia/ICONS/transparent.png}

\includegraphics{https://static01.graylady3jvrrxbe.onion/images/2020/03/15/magazine/15mag-music-highwomen/15mag-music-highwomen-master675-v2.jpg}

\hypertarget{23}{%
\subparagraph{23}\label{23}}

\hypertarget{the-highwomen}{%
\subparagraph{The Highwomen}\label{the-highwomen}}

`Redesigning Women'

\href{https://open.spotify.com/track/31S0wn0oWPO6EVX084IVU1?si=YFLxI794RY2hCnyz-_XfOw}{
Full Track}

Four of country's leading women, in one act. By Lindsay Zoladz

\href{https://open.spotify.com/track/31S0wn0oWPO6EVX084IVU1?si=YFLxI794RY2hCnyz-_XfOw}{Full
Track }

Next

Four of country's leading women, in one act. By Lindsay Zoladz In 2015,
country music was rocked by a controversy ridiculous enough to earn the
name ``tomato-gate.'' Explaining why mainstream country stations played
only a handful of female artists, the radio consultant Keith Hill
reached for a metaphor: ``I play great female records, and we've got
some right now. They're just not the lettuce in our salad.'' The lettuce
included top-selling male artists like Blake Shelton and Keith Urban;
the women, Hill went on to explain, were the tomatoes. Read more

In 2015, country music was rocked by a controversy ridiculous enough to
earn the name ``tomato-gate.'' Explaining why mainstream country
stations played only a handful of female artists, the radio consultant
Keith Hill reached for a metaphor: ``I play great female records, and
we've got some right now. They're just not the lettuce in our salad.''
The lettuce included top-selling male artists like Blake Shelton and
Keith Urban; the women, Hill went on to explain, were the tomatoes.

From Kacey Musgraves to Kelsea Ballerini, Miranda Lambert to Maren
Morris, we are living in a golden age of women in country --- but most
of them are succeeding despite the attitudes of men like Hill. On
``Redesigning Women,'' the debut single from the Highwomen, you hear
four women's voices --- Morris, Amanda Shires, Brandi Carlile and
Natalie Hemby. Shires has said the idea of assembling an all-female
supergroup came to her when her daughter told her she wanted to be a
musician: ``I thought the worst thing that could happen would be that
she would go for country, because there are only currently two women's
voices that you can actually hear.''

As ``Redesigning Women'' reminds us, plenty of women toil half their
lives in the kitchen --- which means they can throw together a dish with
a lot more depth than a sad, wilting bowl of lettuce and a couple of
tomatoes.

Lindsay Zoladz is a critic, a reporter and an essayist in Brooklyn. She
was previously a staff writer for The Ringer and before that the pop
critic for New York magazine. It is her first time writing for the
magazine. Illustration by Denise Nestor. Source photograph: Matt
Baron/Shutterstock.

`10 \%'

\includegraphics{https://static01.graylady3jvrrxbe.onion/packages/flash/multimedia/ICONS/transparent.png}

\includegraphics{https://static01.graylady3jvrrxbe.onion/images/2020/03/15/magazine/15mag-illustration-07/15mag-illustration-07-master675-v2.jpg}

\hypertarget{24}{%
\subparagraph{24}\label{24}}

\hypertarget{kaytranada}{%
\subparagraph{Kaytranada}\label{kaytranada}}

`10 \%'

\href{https://open.spotify.com/track/41SwdQIX8Hy2u6fuEDgvWr?si=iBwFWhC1QzuhIiBvdv9D3w}{
Full Track}

The Canadian producer's music can help you feel surrounded. By Angela
Flournoy

\href{https://open.spotify.com/track/41SwdQIX8Hy2u6fuEDgvWr?si=iBwFWhC1QzuhIiBvdv9D3w}{Full
Track }

Next

The Canadian producer's music can help you feel surrounded. By Angela
Flournoy At the beginning of 2020, I wanted to close my eyes and move. I
did not feel well for a number of reasons. I worried that the country
was maybe going to start a new war, and maybe going to re-elect its
president; I struggled with a moral and professional dilemma; and my
mother was suddenly, seriously ill. I was back home in suburban Los
Angeles --- in her home, holed up in my old bedroom --- when I wasn't
shuttling back and forth to the hospital. I was in a funk to end all
funks. What I needed was to move, to sweat, to swing into a dark place
filled with other people and let out a wail. But there was no time to
sneak off to a club or a bar. So instead, I listened to an
electronic-disco album called ``Bubba,'' by Kay­tranada, many, many
times. Read more

At the beginning of 2020, I wanted to close my eyes and move. I did not
feel well for a number of reasons. I worried that the country was maybe
going to start a new war, and maybe going to re-elect its president; I
struggled with a moral and professional dilemma; and my mother was
suddenly, seriously ill. I was back home in suburban Los Angeles --- in
her home, holed up in my old bedroom --- when I wasn't shuttling back
and forth to the hospital. I was in a funk to end all funks. What I
needed was to move, to sweat, to swing into a dark place filled with
other people and let out a wail. But there was no time to sneak off to a
club or a bar. So instead, I listened to an electronic-disco album
called ``Bubba,'' by Kay­tranada, many, many times.

Kay­tranada, born Louis Kevin Celestin, is a 27-year-old
Haitian-Canadian producer from Montreal. He made his first album,
``99.9\%'' (2016), in his mother's house while sharing a basement
bedroom with his younger brother. He came out as gay shortly before the
album's release. Sometimes I imagine Kay­tranada in his early 20s,
having never yet been with a man
(\href{https://www.thefader.com/2016/04/05/kaytranada-album-interview-99-percent}{as
he told The Fader in 2016}), sharing a bedroom in his family home.
Perhaps he also felt a need to swing into a dark place filled with
people and let out a wail. Perhaps this is how disco found him. It is a
genre sometimes derided for prizing feeling over thinking, for using
thumping bass and simple repetition to get people onto a dance floor.
But that discounts the healing that can happen when people feel the urge
to move.

``Bubba,'' Kay­tranada's second album, is 51 minutes and 17 tracks long,
but it feels like a single D.J. set, with smooth transitions from groove
to upbeat groove. Toward the middle, you reach the song ``10\%,'' whose
hook has the Colombian-American singer Kali Uchis asking a question:
``You keep on takin' from me, but where's my 10 percent?'' Kay­tranada
layers Uchis's voice so that there is a slight echo to her inquiry; it
sounds as if she's right next to the listener's ear, cool but insistent,
whispering while dancing. ``10\%'' is a song about exploitation,
presumably by some manager or business handler (``Run my money to
me/Don't act like you didn't know''), but it is also a song about being
weary of pretense. ``You're trying way too hard/Ego is not your
friend,'' Uchis sings, though she isn't going to stop to deal with the
offender; she's got time.

The composition of the song itself seems to eschew showy effort, which
is not the same as being uncomplicated. ``10\%'' is a song whose
disparate parts --- steady conga drums, synthesized harps, jaunty
violins --- are all in service to a single purpose: movement. Most of
Kay­tranada's dance songs have this propulsive quality, owing in large
part to his unique use of funk-inflected drum sequences; they drive the
rhythm forward as opposed to holding it in a steady, stultifying loop.

``10\%'' is a song you can dance to with your eyes closed. The lyrics
provide texture, but it's the rhythm itself that advises you on what to
do, and you can do it as awkwardly or as expertly as you're physically
capable. Twirl, two-step or kind of roller-skate in place, sure. In this
way, the song is selling what Soul­Cycle is ostensibly selling, what the
forever-upbeat fitness coach on my workout app is selling. Looking good
helps, maybe, but the real item for sale is feeling good. And whether or
not you can find that dark dance floor full of strangers, ``10\%'' is
also selling the feeling of not being alone. All music has elements of
emotional aspiration (we sing a love ballad and feel our own hearts
swell), but dance music offers a particular, seductive aspiration ---
the possibility of being in physical and emotional communion with
others, if only for three minutes.

I have yet to find a suitable club or bar. But I have listened to this
song in the dark and imagined myself there, in the midst of a sweaty,
rapturous crowd, with the only people looking at me being those within
arm's reach --- and even then, only for a glimpse, before the music
carries them elsewhere.

Angela Flournoy is the author of the novel ``The Turner House.''
\href{https://www.nytimes3xbfgragh.onion/interactive/2019/03/07/magazine/top-songs.html\#/mariah-carey}{She
wrote about Mariah Carey for the 2019 Music Issue.} Illustration by
Denise Nestor. Source photographs: Stephen Lovekin/Shutterstock.

`The Center Won't Hold'

\includegraphics{https://static01.graylady3jvrrxbe.onion/packages/flash/multimedia/ICONS/transparent.png}

\includegraphics{https://static01.graylady3jvrrxbe.onion/images/2020/03/15/magazine/15mag-illustration-14/15mag-illustration-14-master675-v2.jpg}

\hypertarget{25-1}{%
\subparagraph{25}\label{25-1}}

\hypertarget{sleater-kinney}{%
\subparagraph{Sleater-Kinney}\label{sleater-kinney}}

`The Center Won't Hold'

\href{https://open.spotify.com/track/0SJZFV7p1Nhd9xQnUonHSY?si=2iap8ASQRuS5DQPOtEAaOw}{
Full Track}

The band finds a new sound in an act of creative destruction --- and
loses Janet Weiss. By Hanif Abdurraqib

\href{https://open.spotify.com/track/0SJZFV7p1Nhd9xQnUonHSY?si=2iap8ASQRuS5DQPOtEAaOw}{Full
Track }

Next

The band finds a new sound in an act of creative destruction --- and
loses Janet Weiss. By Hanif Abdurraqib It is vital for artists to have
an understanding of what they are good at, so that they might revel in
the satisfaction of departure when the time comes. Through their 25
years and 10 albums, Sleater-Kinney have proved adept at a wide array of
moves: volume and tone modulation, lyrics that balance playful cynicism
with biting reality. And still, nothing consistently moves me to the
edge of my seat like a Sleater-Kinney title track, especially if that
title track opens the album it's named after. These openers immediately
draw a line in the sand, as if to say: If you can't make it through
this, you're not going to like what's on the other side. Read more

It is vital for artists to have an understanding of what they are good
at, so that they might revel in the satisfaction of departure when the
time comes. Through their 25 years and 10 albums, Sleater-Kinney have
proved adept at a wide array of moves: volume and tone modulation,
lyrics that balance playful cynicism with biting reality. And still,
nothing consistently moves me to the edge of my seat like a
Sleater-Kinney title track, especially if that title track opens the
album it's named after. These openers immediately draw a line in the
sand, as if to say: If you can't make it through this, you're not going
to like what's on the other side.

\href{https://www.nytimes3xbfgragh.onion/2019/08/14/arts/music/sleater-kinney-the-center-wont-hold-review.html}{``The
Center Won't Hold,''}like many of its kin in the Sleater-Kinney
title-track canon, takes its position at the top of the band's 2019
album (their second since their 2015 return from hiatus). The song pulls
a magic trick: It's a song inside a song. In the first part, which lasts
about 2 minutes 8 seconds, voices chant, ``The center won't hold'' over
and over again in a heavy sigh, on every third beat, with Carrie
Brownstein wringing the emotion out of the verse. The mood is both
ominous and cocoonlike, akin to watching tree branches lash a window
during a storm while someone familiar holds you. It is also the sound of
Sleater-Kinney turning away from whatever expectations a listener might
have held.

When you listen to the song on repeat, the final 56 seconds feel
inevitable, but I would give anything to return to the moment when I
first heard it kick out of one gear and into the next. It feels like the
steep drop of a roller coaster after a meandering journey to the top.
The guitars grow louder, faster. The unmistakable drums of Janet Weiss
are unchained. The line ``The center won't hold'' gets a new ferocity.
As gifted as Sleater-Kinney have shown themselves to be as lyricists, a
secondary excitement comes when those lyrics are honed to a fine point.
A repeated line or two act as the glue for the instruments' tearing a
room apart. Here, Corin Tucker loops the words with increasing volume,
an attempt to keep up with the growing growl and thrash of sound at her
back. ``The Center Won't Hold'' is the perfect encapsulation of
Sleater-Kinney at their best, eager and attacking.

The song's lyrics and the album's title bring to mind the W.B. Yeats
poem ``The Second Coming'' --- ``Things fall apart; the centre cannot
hold;/Mere anarchy is loosed upon the world'' --- it's hard not to think
of them as a parable. The band didn't emerge from this album as the same
band that went into the making of it. Janet Weiss, one of the great rock
drummers of all time, played on the album but
\href{https://www.nytimes3xbfgragh.onion/2019/07/31/arts/music/sleater-kinney-st-vincent.html}{departed
the band before the album came out.} For those who have known
Sleater-Kinney and loved them for a long time, it might feel as if the
band is in peak form for one last time.

Hearing the album all the way through, you can understand why Weiss
didn't feel as if her role in the band made as much sense as it used to.
The back half of the title track, with all its sonic ferocity, is an
exception to the otherwise-glossy quality of the songs. The album still
teems with Sleater-Kinney's characteristic wit and rage, but it's
packaged in a lighter and more dreamlike set of sounds. There isn't as
much room for Weiss's signature drumming.

``The Center Won't Hold'' found a band trying newness, shifting from one
template to the next throughout the album but not sounding fully
comfortable in any of them. It would be easy for a longtime fan to feel
disappointment. But I eventually set aside nostalgia to allow myself
gratitude and admiration --- not entirely for the music itself but for
the aims of the band in this new era. Consistent greatness, paired with
tenure, can be a recipe for complacency. Someone, somewhere, will always
be eager for a Sleater-Kinney album, even if it is the exact same album
as the one before it. But Corin Tucker and Carrie Brownstein have always
been curious about the places they could push their sound, and on this
album, they seem to be committing to a new set of curiosities --- for
the sake of survival, but also for the sake of refusing stagnation. I'm
thankful for a band willing to turn corners, even as I'm heartbroken
about a member of that band's being left behind.

When praising evolution and risk and becoming something beyond what
people expect of you, it must also be said that the act is often one of
undoing. The last 56 seconds of ``The Center Won't Hold'' feel like the
site of an old fight that was neither won nor lost, brilliant anarchy
for anarchy's sake. A band, pushing themselves and their sound to its
limit, before turning out the lights on an era and running toward a new
one.

Hanif Abdurraqib is a poet, an essayist and a cultural critic from
Columbus, Ohio.
\href{https://www.nytimes3xbfgragh.onion/interactive/2019/03/07/magazine/top-songs.html\#/bruce-springsteen}{He
wrote about Bruce Springsteen in the 2019 Music Issue.} Illustration by
Denise Nestor. Source photograph: Chad Batka for The New York Times.

Photographs by Arielle Bobb-Willis for The New York Times.

Videos by Arielle Bobb-Willis and Jacqueline Ayala for The New York
Times.

Illustrations by Denise Nestor.

Stylists: Antonoff: Savannah White; Eilish: Samantha Burkhart; Howard:
Nonja McKenzie; King Princess: Sean Knight; Koenig: Evan Simonitsch; Lil
Nas X: Hodo Musa; Megan Thee Stallion: E.J. King.

Grooming: Antonoff: Nicole Elle King; Koenig: Candice Birns.

Hair: Eilish: Mara Roszak; Howard: Monae Everett; King Princess: Jerome
Terry; Lil Nas X: Britney Thomas; Megan Thee Stallion: Kellon Williams.

Makeup: Eilish: Robert Rumsey; Howard: Monae Everett; King Princess:
Sara Tagaloa; Lil Nas X: Christina Guerra.

Manicure: Lil Nas X: Christina Guerra.

Additional design and development by Jacky Myint.

Write a comment

\begin{itemize}
\item
\item
\item
\item
\end{itemize}

Advertisement

\protect\hyperlink{after-bottom}{Continue reading the main story}

\hypertarget{site-index}{%
\subsection{Site Index}\label{site-index}}

\hypertarget{site-information-navigation}{%
\subsection{Site Information
Navigation}\label{site-information-navigation}}

\begin{itemize}
\tightlist
\item
  \href{https://help.nytimes3xbfgragh.onion/hc/en-us/articles/115014792127-Copyright-notice}{©~2020~The
  New York Times Company}
\end{itemize}

\begin{itemize}
\tightlist
\item
  \href{https://www.nytco.com/}{NYTCo}
\item
  \href{https://help.nytimes3xbfgragh.onion/hc/en-us/articles/115015385887-Contact-Us}{Contact
  Us}
\item
  \href{https://www.nytco.com/careers/}{Work with us}
\item
  \href{https://nytmediakit.com/}{Advertise}
\item
  \href{http://www.tbrandstudio.com/}{T Brand Studio}
\item
  \href{https://www.nytimes3xbfgragh.onion/privacy/cookie-policy\#how-do-i-manage-trackers}{Your
  Ad Choices}
\item
  \href{https://www.nytimes3xbfgragh.onion/privacy}{Privacy}
\item
  \href{https://help.nytimes3xbfgragh.onion/hc/en-us/articles/115014893428-Terms-of-service}{Terms
  of Service}
\item
  \href{https://help.nytimes3xbfgragh.onion/hc/en-us/articles/115014893968-Terms-of-sale}{Terms
  of Sale}
\item
  \href{https://spiderbites.nytimes3xbfgragh.onion}{Site Map}
\item
  \href{https://help.nytimes3xbfgragh.onion/hc/en-us}{Help}
\item
  \href{https://www.nytimes3xbfgragh.onion/subscription?campaignId=37WXW}{Subscriptions}
\end{itemize}
