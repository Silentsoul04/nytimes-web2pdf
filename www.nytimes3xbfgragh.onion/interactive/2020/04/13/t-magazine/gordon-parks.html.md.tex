\hypertarget{the-man-who-paved-the-way-for-black-directors-in-hollywood}{%
\section{The Man Who Paved the Way for Black Directors in
Hollywood}\label{the-man-who-paved-the-way-for-black-directors-in-hollywood}}

April 13, 2020

\begin{itemize}
\item
\item
\item
\item
\end{itemize}

Gordon Parks's career made it possible for the next generation to fight
for their rightful place in the mainstream --- only to face the same
opposition he had.

\href{https://www.nytimes3xbfgragh.onion/interactive/2020/04/13/t-magazine/culture-issue-2020.html}{We
Are Family}

\hypertarget{chapter-1-heirs-and-alumni}{%
\subparagraph{Chapter 1: Heirs and
Alumni}\label{chapter-1-heirs-and-alumni}}

\hypertarget{previous}{%
\subparagraph{Previous}\label{previous}}

\hypertarget{next}{%
\subparagraph{Next}\label{next}}

\hypertarget{the-man-who-paved-the-way-for-black-directors-in-hollywood-1}{%
\section{The Man Who Paved the Way for Black Directors in
Hollywood}\label{the-man-who-paved-the-way-for-black-directors-in-hollywood-1}}

\hypertarget{the-directors}{%
\subsection{The Directors}\label{the-directors}}

In 1969, Gordon Parks became the first black director to make a major
Hollywood studio film. His career made it possible for the next
generation to fight their way into the mainstream --- only to face the
same opposition Parks had.

By \href{https://www.nytimes3xbfgragh.onion/by/a-o--scott}{A.O. Scott}

April 13, 2020

SHARE

IN 1968, 20 YEARS after he was hired as Life's first African-American
staff photographer,
\href{https://www.nytimes3xbfgragh.onion/topic/person/gordon-parks}{Gordon
Parks} prepared to demolish another color line. ``You're about to become
Hollywood's first black director,'' he was told by Kenneth Hyman, the
head of production at Warner Bros., during their first meeting. The
studio wanted Parks, by then an accomplished writer, documentarian, poet
and composer as well as a famous photographer, to adapt his 1963 novel,
``\href{https://www.nytimes3xbfgragh.onion/watching/titles/the-learning-tree}{The
Learning Tree},'' for the screen. In addition to directing, he would
write the screenplay and the musical score, and serve as producer.

As Parks recalls in his memoir
``\href{https://www.simonandschuster.com/books/A-Hungry-Heart/Gordon-Parks/9780743269032}{A
Hungry Heart},'' published in 2005, the year before his death, Hyman
told him, ``I can think of only two directors who attempted to do what
you are about to do:
\href{https://www.nytimes3xbfgragh.onion/1985/10/11/arts/orson-welles-is-dead-at-70-innovator-of-film-and-stage.html}{Orson
Welles} and
\href{https://www.nytimes3xbfgragh.onion/topic/person/charles-chaplin}{Charlie
Chaplin}.'' No pressure there. But Parks would set out to make ``The
Learning Tree'' conscious of a burden that Welles and Chaplin, cinematic
pioneers though they were, had never faced. He knew that ``a multitude
of hopeful young black directors would be watching, counting on me to
successfully open those closed doors.''

The T List \textbar{}

Sign up here

And so he did. ``The Learning Tree,'' shot in Fort Scott, Kan., in
wondrous wide-screen, is a coming-of-age story drawn from Parks's own
prairie childhood, at once gently nostalgic and unflinching in its
depiction of 1920s-vintage American racism. (Parks was born in Fort
Scott in 1912.) It opened the doors for subsequent waves of black
directors to break into Hollywood --- including
\href{https://www.imdb.com/name/nm0399737/}{Reginald Hudlin},
\href{https://www.imdb.com/name/nm0870186/}{Robert Townsend},
\href{https://www.nytimes3xbfgragh.onion/2017/09/06/business/media/burnett-sutherland-honorary-oscars.html}{Charles
Burnett}, \href{https://www.imdb.com/name/nm0225416/}{Ernest Dickerson}
and \href{https://www.imdb.com/name/nm0776317/}{Michael Schultz}, all of
whom can be numbered among the students of ``The Learning Tree,'' the
creative children of Gordon Parks. (There are many others, including
\href{https://www.nytimes3xbfgragh.onion/topic/person/spike-lee}{Spike
Lee},
\href{https://www.nytimes3xbfgragh.onion/interactive/2020/04/13/t-magazine/daughters-of-the-dust.html}{Julie
Dash} and Parks's actual son
\href{https://www.imdb.com/name/nm0662899/}{Gordon Parks Jr.}, the
director of the blaxploitation classic
``\href{https://www.nytimes3xbfgragh.onion/2018/06/18/watching/blaxpoitation-films-brown-sugar.html}{Superfly},''
who died in a plane crash in Kenya in 1979.)

The Parks legacy that flows through their work is less a matter of
direct influence than of spirit, ethic and sensibility. ``The Learning
Tree'' is forthrightly political, and also tender, sexy, comical and
full of acutely observed and remembered details. You could say the same
about Schultz's
``\href{https://www.nytimes3xbfgragh.onion/watching/titles/movies/1000120485}{Cooley
High}'' (1975), Burnett's
``\href{https://www.imdb.com/title/tt0076263/}{Killer of Sheep}''
(1978), Dickerson's
``\href{https://www.nytimes3xbfgragh.onion/watching/titles/movies/1000121318}{Juice}''
(1992), Townsend's
``\href{https://www.nytimes3xbfgragh.onion/watching/titles/movies/1000022175}{Hollywood
Shuffle}'' (1987) and Reginald and Warrington Hudlin's
``\href{https://www.nytimes3xbfgragh.onion/watching/titles/house-party}{House
Party}'' (1990) --- all wildly different in method and mood but
enlivened by the confident local knowledge and affectionate humanism
that Parks brought from his writing and photography into American film.

Like other firsts in African-American history, from
\href{https://www.nytimes3xbfgragh.onion/topic/person/jackie-robinson}{Jackie
Robinson} to
\href{https://www.nytimes3xbfgragh.onion/topic/person/barack-obama}{Barack
Obama}, Parks did not come out of nowhere. His breakthrough was, like
theirs, both a herald of rapidly changing times and a sign of how
belated change can feel --- simultaneously a matter of \emph{Already?}
and \emph{At last!} He was hardly the first black American to wield a
movie camera, and ``The Learning Tree'' was not the first Hollywood
production of its time to address the black experience. The late 1960s
saw the rise of independent African-American filmmakers like
\href{https://www.nytimes3xbfgragh.onion/2014/08/27/arts/william-greaves-a-documentarian-and-pioneering-journalist-dies-at-87.html}{William
Greaves} and
\href{https://www.nytimes3xbfgragh.onion/2013/09/20/arts/design/melvin-van-peebles-headlines-a-group-art-show.html}{Melvin
Van Peebles} and the ascendance, in Hollywood, of racially enlightened
dramas like
``\href{https://www.nytimes3xbfgragh.onion/watching/recommendations/watching-film-in-the-heat-of-the-night}{In
the Heat of the Night}'' and
``\href{https://www.nytimes3xbfgragh.onion/watching/titles/guess-whos-coming-to-dinner}{Guess
Who's Coming to Dinner},'' both released in 1967.

``The Learning Tree'' is something else, though: an absolutely personal
film, entwined with its creator's own experiences, that lays
authoritative claim to a place in the American mainstream. At Life (and
before that at the New Deal-era Farm Security Administration), Parks was
known for his intensive, intimate portraits of housing projects,
working-class neighborhoods and poor, rural towns, and there was always
a risk, given the institutional whiteness of the Time Life Corporation,
that those images could be misinterpreted as exotic. But his aesthetic
rigor --- the beauty and integrity of those images --- ensured that
Parks was doing more than explaining black life to white America. He
was, like his exact contemporary
\href{https://www.nytimes3xbfgragh.onion/topic/person/ralph-ellison}{Ralph
Ellison} (who grew up one state south of Parks, in Oklahoma, and who
like Parks eventually went north) committed to the grand midcentury
project of explaining America to itself.

The idealism of that enterprise can look bittersweet in retrospect, and
by the time Parks turned to moviemaking, it had started to come undone.
When it comes to race, Hollywood's doors have a way of closing suddenly,
or leading into half-empty rooms and down long, dim hallways. After
``The Learning Tree,'' Parks directed four more features:
``\href{https://www.nytimes3xbfgragh.onion/watching/titles/movies/1000012133}{Shaft}''
(1971),
``\href{https://www.nytimes3xbfgragh.onion/watching/titles/movies/1000146702}{Shaft's
Big Score!}'' (1972),
``\href{https://www.nytimes3xbfgragh.onion/watching/titles/movies/1000115487}{The
Super Cops}'' (1974) and
``\href{https://www.nytimes3xbfgragh.onion/watching/titles/movies/1000086190}{Leadbelly}''
(1976). (He also embedded with the Black Panthers in Oakland, Calif., on
assignment for Life and helped found Essence magazine.) Though the
critical and commercial fortunes of those movies varied, they represent
the kind of work that might have laid the foundation for a long
Hollywood career. After ``Leadbelly,'' an ambitious musical biopic
released with minimal promotional support during a change of regime at
Paramount (and, in my opinion, one of the great neglected movies of its
decade), Parks never directed another studio film.

THE HISTORY OF African-American cinema, like the larger national history
it refracts, is a complicated chronicle of progress and retrenchment,
pulled backward by long habits of exclusion and condescension, and
pushed forward by the grace and tenacity of artists like Parks and his
followers. None of his heirs has had an easy path through Hollywood.
Some started in relative boom times (the '70s, the early '90s) only to
face industry indifference when fashions changed. Many moved between the
studios and the independent sphere, or between movies and television. To
study their filmographies is to admire their resourcefulness, and also
to contemplate careers marked by frustration: movies that were meagerly
budgeted and poorly marketed, and a great many that simply never got
made. Not to mention the masterpieces that were ignored or undervalued
in their time --- a list that would include, at a minimum, Burnett's
``\href{https://www.nytimes3xbfgragh.onion/watching/titles/movies/1000115143}{Nightjohn}''
(1996, about a slave who learns to read in the pre-Civil War South),
Hudlin's
``\href{https://www.nytimes3xbfgragh.onion/watching/titles/movies/1000114177}{The
Great White Hype}'' (also 1996, about a boxing promoter's scheme to make
his black fighter more popular) and Schultz's magisterial
``\href{https://www.nytimes3xbfgragh.onion/watching/titles/movies/1000082349}{Car
Wash}'' (1976), a day-in-the-life farce set at the Dee-Luxe Car Wash in
downtown Los Angeles.

The struggle and the art go together, which isn't to say that the art is
simply an expression or representation of the struggle, even when, as in
Townsend's ``Hollywood Shuffle,'' the struggle is the subject. That
movie pokes satirical fun at the American movie industry's race problems
--- its appetite for depictions of black servility, criminality and
suffering; its indifference to the tastes of some of its most reliable
consumers; its soft and hard bigotries --- within the context of a story
of lower-middle-class striving. For all its flights of whimsy and
fantasy, it remains grounded in the realities of work, love and family.

Which is much like ``The Learning Tree,'' a portrait of the artist (a
teenager named Newt Winger, played by Kyle Johnson) as a young man
discovering both his own potential and the limits the world places upon
it. That is a durable theme of American literature, and also --- in a
way that is both distinctive and absolutely central --- of
African-American film. It may not be something Gordon Parks invented,
but it is something he bequeathed.

A.O. Scott is a critic at large at The New York Times and the author of
``Better Living Through Criticism.'' Bon Duke is a director who works in
both still and moving images. Top photo, grooming and hair: Monique
Samala. Production: Maritza Carbajal.

\hypertarget{we-are-family-1}{%
\subsubsection{We Are Family}\label{we-are-family-1}}

\hypertarget{chapter-1-heirs-and-alumni-1}{%
\paragraph{Chapter 1: Heirs and
Alumni}\label{chapter-1-heirs-and-alumni-1}}

\href{/interactive/2020/04/13/t-magazine/black-art-galleries.html}{}

\hypertarget{the-artists}{%
\subparagraph{The Artists}\label{the-artists}}

\href{/interactive/2020/04/13/t-magazine/italian-fashion-design-houses.html}{}

\hypertarget{the-dynasties}{%
\subparagraph{The Dynasties}\label{the-dynasties}}

\href{/interactive/2020/04/13/t-magazine/gordon-parks.html}{}

\hypertarget{the-directors-1}{%
\subparagraph{The Directors}\label{the-directors-1}}

\href{/interactive/2020/04/13/t-magazine/enrique-olvera-chef.html}{}

\hypertarget{the-disciples}{%
\subparagraph{The Disciples}\label{the-disciples}}

\href{/interactive/2020/04/13/t-magazine/royal-academy-antwerp.html}{}

\hypertarget{the-graduates}{%
\subparagraph{The Graduates}\label{the-graduates}}

\hypertarget{chapter-2-reunions-and-reconsiderations}{%
\paragraph{Chapter 2: Reunions and
Reconsiderations}\label{chapter-2-reunions-and-reconsiderations}}

\href{/interactive/2020/04/13/t-magazine/ninth-street-greenwich-village-neighbors.html}{}

\hypertarget{the-neighbors}{%
\subparagraph{The Neighbors}\label{the-neighbors}}

\href{/interactive/2020/04/13/t-magazine/omen-restaurant-nyc.html}{}

\hypertarget{the-regulars}{%
\subparagraph{The Regulars}\label{the-regulars}}

\href{/interactive/2020/04/13/t-magazine/hair-musical-broadway.html}{}

\hypertarget{hair-1967}{%
\subparagraph{Hair (1967)}\label{hair-1967}}

\href{/interactive/2020/04/13/t-magazine/sweeney-todd-revival.html}{}

\hypertarget{sweeney-todd-2005-revival}{%
\subparagraph{Sweeney Todd (2005
Revival)}\label{sweeney-todd-2005-revival}}

\href{/interactive/2020/04/13/t-magazine/daughters-of-the-dust.html}{}

\hypertarget{daughters-of-the-dust-1991}{%
\subparagraph{Daughters of the Dust
(1991)}\label{daughters-of-the-dust-1991}}

\hypertarget{chapter-3-legends-pioneers-and-survivors}{%
\paragraph{Chapter 3: Legends Pioneers and
Survivors}\label{chapter-3-legends-pioneers-and-survivors}}

\href{/interactive/2020/04/13/t-magazine/butch-stud-lesbian.html}{}

\hypertarget{the-renegades}{%
\subparagraph{The Renegades}\label{the-renegades}}

\href{/interactive/2020/04/13/t-magazine/act-up-aids.html}{}

\hypertarget{the-activists}{%
\subparagraph{The Activists}\label{the-activists}}

\href{/interactive/2020/04/13/t-magazine/artist-recluse.html}{}

\hypertarget{the-shadows}{%
\subparagraph{The Shadows}\label{the-shadows}}

\href{/interactive/2020/04/13/t-magazine/black-actresses-bassett-berry-blige-henson-whitfield-elise.html}{}

\hypertarget{the-veterans}{%
\subparagraph{The Veterans}\label{the-veterans}}

\hypertarget{chapter-4-the-new-guard}{%
\paragraph{Chapter 4: The New Guard}\label{chapter-4-the-new-guard}}

\href{/interactive/2020/04/13/t-magazine/asian-american-fashion-designers.html}{}

\hypertarget{the-designers}{%
\subparagraph{The Designers}\label{the-designers}}

\href{13tmag-beauties.html}{}

\hypertarget{the-beauties}{%
\subparagraph{The Beauties}\label{the-beauties}}

\href{/interactive/2020/04/13/t-magazine/nyc-downtown-nightlife-party-scene.html}{}

\hypertarget{the-scenemakers}{%
\subparagraph{The Scenemakers}\label{the-scenemakers}}

\href{/interactive/2020/04/13/t-magazine/maria-cornejo-olivier-rousteing-telfar-clemens-alessandro-michele.html\#olivier-rousteing-and-co}{}

\hypertarget{olivier-rousteing-and-co}{%
\subparagraph{Olivier Rousteing and
Co.}\label{olivier-rousteing-and-co}}

\href{/interactive/2020/04/13/t-magazine/maria-cornejo-olivier-rousteing-telfar-clemens-alessandro-michele.html\#maria-cornejo-and-co}{}

\hypertarget{maria-cornejo-and-co}{%
\subparagraph{Maria Cornejo and Co.}\label{maria-cornejo-and-co}}

\href{/interactive/2020/04/13/t-magazine/maria-cornejo-olivier-rousteing-telfar-clemens-alessandro-michele.html\#telfar-clemens-and-co}{}

\hypertarget{telfar-clemens-and-co}{%
\subparagraph{Telfar Clemens and Co.}\label{telfar-clemens-and-co}}

\href{/interactive/2020/04/13/t-magazine/maria-cornejo-olivier-rousteing-telfar-clemens-alessandro-michele.html\#alessandro-michele-and-co}{}

\hypertarget{alessandro-michele-and-co}{%
\subparagraph{Alessandro Michele and
Co.}\label{alessandro-michele-and-co}}

\href{/interactive/2020/04/13/t-magazine/foreign-correspondents.html}{}

\hypertarget{the-journalists}{%
\subparagraph{The Journalists}\label{the-journalists}}

\begin{itemize}
\item
\item
\item
\item
\end{itemize}

Advertisement

\protect\hyperlink{after-bottom}{Continue reading the main story}

\hypertarget{site-index}{%
\subsection{Site Index}\label{site-index}}

\hypertarget{site-information-navigation}{%
\subsection{Site Information
Navigation}\label{site-information-navigation}}

\begin{itemize}
\tightlist
\item
  \href{https://help.nytimes3xbfgragh.onion/hc/en-us/articles/115014792127-Copyright-notice}{©~2020~The
  New York Times Company}
\end{itemize}

\begin{itemize}
\tightlist
\item
  \href{https://www.nytco.com/}{NYTCo}
\item
  \href{https://help.nytimes3xbfgragh.onion/hc/en-us/articles/115015385887-Contact-Us}{Contact
  Us}
\item
  \href{https://www.nytco.com/careers/}{Work with us}
\item
  \href{https://nytmediakit.com/}{Advertise}
\item
  \href{http://www.tbrandstudio.com/}{T Brand Studio}
\item
  \href{https://www.nytimes3xbfgragh.onion/privacy/cookie-policy\#how-do-i-manage-trackers}{Your
  Ad Choices}
\item
  \href{https://www.nytimes3xbfgragh.onion/privacy}{Privacy}
\item
  \href{https://help.nytimes3xbfgragh.onion/hc/en-us/articles/115014893428-Terms-of-service}{Terms
  of Service}
\item
  \href{https://help.nytimes3xbfgragh.onion/hc/en-us/articles/115014893968-Terms-of-sale}{Terms
  of Sale}
\item
  \href{https://spiderbites.nytimes3xbfgragh.onion}{Site Map}
\item
  \href{https://help.nytimes3xbfgragh.onion/hc/en-us}{Help}
\item
  \href{https://www.nytimes3xbfgragh.onion/subscription?campaignId=37WXW}{Subscriptions}
\end{itemize}
