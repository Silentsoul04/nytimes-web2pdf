Sections

SEARCH

\protect\hyperlink{site-content}{Skip to
content}\protect\hyperlink{site-index}{Skip to site index}

\href{https://www.nytimes3xbfgragh.onion/section/us}{U.S.}

\href{https://myaccount.nytimes3xbfgragh.onion/auth/login?response_type=cookie\&client_id=vi}{}

\href{https://www.nytimes3xbfgragh.onion/section/todayspaper}{Today's
Paper}

\href{/section/us}{U.S.}\textbar{}Coronavirus Testing Needs to Triple
Before the U.S. Can Reopen, Experts Say

\url{https://nyti.ms/2VySMpw}

\begin{itemize}
\item
\item
\item
\item
\item
\end{itemize}

\hypertarget{the-coronavirus-outbreak}{%
\subsubsection{\texorpdfstring{\href{https://www.nytimes3xbfgragh.onion/news-event/coronavirus?name=styln-coronavirus-national\&region=TOP_BANNER\&block=storyline_menu_recirc\&action=click\&pgtype=Interactive\&impression_id=eb4cea10-f1c1-11ea-a22b-e581da043942\&variant=undefined}{The
Coronavirus
Outbreak}}{The Coronavirus Outbreak}}\label{the-coronavirus-outbreak}}

\begin{itemize}
\tightlist
\item
  live\href{https://www.nytimes3xbfgragh.onion/2020/09/08/world/covid-19-coronavirus.html?name=styln-coronavirus-national\&region=TOP_BANNER\&block=storyline_menu_recirc\&action=click\&pgtype=Interactive\&impression_id=eb4cea11-f1c1-11ea-a22b-e581da043942\&variant=undefined}{Latest
  Updates}
\item
  \href{https://www.nytimes3xbfgragh.onion/interactive/2020/us/coronavirus-us-cases.html?name=styln-coronavirus-national\&region=TOP_BANNER\&block=storyline_menu_recirc\&action=click\&pgtype=Interactive\&impression_id=eb4cea12-f1c1-11ea-a22b-e581da043942\&variant=undefined}{Maps
  and Cases}
\item
  \href{https://www.nytimes3xbfgragh.onion/interactive/2020/science/coronavirus-vaccine-tracker.html?name=styln-coronavirus-national\&region=TOP_BANNER\&block=storyline_menu_recirc\&action=click\&pgtype=Interactive\&impression_id=eb4cea13-f1c1-11ea-a22b-e581da043942\&variant=undefined}{Vaccine
  Tracker}
\item
  \href{https://www.nytimes3xbfgragh.onion/2020/09/02/your-money/eviction-moratorium-covid.html?name=styln-coronavirus-national\&region=TOP_BANNER\&block=storyline_menu_recirc\&action=click\&pgtype=Interactive\&impression_id=eb4cea14-f1c1-11ea-a22b-e581da043942\&variant=undefined}{Eviction
  Moratorium}
\item
  \href{https://www.nytimes3xbfgragh.onion/interactive/2020/09/02/magazine/food-insecurity-hunger-us.html?name=styln-coronavirus-national\&region=TOP_BANNER\&block=storyline_menu_recirc\&action=click\&pgtype=Interactive\&impression_id=eb4cea15-f1c1-11ea-a22b-e581da043942\&variant=undefined}{American
  Hunger}
\end{itemize}

Advertisement

\protect\hyperlink{after-top}{Continue reading the main story}

\hypertarget{coronavirus-testing-needs-to-triple-before-the-us-can-reopen-experts-say}{%
\section{Coronavirus Testing Needs to Triple Before the U.S. Can Reopen,
Experts
Say}\label{coronavirus-testing-needs-to-triple-before-the-us-can-reopen-experts-say}}

By \href{https://www.nytimes3xbfgragh.onion/by/keith-collins}{Keith
Collins}April 18, 2020

\begin{itemize}
\item
\item
\item
\item
\end{itemize}

As some governors
\href{https://www.nytimes3xbfgragh.onion/2020/04/17/us/coronavirus-cases-news-update.html\#link-35413772}{consider}\href{https://www.nytimes3xbfgragh.onion/2020/04/17/us/coronavirus-cases-news-update.html\#link-35413772}{easing
social
distancing}\href{https://www.nytimes3xbfgragh.onion/2020/04/17/us/coronavirus-cases-news-update.html\#link-35413772}{restrictions},
new estimates by researchers at Harvard University suggest that the
United States cannot safely reopen unless it conducts more than three
times the number of coronavirus tests it is currently administering over
the next month.

\hypertarget{average-daily-coronavirus-tests-in-the-us}{%
\subsubsection{Average Daily Coronavirus Tests in the
U.S.}\label{average-daily-coronavirus-tests-in-the-us}}

Level needed

to safely

reopen

152 tests

150

tests per 100,000 people

(7-day average)

120

90

Current level

45 tests

60

30

March 4

April 15

May 15

150

tests per 100,000 people

(7-day average)

Level to reopen

152 tests

120

90

60

Current level

45 tests

30

March 4

April 15

May 15

An average of 146,000 people per day have been tested for the
coronavirus nationally so far this month, according to the
\href{https://covidtracking.com/}{COVID Tracking Project}, which on
Friday reported 3.6 million total tests across the country. To reopen
the United States by mid-May, the number of tests performed every day
should be 500,000 to 700,000, according to
\href{https://globalepidemics.org/2020/04/18/why-we-need-500000-tests-per-day-to-open-the-economy-and-stay-open/}{the
Harvard estimates}, which is a daily minimum of about 152 tests per
100,000 people.

That level of testing is necessary to identify the majority of people
who are infected and isolate them from people who are healthy, according
to the researchers. About 20 percent of those tested so far were
positive for the virus, a rate that the researchers say is too high.

``If you have a very high positive rate, it means that there are
probably a good number of people out there who have the disease who you
haven't tested,'' said Ashish Jha, the director of the Harvard Global
Health Institute. ``You want to drive the positive rate down, because
the fundamental element of keeping our economy open is making sure
you're identifying as many infected people as possible and isolating
them.''

The researchers said that expanded testing could reduce the rate to 10
percent, which is the maximum rate recommended by the World Health
Organization. In Germany, that number is 7 percent, and in South Korea,
it is closer to 3 percent.

There is variation in the rate of testing and positive results among
states, but most need to administer more tests to get to the level the
researchers suggest --- a minimum of about 152 tests per 100,000 people
each day.

\hypertarget{average-daily-coronavirus-tests}{%
\subsubsection{Average Daily Coronavirus
Tests}\label{average-daily-coronavirus-tests}}

\hypertarget{per-100000-people}{%
\paragraph{Per 100,000 people}\label{per-100000-people}}

Alabama

Ala.

44152

March 4

April 15

May 15

Mar. 4

Apr. 15

May 15

Alaska

Alaska

31

March 4

April 15

May 15

Mar. 4

Apr. 15

May 15

Arizona

Ariz.

21

March 4

April 15

May 15

Mar. 4

Apr. 15

May 15

Arkansas

Ark.

35

March 4

April 15

May 15

Mar. 4

Apr. 15

May 15

California

Calif.

26

March 4

April 15

May 15

Mar. 4

Apr. 15

May 15

Colorado

Colo.

29

March 4

April 15

May 15

Mar. 4

Apr. 15

May 15

Connecticut

Conn.

85

March 4

April 15

May 15

Mar. 4

Apr. 15

May 15

Delaware

Del.

67

March 4

April 15

May 15

Mar. 4

Apr. 15

May 15

District of Columbia

D.C.

66

March 4

April 15

May 15

Mar. 4

Apr. 15

May 15

Florida

Fla.

47

March 4

April 15

May 15

Mar. 4

Apr. 15

May 15

Georgia

Ga.

34

March 4

April 15

May 15

Mar. 4

Apr. 15

May 15

Hawaii

Hawaii

47

March 4

April 15

May 15

Mar. 4

Apr. 15

May 15

Idaho

Idaho

28

March 4

April 15

May 15

Mar. 4

Apr. 15

May 15

Illinois

Ill.

47

March 4

April 15

May 15

Mar. 4

Apr. 15

May 15

Indiana

Ind.

37

March 4

April 15

May 15

Mar. 4

Apr. 15

May 15

Iowa

Iowa

27

March 4

April 15

May 15

Mar. 4

Apr. 15

May 15

Kansas

Kan.

25

March 4

April 15

May 15

Mar. 4

Apr. 15

May 15

Kentucky

Ky.

19

March 4

April 15

May 15

Mar. 4

Apr. 15

May 15

Louisiana

La.

124

March 4

April 15

May 15

Mar. 4

Apr. 15

May 15

Maryland

Md.

41

March 4

April 15

May 15

Mar. 4

Apr. 15

May 15

Massachusetts

Mass.

92

March 4

April 15

May 15

Mar. 4

Apr. 15

May 15

Michigan

Mich.

34

March 4

April 15

May 15

Mar. 4

Apr. 15

May 15

Minnesota

Minn.

24

March 4

April 15

May 15

Mar. 4

Apr. 15

May 15

Mississippi

Miss.

84

March 4

April 15

May 15

Mar. 4

Apr. 15

May 15

Missouri

Mo.

35

March 4

April 15

May 15

Mar. 4

Apr. 15

May 15

Montana

Mont.

29

March 4

April 15

May 15

Mar. 4

Apr. 15

May 15

Nebraska

Neb.

28

March 4

April 15

May 15

Mar. 4

Apr. 15

May 15

Nevada

Nev.

33

March 4

April 15

May 15

Mar. 4

Apr. 15

May 15

New Hampshire

N.H.

32

March 4

April 15

May 15

Mar. 4

Apr. 15

May 15

New Jersey

N.J.

70

March 4

April 15

May 15

Mar. 4

Apr. 15

May 15

New Mexico

N.M.

72

March 4

April 15

May 15

Mar. 4

Apr. 15

May 15

New York

N.Y.

118

March 4

April 15

May 15

Mar. 4

Apr. 15

May 15

North Carolina

N.C.

34

March 4

April 15

May 15

Mar. 4

Apr. 15

May 15

North Dakota

N.D.

52

March 4

April 15

May 15

Mar. 4

Apr. 15

May 15

Ohio

Ohio

22

March 4

April 15

May 15

Mar. 4

Apr. 15

May 15

Oklahoma

Okla.

57

March 4

April 15

May 15

Mar. 4

Apr. 15

May 15

Oregon

Ore.

30

March 4

April 15

May 15

Mar. 4

Apr. 15

May 15

Pennsylvania

Pa.

44

March 4

April 15

May 15

Mar. 4

Apr. 15

May 15

Puerto Rico

P.R.

13

March 4

April 15

May 15

Mar. 4

Apr. 15

May 15

Rhode Island

R.I.

185

March 4

April 15

May 15

Mar. 4

Apr. 15

May 15

South Carolina

S.C.

28

March 4

April 15

May 15

Mar. 4

Apr. 15

May 15

South Dakota

S.D.

50

March 4

April 15

May 15

Mar. 4

Apr. 15

May 15

Tennessee

Tenn.

51

March 4

April 15

May 15

Mar. 4

Apr. 15

May 15

Texas

Texas

27

March 4

April 15

May 15

Mar. 4

Apr. 15

May 15

Utah

Utah

51

March 4

April 15

May 15

Mar. 4

Apr. 15

May 15

Vermont

Vt.

76

March 4

April 15

May 15

Mar. 4

Apr. 15

May 15

Virginia

Va.

23

March 4

April 15

May 15

Mar. 4

Apr. 15

May 15

Washington

Wash.

34

March 4

April 15

May 15

Mar. 4

Apr. 15

May 15

West Virginia

W.Va.

41

March 4

April 15

May 15

Mar. 4

Apr. 15

May 15

Wisconsin

Wis.

25

March 4

April 15

May 15

Mar. 4

Apr. 15

May 15

Wyoming

Wyo.

56

March 4

April 15

May 15

Mar. 4

Apr. 15

May 15

Notes: Testing levels shown are a 7-day moving average. Reporting
methods and regularity can vary. Some fluctuations in the data can be
because of a gap in reporting or a change in the reporting method. Maine
was excluded because consistent and reliable data was not available.

In most states, people who had severe symptoms, worked in health
facilities or were otherwise hospitalized were given priority for
testing. The goal of the testing level recommended by the researchers
would be to test nearly everyone who has mild or severe flulike
symptoms, and an average of 10 contacts for each person who tests
positive for the virus.

A
\href{https://www.nytimes3xbfgragh.onion/2020/03/28/us/testing-coronavirus-pandemic.html}{shortage
of test kits and technical flaws} in the United States significantly
delayed more widespread testing of the virus, letting it spread
undetected for weeks. With
\href{https://www.nytimes3xbfgragh.onion/interactive/2020/us/coronavirus-us-cases.html}{more
than 695,000 cases as of Friday}, the country has the highest number of
known cases in the world.

Health experts said that if the U.S. had tested earlier and more, the
outbreak would have been better contained. Caitlin Rivers, an assistant
professor at the Johns Hopkins Center for Health Security, said that
since that did not happen and the virus spread rapidly, it has made
sense to prioritize testing for those in hospitals in recent weeks.

``It does make sense, given constraints, to focus on people with severe
illness, because you need that information to guide their care,'' Dr.
Rivers said.

In mid-May, however, when the researchers estimate infections will have
subsided somewhat and states are looking to reopen their economies, Dr.
Jha believes vastly expanded testing will be crucial.

``I want to be able to identify everybody who is even mildly
symptomatic,'' he said. ``So when I wake up one morning and have a sore
throat and a fever, I should be able to go get tested. And then I want
to be able to test all of my contacts if I turn out to be positive, so
that I can do the test, trace and isolate strategy that's so critical to
allowing us to open up and stay open.''

New Jersey currently has the highest positive rate among all states ---
about half of the 157,000 coronavirus tests that it has conducted so far
have come back positive. That suggests that many of the people in the
state who have the coronavirus have not been tested. And recently,
testing in New Jersey has started to decline.

\hypertarget{average-daily-tests-in-new-jersey}{%
\subsubsection{Average Daily Tests in New
Jersey}\label{average-daily-tests-in-new-jersey}}

Level needed

to safely

reopen

152 tests

200

tests per 100,000 people

(7-day average)

Estimated

actual cases

150

100

Current level

70

50

Identified

cases

0

March 4

May 15

April 15

200

tests per 100,000

people (7-day

average)

Estimated

actual cases

Level to reopen

152 tests

150

100

Current level

70

50

Identified

cases

0

March 4

May 15

April 15

Note: Estimated cases are five times the number of cases identified. The
researchers estimate that at least 80 percent of people who have the
virus have not been tested.

The state's testing has been
\href{https://www.nytimes3xbfgragh.onion/2020/04/13/nyregion/coronavirus-testing.html}{hindered
by setbacks}, including long lines at testing facilities and a shortage
of health care workers and testing swabs.

Christopher Neuwirth, an assistant commissioner at New Jersey's Health
Department, said the state's recent lag in testing could also indicate
that fewer people are trying to get tested.

``In the days and weeks up until testing was available, there was a huge
pent-up demand,'' Mr. Neuwirth said. After that, he said, the public
became more aware of the criteria for testing. ``People really paid
attention and really heard the messaging that you have to be
symptomatic.''

That messaging will have to be reversed, Dr. Jha said, once testing is
expanded.

``We need to switch from saying to people, `if you have mild symptoms,
if you're not feeling super sick, don't come and stay at home,' to `if
you have any symptoms, you need to come in to get tested right away,' ''
he said.

Sources: \href{https://covidtracking.com/}{The COVID Tracking Project};
Ashish Jha, director of the Harvard Global Health Institute and a
professor of health policy at the Harvard T.H. Chan School of Public
Health; the World Health Organization; Washington State Department of
Health

\begin{itemize}
\item
\item
\item
\item
\end{itemize}

Advertisement

\protect\hyperlink{after-bottom}{Continue reading the main story}

\hypertarget{site-index}{%
\subsection{Site Index}\label{site-index}}

\hypertarget{site-information-navigation}{%
\subsection{Site Information
Navigation}\label{site-information-navigation}}

\begin{itemize}
\tightlist
\item
  \href{https://help.nytimes3xbfgragh.onion/hc/en-us/articles/115014792127-Copyright-notice}{©~2020~The
  New York Times Company}
\end{itemize}

\begin{itemize}
\tightlist
\item
  \href{https://www.nytco.com/}{NYTCo}
\item
  \href{https://help.nytimes3xbfgragh.onion/hc/en-us/articles/115015385887-Contact-Us}{Contact
  Us}
\item
  \href{https://www.nytco.com/careers/}{Work with us}
\item
  \href{https://nytmediakit.com/}{Advertise}
\item
  \href{http://www.tbrandstudio.com/}{T Brand Studio}
\item
  \href{https://www.nytimes3xbfgragh.onion/privacy/cookie-policy\#how-do-i-manage-trackers}{Your
  Ad Choices}
\item
  \href{https://www.nytimes3xbfgragh.onion/privacy}{Privacy}
\item
  \href{https://help.nytimes3xbfgragh.onion/hc/en-us/articles/115014893428-Terms-of-service}{Terms
  of Service}
\item
  \href{https://help.nytimes3xbfgragh.onion/hc/en-us/articles/115014893968-Terms-of-sale}{Terms
  of Sale}
\item
  \href{https://spiderbites.nytimes3xbfgragh.onion}{Site Map}
\item
  \href{https://help.nytimes3xbfgragh.onion/hc/en-us}{Help}
\item
  \href{https://www.nytimes3xbfgragh.onion/subscription?campaignId=37WXW}{Subscriptions}
\end{itemize}
