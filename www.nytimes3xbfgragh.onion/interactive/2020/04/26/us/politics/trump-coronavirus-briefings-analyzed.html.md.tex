Sections

SEARCH

\protect\hyperlink{site-content}{Skip to
content}\protect\hyperlink{site-index}{Skip to site index}

\hypertarget{comments}{%
\subsection{\texorpdfstring{\protect\hyperlink{commentsContainer}{Comments}}{Comments}}\label{comments}}

\href{}{260,000 Words, Full of Self-Praise, From Trump on the
Virus}\href{}{Skip to Comments}

The comments section is closed. To submit a letter to the editor for
publication, write to
\href{mailto:letters@NYTimes.com}{\nolinkurl{letters@NYTimes.com}}.

\hypertarget{260000-words-full-of-self-praise-from-trump-on-the-virus}{%
\section{260,000 Words, Full of Self-Praise, From Trump on the
Virus}\label{260000-words-full-of-self-praise-from-trump-on-the-virus}}

By \href{https://www.nytimes3xbfgragh.onion/by/jeremy-w-peters}{Jeremy
W. Peters},
\href{https://www.nytimes3xbfgragh.onion/by/elaina-plott}{Elaina Plott}
and \href{https://www.nytimes3xbfgragh.onion/by/maggie-haberman}{Maggie
Haberman}April 26, 2020

\begin{itemize}
\item
\item
\item
\item
\item
  \emph{1437}
\end{itemize}

March 9

March 10

March 10

March 11

March 11

March 13

March 14

March 15

March 16

March 17

March 17

March 18

March 18

March 19

March 19

March 20

March 21

March 22

March 23

March 24

March 24

March 25

March 26

March 27

March 27

March 29

March 30

March 31

April 1

April 2

April 3

April 4

April 5

April 6

April 8

April 9

April 10

April 13

April 14

April 15

April 16

April 17

Three journalists from The New York Times reviewed more than 260,000
words spoken by President Trump during the pandemic. Here's what we
learned.

 By \href{https://www.nytimes3xbfgragh.onion/by/jeremy-w-peters}{Jeremy
W}\href{https://www.nytimes3xbfgragh.onion/by/jeremy-w-peters}{.}\href{https://www.nytimes3xbfgragh.onion/by/jeremy-w-peters}{Peters},
\href{https://www.nytimes3xbfgragh.onion/by/elaina-plott}{Elaina Plott}
and \href{https://www.nytimes3xbfgragh.onion/by/maggie-haberman}{Maggie
Haberman} April 26, 2020

At his White House news briefing on the coronavirus on March 19,
President Trump offered high praise for the commissioner of the Food and
Drug Administration, Stephen Hahn. ``He's worked, like, probably as hard
or harder than anybody,'' Mr. Trump said. Then he corrected himself:
``Other than maybe Mike Pence --- or me.''

On March 27, Mr. Trump boasted about marshaling federal resources to
fight the virus, ignoring his
\href{https://www.nytimes3xbfgragh.onion/2020/04/11/us/politics/coronavirus-trump-response.html}{early
failures} and smearing previous administrations. ``Nobody has done
anything like we've been able to do,'' he claimed. ``And everything I
took over was a mess. It was a broken country in so many ways. In so
many ways.''

And on April 13, Mr. Trump insisted that governors were so satisfied
with his performance they hadn't asked for anything on a recent
conference call. ``There wasn't even a statement of like, `We think you
should do this or that,''' he said. ``I heard it was, like, just a
perfect phone call.''

The self-regard, the credit-taking, the audacious rewriting of recent
history to cast himself as the hero of the pandemic rather than the
president who was slow to respond: Such have been the defining features
of Mr. Trump's use of the bully pulpit during the coronavirus outbreak.

The New York Times analyzed every word Mr. Trump spoke at his
\href{https://www.whitehouse.gov/remarks/}{White House briefings} and
other presidential remarks on the virus --- more than 260,000 words ---
from March 9, when the outbreak began leading to widespread disruptions
in daily life, through mid-April. The transcripts show striking patterns
and repetitions in the messages he has conveyed, revealing a display of
presidential hubris and self-pity unlike anything historians say they
have seen before.

By far the most recurring utterances from Mr. Trump in the briefings are
self-congratulations, roughly 600 of them, which are often predicated on
exaggerations and falsehoods. He does credit others (more than 360
times) for their work, but he also blames others (more than 110 times)
for inadequacies in the state and federal response.

Mr. Trump's attempts to display empathy or appeal to national unity
(about 160 instances) amount to only a quarter of the number of times he
complimented himself or a top member of his team.

Here is what a week of the analysis looks like:

\hypertarget{excerpts-from-a-week-of-briefings}{%
\subsubsection{Excerpts From a Week of
Briefings}\label{excerpts-from-a-week-of-briefings}}

``But the authority of the President of the United States, having to do
with the subject we're talking about, is total.''

Exaggerations and falsehoods

``This is a horrible thing that happened to our country. This is a
horrible thing that happened to 184 countries all over the world. This
is a horrible thing, and there was no reason for it. It should never,
ever happen again.''

``But, you know, sort of --- I've been brutalized for the last four
years. I used to do well before I decided to run for politics. But I
guess I'm doing okay because, to the best of my knowledge, I'm the
president of the United States, despite the things that are said.''

Displays empathy or appeals to unity

``I watched the governor of Arkansas, Asa. You saw that. He --- I
thought he was terrific. I watched the governor of Oklahoma over the
weekend being interviewed. He was terrific.''

Self-congratulations

Credits others

``You know, they're supposed to buy their own stockpile. They have state
stockpiles. They're supposed to be using that. And unfortunately, most
of the states weren't there. And a lot of people didn't want to talk
about it, but they weren't there.''

Blaming others

``We inherited a stockpile where the cupboards were bare.''

Blaming others

``And we --- we had a broken system and now we have a great system. We
have a system where other countries are coming to us and saying, `We
want to get some of those tests.'''

Exaggerations and falsehoods

``But, you know, sort of --- I've been brutalized for the last four
years. I used to do well before I decided to run for politics. But I
guess I'm doing okay because, to the best of my knowledge, I'm the
president of the United States, despite the things that are said.''

Self-congratulations

``Our country has suffered. The world has suffered. One hundred and
eighty-four other countries have suffered.''

Displays empathy or appeals to unity

``I watched the governor of Arkansas, Asa. You saw that. He --- I
thought he was terrific. I watched the governor of Oklahoma over the
weekend being interviewed. He was terrific.''

Credits others

The Times analyzed 42 press briefing transcripts and other remarks by
Mr. Trump on the virus from March 9 to April 17, using transcripts from
whitehouse.gov. Links are provided to the transcripts in the sections of
quotes below. The visualization above shows only portions of briefings
when Mr. Trump is speaking.

Mr. Trump has mentioned his immediate predecessor, President Barack
Obama, roughly 10 times, sometimes in response to a question. And he has
referred to previous administrations about 30 times, often accusing them
of leaving him with faulty conditions. Mr. Trump has mentioned
governors, individually or as a group, about 400 times, alternating
between compliments and criticism.

While other presidents treated moments of crisis as an opportunity to
bring the nation together, Mr. Trump, bereft of his signature campaign
rallies, has used the evening television appearances as a branding
exercise to promote himself. The briefings became so problematic ---
especially after Mr. Trump's
\href{https://www.nytimes3xbfgragh.onion/2020/04/24/us/politics/trump-inject-disinfectant-bleach-coronavirus.html}{dangerous
suggestion} last week that injecting disinfectant could help people who
are sick with the virus --- that the White House is now considering
\href{https://www.nytimes3xbfgragh.onion/2020/04/26/us/politics/trump-disinfectant.html}{limiting
them}.

He has regularly used hyperbole to try to cast his leadership as
historic in scope, even placing himself in the pantheon of presidents
like Lincoln and Roosevelt who led the nation through some of its
darkest moments. ``We have done a job, the likes of which nobody has
ever done,'' he declared at his April 13 briefing.

Less frequently, he has mentioned the hard work and dedication of
ordinary Americans like nurses and truck drivers. He can sometimes be
generous when he credits the work of state and city leaders, including
Democrats, though he often does so while mentioning that they have been
appreciative of him. (Gov. Gavin Newsom of California was ``gracious'';
Mayor Bill de Blasio of New York City, ``very nice.'') He has repeatedly
singled out ``great companies'' and the ``greatest business executives
in the world,'' but individual workers less so.

And he has mentioned the coronavirus's staggering toll --- nearly 50,000
Americans dead
\href{https://www.nytimes3xbfgragh.onion/interactive/2020/us/coronavirus-us-cases.html}{as
of April 26}, and hundreds of thousands of others sick --- only
fleetingly.

There has always been someone or something else to deflect blame for the
various breakdowns in the government's response. But Mr. Trump's targets
have shifted over the last several weeks, showing a clear but disorderly
progression of his message as he struggled to focus attention

\begin{quote}
elsewhere.
\end{quote}

\hypertarget{blames-others}{%
\subsection{Blames others}\label{blames-others}}

\begin{quote}
\hypertarget{when-i-took-this-over-it-was-an-empty-box-we-didnt-have-testing-we-didnt-have-anything-we-had-a-broken-system-there-we-had-a-broken-system-with-stockpiling-we-had-a-lot-of-broken-systems-and-im-not-just-blaming-president-obama-you-go-long-before-that--march-27}{%
\subsubsection{\texorpdfstring{``When I took this over, it was an empty
box. We didn't have testing. We didn't have anything. We had a broken
system there. We had a broken system with stockpiling. We had a lot of
broken systems. And I'm not just blaming President Obama. You go long
before that.''
\href{https://www.whitehouse.gov/briefings-statements/remarks-president-trump-vice-president-pence-members-coronavirus-task-force-press-briefing-13/}{»}
\textbf{March
27}}{``When I took this over, it was an empty box. We didn't have testing. We didn't have anything. We had a broken system there. We had a broken system with stockpiling. We had a lot of broken systems. And I'm not just blaming President Obama. You go long before that.'' » March 27}}\label{when-i-took-this-over-it-was-an-empty-box-we-didnt-have-testing-we-didnt-have-anything-we-had-a-broken-system-there-we-had-a-broken-system-with-stockpiling-we-had-a-lot-of-broken-systems-and-im-not-just-blaming-president-obama-you-go-long-before-that--march-27}}
\end{quote}

\begin{quote}
\hypertarget{nobody-has-done-anything-like-weve-been-able-to-do-and-everything-i-took-over-was-a-mess-it-was-a-broken-country-in-so-many-ways-in-so-many-ways-other-than-this-we-had-a-bad-testing-system-we-had-a-bad-stockpile-system-we-had-nothing-in-the-stockpile-system--march-27}{%
\subsubsection{\texorpdfstring{``Nobody has done anything like we've
been able to do. And everything I took over was a mess. It was a broken
country in so many ways. In so many ways other than this. We had a bad
testing system. We had a bad stockpile system. We had nothing in the
stockpile system.''
\href{https://www.whitehouse.gov/briefings-statements/remarks-president-trump-vice-president-pence-members-coronavirus-task-force-press-briefing-13/}{»}
\textbf{March
27}}{``Nobody has done anything like we've been able to do. And everything I took over was a mess. It was a broken country in so many ways. In so many ways other than this. We had a bad testing system. We had a bad stockpile system. We had nothing in the stockpile system.'' » March 27}}\label{nobody-has-done-anything-like-weve-been-able-to-do-and-everything-i-took-over-was-a-mess-it-was-a-broken-country-in-so-many-ways-in-so-many-ways-other-than-this-we-had-a-bad-testing-system-we-had-a-bad-stockpile-system-we-had-nothing-in-the-stockpile-system--march-27}}
\end{quote}

\begin{quote}
\hypertarget{ill-let-you-know-someday--lets-see-what-happens--but-i-may-let-you-know-whos-not-doing-their-job-i-can-tell-you-the-ones-that-are-good-both-republican-and-democrat-and-the-ones-who-dont-know-what-theyre-doing-but-we-help-some-of-the-ones-that-dont-know-what-theyre-doing--april-13}{%
\subsubsection{\texorpdfstring{``I'll let you know someday --- let's see
what happens --- but I may let you know who's not doing their job. I can
tell you the ones that are good, both Republican and Democrat, and the
ones who don't know what they're doing. But we help some of the ones
that don't know what they're doing.''
\href{https://www.whitehouse.gov/briefings-statements/remarks-president-trump-vice-president-pence-members-coronavirus-task-force-press-briefing-25/}{»}
\textbf{April
13}}{``I'll let you know someday --- let's see what happens --- but I may let you know who's not doing their job. I can tell you the ones that are good, both Republican and Democrat, and the ones who don't know what they're doing. But we help some of the ones that don't know what they're doing.'' » April 13}}\label{ill-let-you-know-someday--lets-see-what-happens--but-i-may-let-you-know-whos-not-doing-their-job-i-can-tell-you-the-ones-that-are-good-both-republican-and-democrat-and-the-ones-who-dont-know-what-theyre-doing-but-we-help-some-of-the-ones-that-dont-know-what-theyre-doing--april-13}}
\end{quote}

First it was the virus itself, which Mr. Trump described on March 16 as
an ``invisible enemy'' that ``came out from nowhere.''

Two days later, he said at the start of the briefing that the country
was at ``war against the Chinese virus.'' But after about a week, Mr.
Trump dropped that phrase and refocused his blame on other targets,
while soft-pedaling his criticism of China. The world might have been
better prepared, he said on April 17, ``if a certain country did what
they should have done.''

He has also attacked the World Health Organization --- a ``very
China-centric'' entity, he said, that ``minimized the threat very
strongly.''

The April 13 briefing shows the full range of Mr. Trump's statements:

\hypertarget{remarks-by-president-trump-during-coronavirus-task-force-press-briefing}{%
\subsection{Remarks by President Trump During Coronavirus Task Force
Press
Briefing}\label{remarks-by-president-trump-during-coronavirus-task-force-press-briefing}}

April 13, 2020

James S. Brady Press Briefing Room

THE PRESIDENT: Thank you very much, everyone. My administration will do
everything possible to help those communities get back on their feet.
We're speaking with the governors and representatives. FEMA is already
on its way, and they got there --- as soon as we heard the word, I said,
``Get out there.'' So FEMA is there, and you know the great job that
FEMA does. It's, really, something very special.

So we just want to say: Warmest condolences, and we're with you all the
way. It's a tough deal. That was a bad, bed level five. That was a bad
group. That's as high as it gets. It was a bad grouping of tornadoes.
Something that's something incredible, the power --- the horrible,
destructive power.

America is continuing to make critical progress in our war against the
virus. Over the weekend, the number of daily new infections remained
flat nationwide. Flat. Hospitalizations are slowing in hotspots like New
York, New Jersey, Michigan, and Louisiana. This is clear evidence that
our aggressive strategy to combat the virus is working and that
Americans are following the guidelines. It's been incredible what
they've done.

You looked at the charts, and the charts are --- and the models from
early on --- predictions were 100 {[}thousand{]} and 120,000 people look
like. If they did well, they were going to unfortunately perish. And
we're going to be, hopefully, way, way below that number. So that will
be a sign of people doing things right, but it's still just a horrible
thing all over the world. A hundred and eighty-four countries.

This is all a tribute to our wonderful healthcare advisors and experts
who have been with us right from the beginning. We appreciate it so
much. In fact, Dr. Fauci is here. Maybe I could ask Tony to say a few
words before we go any further.

Thank you very much. Tony, please.

-\/-

So, Mr. President, the question is ---

THE PRESIDENT: And, by the way, the travel ban, that was earlier. The
travel ban was done earlier.

You couldn't give us a date when he talked to you.

THE PRESIDENT: And if you look at statistics --- I happened to write a
couple of them down. If you look at statistics --- so, on January 6th
--- that's long before the dates you're talking about --- there were ---
CDC issued a travel notice for Wuhan, China, a notice, before there was
even a confirmed case of the virus in the United States. That's on
January 6th. This is all documented. Because we have so much fake news,
I like to document things. January 6th, long before the dates we're
talking about, CDC issued travel notice to Wuhan --- for Wuhan.

On January 11th, we have zero cases in the United States. Zero. We don't
have any cases. So there are no cases reported that we know of. This is
January 11th. The CDC issued a level one travel notice health --- for
health, while there were still no confirmed cases. So we had zero cases.
People want me to act. I'm supposed to close down the economy --- the
greatest economy in the history of the world --- and we don't have one
case confirmed in the United States. That's January 11th.

On January 17th, the CDC began implementing public health entry
screenings at three major U.S. airports that received the greatest
volume of passengers from Wuhan, at my instructions. There was not a
single case of the coronavirus in the United States. So on January 17th,
there wasn't a case, and the fake news is saying, ``Oh, he didn't act
fast enough.'' Well, you remember what happened. Because when I did act,
I was criticized by Nancy Pelosi, by Sleepy Joe Biden. I was criticized
by everybody. In fact, I was called xenophobic. I was asking Biden to
please define that for me. I was called other things by Democrats and
some others --- not too many others, actually. So that --- by the media,
definitely.

Now, on January 21st --- this is long before the time we're talking,
because when Tony is talking, I believe he's talking about the end of
February. On January 21st, okay --- still early --- there was one case
of the virus. At that time, we called it the ``Wuhan virus,'' right?
Wuhan. There was one case in the whole United States. We have one case.
This is all documented. It all comes from you. A lot of it comes from
you people.

On January 21st, the CDC activated an emergency operation center. There
was just one case, one person. That's why that ad was such a phony.
There was one person in the United States. You know they used the ad,
``There's only one person\ldots{}'' That statement was made at that
time. One case. In the whole United States, one case. I'm supposed to
shut down the government? The biggest --- the biggest economy in the
history of the world. ``Shut it down. We have one case.''

Seven cases were on January 31st. Now, on January 21st, there was a
case. Not one person had died. You heard that, Steve, right? Not one
person.

So we have this massive country, the United States of America. We have
the greatest economy in the world --- bigger than China's, by a lot ---
right? --- because of what we've done over the last three and a half
years, prior to the virus, but including the virus. So we have the
biggest economy, the greatest economy we've ever had; the highest
employment numbers; the best employment numbers; best unemployment
numbers also. The best of everything.

So, on January 31st --- think of it --- not one person has died. Not
one. Nobody died. Not one, Jon. I don't think you'll find any. This is
reported by CDC, confirmed by the news --- which doesn't mean anything
to me because they don't tell the truth. But CDC reported: January 31st,
not one person has died.

And I issued a travel restriction from China. Think of it. So nobody
died, and I issued. You can't get earlier than that.

So we have nobody died, and I said, ``China, you can't come in. I'm
sorry.'' Because I saw what was going on. It wasn't so much what I was
told; it was that I saw what was going on and I didn't like it. But I
didn't speak to Tony about it. Didn't speak to very many people about
it. I didn't like it.

So what did I do? Ready? January 31st: In the United States, not one
person had died because of the --- again, the Wuhan virus. So I issued
travel restrictions on that date, even though nobody died, and I got
brutalized over it by the press because I was way too early; I shouldn't
have done it. Brutalized by the press. But, you know, sort of --- I've
been brutalized for the last four years. I used to do well before I
decided to run for politics. But I guess I'm doing okay because, to the
best of my knowledge, I'm the President of the United States, despite
the things that are said.

So, then, first mandatory quarantine in more than 50 years, we did.
First mandatory in 50 years. The same restrictions that the Democrats
and the media called ``xenophobic.'' Now, Joe Biden said, ``He's a
racist.'' He called me a racist because I said, ``We're shutting down
entry from China. We're shutting it down.'' He called me xenophobic, and
he called me racist and other things.

Since then, on a Friday night, two weeks ago, Joe Biden issued a state-
--- it wasn't him; he didn't write it. I'm sure he doesn't even know
that it was issued. But the people from his campaign --- who are smart.
The people that write those little PR releases are pretty smart,
reasonably good. Not the best, but they're not bad. But they issued a
statement, saying that Joe Biden agrees that the Pres- --- that
President Trump was right to close it down to China. Now, he did that.

Now, he issued it on Friday night. We've all heard about that, Jon ---
Friday nights, right? In fact, his was later Friday night than I ever
released mine on Friday night. Okay?

(Inaudible) seems pretty late.

THE PRESIDENT: So he did --- he did it pretty late. I mean, you know,
like at 11 o'clock in the evening or something. You know, that's pretty
late. Anyway. So Joe Biden issued --- and it's one of those things.

But, in February, Nancy Pelosi said we should come to Chinatown. This is
late February. ``Come to Chinatown. We think it's very safe. Come here.
Let's all have the big parade --- Chinatown parade.'' Probably referring
to San Francisco. And that's it.

But I took this action early. And so the story in the New York Times was
a total fake. It's a fake newspaper and they write fake stories. And
someday --- hopefully in five years, when I'm not here --- those papers
are all going out of business because nobody is going to want to read
them. But now they like them because they write about me.

Now, with that, I have a couple of interesting --- we have a few clips
that we're just going to put up. We could turn the lights a little bit
lower. I think you'll find them interesting. And then we'll answer some
questions. I'll ask you some questions because you're so guilty, but
forget it. But most importantly, we're going to get back onto the reason
we're here, which is the success we're having. Okay?

Please, you could put it on. Thank you.

(A video is played.)

So we could give you hundreds of clips like that from governors ---
including Democratic or ``Democrat,'' as I call them, governors, which
is actually the correct term. We could give you hundreds of clips just
like that. We have them. We didn't want this to go on too long, but I
just want to say it's --- you know, it's very sad when people write
false stories like, in that case, I guess it was gotten mostly from the
New York Times, which is a highly --- I mean, if you had libel laws,
they would have been out of business even before they'll end up going
out of business. So it's too bad.

But we could have given you --- you saw the statements. We have hundreds
of statements. Hundreds of statements, including from Democrats and
Democrat governors.

And if you look, they were all saying, ``We need ventilators. We
need\ldots{}'' You don't hear ``ventilators'' anymore. They have all the
ventilators they need, which we were right about. We said, ``You're
asking for too many. You don't need that.'' And, in all fairness, these
two people right here --- Dr. Birx, Dr. Fauci --- they said, ``I don't
think they need that many ventilators.'' And I said, ``I agree.''

At one point --- and I'm not knocking New York for this, but they were
asking --- you remember? --- 40,000 ventilators. And that's more than
they have all over the country. And we got them a lot of ventilators,
and nobody has complained.

We got them, as you know, beautiful --- we built hospital rooms all over
the country. The governor of Louisiana, John Bel Edwards, was very nice.
He said, ``You know what? You don't have to build a second hospital.''
Because good news is happening. They're not able to fill the beds. They
needed two hospitals. We built one; it was perfect. We're getting ---
we're just starting the other. I called him up. I said, ``Do you think
we should build the second one? I don't think you're going to need it.''
He said, ``Let me get back.'' He got back. We didn't need it.

With Governor Cuomo, in all good spirit and faith, he wanted to have the
Jacob Javits Center done. And we built 2,900 incredible beds.
Incredible. Then we make it --- we made it COVID --- and --- or, to be
exactly accurate, COVID-19, and --- which was a lot of work. We had to
change the ductwork. We had to seal up certain areas. We had to put
areas of rooftop things over the beds. We did a lot of work. And we had
it --- but they never really had too much use for it.

And they called also --- Mayor de Blasio, rightfully, he called. He
said, ``Would it be possible to get more medical help?'' So, now, not
only are we building facilities, we're --- they're asking us for help
because they're unable to man it. And we got him the help. We got Mayor
de Blasio a lot of help.

Then, when the Javits Center wasn't used much --- and then, as you know,
the Mercy --- and we took the Mercy and we took the Comfort, and we made
them both --- Los Angeles and New York --- we made them COVID-adaptable,
which was not easy to do. And we didn't get almost any people sent
there. They didn't need them at the beginning because they didn't need
it for anything but this because there were fewer accidents, fewer
motorcycles, fewer everything.

And what we did was like an incredible job, but they didn't need them.
It turned out --- they were there. We were ready. You know the
expression? They have an expression: ``Ready, willing, and able.'' We
were ready, willing, and able. What the Army Corps of Engineers did was
a miracle. What --- what FEMA did was a miracle. What the doctors did.

So I got a call two days ago from the mayor of New York. He said,
``Could you help us even more with medical personnel?'' And we sent ---
I think it was 448 doctors, nurses, and respiratory experts. Real
experts. And I got a call from the mayor and he said, ``I want to tell
you: Incredible --- these people are incredible.'' He said, ``They
lifted the spirits of the hospital workers from New York City like
nothing I've ever seen.''

He's --- he was unbelievable, what he said. It was really appreciated.
And I let them know that. I let the military people --- he said, ``They
went in there so brave, so incredible. They lifted the spirits of
everybody.'' We did all of this work, but when you read the phony
stories, you --- nobody --- nobody acknowledges this.

And it doesn't have to be acknowledged, from my standpoint, but it does
have to be acknowledged from the great work that these doctors, nurses,
the Army Corps of Engineers, FEMA --- all these people, they've done
this incredible job. And they shouldn't be abused because --- you take a
look at what's happened. Nobody is asking for ventilators, except
outside of our country.

Outside of our country, they're calling me --- every country. They're
calling me. So many countries. And I'm going to try and help them
because we have thousands of ventilators being built. But nobody is
asking for ventilators. Nobody is asking for beds because we built
hospitals. I think we built 20,000 beds in a period of a couple of
weeks.

The job they've done is incredible. With all of that being said, I'm
getting along very well with governors. And if I wouldn't, Mike Pence
had a call today with the governors and it was like a 10. I said, ``How
was it?'' He said, ``It's a 10.'' He was one of my expressions,
actually. But he said it was like a 10.

And I'm sure you people were probably on the call, although you weren't
supposed to be. But you were sitting in somebody's office listening to
it, because every time we have these --- and, you know, and you would
know that for weeks those calls have been very good. But there wasn't a
raised voice. There wasn't even a statement of like, ``We think you
should do this or that.'' I heard it was, like, just a perfect phone
call. It might not be reported that way. They'll say, ``I thought that
somebody maybe slightly raised\ldots{}'' Didn't even raise a voice.

My only point of saying this, because I want to get back to why we're
here: The press has not treated these incredible people who've done such
a great job, they haven't treated them fairly. They're way off. We were
way ahead of schedule.

And remember this --- because the Times story was a fake --- but
everything --- remember this: Everything we did, I was criticized
because I was too early. If I waited longer, it would have been --- you
would have been critic- --- if I went way early, if I went three months
earlier, I would have been criticized --- you know, criticized for being
way too early.

So, with all of that being said, we understand it. I think I've educated
a lot of people as to the press. And I would love to be able to say that
we have a very honest press. Honestly, Jon, there'd be nothing I would
be more proud of if the press would work --- and I don't mind being
criticized, but not when they're wrong. Not when people have done a
great job.

Yes.

Can I just ask you about the video? Because I've never seen a video like
that played in this room. It looks a bit like a campaign ad. Who --- who
produced that video for you?

THE PRESIDENT: That was done by a group in the office, and it was done
just by --- we just put some clips together. I could give you --- I'll
bet you I have over 100 more clips even better than them. They were just
pieced together over the last two hours. That was just --- oh, we have
far better than that. That's nothing compared to some of them.

But this was produced here in the White House by ---

THE PRESIDENT: Yeah, this was done by Dan and a group of people, and
they just put it together in a period of probably less than two hours.

Why did you feel the need to do that?

THE PRESIDENT: Because we're getting fake news, and I like to have it
corrected. They're saying what a great job we're doing, and the media
--- these are the governors of California, governor of New Jersey,
governor of New York.

Look, in New York, we work very close with Andrew. In New York,
ventilators --- we're going to be probably --- we didn't --- they didn't
have a problem. We got them tremendous numbers of --- thousands, but we
got them a tremendous number of ventilator. You don't hear ventilators
are a problem. Beds were going to be a problem. I mean, I'm happy about
it. The Javits Center, which is incredible, is almost empty because they
don't need them. That's good news, not bad news. I'm --- you know, I'm
not saying, ``Gee, I wish more people were there.'' I don't want more
people there.

We brought in the boat. We brought in the Comfort. And the Comfort was
originally not supposed to be for this at all --- the coronavirus. It
was not supposed to be for that at all. They called, they said, ``Could
we have it?'' That was a number of weeks ago. We said, ``We don't think
you need it, but if you need it, we'll do it.'' Then they said, ``Could
you get the medical personnel to run the Javits Center? Could you get
the medical personnel to run the ship?'' We said, ``If it's necessary,
we will.'' And we did. There were military personnel. That's the ones
that Mayor de Blasio was so great to, in terms of his statements. I
mean, I really appreciated his statements. He was so impressed with
them, and I am too. The level of genius and bravery. They're great
people, the military people.

And we pieced that together --- I would say it took less than two hours.
It was done in house.

Steve?

But just to be clear, this was produced by government employees, by
people here at the White House --- this campaign-style video here?

THE PRESIDENT: I wouldn't use the word ``produced.'' All they did was
took some clips, and they just ran them for you. And the reason they did
is to keep you honest. Now, I don't think that's going to work. It's not
going to have any impact. But just think of it: You heard the clips, you
heard what I said. They said I acted late on closing down the country.

Some people wish we never closed it down. Now, if we didn't, we would
have lost hundreds of thousands of people. You know, interestingly ---
so I'm --- I'm against that. We did the right thing. Everything we did
was right. If we would have closed down ---

You don't think you made any mistakes along the way here? You think
everything you did was right?

THE PRESIDENT: Well --- well, look, governors should have had
ventilators; they chose not to have them. We were able to get them
ventilators. You haven't heard --- other than, you know, there was a lot
of panic, a lot of screaming, ``We want ventilators.'' They got the
ventilators. You don't have that anymore. And the surge is supposed to
be coming now. And if they do need ventilators, Jon, we've got almost
10,000 that are ready to rock. We have people standing with those
ventilators right now. If you needed 2,000 in New York, which you don't,
but if you did, we can have them here in less than 24 hours. We're ready
to rock. This was a great --- this was a great military --- and beyond
that --- operation.

Let's get back to the regular. Shouldn't we get back to the regular? We
could talk about this, but all I'm doing is this: I could have given you
--- like, those are four or five clips that we just played for you. I
could have given you hundreds of people --- I mean, Gavin was on
television two days ago with one of your competitors, singing praises.
He says, ``Look\ldots{}'' You know, the question was asked in a negative
way. He said, ``Look, I know what you want to say but --- want me to
say, but he's been really good. It's hard for me to say that. In fact,
it's impossible for me to say it.'' Gavin Newsom, the governor of
California.

I have many clips from many --- I have some clips from Anthony, that I
didn't want to put up, which were really good. I think Anthony would be
the first one to say, when I closed the country to China, when I closed
the --- the China ban, whatever you want to call it --- Anthony said I
saved a lot of lives by doing that. I mean, am I correct? I don't want
to put words in Anthony's mouth, by the way, and I like him.

Today, I walk in and I hear I'm going to fire him. I'm not firing --- I
think he's a wonderful guy.

But why did you tweet something that said ``Fire Fauci''? Why did you
---

You retweeted the hashtag \#FireFauci.

THE PRESIDENT: I retweeted somebody. I don't know. They said ``fire.''
It doesn't matter.

Did you notice that when you retweeted it?

THE PRESIDENT: Yeah, I notice everything.

So you retweeted it even though it said, ``Time to fire Fauci''?

THE PRESIDENT: No, no, that's somebody's opinion. All that is is an
opinion.

But you read it and you elevated it.

THE PRESIDENT: No, I was called about that. I said, ``I'm not firing
him.'' In fact, if you ask your friends in the office --- in the public
relations office, I was immediately called upon that. And I said, ``No,
I like him. I think he's terrific.'' Because this was a person's view.
Not everybody is happy with Anthony. Not everybody is happy with
everybody.

But I will tell you, we have done a job, the likes of which nobody has
ever done. The mobilization, getting of equipment, all of the things
we've done --- nobody has ever done a job like this. We have 50
governors --- and territories, by the way. People don't ever mention
that we have territories. We have 50 governors and territories, and many
of those governors are Democrats, and they can't find anything to
complain about.

And honestly, many of them didn't do their jobs. I'll let you know
someday --- let's see what happens --- but I may let you know who's not
doing their job. I can tell you the ones that are good, both Republican
and Democrat, and the ones who don't know what they're doing. But we
help some of the ones that don't know what they're doing. They should
have had their own stockpiles.

And now, if they want, we can build them stockpiles of ventilators.

The hardest thing is a ventilator, because it's expensive, it takes a
while to get. We got them, and nobody believed we did.

Now, many of the governors were asking for far too many. And we said
they were asking for far too many. We talked and we said --- you said
very strongly that they just don't need that many. You said they don't
need that many beds --- Deborah.

So that's it. Steve, go ahead.

To be clear, you and Dr. Fauci are on the same page?

THE PRESIDENT: Yeah, we have been from the beginning. I don't know what
it is exactly. But if I put somebody's opinion up --- you know, I don't
mind controversy. I think controversy is a good thing, not a bad thing.
But I want it to be honest controversy.

Now, when I got a call --- I got a call not very quickly; nobody, you
know, saw that as being any big deal --- they said, ``How are you doing
with Dr. Fauci?'' I said, ``I'm doing great.'' And I didn't talk to Dr.
Fauci, even until we just got here. Dr. Fauci asked one of the people if
he could get up and speak, and he did.

So he said that once you ---

THE PRESIDENT: And they totally misinterpreted him. I saw what they did.

Can I --- can I ask you? He said the question was hypothetical. But what
he was just acknowledging is that lives would have been saved if the
mitigation practices were put into place earlier. That seems obvious. Do
you not agree with that?

THE PRESIDENT: Here's the thing. No, no. What he really is saying
though: ``But how could you have done it?'' Look, I just went over stats
with you. Right here. Right here.

How do you close it up? You have no deaths and no cases on January 11th.
Doctor, would you recommend closing the United States of America?''
``Oh, this must be terrible. How many cases do we have?'' ``None.'' How
many deaths do we have?'' ``None.''

January 17th --- go back another week. On January 17th --- this is 10
days before I did the --- little bit less than 10 days before I did the
ban. I did a ban on China. You think that was easy? I then did a ban on
Europe. And a lot of people said that was an incredible thing to do,
because you look at Spain --- and, by the way, we're doing very well
because when you look at all of those flat graphs and you add it all up,
the United States is very low, and per capita we're very low. We're
doing very well.

But how do you close up the United States of America? So, on January
6th, no deaths. On January 11th, no deaths. And no --- no cases. On
January 17th, no cases, no cases, no deaths. I'm supposed to close up
the United States of America when I have no cases?

You didn't close down until the middle of March. ``Should you have
closed it down earlier?'' --- that's the question.

THE PRESIDENT: I closed down from China.

It's not about January.

THE PRESIDENT: Excuse me, I closed it down from China. And, by the way,
some people think I should have waited longer and maybe ridden it out. I
disagree with them. Okay? But it was thought of. I mean, that was an
alternative. You know, there are a lot of people that would have said,
``Let's ride it out.'' Now, I'll give you the good news. If I would have
done that, it would have been, I think, catastrophic because their
numbers are, Anthony, 1.6 to 2.2 million people would have died if we
tried to do that.

And I did this last time: Cut it in half. Don't say 2.2 million. Cut it
in more than half. Say a million people died --- well, that's much more
than the Civil War. Cut it in half. Take the million and cut it in half;
that's 500,000 people would have died. Now, that number we would have
reached. Okay? That would have been easy to reach if we did nothing. So
we did the right thing, and our timing was very good.

But here's the one thing --- and you have to say this --- when you ask
me, ``Why didn't you do this?'' --- how come when I did the ban on China
and some very, very --- instituted some very tough things, how come
Nancy Pelosi, a month later, is in Chinatown, saying, ``Let's all march.
This is not going to happen''? How come we have many of the experts from
CNN and many other networks --- if you call CNN a network; I don't,
personally. But we have CNN, we have many other places, and they're all
saying, ``He doesn't need to do it. He doesn't need to do it.''

All I'm saying is this: How do you close down the greatest economy in
the history of the world when, on January 17th, you have no cases and no
death; when on January 21st, you have one case and no death? One case.
Think of it. Now, we're supposed to close down the country?

But here's what happened: When, on January 31st, I instituted the ban,
Joe Biden went crazy. He said, ``You don't need the ban. You\ldots{}''
He didn't go crazy. Like, he has --- he didn't even know what the hell
the ban was. But he --- so he didn't go crazy. But he did say ---

Your ban bought you time.

THE PRESIDENT: He did call me xenophobic.

What did you do with that time?

THE PRESIDENT: Wait a minute. He called me xenophobic; he called me a
racist because --- he has since apologized and he said I did the right
thing. So when you say, ``Why didn't you do this?'' Every Democrat
thought I made a mistake when I did it. I saved tens of thousands, maybe
hundreds of thousands of lives, by doing that.

But what did you do with that time that you bought? The argument is that
you bought yourself some time and you didn't use it to prepare
hospitals, you didn't use it to ramp up testing. Right now ---

THE PRESIDENT: You're so --- you're so ---

--- nearly 20 million people are unemployed.

THE PRESIDENT: You're so disgraceful. It's so disgraceful the way you
say that.

Tens of thousands of Americans are dead.

THE PRESIDENT: Let me just --- listen.

How is this sizzle reel ---

THE PRESIDENT: I just went over it.

--- or this rant supposed to make people feel confident ---

THE PRESIDENT: I just went over it.

--- in an unprecedented crisis?

THE PRESIDENT: Nobody thought we should do it. And when I did it ---

But what did you do with the time that you bought?

THE PRESIDENT: You know what we did?

The month of February. That ---

THE PRESIDENT: You know what we did?

That video has a gap. The entire month of February.

THE PRESIDENT: What do you do --- what do you do when you have no case
in the whole United States when you ---

You had cases in February.

THE PRESIDENT: Excuse me. You reported it: zero cases, zero deaths on
January 17th.

January.

January, February --- the entire month of February.

THE PRESIDENT: January. I said in January.

Your video has a complete gap: the month of February.

THE PRESIDENT: On January 30th ---

What did your administration do in February with the time that your
travel ban bought you?

THE PRESIDENT: A lot.

What?

THE PRESIDENT: A lot. And, in fact, we'll give you a list --- what we
did. In fact, part of it was up there. We did a lot.

It wasn't in the video. The video had a gap.

THE PRESIDENT: Look. Look, you know you're a fake. You know that. Your
whole network --- the way you cover it --- is fake. And most of you ---
and not all of you --- but the people are wise to you. That's why you
have a lower --- a lower approval rating than you've ever had before,
times probably three.

Twenty million people now are unemployed.

THE PRESIDENT: And when you ask me that questions ---

Tens of thousands of people are dead, Mr. President.

THE PRESIDENT: Let me ask you this: Why didn't Biden --- why did Biden
apologize? Why did he write a letter of apology?

I don't think the unemployed people right now care about why Joe Biden
didn't apologize to you, sir.

THE PRESIDENT: Why did the Democrats think that I acted too quickly? You
know why? Because they really thought that I acted too quickly. We have
done a great job.

So get us the list of what you did in February.

THE PRESIDENT: Now, I could've --- I could've kept it open. And I
could've done what some countries are doing. They're getting beat up
pretty badly. I could have kept it open. I thought of keeping it open
because nobody has ever heard of closing down a country, let alone the
United States of America. But if I would have done that, we would have
had hundreds of thousands of people that would right now be dead.

We've done this right. And we really --- we really have done this right.
The problem is the press doesn't cover it the way it should be.

Go ahead. One more question. Steve, go ahead.

There's a debate over what authority you have to order the country
reopened. What authority do you have?

THE PRESIDENT: Well, I have the ultimate authority, but we're going to
get into that in a minute. We're going to just finish this up. We're
going to tell you about other things that we've done right.

But I will say this: Had we said, ``Let's just keep going and let's not
do a closing'' --- whether it's 2.2 that they, at one point, predicted
as an outside or 1.6 at a lower number; you cut it all the way down to
6- or 7- or 800,000; take just a fraction of the number that could have
happened, it literally would have been more than the Civil War. It would
have been a disaster.

So, the minimum number was 100,000. And I think --- I feel pretty good
that we're going to be substantially below, Anthony, the 100,000. And I
hope we will.

All right. So, today, the Department of Health and Human Services is
announcing five new contracts to procure large numbers of additional
ventilators under the Defense Production Act, which we used a lot, by
the way --- which you didn't like to talk about --- in addition to the
1,300 we received today. We received, today, 1,300 additional
ventilators. Now, we're probably not going to need them, but we can add
that to our stockpile, which is very big, and we can move it around
should the surge take place and should it be a very substantial surge.
We're ready to --- we're ready to rock.

The contracts are with General Electric, Hillrom, Medtronic, ResMed, and
Vyaire, combined with the DPA contracts that we announced last week with
General Motors and Philips and two other contracts with Hamilton and
ZOLL. We're adding 6,190 ventilators to the Strategic National
Stockpile, of which we have a lot already --- thousands --- close to
10,000. But this will be added by May 8th, another 29,000 by the end of
May. And more than 120,000 total we will have by the end of the year.

Now, we're going to help other countries. We're going to help states if
they need it. We may help some states' stockpile. You know, they're
supposed to buy their own stockpile. They have state stockpiles. They're
supposed to be using that. And unfortunately, most of the states weren't
there. And a lot of people didn't want to talk about it, but they
weren't there. We will talk about it at the right time, if you want to.
I --- at this point, I'm more focused on getting past this nightmare of
a epidemic or a pandemic --- anything you want to call it. We got to get
past it.

No one who has needed a ventilator has not gotten a ventilator. Think of
that. You know, you heard all about ventilators, ventilators. ``We need
ventilators.'' Because they didn't have them. Because the states should
have had them. No one who has needed a ventilator has not gotten a
ventilator. No one who has needed a hospital bed has been denied a
hospital bed. That's not even really our responsibility.

Now, if we can help, we're going to do it. But that's where the Army
Corps of Engineers did such a great job. We built over 20,000 beds. In
fact, we built thousands more than we've actually needed to be safe. We
wanted to be safe and we really --- they rose to this incredible
occasion.

I mean, we built more beds than we thought. We thought, in Louisiana, we
were going heavy. And again, when I called the governor, I said, ``Maybe
we shouldn't build that second hospital, because we don't want to build
it if you don't need it.'' He called back, he said, ``I don't think
we're going to need it.'' They had 1,000 rooms, 1,000 beds, and they
used a lot of them, but they didn't need the other one so we stopped it
because we don't want to waste.

But we're prepared to build thousands more should we need it. I don't
think we're going to need it because it looks like we're plateauing and
maybe even, in many cases, coming down.

In addition, we've ordered a total of 60 mobile decontamidation
{[}sic{]} --- -contamination systems. So the decontamination system from
Battelle, in Ohio, is an incredible thing because it takes the masks,
and up to 20 times you can decontaminate a mask. And I've been asking
from the beginning: ``Why can't we sterilize and sanitize these masks?''
And it turned out we can. And there was a great company in Ohio, they
sent us some great equipment, and they're doing that now.

And now we're going to have more than 33 million N95 masks per week will
be cleaned, decontaminated, and it'll be great. It's something that,
frankly, I think people should have thought of a long time ago.

Five more flights landed today as part of the Project Air Bridge --- our
massive air lift operation to bring personal protective equipment into
the United States, which has now delivered nearly half a million N95
masks, 370 million gloves, 25 million surgical masks, and 4.9 million
gowns. So, we have millions of gowns, gloves, masks, all surgical
equipment coming in should the states need it.

Now, the states --- the states are supposed to be buying their own
stuff. But should they need it, we are ready to give them, because we're
building up our stockpile again like crazy.

Remember, I --- and you saw the stories. I inherited --- this
administration --- Mike, myself, the whole administration, we inherited
a stockpile where the cupboards were bare. There was nothing. And I say
it and I'll say it again: Just like we didn't have ammunition, we didn't
have medical supplies, we didn't have ventilators, we didn't have a lot
of things that should have been had.

And you can read your own stories on that because you know what
happened: They didn't want to spend the money. But we did.

To date, we've facilitated the supply of more than 38 million N95 masks
nationwide. This week, we'll be sending 2 million N95 masks to the
Commonwealth of Pennsylvania. The Vice President will go into more
detail. He's got great detail on that, and I think it's a pretty amazing
story. We have a lot of masks already in stock, and we have more coming.

We're further expanding hospital surge capacity in key areas of the
opening, and we have a portion of certain VA hospitals and non-veteran
coronavirus patients, including at the East Orange, New Jersey Medical
Center, as well as facilities in Manhattan and Brooklyn. They're ready.
They're able. They're beautiful. Hopefully, we won't need too many of
them because, frankly, we built everything that the governors wanted.
And in many cases, it's too much. We told them it was too much, but we
wanted to err on the side of caution.

The United States has now conducted nearly 3 million tests for the
virus. Three million --- the most of any nation. We are performing
approximately 150,000 tests every single day and our rate of testing is
especially high in areas hardest hit by the virus, if you look. And
that's really --- and it has hit some areas --- the virus --- very, very
hard. For example, per capita testing in New York is higher than the
rest of the world.

The NIH, CDC, and FDA are also currently validating several antibody
tests that will allow us to determine whether someone has already had
the virus and potentially become immune to infection. We're looking at
that. The antibody tests are going to be very interesting, over the next
short while. A lot of things are being developed, as we speak.

In the race to develop effective treatments, the drug company Gilead
announced that its drug, remdesivir, has shown promising results ---
very promising --- in compassionate use settings. In addition, the FDA
has just granted emergency use authorization for a device that removes
certain proteins from the bloodstream, possibly preventing a patient's
immune system from overreacting to the virus and damaging vital organs,
which is a big problem.

Furthermore, over the last seven days, my administration has deployed
roughly 28 million doses of hydroxychloroquine from our National
Stockpile. We have millions of doses that we bought and many people are
using it all over the country. And just recently, a friend of mine told
me he got better because of the use of that --- that drug. So, who
knows? And you combine it with Z-Pak, you combine it with Zinc ---
depending on your doctor's recommendation. And it's having some very
good results, I'll tell you.

I think if anybody recommended it other than me, it would be used all
over the place, to be honest with you. I think the fact that I
recommended it, I probably set it back a lot. But it's a lot of good
things that are happening with it. A lot of good tests.

Scientists are also pursuing a blood therapy known as convalescent
plasma. Convalescent plasma. This therapy uses antibodies from the blood
of recovered patients to treat those who are sick. And this is something
that actually is a very old procedure, but it's done in a very modern
way.

During this difficult time, we're also working to ensure that the 2020
Census is completed safely and accuracy. We may be asking for an
extension because, obviously, they can't be doing very much right now.
They wouldn't even be allowed to do it. So, the Census, we're going to
be asking for a delay --- a major delay, I think. How can you possibly
be knocking on doors for a long period of time now?

The Census Bureau recently made the decision to temporarily suspend its
field operation data-collection activities to help stop the spread. In
addition, while millions of Americans continue to complete their
questionnaire online, the Census Bureau has asked Congress for a 120
extension. I don't know that you even have to ask them. This is called
``an act of God.'' This is called a --- a situation that has to be ---
they have to give in. I think 120 days isn't nearly enough.

My administration is also taking bold action to help American workers.
On Friday, Americans began receiving the cash payments authorized by a
historic \$2 trillion relief bill.

By the end of the week, nearly 80 million Americans will receive a total
of \$147 billion. And from what the Secretary of the Treasury tells me,
that's very much on time and going very nicely. He'll be speaking in a
moment. And payments --- these payments go directly into the banks and
the bank accounts of these people. Millions of additional payments will
follow. The typical family of four will receive \$3,400. That's for a
family of four. That's something.

Additionally, through our Paycheck Protection Program --which is a
tremendous success, and they should extend it and increase it. This has
been a tremendous success. So successful that the banks are taking a
little bit longer to distribute the money, but it's going rapidly.

We've now processed over \$200 billion in loans to help small businesses
retain their workers. Now we urgently need lawmakers to set aside the
partisan agendas and to replenish this program with new funds because
it's really something that has been an incredible success. And they need
more money to keep it going to take care of these business and keep them
--- keep them open.

I want to thank the many governors, health professionals, scientists,
and business leaders for their incredible hard work and input over the
past month, and even long beyond a month, Mike, I would say. You know,
we've been working together with a lot of for, it seems like, forever.

I've been having many discussions with my team and top experts, and
we're very close to completing a plan to open our country, hopefully
even ahead of schedule. And that's so important.

We will soon finalize new and very important guidelines to give
governors the information they need to start safely opening their
states. My administration's plan and corresponding guidelines will give
the American people the confidence they need to begin returning to
normal life. That's what we want. We want to have our country open. We
want to return to normal life. Our country is going to be open, and it's
going to be successfully opened. And we'll be explaining over a very
short number of days exactly what is going to be.

We've also, as you probably heard, developed a committee. We're actually
calling it a number of committees with the most prominent people in the
country, the most successful people in the various fields. And we'll be
announcing them tomorrow.

This weekend, the United States also helped facilitate an unprecedented
agreement among the 23 nations of OPEC Plus --- that's OPEC plus
additional energy-producing nations --- representing many of the world's
largest oil-producing countries to stabilize oil markets. And we have,
in fact --- and I think you've seen a big stabilization over the last
couple of days.

Together, countries around the world will cut oil production by
approximately 20 million barrels. People are saying 10 million, but we
think that the number that they'll actually hit is going to be closer to
20 million barrels a day. And that will help a lot with saving jobs all
over Texas and Oklahoma and North Dakota and many of other --- other of
our big energy states.

This historic action will help nearly 11 million American workers who
are supported by the U.S. oil and gas industry. It's a very monumental
agreement. I want to thank Saudi Arabia and the King of Saudi Arabia,
the Crown Prince of Saudi Arabia --- both. I want to thank President
Putin of Russia. And I want to thank a very good friend of mine --- a
man who's become a friend of mine: The President of Mexico showed great
flexibility. President López Obrador. He showed great flexibility and
--- and tremendous intelligence doing what he did. It was not that easy
for him. And I want to thank Mexico and the president. This is a very
historic deal. Very historic. So we'll see how it all goes.

In this time and challenge --- and we are certainly in a time probably
like we haven't been in many, many decades --- we are strengthened and
sustained by the bonds of love and loyalty that unite all Americans. I'm
so proud of the American people.

Everywhere you look, you see the patriotism of our people shining
through and the courage of our doctors and nurses on the frontlines, and
the dedication of our food supply workers, and in the commitment of
every citizen to achieving victory over the virus. That's what's going
to happen. It's going to happen sooner than people think. And we're
going to be smart about it. Very, very smart about it. We're going to be
safe about it. We're going to be listening also to the great doctors and
medical professionals.

Together, we're beating back the invisible enemy and we're paving the
way for great resurgence. Really, a great resurgence for American
prosperity. Our country wants to go back. They want to go back to work.
They're going to go back safely, and that's what we want.

I'd like now to ask Vice President Pence to say a few words, followed by
Dr. Fauci and Dr. Birx. I think before we --- before we do this ---
because I know there's an emergency where they want Steve to come. So
what I'll do is I'll ask Steve to come up --- Secretary of the Treasury.
You can talk a little bit and then maybe take a couple of questions
about what's happening. Tell them the success we're having.

Thank you very much.

-\/-

I just wonder --- I wanted to ask you, Mr. President, what you think ---
what --- if you could sketch for us what reopening the economy looks
like. Do you think it's going to be everything open? Or do you think ---

THE PRESIDENT: Well, I'll be doing that over the next few days because
we'll probably be making a statement about that and exactly what it
looks like.

I know what that looks like, but I also want to get the advice, in a
sense. We have some of the --- the biggest from every business on this
council. We're actually setting up a number of different councils or
committees, I guess you could call them. And we have a lot of smart
people. I think that they will give us some also good advice. But, no,
we want to be very, very safe. At the same time, we got to get our
country open.

Yeah, I understand that, Mr. President. Do you think there is a
possibility then that what you do is you open it incrementally? Do you
think people will go back to restaurants, to concerts, the cinema?

THE PRESIDENT: I do think so. Eventually, they will. Yeah.

And let me ask one final ---

THE PRESIDENT: I think eventually they will do that. And I think we're
going to --- boom --- I think we're going to --- I think it's going to
go quickly. Our people want to get back to work, and I think there's a
pent-up demand like there hasn't been in a long time.

And that's why --- and that's why you see the stock market --- I mean,
to think that the stock market is at the level it is right now, with all
that this world and this country has gone. And look at the European
Union, how badly decimated they've been. Look at other countries. Look
at China, by the way. I've seen the numbers. Look at China. Look at how
these countries have been just decimated by this.

And to think that the stock market is at this tremendously high number.
Not that much --- you know, it was looking a little bleak for a while,
but it --- it hit a certain point and then started going up. I think
that's a great tribute to the fact that there's a demand.

Yeah.

Mr. President, thank you, sir. In regards to some of your tweets earlier
today, and I think it was Steve's question, my question to you is: What
provision in the Constitution gives the President the power to open or
close state economies? And then ---

THE PRESIDENT: Numerous provisions. We'll give you a legal brief if you
want.

And then --- we'll be looking forward to that, sir. But following up:
What happens if you say, for instance, ``We want states to reopen but
California or New York do not open''? What would you do then?

THE PRESIDENT: Well, I think everybody wants to open. I mean, I guess,
you know, that could happen, but I don't think that would happen.

Go ahead please.

It's been states that have closed, ordered schools closed. It's been
states that have ordered businesses like restaurants and bars closed.

THE PRESIDENT: That's because I let that happen because I would have
preferred that. I let that happen. But if I wanted to, I could have
closed it up. But I let that happen and I like the way they've done it.
And the seven that remained really in sort of a semi-lockdown --- if you
look at those states, they've really done a very good job. They're very
much different from a New York or from other places where they've been
hit very hard.

So you're prepared then to bigfoot states and say, ``I order you to open
your schools, I order to force kids to be able to go''?

THE PRESIDENT: Go ahead, please. Yeah.

Yes, Mr. President. Following up on that, there are two consortiums of
states today --- California, Oregon, and Washington on the West Coast;
Northeastern states --- totally representing about 100 million people,
who have said they're going to cooperate and decide when to reopen those
states.

THE PRESIDENT: Well, they can decide, but ---

Does that undermine what you're trying to do?

THE PRESIDENT: No, not at all. Let me just tell you --- very simple. I'm
going to put it very simply: The President of the United States has the
authority to do what the President has the authority to do, which is
very powerful. The President of the United States calls the shots.

If we weren't here for the states, you would have had a problem in this
country like you've never seen before. We were here to back them up. And
we're backing --- and we've more than backed them up. We did a job that
nobody ever thought was possible. It's a decision for the President of
the United States.

Now, with that being said, we're going to work with the states because
it's very important. You have local governments, they're pinpointed.
It's really --- you talk about --- it's like a microchip. They are
pinpointed. We have local government that hopefully will do a good job.
And if they don't do a good job, I'd step in so fast. But no, they can't
do anything without the approval of the President of the United States.

But, Mr. President ---

THE PRESIDENT: Go ahead, please.

So if some states refuse to reopen and you order them to, the 10th
Amendment of the Constitution says all powers that don't reside in the
President or Congress reside in the states. How do you overcome that?

THE PRESIDENT: Well, if some states refuse to open, I would be --- I
would like to see that person run for election. They're going to open.
They're going to all open.

So that's a valid (inaudible).

THE PRESIDENT: I think that's something that's not going to happen. They
want to open. They have to open. They have to get open. Every one of
those states, the people want to go and they want to ---

Now, some will be --- are in a different situation. You have --- I won't
name states now, but I will over the next two or three days. I'm going
to be very specific. But you have some states where this is not the kind
of a problem than it is in New York or Louisiana or Michigan or other
places that got hit very hard. Illinois got hit very hard. But all
states want to open and they want to open as soon as possible. But they
want to open safely and so do I.

Yeah, please.

Thank you, Mr. President. Today, the French President Emmanuel Macron
said that he will keep the shutdown in France until the middle of May.
Does that mean that the U.S. will keep the ban from flights from ---

THE PRESIDENT: No, France got hit very hard. France got hit very hard.
And again, he has to do what he has to do. He's a friend of mine. But
France --- Spain has just been decimated. You look at what's happened in
Italy, it's a very well-known fact, what happened in Italy. No, they
were hit very hard.

-\/-

Do you still see a need for a phase four stimulus? Or is this push to
reopen ---

THE PRESIDENT: Steve, I just want to say we have had ---

--- the economy in lieu of that?

THE PRESIDENT: --- discussions --- wait, excuse me. One second, please.
We've had very good discussions with the airlines. Very good
discussions.

And is it possible to reopen the economy on May the 1st?

THE PRESIDENT: I don't want to say that. You'll be hearing over the next
few days.

-\/-

Mr. President, just to clarify your understanding of your authority
vis-à-vis governors --- just to be very specific: For instance, if a
governor issued a stay-at-home order ---

THE PRESIDENT: When you say ``my authority'' --- the President's
authority. Not mine, because it's not me.

If I could just ask the question ---

THE PRESIDENT: This is --- when somebody is the President of the United
States, the authority is total, and that's the way it's got to be.

It's total? Your authority is total?

THE PRESIDENT: It's total. It's total.

Your authority is total?

THE PRESIDENT: And the governors know that.

So if a governor ---

THE PRESIDENT: The governors know that. No, you have ---

If a governor issues a stay-at-home order, you ---

THE PRESIDENT: --- a couple of bands of --- excuse me. Excuse me. You
have a couple ---

Could you rescind that? Could you rescind that order?

THE PRESIDENT: You have a couple of bands of Democrat governors, but
they will agree to it. They will agree to it.

What if it was a Republican governor?

THE PRESIDENT: But the authority of the President of the United States,
having to do with the subject we're talking about, is total.

Yeah, please, go ahead.

Mr. President, one of the things you ---

THE PRESIDENT: Go ahead, please.

One of the things you --- one of the things you spoke --- we saw in your
video about was the travel ban from Europe. As part of reopening
America, do you want to reopen the borders so that people from Europe,
from the UK ---

THE PRESIDENT: At the right time.

How soon do you think we are from ---

THE PRESIDENT: And a very good question, actually.

Well, I'm going to have to take a look. I wouldn't say Italy is doing
great right now, and I wouldn't say Spain is doing great right now. And
we just heard that France is extending its stay-inside order, right?
Their stay --- they've extended it --- I just see that --- and, I think,
for a short period of time.

But no, when they're back. We want to do it very quickly, but we want to
make sure everything is good.

No, right now we have a very ---

Weeks, months?

THE PRESIDENT: Right now we have a very strong ban. We're going to keep
it that way until they heal.

Weeks, months? What would you ---

THE PRESIDENT: Well, I can't tell you that. I can't tell you that. I
have to see: How are they doing? I mean, France just went for another
two days --- for another two weeks. We have to see.

Jon.

So Dr. Fauci said that you took his advice on the question of
mitigation. He made the recommendation. You accepted it. You put into
place. As you make this next decision, which, as you have said ---

THE PRESIDENT: Well, I'm not sure who --- Jon, I'm not sure who really
gave me advice on the ban. I think I took ---

No, not on the ban. I'm talking about the mitigation.

THE PRESIDENT: I think I took my own advice on the ban. I don't know.

--- the social --- the social distancing, I'm talking about. The
shutting down.

THE PRESIDENT: Okay.

Not of travel, but of activity. So my question is: As you make this next
decision, which you have said may be the most difficult or important
decision of your presidency, will you assure the American people that
you will again take the advice of the doctors --- of Dr. Fauci, of Dr.
Birx? Will you take the advice of the health experts before you do that?

THE PRESIDENT: I will and many other people also. But I will absolutely
take their advice.

But would you go against them?

THE PRESIDENT: Please go ahead.

Yes, Mr. ---

Would you go against their recommendation? If they say you need another
15 or 30 days, would you ---

THE PRESIDENT: I don't think it would be likely because I think we're
not very far from being on the same page.

Please.

Yes, Mr. President, one thing that Governor Cuomo said today is that
states do not have the capacity to do the mass COVID-19 testing ahead of
a reopening as ---

THE PRESIDENT: Well, they have to do it. Look, they're supposed to be
doing it.

He says he can't purchase the diagnostic tests or equipment ---

THE PRESIDENT: Yeah, I know. I know.

--- and needs federal help. So will the states get that?

THE PRESIDENT: Well, they --- they may need help, but ---

Will they get it?

THE PRESIDENT: But they're there. They're on ground. They've got local
mayors, local representatives. They have people that do it.

But he needs the supplies.

THE PRESIDENT: And what we did last time is unprecedented. We literally
rebuilt tests. We --- we rebuilt a whole industry because we inherited
nothing. What we inherited from the previous administration was totally
broken, which somebody should eventually say. Not only were the
cupboards bare, as I say, but we inherited broken testing. Now we have
great testing.

I just left the top executives at Abbott. Who would have thought that
would have happened, where they have such a great test as that?

And, in fact, what I'll do --- I think, unless you have any further
questions for the Secretary of the Treasury --- do you have anybody for
Steve? Anybody?

Mr. Secretary Mnuchin ---

THE PRESIDENT: Is that for the Secretary of the Treasury or for me?

For Secretary Mnuchin, yeah. Yes, sir.

THE PRESIDENT: Because if it's for me, we can wait.

It's for Secretary Mnuchin.

THE PRESIDENT: We have to get him back to work, okay?

-\/-

It's been reported ---

THE PRESIDENT: Well, we didn't want to do the Kennedy Center, just so
you understand. And that that was done --- the Democrats wanted it in.
We didn't want that, but they wanted it in. And we had to agree in order
to get something done for the workers. But we want this to be for the
workers and for companies that employ the workers. That's what we're
looking for. We're not looking for extraneous nonsense.

-\/-

A quick question about something you just said. You said, ``When someone
is President of the United States, their authority is total.'' That is
not true. Who --- who told you that?

THE PRESIDENT: Okay. So you know what we're going to do? We're going to
write up papers on this. It's not going to be necessary, because the
governors need us one way or the other, because ultimately it comes with
the federal government. That being said, we're getting along very well
with the governors, and I feel very certain that there won't be a
problem.

Yeah, please. Go ahead.

Has any governor agreed that you have the authority to decide when their
state opens back up?

THE PRESIDENT: I haven't asked anybody because I don't --- you know why?

Because no one has --- no one has said that.

THE PRESIDENT: Because I don't have to.

Go ahead, please.

But who told you the President has the total authority?

THE PRESIDENT: Enough. Please.

You mentioned the Vice President's call with the governors today.
Governor Hogan of Maryland has urged your administration to ask Congress
for \$500 billion to help stabilize budgetary shortfalls created by
coronavirus.

THE PRESIDENT: It's nice of Governor Hogan, very much. We appreciate
Governor Hogan's statement.

Governor Cuomo said the CARES Act ignored state government shortfalls.

THE PRESIDENT: Cuomo.

Do you support that request?

THE PRESIDENT: Which one? What did he say?

He said the CARES Act ignored the budgetary shortfalls.

THE PRESIDENT: Well, they're looking at things in phase four, where they
have --- you know, where they talk about states, and they're also
talking about hospitals. They're talking about states who have been
battered, and they're also talking about hospitals. And we're certainly
willing to look at that.

Will you urge Congress on their behalf?

THE PRESIDENT: You know, we'll see what we all come back with. But they
are talking about states and they're talking about hospitals.

OAN.

Thank you, Mr. President. The Governor of Michigan, Gretchen Whitmer,
has on Thursday signed an executive order banning the sale of
non-essential goods. If other states follow ---

THE PRESIDENT: The sale of what?

Non-essential goods. She has banned the sale of non-essential goods.
Many are calling this draconian, unconstitutional. As President, do you
think that if other states were to follow her example in the coming
weeks, that the federal government should intervene?

THE PRESIDENT: Well, I don't think that's going to happen. I think it's
very extreme. But she's doing it, and I think it's going to be over a
long way before we have to start thinking about it too much. It is
strong. It's a very strong position to take. But they're making a lot of
progress in Michigan, so let's see how it all works.

What is the status of the funding for the World Health Organization?

THE PRESIDENT: We're going to be talking about that very soon. I'm
getting a full report. I'm not happy with the World Health Organization.
I'm not happy with the World Trade Organization either. We've been
ripped off by everybody.

And we have --- this country, for so many years, has been ripped off by
everybody, whether it's a World Health or World Trade. And they're like
--- I call them the ``Bobbsey Twins.'' They'd look at our country ---
for years and years, we had people that did nothing about it. We're
doing a lot about it. So we'll have a report.

And we'll also --- we're also talking about the World Trade
Organization. But we've made a lot of progress there. We're now winning
cases for the first time, because they know I'll leave if we don't get
treated fairly.

This country --- our country --- was at a point where we rarely, if
ever, won the lawsuits within the World Trade Organization. But now
we're winning a lot of them, because they know I'm not --- I'm not
playing games. We will pull out if we have to. We just won a
7-billion-dollar lawsuit, which was very nice.

Do you expect a decision this week on cutting funding for the WHO?

THE PRESIDENT: Uh, yeah. I would say, by the end of the week, I'm going
to make a decision on that. Yeah. There's a lot of --- right now,
there's a lot of things happening.

On China --- why are there no consequences for China, for the
misinformation that they shared?

THE PRESIDENT: How do you know there are no consequences?

Because you've said. Well, you've been asked, and it appear that there
were no ---

THE PRESIDENT: How do you know there are no consequences?

What are the consequences, Mr. President, for the misinformation ---

THE PRESIDENT: I wouldn't tell you. China will find out. Why would I
tell you?

But people are concerned that they stonewalled, that they gave
misinformation ---

THE PRESIDENT: No, you started off by saying, ``Why are there no
consequences?''

Because you've been asked this a few times, so I'm following up on your
response. Why are there no consequences for China?

THE PRESIDENT: How do you know there are no consequences?

Because we've asked you ---

THE PRESIDENT: You're going to find out.

Q --- and you've said --- you've said that didn't want to have an
consequences because you suggested trade.

THE PRESIDENT: I wouldn't tell you. You'd probably be the last person on
Earth I'd tell.

So you're saying there will be consequences?

THE PRESIDENT: Go ahead. Uh, yeah, please.

Mr. President --- actually, this is a question for Mr. Vice President.
Do you agree with the President's statement and his understanding of
federalism, that his power is total --- like in the way he described it?
Is there anything you'd like to add or any context you'd like to add to
the way he was discussing that?

** PENCE RESPONDED

-\/-

But it sounds like you think his power is a little bit more
circumscribed than totally.

THE PRESIDENT: Well, make no mistake about it: In the long history of
this country, the authority of the President of the United States during
national emergencies is unquestionably plenary. And you can look back
through times of war and other national emergencies. And as the
President said, we'll happily brief that.

But in the days ahead, what the President has charged us to do is to
work with our health experts. We're going to bring together an
extraordinary group of American business leaders to counsel the
President. And then, working with the CDC, we're going to produce new
guidelines, based upon the data, for every state and territory in this
nation. We're going to give them guidance. And, as the President has
indicated, we'll continue to respect the leadership and partnership that
we forge with every governor in America.

But this is an unprecedented time. But I have to tell you: When you look
at the fact --- despite the heartbreaking loss of more than 22,000
Americans --- when you look at the fact of what the health experts told
us this could be, I think I only can feel a sense of gratitude to the
American people, gratitude to the extraordinary team that has counseled
this President, the steps that President Trump has taken, the policies
that governors have implemented all across America.

I mean, we were discussing today, at the task force, that when you look
at the European Union as a whole, they have nearly three times the
mortality rate that the United States of America has today. And that is
a tribute to our extraordinary healthcare workers, their dedication,
their tireless work. But it's also a tribute to the fact that the
American people put into practice the mitigation efforts that the
President counseled the nation to do on the advice of our best
scientists, now more than a month ago, and our hospitals were not
overwhelmed and are not overwhelmed at this hour.

And I have to tell you that standing here today, I couldn't be more
proud to stand alongside this President and to be a part of this team
that has served the American people during this challenging hour.

And I just say to you: To every American looking on, as we see the
numbers leveling and maybe even beginning to go down, I just encourage
you to keep doing what you're doing. Because of the sacrifices that
Americans and American families have made through these mitigation
efforts, you're saving lives and you're seeing our nation through this
time.

-\/-

THE PRESIDENT: All right, go ahead, with the face mask. Go ahead.

Sir, if you can hear me through the mask ---

THE PRESIDENT: Barely.

Can --- can you ---

THE PRESIDENT: I hear you well, actually.

--- the --- the District of Columbia argues that they were shortchanged
in the most recent funding bill because they were treated as a territory
instead of as a state. Will that be made right in phase four?

THE PRESIDENT: Well, we're looking at that certainly. I heard that
complaint, but the mayor seems to be very happy with everything we've
done.

THE VICE PRESIDENT: She was on today.

THE PRESIDENT: I mean, she's actually --- and she was on today, saying
very good things.

Okay, yeah. Go ahead in the back.

Mr. President, you talked about this being the most difficult decision
that you are going to have to take about whether to reopen the economy.
I wonder how much it weighs on your mind the thought that if there is a
second wave, you've reopened the economy and you might have to shut
things down again.

THE PRESIDENT: It does. And I hope that won't happen. I certainly hope
that won't happen, but it does weigh on my mind.

Okay, in the back, go ahead. You had one. Go ahead.

Mr. President?

THE PRESIDENT: Go ahead, please.

Thank you. Okay, thank you. A question for one of my colleagues who
wasn't able to be here. China deployed an aircraft carrier into the
South China Chi {[}sic{]} --- South China Sea this weekend, amid claims
by Chinese state media that COVID has reduced U.S. military readiness in
the region. What kind of responses are you thinking of? Will you have a
response to this action?

THE PRESIDENT: China has their own difficulties. We have a relationship
with China that --- we're not happy with certain things that happened
over the last period of time, as you know, and I've been very explicit
on that. But we know all about that.

And, no, China is --- we've seen what they did. We've seen many other
things that they've done, both pro and con. And we'll be just fine.

On Abbott, you said something earlier, where you said that you're
putting together the economic task force and that you thought that the
recommendations were happening earlier than expected. Did you mean to
suggest that it could be before May 1st that you start recommending that
states open?

THE PRESIDENT: I don't want to say that, but we're going to be putting
out guidelines and recommendations fairly quickly. In a few days.

You're not ruling out that it would be before?

THE PRESIDENT: I'm not going to say. But, look, certain states are doing
very well. Certain big parts of the country are doing very well. They're
doing, really, very well. And so we're going to be putting out
recommendations and guidelines very soon.

Steve?

And would these new guidelines be --- would they fit each area or would
they be a uniform set of guidelines?

THE PRESIDENT: Well, you're going to see. I --- I don't want to tell you
now, but right now we have a very strong indication that we know, pretty
much --- we have some good ideas.

I also do want to get --- I want to have --- we'll have video conference
or at least a conference call with a lot of very good people, having to
do with certain fields, whether it's energy or whether it's
entertainment and restaurants, et cetera, et cetera.

We have to get people back into restaurants. We have to do what we have
to do. Whether it's deductibility or not, we'll see, but it should be
deductibility. You'll get them back so fast. I mean, they used to have
deductibility. The restaurant business, it --- it was one of the hot
businesses. And then they ended it a long time ago, many years ago. But
we may need that to get people back into the restaurants.

Please.

Yeah. Michelle Obama, today, got behind mail-in voting nationwide as a
possible solution to the --- on states. She said it shouldn't be a
partisan issue. Have your advisors told you that that could save lives?
And (inaudible)?

THE PRESIDENT: Absentee ballot, are you talking about? Absentee ballot?

Yeah, and on a massive scale because of the coronavirus.

THE PRESIDENT: Well, I don't know what she did. I mean, I didn't see
that. When did that happen? Today?

Yeah, she's part of --- that's a nonpartisan group.

THE PRESIDENT: Well, I wish her luck. I wish her a lot of luck.

Please, go ahead.

On Abbott Labs, you said you had long ---

Yes, Mr. President. There's a little bit of confusion about your phone
calls yesterday with President Putin. The Kremlin is saying that you
discussed current issues of ensuring strategic security. That wasn't
referred to in the White House readout. Can you enlighten us ---

THE PRESIDENT: We discussed many things. We did discuss China. We
discussed many different things, but we --- it was primarily a call on
the oil, as you can imagine. And they were very helpful in getting a
stabilization price, a stabilization of the number of barrels. I think
the number is going to be closer to 20 --- maybe 15, but closer to 20
than it is going to be to 10.

And I think it was a very important call.

I also spoke with the King of Saudi Arabia and that was a very important
call. And the bottom nine is OPEC Plus. It's called OPEC Plus because
there are other states also, other nations. We came to a very good
agreement.

Please.

You have ---

THE PRESIDENT: Go ahead.

What was the part about strategic security? Was that ---

THE PRESIDENT: I would say mostly we were talking about China. We were
talking about their borders. And we're talking about our borders a
little bit --- our borders with Mexico. Because, as you know, Mexico is
a big part of the deal. And Mexico really --- it was very complex from
the standpoint of Mexico. It was not an easy deal for Mexico. And the
President --- we appreciated a certain amount of flexibility. But we
talked about borders, we talked about China, we talked about Mexico.

But ``strategic security'' sounds more like arms treaties.

THE PRESIDENT: Well, I mean, I --- we did talk about the arms. Yes, we
did. That was a very important part of the call actually. Yeah, good
point.

Please.

So, on Abbott Labs, you said testing is going great. We know that they
have --- these machines have been sent to some of the governors, but
some of them are saying they don't have the materials to actually
conduct the tests.

THE PRESIDENT: Well, they have to get the material. You know, the
governors have to get the material.

The cartridges.

THE PRESIDENT: Now, if they can't get it, they're going to see us.

The government is --- the federal government is distributing those
cartridges.

THE PRESIDENT: I'm talking about the local governments. I'm talking
about governors have to get the material. Now, they have machines. In
fact, we --- we're going to go into --- I'd ask Mike to go into it as
soon as I leave. They have very powerful machines that they don't know
they even have. I'm not talking about Abbott; I'm talking about the
governors. They have machines that are used for this ---

The hospital labs.

THE PRESIDENT: You know what I'm talk- --- do you know what I'm
referring to?

Yeah, they have the two different machines.

THE PRESIDENT: Very big, very powerful machines where, in a certain
state's case, they're only using 10 percent of their capacity and they
didn't know it. That happens to be Illinois.

Jon, please.

Okay, well, real quick. Real quick. These 15-minute tests that you've
sent out, these new ones that you had in the Rose Garden, they say ---
including Governor Sununu in New Hampshire --- that they don't have the
cartridges to actually conduct the test. So when will they get those
cartridges?

THE PRESIDENT: What do you think the answer to that is?

THE VICE PRESIDENT: We're rapidly --- we're rapidly increasing the
numbers, Mr. President.

THE PRESIDENT: Rapidly increasing the numbers.

THE VICE PRESIDENT: And Deb will speak to (inaudible).

Can we look into when?

THE PRESIDENT: Rapidly increases. Well, pretty quickly.

They can --- they can do 50,000 a day, right?

THE PRESIDENT: Well, you have other machines where they can really work.

Can they go up beyond that?

THE PRESIDENT: And a lot of the states have the big machines that can do
a lot. They didn't even know they had them. They didn't even know that
they had them. And Mike is going to be talking about that.

And you remember you mentioned, several weeks ago, the --- that Google
was putting together that website where they would handle the
drive-through testing?

THE PRESIDENT: Yeah. Google and Apple.

Have you --- have you given up on ---

THE PRESIDENT: You mean Google and Apple combination.

Have you --- have you moved past that, because you said ---

THE PRESIDENT: No. A lot of people don't like it from the standpoint of
constitutional rights. I mean, a lot of people don't like it and some
people think it's great. No, they are working on that, as I understand.

How about the testing website? Remember you said a website for Google
and ---

THE PRESIDENT: Yeah, no, I know that. I know that. I know.

And it's only operating in, I think, five counties in California right
now. Is that ---

THE PRESIDENT: That's right. No, Google is looking at it, but Google is
also working with Apple or looking at something. We have the greatest
companies in the world looking at things that, in a year from now,
everything that we're looking at now is going to be obsolete. That's how
good it is.

We have things happening that are unbelievable. I saw a presentation
today that I can't talk about yet, but it's incredible. Plus, I think
they're doing --- Tony --- I think they're doing very well in the
vaccines. They're working hard on the vaccines and I think you'll have
an answer for vaccines. I believe that there's some great things coming
out with respect to that. Now you need a testing period, but you're
going to have some great things.

Please.

Sir, on the contact tracing that Google and Apple are doing --- so a
different subject, on the contact ---

THE PRESIDENT: No, no. This is --- this is the Google and Apple. I don't
know if it's a partnership or what, but they're working on some ---

Correct. So there was the one ---

THE PRESIDENT: They're working on more than one element. They're working
on a couple of different things, Google and Apple. Google is also
working on something, as you know, having to do with testing. I believe
they're doing that in a singular fashion.

So my question is not about the drive-through testing website. Not that.

THE PRESIDENT: Okay.

On Google and Apple's contact tracing that they want to ---

THE PRESIDENT: Yeah. Yeah.

--- they've got this process now where they can put, you know, contact
tracing on your phone.

THE PRESIDENT: I know.

If you opt in, you can be alerted if you've been ---

THE PRESIDENT: That's right.

--- in contact with someone with the coronavirus. Do you --- how do you
feel about that?

THE PRESIDENT: Well, it's an amazing thing, but a lot of people have
some very big constitutional problems with it. You know that. It's an
amazing thing and it would be --- actually, as you know, other countries
are thinking about using something similar but not as good.

Which other countries are thinking about something similar?

THE PRESIDENT: I hear Singapore is. Singapore is. No, Singapore had a
little bit of a setback because they had a --they had a break. You know
that, and --- but they'll take care of it. I know --- I know the folks
in Singapore. They're doing a great job and they're going to put it back
very quickly. But Singapore and other countries are looking at other
things, and some countries are doing other things.

Would you prefer that Americans use some other system?

THE PRESIDENT: Well, I don't want to get into that because we have a
whole constitutional thing. We have more of a constitutional problem
than a mechanical problem, but we will be making a determination on
that. That's something we're going to be discussing with a lot of people
over the next four weeks. That would be a very accurate way of doing it,
but a lot of people have a problem with it.

Yeah, please. Go ahead.

A testing question, maybe for Dr. Fauci as well. Can you talk about
where the antibody test is and how quickly that will be (inaudible)?

THE PRESIDENT: Well, it's moving along. I think I can speak because I
--- I have to leave. Moving along quickly, moving along well. It's a
test that's been going along for many, many years, except now we have
very modern, very incredible versions of it. But that's moving on. The
antibody test have --- moving along very well.

Okay, anymore COVID-19? COVID-19?

Yeah, Steve.

One soldier on the Theodore Roosevelt has died. Has --- have you
determined the status of Captain Crozier, the former commander?

THE PRESIDENT: Well, that's going through the Navy, as I understand it.
The Navy is going to be making decisions on all of that. And they had a
break in --- I don't think the ship should have been stopping in Vietnam
when you have a pandemic, to be honest with you. You know, I don't think
the captain should have been writing letters. He's not Ernest Hemingway,
as I said before, and he shouldn't have been writing letters.

And I don't think --- I don't know who gave the orders to stop in
Vietnam. But they stopped in Vietnam and all of a sudden they get on,
and now you have over 500 sailors and --- and people on the ship that
are affected. I don't know whose idea that was, but that wasn't such a
good idea in the middle of a pandemic.

Yes, please. Jon. Go ahead, Jon.

Just one last --- on this question of constitutionality, I'm just
wondering what changed your view because ---

THE PRESIDENT: Nothing changed it. No, no, I know exactly what you're
going to say. Nothing changed it. The fact that I want to rely on states
or maybe will or maybe have, and the fact that we've gotten on ---
that's one thing. The fact that I don't want to use the power is another
thing. Look ---

But you said from the standpoint of the Constitution ---

THE PRESIDENT: Yes, Constitution.

--- you thought it should be up to the governors.

THE PRESIDENT: Constitutionally. You can look at constitutionally. You
can look at federalism. You can look at it any different way. Jon, the
fact that I don't want to exert my power is much different. We have the
power.

You asked, ``Does the federal government have the power?'' The federal
government has absolute power. It has the power. As to whether or not
I'll use that power, we'll see. I would rather ---

So if New York wants to stay closed down, you can ---

THE PRESIDENT: Jon, I would rather work with the states, because I like
going down to a local government. That's why with --- I guess it's now
seven states not eight that --- because South Carolina did --- you know,
they went away from what we discussed the last time. So that's why I
looked at the individual states; they're doing a very good job. They're
really doing a very good job. I'd rather have them make the decision.

Now, the fact that I'd rather have --- that's fine. But I have the
absolute right to do if I want to. I may not want to. We have a very
good relationship. Now, we'll see what happens.

If you notice, the few states you're talking about, they're all with
Democrat governors. But if governors are doing a good job and they
control it better --- because you don't have somebody in Washington
saying, ``Set up a testing site in the parking lot of a Walmart.'' And
we're in Washington and they're in a state that's very far away. That's
really --- it should be and it should have always been. And I've always
said it was.

But the relationship we have now with the states and governors is very
good. And we'll be announcing, over the next very short period of time,
exactly what we're going to be doing.

Okay, a couple of more. Go ahead.

On coronavirus and Joe Biden: He's the presumptive Democratic nominee.
Do you have any plans yet on when you'll start sharing, or when the
White House will start sharing some of that information about the
coronavirus? Your presidential daily brief?

THE PRESIDENT: Well, nobody has called about coronavirus, about --- from
their standpoint. Look, they had the H1N1, which is swine flu, and that
was a big failure. That was a tremendous failure. They had a lot of
failures. And you take a look at what --- you take a look at the
history. And, you know, 17,000 people died. And you talk about late?
They were so late --- they were late like it never even existed. That
was a --- that was a big problem. Caused a lot of other people a big
problem too.

So, you know, if Joe Biden would like a briefing, I'd certainly get him
a briefing. I don't know what he'd do with it.

Yeah, please.

So, are Jared and Ivanka on --- serving on the new task force? And how
are you going to balance ---

THE PRESIDENT: No.

What?

THE PRESIDENT: No, they're not. No.

Okay.

THE PRESIDENT: Yeah. Yes, go ahead, please.

I just want to clarify. So, in earlier conversation, there was a
description of multiple different councils or tasks --- task forces. Can
you just explain exactly what the structure is and who is going to be on
it?

THE PRESIDENT: Well, you have Mike's task force, which is the White
House Task Force, which really brought us up to this point brilliantly,
I must say. Dealt with governors and dealt with governors all the way
through. And I was on many of those calls. And every call got better and
better and better. It was hostile at the beginning. By the time we
finished --- I mean, today's call was a very good call, a very friendly
call. I think everyone is online.

And again, you don't have anybody driving you crazy, saying they're not
getting ventilators, they're not getting all of the different things,
they need more beds. They have a lot of beds right now.

(Inaudible) hospitals who --- doctors who say they don't have the
supplies that they need.

THE PRESIDENT: And we always err --- and I think it's important for you
know --- we always erred on the sake of ``Give them more.'' Even when we
didn't think --- we didn't think New York needed the beds that they were
asking for. We didn't think they needed the ventilators that they asked
for. And we were right.

Now, on ventilators, we're ready to march. I told you this: We're ready
to march. We have 10,000 ventilators. We're ready to move them anywhere
in the country when we need them, if we need them. We're also building a
lot of ventilators, and that's going to be used at some point, I
believe. You know we're going to have stockpiles, including state
stockpiles if they want to work out some kind of an arrangement with us.
But we're also going to help other countries, whether it's Italy, or
Spain, or other --- France is having a big problem. They all desperately
need --- Germany too --- they need ventilators.

So we're going to have a lot of ventilators. We have a lot --- you heard
the numbers --- we have a lot coming next week. Next week, we have a
tremendous amount coming.

Okay, final question. Steve.

And on the task force --- I'd asked for the task --- how the task force
is going to be structured. Is it one ---

THE PRESIDENT: No. Then we have --- in addition to that, we have a
number of committees. We'll have a transportation committee. We're going
to have a manufacturing committee. You'll see it tomorrow.

We're also having a religious leaders committee. We have a great group
of religious leaders. We're having committees with religious leaders.
You've been seeing what's going on with the churches, and all of that.
And we're going to have a faith leaders committee.

And so we have --- we're going to have a few committees. I'll call them
``committees,'' and then ultimately we're going to make decisions. So
we're going to make decisions fairly quickly, and I think they're going
to be the correct decision. I hope so.

Steve.

So you form the economic task force tomorrow. When do you want them to
have recommendations for you?

THE PRESIDENT: Soon. Soon. And they already know what I want.

Next week or ---

THE PRESIDENT: And so, when we form --- when you say ``form,'' I don't
have to give them instructions. These are very sophisticated people.
These are the best people in their fields. So I don't have to say,
``Gee, let's --- we just met and we're going to meet in two weeks and
here's what we're\ldots{}'' I said, ``Here's what I want.'' We've
already told them. And they're the ---

What did you tell them you want?

THE PRESIDENT: --- the best names in the various businesses and
professions and religions. I mean, they're --- these are the greatest
names. The people that, I think, probably know the best.

So, we've called them and we're going to be speaking to them very soon.
And we want them to have --- if it's questions, or statements, we want
them to have that for us. And we will have either a response, or maybe
--- I mean, ideally we're going to be learning from them. And we'll be
able to do that and put them --- put everything we learned from those
calls into our new guidelines.

So we're going to have new guidelines coming soon. I think it's going to
be very good. I think it's going to be very smooth. And I hope it's
going to be very safe.

Thank you all very much. Thank you. Thank you.

Any thoughts on Stanley Chera, sir? Any thoughts on Stanley Chera ---

THE PRESIDENT: Stanley Chera is a friend of mine for a long time. He's a
great real estate person --- passed away. Was a great real estate
person. Great. Great. Sort of a legend in New York real estate.

He called me a couple of weeks ago, said he tested positive. Stanley is
in his early- to mid-80s, I guess. And Stanley went to the hospital and
he never came out. He went into a coma. He was unconscious for a long
period of time and he never made it. A great man. He left --- very
charitable, really a great philanthropist. A very, very successful
person in Manhattan, in the real estate business. So I got to know him a
lot.

He was so excited when his friend from New York became the President of
the United States. He was like --- like a young boy. And he was not a
young boy, but he was like a young boy. He was so excited. He thought
we'd do such a good job and he was so happy. And he --- he was very
proud of what we've done in this administration.

But he was tested positive, and unfortunately he --- he didn't make it.
It's a very --- to me, it's a very sad thing.

Thank you all very much. Thank you.

\href{https://www.whitehouse.gov/briefings-statements/remarks-president-trump-vice-president-pence-members-coronavirus-task-force-press-briefing-25/}{Read
Full Transcript}

\begin{itemize}
\item
\item
\item
\item
\item
\item
\end{itemize}

He has often spouted contempt at the news media, which he says never
gives him the credit he deserves.

He has claimed falsely that he always responded quickly and assertively
when, in fact, he repeatedly played down the seriousness of the virus.

He has also pointed his finger at the governors.

He has portrayed his decision in January to
\href{https://www.nytimes3xbfgragh.onion/2020/04/04/us/coronavirus-china-travel-restrictions.html}{restrict
certain travel from China} --- which he has exaggerated as having saved
``tens of thousands, maybe hundreds of thousands of lives'' --- as a
heroic me-against-the world struggle.

He has blamed other presidents --- but none more than Mr. Obama ---
claiming they left him with an ``obsolete'' and ``broken'' system and
``horrible tests.'' His references to previous administrations include
accusations of neglect. He is selective with the facts and sometimes
wrong. Mr. Trump falsely insisted, for instance, that there were no
ventilators in the federal emergency surplus.

He has tried to sound notes of unity and compassion.

Viewed simply as a pattern of Mr. Trump's speech, the
self-aggrandizement is singular for an American leader. But his approach
is even more extraordinary because he is taking credit and demanding
affirmation while he asks people to look beyond themselves and bear
considerable hardship to help slow the spread of the virus.

``He doesn't speak the language of transcendence, what we have in
common,'' said Jennifer Mercieca, a historian of American political
rhetoric at Texas A\&M University. Instead, Dr. Mercieca said, he falls
back on a vocabulary he developed over decades promoting himself and his
business.

``Trump's primary goal is to spread good news and information and market
the Trump brand: `Trump is great. The Trump brand is great. The Trump
presidency is great,''' she said. ``It's not the right time or place to
do that.''

At 260,000 words and counting, enough to fill a 700-page book, Mr. Trump
has been writing his own history of the virus, one that is favorable to
him, settles scores and is often at odds with the facts. There were at
least 130 examples of falsehoods or exaggerations. He ignored
\href{https://www.nytimes3xbfgragh.onion/2020/03/27/us/politics/trump-coronavirus-factcheck.html}{his
long public record} of making breezy claims about the virus when he said
on March 17, ``I've felt it was a pandemic long before it was called a
pandemic.'' He falsely described the Obama administration's response to
the H1N1 virus, saying on April 6, ``It was like they didn't even know
it was here.''

\hypertarget{speaks-falsely-or-exaggerates}{%
\subsection{Speaks falsely or
exaggerates}\label{speaks-falsely-or-exaggerates}}

\begin{quote}
\hypertarget{when-the-professionals-need-a-test-when-they-need-tests-for-people-they-can-get-the-test-its-gone-really-well--march-10}{%
\subsubsection{\texorpdfstring{``When the professionals need a test,
when they need tests for people, they can get the test. It's gone really
well.''
\href{https://www.whitehouse.gov/briefings-statements/remarks-president-trump-meeting-republican-senators-2/}{»}
\textbf{March
10}}{``When the professionals need a test, when they need tests for people, they can get the test. It's gone really well.'' » March 10}}\label{when-the-professionals-need-a-test-when-they-need-tests-for-people-they-can-get-the-test-its-gone-really-well--march-10}}
\end{quote}

\begin{quote}
\hypertarget{we-have-a-problem-that-a-month-ago-nobody-ever-thought-about--march-16}{%
\subsubsection{\texorpdfstring{``We have a problem that a month ago
nobody ever
\href{https://www.nytimes3xbfgragh.onion/2020/04/11/us/politics/coronavirus-trump-response.html}{thought
about}.''
\href{https://www.whitehouse.gov/briefings-statements/remarks-president-trump-vice-president-pence-members-coronavirus-task-force-press-briefing-3/}{»}
\textbf{March
16}}{``We have a problem that a month ago nobody ever thought about.'' » March 16}}\label{we-have-a-problem-that-a-month-ago-nobody-ever-thought-about--march-16}}
\end{quote}

\begin{quote}
\hypertarget{this-is--when-somebody-is-the-president-of-the-united-states-the-authority-is-total-and-thats-the-way-its-got-to-be--april-13}{%
\subsubsection{\texorpdfstring{``This is --- when somebody is the
president of the United States, the authority is total, and that's the
way it's got to be.''
\href{https://www.whitehouse.gov/briefings-statements/remarks-president-trump-vice-president-pence-members-coronavirus-task-force-press-briefing-25/}{»}
\textbf{April
13}}{``This is --- when somebody is the president of the United States, the authority is total, and that's the way it's got to be.'' » April 13}}\label{this-is--when-somebody-is-the-president-of-the-united-states-the-authority-is-total-and-thats-the-way-its-got-to-be--april-13}}
\end{quote}

There is no precedent for the platform that Mr. Trump commandeered for
himself through much of March and April: a nationally televised
appearance that can go on for up to two and a half hours, seven days a
week, often without interruption. Critics of the president have
questioned why the cable networks
\href{https://www.nytimes3xbfgragh.onion/2020/03/25/business/media/trump-coronavirus-briefings-ratings.html}{continue
to air the briefings}, saying that decision is, in effect, handing over
to the president control of the day's agenda.

``It was thought that presidents were extraordinarily powerful at the
height of the Cold War when they could ask the three networks for 20
minutes of TV time,'' said Michael Beschloss, a presidential historian.
``But as far as a president's being able to exert influence, I think
this is much greater than that.''

If Mr. Trump is always the hero in his version of events, the way he
describes his role has changed over time.

He
\href{https://www.nytimes3xbfgragh.onion/2020/03/22/us/politics/coronavirus-trump-wartime-president.html}{called
himself a ``wartime president''} at first but mostly dropped the label
by late March.

More recently he has often described himself as a commander leading an
enormous undertaking to reconfigure the nation's supply chain to deliver
badly needed medical supplies like testing kits. ``There's never been
anything like it,'' he said on April 1, despite widespread concerns that
not enough testing was being done.

He credited himself with leading a turnaround --- echoing his campaign
promise to ``Make America Great Again'' --- even though there are still
\href{https://www.nytimes3xbfgragh.onion/2020/04/15/us/coronavirus-testing-trump.html}{serious
shortages and deficiencies} in the nation's testing system.

``We had a broken system,'' he said April 14. ``And now we have a great
system.''

The coronavirus briefings have often contained the same phrases and
themes that he used in his 2016 race.

``It's consistent with the way he campaigned when he said, `I alone can
fix it,''' said John Murphy, a professor at the University of Illinois
at Urbana-Champaign who studies the rhetoric of American presidents and
politicians.

Dr. Murphy said that most presidents avoid taking personal credit
because they appreciate the fact that Americans can draw the connection
themselves between presidential leadership and the country's successes.

With Mr. Trump, there is no such subtlety. ``The level of
self-congratulations that occurs every day at these press conferences is
unprecedented,'' Dr. Murphy added.

The president, often criticized as lacking empathy, does occasionally
express it. The Times found about 60 instances in the analysis. His
usage of unifying language was fleeting at first, rarely more than a
terse sentence about the ``tremendous spirit'' of Americans.

\hypertarget{uses-unifying-language-or-attempts-empathy}{%
\subsection{Uses unifying language or attempts
empathy}\label{uses-unifying-language-or-attempts-empathy}}

\begin{quote}
\hypertarget{for-those-of-you-who-are-feeling-alone-and-isolated-i-want-you-to-know-that-we-are-all-joined-together-as-one-people-eternally-linked-by-our-shared-national-spirit--we-love-our-country--a-spirit-of-courage-and-love-and-patriotism--march-22}{%
\subsubsection{\texorpdfstring{``For those of you who are feeling alone
and isolated, I want you to know that we are all joined together as one
people, eternally linked by our shared national spirit --- we love our
country --- a spirit of courage and love and patriotism.''
\href{https://www.whitehouse.gov/briefings-statements/remarks-president-trump-vice-president-pence-members-coronavirus-task-force-press-briefing-8/}{»}
\textbf{March
22}}{``For those of you who are feeling alone and isolated, I want you to know that we are all joined together as one people, eternally linked by our shared national spirit --- we love our country --- a spirit of courage and love and patriotism.'' » March 22}}\label{for-those-of-you-who-are-feeling-alone-and-isolated-i-want-you-to-know-that-we-are-all-joined-together-as-one-people-eternally-linked-by-our-shared-national-spirit--we-love-our-country--a-spirit-of-courage-and-love-and-patriotism--march-22}}
\end{quote}

\begin{quote}
\hypertarget{i-want-to-start-by-saying-that-our-hearts-go-out-to-the-people-of-new-york-as-they-bear-the-brunt-of-the-coronavirus-pandemic-in-america--april-3}{%
\subsubsection{\texorpdfstring{``I want to start by saying that our
hearts go out to the people of New York as they bear the brunt of the
coronavirus pandemic in America.''
\href{https://www.whitehouse.gov/briefings-statements/remarks-president-trump-vice-president-pence-members-coronavirus-task-force-press-briefing-18/}{»}
\textbf{April
3}}{``I want to start by saying that our hearts go out to the people of New York as they bear the brunt of the coronavirus pandemic in America.'' » April 3}}\label{i-want-to-start-by-saying-that-our-hearts-go-out-to-the-people-of-new-york-as-they-bear-the-brunt-of-the-coronavirus-pandemic-in-america--april-3}}
\end{quote}

\begin{quote}
\hypertarget{you-have-people-that-have-never-asked-for-business-interruption-insurance-and-theyve-been-paying-a-lot-of-money-for-a-lot-of-years-for-the-privilege-of-having-it-and-then-when-they-finally-need-it-the-insurance-company-says-were-not-going-to-give-it-we-cant-let-that-happen--april-10}{%
\subsubsection{\texorpdfstring{``You have people that have never asked
for business interruption insurance, and they've been paying a lot of
money for a lot of years for the privilege of having it. And then when
they finally need it, the insurance company says, ``We're not going to
give it.'' We can't let that happen.''
\href{https://www.whitehouse.gov/briefings-statements/remarks-president-trump-vice-president-pence-members-coronavirus-task-force-press-briefing-24/}{»}
\textbf{April
10}}{``You have people that have never asked for business interruption insurance, and they've been paying a lot of money for a lot of years for the privilege of having it. And then when they finally need it, the insurance company says, ``We're not going to give it.'' We can't let that happen.'' » April 10}}\label{you-have-people-that-have-never-asked-for-business-interruption-insurance-and-theyve-been-paying-a-lot-of-money-for-a-lot-of-years-for-the-privilege-of-having-it-and-then-when-they-finally-need-it-the-insurance-company-says-were-not-going-to-give-it-we-cant-let-that-happen--april-10}}
\end{quote}

His praise of health care workers who put themselves at risk every day
--- ``these are our warriors'' --- and his admiration for the resilience
of the American people have become somewhat more common, especially in
his prepared remarks at the opening of each briefing.

``We've marshaled every instrument of American power, and we've
unleashed our most potent weapon of all: the courage of the American
people,'' he said on April 16.

At some briefings, Mr. Trump has tried to rally the country, saying the
United States will emerge stronger. He often expresses confidence that
the ravaged country will bounce back quickly --- ``like a burst of
light.'' And at perhaps his most effective briefing, he soberly braced
the country for two weeks ahead that would be ``painful,'' as doctors
predicted a large number of deaths.

Once in a while, the deadliness of the virus seems to weigh on Mr.
Trump. He spoke of a friend on March 31 who came down with an especially
bad case and was in a coma. ``Sort of a tough guy, a little older, a
little heavier than he'd like to be,'' the president said. ``It's not
the flu. It's vicious.'' And he has spoken of seeing the images of
Elmhurst Hospital Center, miles from where he grew up in Queens,
\href{https://www.nytimes3xbfgragh.onion/2020/03/25/nyregion/nyc-coronavirus-hospitals.html}{overwhelmed
by coronavirus patients}.

But his laments about the virus's economic toll --- the damage it has
caused ``probably the best economy in the history of the world'' --- are
far more common than remarks about the human toll.

``It's the things that are not there, the things he isn't doing,'' said
Roderick P. Hart, a professor at the University of Texas at Austin who
is an expert in political speech. ``It's what's not there --- that sense
of, `I'm part of the human condition,' the ability to empathize with the
downtrodden and the afflicted --- that's what's so important.''

Isabella Grullón Paz contributed reporting. Produced by Gray Beltran,
Larry Buchanan, Aaron Byrd and Josh Williams.

Read 1437 Comments

\begin{itemize}
\item
\item
\item
\item
\end{itemize}

Advertisement

\protect\hyperlink{after-bottom}{Continue reading the main story}

\hypertarget{site-index}{%
\subsection{Site Index}\label{site-index}}

\hypertarget{site-information-navigation}{%
\subsection{Site Information
Navigation}\label{site-information-navigation}}

\begin{itemize}
\tightlist
\item
  \href{https://help.nytimes3xbfgragh.onion/hc/en-us/articles/115014792127-Copyright-notice}{©~2020~The
  New York Times Company}
\end{itemize}

\begin{itemize}
\tightlist
\item
  \href{https://www.nytco.com/}{NYTCo}
\item
  \href{https://help.nytimes3xbfgragh.onion/hc/en-us/articles/115015385887-Contact-Us}{Contact
  Us}
\item
  \href{https://www.nytco.com/careers/}{Work with us}
\item
  \href{https://nytmediakit.com/}{Advertise}
\item
  \href{http://www.tbrandstudio.com/}{T Brand Studio}
\item
  \href{https://www.nytimes3xbfgragh.onion/privacy/cookie-policy\#how-do-i-manage-trackers}{Your
  Ad Choices}
\item
  \href{https://www.nytimes3xbfgragh.onion/privacy}{Privacy}
\item
  \href{https://help.nytimes3xbfgragh.onion/hc/en-us/articles/115014893428-Terms-of-service}{Terms
  of Service}
\item
  \href{https://help.nytimes3xbfgragh.onion/hc/en-us/articles/115014893968-Terms-of-sale}{Terms
  of Sale}
\item
  \href{https://spiderbites.nytimes3xbfgragh.onion}{Site Map}
\item
  \href{https://help.nytimes3xbfgragh.onion/hc/en-us}{Help}
\item
  \href{https://www.nytimes3xbfgragh.onion/subscription?campaignId=37WXW}{Subscriptions}
\end{itemize}
