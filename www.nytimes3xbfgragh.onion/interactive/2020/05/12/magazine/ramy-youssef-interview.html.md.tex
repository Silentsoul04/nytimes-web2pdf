Sections

SEARCH

\protect\hyperlink{site-content}{Skip to
content}\protect\hyperlink{site-index}{Skip to site index}

\hypertarget{comments}{%
\subsection{\texorpdfstring{\protect\hyperlink{commentsContainer}{Comments}}{Comments}}\label{comments}}

\href{}{Ramy Youssef Is Not Using Comedy to Teach You About
Muslims}\href{}{Skip to Comments}

The comments section is closed. To submit a letter to the editor for
publication, write to
\href{mailto:letters@NYTimes.com}{\nolinkurl{letters@NYTimes.com}}.

Talk

\hypertarget{ramy-youssef-is-not-using-comedy-to-teach-you-about-muslims}{%
\section{Ramy Youssef Is Not Using Comedy to Teach You About
Muslims}\label{ramy-youssef-is-not-using-comedy-to-teach-you-about-muslims}}

By David MarcheseMay 12, 2020

\begin{itemize}
\item
\item
\item
\item
\item
  \emph{9}
\end{itemize}

``My goal isn't to check the boxes of the intelligentsia.''

\includegraphics{https://static01.graylady3jvrrxbe.onion/newsgraphics/2020/04/19/talk/c6cbcfefcb74de853cb758cd01b1f7b5254d2860/close.svg}

\textbf{Talk} May 12, 2020

\hypertarget{ramy-youssef-is-not-using-comedy-to-teach-you-about-muslims-1}{%
\section{Ramy Youssef Is Not Using Comedy to Teach You About
Muslims}\label{ramy-youssef-is-not-using-comedy-to-teach-you-about-muslims-1}}

By David Marchese

SHARE

When Ramy Youssef was growing up in New Jersey, he took for granted that
he and his Egyptian Muslim family weren't going to see a TV show about
people like themselves. ``We'd watch something,'' Youssef recalled,
``and emotionally connect to characters, but it still wasn't us, you
know? It was \emph{them}.'' With ``Ramy,'' the whimsically subversive
hit Hulu show that Youssef, 29, helped create and in which he stars ---
it returns for a second season on May 29 --- he made the series he
didn't get to see back then. Albeit one that's far more sexually frank,
surreal and spiritually searching than his (or just about anyone's)
family might have expected. ``I don't want to try to teach anybody
anything,'' Youssef said. ``I don't want to provide answers. I just want
the show to be thought-provoking.''

\textbf{Your show and your stand-up are so much about the conflicting
feelings you have about faith. We might expect those feelings to arise
in response to struggle. I'm curious whether} \emph{\textbf{success}}
\textbf{presented any spiritual challenges?} Making the show is a
spiritual challenge. There's so much tension around being Muslim, and I
don't want to abuse that tension. Instead it's about, maybe if I talk
about this thing on my show, then some families can have a conversation
that they wouldn't have had without watching. It hasn't been this thing
where it's like: \emph{Man,}
\emph{\href{http://nytimes3xbfgragh.onion\#tooltip-1}{I won a Golden
Globe,}} \emph{and then I was at this party, and maybe I want to do coke
now}.

\textbf{Obviously for any show to be engaging it has to find ways to
create tension. But the depictions of}
\textbf{\href{http://nytimes3xbfgragh.onion\#tooltip-2}{Ramy's mom and
sister}} \textbf{and, without giving too much away about the new season,
his uncle don't exactly shy away from the stereotype of Muslims as
sexually repressed. So how do you think about addressing the dramatic
needs of a TV show when those same needs might be nudging you toward
cultural clichés?} My job is to make something that's grounded in
reality. A lot of people want what I would call sci-fi; people want
television to be what they think the world \emph{should} look like. That
is different than what I do. Am I going to create something where it's
like, this is an openly gay Muslim dude who's just chilling? No. My goal
isn't to check the boxes of the intelligentsia. I'm interested in
creating a real, grounded chapter for a character as opposed to tying
everything up neatly at the end of 25 minutes. And I don't think that we
frame that episode around him being repressed because he's Muslim. What
he's going through is not unique to Muslims. It's been cool sending that
episode to people in the L.G.B.T. community who have watched it and are,
like, `Man, I've dated that guy,' or `Man, I've been that guy.'

\textbf{You don't think that on some level ``Ramy'' --- in trying to be
nuanced or even just in terms of representation --- is an attempt to
show what you think the world should look like?} My show is a filter on
what exists. It's like, here's a private look. You watch a lot of shows,
especially sitcoms, and they're like, oh, cool, we have a friend group
--- let's make sure we have one of every race in it. Yeah, sometimes
that's the group I'm in --- maybe if it's a birthday party --- but if
I'm really hanging out with my friends back in Jersey, I'm hanging out
with my Arab guys. A lot of people put weight on TV to project a
different reality. I don't think that we should have to, because then
I'd be tasked with representing every single Muslim, which is
impossible. The show could never give a complete picture. And it never
will.

\textbf{You said that your hope is for ``Ramy'' to start conversations.
Did it do that for you and your family?} We've been able to talk more
openly about sex. People don't even know how to define things around it,
and that comes from lack of transparency. It comes from a form of
repression. I felt like my comedy could talk about spirituality and
sexuality and in a way that I could do uniquely. If you want a show to
talk about politics or the Muslim ban or whatever --- someone should
make that show. That's not what I'm interested in.

\textbf{Well, as far as repression, I'd be basically mortified to have
conversations with my family about sex. How'd that go for you?} It was
just like a baseline acknowledging of the subject. I used to riff
onstage that when you're Muslim, it's hard to come out as straight. But
the thing I'm talking about now was more about having my parents watch
the show and then being like, oh, wow, they still love me. I didn't know
if they would. So, it wasn't an explicit conversation. It just allowed
us to sit in a more well-rounded reality. But this thing about
conversations that wouldn't happen --- the thing I love about my job is
that I'm allowed to be wrong. I'm a clown. I'm a comedian. Comedy, by
design, is a place to explore the subconscious. The subconscious is not
appropriate. The subconscious is not politically correct. Comedians rail
on political correctness, but it's actually good for comedy in many ways
when it can help you not be reductive, and I don't think there's
anything in my show that's reductive. It might be uncomfortable, but
that's what's going on in my subconscious. A clown is supposed to be
messy.

\textbf{In that regard, do you think there's been a shift in our
expectations of what comedians are supposed to offer?} It's like there's
this weird role reversal in what society wants. They want comedians to
have every talking point be \emph{on} point and every piece of art
equally represent this and that. That's what politicians should do ---
and now our politicians are a joke. I call this the ``Daily Show''
effect. My generation grew up on ``The Daily Show.'' Jon Stewart gave me
solace. He was the only person who I felt defended Muslims when crazy
things were happening; the only place you could find a nuanced take on
Iraq was Jon Stewart's show. He had a massive impact on comedy, because
people started feeling like comedy has to be where you get your news and
your truth. But delivering that just happened to be hiss comedic brand.
That's what \emph{he} did. That doesn't mean that's the role of comedy.
People love to say: ``I get my news from `The Daily Show.''' Cool, but
don't get it from me.

\textbf{Maybe this is an example of taking comedy too seriously, but
there's a bit in}
\textbf{\href{http://nytimes3xbfgragh.onion\#tooltip-3}{your special}}
\textbf{that I keep thinking about. You have this moment where you're
talking about Sept. 11, and you're wondering if the fallout from that
day made America weaker --- because of how it helped lead to Trumpism
--- and made Islam stronger, because people like you started to think
more deeply about your connection to the religion. I'm sure you're right
about this in a lot of ways, but doesn't your line of thinking
presuppose affinities between radical Islam and the Islam that you
practice? Does this question make any sense? Probably it's just that
your joke made me uncomfortable, and I'm trying to figure out why.} I
put a lot of emphasis in that joke on the reaction to what was said
following 9/11. The joke supposes the framing that the terrorists were
doing it from a place of being Muslim, which they weren't. It's almost
like, let me use that faulty logic, you know what I mean? I don't buy
the media framing of 9/11. That's not to be a conspiracy theorist. It's
just to say that this idea of East versus West is so much deeper than
any one act. It's deeper than religion. We're talking about ripple
effects from colonialism and total economic torture. So, no, I don't
believe that joke's framing; the joke is more about, if that \emph{is}
the framing you subscribe to, then do you not see how you've done some
weird circle where maybe 9/11 worked?

\textbf{Either way, the important thing to know is that the}
\textbf{\href{http://nytimes3xbfgragh.onion\#tooltip-4}{ice caps are
melting.}} **** They're melting! That's actually what I'm saying. It's
like, let's talk about the planet. Not to sound like Bernie Sanders, but
almost every conversation about religion could lead to a rant about the
1 percent. That's kind of what's amazing about a period like this one,
during a pandemic, where we all experience something at the same time.
You start to be like, wait, we don't have enough ventilators, but
there's more Walmarts than I can count? \emph{Really}? You start to see
how much of this is like all a {[}expletive{]} joke.

\textbf{Does your faith affect how you think about the pandemic?} I know
I have solace in spiritual connection. What a moment like this does is
make your brain so loud. You could read every article. You could listen
to every podcast. So in my spiritual practice it's like, how do I get
quiet? How do I get to a place where I can just turn that off and have
faith? You know, it's funny because so many of my closest friends are
comics who don't believe in God the way I do. They'll say it's
illogical. A lot of things are illogical! We're dealing with a virus
right now that completely turned the world around in a week, and we're
being led by a reality-TV-show star. So why couldn't Moses part the sea?
You're telling me it's that big of a jump?

\textbf{You've talked in interviews before about meeting}
\textbf{\href{http://nytimes3xbfgragh.onion\#tooltip-5}{Jerrod
Carmichael}} \textbf{at a party and having a conversation with him about
the way having religious faith made you both feel like outsiders. I'm
wondering if you might be able to parse that for me a little more. This
is not exactly an irreligious country. So is your feeling of being an
outsider specifically about being Muslim? Is it about being a believer
within the Hollywood community? Where do you most feel the rub?} There
\emph{are} many people who do believe in God in this country, but
there's also this idea that believing in God is seen as stupid and goes
against liberal values; that believing in God is antiprogressive. That's
the tension. That's the thing where sometimes you feel like when you say
you believe, you should be doing it in a speakeasy. One of the jokes I
did on \href{http://nytimes3xbfgragh.onion\#tooltip-6}{``Colbert''} was
about how I grew up in a town where gay people had to come out to their
religious community, and now I'm a religious person coming out to the
gay community. But there's so much in the values of the faiths that have
lasted that, if followed with an open heart, is totally in line with
being progressive. I do get why people are bothered by religion. You
have people whose religion says ``don't judge'' telling other people
that they can't get married. So there are people who say they're
practicing religion who are not quite practicing it. Everything's a
contradiction.

\textbf{You mentioned the Golden Globes earlier. The response to your}
\textbf{\href{http://nytimes3xbfgragh.onion\#tooltip-7}{acceptance
speech}} \textbf{was so encouraging and lovely. What stands out to you
from that night?} I actually knew my name was going to be called right
before it was. There was this moment when Jennifer Aniston was looking
at the card, and I saw the confusion on her face. There was this frozen
two seconds. I was like: It's my name, and she doesn't know how to say
it.

\textbf{Which I'm sure you've experienced before.} She was like every
substitute teacher I've ever had. But before that I so was not thinking
that I was going to win. I was just like, if I do, all I know is I'm
going to say \emph{Allahu akbar} because that's all I would feel in my
heart.

\textbf{I know you've been working on new stand-up material. What kind
of stuff are you thinking about?} I've been talking a bunch about going
to therapy. I wrote this joke about coming out to my family about seeing
a therapist. My dad was like, I can't believe you're seeing a therapist.
We thought you believed in God! And I'm like, I do. I believe in God.
But God can't be your therapist --- a therapist has to be Jewish.

\textbf{That's actually a rule. Not everyone knows.} You can't see a
not-Jewish therapist. That wouldn't work. You ever try to talk to a
therapist who's not Jewish? It's impossible.

\begin{center}\rule{0.5\linewidth}{\linethickness}\end{center}

David Marchese is a staff writer and the Talk columnist for the
magazine.

Opening illustration: Source photograph by Kirk McKoy/Los Angeles Times,
via Contour by Getty Images

\emph{This interview has been edited and condensed from two
conversations.}

\textbf{Correction:}~May 13, 2020

An earlier version of this~article~omitted part of Ramy Youssef's
response to a question about the framing of Muslims and sexuality on an
episode of his hit show ``Ramy,'' and gave an incomplete reflection of
his thoughts on the show. In addition to saying that he was ``interested
in creating a real, grounded chapter for a character,'' he said, ``And I
don't think that we frame that episode around him being repressed
because he's Muslim.~What he's going through is not unique to Muslims.
It's been cool sending that episode to people in the L.G.B.T. community
who have watched it and are, like, `Man, I've dated that guy,' or `Man,
I've been that guy.'~"

Read 9 Comments

\begin{itemize}
\item
\item
\item
\item
\end{itemize}

Advertisement

\protect\hyperlink{after-bottom}{Continue reading the main story}

\hypertarget{site-index}{%
\subsection{Site Index}\label{site-index}}

\hypertarget{site-information-navigation}{%
\subsection{Site Information
Navigation}\label{site-information-navigation}}

\begin{itemize}
\tightlist
\item
  \href{https://help.nytimes3xbfgragh.onion/hc/en-us/articles/115014792127-Copyright-notice}{©~2020~The
  New York Times Company}
\end{itemize}

\begin{itemize}
\tightlist
\item
  \href{https://www.nytco.com/}{NYTCo}
\item
  \href{https://help.nytimes3xbfgragh.onion/hc/en-us/articles/115015385887-Contact-Us}{Contact
  Us}
\item
  \href{https://www.nytco.com/careers/}{Work with us}
\item
  \href{https://nytmediakit.com/}{Advertise}
\item
  \href{http://www.tbrandstudio.com/}{T Brand Studio}
\item
  \href{https://www.nytimes3xbfgragh.onion/privacy/cookie-policy\#how-do-i-manage-trackers}{Your
  Ad Choices}
\item
  \href{https://www.nytimes3xbfgragh.onion/privacy}{Privacy}
\item
  \href{https://help.nytimes3xbfgragh.onion/hc/en-us/articles/115014893428-Terms-of-service}{Terms
  of Service}
\item
  \href{https://help.nytimes3xbfgragh.onion/hc/en-us/articles/115014893968-Terms-of-sale}{Terms
  of Sale}
\item
  \href{https://spiderbites.nytimes3xbfgragh.onion}{Site Map}
\item
  \href{https://help.nytimes3xbfgragh.onion/hc/en-us}{Help}
\item
  \href{https://www.nytimes3xbfgragh.onion/subscription?campaignId=37WXW}{Subscriptions}
\end{itemize}
