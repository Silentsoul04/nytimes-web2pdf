Sections

SEARCH

\protect\hyperlink{site-content}{Skip to
content}\protect\hyperlink{site-index}{Skip to site index}

\hypertarget{comments}{%
\subsection{\texorpdfstring{\protect\hyperlink{commentsContainer}{Comments}}{Comments}}\label{comments}}

\href{}{Visiting the Museum in My Mind}\href{}{Skip to Comments}

The comments section is closed. To submit a letter to the editor for
publication, write to
\href{mailto:letters@NYTimes.com}{\nolinkurl{letters@NYTimes.com}}.

\hypertarget{visiting-the-museum-in-my-mind}{%
\section{Visiting the Museum in My
Mind}\label{visiting-the-museum-in-my-mind}}

By Hisham MatarMay 15, 2020

\begin{itemize}
\item
\item
\item
\item
\item
  \emph{3}
\end{itemize}

\includegraphics{https://static01.graylady3jvrrxbe.onion/images/2020/05/26/magazine/26mag-quarantine-ruge/26mag-quarantine-ruge-master495.png}

\includegraphics{https://static01.graylady3jvrrxbe.onion/images/2020/05/24/magazine/24mag-quarantine-ruge-03/24mag-quarantine-ruge-03-master495.png}

``The Painter's Daughters Chasing a Butterfly'' by Thomas
GainsboroughNational Gallery, London/ Art Resource, NY

\includegraphics{https://static01.graylady3jvrrxbe.onion/images/2020/05/24/magazine/24mag-quarantine-ruge-03-02/24mag-quarantine-ruge-03-02-master495.png}

``With My Head Hanging Down Before the Parachute Opened,'' by Willi
RugeWilli Ruge/Ullstein Bild, via Getty Images

\includegraphics{https://static01.graylady3jvrrxbe.onion/images/2020/05/26/magazine/26mag-quarantine-ruge/26mag-quarantine-ruge-master495.png}

\hypertarget{something-happens-when-you-fall-two-artworks-that-ask-the-question-what-world-will-we-find-on-the-other-side-of-this}{%
\subsection{\texorpdfstring{Something Happens When You Fall \emph{Two
artworks that ask the question:} \emph{What world will we find on the
other side of
this?}}{Something Happens When You Fall Two artworks that ask the question: What world will we find on the other side of this?}}\label{something-happens-when-you-fall-two-artworks-that-ask-the-question-what-world-will-we-find-on-the-other-side-of-this}}

By Hisham Matar

Shortly after the coronavirus crisis set in, my friend Caroline
Campbell, a curator at London's National Gallery, wrote to tell me about
how, a couple of days after the museum was forced to close its doors to
the public, she found herself wandering through the empty galleries. She
had never known the place to be so quiet nor the birdsong outside so
loud. ``The blinds and shutters were down,'' she wrote. ``The pictures
hung on the walls in heavy silence.'' But what unsettled her most was
how the new atmosphere seemed to alter her relationship to the pictures.
``On this oddest of days,'' she said, ``I was drawn to a picture that
has never interested me greatly. I know Gainsborough's portrait of his
two daughters chasing a butterfly means a lot to many people, but it
hasn't done much for me in the past. It has always seemed schmaltzy and
sentimental. But now I was utterly compelled by it.''

Thomas Gainsborough's ``The Painter's Daughters Chasing a Butterfly,''
from about 1756, is a popular picture, one of those that are easy to
dismiss. ``I began to think,'' Caroline wrote, ``that our past life has
maybe been as futile as chasing butterflies. I also felt, in a weird
way, that Gainsborough's children were trapped, stuck within a frame on
the wall of the gallery, and only our gaze could set them free. What I'm
trying to say, I guess, is that pictures only function when there are
people to look at them. Otherwise, they hang somewhere in limbo,
purposeless and a little lost.''

Like Caroline, I had never paid much attention to ``The Painter's
Daughters.'' I have many times passed it by on my way to another
picture. I looked it up online. There is a strange and wary confidence
to the sisters. The butterfly is precarious, pushed close to the edge of
the painting, and seems to be both free and in mortal danger. The
sisters are holding hands but are far apart enough that they seem
connected yet distinct. It is not clear if the older one is readying
herself to strike. In any event, she knows more than the other and is,
at the moment we catch them, willing to let her younger sister test the
limits a little. Theirs is the murderous curiosity of the innocent. The
butterfly is balanced on what looks like a thorny stalk. The little
sister's hand is already extended halfway there. Her delicate fingers
are dark with intent. If she misses, though, the thorns will be what she
catches. Her face is ignorant of this. The older sister, however, has an
inkling and seems to be both encouraging and restraining her sister.
Part of her purpose seems to be her own amusement. And suddenly, looking
at them together --- the shapes of their figures, the activity of their
dresses --- they resemble the wings of a large butterfly in motion. All
around them is darkness. The trees are pregnant with dark premonitions,
and the small light in the distant sky is weak and dying.

\includegraphics{https://static01.graylady3jvrrxbe.onion/images/2020/05/24/magazine/24mag-quarantine-ruge-03/24mag-quarantine-ruge-03-master495.png}

``The Painter's Daughters Chasing a Butterfly'' by Thomas
GainsboroughNational Gallery, London/ Art Resource, NY

It is a painting about the turning of time, about youth and aging, about
the violence of the innocent and about a father's concern, his fears of
mysterious and real dangers.

\includegraphics{https://static01.graylady3jvrrxbe.onion/images/2020/05/24/magazine/24mag-quarantine-ruge-03-02/24mag-quarantine-ruge-03-02-master495.png}

``With My Head Hanging Down Before the Parachute Opened,'' by Willi
RugeWilli Ruge/Ullstein Bild, via Getty Images

\textbf{In the days after} my exchange with Caroline, further
restrictions were imposed. I began to miss my old routine. For a
majority of my life, I have visited the museum about once a week to
stand in front of a picture for half an hour or so. In London, where I
live for most of the year, I usually go to the National Gallery; in New
York, where I live for the rest of the time, it is the Metropolitan
Museum of Art. Without knowing it, I have, in modest ways, come to
organize my week around this habit. But this, of course, was the least
of it; the coronavirus has so quickly succeeded in affecting almost
every aspect of life. It has encouraged us to be suspicious of one
another. It has exposed the inequities in our society. Its effects are
mental as well as physical, in the sense that it threatens to attack
that most essential of human rights, breathing, and compel our culture
to think that the pandemic is the only thing worth thinking about. As
you would under a dictatorship, you must protect your body as well as
your imagination from it. I declined invitations to write about it and
tried instead to focus on my work. I felt I was turning upon myself.
Time itself has come to resemble a loop.

Yet somehow, whenever I longed for a return to normality, to take the
bus and visit the National Gallery, the most vivid image I had wasn't
from my memory. Instead, I saw myself in Caroline's shoes, walking
through the empty museum to look at the Gainsborough. It was as though,
through a strange mental confluence, no doubt exaggerated by the
situation we are all in, I had swapped places with Caroline. I thought
of something she and I discussed before, in the days when we were able
to meet in person, about how a painting is never finished, that it must
continue to do its work long after it has been hung on the gallery wall,
that a picture relies on us to complete it. This is why, we agreed
passionately, free access to museums is essential. Our culture depends
on it, but also the art depends on it. Now that this age-old
communication was not possible, how can those paintings in the museum,
hanging in the dark, function? The question presented itself with the
nuts and bolts of a practical problem. When the art in the museum is
inaccessible, what happens to it and what happens to us? Have we not
observed how a picture can suddenly become significant to our culture
while another, equally good, perhaps even better, recedes from our
attention? And does this not prove that art is in constant dialogue with
history?

It was in this state of mournful confusion that a set of photographs I
saw years ago --- and for all intents and purposes had forgotten about
--- emerged, like a found object, in my memory. In certain moments,
usually when I wasn't thinking, the images would occur to me. They are
by a little-known German photographer named Willi Ruge (1892-1961). ``I
Photograph Myself During a Parachute Jump'' is a series shot as Ruge
dived off an aircraft headfirst, like a rocket to the ground. The
pictures were taken in 1931, between the two mammoth wars, and therefore
a time, I imagine, like those trees in Gainsborough's painting, pregnant
with dark premonitions.

Ruge was interested in speed. As a boy he wanted to be a pilot. He
sharpened his photographic skills during World War I, when he served at
the front and as an air gunner. It was then that the connection was
reinforced in his mind between photography and danger. He later traveled
to Argentina, where he photographed bodies mingled in tango and took
several disturbing shots as he turned around the muscular tug and pull
of a wild horse some gauchos had chained to a tree under the blazing
midday sun in order to tame it. He took aerial views of trees, each
standing beside its shadow, sparsely dotting a small village in Sudan.
He captured, in startlingly perfect focus, a rider falling off his horse
at a steeplechase race in 1920s England. He took a haunting portrait of
a Royal Air Force pilot, Lt. Col. Frederick F. Minchin, a few months
before he set off on an attempt to cross the Atlantic. The aircraft
disappeared. The pilot's fate remains a mystery. The photograph is
uncanny because Minchin appears to be mesmerized, peering fixedly at
something just out of reach, as though recognizing his destiny and yet
unable to escape it. Ruge caught a similar entranced quality in Rudolf
Caracciola, the top racing driver of the age. He captured
speed-distorted images of him whizzing by in his car and took quieter
photographs of his wife as she leaned over, watching her husband with
trembling pride and fear in her face.

In 1936, well into the Nazi years, Ruge published a series he titled
``In the Air With Our Combat Pilots,'' which included photographs that
lionized the German Air Force. A few years later, during World War II,
he more nakedly used his camera as a tool for propaganda: He
photographed military personnel in strategic meetings, captured machine
gunners and pilots in action and documented the optimistic faces of
young recruits. Ruge seemed to have accepted Nazi rule and certainly
supported it with his work. He, like other German photographers then,
commercially benefited from the exile, incarceration and murder of able
colleagues. And therefore, his work after 1933 exists in a troubling
vacuum, one marked by the enforced disappearance of others, and his work
before that time is touched by a powerful foreboding, as though he
sensed the nearness of the precipice and was darkly drawn to it.

One such example is ``I Photograph Myself During a Parachute Jump.''
Here Ruge becomes the protagonist of his own work. He wanted to capture
the drama from several angles. He, the falling figure, had a camera; a
colleague of his photographed from another aircraft that hovered at a
similar altitude; and a second was on the ground and caught, as Ruge had
done with the Caracciolas, Mrs. Ruge's reaction as she watched her
husband descend from the sky. In ``Photo of Myself at the Moment of My
Jump,'' Ruge appears to be falling backward, his mouth wind-swollen and
horribly open. In ``I Decided to Jump Headfirst,'' taken from the other
aircraft, we see Ruge from a distance. He is upside down, and his arms
and legs are spread wide as he swims in the air a few meters beneath his
aircraft, the faraway earth flat and placid beneath. Then in
``Meanwhile, on the Ground \ldots{} '' we see his wife holding their
newborn son and looking up at the sky, surrounded by other amazed
onlookers; one of them, a lady in an open coat, is literally biting her
fingers. A few seconds later, the panic has set in for Ruge. He takes a
picture that he later calls ``With My Head Hanging Down Before the
Parachute Opened \ldots{} .'' It shows his face from beneath, his
features distorted by fear and effort, and his right hand is reaching to
clinch on to something. The gesture is uncannily reminiscent of
Gainsborough's younger daughter's hand reaching for the butterfly.

\textbf{In the same year} that Ruge dived off the aircraft with a
camera, Virginia Woolf wrote, in ``A Letter to a Young Poet'': ``But it
is October 1931, and for a long time now poetry has shirked contact with
--- what shall we call it? --- Shall we shortly and no doubt
inaccurately call it life? And will you come to my help by guessing what
I mean?'' It is a curious coincidence that just as Ruge was making ``I
Photograph Myself During a Parachute Jump,'' which is, in part at least,
about contact and lack of contact, Woolf was thinking of the same
problem in language and, more specific, poetry's ability to capture life
in the modern age. ``And will you come to my help by guessing what I
mean?'' is surely every work of art's silent plea. Indeed, Ruge's
parachuting self does remind me of the poet that Woolf evokes in her
``Letter,'' dancing ``on the floor of {[}his{]} mind.'' Here are his
feet, in lace-up leather shoes, dangling helplessly above a cityscape.
The architecture, trees and road below seem like a distant dream, a
child's fantasy of what a city might look like. Like the rest of the
photographs in this series, it is about proximity, but also
approximation: how to measure the distance between two points, and what
it is to cross it. They are about being suspended, and what that state
might hold. They reveal a playful but also anxious response to our
relationship to earth, our resistance to nature and the fantasy, which
is also a madness, of wanting to be weaned from it.

I am still surprised that the work of Willi Ruge, which has never before
seriously occupied my attention, should suddenly come so vividly to mind
during these strange and uncertain days. Not only are we uncertain about
how and when we will come out of this; we are also unsure about the
world we will find on the other side. We feel oddly convinced that
nothing will be the same. It is as if the present has woken us up from a
dream. And yet, very much like the painter's daughters and the falling
Ruge, we feel that we are involved in some sort of experiment in living.
The moment is charged with opposing temptations. Will we rethink our
attitudes to nature and to commerce in order to address the health of
our planet and the violent inequities in our society? Will we remember
the compelling force with which we have been reminded of how deeply
interconnected and reliant we are on one another? That we are not only
part of nature but also part of a single organism? Will we honor the
passion we felt for kindness? Will we, in other words, stay true to the
version we glimpsed of our better selves: that of generosity, heroism
and compassion? Or will we yield to the dark calls of those who insist
that the lessons to take from this are that we should fear the stranger
even more, build higher walls and practice a resolute policy of
intolerance to difference and to reason? Our history shows that we are
paradoxically good at learning from the past as well as remaining
somewhat unchanged by it.

Perhaps Ruge's images returned to me because, as with my friend
Caroline, the current circumstances have altered my attention. I think
this is one of the ways that art works; it is there for our pleasure but
also as a tool for thinking. It helps us consider the present and our
place in it. It is what has made Caroline stop at the Gainsborough, and
it is also what has made me recall images by a largely forgotten
photographer whose work I had seen only in passing several years ago and
in reproduction. And from this perspective, it suddenly seems that ``The
Painter's Daughters Chasing a Butterfly'' and ``I Photograph Myself
During a Parachute Jump'' are connected. I think it was the wariness and
suspended state of danger and threat in the Gainsborough that helped
bring to mind Ruge's photographs. Both are about a suspended moment,
where the consequences are in question; a moment, in other words, not
too dissimilar from where we find ourselves today.

\emph{Illustration by Brian Rea.}

Hisham Matar \emph{is the author of the novels ``In the Country of Men''
and ``Anatomy of a Disappearance'' and the memoirs ``The Return'' and
``A Month in Siena.'' He has won numerous prizes, including a Pulitzer.
This is his first article for the magazine.} Brian Rea \emph{is an
illustrator in Los Angeles. His book ``Death Wins a Goldfish'' was
published last year.}

\hypertarget{untitled-series}{%
\subsection{Untitled series}\label{untitled-series}}

\hypertarget{photographs-2020}{%
\subparagraph{Photographs, 2020}\label{photographs-2020}}

\hypertarget{by-jack-davison}{%
\subparagraph{By Jack Davison}\label{by-jack-davison}}

\includegraphics{https://static01.graylady3jvrrxbe.onion/images/2020/05/24/magazine/24mag-quarantine-davison-07/24mag-quarantine-davison-07-master495.jpg}

\includegraphics{https://static01.graylady3jvrrxbe.onion/images/2020/05/24/magazine/24mag-quarantine-davison-03/24mag-quarantine-davison-03-master495.jpg}

\includegraphics{https://static01.graylady3jvrrxbe.onion/images/2020/05/24/magazine/24mag-quarantine-davison-06/24mag-quarantine-davison-06-master495.jpg}

\includegraphics{https://static01.graylady3jvrrxbe.onion/images/2020/05/24/magazine/24mag-quarantine-davison/24mag-quarantine-davison-master495.jpg}

\includegraphics{https://static01.graylady3jvrrxbe.onion/images/2020/05/24/magazine/24mag-quarantine-davison-05/24mag-quarantine-davison-05-master495.jpg}

\includegraphics{https://static01.graylady3jvrrxbe.onion/images/2020/05/24/magazine/24mag-quarantine-davison-04/24mag-quarantine-davison-04-master495.jpg}

\hypertarget{untitled-series-1}{%
\subsection{Untitled series}\label{untitled-series-1}}

\hypertarget{photographs-2020-1}{%
\subparagraph{Photographs, 2020}\label{photographs-2020-1}}

\hypertarget{by-jack-davison-1}{%
\subparagraph{By Jack Davison}\label{by-jack-davison-1}}

``I'm always taking pictures, and I often use times like this, when I'm
not working, to play with different techniques. I was at home thinking,
If I'm never allowed to be in a room with someone to photograph him or
her again, could I still do my job? Is there a way in which I could
still make interesting pictures? These images were born of those
questions. I started playing with projecting images from Zoom or
WhatsApp calls and photographing them. I projected them from my phone or
through my computer onto the walls of my house. Sometimes I disrupted
the image further by putting something in front of my lens, like a small
pane of glass. It gives the photos texture --- the way the image falls
on the wall or how you can sometimes see cracks in the wall. I started
out doing cleaner images using screenshots, but I preferred adding these
extra layers. I was thinking of camera obscura and all the earliest
cameras --- cameras that would draw the outside world in.'' --- J.D.

\includegraphics{https://static01.graylady3jvrrxbe.onion/images/2020/05/26/magazine/26mag-quarantine-insanity-04/26mag-quarantine-insanity-04-master495.png}

\includegraphics{https://static01.graylady3jvrrxbe.onion/images/2020/05/26/magazine/26mag-quarantine-insanity-02/26mag-quarantine-insanity-02-master495.png}

\includegraphics{https://static01.graylady3jvrrxbe.onion/images/2020/05/26/magazine/26mag-quarantine-insanity-02-02/26mag-quarantine-insanity-02-02-master495.png}

\includegraphics{https://static01.graylady3jvrrxbe.onion/images/2020/05/26/magazine/26mag-quarantine-insanity-02-03/26mag-quarantine-insanity-02-03-master495.png}

\includegraphics{https://static01.graylady3jvrrxbe.onion/images/2020/05/26/magazine/26mag-quarantine-insanity-04/26mag-quarantine-insanity-04-master495.png}

\hypertarget{insanity-can-keep-you-sane-if-you-cant-live-normally-why-not-find-little-harebrained-ways-to-warp-reality}{%
\subsection{\texorpdfstring{Insanity Can Keep You Sane \emph{If you
can't live normally, why not find little harebrained ways to warp
reality?}}{Insanity Can Keep You Sane If you can't live normally, why not find little harebrained ways to warp reality?}}\label{insanity-can-keep-you-sane-if-you-cant-live-normally-why-not-find-little-harebrained-ways-to-warp-reality}}

By Molly Young

My quarantine has been fine. I was able to get out of the city; I don't
have the virus; I've lost some work, but not all of it; and just under
17 percent of my immediate family members have fallen seriously ill.
I've made out beautifully, and I feel terribly unhappy: a pair of
conditions that are tough to either reconcile or deny. Like characters
in an Edgar Allan Poe story, every person I know is suddenly confined to
a small space and forced to meditate on existence, death and his or her
own satanic impulses. The physical manifestations of my own dread have
included insomnia, a bumper crop of gray hairs and an absence of
self-control around any form of alcohol or drug. At some point my ribs
became countable. I daydream about which elected officials most deserve
to get Covid. Being an inessential worker in every sense of the word,
there has been no obvious way for me to channel my bad feelings into
civically productive activities. If I can't be productive, I have three
choices: Do nothing, do destructive things or do neutral things.

The adaptations began during my third week of quarantine. Over the
course of a routine internet stroll, I discovered a ``natural lifestyle
coach'' named Tony Riddle. Riddle looks like a Viking warlord and does
stuff like try to run 900 miles across the entire length of Britain
barefoot. He has sinuous muscles and four children and does not believe
in normal furniture, especially chairs, which he considers unnatural and
expendable. His home contains none. Instead, he recommends ``ground
living'': banishing furniture in favor of endless variations on
squatting and kneeling. This, according to him, helps nourish a person's
ankles, knees and hips, as well as ``rewilding'' her feet. I watched a
hypnotic two-minute video of Riddle, serene and barefoot, as he moved
through 10 positions designed to help support my ``ancestral movement
system.'' Was he a maniac? A genius? Only one way to find out.

I spent a day working from the floor, squatting before and around my
computer as though it were a campfire, with glutes aflame and feet
unshod. ``This is how a monkey sees the world,'' I thought, dreamily. It
was the calmest I'd felt in days. If lacking something as fundamental as
furniture did not impair my life, perhaps the same could be true of
recently banned fundamentals, like social contact or walking outdoors
without a muzzle. When people talk about fasting, this is what they talk
about: the surge of power that arises from realizing you don't need what
you thought you did.

\includegraphics{https://static01.graylady3jvrrxbe.onion/images/2020/05/26/magazine/26mag-quarantine-insanity-02/26mag-quarantine-insanity-02-master495.png}

And so other behavior modifications followed, all of them minor acts of
norm-shedding. I wandered around naked and stayed up all night. I paced
thousands of laps around the kitchen table. I slept in places that were
not my bed. I ate a meal without using my hands or any utensils, like a
dog, just to see what it was like. (Sloppy, as I expected.) I coaxed a
group of wild turkeys out of the woods with a trail of sunflower seeds
that I placed in a circle, which they obediently traced. I tried to
attract other birds by sitting quietly on the porch cloaked in seeds but
had to go back inside after being menaced by a squirrel.

\includegraphics{https://static01.graylady3jvrrxbe.onion/images/2020/05/26/magazine/26mag-quarantine-insanity-02-02/26mag-quarantine-insanity-02-02-master495.png}

None of these ``hobbies'' were fascinating or impressive --- and they're
even less so when I type them out --- but they were placating, free and
legal. I couldn't stop finding harebrained new ways to warp reality.
Little by little, my fidelity to personhood diminished. I spent hours
sitting on the carpet against a wall, doing nothing except considering.
I considered investigating the stain under the boiler. I considered
making banana bread. I considered cleaning the gutters. The word
``consider'' implies, correctly, that these thoughts at no point turned
into actions. During one morning of considering I felt my head entering
the wall, or sort of dipping in and out of it. This went on for a minute
before the sensation faded and was replaced with a moment of alarm at
the banality of the hallucination, which was like being on the world's
lamest drug: Instead of experiencing ego death, I momentarily penetrated
a sheet of drywall. What was the life-altering lesson in that? ``Sleep
more,'' maybe.

\includegraphics{https://static01.graylady3jvrrxbe.onion/images/2020/05/26/magazine/26mag-quarantine-insanity-02-03/26mag-quarantine-insanity-02-03-master495.png}

Some years ago a friend of mine lived with a community called the
Bruderhof. The Bruderhof is a constellation of settlements numbering
about 3,000 people, spread over four continents, with roots in
Anabaptism --- a 16th-century radical offshoot of Protestantism that
believes in a separation of church and state and adult baptism, among
other reforms. Members are pacifists who renounce private property, live
simply, dress modestly and --- to judge by the official Bruderhof
website --- have a distinctive sense of humor. (Among the questions in
the site's FAQ section: Sorry, what's Anabaptism? and Do you ever have
fun at the Bruderhof?) ``Amish-adjacent'' is probably the easiest way to
describe them, but they're allowed to have smartphones, drive cars and
upload (utterly delightful!) YouTube videos.

At some point in my friend's residence, a pregnant couple from the
community went to an outside hospital to give birth. The baby was
stillborn. Instead of the planned celebration, a course of mourning
began. The day after the couple returned, a busload of men and women
from a neighboring settlement showed up to take over daily operations.
For 10 days this fleet of visitors cooked, cleaned and performed
whatever tasks needed doing while the home community paused. ``It felt
like something from `Lord of the Rings,' '' my friend told me. ``All of
a sudden this ancillary battalion shows up over the hill, and you feel
like you might win the battle.'' I can't claim to understand the exact
meaning of this protocol, but it makes sense on its face. A grieving
period is marked by altering outer reality to mimic the state of the
bereaved's inner world: absolute inertia, total cessation of routine.
The ratification might not be curative, but at least it would feel
cosmically sensical.

The philosophical underpinning of this process is something the
Bruderhof calls \emph{gelassenheit}, and like a lot of ultimately
untranslatable German words, the meaning has been extensively discussed,
debated and written about on the internet. (The Anabaptist blogging
community is surprisingly robust.) Yielding, waiting, submitting to God,
abandoning the self, surrendering pride, subordinating the individual to
the community --- this is the English word cloud around
\emph{gelassenheit}. One blogger described it as ``an antidote to the
sheer pompous weariness of the world,'' which sounds about right.

There is an argument that behaving unusually is a rational way to
assimilate an altered reality, especially if the alteration is a rotten
one. When you pre-emptively dismiss whatever rules of living are within
your control, like using furniture or wearing clothes, you're injecting
yourself with a tolerable portion of insanity, which works like a
vaccine. I'm childless, but I've watched friends who are isolating with
children slip immediately into a less-extreme version of the same state:
losing their grip, abandoning their routines, witnessing their selves
mutate. If you believe that identity is behavior --- that you are how
you act, not what you think or how you feel --- then you understand that
adjectives like ``normal'' or ``functional'' require constant tending.
If you change your conduct, you can change your life: how simple, and
how daunting! All it took for me to become unrecognizable was to start
acting like a different person. In theory, this should work in reverse
too. When this is all over, I can return to chairs and forks and sleep.
It would probably be for the best. In the meantime, there are plenty of
individuals who haven't spiraled, either because they don't have the
luxury or do have a stronger constitution. In these people I find an
inspiring path back to normalcy.

Next to the house where I'm staying is a road that cuts through a marsh
to a bunch of dumpsters, and I spend hours shuffling between the house
and the dumpsters. It would be more pleasant to walk literally anywhere
else, but I need to be within sprinting distance of home in case the
anxiety takes a stomach-related expression. It doesn't help that the
marsh smells powerfully of sulfur. It's like a Gary Larson sketch of
hell. During one of my customary shuffles between home and dumpsters I
looked up and saw, on the embankment opposite the fetid marsh, a guy in
work boots with his hands on his hips, gazing down at me. I lurched to a
halt, embarrassed to be caught scurrying back and forth like a creepy
little rodent. He lifted a hand and waved.

\emph{Illustration by Brian Rea.}

Molly Young \emph{is a contributing writer for the magazine and the
literary critic for New York magazine.} Brian Rea \emph{is an artist in
Los Angeles. His book ``Death Wins a Goldfish'' was published last
year.}

\hypertarget{aphelile-iv-durban}{%
\subsection{``Aphelile IV, Durban''}\label{aphelile-iv-durban}}

\hypertarget{photograph-2020}{%
\subparagraph{Photograph, 2020}\label{photograph-2020}}

\hypertarget{by-zanele-muholi}{%
\subparagraph{By Zanele Muholi}\label{by-zanele-muholi}}

\includegraphics{https://static01.graylady3jvrrxbe.onion/images/2020/05/24/magazine/24mag-quarantine-muholi/24mag-quarantine-muholi-master495.jpg}

\hypertarget{aphelile-iv-durban-1}{%
\subsection{``Aphelile IV, Durban''}\label{aphelile-iv-durban-1}}

\hypertarget{photograph-2020-1}{%
\subparagraph{Photograph, 2020}\label{photograph-2020-1}}

\hypertarget{by-zanele-muholi-1}{%
\subparagraph{By Zanele Muholi}\label{by-zanele-muholi-1}}

``The image was taken on April 11, in South Africa, in response to the
emergency and the use of gloves and masks as essentials that are
necessary to keep one safe and protected against the pandemic. I am
under lockdown with limited movements and resources to continue
production at my usual pace. I have to make use of what is at my
disposal.'' --- Z.M.

\includegraphics{https://static01.graylady3jvrrxbe.onion/images/2020/05/26/magazine/26mag-quarantine-nature-04/26mag-quarantine-nature-04-master495-v3.png}

\includegraphics{https://static01.graylady3jvrrxbe.onion/images/2020/05/26/magazine/26mag-quarantine-nature-04/26mag-quarantine-nature-04-master495-v3.png}

\hypertarget{the-comfort-of-common-creatures-watching-birds-is-a-way-of-mobilizing-attention-to-turn-it-into-a-means-of-imaginative-escape}{%
\subsection{\texorpdfstring{The Comfort of Common Creatures
\emph{Watching birds is a way of mobilizing attention, to turn it into a
means of imaginative
escape.}}{The Comfort of Common Creatures Watching birds is a way of mobilizing attention, to turn it into a means of imaginative escape.}}\label{the-comfort-of-common-creatures-watching-birds-is-a-way-of-mobilizing-attention-to-turn-it-into-a-means-of-imaginative-escape}}

By Helen Macdonald

I've always hated staying still. I was the child who fidgeted, the
student who couldn't concentrate, who would find any excuse to leave her
desk and wander the streets, not to go shopping, or to meet friends, but
simply to keep moving as a way to escape anxiety. It's a tactic that has
worked for me for many years, but now, like millions of others, I have
to remain within the walls of my home. This confinement is a challenge
for me, one so absurdly unimportant in the face of the current crisis
that I'm uncomfortable even speaking of it. But it's a difficulty all
the same.

Recently I fired up a very old computer, a heavy, slow beast that still
has within it all the research files I was working on nearly two decades
ago. I was searching for a series of photographs I took in an Oxford
library. Here they are. The patterns on the underside of soaring buzzard
wings drawn in pencil upon a flattened cigarette pack. Photographs,
letters. A hand-drawn map of Oflag VIIB, the prisoner-of-war camp in
Eichstätt, Bavaria, where the German Army held Allied officers from 1940
to 1945, and pages and pages of notebooks written by prisoners who spent
their days in this camp, and others, recording the lives of birds that
nested there.

In Britain, comparisons to the Second World War have become a refrain of
the Covid-19 crisis. Myriad political commentators have praised that
era's heroism, the ability of the British people to cope with discomfort
and hardship. They use the war to encourage us to approach the pandemic
with poise and equanimity. Such exhortations traduce the terrible
realities of that war, and tend to flatten the worst aspects of this
crisis too, and this is why I'm hesitant to draw any parallels between
the lives of young men confined in a prison camp and the minor
inconveniences of lockdown. But something about the present
circumstances made me remember those men and want to revisit the notes
they took.

They wrote about arriving in the camp and deciding it was paradise for a
bird-watcher. Of how they watched for hours at a time, alone or in
shifts --- teams of men whose attention was fixed on the goldfinches
that nested within the wire fences, on redstarts and wrynecks or
warblers or crows --- taking exactingly detailed notes of what those
birds were doing every second of their witnessed lives. One watcher,
Peter Conder, who later became the director of Britain's Royal Society
for the Protection of Birds, eventually escaped captivity carrying his
precious notes in a rucksack fashioned from an old shirt. These men told
themselves that their notes would be of scientific importance once the
war was over, but I don't think that's why they made their rigorous
observations. I think doing so brought them comfort; the birds they
watched were free and knew nothing of war, and they were the same kinds
they knew from home. But mostly watching the birds was a way of
mobilizing attention, to turn it into a means of imaginative escape, a
way to counter their own sense of captivity, of powerlessness, futility
and despair.

Over the last weeks, I've often seen it suggested in the press and
social media that spending time in the natural world can bring comfort
and consolation during the pandemic. It's a notion rich with privilege,
for Covid-19 is disproportionately affecting communities with little
financial capital living in urban environments, for whom access to acres
of forests or fields is difficult or impossible. But as these notebooks
show, there are small, local methods of regarding the natural world that
are available to anyone and can lessen the psychological burden of
adversity.

During lockdown I have been spending a considerable amount of time
watching the common birds that visit my small backyard. There are
pigeons, starlings, blackbirds collecting beakfuls of dried grass to
line their nests, sparrows taking dust baths in a patch of bare and
sunlit earth. Watching animals from your home --- and they can be
anything from sparrows to spiders on windowsills --- can give solace
through the shift in perspective that the writer and philosopher Iris
Murdoch called ``unselfing.'' In her book ``The Sovereignty of Good,''
she gave the example of how, when feeling anxious and resentful and
caught up in your own concerns, you might look out of the window and see
a hovering kestrel; stare at it --- and then the world becomes all
kestrel, just for a while. Your brooding self disappears, and when you
return to yourself, your mental suffering has been lessened.

Most of us expect our lives to have familiar trajectories and spend our
days secure in our assumptions about how things will go and what will
happen next. Covid-19 has many terrible effects, but one that is
particularly quiet and strange is how it has unmoored us from that
familiar expectation. Everything is on hold. The future is
indeterminate. We do not know what will happen next. We cannot. The
sparrows that hop on the bricks of my backyard wall have daily routines
I am coming to know, and witnessing them is calming to watch when I have
few of my own. As I look out of my window in lockdown, my attention is
fixed on these birds, rather than trees or distant rooftops, because I
am desperate for novelty, to watch things that alter; for in seeing
change, I can parse time.

While the prison-camp ornithologists took their notes, their lives were
being controlled by camp administrators and guards, their futures
dictated by the frighteningly unknowable progression of the war. They
were crammed into close quarters, fed noisome rations, trapped in a
situation in which they had no control over what would happen to them.
But they could observe. ``I used to watch this pair of goldfinches for
10 hours a day,'' Peter Conder wrote, ``and sometimes up to 13 hours,
with only a minute or two for breaks.'' The simple act of watching the
birds could lessen the grip of dismal circumstances upon these men. But
by making their careful notes, they did something more: grant themselves
a new sense of control. Like his fellow bird-minded internees, Conder
wrote down everything the birds did, every visit they made to the nest,
every song, every flight to and from nearby trees, every hop and scratch
and turn. These men were writing trajectories, borrowed from the lives
of birds, that made the passing of time meaningful. Trajectories that
ran backward rather than forward, but were reassuringly solid and
sustained all the same.

The word ``observation'' comes from the Latin observatio, and its
etymological history spans acts of both observation and observance. As I
read these prison-camp notebooks, I began to see that what these men did
was a form of devotion. They were using the small lives of birds as
things they could orient themselves against. Their patient observations
remind me of how monks in medieval monasteries ordered their days to
fill them with meaning. How they made careful notes of the weather, of
the changing pattern of the stars, and how they timed their prayers
according to the precise positions of celestial objects. I never thought
that I could be content to stay still, but I remember those monks, look
at the notes taken by the men watching small birds behind the wire, and
think, too, of Elisabeth Tova Bailey's book ``The Sound of a Wild Snail
Eating,'' the story of how, confined to bed for many months by a
debilitating mitochondrial disease, she made careful observations of a
tiny snail that had been brought in from outside in a pot of violets,
transforming what she saw into a rich meditation on snails and time and
habitation and purpose at a time when all other life was out of reach.
What she came to understand during that time is something I have been
late in learning. We don't need to strike out into the wild to feel
close to the natural world and receive benison from it. From one place,
we can witness the sweep and dip of the universe about us. The stars
over the monastery gables, the birds on the wire, the street pigeons
that visit the patch of grass behind my house before flying off
elsewhere. We can become deeply connected to the world through paying
the most careful and fearless attention to what we can see, from
wherever it is we must be.

\emph{Illustration by Brian Rea.}

Helen Macdonald \emph{\emph{is a contributing writer for the magazine
and the author of the best-selling memoir ``H Is for Hawk.'' She last
wrote for the magazine about Brexit and the}
\href{https://www.nytimes3xbfgragh.onion/2017/01/05/magazine/in-search-of-post-brexit-england-and-swans.html}{ancient
British ritual of swan upping}*.}* Brian Rea \emph{is an artist in Los
Angeles. His book ``Death Wins a Goldfish'' was published last year.}

\hypertarget{someone-in-georgia-is-having-a-bad-hair-day}{%
\subsection{``Someone in Georgia Is Having a Bad Hair
Day''}\label{someone-in-georgia-is-having-a-bad-hair-day}}

\hypertarget{contuxe9-crayon-and-graphite-on-paper-in-progress-2020}{%
\subparagraph{Conté crayon and graphite on paper (in progress),
2020}\label{contuxe9-crayon-and-graphite-on-paper-in-progress-2020}}

\hypertarget{by-kara-walker}{%
\subparagraph{By Kara Walker}\label{by-kara-walker}}

\includegraphics{https://static01.graylady3jvrrxbe.onion/images/2020/05/24/magazine/24MAG-QUARANTINE-ARTWORKS-02/24MAG-QUARANTINE-ARTWORKS-02-master495.jpg}

\hypertarget{someone-in-georgia-is-having-a-bad-hair-day-1}{%
\subsection{``Someone in Georgia Is Having a Bad Hair
Day''}\label{someone-in-georgia-is-having-a-bad-hair-day-1}}

\hypertarget{contuxe9-crayon-and-graphite-on-paper-in-progress-2020-1}{%
\subparagraph{Conté crayon and graphite on paper (in progress),
2020}\label{contuxe9-crayon-and-graphite-on-paper-in-progress-2020-1}}

\hypertarget{by-kara-walker-1}{%
\subparagraph{By Kara Walker}\label{by-kara-walker-1}}

``The paranoia I have been experiencing has had me flooded emotionally,
and drawing has always been my sorting tool. I just let this emerge
organically, taking the point of view of the virus, which is living its
best life right now and which doesn't take sides, even though people
would like to structure the narrative in such a way, to prove their
political or social infallibility.'' --- K.W.

\includegraphics{https://static01.graylady3jvrrxbe.onion/images/2020/05/26/magazine/26mag-quarantine-paris-03/26mag-quarantine-paris-03-master495.png}

\includegraphics{https://static01.graylady3jvrrxbe.onion/images/2020/05/26/magazine/26mag-quarantine-paris-03/26mag-quarantine-paris-03-master495.png}

\hypertarget{finding-belonging-in-exile-i-didnt-feel-parisian-until-i-escaped-paris}{%
\subsection{\texorpdfstring{Finding Belonging in Exile \emph{I didn't
feel Parisian until I escaped
Paris.}}{Finding Belonging in Exile I didn't feel Parisian until I escaped Paris.}}\label{finding-belonging-in-exile-i-didnt-feel-parisian-until-i-escaped-paris}}

By Thomas Chatterton Williams

In his 1943 book, ``Liberté, liberté chérie,'' Pierre Mendès-France
recounts the various waves of flight that marked the exodus from Paris
in the spring of 1940, at the dawn of the German occupation. ``In the
early days, we saw fast and sumptuous American cars driven by liveried
chauffeurs,'' he writes. These were followed by the ``less shiny, less
new'' vehicles of the middle classes, which were in turn followed days
later by caravans of jalopies, eventually abandoned by the roadside,
``their owners continuing on foot to the next town, then by train,
bicycle, or hitchhiking.'' Next came the cyclists --- ``mostly young,
often carefree'' --- then the pedestrians, ``sometimes whole families,
the man with a rucksack on his shoulder, the woman pushing a cart or
baby carriage.'' Later came the stragglers, ``overwhelmed, feet
bloodied.'' Finally, there were the horse-drawn carriages driven by
peasants, ``laden with sick people, children, old people, agricultural
equipment and furniture,'' he writes. ``Sometimes livestock walked
alongside them, including cows and horses.''

\emph{Plus ça change}. It wasn't nearly so frenetic, but on March 16 my
wife and I, along with our two small children and whatever clothes,
books and toys we could think to grab, ordered a taxi across an empty
Paris and joined a crush of masked travelers at the Gare Montparnasse.
As we idled under the LCD screens, waiting anxiously for the arrival of
the trains that would shoot out of the station to various destinations
along the country's western reaches, I was aware that we were all of us
re-enacting a scene that has played out over and over again throughout
this city's dramatic history.

The day we left, after a week of growing alarm over the spread of the
novel coronavirus and decreasing freedom in the attempt to limit the
contagion, starting with the closure of schools and swiftly followed by
the shuttering of all nonessential businesses, President Emmanuel Macron
was scheduled to address the nation in the evening. He would, as many
expected, soon order total home confinement. The only question for
anyone with options was where to go to endure it. We barely had a chance
to contemplate our decision.

The day before we ended up leaving, it was our stroke of good fortune to
be having lunch with a couple who have their own young children and an
acquaintance in government kind enough to give them advance warning.
These friends patiently impressed upon us the severity of what was about
to happen. We were going to be housebound for the next 15 days, very
likely longer. Our friends would be leaving in a few hours for their
family home near the Atlantic. Would we like to hunker down with them?
At least this way the kids would have a yard to expand into. Once the
order was made official, it would be much more difficult to move around
the country. I grabbed my friend's laptop and reserved what appeared to
be four of the last tickets available to La Baule-Escoublac before
confinement, departing the following morning.

Our decision was a common one. Le Parisien reported that ``more than a
million residents left the Paris region before confinement,'' based on
geolocation data collected by Orange, the country's largest mobile-phone
service provider. The company's chief executive estimated that from
March 13 to March 20, a staggering 17 percent of the population of Paris
and its neighboring suburbs decamped to their country houses, of which
there exist some 3.4 million around the nation.

Contemplating these figures and their implications, I was reminded of
the architectural historian James S. Ackerman's 1990 classic, ``The
Villa: Form and Ideology of Country Houses,'' and its famous description
of the significance of the secondary residence in an urbanized society.
Noting that the ``basic program of the villa'' has remained unchanged
since Roman times, Ackerman offers an explanation: ``It fills a need
that never alters, a need which, because it is not material but
psychological and ideological, is not subject to the influences of
evolving societies and technologies. The villa accommodates a fantasy
which is impervious to reality.'' And though it exists as a respite from
the city, the villa ``cannot be understood apart from the city'' --- its
meaning derives from what it is not.

Few, if any, European societies are as centralized as France. Almost a
full fifth of the French population lives in the Paris metropolitan
area. In terms of cultural and economic dominance, Paris, still Europe's
fashion capital, also combines the functions and prestige of Wall
Street, Hollywood and Washington --- all in one location. Despite
various attempts over the years to decentralize the state, the
inhabitants of the city lord over the rest of France to an extent that
is similar but distinct from the divide between ``real'' America and its
coastal elite. Despite extravagantly rich and diverse geographical
blessings, from snow-capped Alpine slopes to crystalline Mediterranean
shores and the exquisite soils of Champagne and Bordeaux, since
monarchical times, the country has principally understood itself along a
simple binary: Paris/\emph{province}. Recently, such cultural and
political insolubility has provoked serious societal consequences. First
came more than a solid year of Yellow Vest protests and riots, which, in
Paris, sometimes had the feel of a furious guerrilla war being waged on
the town by the country. Now we have Covid-19, and though France
provides a safety net that precludes the spectacular kind of degradation
and suffering currently on display in America, few seem to believe that
we're truly all in this together.

On top of it all --- or as a fundamental aspect of this imbalanced
relationship --- there is that unusually high proportion of second-home
ownership (even if it's worth noting that these homes, while lovely, are
typically modest). In times of crisis, whether man-made or the result of
the pestilent ``flail of God,'' as Camus so memorably phrased it,
Parisians who can are wont to chase their safety to the provinces.
Fleeing Paris is a collective, inherited reflex. And as Mendès-France's
account lays bare, whatever else they are, such stories of escape are
always accounts of privilege --- with regard to the position of the
capital in relation to the rest of the country, as well as the internal
hierarchy of its inhabitants.

As a \emph{parisien d'adoption}, I am only semicognizant of where I may
fit at any given time into the French social fabric. As a foreigner
compelled by an epidemic to abandon my home --- an exile twice over ---
it is difficult, if necessary, for me to think of myself as part of this
other, overarching dynamic. Traveling through France in regular times,
for better or worse, I am simply perceived as an American. But now it's
different. My family has inadvertently participated in a larger,
possibly exploitative interaction that has sown resentment among some
residents of the rural areas we have infiltrated. The locals we've met
have been mostly welcoming and generous, though it's hard to say to what
degree that's because the fear that Parisians would spread the virus
ultimately proved unfounded. In those infrequent but memorable instances
in which a neighbor declines a ``\emph{bonjour},'' and for the first
time I can remember, I think I do detect my wife and friends being
perceived the way that I can be viewed --- not as natives but as
interlopers in this land.

La Baule-Escoublac, the nearby seaside resort we had come through from
Paris, counted 10 refugees for every inhabitant during the last mass
exodus 80 years ago. As fearsome as Covid-19 is, it is not the Nazis.
Still, the mayor of La Baule, Yves Métaireau, estimated that the
population had swelled to more than 40,000 inhabitants from 17,000.
After nearly a decade of expatriate ambivalence, imagining myself not so
much a resident of France as a ``trans-Atlantic commuter,'' to use James
Baldwin's phrase --- with one foot in this society and one foot out ---
in exercising this authentically Parisian need to escape, it feels as
though, suddenly, I've had my position here solidified. Mandatory
confinement is scheduled to expire on May 11, but neither my wife nor I
is so inclined to return to the city right away. We take turns scouring
the internet for houses in the country to rent --- a thoroughly Parisian
activity these days. The pandemic is forcing more and more of us to
reconsider just where we belong.

On a recent afternoon, as my friend and I were waiting a safe distance
behind the next person in line to enter the new organic market and
scrolling through work emails, he looked up suddenly and remarked that
this time away had put a few things in perspective: Maybe it wasn't all
that necessary to live in Paris after all. We'd already laughed at the
fact that we'd gotten in the habit of spotting and mentally separating
ourselves from the conspicuous new arrivals who flooded the area over
the Easter break. A monthlong string of sun-drenched days was still
going strong, and at that moment, I tended to agree with him. I wondered
how many of the million-plus Parisians scattered around the country were
thinking the same thing.

Correction: May 21, 2020

An earlier version of this article misstated the date of Pierre Mendès
France's book ``Liberté, liberté chérie.'' It was published in 1943, not
1977.

\emph{Illustration by Brian Rea.}

Thomas Chatterton Williams *\emph{is a contributing writer for the
magazine and a Distinguished Visiting Fellow at the Hannah Arendt Center
at Bard College.}
\href{https://www.nytimes3xbfgragh.onion/2018/10/11/magazine/jacques-audiard-french-scorsese-sisters-brothers.html}{His
last feature was about the director Jacques Audiard.} *** Brian Rea
\emph{is an artist in Los Angeles. His book ``Death Wins a Goldfish''
was published last year.}

\includegraphics{https://static01.graylady3jvrrxbe.onion/images/2020/05/26/magazine/26mag-pellegrin-03/26mag-pellegrin-03-master495-v2.jpg}

Emma, 6, runs through a field in front of the farmhouse.

\hypertarget{turning-the-camera-from-war-to-family--were-very-fortunate-to-all-be-here-together}{%
\subsection{\texorpdfstring{Turning The Camera From War to Family ****
\emph{``We're very fortunate to all be here
together.''}}{Turning The Camera From War to Family **** ``We're very fortunate to all be here together.''}}\label{turning-the-camera-from-war-to-family--were-very-fortunate-to-all-be-here-together}}

By Paolo Pellegrin

\includegraphics{https://static01.graylady3jvrrxbe.onion/images/2020/05/26/magazine/26mag-pellegrin-13/26mag-pellegrin-13-master495-v2.jpg}

Luna, 10, in her bed in the farmhouse.

I was in Australia, working on a photographic project on the aftermath
of the wildfires, and there was a moment when I realized that this
pandemic was not being contained. It was spreading everywhere. My family
was back in Switzerland, and I was playing these scenarios through my
mind: Borders being closed. What if I get sick? What if I get stuck?
What if my wife, Kathryn, gets sick, and I can't reach her?

So we made a decision that we should all be together. I cut my trip to
Australia short and rushed back home, just before they started imposing
travel restrictions and closing borders.

We live in Geneva in a pretty small apartment. The schools were already
closed, and my wife and I realized it would be very hard for the girls,
Luna and Emma, to quarantine there.

When we left the city for the mountains, I had the thought that we were
going into the unknown with no horizon of what was going to happen or
when it was going to end. I felt I wanted to document this experience,
even just for ourselves, so I made a very deliberate decision to bring
my ``real'' cameras.

We arrived at the farmhouse two days after my birthday, which is on
March 11. We've been here now for more than two months.

Both my wife and I had a longing to be in nature, especially at a time
like this. We'd been to this valley last summer and stayed in this place
then. This is the countryside in Switzerland --- when you get out of the
city, it all looks like this. We're pretty isolated in our little cabin.
We're very fortunate to all be here together.

\includegraphics{https://static01.graylady3jvrrxbe.onion/images/2020/05/26/magazine/26mag-pellegrin-05/26mag-pellegrin-05-master495-v2.jpg}

The forest behind the farmhouse.

\includegraphics{https://static01.graylady3jvrrxbe.onion/images/2020/05/26/magazine/26mag-pellegrin-12/26mag-pellegrin-12-master495-v2.jpg}

Kathryn.

\includegraphics{https://static01.graylady3jvrrxbe.onion/images/2020/05/26/magazine/26mag-pellegrin-14/26mag-pellegrin-14-master495-v2.jpg}

Luna and Emma play together on the bed.

I'm Italian, and in the first few weeks of the pandemic, seeing Northern
Italy deteriorating very, very quickly, there was a sense of being in
completely unexplored territory. That has parallels, surely, with the
type of work I have done in the past as a photographer in conflict
zones. Early on, it was our decision as a family for me not to travel,
not to cover this. This is the first time in decades, maybe in my
career, that I've decided not to cover an event, especially one of this
magnitude.

These photos are very different from my usual work. After decades of a
certain type of photography, very kinetic and very dynamic, I have found
myself looking for moments of silence. I'd never really photographed my
family or the girls very seriously before. Yes, I've photographed them
with an iPhone, as any other parent would. But I had a sense that I
wanted to document this moment. This is the longest I've ever stayed
with my family because I'm always traveling, always leaving, so to have
this time together is very special. At the same time, I do not think of
the pictures as a diary of a quarantine. Obviously there is that
element, but I wanted to touch something that was more timeless and
universal. Something about the girls, about the passage of time, about
changes. Something that was in the moment but that also transcended it.
--- \emph{As told to Adam Sternbergh}

\includegraphics{https://static01.graylady3jvrrxbe.onion/images/2020/05/26/magazine/26mag-pellegrin/26mag-pellegrin-master495-v2.jpg}

Emma.

\includegraphics{https://static01.graylady3jvrrxbe.onion/images/2020/05/26/magazine/26mag-pellegrin-10/26mag-pellegrin-10-master495-v2.jpg}

Luna.

\includegraphics{https://static01.graylady3jvrrxbe.onion/images/2020/05/26/magazine/26mag-pellegrin-11/26mag-pellegrin-11-master495-v2.jpg}

The table in the farmhouse.

\includegraphics{https://static01.graylady3jvrrxbe.onion/images/2020/05/26/magazine/26mag-pellegrin-08/26mag-pellegrin-08-master495-v2.jpg}

Emma plays at a vacant barn near the family's house.

\includegraphics{https://static01.graylady3jvrrxbe.onion/images/2020/05/26/magazine/26mag-pellegrin-07/26mag-pellegrin-07-master495-v2.jpg}

Luna and Emma.

\includegraphics{https://static01.graylady3jvrrxbe.onion/images/2020/05/26/magazine/26mag-pellegrin-04/26mag-pellegrin-04-master495-v2.jpg}

Pellegrin's reflection as he photographed Emma.

\includegraphics{https://static01.graylady3jvrrxbe.onion/images/2020/05/26/magazine/26mag-pellegrin-09/26mag-pellegrin-09-master495-v2.jpg}

The view.

Paolo Pellegrin \emph{is a Magnum photographer who has been documenting
historic events around the world for decades. He last photographed the
artist Anselm Kiefer for the magazine.}

\includegraphics{https://static01.graylady3jvrrxbe.onion/images/2020/05/26/magazine/26mag-quarantine-jamison-2/26mag-quarantine-jamison-2-master495.png}

\includegraphics{https://static01.graylady3jvrrxbe.onion/images/2020/05/26/magazine/26mag-quarantine-jamison-1/26mag-quarantine-jamison-1-master495.png}

\includegraphics{https://static01.graylady3jvrrxbe.onion/images/2020/05/26/magazine/26mag-quarantine-jamison-1-02/26mag-quarantine-jamison-1-02-master495.png}

\includegraphics{https://static01.graylady3jvrrxbe.onion/images/2020/05/26/magazine/26mag-quarantine-jamison-1-03/26mag-quarantine-jamison-1-03-master495.png}

\includegraphics{https://static01.graylady3jvrrxbe.onion/images/2020/05/26/magazine/26mag-quarantine-jamison-2/26mag-quarantine-jamison-2-master495.png}

\hypertarget{when-the-world-went-away-we-made-a-new-one-i-lost-many-things-during-the-quarantine-but-there-can-be-an-unexpected-abundance-inside-a-state-of-loss}{%
\subsection{\texorpdfstring{When the World Went Away, We Made a New One
\emph{I lost many things during the quarantine. But there can be an
unexpected abundance inside a state of
loss.}}{When the World Went Away, We Made a New One I lost many things during the quarantine. But there can be an unexpected abundance inside a state of loss.}}\label{when-the-world-went-away-we-made-a-new-one-i-lost-many-things-during-the-quarantine-but-there-can-be-an-unexpected-abundance-inside-a-state-of-loss}}

By Leslie Jamison

Like so many quarantines, mine began with a series of sudden
subtractions: subways, classes, public spaces, hugs, bookstores, child
care; the bodies of friends in my living room; the bodies of strangers
brushing against mine on the sidewalk; and finally my own body, as the
virus came for me early, insistent but ultimately merciful; shivers and
night sweats and muscle aches rippling from my neck to my heels. During
those weeks of total isolation with my toddler, I lost my sense of taste
and smell, lost everything beyond my doorway --- lost the streets of my
city, which was rapidly flooding with deeper losses I could only
imagine. The wailing sirens made it impossible to forget that the
hospitals were filling with patients on ventilators.

The subtractions of our quarantine came on the heels of other ones. I
signed divorce papers just a month before the city started shutting
down, and as the lockdown's restrictions drew an increasingly tight
perimeter around every household, they cast into sharper relief the ways
mine had been gutted. It felt vaguely like being forced to live in a
building splintered by a wrecking ball before the rebuilding had begun.
Quarantine didn't just take things away; it revealed --- with a harsh,
unrelenting clarity --- what had already been lost.

Once I realized I would be spending many weeks alone at home with my
daughter, I made us a daily schedule with clumsy illustrations: stray
water drops next to ``Mama Shower,'' a cutout octopus next to
``Cleaning/Chores,'' as if we'd deploy eight arms to wipe the door
handles with bleach; a tiger beside our ``Morning Walk,'' as if the
streets of Brooklyn would be full of exotic discoveries. But once I got
sick, even the limited life outlined on our rainbow schedule --- its
cheerful colors radiating compensatory, forced optimism --- now seemed
naïve in its aspirations, anchored by walks we could no longer take,
meals I could no longer taste and activities that required staying
vertical longer than I could manage. The cherry blossoms beyond our
windows seemed tone-deaf in their extravagance. The sunshine arrived
like someone laughing on a hospital ward.

\includegraphics{https://static01.graylady3jvrrxbe.onion/images/2020/05/26/magazine/26mag-quarantine-jamison-1/26mag-quarantine-jamison-1-master495.png}

Before I realized I was sick, I refused to believe my own fatigue,
falling asleep on the couch while I tried to return work email during
naptime. But eventually there was no denying it: the aches running like
electric currents through my legs, wearing me out like exercise. When I
stood after picking up things my daughter dropped or tossed, the corners
of my vision fluttered with dark flecks. The virus claimed my bedroom as
its own, salting my sheets with night sweat. When I woke in the
darkness, body aching in the gloom, I always checked the news on my
phone before I could remember not to.

\includegraphics{https://static01.graylady3jvrrxbe.onion/images/2020/05/26/magazine/26mag-quarantine-jamison-1-02/26mag-quarantine-jamison-1-02-master495.png}

A few days after I lost my sense of taste and smell, I started seeing
articles about this new symptom. That's how it was: bodies in the news,
and the news in our bodies, making us sweat and shiver. It seemed as if
losing my sense of taste was a personalized cosmic joke, a nod to the
eating disorder I had years earlier. But that's the fallacy and hubris
of any misfortune, however minor --- that it was made bespoke, just for
us. I knew this was melodrama and tried not to indulge it before
naptime. Then I could cry alone in the bathroom if I needed to.

\includegraphics{https://static01.graylady3jvrrxbe.onion/images/2020/05/26/magazine/26mag-quarantine-jamison-1-03/26mag-quarantine-jamison-1-03-master495.png}

Maybe the pandemic felt to everyone like a heat-seeking missile
specifically targeting the particular fragilities of any life --- a new
business, a restaurant job, a fractured marriage or its dissolution ---
even as the virus cast its vast, impersonal damage across us all. It
created a certain cognitive dissonance to encounter something as surreal
and unfamiliar as a global pandemic from inside the deadening
familiarity and cloistered banality of our apartment --- an
extraordinary event experienced from inside a parade of days textured by
unceasing ordinariness, the daily loop of domesticity. The teakettle,
the oatmeal-crusted bowls in the sink, the toddler scattering her tiny
hats and gloves across the floor for the umpteenth time, ``Mama FIX
it.'' The days were endless and also irrelevant: Tuesdays were
Wednesdays were Fridays, except sometimes it was raining outside and
sometimes it was sunny and sometimes --- as a neighbor informed us by
text --- someone broke into the vestibule of our building to ransack the
Amazon packages. The past flooded the empty present, filling the
apartment with its ghosts.

I kept remembering the summer I spent recovering from jaw surgery two
decades earlier, not just sequestered in my home but in my body; unable
to eat or speak because my jaw was wired shut for months, 18 years old
and missing the world that was stripped away. I kept remembering the
first time I tried to stop drinking --- at 27, a decade later --- when I
essentially put myself in quarantine, taking a week off my bakery job to
hole up in my brother's empty apartment and Not Drink. In my mind, this
self-sequestering was a cross between a bad schoolgirl's being sent to a
corner of the classroom and a hero's striding off to some remote
mountaintop to confront her archenemy in one-on-one combat. In reality,
it mainly involved eating saltines and foil-wrapped triangles of
spreadable cheese for dinner, and realizing --- at one point --- that it
had been a couple of days since I'd been outside, in part because I was
afraid I lacked the willpower not to stop at a liquor store. I drank
again as soon as I got home.

When I tried to quit again, a few months later, it was not in isolation
but by flinging myself into the unexpected community of recovery
meetings. Remembering those nights in the midst of the pandemic, I
yearned for their physicality: the unfolded origami creases of
strangers' papery palms against my own; the stem of a plastic fork still
warm from someone else's grip as I pronged a vanilla-frosted slice of
sober-anniversary cake; the raspy voices and minty gum-breath of
chain-smokers offering collective prayers. But after six weeks of
studiously avoiding any kind of contact or even proximity with
strangers, I also flinched at the idea of that kind of bodily communion;
it seemed an impossibly beautiful constellation of perilous exposures.

But while the physical proximity of early sobriety felt impossibly far
away, an echo from those days felt eerily close --- the surprise of
finding unexpected abundance inside a state of loss. When you lose what
you rely upon, you start reaching for things you never thought you'd
want, or else the things you already had but always took for granted.
Early sobriety taught me one version of this strange arithmetic by
giving me a way to understand what I was losing --- the sweet oblivion
of getting drunk --- in terms of what it made room for: not just the
sweaty palms and earnest confessions of strangers but also a more acute
presence in my own life.

\textbf{At first, of} course, not-drinking was hell. It was deprivation
and punishment, as if I were trapped in a bare white room while the
cinema reels of boozy nostalgia played on the other side of a glass
wall: the salty pop of gin-soaked olives, the foam of cold beer on warm
summer evenings flickering with fireflies. All that was gone. Only
seltzer remained. But if not-drinking was hell, then sobriety was
something else. Eventually --- not on the first day, or the 20th, but
maybe on the 100th, or the 400th --- the whole world began to open up.
Days weren't just defined by absence --- this is life \emph{minus}
drinking --- but by a new kind of plenitude: the rituals of recovery
meetings, and the voices of strangers in those rooms, telling stories
about loving booze so much they thought their hearts would break from
losing it.

This strange, unsettling affinity with strangers was abundance. The call
to \emph{listen} was abundance. But these weren't the only forms of
abundance. The sensory hyperattention of sobriety was overwhelming, like
staring at the sun: the acid pang of an orange slice on a cold sore, the
ache in the balls of my feet after 12 hours standing beside a giant
mixer in the kitchen of the bakery where I worked. The abundance of
those days rose from the conspiracy of multiple constraints --- the
constraints of sobriety alongside the confines of that cramped kitchen
and those repetitive labors. Even unbeautiful things came to constitute
a strange new lushness, because they felt so ferociously proximate, so
searing and undeniable.

A decade later, quarantine was nothing if not searing and undeniable ---
the broken-record quality of our daily lives insisting on the same
rooms, the same people, the same routines. Recovery meetings happened on
Zoom now, like so much of the rest of my life, and at a distance
couldn't offer the same bodily surrender. Still, while certain kinds of
visceral intimacy were lost, in other ways the meetings felt more
intimate than ever. Every square on the screen was a portal into
someone's home, revealing other sober alcoholics leaning against their
headboards or curled up under blankets, Bluetooth buds carrying the rest
of our voices, cat whiskers swishing suddenly in front of computer
cameras. In our thumbnail boxes, we chanted the serenity prayer in an
out-of-sync patchwork that was somehow more moving for its raggedness,
for the ways it failed to disguise the incompleteness of our medium, the
ways it didn't replace what we'd lost: that room full of body heat and
layer cake, plastic forks passed palm to palm. It was a chorus of
disembodied voices trying our best, straining or fumbling or sometimes
surging toward gratitude; acknowledging all the loss and terror around
us without trying to redeem it.

For the first few meetings I attended, I had Zoom set to ``speaker''
mode because I didn't know there was a ``gallery'' alternative that
could display everyone's faces at once. Whoever was speaking loomed
large, but whenever someone laughed or murmured in recognition their
face would pop briefly to the center of the screen --- the technology
illuminating, just for a moment, the flashes of resonance that had
animated our meetings all along. Sometimes I'd be distracted or
horrified by the sight of my own face in the corner --- wondering if my
expression communicated enough attention, compassion or openhearted
presence --- but one of the best things about speaker mode was that it
let me scroll away from my own face so I didn't see it at all. Scrolling
away from my own face on Zoom became a technological embodiment of what
recovery meetings had been inviting me to do for years: get away from
myself, flee the quarantine of my own heart.

Even outside these meetings, quarantine was enacting a daily alchemy
with the abstract truisms of recovery, making them concrete: \emph{One
day at a time} meant not knowing how long quarantine would last. It
meant: \emph{Just get through this single stretch of hours.} Surrender
existed on all scales. It meant giving up on knowing how the pandemic
would play out across the world. It meant giving up a definite timeline
for when we'd come out of lockdown. It meant letting my daughter pull
all the books off the bookshelf without trying to pick them up. One
morning I sat cross-legged and tried to read passages from a book of
Buddhist meditations --- \emph{holding on to anything blocks wisdom,} I
dutifully repeated --- while my daughter climbed onto my back, heaving
oatmeal breath on my cheek, and pulled the book from my hands. She
enacted its truth by ripping one of its pages.

Ten days into our total isolation, once our apartment had filled with
bags of recycling, I Googled ``toddler art + old trash.'' We ended up
drawing a road on the back of a cardboard diaper box. When she ripped up
one of her picture books, we used the illustrations as decoration, and I
copied a quote onto the cardboard from a poem Sylvia Plath had written
for her newborn son, a poem I memorized at 24 during the months after my
abortion: ``Love, love, I have hung our cave with roses. . . . '' Old
trash was the new cross-stitch. My daughter scribbled over the lines
with marker, and they felt even truer obscured by her scribbles,
spackled with her fish stickers.

Every morning I read the same passage in the Big Book, \emph{It is plain
that a life which includes deep resentment leads only to futility and
unhappiness}, and thought of other people's quarantines --- people with
partners, who curled up with a body each night, or people who'd fled the
city, or people who'd fled the city \emph{with} their partners --- and
tried to surrender that resentment too. I tried to neutralize it with
gratitude. Not gratitude in the dutiful, box-checking, white-knuckled
sense of acknowledging everything I had --- my health, my daughter, my
job --- but in a more immediate sense: for the sunlight on my daughter's
overgrown curls, for the specific weight of her head on my shoulder; for
my students reading from the pandemic diaries I'd asked them to keep, as
we all gathered in our Zoom boxes to listen; for my high school friends
on Zoom, how blunt and broken I could be in their company. I was
grateful for the taste of peanut butter, the first time it returned ---
the first time \emph{any} taste returned. The faint nutty sweetness was
like a stranger standing at the end of a long corridor, barely visible
but \emph{there} --- more than six feet away, but better than no one at
all.

\textbf{During the thickest,} shivery days of my illness --- when it was
just me and my daughter and a photo-copy of my divorce settlement on a
closet shelf, tucked beneath our stash of cloth masks --- it was as if
her tiny, restless body were living for both of us: tasting for us both,
seeking pleasure for us both, radiating energy for us both. She
conducted intense, inscrutable projects, using her tiny wok to carry my
lucky hawk's feather --- found by the side of the road on a sunny day
upstate, in another universe entirely --- to her little wooden kitchen,
where she stirred it with a little wooden knife. What was she doing? Her
eyes gleamed with focus. She wanted to take care of everything. She
tried to put a diaper on her wooden zebra. She tried to put a diaper on
our Dustbuster. She tried to put a diaper on our tube of Clorox wipes
and then tuck it under my comforter. ``Night, night,'' she said. On Day
8 of our isolation, she glanced toward the window and said plaintively,
``Outside.'' On Day 9, we spotted a toddler in a puffy orange coat
lurching toward her father on a driveway across the street, and my
daughter called out: ``Orange baby!'' Seeing another person felt like
spotting a celebrity. When the toddler left, my daughter called out:
``Orange baby come back!''

When my aching muscles felt like knotted ropes draped across the inside
of my body, and I felt incapable of doing much besides lying down, I was
grateful for my daughter's endless appetite for stories. Her desire to
read 20 picture books in a row no longer seemed burdensome, as it did
during the busy crush of normal life; now it seemed more like a way she
was guiding us through the hours. The days were a swirl of body chills
and fantasies: the story about the boy and bear traveling through a
magical land of berries; the story about the mouse who recited poems to
all the other mice spending winter huddled in an old stone wall; the
story about the woman who gazed out the window from her sick bed and
imagined planting lupines across the hills; the story about the dinosaur
who wanted to be a ballerina. It started to seem as if every story were
about quarantine. The mice in the old stone wall were in quarantine. The
woman in her sickbed was in quarantine. The brontosaurus bumping her
head while attempting a jeté was in her own quarantine --- trapped in a
space that was too confined, a world that was too small. The boy and the
bear on the blackberry train were clearly \emph{also} in quarantine;
that's why they were dreaming of this fantastical land full of
strawberry ponies and raspberry fireworks.

At dusk each day, I played Leonard Cohen during bath-time, his scratchy
voice crooning about a Manhattan that no longer existed, and might never
exist again, where \emph{there's music on Clinton Street all through the
evening.} My daughter's tiny palms splashed against the soapy water as
the streets below our window erupted into applause for the doctors and
nurses at the hospital a few blocks away --- and we clapped too, through
the suds of her bath, though no one could hear us.

I tried to feed my daughter at least one new thing each day, as a way of
telling the days apart. Boiled zucchini, sliced rings of pineapple,
raspberries before they fuzzed with tiny white beards of mold. Pasta
shaped like bow ties, pasta shaped like wagon wheels. Peanut butter
straight from the jar. Sometimes I caught myself gazing at her with
jealousy --- she could still \emph{taste}. I missed the taste of
chocolate, the taste of apples, the taste of Cheddar cheese, even the
taste of the instant coffee I drank when the good coffee ran out. Or
certain smells, like the urine tang and compost stink of my daughter's
drooping diapers --- I grew to miss even that.

Missing taste became a way of missing everything. I missed the air,
missed having moments when I was doing something other than picking tiny
wooden teacups off the floor, missed other people --- even a single
other person, even the bodies of strangers --- missed my friend Anna,
who lived five blocks away, now a thousand miles, who brought over
groceries when I was sick: a bulb of fennel and a carton of mushrooms
and pale balls of raw cookie dough, grub worms of knobby turmeric (what
do I . . . ?). Late at night, after our kids were asleep, Anna and I
would trade voice memos telling the stories of objects in our homes,
because the objects in our homes were what we had. She told me about her
stack of overdue library books, the orange earrings she'd given birth
in. It wasn't the same as feeling her arm draped over my shoulders, or
watching our toddlers gazing up at us, side by side, waiting for us to
feed them chunks of apple-cider doughnuts. But it was something that
reached into my marrow, her voice traveling across the city blocks,
filling up the darkness.

A few years into sobriety, I went to a potluck where no one ate or drank
anything. Half of the people who came were alcoholic, or sober addicts,
and the other half struggled with binge eating, so the idea was
basically: What activity can we gather around that doesn't involve
putting something into our bodies? Everyone brought something to read,
and we gathered over flickering candles and listened to one another as
if our voices were food. A few days into the quarantine, when two
friends organized a group of us to read poems aloud and send the audio
files to one another, I thought of that boozeless, foodless potluck, how
grace never arrives as we imagine it. I sat by a window overlooking
empty streets, as my daughter tried to put a wooden cookie in my mouth,
and listened to the disembodied voice of my friend reading William
Meredith's ``Accidents of Birth'': ``to/meet in a room, alive in our
skins,/and the whole galaxy gaping there.''

\textbf{What to do} with the strange incandescence of those two weeks of
total isolation with my daughter --- her sweet voice naming all the
animals in her bath book as the clapping from the streets rose around us
like a hymn? What to do with the eerie, spellbinding video my friend
sent of herself dancing in the middle of a deserted street to a speaker
blaring ``We Are the World'' from a shuttered jewelry store? These
strange beauties did nothing to supply the ventilators our city lacked,
to mitigate the oncoming apex of deaths, to stave off the bankruptcies
or the oncoming recession. They were not a vaccine, or an antibody test,
or even a useless floating hospital docked in the Hudson River. They did
not cure the virus, or redeem the suffering it caused. The sirens kept
blaring as I gave my daughter her baths.

It's easy to subscribe to a fantasy of diminishment as revelation ---
the notion that wisdom is the inevitable yield of hardship. But
sometimes loss just feels like loss, and absence is just absence: the
solipsism of pain; the ache of losing touch; the empty streets and
bankruptcies, the missing ventilators, the bodies stored in the
temporary morgues of moving vans. The trick is how to hold both truths
at once --- absence-as-presence and absence-as-absence --- rather than
letting one obscure the other; how to let fragile, unexpected, imperfect
consolations exist alongside everything they can't console.

Holding both at once lets us honor the pleasures and odd discoveries of
quarantine without blinding ourselves to everything beyond it. It's a
way of seeing that does not back away from what is happening by
pretending people are not dying, and that does not back away from what
is happening by pretending people are not loving and being loved
alongside this death. Because we are also eating brownies. We are
stupefied by the tenderness of a child tucking a tube of Clorox wipes
under the covers. We are brought to tears by the sight of a nurse
walking home from work in hospital scrubs. Suffering and grace live side
by side, as they always have --- in the same homes, or else separated by
those walls we keep between our bodies now, in service of a solidarity
we trust but cannot touch. Grace locks eyes with pain from the other
side of the sidewalk, six feet away, and they both keep walking.

Leslie Jamison \emph{is the author, most recently, of ``Make It Scream,
Make It Burn.'' She last wrote for the magazine about the relationship
between creativity and addiction.} Brian Rea \emph{is an artist in Los
Angeles. His book ``Death Wins a Goldfish'' was published last year.}

\hypertarget{untitled}{%
\subsection{Untitled}\label{untitled}}

\hypertarget{photograph-2020-2}{%
\subparagraph{Photograph, 2020}\label{photograph-2020-2}}

\hypertarget{by-rinko-kawauchi}{%
\subparagraph{By Rinko Kawauchi}\label{by-rinko-kawauchi}}

\includegraphics{https://static01.graylady3jvrrxbe.onion/images/2020/05/24/magazine/24mag-quarantine-kawauchi/24mag-quarantine-kawauchi-master495.jpg}

\hypertarget{untitled-1}{%
\subsection{Untitled}\label{untitled-1}}

\hypertarget{photograph-2020-3}{%
\subparagraph{Photograph, 2020}\label{photograph-2020-3}}

\hypertarget{by-rinko-kawauchi-1}{%
\subparagraph{By Rinko Kawauchi}\label{by-rinko-kawauchi-1}}

``I've been staying home in Chiba Prefecture in Japan. My daughter was
playing with one of her arts-and-crafts creations, a plastic bottle
painted to work like a kaleidoscope. I affixed it to my camera, using it
to photograph my garden. It made me understand the possibility, and the
variety, of the future --- how changing one detail can make the whole
world look like a different place.'' --- R.K.

\includegraphics{https://static01.graylady3jvrrxbe.onion/images/2020/05/24/magazine/24mag-diary-04/24mag-diary-04-master495.png}

\includegraphics{https://static01.graylady3jvrrxbe.onion/images/2020/05/24/magazine/24mag-diary-04/24mag-diary-04-master495.png}

\hypertarget{we-cant-comprehend-this-much-sorrow-historys-first-draft-is-almost-always-wrong--but-we-still-have-to-try-and-write-it}{%
\subsection{\texorpdfstring{We Can't Comprehend This Much Sorrow
\emph{History's first draft is almost always wrong --- but we still have
to try and write
it.}}{We Can't Comprehend This Much Sorrow History's first draft is almost always wrong --- but we still have to try and write it.}}\label{we-cant-comprehend-this-much-sorrow-historys-first-draft-is-almost-always-wrong--but-we-still-have-to-try-and-write-it}}

By Teju Cole

This year has been a blur, but I remember one day clearly: Sunday, March
8. It was the last day I ate at a restaurant, the last day I went to a
concert (Red Baraat at the Sinclair in Cambridge, Mass.) and the last
day I hugged a friend. It was also the first time I thought that I
should begin writing about what was going on.

That thought was immediately followed by its negation: Why bother? The
same incidents, the same references and the same outrages would
inevitably be picked over by other writers; for all our social
distancing, we'd all be crowding around the same material. I also knew
that anything I wrote could soon be --- in fact was almost certain to be
--- contradicted by new developments. But what worried me most was that
certain points of emphasis in my writing would later prove to have been
misjudged, and that this would somehow reveal that my heart had been in
the wrong place all along.

By mid-April, the daily death toll had risen to terrifying heights. But
then those numbers fluctuated and at times fell, and it appeared that
the worst was over. We seemed to be in for a significant sequence of
days. So I set down a week's worth of observations, hoping to capture,
with no attempt at being comprehensive, a time when my feelings were as
raw as my understanding of what was happening.

Saturday, April 11

---

Yesterday's death toll from Covid-19 in New York State was 996 people.
It is unbelievable to be living through this. I am in Cambridge, but I
am obsessed by the New York numbers. We know that history is not over,
that enormous events are inevitable. But the mind takes time to adjust.
It still feels unreal to wake up every morning in a world as strange as
dreams.

I saw a photograph by a Bosnian photographer named Ziyah Gafic. Posted
on his Instagram account, it showed undertakers in a Sarajevo graveyard.
There were four of them, seated, standing, resting on gravestones. They
were masked and clad in the full-body white suits of the familiar new
priesthood. I was reminded of a pen-and-ink drawing of beekeepers by
Bruegel. The same muffled air of mystery.

My sister, a doctor in Michigan, texted me a couple of days ago: ``My
first patient today was a 22-year-old prison worker who had been exposed
and was now short of breath. I felt a wave of panic. Then I just did my
job.''

Sunday, April 12

---

Yesterday's death toll from Covid-19 in New York State was 997 people. I
am looking at the bird feeder outside my window. A squirrel has figured
out how to clamber up the supporting pole, and now it hangs on, busily
eating, looking both awkward and comfortable. The boreal chickadees are
irritated. From time to time a cardinal appears, like a splash of red
paint.

A scene in a film I watched the other day --- a not very good one from
2013 called ``On My Way,'' starring Catherine Deneuve --- was set in a
restaurant. The clinking of glasses, that sound alone, filled me with
such longing. Much of what I miss is tied to dumb privilege. And
yesterday, outside the wine shop (speaking of privilege) in nearby
Belmont, as a number of us waited for our pickup --- masked, observing
appropriate social distance --- one man with a thick Latin American
accent said: ``Is like a movie. But nobody waking yet.'' I was suddenly
spooked, as in a dream when you become aware you're in a dream only
after someone says it.

Monday, April 13

---

Yesterday's death toll from Covid-19 in New York State was 957 people.
My mother turned 70 last week. A few days before her birthday, she sent
a carefully worded text to all her friends and family asking them to
celebrate with her by staying in their homes. My parents live in Lagos,
Nigeria, and they are taking the quarantine rules seriously. Their daily
situation over there is much the same as mine here in Cambridge. We are
fortunate in the same ways: There is enough food in the pantry for a few
weeks, and there is no pressing need to leave the house. It is hard for
others in Nigeria. Huge numbers of daily wage earners, street traders
and informal workers have to feed themselves and their dependents. How
are they doing it? Asking certain people to stay home for the sake of
society is absurd, because these are people society has never cared
about. ``Stay home so people won't die'' is a hell of a thing to say to
those who are dying of hunger. I keep thinking about floods, and how
only after the waters recede do the bodies of the drowned become
visible.

I'm listening to kora music played by Ballaké Sissoko. I'm listening to
Beethoven, the early sonatas. In these bruising days, any delicately
made thing quickens the heart.

Tuesday, April 14

---

Yesterday's death toll from Covid-19 in New York State was 882 people.
It surprises me that I can still listen to so much music every day, and
that I still want to. Listening to MF DOOM, I find myself thinking about
what hip-hop is and what it means in a moment like this. Music of all
kinds thrives socially, but some genres tolerate solitude better than
others. Hip-hop isn't really alive unless someone else hears it: It is
social all the way down to its genetic code. These days, DJ Nice is
hosting parties that are viewed by a huge audience on Instagram Live.
Last weekend, also on Instagram Live, there was a battle between DJ
Premier, who was spinning from isolation in New York, and RZA, who was
spinning from isolation in California. I watched and listened with
thousands of others, creating with them the feeling of being in a crowd,
wishing that the crowd could be embodied.

Wednesday, April 15

---

Yesterday's death toll from Covid-19 in New York State was 1,003 people.
It is a clear, bright day in Cambridge. Flowers are blooming outside. K
fills the house with cuttings from her tulip beds and from the cherry
tree. There are four grinning daffodils in a clear vase.

People are dying in hospitals and dying at home. The official tolls are
almost certainly an undercount. The morgues are overflowing. Those are
the facts. But where is the grief? When we first started getting the
news out of Italy, and then Spain, with frightening daily numbers
comparable with what is now happening in New York, that news seemed to
be delivered with holy awe. In El País, for example, each day's news was
led by the previous day's dead, a number that was often in the paper's
main headline. In the American papers, I usually have to do some
searching to find how many people have died in the past day. The front
pages here seem to often carry news of the financial markets or of the
political squabbles of the day. But what I want is to be directly
confronted with the fact, the enormity, the irreducible sadness of all
these deaths.

A friend of mine attended a Zoom funeral last week for his grandmother.
My neighbor attended a Zoom funeral last week for his father. A tragic,
sordid phrase that wouldn't have meant anything just a few weeks ago:
``Zoom funeral.'' Zoom is a for-profit company, and it is currently very
profitable.

Thursday, April 16

---

Yesterday's death toll from Covid-19 in New York State was 936 people.
The numbers are bad in Massachusetts, over a hundred per day at the
moment, but nowhere near as grim as in New York.

I talked to my mother on the phone. She says there are roving gangs of
young men in Lagos. They are stopping and raiding food trucks and
sometimes invading people's homes. I tell her, thinking of Jean Valjean,
that you can hardly blame a hungry man for stealing bread. She says,
``That's true, but how would you feel if it was your home they
invaded?''

I haven't been outside in a few days. I go out after 4 p.m., and the
streets are busy with walkers and joggers. There is a profusion of
beautiful spring flowers, and many more masks on the streets now, in
keeping with the latest directives. I'm wearing one.

In Bruegel's drawing, there are three beekeepers and a fourth person,
whose body is turned away from us. He has climbed up a tree, presumably
to take the eggs out of a bird's nest. Is the drawing meant to contrast
the active life (the man in the tree) with the contemplative life (the
beekeepers)? No one knows for sure. It's not only the atmosphere of the
drawing that is mysterious; its meaning, too, is unknown and has eluded
generations of scholars.

When I return from my walk, I have a phone chat with my cousin, who
works as a sorter at one of Amazon's warehouses in the Atlanta area. She
immigrated to the United States in September and has just been upgraded
to full-time status at the warehouse. Her younger sister, who has been
here longer, is a full-time nurse on the Covid-19 unit of a large
teaching hospital. She tells me she's being offered overtime, for good
pay, but that she doesn't want to do it because of the elevated risk.

Unbelievable to be living through this? No, it's believable. I believe
it, the way you wake up in the middle of the night on a trans-Atlantic
flight and believe: I am 35,000 feet above sea level, moving at
tremendous speed through freezing air.

Friday, April 17

---

Yesterday's death toll from Covid-19 in New York State was 732 people. I
can hardly concentrate in daytime. At night, I read Annie Ernaux's ``The
Years.'' You can feel the pulse and intelligence of Ernaux's mind, her
technical facility, the range of her assessments over several decades of
French history. The book, which mixes history with memoir, is good
writing. Eventually, there will be good writing about our moment as
well. If journalism is the first rough draft of history, perhaps a
journal is the first rough draft of literature. But grief makes me sour.
I feel as though I've read the same piece of white writing 30 times in
the past month.

Much of it is concerned with inconveniences, and some of it is jokey. I
understand these collective attempts at lightness, but I quarrel with
them, because I know that in the United States there is no
``collective.'' Levity in the midst of sorrow can be a consolation if
the sorrow is shared to begin with. But here, where everything is
divided, where the unscathed can't quite believe the wounded, the levity
sounds like anything but solidarity. Covid-19 was initially heralded as
a great equalizer, and there was some evidence of this in some
countries. But it arrived in America and immediately became American:
classist, capitalist, complacent.

The words Samuel Beckett wrote to his friend Alan Schneider in 1963 feel
like a lifeline: ``I offer you only my deeply affectionate and
compassionate thoughts and wish for you only that the strange thing may
never fail you, whatever it is, that gives us the strength to live on
and on with our wounds.''

Saturday, April 18

---

Yesterday's death toll from Covid-19 in New York State was 804 people.
There's consolation in the falling numbers, sure, but I mostly think
about how, with better leadership, far fewer people would be bereaved
right now. But leaders are nothing without their followers, and many
Americans have decided to inhabit an alternate reality. What is newly
shocking is realizing that their fanciful reality is impervious to
everything, even a horrifying daily death toll.

It began to rain just before I slept last night. This morning there was
snow on the ground in patches. K had read the weather report. To protect
her roses, she had brought in what she could and had covered what she
couldn't. The tulips out front look sad. The small birds are at the
feeder, but there's no squirrel to bother them. It's almost a week since
I saw the cardinal. My sister has her first week off in a very long
time. Being a doctor means witnessing suffering and death, but this past
month has been something else. I feel visceral relief for her, as though
I were the one on vacation.

I want to weep. I can't weep; I can't write either. Is shock necessarily
naïve? Can't it also be evidence of taking in, and being affected by,
new information?

I'm still thinking about ``Zoom funerals.'' There's good reason the
earliest surviving traces of many civilizations have to do with the
burial of the dead: tombs, barrows, tumuli. In Sophocles' ``Antigone,''
King Creon denies funeral rites to Polynices. Polynices is dead anyway
and won't know in what manner his body is disposed of --- but Antigone
knows, and we know, and it is what the living know that matters. Our
need for proper ritual will never subside.

We are eager to find out what an old text can say to our new situation.
But ``Antigone'' won't tell you what to do in the time of Covid-19. The
play is about individual conscience against the state, loyalty to
family, funerary customs, the clash between two varieties of
self-contradiction and, above all, the workings of tragedy. Tragedy is
not simply that something bad happens; it's that one thing leads to
another: if this, then that, and if that, then the other thing. In
Seamus Heaney's version of ``Antigone,'' ``The Burial at Thebes,'' he
has the chorus declare: ``It starts like an undulation underwater,/A
surge that hauls black sand up off the bottom,/Then turns itself into a
tidal current. . . .''

One thing leads to another. Polynices dies in battle, and Antigone, for
defying the directive to leave him unburied, is sentenced to death. She
hangs herself. Haemon, to whom she was engaged, kills himself with a
sword. Eurydice, grieving her son, also commits suicide. Hubris,
cruelty, and next thing you know, an entire generation is brought to
grief. We can see the tidal current and the wreckage in its wake; but
why has it happened? All we know is that different choices would have
led to a different outcome.

Teju Cole \emph{teaches writing at Harvard. His photobook "Fernweh" was
published in February.} Brian Rea \emph{is an artist in Los Angeles. His
book ``Death Wins a Goldfish'' was published last year.}

\hypertarget{the-moon}{%
\subsection{``The Moon''}\label{the-moon}}

\hypertarget{collage-on-paper-2020}{%
\subparagraph{Collage on paper, 2020}\label{collage-on-paper-2020}}

\hypertarget{by-lorna-simpson}{%
\subparagraph{By Lorna Simpson}\label{by-lorna-simpson}}

\includegraphics{https://static01.graylady3jvrrxbe.onion/images/2020/05/24/magazine/24MAG-QUARANTINE-ARTWORKS-03/24MAG-QUARANTINE-ARTWORKS-03-master495.jpg}

\hypertarget{the-moon-1}{%
\subsection{``The Moon''}\label{the-moon-1}}

\hypertarget{collage-on-paper-2020-1}{%
\subparagraph{Collage on paper, 2020}\label{collage-on-paper-2020-1}}

\hypertarget{by-lorna-simpson-1}{%
\subparagraph{By Lorna Simpson}\label{by-lorna-simpson-1}}

``I created this work at the end of April in my home in Los Angeles. I
was moved by the sense of endless days flowing one into another, with
the news of death, the continuation of life, the insanity, the
tenderness and the sadness all continuing.'' --- L.S.

\includegraphics{https://static01.graylady3jvrrxbe.onion/images/2020/05/26/magazine/26mag-quarantine-hair/26mag-quarantine-hair-master495.png}

\includegraphics{https://static01.graylady3jvrrxbe.onion/images/2020/05/26/magazine/26mag-quarantine-hair-02/26mag-quarantine-hair-02-master495.png}

\includegraphics{https://static01.graylady3jvrrxbe.onion/images/2020/05/26/magazine/26mag-quarantine-hair-03/26mag-quarantine-hair-03-master495.png}

\includegraphics{https://static01.graylady3jvrrxbe.onion/images/2020/05/26/magazine/26mag-quarantine-hair-04/26mag-quarantine-hair-04-master495.png}

\includegraphics{https://static01.graylady3jvrrxbe.onion/images/2020/05/26/magazine/26mag-quarantine-hair-05/26mag-quarantine-hair-05-master495.png}

\includegraphics{https://static01.graylady3jvrrxbe.onion/images/2020/05/26/magazine/26mag-quarantine-hair-06/26mag-quarantine-hair-06-master495.png}

\includegraphics{https://static01.graylady3jvrrxbe.onion/images/2020/05/26/magazine/26mag-quarantine-hair/26mag-quarantine-hair-master495.png}

\hypertarget{when-the-world-unravels-braid-your-own-hair-ive-worn-my-hair-in-the-same-exact-style-for-six-years-learning-to-style-it-myself-has-brought-some-comfort-of-normalcy}{%
\subsection{\texorpdfstring{When the World Unravels, Braid Your Own Hair
\emph{I've worn my hair in the same exact style for six years. Learning
to style it myself has brought some comfort of
normalcy.}}{When the World Unravels, Braid Your Own Hair I've worn my hair in the same exact style for six years. Learning to style it myself has brought some comfort of normalcy.}}\label{when-the-world-unravels-braid-your-own-hair-ive-worn-my-hair-in-the-same-exact-style-for-six-years-learning-to-style-it-myself-has-brought-some-comfort-of-normalcy}}

By Jazmine Hughes

A few weeks ago, I dreamed that I was scrolling Twitter and came across
a post that stopped my thumb cold: ``CUOMO OPENS BRAIDING SALONS.''
Dream me --- hair loosened into its natural Afro, like real me --- was
relieved, elated. I read the series of tweets detailing all the
precautions that salons and customers would be taking: wiping down
chairs regularly, disinfecting combs, wearing masks. The tweets told me
that I needed to be smart and stay cautious, but that the worst was
over, cosmetologically speaking. After three months, one week and six
days, I could call my braider, Sonia, once again and pay \$150 for her
to spend six to eight hours slowly yanking my hair into dozens of tidy,
light braids, completed mostly during commercial breaks of talk shows,
while I ran down my phone battery and balanced Chinese takeout on my
lap. I couldn't get her on the phone fast enough.

Since adolescence, my hair has been a steady proxy for my anxieties: I
can track extreme stress by the nightmares I've had of sudden hair loss
or a botched chemical job. In the most important ways, I've been spared
by Covid-19 --- my family healthy, my job secure, my lodgings stable ---
but I've still been restless with worry, unable to fall asleep before
dawn. During those sleepless hours, I watched YouTube hair tutorials,
each one titled with a technique (``most natural crochet ever,'' ``easy
protective style'') and a promise (``UNDER TWO HOURS,'' ``easy enough
for beginners''). Maybe I could learn something new.

I've never known how to do much with my hair because I never wanted to
learn: Somewhere in middle school I got it in my head that smart women
didn't care about their looks, and all photos from the following decade
prove that I took that to heart. As a child, I bonded with my mother or
grandmother or sisters on nights when I could harangue them into
wrapping my chemically straightened hair under a scarf. In college, I'd
visit a salon during academic breaks, then persuade my friends to ``get
the back of my head'' when I later touched up my roots with relaxer, but
mostly I kept my hair short. For a majority of the past decade, I've let
a braider handle my hair, rarely styling it myself beyond slicking it up
into a bun or just letting it assume its own shape.

\includegraphics{https://static01.graylady3jvrrxbe.onion/images/2020/05/26/magazine/26mag-quarantine-hair-02/26mag-quarantine-hair-02-master495.png}

The pernicious theory that natural black hair is difficult to take care
of is only true if you're extremely lazy, which I am. I regard doing my
hair regularly with the same enthusiasm as brushing my teeth or filing
expenses --- banalities that counterbalance the dance parties and first
dates and beach days of life. Accordingly, I've worn my hair in the same
exact style for six years. I first had it braided on a whim one summer
afternoon and quickly decided that I'd never do it again. But then I
bungled removing the braids, getting the extensions so knotted in with
my actual hair that I had to cut out both, after which I decided that
having braids for a while might be nice, actually. I've maintained them
through changes in addresses and dress sizes and jobs and relationships,
through tattoos added on my arms and nose rings extracted from my face,
the most enduring change to my body I'd ever paid for. Every morning and
evening, at home or away, in darkness or in light, sick, sad, drunk, I
have looked at myself and seen the braids too.

\includegraphics{https://static01.graylady3jvrrxbe.onion/images/2020/05/26/magazine/26mag-quarantine-hair-03/26mag-quarantine-hair-03-master495.png}

But confined in my apartment for the past two months, I recognize very
little about my life. The faces of my friends, now lit by their computer
screens, seem slightly off, distorted by grief or fatigue or technology;
my neighborhood is unnaturally hushed, devoid of the charcoal smokers
dotting the sidewalk that mark the onset of nice weather; my handwriting
has deteriorated from disuse. A few days before the stay-at-home orders
began in New York, I took my braids down in preparation for my next
appointment, leaving my hair to erupt into springy sections. So in the
mirror now, my image is never quite what I expect to see. I feel mildly
disoriented, as if I had followed a recipe but somehow the dish didn't
come out the way it looked in the picture. I had worn braids not for
vanity but for control: without them, my hair, emboldened with its own
direction, often looks the way it wants to, which can be different every
day. With them, I look exactly the same, day in and day out, and I don't
have to do a thing.

\includegraphics{https://static01.graylady3jvrrxbe.onion/images/2020/05/26/magazine/26mag-quarantine-hair-04/26mag-quarantine-hair-04-master495.png}

During the first few days of self-isolation, my hair was my victory
garden: I tended to it, feeding it protein and moisture, oil and cremes,
throwing myself into its maintenance just to give myself something to
do. I clarified it with Aztec clay, I rinsed it with apple-cider
vinegar, I softened it with shea butter, I grew it with biotin. Over
time, though, the instinct to settle my idle hands decreased, and in its
place arose the desire for self-recognition. It's not that I don't like
how I appear in the mirror now but rather that I've had very little say
in the matter. Our personal choices have eroded; in isolation, I'd
expected to miss all the normal things, like joining my friends late at
dinner or running into someone on the subway, but I've been surprised by
the other things: putting on eyeliner, wearing a new perfume, the
temporary alterations that feel like a necessary part of autonomy.

\includegraphics{https://static01.graylady3jvrrxbe.onion/images/2020/05/26/magazine/26mag-quarantine-hair-05/26mag-quarantine-hair-05-master495.png}

Style, then, feels like the last indulgence. Economists cite the
lipstick effect, a glossy theory that, in times of upheaval, sales of
small indulgences like cosmetics increase --- hard-won treats that
lessen the impact. In the past few months, by some counts, nail-polish
-purchases- have soared, most likely to people eager for a diversion.
(Or maybe because we're staring at our hands more than ever.) The
results of our bodies left to their own devices --- leg hair unwaxed,
bangs overgrown, grays returning --- seem striking because we can't fix
what we don't like as easily as before. Most of the time, we can look
more or less how we want, within good reason, but self-isolation has
forced aesthetic choices on us, reducing us to mere bodies awaiting the
inevitable.

\includegraphics{https://static01.graylady3jvrrxbe.onion/images/2020/05/26/magazine/26mag-quarantine-hair-06/26mag-quarantine-hair-06-master495.png}

The interim between what we look like now, dead-eyed and bushy all over,
and how we might want to look feels like a vacation that has gone on for
too long, its appeal loosening by the day. Some people have resorted to
the drastic, shaving and dyeing and bleaching and growing out, sprinting
toward the unknown. But for the most part, all I wanted is for something
to resemble the way it appeared to me before.

Holed up in my home, I did not feel the same, a state I had limited
control over, but I also did not look the same, and I figured that was
the one thing I could change. In the middle of the night, I am convinced
I can do anything, and after a few YouTube videos, I figured I could
try, for the first time, to spend an hour blowing out my hair, another
braiding it into eight neat cornrows that lay flat against my skull,
braid those braids into a single braid, then loop 100 faux dreadlocks
into each cornrow. The women on my screen made it look quick and easy;
my insomnia reasoned that it gave me something to do.

So one Saturday morning, I angled my laptop on my bathroom sink, a
tutorial queued up --- in the harsh reality of day --- to do a measly
two cornrows. My first attempt went poorly --- I tried to get the braids
to adhere to the length of my scalp, but they crept off the nape of my
neck no matter how many times I restarted. (One comment to the tutorial
read: ``Shout-out to all my black girls who still don't understand how
to cornrow even after watching all the videos on YouTube.'') I wore them
for four days anyway, a giddy black Pippi Longstocking. I began idly
braiding my hair while on the phone or watching TV, the plaits growing
smaller, my completion time shortening, my fingers becoming used to
their job. My long-held evasion had stopped making sense --- this wasn't
that bad! I speculated about my burgeoning expertise: Maybe I could
start a natural hair YouTube channel myself; maybe I'd be the valued
friend who ``can do hair.'' I stood to make so much money when this was
all over! I looked in the mirror after a blind braiding during a long
phone call. Practice had not made perfect. Practice had barely made OK.

I gave up on the dream of the eight cornrows. The YouTube algorithm led
me to something easier: Marley twists, done with a particular type of
synthetic hair, attached to the scalp with skillfully hidden rubber
bands. They weren't the same as box braids, but they'd provide the same
heft, the approximate length, the framing around my face in the bathroom
mirror. The beauty-supply store in my neighborhood remained open,
stocked with masks and gloves and bundles of hair. The tutorial promised
I could complete it in under two hours, which it underestimated by about
two hours. The final result turned out sloppy and elementary --- the
parts are not well defined, the rubber band is too far away from the
root, I ran out of hair. But I am comforted by its weight. Without
thinking, I've restarted my habit of gathering up the strands and tying
them into a bun on my head, grasping at anything familiar.

\emph{Illustration by Brian Rea.}

Jazmine Hughes \emph{is a story editor for the magazine. She last wrote
about learning how to swim.} Brian Rea \emph{is an artist in Los
Angeles. His book ``Death Wins a Goldfish'' was published last year.}

\hypertarget{beneath-an-unforgiving-sun}{%
\subsection{``Beneath an Unforgiving
Sun''}\label{beneath-an-unforgiving-sun}}

\hypertarget{oil-on-canvas-2020}{%
\subparagraph{Oil on canvas, 2020}\label{oil-on-canvas-2020}}

\hypertarget{by-titus-kaphar}{%
\subparagraph{By Titus Kaphar}\label{by-titus-kaphar}}

\includegraphics{https://static01.graylady3jvrrxbe.onion/images/2020/05/24/magazine/24MAG-QUARANTINE-ARTWORKS/24MAG-QUARANTINE-ARTWORKS-master495-v2.jpg}

\hypertarget{beneath-an-unforgiving-sun-1}{%
\subsection{``Beneath an Unforgiving
Sun''}\label{beneath-an-unforgiving-sun-1}}

\hypertarget{oil-on-canvas-2020-1}{%
\subparagraph{Oil on canvas, 2020}\label{oil-on-canvas-2020-1}}

\hypertarget{by-titus-kaphar-1}{%
\subparagraph{By Titus Kaphar}\label{by-titus-kaphar-1}}

``The only real thing that exists in the painting is the absence, the
hole. When I stepped back from the painting, I felt as if I wanted to
understand the story more. Where have these children gone? What is the
story here? The trauma that these women are conveying is one that can
exist in the context of the 1960s or in the context of Covid-19. It's
not that I set out to talk about this moment. But to some degree, loss
is loss.'' --- T.K.

\includegraphics{https://static01.graylady3jvrrxbe.onion/images/2020/05/26/magazine/26mag-quarantine-oconnell/26mag-quarantine-oconnell-master495.png}

\includegraphics{https://static01.graylady3jvrrxbe.onion/images/2020/05/26/magazine/26mag-quarantine-oconnell/26mag-quarantine-oconnell-master495.png}

\hypertarget{you-can-be-homesick-at-home-the-lockdown-revealed-an-uncanny-and-alienating-version-of-my-surroundings}{%
\subsection{\texorpdfstring{You Can Be Homesick at Home \emph{The
lockdown revealed an uncanny and alienating version of my
surroundings.}}{You Can Be Homesick at Home The lockdown revealed an uncanny and alienating version of my surroundings.}}\label{you-can-be-homesick-at-home-the-lockdown-revealed-an-uncanny-and-alienating-version-of-my-surroundings}}

By Mark O'Connell

About a week into lockdown, I was walking to a park near my house in
Dublin for my state-sanctioned daily constitutional when I encountered,
kneeling on the sidewalk, two middle-aged men in leather jackets, their
shaved heads bowed and gleaming in the light of the midday sun. As I
stepped into the street to give these men their space, I saw that they
were kneeling at the entrance to an Orthodox church, their hands clasped
before them in supplication. I kept walking, but I couldn't put the men
out of my mind. I kept seeing them: eyes closed, kneeling, praying to a
locked door.

At that time, I was waking up in the morning and, before I was even
conscious, I was aware of a bad feeling in my stomach, both very hard
and very tender. It was a feeling I couldn't quite identify, but one
that reminded me of something I hadn't felt in a long time. Suspecting I
might feel better if I got some exercise, I bought a bike on the
internet. I had never really cycled around Dublin before. I had barely
cycled at all since I was a kid. Now that there were so few cars on the
roads, it seemed like a good time to start, and cycling as an activity
suddenly seemed to represent the highest form of freedom available. I
cycled through the streets of the becalmed city, marveling at this place
I had lived in my whole adult life, everything at once perfectly
familiar and unrecognizably strange. I was curious about this uncanny
place that was my home, but I was also curious about this feeling I had,
the hard and tender pain in my gut. I wasn't just sad --- it was
something else, like a complication of the underlying condition of
sadness.

I cycled around Trinity College, and past the National Library, where
before the virus hit I used to work several days a week. I cycled by
Hodges Figgis, the large bookstore on Dawson Street where I had, over
the years, spent hundreds of aimless hours browsing. Everything was shut
and dark. But it wasn't just that. The whole city suddenly felt like a
massive Potemkin village, as though its former vitality and noise and
humor had been an illusion, and the streets and the buildings were now
revealed as an elaborate facade. It was somehow hard to believe any of
it had ever been real.

I cycled through Temple Bar, a neighborhood normally lousy with
tourists, my bike rattling over the slick cobbles of the empty streets.
In the darkened window of an art-book store, among a display of zines
and photography books, I saw a sign that read: ``Home is where your
bookshop is.'' I looked at this dumbly for a while, and understood at
last what the feeling was: It was homesickness. I was homesick.

I was a clingy child, and homesickness was a major theme of my early
years. Sleepovers at friends' houses invariably ended prematurely, not
long after bedtime, with a call to my parents, who would come and get me
and drive me home in my pajamas. I didn't like having to do this, and
the older I got the more ashamed I felt about it. One of my earliest
memories: I was maybe 3 or 4, and my parents went away for a few days,
and my older sister and I stayed with friends of the family. Their house
had a room we children were not allowed into --- a so-called ``good
room,'' reserved for entertaining guests, a common-enough thing in Irish
houses. Somehow, I conflated the room that I couldn't enter with the
parents I couldn't see; in my mind, this room was where they had
disappeared to. I remember standing outside in the front garden, peering
through the gap between the curtains into that locked room and trying to
catch a glimpse inside, of my parents, of home.

Something of this memory --- of standing in front of a shut door with a
sense that home was somehow behind it --- clung to the experience of
cycling through Dublin during lockdown. I was homesick for the place I
was still in. But how could I be homesick? If anything, surely I was the
\emph{opposite} of homesick. I was sick of being at home, of having to
stay within the two-kilometer radius around my house, beyond which I was
legally restrained from venturing. How could I be homesick when I was
spending almost all of my time in my house, with the people I loved most
in the world?

But I was not at home. This was an odd thing to realize, given that in a
strictly technical sense I clearly was at home. But I knew that I was
not, because if I had been at home, I would have been in a different
city. I would have been relishing some overheard snatch of absurdity on
the street. I would have been irritated by a busker on Grafton Street
belting out ``My Heart Will Go On'' on the panpipes. I would have been
doing things whose former banality now seemed impossibly idyllic, like
wondering whether going for one more takeout flat white would be
overdoing it; deciding it would be, and then going for it anyway. Or
running into a friend on the street, and then running into a second
friend while I was still catching up with the first.

But none of these things were possible right now. I was home, was the
thing, but I was not at home. I was no more at home than those men on
the sidewalk were at church. I was kneeling in front of a locked door,
feeling that home was somewhere behind it. I was peering through a
window, desperate for a glimpse of everything I missed so badly.

\emph{Illustration by Brian Rea.}

Mark O'Connell \emph{is a writer based in Dublin. His last article for
the magazine about the Chernobyl Exclusion Zone was adapted from his
recent book "Notes From an Apocalypse."} ** Brian Rea \emph{is an artist
in Los Angeles. His book ``Death Wins a Goldfish'' was published last
year.}

\includegraphics{https://static01.graylady3jvrrxbe.onion/images/2020/05/26/magazine/26mag-quarantine-dusting/26mag-quarantine-dusting-master495.png}

\includegraphics{https://static01.graylady3jvrrxbe.onion/images/2020/05/26/magazine/26mag-quarantine-dusting/26mag-quarantine-dusting-master495.png}

\hypertarget{youre-never-alone-in-a-dusty-apartment-your-dust-is-you-and-the-life-outside-your-window-and-the-life-of-every-tenant-before-you-gross--but-who-right-now-can-turn-down-company}{%
\subsection{\texorpdfstring{You're Never Alone in a Dusty Apartment
\emph{Your dust is you, and the life outside your window, and the life
of every tenant before you. Gross --- but who right now can turn down
company?}}{You're Never Alone in a Dusty Apartment Your dust is you, and the life outside your window, and the life of every tenant before you. Gross --- but who right now can turn down company?}}\label{youre-never-alone-in-a-dusty-apartment-your-dust-is-you-and-the-life-outside-your-window-and-the-life-of-every-tenant-before-you-gross--but-who-right-now-can-turn-down-company}}

By Jamie Lauren Keiles

A new intrusive thought to consider: On average, the human body sheds
its entire bag of skin --- more than a billion cells --- every 28 days.
Since the beginning of self-quarantine, I have shed my husk more than
twice, casting off thousands of skin cells each second. Every morning,
after my waking groan, I sat up half-awake in bed, watching the shards
of my disintegrating self turning over in a tone-deaf ray of sun. Free
of the virus, that was how I was dying --- incrementally, and mainly
from self-pity. Others are not as lucky, I thought, with the grandeur of
a person who fights by doing nothing, whose suffering takes up negative
space. The neighbors banged pots and pans to alleviate their guilt. I
thought about skin as it floated through the air, landed on the bureau,
remade itself as dust.

Dust, as a collective noun, means almost nothing. The word is suggestive
of dryness and smallness --- tiny unknown pieces, combined with other
pieces, which remain to be discovered like the bottom of the sea. There
is road dust and coal dust and stardust and space dust. Household dust,
a lively field of scientific study, is thought to contain domestic life
in microcosm: mainly sloughed-off skin and hair, but also sweater fibers
and pet dander, dried-out bugs and tracked-in outdoor dirt. A zillion
motes of dust make up a singular ``dust bunny.'' Like a fine wine or the
neuroses of siblings, each is unique to its own environment. My dust is
me, and the friends I can't have over. Yours is you, and the life
outside your window, and the life of every tenant before you. Gross ---
but who right now can turn down company?

The word for both adding and removing dust is ``dust.'' I dusted my way
through self-quarantine with a box of dollar-store dryer sheets,
encountering my home on a microscopic scale, as nothing but motes
landing on horizontal planes. Experts say you should dust from high to
low, letting the dust from the higher-up realms settle in the lower-down
ones before you clean them. This is a pretty good housekeeping tip,
presupposing the duster believes that cleanliness is the most desirable
outcome. Right now I don't know what I believe. In quarantine, I dust
for distraction. I dust the baseboards. I dust the dark side of the fan
blades. I dust the tops of the light bulbs in my lamps. I dust for a
universe I can control. The thing about dusting is it is endless. Even
as you dust, you make dust.

The management of household dust was a petty bourgeois fixation from the
start. The chore was first emphasized in the mid-1800s, as dust from the
just-industrialized street was beginning to threaten the newly sealed
middle-class home. In her 1966 book, ``Purity and Danger,'' the British
anthropologist Mary Douglas described this encroachment as matter ``out
of place.'' Cleanliness was next to godliness, they said then, as they
do now. Dirt symbolized a collapse of social order. The maintenance of
this order usually fell on women, whose level of commitment to dust
management symbolized moral piety, and later, in the inverse,
liberation. In her 1949 book, ``The Second Sex,'' Simone de Beauvoir
suggested that a fixation on housework offered women a flight from
themselves and the world. ``Few tasks are more like the torture of
Sisyphus than housework,'' she wrote. ``The clean becomes soiled, the
soiled is made clean, over and over, day after day.'' In de Beauvoir's
view, the life of Sisyphus was torture. At least he got to go outside.

Self-quarantine has me thinking and acting in all kinds of backward
ways. Victorian metaphors are tantalizing, with cosseted elites at home
baking bread as labor outside is ruthlessly exploited. Even as
scientific thinking has evolved, a facile sense of germ theory prevails,
with particulate life understood in stubbornly moral terms. Inside the
house, the dust bunnies are friendly. Outside, there's the wet malintent
of disease. People used to gossip about social behaviors, but now
microbial rumors spread: The virus thrives on your reusable grocery
bags; it can leap from the mouth of an oncoming jogger and hang in the
air for up to a week. All we get told for sure is wash your hands. Log
on to check the death count every day and wonder when it will finally be
over. At least with the dusting, there is evidence of progress.

As I sit and write, a new layer of dust accumulates. Later on this
evening, I'll make another round, Swiffering the baseboards, wiping down
the ledges, dragging a sock along the porcelain toilet tank. The
underside of a dusty cloth has the cozy, filthy horror of a yellow
Q-tip. Better out than in! Coping mechanisms are always some combination
of perverse and pathetic. Through dusting I've found a means of control,
a form of endlessness I can contain.

Right now, the world has only two scales --- big and small --- and both
go on forever. I am so, so tired of endlessness: the unrelenting
boredom, the cycles of self-pity, the constant systemic breakdown, the
eternity of death. I long to think about big, dumb things that have an
end: a steak from a restaurant, the nave in a church, a hug from a
friend of a friend, the Grand Canyon.

\emph{Illustration by Brian Rea.}

\emph{Jamie Lauren Keiles is a contributing writer for the magazine
whose last article was about the film adaptation of the musical
``Cats.''} Brian Rea \emph{is an artist in Los Angeles. His book ``Death
Wins a Goldfish'' was published last year.}

\includegraphics{https://static01.graylady3jvrrxbe.onion/images/2020/05/26/magazine/26mag-quarantine-intro-04/26mag-quarantine-intro-04-master495.png}

\includegraphics{https://static01.graylady3jvrrxbe.onion/images/2020/05/26/magazine/26mag-quarantine-intro/26mag-quarantine-intro-master495-v2.png}

\includegraphics{https://static01.graylady3jvrrxbe.onion/images/2020/05/26/magazine/26mag-quarantine-intro-07/26mag-quarantine-intro-07-master495-v2.png}

\includegraphics{https://static01.graylady3jvrrxbe.onion/images/2020/05/26/magazine/26mag-quarantine-intro-04/26mag-quarantine-intro-04-master495.png}

\hypertarget{the-truth-about-cocoons-what-caterpillars-really-go-through-in-there-has-applications-for-our-moment}{%
\subsection{\texorpdfstring{The Truth About Cocoons \emph{What
caterpillars really go through in there has applications for our
moment.}}{The Truth About Cocoons What caterpillars really go through in there has applications for our moment.}}\label{the-truth-about-cocoons-what-caterpillars-really-go-through-in-there-has-applications-for-our-moment}}

By Sam Anderson

Every child knows about cocoons. They're one of the first things we
learn about the natural world. What the ABCs are to language, cocoons
are to biology. They introduce us to the wonder of metamorphosis: A
little blobby squirmy thing disappears into a sac and emerges as a
flamboyant colorful flappy thing. Magic! In fact, that is exactly how we
tend to learn about it --- as one of nature's great magic tricks, if not
inexplicable then largely unexplained. The emphasis always seems to be
on the before and the after, never the during.

Lately, I have found myself wondering --- as I sit here hunched inside
my dark house, for infinity weeks, hardly moving, wearing the same green
sweatshirt while eating the same four snacks --- about cocoons. I don't
really care anymore what goes in or what comes out; those are questions
for different times, for ancient pasts and distant futures. I'm
interested in precisely the part of the story that tends to be skipped:
the confinement, the waiting, the darkness, the change.

What is it actually like inside a cocoon? Is it cozy and peaceful? Or
cramped and dim? Is the bug's stay voluntary, involuntary or something
in between? And what really happens during that seemingly magical
change? Is it inspiring and wondrous? Or is it unpleasant and grim? What
did I not learn in kindergarten?

\includegraphics{https://static01.graylady3jvrrxbe.onion/images/2020/05/26/magazine/26mag-quarantine-intro/26mag-quarantine-intro-master495-v2.png}

It turns out that the inside of a cocoon is --- at least by
outside-of-a-cocoon standards --- pretty bleak. Terrible things happen
in there: a campaign of grisly desolation that would put most horror
movies to shame. What a caterpillar is doing, in its self-\/-imposed
quarantine, is basically digesting itself. It is using enzymes to reduce
its body to goo, turning itself into a soup of ex-caterpillar --- a
nearly formless sludge oozing around a couple of leftover essential
organs (tracheal tubes, gut).

\includegraphics{https://static01.graylady3jvrrxbe.onion/images/2020/05/26/magazine/26mag-quarantine-intro-07/26mag-quarantine-intro-07-master495-v2.png}

Only after this near-total self-annihilation can the new growth begin.
Inside that gruesome mush are special clusters of cells called
``imaginal discs,'' which sounds like something from a Disney movie but
which I have been assured is actual biology. Imaginal discs are
basically the seeds of crucial butterfly structures: eyes, wings,
genitalia and so on. These parts gorge themselves on the protein of the
deconstructed caterpillar, growing exponentially, taking form, becoming
real. That's how you get a butterfly: out of the horrid meltdown of a
modest caterpillar.

\textbf{The caterpillar I know} best was created, out of scraps of
painted paper, by a man who grew up in Nazi Germany: Eric Carle. ``The
Very Hungry Caterpillar'' is a small, eccentric masterpiece --- a
children's book, yes, but also a formative exploration of the complex
nature of change. Like many children, I read the book so many times that
it blended into my psyche as a kind of background color. Years later, I
read it again so many times with my own children that it blended into
their psyches too, and back into my own again, this time in a deeper
color.

Recently, when I reread ``The Very Hungry Caterpillar'' alone, in
quarantine, it felt less like reading than remembering. And yet I was
also slightly surprised. Carle's caterpillar is, of course, more than
just a caterpillar; it is a classic existentialist antihero --- a lonely
creature of pure need, guided by only its own ravenousness, skirting the
knife's edge between self-destruction and growth. What surprised me this
time was that so little of the book is devoted to that final act of
transformation --- the magic change we are always taught to anticipate.
Mainly, the book is a catalog of eating. Ridiculous, ceaseless,
maniacal, necessary but also unwise eating.

Things start out fine. The caterpillar tunnels through one food after
another: an apple on Monday, two pears on Tuesday, three plums on
Wednesday and so forth. He always leaves a little hole in his wake, a
literal hole in the book's pages --- his hunger is so powerful that it
begins to destroy the very story of his hunger. By Saturday, the
caterpillar has abandoned all sense and started bingeing on junk food: a
piece of chocolate cake, an ice cream cone, a pickle, a slice of cherry
pie, a cupcake --- the kind of caloric intake, surely, that will not
help an insect with its ultimate long-term goals. Sure enough, the
caterpillar ends up curled at the bottom of the page with a stomachache,
eyes squished and drooping, mouth compressed to a miserable line.

This moment in the book hit me deeply. Staring into that caterpillar's
wretched face felt like looking into a mirror. In quarantine, I, too,
have been ravenously bingeing. When I pass through a room, I leave it
practically riddled with holes. I have worked my way through bags of
chips, bags of gummy candy, bags of unsalted walnuts, jars of peanut
butter, tubs of ice cream, boxes of cereal. And the consumption goes far
beyond food. Along with the rest of America, I devoured the cursed
docu-series ``Tiger King'' with hardly a break. I have been playing so
many hours of the Nintendo game ``Animal Crossing'' that my wife
sometimes peeks cautiously into the room and asks: ``Is there some way I
can help you? Are you sure you want to still be playing this?''

I am speaking to you now from the inside of my cocoon. You, presumably,
are listening from inside your own. It could be an apartment, a room, a
closet or a farmhouse out in the middle of nowhere on the site of an
abandoned commune. Perhaps, like me, you have been living a
self-destructive quarantine cocoon life. Maybe you are not using junk
food or video games to numb the pain, but you are probably using
something. The data suggests that most of us are. A friend recently sent
me his phone's weekly screen-time report. It was down 15 percent, to 10
hours 50 minutes a day. Utility companies say that people are sleeping
in --- spikes in energy use that typically occurred around 6 a.m. are
now happening much later. The very shape of society is dissolving in
front of our eyes. ``Workdays,'' ``mealtimes,'' ``holidays'' --- all
these rituals that were designed to help us shape our days have now been
drained of meaning. Society has become, in large part, boneless. A soup
of ex-society.

Watching it happen, all over the planet, has been horrible. There is no
other word for it. It has been horrible inside overrun hospitals --- the
multiplying hopelessness, the gasping of those who can't be saved. It
has been differently horrible inside our own houses, where we wonder how
to help as we refresh internet spreadsheets and watch the numbers rise:
cases, tests, deaths. It has been horrible to read about the stories
behind those numbers. It has been horrible to watch for tiny signs of
illness in the people we love the most. Horrible to feel our own
foreheads, wondering if we are warm. Horrible to be warm. Horrible to be
cool. Horrible not to be sure.

How do we even begin to process all of this --- this cataclysm that is
happening simultaneously in slow motion and all at once, on distant
continents and inside our own cells? Months in, we still have no idea
when it will end or what we will all come out looking like. The
metamorphoses are happening mostly in private, all over the place, in
billions of individual pods --- acts of internal self-\/-destruction and
rebuilding, subtle shifts and whole revolutions.

I keep asking myself, as pretzel crumbs spill down the front of my
sweatshirt, what I have learned from quarantine. On most days, honestly,
it feels like nothing. The main question we seem to answer in quarantine
is circular: How would we handle ourselves in quarantine? On
particularly bad days, as I blink myself awake at 11 a.m. and promptly
reopen Twitter, I feel that I've actually learned less than nothing ---
that my knowledge is rolling backward. Before the world shut down, I
swear on my Cool Ranch Doritos, I was working on myself, exercising
regularly and making plans and sticking to them. I was generating lists,
and lists of lists, and lists of lists of lists. I listened to a whole
audiobook about forming good habits. But now all of that has gone away.
For the moment at least, in the face of this horror and sadness and
confusion, as I watch the curves grow worse on the terrifying graphs, my
fantasies of self-\/-improvement have evaporated. Maybe they weren't
essential. Maybe they needed to be rethought. Or maybe they just needed
to go dormant for a while.

\textbf{A caterpillar doesn't} choose to go into its cocoon. ``Cocoon,''
actually, in the case of butterflies, is the wrong word. My apologies to
Eric Carle, but it is actually a chrysalis, a hard shell that was inside
the caterpillar's body the whole time. From the ancient Greek chrysos,
``gold'' --- a golden envelope of internal self-\/-destruction. In order
to expose the chrysalis, the caterpillar just has to slough off its
chubby outer layer. It seals itself inside itself. No decision is really
made; it's just a matter of hormonal cascades --- cues beyond the
understanding of the creature being cued. The caterpillar stays inside
because it has to. And when it comes out, it is a different thing. We
tell that story one way. The caterpillar no doubt experiences it very
differently.

Is a butterfly's life any better than a caterpillar's? Was all that
suffering in the cocoon worth it? Is a volcano happier after it erupts?
These are children's questions. It doesn't even make sense to ask. The
bug has no choice. The cocoon is forced upon it. And then nature runs
its course.

But here is one thing I have learned in quarantine: The world can stop.
This is not a small thing, nor is it easy. It is not necessarily good or
bad. History teaches us, again and again, that change does not mean
progress, that metamorphosis does not mean improvement. When society
dissolves, we --- the ones who make up society --- dissolve right along
with it. When our familiar structures become unrecognizable, we become
unrecognizable to ourselves. This means suffering. We are confused and
miserable and terrified and heart-broken. We find ourselves continually
annoyed with our families. We sleep too long or not long enough, eat too
much or not enough. We stop exercising and attend absurd virtual
meetings meant to simulate work or school. We clap outside at 7 p.m.

But I believe that the situation, as bleak as it may seem, may also
contain the seeds of its own transformation --- somewhere very deep in
the formless mush.

Let's not forget that in our horrendous confusion --- in spite of it,
because of it --- we managed to do something amazing. We chose to go
dormant. We changed almost everything in the world, almost overnight.
This required a kind of collective action that, frankly, would have
struck me as impossible five months ago. There are, of course, outliers,
loudmouths and nihilists and malcontents. They will always exist. But
enough people are not that. Enough of us have found enough reasons to
change, and it has made an actual difference. We are in the middle of
creating whatever the new world will be. We did it, and we are doing it,
every day.

Meanwhile, I am still sitting here in my chrysalis, ravenous, sad,
confused, feeling simultaneously changed and unchanged. Perhaps one of
these afternoons, many months from now, I will be nibbling away at
whatever happens to be in front of me, and it will turn out not to be
more gummy bears but something else, the actual wall of my enclosure,
and I will eat enough to make a hole, and then I will look out, with a
whole new kind of eyes, to see what sort of world is waiting for me, and
what I have become in it.

\emph{Illustration by Brian Rea.}

Sam Anderson \emph{is a staff writer for the magazine and the author of
``Boom Town,'' a book about Oklahoma City. He last wrote a profile of
the musician Weird Al Yankovic.} Brian Rea \emph{is an artist in Los
Angeles. His book ``Death Wins a Goldfish'' was published last year.}

What We've Learned in Quarantine

\begin{itemize}
\item
  \protect\hyperlink{willi-ruge-photographs}{}

  \hypertarget{something-happens-when-you-fall}{%
  \subparagraph{Something Happens When You
  Fall}\label{something-happens-when-you-fall}}
\item
  \protect\hyperlink{insanity}{}

  \hypertarget{insanity-can-keep-you-sane}{%
  \subparagraph{Insanity Can Keep You
  Sane}\label{insanity-can-keep-you-sane}}
\item
  \protect\hyperlink{nature}{}

  \hypertarget{the-comfort-of-common-creatures}{%
  \subparagraph{The Comfort of Common
  Creatures}\label{the-comfort-of-common-creatures}}
\item
  \protect\hyperlink{paris}{}

  \hypertarget{finding-belonging-in-exile}{%
  \subparagraph{Finding Belonging in
  Exile}\label{finding-belonging-in-exile}}
\item
  \protect\hyperlink{recovery}{}

  \hypertarget{there-can-be-abundance-in-deprivation}{%
  \subparagraph{There Can Be Abundance in
  Deprivation}\label{there-can-be-abundance-in-deprivation}}
\item
  \protect\hyperlink{diary}{}

  \hypertarget{historys-first-draft}{%
  \subparagraph{History's First Draft}\label{historys-first-draft}}
\item
  \protect\hyperlink{hair}{}

  \hypertarget{my-hair-my-self}{%
  \subparagraph{My Hair, My Self}\label{my-hair-my-self}}
\item
  \protect\hyperlink{homesickness}{}

  \hypertarget{homesickness-at-home}{%
  \subparagraph{Homesickness at Home}\label{homesickness-at-home}}
\item
  \protect\hyperlink{dusting}{}

  \hypertarget{the-companionship-of-dust}{%
  \subparagraph{The Companionship of
  Dust}\label{the-companionship-of-dust}}
\item
  \protect\hyperlink{cocoons}{}

  \hypertarget{the-truth-about-cocoons}{%
  \subparagraph{The Truth About Cocoons}\label{the-truth-about-cocoons}}
\end{itemize}

\includegraphics{https://static01.graylady3jvrrxbe.onion/images/2020/05/26/magazine/26mag-quarantine-ruge/26mag-quarantine-ruge-master495.png}

\includegraphics{https://static01.graylady3jvrrxbe.onion/images/2020/05/26/magazine/26mag-quarantine-insanity-04/26mag-quarantine-insanity-04-master495.png}

\includegraphics{https://static01.graylady3jvrrxbe.onion/images/2020/05/26/magazine/26mag-quarantine-nature-04/26mag-quarantine-nature-04-master495-v3.png}

\includegraphics{https://static01.graylady3jvrrxbe.onion/images/2020/05/26/magazine/26mag-quarantine-paris-03/26mag-quarantine-paris-03-master495.png}

\includegraphics{https://static01.graylady3jvrrxbe.onion/images/2020/05/26/magazine/26mag-quarantine-jamison-2/26mag-quarantine-jamison-2-master495.png}

\includegraphics{https://static01.graylady3jvrrxbe.onion/images/2020/05/24/magazine/24mag-diary-04/24mag-diary-04-master495.png}

\includegraphics{https://static01.graylady3jvrrxbe.onion/images/2020/05/26/magazine/26mag-quarantine-hair/26mag-quarantine-hair-master495.png}

\includegraphics{https://static01.graylady3jvrrxbe.onion/images/2020/05/26/magazine/26mag-quarantine-oconnell/26mag-quarantine-oconnell-master495.png}

\includegraphics{https://static01.graylady3jvrrxbe.onion/images/2020/05/26/magazine/26mag-quarantine-dusting/26mag-quarantine-dusting-master495.png}

\includegraphics{https://static01.graylady3jvrrxbe.onion/images/2020/05/26/magazine/26mag-quarantine-intro-04/26mag-quarantine-intro-04-master495.png}

Group 126

Group 362

What We've Learned in Quarantine

Read 3 Comments

\begin{itemize}
\item
\item
\item
\item
\end{itemize}

Advertisement

\protect\hyperlink{after-bottom}{Continue reading the main story}

\hypertarget{site-index}{%
\subsection{Site Index}\label{site-index}}

\hypertarget{site-information-navigation}{%
\subsection{Site Information
Navigation}\label{site-information-navigation}}

\begin{itemize}
\tightlist
\item
  \href{https://help.nytimes3xbfgragh.onion/hc/en-us/articles/115014792127-Copyright-notice}{©~2020~The
  New York Times Company}
\end{itemize}

\begin{itemize}
\tightlist
\item
  \href{https://www.nytco.com/}{NYTCo}
\item
  \href{https://help.nytimes3xbfgragh.onion/hc/en-us/articles/115015385887-Contact-Us}{Contact
  Us}
\item
  \href{https://www.nytco.com/careers/}{Work with us}
\item
  \href{https://nytmediakit.com/}{Advertise}
\item
  \href{http://www.tbrandstudio.com/}{T Brand Studio}
\item
  \href{https://www.nytimes3xbfgragh.onion/privacy/cookie-policy\#how-do-i-manage-trackers}{Your
  Ad Choices}
\item
  \href{https://www.nytimes3xbfgragh.onion/privacy}{Privacy}
\item
  \href{https://help.nytimes3xbfgragh.onion/hc/en-us/articles/115014893428-Terms-of-service}{Terms
  of Service}
\item
  \href{https://help.nytimes3xbfgragh.onion/hc/en-us/articles/115014893968-Terms-of-sale}{Terms
  of Sale}
\item
  \href{https://spiderbites.nytimes3xbfgragh.onion}{Site Map}
\item
  \href{https://help.nytimes3xbfgragh.onion/hc/en-us}{Help}
\item
  \href{https://www.nytimes3xbfgragh.onion/subscription?campaignId=37WXW}{Subscriptions}
\end{itemize}
