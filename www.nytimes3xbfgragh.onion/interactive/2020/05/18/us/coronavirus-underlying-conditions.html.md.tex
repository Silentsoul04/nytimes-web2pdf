Sections

SEARCH

\protect\hyperlink{site-content}{Skip to
content}\protect\hyperlink{site-index}{Skip to site index}

\href{https://www.nytimes3xbfgragh.onion/section/us}{U.S.}

\href{https://myaccount.nytimes3xbfgragh.onion/auth/login?response_type=cookie\&client_id=vi}{}

\href{https://www.nytimes3xbfgragh.onion/section/todayspaper}{Today's
Paper}

\href{/section/us}{U.S.}\textbar{}Where Chronic Health Conditions and
Coronavirus Could Collide

\url{https://nyti.ms/2LEZxS1}

\begin{itemize}
\item
\item
\item
\item
\item
\item
\end{itemize}

\hypertarget{the-coronavirus-outbreak}{%
\subsubsection{\texorpdfstring{\href{https://www.nytimes3xbfgragh.onion/news-event/coronavirus?name=styln-coronavirus-national\&region=TOP_BANNER\&block=storyline_menu_recirc\&action=click\&pgtype=Interactive\&impression_id=9ff1a4f0-f2a9-11ea-b26b-4f4bf8e5c9c3\&variant=undefined}{The
Coronavirus
Outbreak}}{The Coronavirus Outbreak}}\label{the-coronavirus-outbreak}}

\begin{itemize}
\tightlist
\item
  live\href{https://www.nytimes3xbfgragh.onion/2020/09/09/world/covid-19-coronavirus.html?name=styln-coronavirus-national\&region=TOP_BANNER\&block=storyline_menu_recirc\&action=click\&pgtype=Interactive\&impression_id=9ff1cc00-f2a9-11ea-b26b-4f4bf8e5c9c3\&variant=undefined}{Latest
  Updates}
\item
  \href{https://www.nytimes3xbfgragh.onion/interactive/2020/us/coronavirus-us-cases.html?name=styln-coronavirus-national\&region=TOP_BANNER\&block=storyline_menu_recirc\&action=click\&pgtype=Interactive\&impression_id=9ff1cc01-f2a9-11ea-b26b-4f4bf8e5c9c3\&variant=undefined}{Maps
  and Cases}
\item
  \href{https://www.nytimes3xbfgragh.onion/interactive/2020/science/coronavirus-vaccine-tracker.html?name=styln-coronavirus-national\&region=TOP_BANNER\&block=storyline_menu_recirc\&action=click\&pgtype=Interactive\&impression_id=9ff1cc02-f2a9-11ea-b26b-4f4bf8e5c9c3\&variant=undefined}{Vaccine
  Tracker}
\item
  \href{https://www.nytimes3xbfgragh.onion/2020/09/02/your-money/eviction-moratorium-covid.html?name=styln-coronavirus-national\&region=TOP_BANNER\&block=storyline_menu_recirc\&action=click\&pgtype=Interactive\&impression_id=9ff1cc03-f2a9-11ea-b26b-4f4bf8e5c9c3\&variant=undefined}{Eviction
  Moratorium}
\item
  \href{https://www.nytimes3xbfgragh.onion/2020/09/09/upshot/coronavirus-surprise-test-fees.html?name=styln-coronavirus-national\&region=TOP_BANNER\&block=storyline_menu_recirc\&action=click\&pgtype=Interactive\&impression_id=9ff1cc04-f2a9-11ea-b26b-4f4bf8e5c9c3\&variant=undefined}{Surprise
  Test Fees}
\end{itemize}

Advertisement

\protect\hyperlink{after-top}{Continue reading the main story}

\hypertarget{comments}{%
\subsection{\texorpdfstring{\protect\hyperlink{commentsContainer}{Comments}}{Comments}}\label{comments}}

\href{}{Where Chronic Health Conditions and Coronavirus Could
Collide}\href{}{Skip to Comments}

The comments section is closed. To submit a letter to the editor for
publication, write to
\href{mailto:letters@NYTimes.com}{\nolinkurl{letters@NYTimes.com}}.

\hypertarget{where-chronic-health-conditions-and-coronavirus-could-collide}{%
\section{Where Chronic Health Conditions and Coronavirus Could
Collide}\label{where-chronic-health-conditions-and-coronavirus-could-collide}}

By \href{https://www.nytimes3xbfgragh.onion/by/nadja-popovich}{Nadja
Popovich},
\href{https://www.nytimes3xbfgragh.onion/by/anjali-singhvi}{Anjali
Singhvi} and Matthew ConlenMay 18, 2020

\begin{itemize}
\item
\item
\item
\item
\item
  \emph{310}
\end{itemize}

Where people

are more at risk

for severe

Covid-19 illness

As the new coronavirus continues to spread over the next months,
\href{https://www.nytimes3xbfgragh.onion/interactive/2020/04/30/opinion/coronavirus-covid-vaccine.html}{and
maybe even years}, it could exact a heavy new toll in areas of the
United States that have not yet seen major outbreaks but have high rates
of diabetes, obesity, high blood pressure and other chronic health
conditions.

Large parts of the South and Appalachia are especially vulnerable,
according to a health-risk index created for The New York Times by
\href{https://www.policymap.com/2020/05/policymap-covid19-quick-maps/}{PolicyMap},
a company that analyzes local health data. The index for the first time
identifies counties with high rates of the underlying conditions that
\href{https://www.cdc.gov/mmwr/volumes/69/wr/mm6915e3.htm}{increase
residents' risk of becoming severely ill} if they are infected with the
coronavirus.

Even in lower-risk counties, a significant proportion of the population
is living with these conditions.

WASH.

Covid-19 health risk

MONT.

MAINE

N.D.

Lower

Higher

MINN.

ORE.

VT.

N.H.

WIS.

IDAHO

S.D.

N.Y.

MASS.

WYO.

MICH.

R.I.

CONN.

IOWA

PA.

NEB.

NEV.

N.J.

OHIO

ILL.

MD.

IND.

UTAH

DEL.

D.C.

COLO.

W.VA.

CALIF.

VA.

KAN.

West Virginia has not

yet seen a significant

outbreak, but it has

one of the most

medically vulnerable

populations.

MO.

KY.

N.C.

TENN.

OKLA.

ARIZ.

ARK.

N.M.

S.C.

Much of Colorado has a lower relative risk for severe Covid-19 illness,
but more than 20 percent of the adult population is estimated to be
obese.

~

~

ALA.

GA.

MISS.

LA.

TEXAS

FLA.

ALASKA

Alabama has a high

burden of chronic health

conditions. In many

counties, more than 40

percent of adults are

estimated to be obese.

HAWAII

Covid-19 health risk

Lower

Higher

3

1

2

West Virginia has not yet seen a significant outbreak, but

it has one of the most medically vulnerable populations.

1

Alabama has a high burden of chronic health conditions. In many
counties, more than 40 percent of adults are estimated to be obese.

2

Much of Colorado has a lower relative risk for severe Covid-19 illness,
but more than 20 percent of the adult population is estimated to be
obese.

3

WASH.

Covid-19 health risk

MONT.

MAINE

N.D.

Lower

Higher

MINN.

ORE.

VT.

N.H.

WIS.

IDAHO

S.D.

N.Y.

MASS.

WYO.

MICH.

R.I.

CONN.

IOWA

PA.

NEB.

NEV.

N.J.

OHIO

ILL.

MD.

IND.

UTAH

DEL.

D.C.

COLO.

W.VA.

CALIF.

West Virginia

has not yet

seen a

significant

outbreak, but it

has one of the

most medically

vulnerable

populations.

VA.

KAN.

MO.

KY.

N.C.

TENN.

ARIZ.

OKLA.

ARK.

N.M.

S.C.

Much of Colorado has a lower

relative risk for severe Covid-19

illness, but more than 20

percent of the adult population

is estimated to be obese.

ALA.

GA.

MISS.

LA.

TEXAS

FLA.

ALASKA

Alabama has a high burden

of chronic health conditions.

In many counties, more than

40 percent of adults are

estimated to be obese.

HAWAII

Note: The health risk shown for a given county is compared with the
national average.

Public health experts warn that these areas may not be adequately
prepared for new waves of infection, even as some have lifted
restrictions meant to curb the spread of the virus.

``Places that have not seen a lot of infection yet should be thinking
about what infection is going to mean once they have an outbreak
there,'' said Micaela E. Martinez, a professor at Columbia University's
Mailman School of Public Health.

``This infection is highly contagious and we have no vaccine, so it will
inevitably sweep through our populations unless we have very tight
measures in place to prevent that from happening,'' Dr. Martinez said.
Once it does, the overall health of a community will matter, she added.

The map above shows where U.S residents are at increased risk for severe
Covid-19 illness, compared with the national average. It is based on the
estimated proportion of adults in each county who have one or more of
these conditions: diabetes, high blood pressure, obesity, heart disease
and chronic lung disease, using survey data from the Centers for Disease
Control and Prevention.

A
\href{https://www.nytimes3xbfgragh.onion/2020/04/23/health/coronavirus-patients-risk.html}{majority
of patients} hospitalized with Covid-19 in the New York City area, an
early epicenter of the nation's outbreak, had one or more underlying
health conditions. Studies from the C.D.C. and others suggest that, once
infected with the coronavirus, people with such conditions are at
particular risk for severe illness, including hospitalization and death.
The conditions do not on their own increase a person's chance of
catching the disease.

\hypertarget{high-risk-counties-where-coronavirus-cases-are-growing}{%
\subsubsection{High-Risk Counties Where Coronavirus Cases Are
Growing}\label{high-risk-counties-where-coronavirus-cases-are-growing}}

\includegraphics{https://static01.graylady3jvrrxbe.onion/newsgraphics/2020/05/05/underlying-conidtions-maps/285b87a8f8b4316698542969359e7b75c0b957dc/table-key.png}

County

Cases per 10k

One-week growth rate

Health risk

Obesity

Diabetes

High blood pressure

Lung disease

Heart disease

Notes: The health risk shown for a given county is compared with the
national average. The average daily growth rate of total coronavirus
cases is for the seven-day period ending May 14, 2020. Counties with at
least 20 total reported cases and positive growth rates are shown. A
person may have more than one health condition; numbers are rounded.

The index does not include age as a separate risk factor, though several
of these conditions are more common in older people. The new coronavirus
has been particularly brutal for seniors, who are hospitalized and die
at higher rates than younger people with the virus. But underlying
health conditions appear to increase the risk of serious illness across
age groups.

In all,
\href{https://www.medrxiv.org/content/10.1101/2020.05.02.20088781v1}{m}\href{https://www.medrxiv.org/content/10.1101/2020.05.02.20088781v1}{ore
than half
o}\href{https://www.medrxiv.org/content/10.1101/2020.05.02.20088781v1}{f
U.S. adults} have at least one condition that increases their risk of
becoming seriously ill if infected.

``It's important to recognize that it's not some `other people' that are
vulnerable,'' said Caitlin M. Rivers, an assistant professor at Johns
Hopkins Center for Health Security. ``The kinds of conditions that put
people at higher risk are extremely common.''

But while many people have these conditions, there are important
disparities by age, race and income.

Heart disease and hypertension are more common among older people.
Obesity rates are
\href{https://www.cdc.gov/nchs/data/databriefs/db288.pdf}{similar across
age groups} in America, but
\href{https://www.cdc.gov/pcd/issues/2019/18_0579.htm}{higher among
black
and}\href{https://www.cdc.gov/pcd/issues/2019/18_0579.htm}{Hispanic
people} than among white people.

Lower-income groups and communities of color face significant health
disparities, including higher rates of chronic health conditions, as
well as lower health insurance rates and more limited access to health
care.

Such disparities reflect historical and current inequalities, said Nancy
Krieger, a professor of social epidemiology at the Harvard T.H. Chan
School of Public Health, including a long-term lack of resources and
opportunity.

And these same groups may also be at higher risk for catching the virus.
Black and Hispanic workers
\href{https://www.cdc.gov/coronavirus/2019-ncov/need-extra-precautions/racial-ethnic-minorities.html}{more
frequently} have jobs classified as critical or otherwise lack the
ability to stay home, and they more often live in multigenerational and
crowded homes where it is harder to isolate if infected with the
coronavirus. ``It's a compounding of risk,'' Dr. Krieger said.

In New York City, the impact of coronavirus ---
\href{https://www1.nyc.gov/assets/doh/downloads/pdf/imm/covid-19-deaths-race-ethnicity-05142020-1.pdf}{both
in cases and deaths} --- has been disproportionately concentrated in
\href{https://time.com/5821212/coronavirus-low-income-communities/}{lower-income
neighborhoods} and
\href{https://www1.nyc.gov/assets/doh/downloads/pdf/imm/covid-19-deaths-race-ethnicity-04082020-1.pdf}{communities
of color}. Similar racial and socioeconomic inequalities have been seen
in outbreaks across the country.

Identifying high-risk communities could help policymakers better
anticipate future outbreaks.

``If you're an official or city planner or hospital administrator, you
want to know what to expect when the disease hits,'' said Dr. Leora
Horwitz, a co-author of a study that evaluated underlying conditions
among coronavirus patients hospitalized in the NYU Langone Health
System. Or, she added, if it hits again.

Dr. Krieger of Harvard said data on health risk should inform targeted
interventions and public health messaging. ``Can we pinpoint where more
testing needs to be done or made accessible and affordable?'' she said.
``Where is the greatest need for community outreach, for contact
tracing?''

In the end, keeping new coronavirus infection and death rates low for
people at high health risk requires keeping them low over all, Dr.
Horwitz said. ``The more disease is in the community, the more disease
among people at high risk, period.''

Methodology

\href{https://www.policymap.com/}{PolicyMap} created the Covid-19 health
risk index using the estimated county-level prevalence of five
underlying health conditions: obesity, diabetes, high blood pressure,
heart disease and chronic obstructive pulmonary disease (COPD). These
conditions put people at higher risk for serious Covid-19 illness,
according to the
\href{https://www.cdc.gov/coronavirus/2019-ncov/need-extra-precautions/people-at-higher-risk.html?CDC_AA_refVal=https\%3A\%2F\%2Fwww.cdc.gov\%2Fcoronavirus\%2F2019-ncov\%2Fspecific-groups\%2Fhigh-risk-complications.html}{C.D.C.}
Other conditions, like kidney disease, also increase the risk of severe
Covid-related sickness but are less prevalent in the population. Asthma
was identified as an at-risk condition by the C.D.C. early on, but data
on asthma risk
\href{https://www.nytimes3xbfgragh.onion/2020/04/16/health/coronavirus-asthma-risk.html}{has
been inconsistent} so far, so the condition is not included in this
index.

PolicyMap estimated county-level prevalence for high blood pressure,
COPD and heart disease using a statistical analysis of state-level
health data from the C.D.C.'s 2017 Behavioral Risk Factor Surveillance
System survey. Estimates for diabetes and obesity at the county level
are from the C.D.C.'s U.S. Diabetes Surveillance System. PolicyMap
combined these estimates, normalized by population, to get a single risk
index for each county. The index does not account for people who have
more than one condition. It assesses the overall health burden in a
community and is not a representation of the number of people with these
health conditions. The data is available for download
\href{https://www.policymap.com/issues/covid-19/\#download-data}{here}.

Coronavirus case data is from a
\href{https://www.nytimes3xbfgragh.onion/article/coronavirus-county-data-us.html}{New
York Times database} of reports from state and local health agencies and
hospitals, as of May 14, 2020. Population figures are from the 2014-18
American Community Survey.

Additional work by Josh Katz.

Read 310 Comments

\begin{itemize}
\item
\item
\item
\item
\end{itemize}

Advertisement

\protect\hyperlink{after-bottom}{Continue reading the main story}

\hypertarget{site-index}{%
\subsection{Site Index}\label{site-index}}

\hypertarget{site-information-navigation}{%
\subsection{Site Information
Navigation}\label{site-information-navigation}}

\begin{itemize}
\tightlist
\item
  \href{https://help.nytimes3xbfgragh.onion/hc/en-us/articles/115014792127-Copyright-notice}{©~2020~The
  New York Times Company}
\end{itemize}

\begin{itemize}
\tightlist
\item
  \href{https://www.nytco.com/}{NYTCo}
\item
  \href{https://help.nytimes3xbfgragh.onion/hc/en-us/articles/115015385887-Contact-Us}{Contact
  Us}
\item
  \href{https://www.nytco.com/careers/}{Work with us}
\item
  \href{https://nytmediakit.com/}{Advertise}
\item
  \href{http://www.tbrandstudio.com/}{T Brand Studio}
\item
  \href{https://www.nytimes3xbfgragh.onion/privacy/cookie-policy\#how-do-i-manage-trackers}{Your
  Ad Choices}
\item
  \href{https://www.nytimes3xbfgragh.onion/privacy}{Privacy}
\item
  \href{https://help.nytimes3xbfgragh.onion/hc/en-us/articles/115014893428-Terms-of-service}{Terms
  of Service}
\item
  \href{https://help.nytimes3xbfgragh.onion/hc/en-us/articles/115014893968-Terms-of-sale}{Terms
  of Sale}
\item
  \href{https://spiderbites.nytimes3xbfgragh.onion}{Site Map}
\item
  \href{https://help.nytimes3xbfgragh.onion/hc/en-us}{Help}
\item
  \href{https://www.nytimes3xbfgragh.onion/subscription?campaignId=37WXW}{Subscriptions}
\end{itemize}
