Sections

SEARCH

\protect\hyperlink{site-content}{Skip to
content}\protect\hyperlink{site-index}{Skip to site index}

\hypertarget{comments}{%
\subsection{\texorpdfstring{\protect\hyperlink{commentsContainer}{Comments}}{Comments}}\label{comments}}

\href{}{Sports Have Been on Pause. It's Time for a Reboot.}\href{}{Skip
to Comments}

The comments section is closed. To submit a letter to the editor for
publication, write to
\href{mailto:letters@NYTimes.com}{\nolinkurl{letters@NYTimes.com}}.

\hypertarget{sports-have-been-on-pause-its-time-for-a-reboot}{%
\section{Sports Have Been on Pause. It's Time for a
Reboot.}\label{sports-have-been-on-pause-its-time-for-a-reboot}}

By The New York Times Sports StaffJuly 22, 2020

\begin{itemize}
\item
\item
\item
\item
\item
  \emph{158}
\end{itemize}

\includegraphics{https://static01.graylady3jvrrxbe.onion/newsgraphics/2020/07/08/sports-reboot/34344eaf8f63f48ec0368433ad270adfd25eca8c/baseball_sm.png}

\includegraphics{https://static01.graylady3jvrrxbe.onion/newsgraphics/2020/07/08/sports-reboot/34344eaf8f63f48ec0368433ad270adfd25eca8c/basketball_sm.png}

\includegraphics{https://static01.graylady3jvrrxbe.onion/newsgraphics/2020/07/08/sports-reboot/34344eaf8f63f48ec0368433ad270adfd25eca8c/football_sm.png}

\includegraphics{https://static01.graylady3jvrrxbe.onion/newsgraphics/2020/07/08/sports-reboot/34344eaf8f63f48ec0368433ad270adfd25eca8c/hockey_sm.png}

\includegraphics{https://static01.graylady3jvrrxbe.onion/newsgraphics/2020/07/08/sports-reboot/34344eaf8f63f48ec0368433ad270adfd25eca8c/tennis_sm.png}

\includegraphics{https://static01.graylady3jvrrxbe.onion/newsgraphics/2020/07/08/sports-reboot/34344eaf8f63f48ec0368433ad270adfd25eca8c/golf_sm.png}

\includegraphics{https://static01.graylady3jvrrxbe.onion/newsgraphics/2020/07/08/sports-reboot/34344eaf8f63f48ec0368433ad270adfd25eca8c/soccer_sm.png}

\hypertarget{sports-have-been-on-pause}{%
\section{Sports Have Been on Pause.}\label{sports-have-been-on-pause}}

It's Time for a Reboot.

By The Sports Desk\\
Illustrations by Eden Weingart\\
July 22, 2020

\includegraphics{https://static01.graylady3jvrrxbe.onion/newsgraphics/2020/07/08/sports-reboot/34344eaf8f63f48ec0368433ad270adfd25eca8c/mobilecover.png}

In every sport, athletes and teams use the timeout to strategize and, if
needed, to reboot. They catch their breath. They regroup. They correct
weaknesses and alter game plans. Now the entire sports world has had the
ultimate timeout, a once-in-a-lifetime chance to reconsider its product
before the games resume. Now is the time for bold improvements --- or
whimsical ones. How should each sport rethink itself moving forward?

Here are some ideas for how to improve baseball, basketball, football,
hockey, tennis, golf, and soccer. We'd also like to hear from you.

\begin{itemize}
\item
\item
\item
\item
\item
  \emph{158}
\end{itemize}

In All Sports

\textbf{Realign all leagues and conferences with quick travel in mind.}

Long flights are bad for the environment and potential incubators of
illness. Create better geographic clusters to resurrect and build
intense rivalries. (You, too, college sports, you conference-sprawling
mess.) **** Picture a California-only division in baseball, or the Jets
and Giants in the same division.

\textbf{Overhaul the way sports are watched.}

It's time to modernize ordinary television viewing beyond the elevated
sideline views and muffled crowd noise we've had for decades. Put more
overhead cameras on cables. Put tiny cameras on players, and even on
officials. Mic up everyone. Sure, give it an R rating and use a
seven-second delay so opponents don't steal plays. But immerse the fans
in new and inventive ways.

\textbf{Simplify the rules that always create controversies and explain
them.}

The average fan can no longer accurately say if something is a catch in
football, a travel in basketball, a balk or a checked swing in baseball,
a legal check in hockey, an illegal tackle in soccer. If you can't
adequately explain and consistently enforce the rules, rewrite them in a
way in which you can.

\textbf{Make fields and courts bigger, or remove some players, and give
the artists more canvas on which to create.}

Many of the field dimensions and team sizes in sports go back to an era
when athletes were not as fast, big and optimally trained. Imagine the
much-loved Olympic-size rink for the N.H.L., a wider court in the
N.B.A., the use of the alleys in singles tennis, or a 10-on-10 N.F.L.

In M.L.B.

\includegraphics{https://static01.graylady3jvrrxbe.onion/packages/flash/multimedia/ICONS/transparent.png}

\includegraphics{https://static01.graylady3jvrrxbe.onion/newsgraphics/2020/07/08/sports-reboot/34344eaf8f63f48ec0368433ad270adfd25eca8c/baseball.png}

\textbf{Speed up the game, but for real.}

Eliminate mound visits. Reduce warm-up pitches. Stick to two-minute gaps
between innings. Keep batters in the box and pitchers on the mound.

\textbf{Limit the incentive to substitute by reducing game-day rosters.}

There are nine players to a side, but baseball has 26-man rosters. Why
not have a reduced game-day roster, like the N.F.L. and N.H.L., chosen
before each game. Maybe 18? Then we'll see how many pitching changes are
really necessary.

\textbf{Eliminate the designated hitter.}

The momentum seems to be moving toward a D.H. in both leagues, but
baseball is a more strategic game when pitchers hit, and there's no
reason pitchers can't be good hitters. (Most were stellar hitters in
their youth, and, really, there's not enough time for batting practice
in those four days between starts?) Force managers to make hard
decisions. Your best hitter can't play the field? Prop him up somewhere
and work around it.

\textbf{If you own a team that finishes last in the division three years
in a row, you and your family must divest entirely.}

But the team stays put.

\textbf{If leagues can have a salary cap, they can have a concessions
cap, too.}

Tie ticket and concessions prices to the team's current record or its
last championship. Last place? Your beer is a buck.

\textbf{Enforce the strike zone that is in the rule book.}

Children are taught that strikes are from the knees to the nipples, but
major leaguers cannot handle anything above the belt? Get batters
swinging. And if umpires cannot do the job consistently, move to
automatic ball-strike technology.

\includegraphics{https://static01.graylady3jvrrxbe.onion/packages/flash/multimedia/ICONS/transparent.png}

\includegraphics{https://static01.graylady3jvrrxbe.onion/newsgraphics/2020/07/08/sports-reboot/34344eaf8f63f48ec0368433ad270adfd25eca8c/batflip.png}

\textbf{Encourage bat flips.}

There is nothing in the rulebook prohibiting bat flips and fist pumps
--- just your grandfather's old-fashioned sense of decorum.

\textbf{Make the bases bigger.}

Just a few inches will decrease the distance between bases and help
bring speed back into the game.

\textbf{Give outfielders a chance to make over-the-wall catches.}

Mandate that every stadium have a gap between the fans and the outfield
wall to prevent fans from interfering with home runs. The highlights
industry thanks you in advance.

\textbf{Every team must have a mascot.}

That means you, Yankees, Dodgers and Angels. And those sausage and
president races (and electronic races on the scoreboard) must have a
payoff for the fans. Assign a favorite, and then give out food if it
wins.

In Pro Basketball

\includegraphics{https://static01.graylady3jvrrxbe.onion/packages/flash/multimedia/ICONS/transparent.png}

\includegraphics{https://static01.graylady3jvrrxbe.onion/newsgraphics/2020/07/08/sports-reboot/34344eaf8f63f48ec0368433ad270adfd25eca8c/basketball.png}

\textbf{You wouldn't cut away from the climax of a movie, so why keep
doing it during a big game?}

More than other major sports, the N.B.A.'s drama is mostly found in the
final two minutes --- a span that can take 20 minutes in real time. The
N.B.A. is addressing this, slowly, by limiting late timeouts. But keep
the cameras rolling, and the audience in the arena, with no commercials.

\includegraphics{https://static01.graylady3jvrrxbe.onion/packages/flash/multimedia/ICONS/transparent.png}

\includegraphics{https://static01.graylady3jvrrxbe.onion/newsgraphics/2020/07/08/sports-reboot/34344eaf8f63f48ec0368433ad270adfd25eca8c/fourpointline.png}

\textbf{What's more thrilling than a 3-pointer? How about a 4-pointer?}

Players like Damian Lillard and Stephen Curry have stretched the floor
and the imagination by being efficient scorers from beyond 30 feet.
Reward them.

\textbf{Create a soccer-style cup competition.}

Take the best thing about college basketball --- the N.C.A.A. tournament
--- to the pros by creating a knockout competition to run concurrent
with the dog days of January and February. With 30 teams, a handful of
first-round byes produces a neat 32-team bracket. Knock five games off
the regular-season to account for it, and play it over three weekends.
Cap it with a grand final four at a neutral site, and call it the Kobe
Bryant Cup, since the soccer-loving Bryant would have reveled in it.

\textbf{Make ``posterize'' more than a metaphor.}

Players dunked on memorably must be made into actual posters, to be
handed out to children at the next game.

\textbf{Allow hanging on the rim after a dunk.}

If you hang on the rim after a dunk, sure, great flex --- but you only
hurt yourself by not getting back on defense.

\textbf{Reseed the playoffs after every round.}

The N.B.A. postseason is, essentially, a predetermined bracket, not
unlike the N.C.A.A. tournament or a tennis major. But when upsets
happen, the bracket remains rigid. Follow the N.F.L. and reward
regular-season success by reseeding after each round, always pitting the
remaining team with the best record against the team with the worst.

\textbf{The trophy tour should not be limited to hockey.}

One of the coolest traditions in sports is the one in which the Stanley
Cup spends one day with every player, coach, trainer and manager from
the winning team. Champions everywhere should be afforded the same
privilege. Imagine N.B.A. players and their social-media accounts during
a day with the Larry O'Brien Trophy.

In the N.F.L.

\includegraphics{https://static01.graylady3jvrrxbe.onion/packages/flash/multimedia/ICONS/transparent.png}

\includegraphics{https://static01.graylady3jvrrxbe.onion/newsgraphics/2020/07/08/sports-reboot/34344eaf8f63f48ec0368433ad270adfd25eca8c/football.png}

\textbf{Lose headset communications.}

Limit the interaction with the sidelines and press boxes, where coaches
overscript the action, and put athletes back in charge of the action.
Better yet: Let quarterbacks call all the plays, drawn in the dirt.
(Note: Need dirt.)

\textbf{Get serious about helmet technology and brain injuries.}

**** Every shot to (and from) the head must be penalized --- even
``accidental'' ones --- until they're rare. And it's long past time that
the N.F.L., which admits to the game's damaging effects on brains,
invests fully in a moonshot-type reimagining of the helmet. The hard
shell was designed decades ago to prevent skull fractures, not
concussions. Go full marshmallow?

\includegraphics{https://static01.graylady3jvrrxbe.onion/packages/flash/multimedia/ICONS/transparent.png}

\includegraphics{https://static01.graylady3jvrrxbe.onion/newsgraphics/2020/07/08/sports-reboot/34344eaf8f63f48ec0368433ad270adfd25eca8c/footballtee.png}

\textbf{Eliminate the kickoff, which adds nothing but time and an
increased risk of violent collisions.}

Any decent N.F.L. kicker is a guaranteed touchback, anyway.

\textbf{Allow an untimed play as an alternative to the onside kick.}

Owners tabled this proposal last month after robust discussion. For
teams wanting the ball back, an untimed fourth-and-15 play from their
25-yard line would spice up late-and-close situations, allowing skill
--- and not, say, the vagaries of a bouncing, prolate spheroid --- to
help decide games.

\textbf{Make every draft virtual, eliminating war rooms and green
rooms.}

The N.F.L. draft, especially, is an extravagant, made-for-television
spectacle, which is why the scaled-back edition from April felt more
captivating, wholesome, authentic and, yes, watchable. Give us more
coaches in their basement and more dogs, and fewer bro-hugs with
Commissioner Roger Goodell.

\textbf{Close the Pandora's box of instant replay}.

If there is a challenge, give a three-person panel of retired officials
60 seconds to vote. Majority wins, and we move on. Let lingering
disputes be contested as they should be: with screaming arguments and
temper tantrums.

In the N.H.L.

\includegraphics{https://static01.graylady3jvrrxbe.onion/packages/flash/multimedia/ICONS/transparent.png}

\includegraphics{https://static01.graylady3jvrrxbe.onion/newsgraphics/2020/07/08/sports-reboot/34344eaf8f63f48ec0368433ad270adfd25eca8c/hockey.png}

\textbf{End the season in May.}

Most years, the N.B.A. and Stanley Cup finals are played concurrently in
June, but the N.B.A. usually gets at least double the audience and
triple the attention. Hockey has no great history with an 82-game
schedule. Reduce it to 70 games, as it was through most years of the
Original Six. Then start the playoffs a month earlier and finish them in
May, when the spotlight is yours, mostly.

\textbf{No Fighting.}

It's time. We know too much to let it continue. Throw a punch and get
ejected, the way it is in every other sport.

\textbf{Reduce goalie pads back to rational size.}

Safety first, of course. But leg pads the width of a snow plow? Pads
meant to add bulk, not protection? No.

\textbf{The World Cup is the best format, so use it for other sports,
like hockey.}

Sports are better when they have a sense of jeopardy. The N.H.L. regular
season is too long, and the Stanley Cup playoffs are a slog. Imagine,
after a short season for seeding, a group stage, followed by knockout
rounds and a grand final.

\includegraphics{https://static01.graylady3jvrrxbe.onion/packages/flash/multimedia/ICONS/transparent.png}

\includegraphics{https://static01.graylady3jvrrxbe.onion/newsgraphics/2020/07/08/sports-reboot/34344eaf8f63f48ec0368433ad270adfd25eca8c/trophy.png}

In Tennis

\includegraphics{https://static01.graylady3jvrrxbe.onion/packages/flash/multimedia/ICONS/transparent.png}

\includegraphics{https://static01.graylady3jvrrxbe.onion/newsgraphics/2020/07/08/sports-reboot/34344eaf8f63f48ec0368433ad270adfd25eca8c/tennis.png}

\textbf{You get one serve, not two, as in table tennis.}

Go big if you're feeling frisky. Go easy, like a typical second serve,
if you're hesitant. But players would probably create hybrids, giving
themselves an array of options that would lead to interesting rallies,
which is what we all want. Bonus: One serve per point speeds up matches.

\textbf{No apologizing for net-cord winners.}

Come on. Inside you're laughing. It's not as if you did it on purpose.
And let serves? Play them, if they're in.

\textbf{Five sets for everyone, but with a high-stakes finish.}

The measure of a great match should not be the time it consumes or the
toll it takes on the participants. Let men and women play best-of-five,
but replace the fifth set with a supertiebreaker.

\textbf{Create one rule on in-match coaching.}

Currently, men's players are not allowed to receive it in individual
events, but women are (though not in the Grand Slams). It's time to end
the confusion, and to preserve a precious and unusual tennis tradition:
letting players make tactical and mental adjustments on their own, under
pressure.

\textbf{End the time-sucking habit of between-the-point towel service}.

Players don't need children to fetch a towel between points and then
take the sweaty thing away. It's elitist, gross and time-consuming. Oh,
and viruses.

\includegraphics{https://static01.graylady3jvrrxbe.onion/packages/flash/multimedia/ICONS/transparent.png}

\includegraphics{https://static01.graylady3jvrrxbe.onion/newsgraphics/2020/07/08/sports-reboot/34344eaf8f63f48ec0368433ad270adfd25eca8c/towel.png}

\textbf{Let the players call their own lines and challenge decisions
with electronics.}

Retire the stoic (and fallible) line judges, but keep a human element by
having players make their own calls. If the opponent wants a challenge,
the drama only builds, and the technology exists to quickly get it
right.

In Golf

\includegraphics{https://static01.graylady3jvrrxbe.onion/packages/flash/multimedia/ICONS/transparent.png}

\includegraphics{https://static01.graylady3jvrrxbe.onion/newsgraphics/2020/07/08/sports-reboot/34344eaf8f63f48ec0368433ad270adfd25eca8c/golf.png}

\textbf{Restrict the flight of the golf ball}

Most sports do not let technological advances overtake the essence of
the game --- no aluminum bats in Major League Baseball, for example ---
but modern pros overpower even the world's best golf courses with modern
equipment. The adjustments might differ from tour to tour, but no one
but the pros themselves would notice the difference if the best drives
went 300 yards, not closer to 400.

\textbf{Bring the tours together occasionally.}

Enact what the PGA Tour has long promised --- regular joint events, with
the top men and women playing alongside each other on the same course,
and during the same week. Then award women's and men's titles.

\textbf{Altered formats and team competitions.}

Besides the major championships, the most anticipated events are team
ones like the Ryder Cup, or those with an unusual twist. Throw in some
events with a different scoring format. Let golfers choose a partner and
turn a couple of tournaments into best-ball or scramble events. Escape
the mundane of yet another forgettable stroke-play event.

\textbf{Stop with the shushing.}

If a baseball player can connect on a diving, 95-mile-an-hour cutter
with 55,000 people screaming, a golfer can hit a driver or strike a putt
without complete silence.

In Soccer

\includegraphics{https://static01.graylady3jvrrxbe.onion/packages/flash/multimedia/ICONS/transparent.png}

\includegraphics{https://static01.graylady3jvrrxbe.onion/newsgraphics/2020/07/08/sports-reboot/34344eaf8f63f48ec0368433ad270adfd25eca8c/soccer.png}

\textbf{You're only offside if your whole body is offside.}

No longer can you be offside by a fingernail or a toe or because of your
aura. And you can be called offside by video only if you don't need a
load of lines to prove it.

\textbf{The scoreboard clock is the clock.}

No more of this dance where the referee gets to be the only one in the
stadium who knows how much time is left. Or reduce matches to an hour
but stop the clock when the ball is off the field.

\includegraphics{https://static01.graylady3jvrrxbe.onion/packages/flash/multimedia/ICONS/transparent.png}

\includegraphics{https://static01.graylady3jvrrxbe.onion/newsgraphics/2020/07/08/sports-reboot/34344eaf8f63f48ec0368433ad270adfd25eca8c/orangecard.png}

\textbf{Adopt an orange card.}

An orange card comes into play for fouls that are too serious for a mere
yellow, but not quite so bad that they warrant complete expulsion. They
come with a 10-minute banishment in some sort of penalty box.

\textbf{Stop letting men's soccer run women's soccer.}

There is no reason for women's soccer, a sport experiencing its boom in
the 21st century, to adopt structures and rules created by men in the
Victorian era. Want to tweak some rules? Go for it. Want to have
continental superleagues, as they do in European basketball and the
N.B.A.? Go for that, too. Then create a global club championship for the
top teams in each league.

\textbf{Head injury substitutes.}

If soccer is going to stick with the five substitutes it adopted after
the pandemic, designate at least one as a temporary sub who can be used
(and reused) to give any player with a head injury time for a proper
medical assessment. Ten minutes ought to do it, and then --- if a doctor
clears the injured player --- the manager decides which player gets to
continue.

\textbf{Back up the penalty kick.}

In most leagues, penalty kicks are successful about 75 percent of the
time. Back them up so that they are roughly a 50-50 proposition. Too
many important matches turn on one questionable call and a perfunctory
penalty kick. And shootouts would be more interesting if fans celebrated
the makes, not the misses.

\textbf{Extra-time sudden-death multiball}.

Every two minutes of extra time, each team loses one player and an
additional ball is put into play, until someone scores.

In College Sports

\textbf{Blow them up, entirely.}

Design and production by Michael Beswetherick and Eden Weingart

Read 158 Comments

\begin{itemize}
\item
\item
\item
\item
\end{itemize}

Advertisement

\protect\hyperlink{after-bottom}{Continue reading the main story}

\hypertarget{site-index}{%
\subsection{Site Index}\label{site-index}}

\hypertarget{site-information-navigation}{%
\subsection{Site Information
Navigation}\label{site-information-navigation}}

\begin{itemize}
\tightlist
\item
  \href{https://help.nytimes3xbfgragh.onion/hc/en-us/articles/115014792127-Copyright-notice}{©~2020~The
  New York Times Company}
\end{itemize}

\begin{itemize}
\tightlist
\item
  \href{https://www.nytco.com/}{NYTCo}
\item
  \href{https://help.nytimes3xbfgragh.onion/hc/en-us/articles/115015385887-Contact-Us}{Contact
  Us}
\item
  \href{https://www.nytco.com/careers/}{Work with us}
\item
  \href{https://nytmediakit.com/}{Advertise}
\item
  \href{http://www.tbrandstudio.com/}{T Brand Studio}
\item
  \href{https://www.nytimes3xbfgragh.onion/privacy/cookie-policy\#how-do-i-manage-trackers}{Your
  Ad Choices}
\item
  \href{https://www.nytimes3xbfgragh.onion/privacy}{Privacy}
\item
  \href{https://help.nytimes3xbfgragh.onion/hc/en-us/articles/115014893428-Terms-of-service}{Terms
  of Service}
\item
  \href{https://help.nytimes3xbfgragh.onion/hc/en-us/articles/115014893968-Terms-of-sale}{Terms
  of Sale}
\item
  \href{https://spiderbites.nytimes3xbfgragh.onion}{Site Map}
\item
  \href{https://help.nytimes3xbfgragh.onion/hc/en-us}{Help}
\item
  \href{https://www.nytimes3xbfgragh.onion/subscription?campaignId=37WXW}{Subscriptions}
\end{itemize}
