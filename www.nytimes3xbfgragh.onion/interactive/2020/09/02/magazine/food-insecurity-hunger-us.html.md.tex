Sections

SEARCH

\protect\hyperlink{site-content}{Skip to
content}\protect\hyperlink{site-index}{Skip to site index}

\hypertarget{comments}{%
\subsection{\texorpdfstring{\protect\hyperlink{commentsContainer}{Comments}}{Comments}}\label{comments}}

\href{}{America at Hunger's Edge}\href{}{Skip to Comments}

The comments section is closed. To submit a letter to the editor for
publication, write to
\href{mailto:letters@NYTimes.com}{\nolinkurl{letters@NYTimes.com}}.

\hypertarget{america-at-hungers-edge}{%
\section{America at Hunger's Edge}\label{america-at-hungers-edge}}

Photographs by Brenda Ann KenneallySept. 2, 2020

\begin{itemize}
\item
\item
\item
\item
\item
  \emph{761}
\end{itemize}

A shadow of hunger looms over the United States. In the pandemic
economy, nearly one in eight households doesn't have enough to eat. The
lockdown, with its epic lines at food banks, has revealed what was
hidden in plain sight: that the struggle to make food last long enough,
and to get food that's healthful --- what experts call `food insecurity'
--- is a persistent one for millions of Americans.

\includegraphics{https://static01.graylady3jvrrxbe.onion/newsgraphics/2020/08/17/hunger/assets/images/014_laura_horsbourgh_may_27_parma_oh_002-1440_x2.jpg}

\includegraphics{https://static01.graylady3jvrrxbe.onion/newsgraphics/2020/08/17/hunger/assets/images/035_deborah_sulton__june_25__jackson_ms_004-1440_x2.jpg}

\includegraphics{https://static01.graylady3jvrrxbe.onion/newsgraphics/2020/08/17/hunger/assets/images/034_deidre_lyons__june_24__30__july_4_jackson_ms_005-1440_x2.jpg}

\includegraphics{https://static01.graylady3jvrrxbe.onion/newsgraphics/2020/08/17/hunger/assets/images/021_marolga_hernandez_june_8_chicago_il_003_toned-1440_x2.jpg}

\hypertarget{america-at}{%
\section{America at}\label{america-at}}

Hunger's Edge

Photographs by Brenda Ann Kenneally

September 2, 2020

Beginning in May, Brenda Ann Kenneally set out across the country, from
New York to California, to capture the routines of Americans who
struggle to feed their families, piecing together various forms of food
assistance, community support and ingenuity to make it from one month to
the next.

Food insecurity is as much about the threat of deprivation as it is
about deprivation itself: A food-insecure life means a life lived in
fear of hunger, and the psychological toll that takes. **** Like many
hardships, this burden falls disproportionately on Black and Hispanic
families, who are almost twice as likely to experience food insecurity
as white families.

Troy, N.Y.

May 9

Like so many who live at hunger's edge, the members of the extended
Stocklas family --- whom Kenneally has photographed for years --- gain
and lose food stamps depending on fluctuating employment status in an
unstable economy. They often have trouble stretching their funds to the
end of the month, so they pool resources to provide family-style dinners
for all.

Tap to cycle through images

\includegraphics{https://static01.graylady3jvrrxbe.onion/newsgraphics/2020/08/17/hunger/assets/images/05_kandace__may_9__troy_ny_002-1440_x2.jpg}

\includegraphics{https://static01.graylady3jvrrxbe.onion/newsgraphics/2020/08/17/hunger/assets/images/05_kandace__may_9__troy_ny_007-1440_x2.jpg}

\includegraphics{https://static01.graylady3jvrrxbe.onion/newsgraphics/2020/08/17/hunger/assets/images/05_kandace__may_9__troy_ny_005-1440_x2.jpg}

\includegraphics{https://static01.graylady3jvrrxbe.onion/newsgraphics/2020/08/17/hunger/assets/images/05_kandace__may_9__troy_ny_001-1440_x2.jpg}

When Kandice Zakrzewski, 25, was no longer eligible for food stamps, she
stopped buying Lactaid for her son, Matthew Ratleph, 2. ``We had to give
that up for him. And just say `You can't drink milk.' Or we have to
water it down.''

Just days before Kenneally arrived, the governor closed schools
statewide, creating a new source of stress for food-insecure families,
which often rely on free school lunches to keep their school-age
children fed. This made the family's big collective meals all the more
crucial. ``Even if it's just pitching in \$10 when we don't have food
stamps,'' Kandice Zakrzewski says, ``we all pitch in.''

\includegraphics{https://static01.graylady3jvrrxbe.onion/newsgraphics/2020/08/17/hunger/assets/images/05_kandace__may_9__troy_ny_003-1440_x2.jpg}

Zakrzewski's son Brayden Ratleph, 6.

\includegraphics{https://static01.graylady3jvrrxbe.onion/newsgraphics/2020/08/17/hunger/assets/images/05_kandace__may_9__troy_ny_004-1440_x2.jpg}

Kayla Stocklas. ``We just kind of get our food and just all do our own
thing," Zakrzewski says.

Gary, Ind.

June 6

Late last year, Doris Hall, 63, moved back to Gary, her hometown, to
look after her great-grandchildren --- ``so they don't have to be in
daycare,'' she says. On weekends, she takes in as many as nine of the
children --- occasionally all 14 --- so that their parents can work.

Tap to cycle through images

\includegraphics{https://static01.graylady3jvrrxbe.onion/newsgraphics/2020/08/17/hunger/assets/images/020_doris__june_6__gary_in_005-1440_x2.jpg}

\includegraphics{https://static01.graylady3jvrrxbe.onion/newsgraphics/2020/08/17/hunger/assets/images/020_doris__june_6__gary_in_001-1440_x2.jpg}

\includegraphics{https://static01.graylady3jvrrxbe.onion/newsgraphics/2020/08/17/hunger/assets/images/020_doris__june_6__gary_in_004-1440_x2.jpg}

\includegraphics{https://static01.graylady3jvrrxbe.onion/newsgraphics/2020/08/17/hunger/assets/images/020_doris__june_6__gary_in_002-1440_x2.jpg}

Hall's rules are strict: naptime in the afternoon, bedtime at 9 p.m. and
most important, whatever she cooks, they must eat.

For lunch, it's ``microwaveable stuff,'' like corndogs, hot dogs and
chicken nuggets that Hall picks up at the nearby food bank. Dinners
vary: spaghetti, chicken, soups, tacos. When she has a rare moment to
eat alone, she makes her favorite meal for herself: greens and tacos.

\includegraphics{https://static01.graylady3jvrrxbe.onion/newsgraphics/2020/08/17/hunger/assets/images/020_doris__june_6__gary_in_003-1440_x2.jpg}

Some of Hall's great-grandchildren waiting for lunch.

\includegraphics{https://static01.graylady3jvrrxbe.onion/newsgraphics/2020/08/17/hunger/assets/images/020_doris__june_6__gary_in_006-1440_x2.jpg}

Armani and Kimani Lacy, 5. ``I never liked cooking,'' Hall says, ``but
now that I've been taking care of the grandkids, I stay in the
kitchen.''

In the face of deprivation, food-insecure families often seize any
opportunity to get and store food when it's available.

\includegraphics{https://static01.graylady3jvrrxbe.onion/newsgraphics/2020/08/17/hunger/assets/images/034_deidre_lyons__june_24__30__july_4_jackson_ms_000-1440_x2.jpg}

In the middle of a food desert in Jackson, Miss., a family's freezer
holds as much as it can.

\includegraphics{https://static01.graylady3jvrrxbe.onion/newsgraphics/2020/08/17/hunger/assets/images/012_claire__may_19__erie_pa_001-1440_x2.jpg}

Freezing milk in Erie, Pa., so nothing goes to waste.

\includegraphics{https://static01.graylady3jvrrxbe.onion/newsgraphics/2020/08/17/hunger/assets/images/047_george_ruder__july_23__tucson_az_003-1440_x2.jpg}

In Tucson, Ariz., a ``cheap soup'' made on the stovetop.

\includegraphics{https://static01.graylady3jvrrxbe.onion/newsgraphics/2020/08/17/hunger/assets/images/017_latia_hinkle__june_3__gary_in_006-1440_x2.jpg}

In Gary, Ind., assembling a full meal from individual school-lunch
portions of taco meat.

\includegraphics{https://static01.graylady3jvrrxbe.onion/newsgraphics/2020/08/17/hunger/assets/images/033_patricia_luckett__june_23__jackson_ms_004-1440_x2.jpg}

Stockpiling supplies in Jackson, Miss.

\includegraphics{https://static01.graylady3jvrrxbe.onion/newsgraphics/2020/08/17/hunger/assets/images/017_latia_hinkle__june_3__gary_in_007-1440_x2.jpg}

School lunches, and the single-serving milk cartons they often contain,
are a mainstay for food-insecure households, like this one in Gary, Ind.

Kenneally arrived in Illinois in early June, soon after nationwide
unemployment claims filed during the pandemic had topped 40 million.

Cicero, Ill.

June 8

In Cicero, just west of Chicago, Jennifer Villa, 29, was living in an
apartment with a kitchen that needed plumbing repairs. She and her
family were already struggling --- a disability makes it hard for her to
work --- and the pandemic had meant less fresh food and even longer
pantry lines.

Tap to cycle through images

\includegraphics{https://static01.graylady3jvrrxbe.onion/newsgraphics/2020/08/17/hunger/assets/images/023_jennifer__june_8__chicago_il_001-1440_x2.jpg}

\includegraphics{https://static01.graylady3jvrrxbe.onion/newsgraphics/2020/08/17/hunger/assets/images/023_jennifer__june_8__chicago_il_000-1440_x2.jpg}

\includegraphics{https://static01.graylady3jvrrxbe.onion/newsgraphics/2020/08/17/hunger/assets/images/023_jennifer__june_8__chicago_il_005-1440_x2.jpg}

\includegraphics{https://static01.graylady3jvrrxbe.onion/newsgraphics/2020/08/17/hunger/assets/images/023_jennifer__june_8__chicago_il_004-1440_x2.jpg}

\includegraphics{https://static01.graylady3jvrrxbe.onion/newsgraphics/2020/08/17/hunger/assets/images/023_jennifer__june_8__chicago_il_003-1440_x2.jpg}

Temporarily without a working kitchen, Villa organized the food she
received from food pantries in the alleyway outside her home.

Whenever food deliveries came, Villa's kids would celebrate. ``Oh,
Mommy, we're going to have food tonight,'' they would tell her. ``We're
not going to go to sleep with no food in our tummy.''

\includegraphics{https://static01.graylady3jvrrxbe.onion/newsgraphics/2020/08/17/hunger/assets/images/023_jennifer__june_8__chicago_il_007-1440_x2.jpg}

Armani Rodriguez

\includegraphics{https://static01.graylady3jvrrxbe.onion/newsgraphics/2020/08/17/hunger/assets/images/023_jennifer__june_8__chicago_il_006-1440_x2.jpg}

Sonia Rodriguez

St. Louis

June 12

By June, the social upheavals following the killing of George Floyd
created even more instability for some families. Kenneally visited
Manausha Russ, 28, a few days after protests led to the closure of a
nearby Family Dollar, where Russ used to get basics like milk, cereal
and diapers. ``The stores by my house were all looted,'' she says.

Tap to cycle through images

\includegraphics{https://static01.graylady3jvrrxbe.onion/newsgraphics/2020/08/17/hunger/assets/images/026_manausha_russ__june_11__st_louis_mo_004-1440_x2.jpg}

\includegraphics{https://static01.graylady3jvrrxbe.onion/newsgraphics/2020/08/17/hunger/assets/images/026_manausha_russ__june_11__st_louis_mo_003-1440_x2.jpg}

\includegraphics{https://static01.graylady3jvrrxbe.onion/newsgraphics/2020/08/17/hunger/assets/images/026_manausha_russ__june_11__st_louis_mo_007-1440_x2.jpg}

\includegraphics{https://static01.graylady3jvrrxbe.onion/newsgraphics/2020/08/17/hunger/assets/images/026_manausha_russ__june_11__st_louis_mo_001-1440_x2.jpg}

From left, Aliza, 1, Nyla, 6, Amarri, 5, and Kadynce, 8, with their
mother, Manausha Russ.

Russ lives with her four daughters on the west side of St. Louis. She
receives about \$635 per month in food stamps, but with the girls at
home all day, and her partner, Lamarr, there too, it isn't always
sufficient. ``Some days I feel like I have a lot,'' she says, ``and some
days I feel like I don't have enough.''

\includegraphics{https://static01.graylady3jvrrxbe.onion/newsgraphics/2020/08/17/hunger/assets/images/026_manausha_russ__june_11__st_louis_mo_006-1440_x2.jpg}

The family moved into their current apartment about six months ago.

\includegraphics{https://static01.graylady3jvrrxbe.onion/newsgraphics/2020/08/17/hunger/assets/images/026_manausha_russ__june_11__st_louis_mo_005-1440_x2.jpg}

Russ doesn't have a dining table or chairs yet, so the girls eat on the
floor.

Memphis

June 19

In so many places, Kenneally found food-insecure families were helping
one another out despite their own hardship. Here, in a condominium
complex on the city's east side, a neighbor picked up free school
lunches and distributed them to children in the building, including the
Boughton sisters: Brooklyn, 4, on the far right, Chynna, 9, and Katie,
8, seen here with a neighbor's toddler who has since moved away.

\includegraphics{https://static01.graylady3jvrrxbe.onion/newsgraphics/2020/08/17/hunger/assets/images/032_donna_rice__nicole__kim__june_19__memphis_tn_004_tighter_horizontal_crop-1440_x2.jpg}

Most of the families Kenneally photographed had struggled to feed
themselves adequately for years. But she also met families who had been
thrown into food insecurity by the pandemic.

Facing\\
Hunger\\
For The\\
First\\
Time

Text by Tim Arango

\includegraphics{https://static01.graylady3jvrrxbe.onion/newsgraphics/2020/08/17/hunger/assets/images/014_laura_horsbourgh__may_27__parma_oh_003-1440_x2.jpg}

In the Horsburgh household, trips to pick up donated food --- a service
the family had not needed for years, before Covid-19 --- became a
diversion for the children stuck at home.

\includegraphics{https://static01.graylady3jvrrxbe.onion/newsgraphics/2020/08/17/hunger/assets/images/012_claire__may_21__erie_pa_005-1440_x2.jpg}

Claire Hudson with her son. Hudson has begun bringing food to the
homeless in Erie, Pa.

The federal government's food-stamp program has been dramatically
expanded to confront the economic devastation of the pandemic. But even
that hasn't been enough, as the ranks of the needy grow.

\includegraphics{https://static01.graylady3jvrrxbe.onion/newsgraphics/2020/08/17/hunger/assets/images/031_ciara_young__june_18__memphis_tn_002-1440_x2.jpg}

Ciara Young (right) and family, Memphis. Young lost her job in the
pandemic.

In long conversations around the country this August --- at kitchen
tables, in living rooms and sitting in cars in slow-moving food lines
with rambunctious children in the back --- Americans reflected on their
new reality. The shame and embarrassment. The loss of choice in
something as basic as what to eat. The worry over how to make sure their
children get a healthy diet. The fear that their lives will never get
back on track.

\includegraphics{https://static01.graylady3jvrrxbe.onion/newsgraphics/2020/08/17/hunger/assets/images/048_alexis_cazimero__july_31__san_diego_ca_006-1440_x2.jpg}

Alexis Cazimero now drives around San Diego County with her younger
children, seen here, distributing food to families like hers.

``Folks who had really good jobs and were able to pay their bills and
never knew how to find us,'' says Ephie Johnson, the president and chief
executive of Neighborhood Christian Charities. ``A lot of people had
finally landed that job, were helping their family, and able to do a
little better. And then this takes you out.''

Read the Full Essay

\href{https://www.nytimes3xbfgragh.onion/2020/09/03/us/food-pantries-hunger-us.html}{}

Minivans at the Food Pantry:\\
Meet America's New Needy
\includegraphics{https://static01.graylady3jvrrxbe.onion/newsgraphics/2020/08/17/hunger/assets/images/tiny_arrow_black.svg}

by Tim Arango

By late June, Kenneally had reached Mississippi, where the economic toll
of Covid-19 was falling hard on some of America's most chronically
impoverished areas, where residents have lived under hunger's shadow for
years. The pandemic dropped the state's labor participation rate to just
53 percent, the lowest in the nation.

\includegraphics{https://static01.graylady3jvrrxbe.onion/newsgraphics/2020/08/17/hunger/assets/images/033_patricia_luckett__june_23__jackson_ms_001-1440_x2.jpg}

Patricia Luckett, 57, has no car, so she sometimes takes a 30-minute
walk to get food from a local social services organization in Jackson,
Miss.

\includegraphics{https://static01.graylady3jvrrxbe.onion/newsgraphics/2020/08/17/hunger/assets/images/033_patricia_luckett__june_23__jackson_ms_003-1440_x2.jpg}

Luckett at home. ``I'm a country girl,'' she says. ``I love to cook.''

\includegraphics{https://static01.graylady3jvrrxbe.onion/newsgraphics/2020/08/17/hunger/assets/images/038_karen_cotton__june_27__jackson_ms_001-1440_x2.jpg}

Karen Cotton, 40, and her sons Jayden Brooks (left), 8, and Adrian
Brooks, 11, in Jackson, Miss.

\includegraphics{https://static01.graylady3jvrrxbe.onion/newsgraphics/2020/08/17/hunger/assets/images/038_karen_cotton__june_27__jackson_ms_002-1440_x2.jpg}

Adrian holds one of his favorite snacks. ``It's not filled with sugar,
so I buy them in bulk,'' Cotton says. ``This is what I call a healthy
snack.''

\includegraphics{https://static01.graylady3jvrrxbe.onion/newsgraphics/2020/08/17/hunger/assets/images/deborah_sulton-1440_x2.jpg}

Deborah Sulton, 66, who has lived in Jackson all her life, has 25
grandchildren and 13 great-grandchildren.

\includegraphics{https://static01.graylady3jvrrxbe.onion/newsgraphics/2020/08/17/hunger/assets/images/035_deborah_sulton__june_25__jackson_ms_002-1440_x2.jpg}

``The way I feed is, I cook like a cook for an army,'' Sulton says.

\includegraphics{https://static01.graylady3jvrrxbe.onion/newsgraphics/2020/08/17/hunger/assets/images/035_deborah_sulton__june_25__jackson_ms_003-1440_x2.jpg}

Aydin Sulton, 2, one of the many relatives Sulton helps support with her
Social Security income and food stamps.

\includegraphics{https://static01.graylady3jvrrxbe.onion/newsgraphics/2020/08/17/hunger/assets/images/035_deborah_sulton__june_25__jackson_ms_006-1440_x2.jpg}

``I have fed a lot of people and a lot of kids in the community,''
Sulton says. ``Whatever I got, I will share it because I get my
blessings back in return.''

Even before the pandemic, more than half of Mississippi's seniors --- 56
percent --- experienced regular shortfalls in food. One in 4
Mississippians is now experiencing food insecurity, according to the
nonprofit Feeding America.

\includegraphics{https://static01.graylady3jvrrxbe.onion/newsgraphics/2020/08/17/hunger/assets/images/040_nita_obryant_and_helen__july_2__greenwood_ms_001-1440_x2.jpg}

The kitchen of Helen O'Bryant and her daughter, Nita, in Greenwood,
Miss.

\includegraphics{https://static01.graylady3jvrrxbe.onion/newsgraphics/2020/08/17/hunger/assets/images/040_nita_obryant_and_helen__july_2__greenwood_ms_002-1440_x2.jpg}

``We're learning to enjoy life a little better,'' says Helen, 72,
sitting beside Nita, 45. ``Cooking helps, it really does. It's something
I want to do for my daughter.''

\includegraphics{https://static01.graylady3jvrrxbe.onion/newsgraphics/2020/08/17/hunger/assets/images/041_thaddeus_whitehead__july_2__greenwood_ms_002-1440_x2.jpg}

Thaddeus Whitehead, 41, with his children Angel, 7, and D'angelo, 8, in
Greenwood, Miss.

\includegraphics{https://static01.graylady3jvrrxbe.onion/newsgraphics/2020/08/17/hunger/assets/images/041_thaddeus_whitehead__july_2__greenwood_ms_001-1440_x2.jpg}

Whitehead says he spends about \$150 on groceries every two weeks, plus
he gets boxes of food from a nearby church.

\includegraphics{https://static01.graylady3jvrrxbe.onion/newsgraphics/2020/08/17/hunger/assets/images/041_thaddeus_whitehead__july_2__greenwood_ms_003-1440_x2.jpg}

He also catches bream from a local lake. ``I cook about 10, and I put
the rest of them up in the freezer. Then when we need to eat them, I
defrost them.''

Jackson, Miss.

June 30

The city of Jackson (population 164,000) is often classified as a ``food
desert'' for its high rate of food insecurity and the scarcity of
well-stocked stores. Deidre Lyons lives there with her three kids,
sister, niece and father. Lyons, 28, receives \$524 a month in food
stamps, but without access to a car, she can't easily get to a grocery
store to use them.

Tap to cycle through images

\includegraphics{https://static01.graylady3jvrrxbe.onion/newsgraphics/2020/08/17/hunger/assets/images/034_deidre_lyons__june_24__30__july_4_jackson_ms_010-1440_x2.jpg}

\includegraphics{https://static01.graylady3jvrrxbe.onion/newsgraphics/2020/08/17/hunger/assets/images/034_deidre_lyons__june_24__30__july_4_jackson_ms_006-1440_x2.jpg}

\includegraphics{https://static01.graylady3jvrrxbe.onion/newsgraphics/2020/08/17/hunger/assets/images/034_deidre_lyons__june_24__30__july_4_jackson_ms_012-1440_x2.jpg}

\includegraphics{https://static01.graylady3jvrrxbe.onion/newsgraphics/2020/08/17/hunger/assets/images/034_deidre_lyons__june_24__30__july_4_jackson_ms_011-1440_x2.jpg}

\includegraphics{https://static01.graylady3jvrrxbe.onion/newsgraphics/2020/08/17/hunger/assets/images/034_deidre_lyons__june_24__30__july_4_jackson_ms_002-1440_x2.jpg}

\includegraphics{https://static01.graylady3jvrrxbe.onion/newsgraphics/2020/08/17/hunger/assets/images/034_deidre_lyons__june_24__30__july_4_jackson_ms_013-1440_x2.jpg}

\includegraphics{https://static01.graylady3jvrrxbe.onion/newsgraphics/2020/08/17/hunger/assets/images/034_deidre_lyons__june_24__30__july_4_jackson_ms_001-1440_x2.jpg}

Lyons's daughter Tianna, 1.

``My kids, they love to eat,'' says Lyons, whose cousin will
occasionally drive her to the grocery store when she isn't caring for
her own children. ``My kids eat whatever we cook because they aren't
picky eaters. I'm hoping they stay like that.''

\includegraphics{https://static01.graylady3jvrrxbe.onion/newsgraphics/2020/08/17/hunger/assets/images/034_deidre_lyons__june_24__30__july_4_jackson_ms_003-1440_x2.jpg}

Janiya in late June outside Robinson Food Mart.

\includegraphics{https://static01.graylady3jvrrxbe.onion/newsgraphics/2020/08/17/hunger/assets/images/034_deidre_lyons__june_24__30__july_4_jackson_ms_008-1440_x2.jpg}

Jaheim buying a corndog at the Dude With the Food, a convenience store
within walking distance of home.

The causes of chronic food insecurity are many: unemployment; low wages;
unaffordable or unstable housing; rising medical costs; unreliable
transportation.

How\\
Hunger\\
Persists\\
In\\
America

Text by Adrian Nicole LeBlanc

\includegraphics{https://static01.graylady3jvrrxbe.onion/newsgraphics/2020/08/17/hunger/assets/images/01_toni__march_25__ny_001-1440_x2.jpg}

At a homeless shelter in Menands, N.Y., in early spring.

Treating hunger as a temporary emergency, instead of a symptom of
systemic problems, has always informed the American response to it ---
and as a result government programs have been designed to alleviate each
peak, rather than address the factors that produce them.

\includegraphics{https://static01.graylady3jvrrxbe.onion/newsgraphics/2020/08/17/hunger/assets/images/03_miriam__may_3__ny_001-1440_x2.jpg}

A resident at a low-income apartment building in Utica, N.Y., takes a
meal to a friend.

\includegraphics{https://static01.graylady3jvrrxbe.onion/newsgraphics/2020/08/17/hunger/assets/images/032_donna_rice__nicole__kim__june_19__memphis_tn_002-1440_x2.jpg}

Receiving lunches provided by the Y.M.C.A. in Memphis, delivered by a
neighbor.

Food banks are supposed to fill in the gaps, but more than 37 million
Americans are food insecure, according to the U.S.D.A. ``We call it an
emergency food system, but it's a 50-year emergency,'' says Noreen
Springstead, executive director of WhyHunger, which supports grass-roots
food organizations.

Read the Full Essay

\href{https://www.nytimes3xbfgragh.onion/2020/09/02/magazine/food-security-united-states.html}{}

How Hunger Persists in a Rich Country Like America
\includegraphics{https://static01.graylady3jvrrxbe.onion/newsgraphics/2020/08/17/hunger/assets/images/tiny_arrow_black.svg}

by Adrian Nicole LeBlanc

In early July, the pandemic was cresting in Texas just as Kenneally
arrived.

Houston

July 6

Kelly Rivera, a single mother with three kids who makes \$688 every two
weeks as a teacher's aide, goes to the food bank on Wednesdays to
supplement what she is able to buy with food stamps. ``There are times
they give you what you need, and there are times they don't give you
what you need,'' she says. ``You can't be picky.''

Tap to cycle through images

\includegraphics{https://static01.graylady3jvrrxbe.onion/newsgraphics/2020/08/17/hunger/assets/images/042_kelly_rivera__july_6__houston_tx_002-1440_x2.jpg}

\includegraphics{https://static01.graylady3jvrrxbe.onion/newsgraphics/2020/08/17/hunger/assets/images/042_kelly_rivera__july_6__houston_tx_001-1440_x2.jpg}

\includegraphics{https://static01.graylady3jvrrxbe.onion/newsgraphics/2020/08/17/hunger/assets/images/042_kelly_rivera__july_6__houston_tx_013-1440_x2.jpg}

\includegraphics{https://static01.graylady3jvrrxbe.onion/newsgraphics/2020/08/17/hunger/assets/images/042_kelly_rivera__july_6__houston_tx_004-1440_x2.jpg}

\includegraphics{https://static01.graylady3jvrrxbe.onion/newsgraphics/2020/08/17/hunger/assets/images/042_kelly_rivera__july_6__houston_tx_007-1440_x2.jpg}

\includegraphics{https://static01.graylady3jvrrxbe.onion/newsgraphics/2020/08/17/hunger/assets/images/042_kelly_rivera__july_6__houston_tx_003-1440_x2.jpg}

Rivera's children eat Cheetos on the couch after returning home from
picking up groceries. (From left: Destiny, 4, Ana, 6, Jonathan, 3.)

The family had to wait for hours at the Catholic Charities in 100-degree
heat. But Rivera has a message for her struggling neighbors who are too
proud to visit food banks: ``Don't be ashamed. That is what the
community is there for, to help.''

\includegraphics{https://static01.graylady3jvrrxbe.onion/newsgraphics/2020/08/17/hunger/assets/images/042_kelly_rivera__july_6__houston_tx_010-1440_x2.jpg}

Rivera waiting in a long food line.

\includegraphics{https://static01.graylady3jvrrxbe.onion/newsgraphics/2020/08/17/hunger/assets/images/042_kelly_rivera__july_6__houston_tx_012-1440_x2.jpg}

Ana, left, and Destiny sit in their car, waiting in a parking lot to be
allowed into the actual food line, where they will wait even longer.

Hatch, N.M.

July 13

Some 800 miles west in New Mexico, near the town of Hatch, workers pick
onions for \$15 a box, which translates to less than a minimum wage for
many workers. There are no food pantries nearby, and so the workers are
forced to eat extremely simply on their earnings, making nearly
everything they eat from scratch.

Tap to cycle through images

\includegraphics{https://static01.graylady3jvrrxbe.onion/newsgraphics/2020/08/17/hunger/assets/images/045_onion_families__july__hatch__nm_021-1440_x2.jpg}

\includegraphics{https://static01.graylady3jvrrxbe.onion/newsgraphics/2020/08/17/hunger/assets/images/045_onion_families__july__hatch__nm_014-1440_x2.jpg}

\includegraphics{https://static01.graylady3jvrrxbe.onion/newsgraphics/2020/08/17/hunger/assets/images/045_onion_families__july__hatch__nm_007-1440_x2.jpg}

\includegraphics{https://static01.graylady3jvrrxbe.onion/newsgraphics/2020/08/17/hunger/assets/images/045_onion_families__july__hatch__nm_012-1440_x2.jpg}

\includegraphics{https://static01.graylady3jvrrxbe.onion/newsgraphics/2020/08/17/hunger/assets/images/045_onion_families__july__hatch__nm_009-1440_x2.jpg}

\includegraphics{https://static01.graylady3jvrrxbe.onion/newsgraphics/2020/08/17/hunger/assets/images/045_onion_families__july__hatch__nm_008-1440_x2.jpg}

\includegraphics{https://static01.graylady3jvrrxbe.onion/newsgraphics/2020/08/17/hunger/assets/images/045_onion_families__july__hatch__nm_002-1440_x2.jpg}

At the beginning of the onion season, a priest came to the field at dawn
to bless the harvest.

Juan Pablo Reyes is using the money he made picking onions to help pay
for college. ``People that work at the bottom of the food chain,
cultivating all these different crops, are basically the builders of our
country,'' he says.

\includegraphics{https://static01.graylady3jvrrxbe.onion/newsgraphics/2020/08/17/hunger/assets/images/045_onion_families__july__hatch__nm_011-1440_x2.jpg}

Yasmin and Yeslin Reyes, 11, are the only members of their family who
don't work in the onion fields, but that will change next summer.

\includegraphics{https://static01.graylady3jvrrxbe.onion/newsgraphics/2020/08/17/hunger/assets/images/045_onion_families__july__hatch__nm_010-1440_x2.jpg}

Their older brothers started when they turned 12, and the same is
expected of them.

\includegraphics{https://static01.graylady3jvrrxbe.onion/newsgraphics/2020/08/17/hunger/assets/images/045_onion_families__july__hatch__nm_013-1440_x2.jpg}

Juan Pablo's high school graduation cap says ``Proud Immigrant'' and has
flowers in the colors of the Mexican flag.

Leaving New Mexico, Kenneally headed west across Arizona. She finished
her journey in Southern California at the end of July. The story there
was no different than it had been across the country, except that
wildfires were also beginning to ravage the state --- yet another crisis
in a year full of them.

San Diego

July 31

An event planner and hairstylist who has been out of work since early in
the pandemic, Alexis Frost Cazimero, 40, now spends her days driving
around the county with three of her children --- Mason, 6 (not
pictured); Carson, 5; and Coco, 1 --- collecting food for her family and
for neighbors and friends who are unable to leave their homes or
reluctant to seek help.

\includegraphics{https://static01.graylady3jvrrxbe.onion/newsgraphics/2020/08/17/hunger/assets/images/048_alexis_cazimero__july_31__san_diego_ca_012-1440_x2.jpg}

Cazimero says she is grateful she has been able to help others. ``Being
that person in the community that shares and brings resources to the
people that can't get them brings purpose to my family.''

\includegraphics{https://static01.graylady3jvrrxbe.onion/newsgraphics/2020/08/17/hunger/assets/images/048_alexis_cazimero__july_31__san_diego_ca_014-1440_x2.jpg}

Adam Cazimero, 40, Coco (standing), Mason and Carson.

\includegraphics{https://static01.graylady3jvrrxbe.onion/newsgraphics/2020/08/17/hunger/assets/images/cazimero_final_pic-1440_x2.jpg}

Kenneally's photographs reveal the fragility of American life, exposed
and exacerbated by the pandemic. They show us how close to the edge so
many families live, how vulnerable and insecure their arrangements are,
and also how resilient they can be when faced with a crisis.

But nothing stands out from these images more vividly than the children:
eating whatever they can, whenever and wherever they can, somehow
managing to maintain, in the midst of this historically desperate time,
some innocence and some hope.

\includegraphics{https://static01.graylady3jvrrxbe.onion/newsgraphics/2020/08/17/hunger/assets/images/045_onion_families__july__hatch__nm_020-1440_x2.jpg}

Hatch, N.M.

\includegraphics{https://static01.graylady3jvrrxbe.onion/newsgraphics/2020/08/17/hunger/assets/images/017_latia_hinkle__june_9__gary_in_008-1440_x2.jpg}

Gary, Ind.

\includegraphics{https://static01.graylady3jvrrxbe.onion/newsgraphics/2020/08/17/hunger/assets/images/015_asparagus__may_28__parma_oh_001-1440_x2.jpg}

Parma, Ohio

\includegraphics{https://static01.graylady3jvrrxbe.onion/newsgraphics/2020/08/17/hunger/assets/images/043_teresa_beltran__july_8__houston_tx_001-1440_x2.jpg}

Houston

They are the greatest victims of the food-insecurity crisis. Research
has shown long-term links between food insecurity and a wide variety of
health issues in children --- elevated risks of asthma and other chronic
illnesses, lags in educational attainment. And according to a Brookings
Institution researcher, the number of U.S. children in need of immediate
food assistance is approximately 14 million.

\includegraphics{https://static01.graylady3jvrrxbe.onion/newsgraphics/2020/08/17/hunger/assets/images/04_luckas_antzak__may_8__oneida_ny_004-1440_x2.jpg}

Oneida, N.Y.

\includegraphics{https://static01.graylady3jvrrxbe.onion/newsgraphics/2020/08/17/hunger/assets/images/019_rosy__june_4__east_chicago_in_004-1440_x2.jpg}

East Chicago, Ind.

\includegraphics{https://static01.graylady3jvrrxbe.onion/newsgraphics/2020/08/17/hunger/assets/images/024_keyana__june_10__florrisant_mo_003-1440_x2.jpg}

Florissant, Mo.

For most of these children, the pandemic did not cause the instability
that plagues their lives; when it is over, they will face a crisis no
less acute, one that has persisted in this country for generations.

\includegraphics{https://static01.graylady3jvrrxbe.onion/newsgraphics/2020/08/17/hunger/assets/images/024_keyana__june_10__florrisant_mo_002-1440_x2.jpg}

Florissant, Mo.

\includegraphics{https://static01.graylady3jvrrxbe.onion/newsgraphics/2020/08/17/hunger/assets/images/042_kelly_rivera__july_7__houston_tx_015_vertical_tighter_crop-1440_x2.jpg}

Houston

\includegraphics{https://static01.graylady3jvrrxbe.onion/newsgraphics/2020/08/17/hunger/assets/images/045_onion_families__july__hatch__nm_016-1440_x2.jpg}

Hatch, N.M.

In the richest nation on earth, they live at the edge of hunger.

Kenneally visited many food distribution sites along her journey,
including ones run by: the
\href{https://www.salvationarmyusa.org/usn/ways-to-give/}{Salvation
Army}, \href{https://catholiccharitiesny.org/}{Catholic Charities},
\href{https://nwpafoodbank.org/}{Second Harvest Food Bank of Northwest
Pennsylvania}, \href{https://www.parmacityschools.org/}{Parma City
School District},
\href{https://stlfoodbank.org/}{St}\href{https://stlfoodbank.org/}{.
Louis Area Foodbank},
\href{https://www.operationfoodsearch.org/}{Operation Food Search},
\href{https://www.ncclife.org/}{Neighborhood Christian Centers},
\href{https://www.ymcamemphis.org/}{Y.M.C.A}\href{https://www.ymcamemphis.org/}{.
of Memphis and
t}\href{https://www.ymcamemphis.org/}{he}\href{https://www.ymcamemphis.org/}{}\href{https://www.ymcamemphis.org/}{Mid-South},
\href{https://stewpot.org/}{Stewpot Community Services},
\href{https://www.houstonfoodbank.org/}{Houston Food Bank},
\href{https://www.houstonisd.org/Page/126421}{Houston Independent School
District}, \href{https://www.communityfoodbank.org/}{Community Food Bank
of Southern Arizona} and \href{https://tihan.org/donate/}{Tucson
Interfaith HIV/AIDS Network}.

Brenda Ann Kenneally is a multimedia journalist who, over 30 years, has
produced participatory media projects with families from her home
community, including ``Upstate Girls: Unraveling Collar City.'' She is
currently assembling a multimedia autobiography, charting her experience
from being a disenfranchised youth to becoming a Guggenheim fellow and
frequent contributor to the magazine.
\href{https://www.nytimes3xbfgragh.onion/2020/09/02/insider/food-insecurity-families.html}{Read
more about Kenneally's journey.}

Adrian Nicole LeBlanc, an independent journalist and MacArthur fellow,
was embedded in an assisted-living facility as Kenneally began her trip
for this issue. They have worked together since 2003.

Tim Arango is a Los Angeles-based national correspondent for The Times.
He spent seven years as Baghdad bureau chief and also covered Turkey.
Before heading overseas, he had been a media reporter for The Times
since 2007.

Additional Reporting by Maddy Crowell, Lovia Gyarkye, Concepción de
León, Jaime Lowe, Jake Nevins, Kevin Pang and Malia Wollan.

Photo Editors: Amy Kellner and Rory Walsh.

Design by Aliza Aufrichtig and Eden Weingart.

Read 761 Comments

\begin{itemize}
\item
\item
\item
\item
\end{itemize}

Advertisement

\protect\hyperlink{after-bottom}{Continue reading the main story}

\hypertarget{site-index}{%
\subsection{Site Index}\label{site-index}}

\hypertarget{site-information-navigation}{%
\subsection{Site Information
Navigation}\label{site-information-navigation}}

\begin{itemize}
\tightlist
\item
  \href{https://help.nytimes3xbfgragh.onion/hc/en-us/articles/115014792127-Copyright-notice}{©~2020~The
  New York Times Company}
\end{itemize}

\begin{itemize}
\tightlist
\item
  \href{https://www.nytco.com/}{NYTCo}
\item
  \href{https://help.nytimes3xbfgragh.onion/hc/en-us/articles/115015385887-Contact-Us}{Contact
  Us}
\item
  \href{https://www.nytco.com/careers/}{Work with us}
\item
  \href{https://nytmediakit.com/}{Advertise}
\item
  \href{http://www.tbrandstudio.com/}{T Brand Studio}
\item
  \href{https://www.nytimes3xbfgragh.onion/privacy/cookie-policy\#how-do-i-manage-trackers}{Your
  Ad Choices}
\item
  \href{https://www.nytimes3xbfgragh.onion/privacy}{Privacy}
\item
  \href{https://help.nytimes3xbfgragh.onion/hc/en-us/articles/115014893428-Terms-of-service}{Terms
  of Service}
\item
  \href{https://help.nytimes3xbfgragh.onion/hc/en-us/articles/115014893968-Terms-of-sale}{Terms
  of Sale}
\item
  \href{https://spiderbites.nytimes3xbfgragh.onion}{Site Map}
\item
  \href{https://help.nytimes3xbfgragh.onion/hc/en-us}{Help}
\item
  \href{https://www.nytimes3xbfgragh.onion/subscription?campaignId=37WXW}{Subscriptions}
\end{itemize}
