Sections

SEARCH

\protect\hyperlink{site-content}{Skip to
content}\protect\hyperlink{site-index}{Skip to site index}

\hypertarget{comments}{%
\subsection{\texorpdfstring{\protect\hyperlink{commentsContainer}{Comments}}{Comments}}\label{comments}}

\href{}{I Got Blown Up in Iraq. Years Later, Amputating My Leg Set Me
Free.}\href{}{Skip to Comments}

The comments section is closed. To submit a letter to the editor for
publication, write to
\href{mailto:letters@NYTimes.com}{\nolinkurl{letters@NYTimes.com}}.

At War

\hypertarget{i-got-blown-up-in-iraq-years-later-amputating-my-leg-set-me-free}{%
\section{I Got Blown Up in Iraq. Years Later, Amputating My Leg Set Me
Free.}\label{i-got-blown-up-in-iraq-years-later-amputating-my-leg-set-me-free}}

By Elana DuffySept. 10, 2020

\begin{itemize}
\item
\item
\item
\item
\item
  \emph{164}
\end{itemize}

\includegraphics{https://static01.graylady3jvrrxbe.onion/packages/flash/multimedia/ICONS/transparent.png}

\includegraphics{https://static01.graylady3jvrrxbe.onion/newsgraphics/2020/09/01/amputee/assets/images/montgomeryelanaduffyfinal-003-2000.png}

In 2005, an I.E.D. in Iraq left me barely able to walk.

I spent 14 years living with chronic pain.

Now I'm free.

\includegraphics{https://static01.graylady3jvrrxbe.onion/packages/flash/multimedia/ICONS/transparent.png}

\includegraphics{https://static01.graylady3jvrrxbe.onion/newsgraphics/2020/09/01/amputee/assets/images/toper_small-2000.jpg}

I only have flashes of memory from getting hit in a roadside bomb attack
near Balad, Iraq, on Oct. 4, 2005. I saw the concerned driver reaching
out to shake me awake, but at the time I couldn't recall who he was.
With growing panic, I realized that I couldn't remember my interpreter's
name or what information I needed to get from my source who had brought
me out on my 12th convoy in 12 days. I'd worked with an infantry team of
U.S. and Iraqi troops for months, but their faces were suddenly
unrecognizable. Something was wrong.

\includegraphics{https://static01.graylady3jvrrxbe.onion/packages/flash/multimedia/ICONS/transparent.png}

\includegraphics{https://static01.graylady3jvrrxbe.onion/newsgraphics/2020/09/01/amputee/assets/images/accident-2000.JPG}

Stepping out of the truck, the sun was too bright through my sunglasses,
noises too loud but indistinguishable in my bleeding ears. A searing
jolt shot up my right leg with every step. I tumbled partway down a
hill, feeling my ankle give out despite having it wrapped and my boot
laces pulled as tight as possible.

Two weeks earlier on a remote mission to gather information about an oil
field, I had been in a collision that killed three Iraqis and wounded
multiple others, including four of us who were in an armored truck. The
impact had thrown me forward with such force that my right boot slammed
into the M16 rifle holder on the back of the passenger seat of the
Humvee, catching the lace and twisting as we bounced around the road.
The medics on the nearest base had no imaging equipment, so we called it
a sprain, and I got back on the road. I was an intelligence collector
and investigator, and I couldn't let my leg distract me from my work.

What I didn't know at the time was that the injuries I suffered in these
two incidents had inflicted such complex and significant damage to my
body that they would end my military career and leave me in nearly
constant, and sometimes excruciating, pain. My daily life and my
personal relationships were overtaken by my injuries and the new ones
they repeatedly caused.

Back in the United States after my Iraq deployment, an ankle assessment
indicated that I had several ligament tears and a chip broken away from
the bone. The injury would require rigorous treatment to heal properly.
Physical therapy. Steroid injections. Heat and ultrasound therapy. Rinse
and repeat. For over a year, Army doctors tried to fix that ankle, but
they didn't factor in the brain injury I had incurred from the bomb
blast, since the hemorrhage I had suffered wasn't found for roughly
another year. By 2008, the blinding migraines, narrowing vision and
changes in cognition finally granted me an emergency trip to Walter Reed
Army Medical Center to remove the mass that had coagulated in my brain.

I swapped leg treatments for four years of physical, occupational,
speech, cognitive and vision therapy. Balance boards and drills with
weird flashing lights. I improved enough to qualify for and transfer
into an elite unit and keep going as a human-intelligence operative. I
re-enlisted, planning to stay in the Army for another 13 or 14 years and
make a full career of it. I volunteered for another deployment --- and
that's when I was told that the complications from my leg and head
injuries made me nondeployable and ineligible for warrant-officer
candidate school. I was now a sergeant first class with no prospects for
promotion. The Army medically retired me in 2012.

\includegraphics{https://static01.graylady3jvrrxbe.onion/packages/flash/multimedia/ICONS/transparent.png}

\includegraphics{https://static01.graylady3jvrrxbe.onion/newsgraphics/2020/09/01/amputee/assets/images/montgomeryelanaduffyfinal-001-2000.png}

Every year since the accident, I tore at least one piece of soft tissue
in my right ankle. But I accepted it as how things were going to be, and
I went through all the therapies again at V.A. hospitals. I lived with
the pain that no physician could do anything about.

Over time, the muscles in my ankle atrophied because of the nerve damage
caused by the brain injury. No matter what I did, I couldn't get
stronger. I couldn't sleep, I couldn't step off a curb without feeling
like my leg would snap. Hiking, mountaineering, rock climbing and even
just walking across town brought pain.

Because I was always injured and in constant pain, I thought I was a
burden to those I cared about. I withdrew from walking to visit friends
or participating in anything that would keep me on my feet too long. I
leaned heavily on a handful of people I was comfortable confiding in,
but that didn't ease the guilt that weighed on me for asking so much and
offering so little in return. Already diagnosed with post-traumatic
stress disorder, my depression steadily worsened until at times I no
longer cared if I was alive or dead.

\includegraphics{https://static01.graylady3jvrrxbe.onion/packages/flash/multimedia/ICONS/transparent.png}

\includegraphics{https://static01.graylady3jvrrxbe.onion/newsgraphics/2020/09/01/amputee/assets/images/montgomeryelanaduffyfinal-004-2000.png}

\includegraphics{https://static01.graylady3jvrrxbe.onion/packages/flash/multimedia/ICONS/transparent.png}

\includegraphics{https://static01.graylady3jvrrxbe.onion/newsgraphics/2020/09/01/amputee/assets/images/montgomeryelanaduffyfinal-005-2000.png}

I was an intelligence collector and investigator, and I couldn't let my
leg distract me from my work.

\includegraphics{https://static01.graylady3jvrrxbe.onion/packages/flash/multimedia/ICONS/transparent.png}

\includegraphics{https://static01.graylady3jvrrxbe.onion/newsgraphics/2020/09/01/amputee/assets/images/working-2000.jpg}

\includegraphics{https://static01.graylady3jvrrxbe.onion/packages/flash/multimedia/ICONS/transparent.png}

\includegraphics{https://static01.graylady3jvrrxbe.onion/newsgraphics/2020/09/01/amputee/assets/images/montgomeryelanaduffyfinal-002-2000.png}

In 2019, I was recovering from reconstructive surgery on my left foot,
which had weakened after more than a decade of favoring my right side,
when I took a wrong step, and suffered what the M.R.I. report described
as ``internal derangement.'' Once again I couldn't walk. By this point,
I was living in New York City, and I had started my own small business.
I was constantly active --- still rock climbing and hiking and just
walking around town as much as I could in between inevitable injuries
that had started occurring more frequently.

The damage was by now so extensive that surgery was unavoidable. Most
doctors advised me to try a complete joint-and-foot fusion, but I was
reluctant. It would only be a partial fix, and nothing would eliminate
the neuropathic pain because that was coming from my beat-up brain.

On my 39th birthday, I met with Dr. S. Robert Rozbruch, a limb
reconstruction and replacement surgeon for Weill Cornell Medical Center
and the Hospital for Special Surgery in New York. He first listed
treatment options similar to what I had already heard, but then he
presented me with a new choice: a drastic option, he admitted, but one
he believed could improve my quality of life. Osseointegration, a
procedure originally devised for implanting teeth, had been pioneered
for prosthetic limbs in Sweden, then used more extensively in Australia.
A limited number of American surgeons were now qualified to perform it.
Rozbruch was one such surgeon. He was recommending amputation of my
right leg at midcalf.

\includegraphics{https://static01.graylady3jvrrxbe.onion/packages/flash/multimedia/ICONS/transparent.png}

\includegraphics{https://static01.graylady3jvrrxbe.onion/newsgraphics/2020/09/01/amputee/assets/images/montgomeryelanaduffyfinal-008-2000.png}

\includegraphics{https://static01.graylady3jvrrxbe.onion/packages/flash/multimedia/ICONS/transparent.png}

\includegraphics{https://static01.graylady3jvrrxbe.onion/newsgraphics/2020/09/01/amputee/assets/images/montgomeryelanaduffyfinal-006-2000.png}

He presented me with a new choice ... one he believed could improve my
quality of life.

\includegraphics{https://static01.graylady3jvrrxbe.onion/packages/flash/multimedia/ICONS/transparent.png}

\includegraphics{https://static01.graylady3jvrrxbe.onion/newsgraphics/2020/09/01/amputee/assets/images/montgomeryelanaduffyfinal-007-2000.png}

\includegraphics{https://static01.graylady3jvrrxbe.onion/packages/flash/multimedia/ICONS/transparent.png}

\includegraphics{https://static01.graylady3jvrrxbe.onion/newsgraphics/2020/09/01/amputee/assets/images/montgomeryelanaduffyfinal-012-2000.png}

In traditional amputations, the limb is removed and then the stump is
fitted into a socket prosthetic, which can lead to issues like poor fit
and a lack of control over the prosthetic. In osseointegration, the limb
is amputated and then a titanium rod is fitted into the bone, with a
node extending a couple inches below the stump. From that node, in my
case, I would be able to screw on attachments for walking, kayaking,
diving and climbing.

Since 2001, some 2,180 U.S. service members have endured major limb
amputations, according to the Army Surgeon General's office. Of that
number, at least 158 underwent amputation a year or more after their
injury --- likely people who had endured injuries that needed to be
stabilized before amputation was possible or who opted for amputation
after other treatments failed. But very few people in the United States
had undergone osseointegration, and only about a half dozen had the
procedure done for a leg injury below the knee. I would be one of the
first if I pursued it. The risk of the operation going wrong was low,
but was I really willing to cut off a part of my leg?

\includegraphics{https://static01.graylady3jvrrxbe.onion/packages/flash/multimedia/ICONS/transparent.png}

\includegraphics{https://static01.graylady3jvrrxbe.onion/newsgraphics/2020/09/01/amputee/assets/images/montgomeryelanaduffyfinal-009-2000.png}

When everything is going right and a body part is doing its job, we
don't have to think about it; we have the luxury of taking it for
granted. But pain had kept my foot on my mind every day for 14 years. I
hated the load it placed on me and the strain I then placed on others. I
hated the feeling of having to monitor every step I took, knowing a
simple misstep could and often did end in a visit to the hospital. I
hated feeling weak, hated realizing it was holding me back.

I weighed my options for more than a month: more cycles of surgery,
physical therapy, injections, all without the promise of pain relief;
or, endure several months of intense recovery and wear a prosthetic for
the rest of my life. I consulted with my parents, my boyfriend, Paul,
and my therapist, but it was seeing my diving, climbing and
mountaineering gear piled in the corner of a closet that made my
decision: I scheduled the surgery for Aug. 5, 2019.

\includegraphics{https://static01.graylady3jvrrxbe.onion/packages/flash/multimedia/ICONS/transparent.png}

\includegraphics{https://static01.graylady3jvrrxbe.onion/newsgraphics/2020/09/01/amputee/assets/images/montgomeryelanaduffyfinal-011-2000.png}

\includegraphics{https://static01.graylady3jvrrxbe.onion/packages/flash/multimedia/ICONS/transparent.png}

\includegraphics{https://static01.graylady3jvrrxbe.onion/newsgraphics/2020/09/01/amputee/assets/images/montgomeryelanaduffyfinal-013-2000.png}

Pain had kept my foot on my mind every day for 14 years.

\includegraphics{https://static01.graylady3jvrrxbe.onion/packages/flash/multimedia/ICONS/transparent.png}

\includegraphics{https://static01.graylady3jvrrxbe.onion/newsgraphics/2020/09/01/amputee/assets/images/montgomeryelanaduffyfinal-014-2000.png}

In the operating room, Rozbruch hammered the rod into my newly exposed
tibia, and the plastic surgeon painstakingly reattached the severed
nerves to the bottom of my remaining calf muscle. It took only four
hours for me to become bionic. When I came to, I was on a cocktail of
painkillers and anesthesia, but I was lucid and felt well enough to be
ravenously hungry. Strange as it sounds, for the first time in a long,
long while, I wasn't thinking about my leg.

After five days, the hospital released me to the care of Paul and my
cats. I hopped from the couch to the fridge for water, food and ice
packs, and generally cursed the slow pace of bone growth. Once, when
Paul wasn't home, my peg got stuck in the small folding stool I used for
balance in the shower. I pushed and pulled, but the peg wouldn't budge,
and I briefly considered living the rest of my life with a nine-inch,
black step stool gracing the bottom of my calf like a plastic hoop
skirt. I was eventually able to yank my peg out, with a little bruising
and some blood.

I hate asking for help: as a woman in the military, showing weakness was
not an option. But my new foot wouldn't be ready for a few months, and
until then, I needed assistance doing everything from getting to the
doctor to making dinner. Paul's parents transported me to and from the
hospital almost weekly, while Paul did what he could to keep me
comfortable at home. He cleaned, he cooked, he scooped cat litter, he
made sure everything I needed for the day was within easy reach before
he left for work in the morning. The feeling of being a burden, of
straining these relationships, weighed on me. I promised myself that
when I was able, I would make it up to them.

Slowly, the pain from surgery began to subside, and in September I
experienced my first hours without any discomfort in more than a decade.
I had gone into the surgery knowing it was meant to put an end to my
chronic leg problems, and this specific procedure was also supposed to
reduce phantom leg pain, but it had been performed so few times that I
had had no one to ask beforehand about the end results. Even though I
was still confined to the couch and didn't even have a prosthetic yet to
stand on in the shower, I felt more free than I had in years.

\includegraphics{https://static01.graylady3jvrrxbe.onion/packages/flash/multimedia/ICONS/transparent.png}

\includegraphics{https://static01.graylady3jvrrxbe.onion/newsgraphics/2020/09/01/amputee/assets/images/montgomeryelanaduffyfinal-019-2000.png}

\includegraphics{https://static01.graylady3jvrrxbe.onion/packages/flash/multimedia/ICONS/transparent.png}

\includegraphics{https://static01.graylady3jvrrxbe.onion/newsgraphics/2020/09/01/amputee/assets/images/montgomeryelanaduffyfinal-015-2000.png}

I hate asking for help: as a woman in the military, showing weakness was
not an option.

\includegraphics{https://static01.graylady3jvrrxbe.onion/packages/flash/multimedia/ICONS/transparent.png}

\includegraphics{https://static01.graylady3jvrrxbe.onion/newsgraphics/2020/09/01/amputee/assets/images/montgomeryelanaduffyfinal-016-2000.png}

\includegraphics{https://static01.graylady3jvrrxbe.onion/packages/flash/multimedia/ICONS/transparent.png}

\includegraphics{https://static01.graylady3jvrrxbe.onion/newsgraphics/2020/09/01/amputee/assets/images/montgomeryelanaduffyfinal-018-2000.png}

It has been just over 12 months since my surgery, and I'm happier with
Peggy (the name I've given the new leg) than I'd been with my natural
foot in many years. I can't run yet, but I can step off a curb without
my ankle rolling under my weight. I am still working on stairs, but
going uphill is so much easier without needles of pain shooting up my
leg. My balance, while still affected from the brain injury, has
improved enough to keep me upright more than double the time I was
managing in my pre-surgery testing. I still have headaches and dizzy
spells, though they no longer end with me tearing a ligament because of
a fall.

Most important, I am living mostly pain-free, and I was finally able to
get back on a rock wall. My climbing foot is about a third of the size
of my real foot to reduce torque, and it isn't flexible. I've had to
modify techniques I had practiced for 25 years, because I am not sure
exactly what part of the prosthetic is on the rock and I can't shift
weight from the heel to the toes. But I'm learning; I even built a
bouldering wall in my yard to practice.

The years of relying on others took a toll on some of my relationships
as I'd always feared. Paul moved out, and some friends drifted away as I
took time to regain my strength and adjust my balance before I could get
back to the things we all enjoyed like boating and diving. As
heartbreaking as those losses are, I have no regrets over the loss of my
foot. Without the pain, I'm more focused on what I am doing and what I
can do next. I can put on music and dance around my apartment while
making dinner, or go for a walk while taking work calls. I am about to
get a swimming foot in time for late-season beach trips, and I even have
a running foot in the design stages. The choice I made was a hard one,
and there are still hard days because of it. But giving up my leg meant
getting back the rest of my life.

\includegraphics{https://static01.graylady3jvrrxbe.onion/packages/flash/multimedia/ICONS/transparent.png}

\includegraphics{https://static01.graylady3jvrrxbe.onion/newsgraphics/2020/09/01/amputee/assets/images/montgomeryelanaduffyfinal-017-2000.png}

\includegraphics{https://static01.graylady3jvrrxbe.onion/packages/flash/multimedia/ICONS/transparent.png}

\includegraphics{https://static01.graylady3jvrrxbe.onion/newsgraphics/2020/09/01/amputee/assets/images/montgomeryelanaduffyfinal-022-2000.png}

Without the pain, I'm more focused on what I am doing and what I can do
next.

\includegraphics{https://static01.graylady3jvrrxbe.onion/packages/flash/multimedia/ICONS/transparent.png}

\includegraphics{https://static01.graylady3jvrrxbe.onion/newsgraphics/2020/09/01/amputee/assets/images/montgomeryelanaduffyfinal-020-2000.png}

\includegraphics{https://static01.graylady3jvrrxbe.onion/packages/flash/multimedia/ICONS/transparent.png}

\includegraphics{https://static01.graylady3jvrrxbe.onion/newsgraphics/2020/09/01/amputee/assets/images/montgomeryelanaduffyfinal-023-2000.png}

Elana Duffy is a Purple Heart veteran with 10 years of service in the
U.S. Army. She also founded the artificial-intelligence company
\href{http://www.pathfinder.vet}{www.pathfinder.vet} to help service
members and veterans connect with local benefits and resources.

Philip Montgomery is a photographer whose current work chronicles the
fractured state of America. For the magazine, he recently chronicled the
nation's coronavirus outbreak, going inside New York City public
hospitals, a funeral home in the Bronx and shuttered businesses.

\emph{Archival photography via Elana Duffy.}

Design and production by Shannon Lin.

Read 164 Comments

\begin{itemize}
\item
\item
\item
\item
\end{itemize}

Advertisement

\protect\hyperlink{after-bottom}{Continue reading the main story}

\hypertarget{site-index}{%
\subsection{Site Index}\label{site-index}}

\hypertarget{site-information-navigation}{%
\subsection{Site Information
Navigation}\label{site-information-navigation}}

\begin{itemize}
\tightlist
\item
  \href{https://help.nytimes3xbfgragh.onion/hc/en-us/articles/115014792127-Copyright-notice}{©~2020~The
  New York Times Company}
\end{itemize}

\begin{itemize}
\tightlist
\item
  \href{https://www.nytco.com/}{NYTCo}
\item
  \href{https://help.nytimes3xbfgragh.onion/hc/en-us/articles/115015385887-Contact-Us}{Contact
  Us}
\item
  \href{https://www.nytco.com/careers/}{Work with us}
\item
  \href{https://nytmediakit.com/}{Advertise}
\item
  \href{http://www.tbrandstudio.com/}{T Brand Studio}
\item
  \href{https://www.nytimes3xbfgragh.onion/privacy/cookie-policy\#how-do-i-manage-trackers}{Your
  Ad Choices}
\item
  \href{https://www.nytimes3xbfgragh.onion/privacy}{Privacy}
\item
  \href{https://help.nytimes3xbfgragh.onion/hc/en-us/articles/115014893428-Terms-of-service}{Terms
  of Service}
\item
  \href{https://help.nytimes3xbfgragh.onion/hc/en-us/articles/115014893968-Terms-of-sale}{Terms
  of Sale}
\item
  \href{https://spiderbites.nytimes3xbfgragh.onion}{Site Map}
\item
  \href{https://help.nytimes3xbfgragh.onion/hc/en-us}{Help}
\item
  \href{https://www.nytimes3xbfgragh.onion/subscription?campaignId=37WXW}{Subscriptions}
\end{itemize}
