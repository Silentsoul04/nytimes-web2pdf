Sections

SEARCH

\protect\hyperlink{site-content}{Skip to
content}\protect\hyperlink{site-index}{Skip to site index}

\href{https://www.nytimes3xbfgragh.onion/section/us}{U.S.}

\href{https://myaccount.nytimes3xbfgragh.onion/auth/login?response_type=cookie\&client_id=vi}{}

\href{https://www.nytimes3xbfgragh.onion/section/todayspaper}{Today's
Paper}

\href{/section/us}{U.S.}\textbar{}California, Oregon and Washington Fire
Tracking Maps

\url{https://nyti.ms/3bIk2td}

\begin{itemize}
\item
\item
\item
\item
\item
\end{itemize}

\hypertarget{wildfires-in-the-west}{%
\subsubsection{\texorpdfstring{\href{https://www.nytimes3xbfgragh.onion/spotlight/california-wildfires?name=styln-california-wildfires\&region=TOP_BANNER\&block=storyline_menu_recirc\&action=click\&pgtype=Interactive\&impression_id=6cd8b8d0-f52c-11ea-a92a-d1075f6219dd\&variant=undefined}{Wildfires
in the West}}{Wildfires in the West}}\label{wildfires-in-the-west}}

\begin{itemize}
\tightlist
\item
  live\href{https://www.nytimes3xbfgragh.onion/2020/09/12/us/wildfires-live-updates.html?name=styln-california-wildfires\&region=TOP_BANNER\&block=storyline_menu_recirc\&action=click\&pgtype=Interactive\&impression_id=6cd8b8d1-f52c-11ea-a92a-d1075f6219dd\&variant=undefined}{Fires
  Updates}
\item
  \href{https://www.nytimes3xbfgragh.onion/interactive/2020/us/fires-map-tracker.html?name=styln-california-wildfires\&region=TOP_BANNER\&block=storyline_menu_recirc\&action=click\&pgtype=Interactive\&impression_id=6cd8b8d2-f52c-11ea-a92a-d1075f6219dd\&variant=undefined}{Maps
  of the Fires}
\item
  \href{https://www.nytimes3xbfgragh.onion/article/wildfires-photos-california-oregon-washington-state.html?name=styln-california-wildfires\&region=TOP_BANNER\&block=storyline_menu_recirc\&action=click\&pgtype=Interactive\&impression_id=6cd8b8d3-f52c-11ea-a92a-d1075f6219dd\&variant=undefined}{Photos}
\item
  \href{https://www.nytimes3xbfgragh.onion/2020/09/10/us/climate-change-california-wildfires.html?name=styln-california-wildfires\&region=TOP_BANNER\&block=storyline_menu_recirc\&action=click\&pgtype=Interactive\&impression_id=6cd8b8d4-f52c-11ea-a92a-d1075f6219dd\&variant=undefined}{A
  Climate Reckoning}
\item
  \href{https://www.nytimes3xbfgragh.onion/article/wildfires-california-oregon-washington.html?name=styln-california-wildfires\&region=TOP_BANNER\&block=storyline_menu_recirc\&action=click\&pgtype=Interactive\&impression_id=6cd8b8d5-f52c-11ea-a92a-d1075f6219dd\&variant=undefined}{Answers
  to Your Questions}
\item
  \href{https://www.nytimes3xbfgragh.onion/2020/09/09/us/california-wildfires.html?name=styln-california-wildfires\&region=TOP_BANNER\&block=storyline_menu_recirc\&action=click\&pgtype=Interactive\&impression_id=6cd8dfe0-f52c-11ea-a92a-d1075f6219dd\&variant=undefined}{Newsletter}
\end{itemize}

Advertisement

\protect\hyperlink{after-top}{Continue reading the main story}

\hypertarget{california-oregon-and-washington-fire-tracking-maps}{%
\section{California, Oregon and Washington Fire Tracking
Maps}\label{california-oregon-and-washington-fire-tracking-maps}}

By \href{https://www.nytimes3xbfgragh.onion/by/matthew-bloch}{Matthew
Bloch}, Scott Reinhard,
\href{https://www.nytimes3xbfgragh.onion/by/bryan-pietsch}{Bryan
Pietsch} and Lucy TompkinsUpdated Sept. 12, 2020

\begin{itemize}
\item
\item
\item
\item
\end{itemize}

Latest extent

7-day extent

Earlier areas

California, Oregon and Washington are enduring a wildfire season of
historic proportions. By the end of Friday night, the fires together had
\href{https://www.nytimes3xbfgragh.onion/2020/09/12/us/wildfire-deaths.html}{claimed
at least 17 lives}, and had left millions of acres of land in embers.

\hypertarget{air-quality-map}{%
\subsection{Air Quality Map}\label{air-quality-map}}

Moderate

Unhealthy for sensitive groups

Unhealthy

Very unhealthy

Hazardous

Source: Sensor data from \href{https://www2.purpleair.com}{PurpleAir}.
Readings
\href{https://www.mdpi.com/2073-4433/11/8/856/htm?fbclid=IwAR2MAjSqcn-5TvQZY5SP-kjfB1su2IO8Zlrl9XBtZQ-Dnv4jxnN6VbtZqGw}{have
been adjusted} to better account for the properties of wood smoke. The
levels on the map are based on the
\href{https://www3.epa.gov/airnow/aqi_brochure_02_14.pdf}{air quality
index}, a measure of how clean or unhealthy the air is.

\href{https://www.nytimes3xbfgragh.onion/2020/09/12/health/fires-air-california.html}{A}\href{https://www.nytimes3xbfgragh.onion/2020/09/12/health/fires-air-california.html}{ir
quality} has become a large concern, as large clouds of ash and smoke
spread across the region.

The
\href{https://www.nytimes3xbfgragh.onion/2020/09/12/us/wildfires-live-updates.html\#link-7a6a9ef4}{health
effects of wildfire smoke} are not fully understood, but it has been
linked to serious health problems and
\href{https://www.sciencedirect.com/science/article/pii/S0160412019326935}{research
suggests} that the effects don't necessarily go away when skies clear.

Here are some of the major fires that are burning across the three
states. See our
\href{https://www.nytimes3xbfgragh.onion/2020/09/12/us/wildfires-live-updates.html}{live
coverage for more ›}

\hypertarget{california-north-complex}{%
\subsection{California: North Complex}\label{california-north-complex}}

Latest extent

7-day extent

Earlier areas

The North Complex has been burning for weeks since it was sparked by
lightning on Aug. 17 in Northern California. Powerful winds this week
caused the fire to spread rapidly; it has burned over 250,000 acres and
is roughly a quarter contained.

The fire has killed nine people and destroyed or damaged 2,000
structures. All evacuation warnings that had been issued were changed to
mandatory orders in Butte, Yuba and Plumas Counties.

\hypertarget{california-creek-fire}{%
\subsection{California: Creek Fire}\label{california-creek-fire}}

Latest extent

7-day extent

Earlier areas

The Creek Fire began Sept. 4 in the Sierra National Forest and has grown
rapidly to more than 175,000 acres.

Over Labor Day weekend, the fire trapped hundreds of hikers and their
dogs in the forest, requiring
\href{https://www.nytimes3xbfgragh.onion/2020/09/08/us/california-wildfires-helicopter-rescue.html}{helicopter
rescue missions} for more than 350 people by the California National
Guard.

It is barely contained and has destroyed hundreds of structures, damaged
another dozen and threatened thousands more. Mandatory evacuations are
in place in surrounding Madera County.

\hypertarget{california-bobcat-fire}{%
\subsection{California: Bobcat Fire}\label{california-bobcat-fire}}

Latest extent

7-day extent

Earlier areas

About 25 miles northeast of Los Angeles, the Bobcat Fire has burned more
than 20,000 acres and is barely contained. Extremely dry brush in areas
with ``little to no fire history'' on record have fueled the fire, which
was burning with a ``rapid rate of spread,'' the U.S. Forest Service
said.

Though evacuation warnings were issued for foothill communities like
Pasadena and Arcadia, no mandatory evacuation orders were in effect on
Friday as the fire moved farther into Angeles National Forest, away from
the residential areas.

\hypertarget{california-august-complex}{%
\subsection{California: August
Complex}\label{california-august-complex}}

Latest extent

7-day extent

Earlier areas

The August Complex was initially 37 fires in the Mendocino National
Forest that were started by lightning on Aug. 17. On Thursday, it
\href{https://www.nytimes3xbfgragh.onion/2020/09/10/us/wildfires-live-updates.html\#link-1471a6ef}{became
the largest fire in modern California history}, overtaking a record set
in 2018 in the same part of the state.

As of Friday afternoon, the August Complex had burned nearly a half
million acres across five counties and was a quarter contained.

\hypertarget{oregon-almeda-fire}{%
\subsection{Oregon: Almeda Fire}\label{oregon-almeda-fire}}

Latest extent

7-day extent

Earlier areas

With startling speed, the Almeda Fire forced thousands of people to
evacuate with little notice, rapidly burning through residential areas,
and leaving several communities largely destroyed.

Two people have been killed, and roughly 600 homes and 100 commercial
buildings were damaged or destroyed as of Friday afternoon. The fire was
50 percent contained as firefighters worked to put out hot spots.

The fire is being investigated as
\href{https://www.nytimes3xbfgragh.onion/2020/09/12/us/wildfires-live-updates.html\#link-18416023}{possible
arson}, and a man was charged on Friday with starting a fire in a small
town as the larger blaze moved toward the area.

\hypertarget{oregon-holiday-farm-fire}{%
\subsection{Oregon: Holiday Farm Fire}\label{oregon-holiday-farm-fire}}

Latest extent

7-day extent

Earlier areas

East of Eugene, the Holiday Farm Fire had burned over 150,000 acres and
was uncontained on Friday. It is expected to threaten residential and
commercial areas as it grows. The fire burned through several
communities, with Blue River bearing the brunt of the damage. It has
claimed at least one life.

\hypertarget{oregon-riverside-beachie-creek-and-lionshead-fires}{%
\subsection{Oregon: Riverside, Beachie Creek and Lionshead
Fires}\label{oregon-riverside-beachie-creek-and-lionshead-fires}}

Latest extent

7-day extent

Earlier areas

Three fires, each larger than 100,000 acres, are burning east of Salem.
The Riverside Fire was uncontained on Friday and had grown to more than
130,000 acres.

Just south of that fire, the Beachie Creek and Lionshead Fires collided,
with the Beachie Creek Fire --- which has killed two people and injured
two more --- burning over 180,000 acres and the Lionshead Fire burning
roughly 110,000 acres.

\hypertarget{washington-cold-springs-and-pearl-hill-fires}{%
\subsection{Washington: Cold Springs and Pearl Hill
Fires}\label{washington-cold-springs-and-pearl-hill-fires}}

Latest extent

7-day extent

Earlier areas

In Northern Washington, the Cold Springs Fire had burned over 185,000
acres as of Friday afternoon and was 25 percent contained. The nearby
Pearl Hill Fire --- which started Sept. 7, a day after the Cold Springs
blaze --- had burned 178,000 acres, but was 50 percent contained.

Both fires spread quickly because of strong winds and low humidity, and
temperatures in the area on Friday hovered in the low 90s. A
\href{https://www.nytimes3xbfgragh.onion/2020/09/09/us/fires-oregon-california-live-updates.html\#link-1a2d0777}{1-year-old
was killed} in the Cold Springs Fire on Wednesday, and his parents were
both severely injured.

Source: NASA's Fire Information for Resource Management System. Areas
shown on the maps are estimates of where there were fires based on
analysis of satellite imagery data.

\begin{itemize}
\item
\item
\item
\item
\end{itemize}

Advertisement

\protect\hyperlink{after-bottom}{Continue reading the main story}

\hypertarget{site-index}{%
\subsection{Site Index}\label{site-index}}

\hypertarget{site-information-navigation}{%
\subsection{Site Information
Navigation}\label{site-information-navigation}}

\begin{itemize}
\tightlist
\item
  \href{https://help.nytimes3xbfgragh.onion/hc/en-us/articles/115014792127-Copyright-notice}{©~2020~The
  New York Times Company}
\end{itemize}

\begin{itemize}
\tightlist
\item
  \href{https://www.nytco.com/}{NYTCo}
\item
  \href{https://help.nytimes3xbfgragh.onion/hc/en-us/articles/115015385887-Contact-Us}{Contact
  Us}
\item
  \href{https://www.nytco.com/careers/}{Work with us}
\item
  \href{https://nytmediakit.com/}{Advertise}
\item
  \href{http://www.tbrandstudio.com/}{T Brand Studio}
\item
  \href{https://www.nytimes3xbfgragh.onion/privacy/cookie-policy\#how-do-i-manage-trackers}{Your
  Ad Choices}
\item
  \href{https://www.nytimes3xbfgragh.onion/privacy}{Privacy}
\item
  \href{https://help.nytimes3xbfgragh.onion/hc/en-us/articles/115014893428-Terms-of-service}{Terms
  of Service}
\item
  \href{https://help.nytimes3xbfgragh.onion/hc/en-us/articles/115014893968-Terms-of-sale}{Terms
  of Sale}
\item
  \href{https://spiderbites.nytimes3xbfgragh.onion}{Site Map}
\item
  \href{https://help.nytimes3xbfgragh.onion/hc/en-us}{Help}
\item
  \href{https://www.nytimes3xbfgragh.onion/subscription?campaignId=37WXW}{Subscriptions}
\end{itemize}
