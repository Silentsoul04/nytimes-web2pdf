Sections

SEARCH

\protect\hyperlink{site-content}{Skip to
content}\protect\hyperlink{site-index}{Skip to site index}

\href{https://myaccount.nytimes3xbfgragh.onion/auth/login?response_type=cookie\&client_id=vi}{}

\href{https://www.nytimes3xbfgragh.onion/section/todayspaper}{Today's
Paper}

\begin{itemize}
\item
  \href{https://www.nytimes3xbfgragh.onion/live/2020/09/09/us/trump-vs-biden?action=click\&pgtype=Article\&state=default\&region=TOP_BANNER\&context=storylines_menu}{Election
  Updates}
\item
  \href{https://www.nytimes3xbfgragh.onion/interactive/2020/09/08/us/elections/results-new-hampshire-primary-elections.html?action=click\&pgtype=Article\&state=default\&region=TOP_BANNER\&context=storylines_menu}{New
  Hampshire Results}
\item
  \href{https://www.nytimes3xbfgragh.onion/interactive/2020/us/elections/election-states-biden-trump.html?action=click\&pgtype=Article\&state=default\&region=TOP_BANNER\&context=storylines_menu}{Paths
  to 270}
\item
  \href{https://www.nytimes3xbfgragh.onion/interactive/2020/08/31/us/politics/vote-by-mail-deadlines.html?action=click\&pgtype=Article\&state=default\&region=TOP_BANNER\&context=storylines_menu}{Voting
  by Mail}
\item
  \href{https://www.nytimes3xbfgragh.onion/interactive/2019/us/elections/2020-presidential-election-calendar.html?action=click\&pgtype=Article\&state=default\&region=TOP_BANNER\&context=storylines_menu}{Key
  Dates}
\item
  \href{https://www.nytimes3xbfgragh.onion/newsletters/politics?action=click\&pgtype=Article\&state=default\&region=TOP_BANNER\&context=storylines_menu}{Politics
  Newsletter}
\end{itemize}

\hypertarget{donald-trump-jr-and-nikki-haley-attack-biden-and-praise-the-president}{%
\section{Donald Trump Jr. and Nikki Haley Attack Biden and Praise the
President}\label{donald-trump-jr-and-nikki-haley-attack-biden-and-praise-the-president}}

Last Updated

Aug. 28, 2020, 8:22 a.m. ET

Aug. 28, 2020, 8:22 a.m. ET

Tim Scott, Kimberly Guilfoyle and Steve Scalise also spoke at the
Republican convention. Earlier in the day, President Trump baselessly
accused Democrats of ``using Covid to steal the election.''

\emph{This briefing has ended. Follow}
\href{https://www.nytimes3xbfgragh.onion/live/2020/08/25/us/rnc-convention-election}{\emph{our
latest coverage of the Biden vs. Trump 2020 election here}}\emph{.}

\hypertarget{heres-what-you-need-to-know}{%
\subsubsection{Here's what you need to
know:}\label{heres-what-you-need-to-know}}

\begin{itemize}
\item
  \protect\hyperlink{tim-scott-offered-his-own-optimistic-biography-after-a-night-of-doom}{}

  Tim Scott offered his own optimistic biography after a night of doom.
\item
  \protect\hyperlink{were-fact-checking-the-republican-national-convention}{}

  We're fact-checking the Republican National Convention.
\item
  \protect\hyperlink{donald-trump-jr-pelted-biden-and-defended-his-father}{}

  Donald Trump Jr. pelted Biden and defended his father.
\item
  \protect\hyperlink{Nikki-Haley-who-once-distanced-herself-from-Trump-offered-a-full-throated-endorsement}{}

  Nikki Haley, who once distanced herself from Trump, offered a
  full-throated endorsement.
\item
  \protect\hyperlink{a-cuban-american-speaker-warned-of-communism-with-florida-voters-in-mind}{}

  A Cuban-American speaker warned of communism, with Florida voters in
  mind.
\item
  \protect\hyperlink{republicans-rewrote-history-on-the-coronavirus}{}

  Republicans rewrote history on the coronavirus.
\item
  \protect\hyperlink{steve-scalise-offered-praise-for-trump-and-repeated-the-falsehood-that-biden-plans-to-defund-the-police}{}

  Steve Scalise offered praise for Trump, and repeated the falsehood
  that Biden plans to `defund' the police.
\item
  \protect\hyperlink{kimberly-guilfoyle-delivered-a-dark-denunciation-of-democrats}{}

  Kimberly Guilfoyle delivered a dark denunciation of Democrats.
\item
  \protect\hyperlink{mark-and-patricia-mccloskey-who-brandished-guns-at-peaceful-protesters-made-a-fear-mongering-pitch}{}

  Mark and Patricia McCloskey, who brandished guns at peaceful
  protesters, made a fear-mongering pitch.
\end{itemize}

\hypertarget{republicans-mounted-a-revisionist-defense-of-trumps-record}{%
\subsection{\texorpdfstring{\protect\hyperlink{republicans-mounted-a-revisionist-defense-of-trumps-record}{Republicans
mounted a revisionist defense of Trump's
record.}}{Republicans mounted a revisionist defense of Trump's record.}}\label{republicans-mounted-a-revisionist-defense-of-trumps-record}}

\includegraphics{https://static01.graylady3jvrrxbe.onion/images/2020/08/24/us/politics/24-live-repubs-ledeall1/merlin_176142300_d8a1eaf7-e930-4114-aadb-9d185e7d98b6-articleLarge.jpg?quality=75\&auto=webp\&disable=upscale}

President Trump and his political allies mounted a fierce and misleading
defense of his political record on the first night of the Republican
convention on Monday, while unleashing a barrage of attacks on Joseph R.
Biden Jr. and the Democratic Party that were unrelenting in their
bleakness.

Hours after Republican delegates formally nominated Mr. Trump for a
second term, the president and his party made plain that they intended
to engage in sweeping revisionism about Mr. Trump's management of the
coronavirus pandemic, his record on race relations and much else.

And they laid out a dystopian picture of what the United States would
look like under a Biden administration, warning of a ``vengeful mob''
that would lay waste to suburban communities and turn quiet
neighborhoods into war zones.

At times, the speakers and prerecorded videos appeared to be describing
an alternate reality: one in which the nation was not nearing 180,000
dead from the coronavirus; in which Mr. Trump had not consistently
ignored serious warnings about the disease; in which the president had
not spent much of his term appealing openly to xenophobia and racial
animus; and in which someone other than Mr. Trump had presided over an
economy that began crumbling in the spring.

Donald Trump Jr., the president's son, praised his father's management
of the virus, one of several segments asserting an unsupported narrative
that the president had been a sturdy leader in a crisis even as polls
show Americans believe he has handled the pandemic poorly.

``As the virus began to spread, the president acted quickly and ensured
ventilators got to hospitals that needed them most,'' the president's
son said, making no mention of the millions of Americans sickened and
killed or the complaints from governors that they were not receiving the
necessary equipment. ``There is more work to do, but there is light at
the end of the tunnel.''

It was part of a vehement address the younger Mr. Trump delivered that
framed the election as a choice between ``church, work and school''
against ``rioting, looting and vandalism.''

The scorched-earth rhetoric and knowing references to phrases like
``cancel culture'' would not have been out of place during a Fox News
prime-time segment. By that measure, the arguments might help lure some
wavering Republicans, uneasy with the president's handling of the virus,
back to Mr. Trump. But it was far from clear that the programming would
appeal to any undecided voters.

\hypertarget{tim-scott-offered-his-own-optimistic-biography-after-a-night-of-doom}{%
\subsection{\texorpdfstring{\protect\hyperlink{tim-scott-offered-his-own-optimistic-biography-after-a-night-of-doom}{Tim
Scott offered his own optimistic biography after a night of
doom.}}{Tim Scott offered his own optimistic biography after a night of doom.}}\label{tim-scott-offered-his-own-optimistic-biography-after-a-night-of-doom}}

\includegraphics{https://static01.graylady3jvrrxbe.onion/images/2020/08/24/us/politics/24vid-rnc-tim-scott/merlin_176144097_6b028b03-8a36-4fcd-b32e-0522095ff841-videoSixteenByNine3000.jpg}

Senator Tim Scott of South Carolina is President Trump's highest-profile
Black supporter, but he didn't use his convention speech to make a
stalwart defense of Mr. Trump's first term.

Instead Mr. Scott, the first Black person to serve in both the House and
Senate, offered his own life story, from growing up sharing a bedroom
with his mother and brother in his grandparents' home to first getting
elected to Congress in 2010.

And while the rest of the speakers on the first night of Mr. Trump's
convention painted the picture of an apocalyptic nation needing to be
rescued by its incumbent president, Mr. Scott offered a far more
optimistic vision of America --- one where his rise ``from cotton to
Congress in one lifetime'' is possible.

``Our nation's arc always bends back toward fairness. We are not fully
where we want to be, but I thank God almighty we are not where we used
to be,'' Mr. Scott said. ``We are always striving to be better. When we
stumble, and we will, we pick ourselves back up and try again. We don't
give into cancel-culture, or the radical --- and factually baseless ---
belief that things are worse today than in the 1860s or the 1960s.''

It is not the first time Mr. Trump has employed Mr. Scott to defend his
stewardship in the White House. Earlier this year Mr. Scott defended Mr.
Trump's handling of the nationwide street protests sparked by the police
killing of George Floyd in Minnesota.

And at the same time Mr. Scott said former Vice President Joseph R.
Biden Jr., who won the Democratic presidential nomination largely on the
strength of his support from Black voters, was both out of touch with
Black America and took its support for granted.

``Joe Biden said if a Black man didn't vote for him, he wasn't truly
Black,'' Mr. Scott said. ``Joe Biden said Black people are a monolithic
community. It was Joe Biden who said poor kids can be just as smart as
white kids.''

Mr. Scott also condemned Mr. Biden for writing
\href{https://www.nytimes3xbfgragh.onion/2019/06/25/us/joe-biden-crime-laws.html}{the
1994 crime bill} ``that put millions of Black Americans behind bars.''
That political attack is notable given that much of the rest of the
evening was devoted to Mr. Trump's call to be more aggressive with
protesters, many of whom are Black.

And, like so many other Republican convention speakers, Mr. Scott said
Mr. Trump alone was holding back the Democrats from permanently altering
the American character.

``Make no mistake: Joe Biden and Kamala Harris want a cultural
revolution. A fundamentally different America,'' Mr. Scott said. ``If we
let them, they will turn our country into a socialist utopia, and
history has taught us that path only leads to pain and misery,
especially for hard-working people hoping to rise.''

--- \href{https://www.nytimes3xbfgragh.onion/by/reid-j-epstein}{Reid J.
Epstein}

\hypertarget{advertisement}{%
\subsubsection{Advertisement}\label{advertisement}}

\protect\hyperlink{after-dfp-ad-mid1}{Continue reading the main story}

\hypertarget{were-fact-checking-the-republican-national-convention}{%
\subsection{\texorpdfstring{\protect\hyperlink{were-fact-checking-the-republican-national-convention}{We're
fact-checking the Republican National
Convention.}}{We're fact-checking the Republican National Convention.}}\label{were-fact-checking-the-republican-national-convention}}

Our team of reporters who cover the Pentagon, Congress, health care and
more are fact checking the speeches on the first night of the Republican
National Convention.

\href{https://www.nytimes3xbfgragh.onion/live/2020/08/24/us/rnc-fact-check}{See
the claims and how they stack up against the truth here}.

--- The New York Times

\hypertarget{donald-trump-jr-pelted-biden-and-defended-his-father}{%
\subsection{\texorpdfstring{\protect\hyperlink{donald-trump-jr-pelted-biden-and-defended-his-father}{Donald
Trump Jr. pelted Biden and defended his
father.}}{Donald Trump Jr. pelted Biden and defended his father.}}\label{donald-trump-jr-pelted-biden-and-defended-his-father}}

\includegraphics{https://static01.graylady3jvrrxbe.onion/images/2020/08/24/us/politics/24vid-rnc-trump-jr/24vid-rnc-trump-jr-videoSixteenByNine3000.jpg}

Donald Trump Jr., the president's eldest son, delivered a
blood-and-thunder attack on Joseph R. Biden Jr. in an opening-night
keynote address at the Republican National Convention that predicted a
national descent into anarchy, violence and oppression if voters choose
not to re-elect his father.

The younger Mr. Trump cast the November election as a referendum on the
country's survival, echoing the sentiments of many of his fellow
first-night convention speakers --- as well as those of
\href{https://www.nytimes3xbfgragh.onion/2020/08/19/us/politics/obama-speech.html}{President
Barack Obama} and other Democrats who made similar assertions, with a
vastly different set of conclusions, at their party's gathering last
week.

In doing so, Mr. Trump's son was trying to beat back a criticism of his
father that has resonated with many voters --- that the president cares
more about his self-interest than the greater good of the nation.

``In the past, both parties believed in the goodness of America,'' he
said. ``We agreed on where we wanted to go. We just disagreed about how
to get there.''

He added: ``This time, the other party is attacking the very principles
on which our nation was founded --- freedom of thought, freedom of
speech, freedom of religion, the rule of law.''

In a speech that resembled a torqued-up version of the Fox News segments
that have become his trademark, Mr. Trump pelted the Democratic nominee
with nicknames, including ``Beijing Biden,'' and claimed the former vice
president's decades in Washington made him ``the Loch Ness Monster of
the Swamp.''

The younger Mr. Trump,
\href{https://www.nytimes3xbfgragh.onion/2020/08/24/magazine/donald-trump-jr.html?action=click\&module=Top\%20Stories\&pgtype=Homepage}{who
has emerged as a marquee fund-raiser for his father after playing a
relatively minor role in the 2016 campaign}, began with what amounted to
a conventional defense of an incumbent. He credited his father for ``the
greatest prolonged recovery in American history'' and ``the lowest
unemployment rate'' in a half-century.

Then came the pivot: ``Courtesy of the Chinese Communist Party, the
virus struck.''

His hard-edge, goading speech presented a stark contrast with the
relaxed, informal reminiscences of Mr. Biden's family over the four
nights of the Democratic convention last week, culminating in an
interview with Mr. Biden's granddaughters in which they jokingly
complained about how often he called to ask how they were doing.

The president did not need to call. He appeared onscreen, at a news
conference earlier in the day, and during the night for an extended
segment with frontline coronavirus workers at the White House.

His omnipresence, in a sense, upstaged his son's big speech, which was
written with the assistance of the former White House aide Cliff Sims,
the author of a tell-all book about the president's staff, according to
a person with knowledge of the situation.

Earlier in the night, speakers, including the party chairwoman, Ronna
McDaniel, had called out the Democrats for painting too gloomy a picture
of the country.

But toward the end of his address, the younger Mr. Trump pulled the
blackout curtain all the way down.

He portrayed Democrats not as honorable political opponents with a
different vision of the country than Mr. Trump, but as an organization
bent on destroying the American way of life, embodied and defended by
his family's patriarch.

``People of faith are under attack,'' Donald Trump Jr. said. ``You're
not allowed to go to church, but mass chaos in the streets gets a pass.
It's almost like this election is shaping up to be church, work and
school versus rioting, looting and vandalism.''

--- \href{https://www.nytimes3xbfgragh.onion/by/glenn-thrush}{Glenn
Thrush}

\hypertarget{nikki-haley-who-once-distanced-herself-from-trump-offered-a-full-throated-endorsement}{%
\subsection{\texorpdfstring{\protect\hyperlink{Nikki-Haley-who-once-distanced-herself-from-Trump-offered-a-full-throated-endorsement}{Nikki
Haley, who once distanced herself from Trump, offered a full-throated
endorsement.}}{Nikki Haley, who once distanced herself from Trump, offered a full-throated endorsement.}}\label{nikki-haley-who-once-distanced-herself-from-trump-offered-a-full-throated-endorsement}}

\includegraphics{https://static01.graylady3jvrrxbe.onion/images/2020/08/24/us/politics/24vid-rnc-haley1/24vid-rnc-haley1-videoSixteenByNine3000.jpg}

Nikki R. Haley, the former ambassador to the United Nations and a former
governor of South Carolina, tried to link Joseph R. Biden Jr. to the
``socialist left'' on Monday during the Republican National Convention,
despite Mr. Biden's record as a moderate.

Ms. Haley delivered a lengthy speech that harkened back to her work as
an ambassador and a governor, simultaneously making the case for
President Trump while also setting the stage for her own possible
presidential run in 2024.

``In much of the Democratic Party, it's now fashionable to say that
America is racist,'' Ms. Haley said. ``That is a lie. America is not a
racist country.''

She then invoked her parents, both immigrants from India who settled in
the South.

``This is personal for me,'' she said. ``My father wore a turban. My
mother wore a sari. I was a brown girl in a Black and white world.''

She added that her family had ``faced discrimination and hardship,'' but
``never gave in to grievance and hate,'' and she praised voters in South
Carolina for choosing her ``as their first minority and first female
governor.''

``America is a story that's a work in progress,'' she said. ``Now is the
time to build on that progress, and make America even freer, fairer and
better for everyone. That's why it's so tragic to see so much of the
Democratic Party turning a blind eye towards riots and rage.''

Ms. Haley also falsely portrayed Mr. Biden as ``good for Iran and ISIS''
and ``great for Communist China.''

``He's a godsend to everyone who wants America to apologize, abstain and
abandon our values,'' she said of the former vice president. ``Donald
Trump takes a different approach. He's tough on China, and he took on
ISIS and won. And he tells the world what it needs to hear.''

Her portrayal exaggerated the progress with the Islamic State, which has
been pushed out of its so-called caliphate, but continues to carry out
attacks in Iraq and Syria. And some of the territorial gains made by
American troops and their allies predate the Trump administration.

She also falsely stated that the Obama administration ``let North Korea
threaten America,'' but ``President Trump rejected that weakness, and we
passed the toughest sanctions on North Korea in history.''

In fact, President Barack Obama considered North Korea to be the most
urgent national security issue in his final year. He persuaded the
United Nations to impose a harsh set of sanctions on the country and, as
he was leaving office, urged Mr. Trump to address the issue as soon as
possible.

The Trump administration did get the United Nations to impose additional
sanctions, but Mr. Trump also began
\href{https://www.nytimes3xbfgragh.onion/2018/06/12/world/asia/north-korea-summit.html}{high-level
diplomatic talks} with North Korea and halted large-scale military
exercises with South Korea to accede to the wishes of Kim Jong-un, the
leader of North Korea. National security hawks on North Korea say Mr.
Trump's concessions weakened Washington's leverage.

Ms. Haley mentioned only briefly the pandemic that has left more than
175,000 people in the United States dead and millions in the country
unemployed.

Though she had once distanced herself from Mr. Trump, Ms. Haley has
become a fierce defender of him in recent months.

In an
\href{https://www.nytimes3xbfgragh.onion/2020/04/08/opinion/nikki-haley-governor-coronavirus-trump.html}{essay
published} in The New York Times in April, Ms. Haley defended Mr.
Trump's response to the
\href{https://www.nytimes3xbfgragh.onion/news-event/coronavirus}{coronavirus
pandemic}, writing ``Once a crisis hits, state responsibility is
primary.''

``The federal government can provide crucial resources, but the burden
is on the governor and her team to distribute them,'' she added.

But her support is remarkable largely because it has
\href{https://www.nytimes3xbfgragh.onion/2019/11/19/us/politics/nikki-haley-trump.html}{not
been constant}. Last August, she pushed back against the president when
he
\href{https://www.nytimes3xbfgragh.onion/2019/08/02/us/trump-elijah-cummings-house-robbed.html}{cast
attention on an attempted break-in} at the Baltimore home of
Representative Elijah E. Cummings.

``This is so unnecessary,'' Ms. Haley
\href{https://twitter.com/nikkihaley/status/1157316435221405697?lang=en}{wrote
on Twitter}.

During her time as ambassador under Mr. Trump, Ms. Haley appeared to
perfect the art of distancing herself from the most criticized policies
of the White House, even while she stayed publicly loyal to Mr. Trump.
She managed to toe a tougher line on Russia than her boss, but she left
on good terms. Since then, Ms. Haley has
\href{https://www.nytimes3xbfgragh.onion/2020/08/08/us/politics/kristi-noem-pence-trump.html}{frequently
been floated} as a possible replacement for Vice President Mike Pence.

---
\href{https://www.nytimes3xbfgragh.onion/by/jennifer-medina}{Jennifer
Medina}, \href{https://www.nytimes3xbfgragh.onion/by/edward-wong}{Edward
Wong} and \href{https://www.nytimes3xbfgragh.onion/by/eric-schmitt}{Eric
Schmitt}

\hypertarget{advertisement-1}{%
\subsubsection{Advertisement}\label{advertisement-1}}

\protect\hyperlink{after-dfp-ad-mid2}{Continue reading the main story}

\hypertarget{a-cuban-american-speaker-warned-of-communism-with-florida-voters-in-mind}{%
\subsection{\texorpdfstring{\protect\hyperlink{a-cuban-american-speaker-warned-of-communism-with-florida-voters-in-mind}{A
Cuban-American speaker warned of communism, with Florida voters in
mind.}}{A Cuban-American speaker warned of communism, with Florida voters in mind.}}\label{a-cuban-american-speaker-warned-of-communism-with-florida-voters-in-mind}}

\includegraphics{https://static01.graylady3jvrrxbe.onion/images/2020/08/24/us/politics/24-live-maximoalvarez/24-live-maximoalvarez-articleLarge.png?quality=75\&auto=webp\&disable=upscale}

Maximo Alvarez, a Cuban-American businessman and supporter of President
Trump from South Florida, claimed on Monday at the Republican convention
that the president was ``fighting against the forces of anarchy and
communism,'' echoing G.O.P. messaging that has tried to tie Joseph R.
Biden Jr. to the Democratic Party's more progressive politicians.

``What about his opponent and the rest of the D.C. swamp?'' Mr. Alvarez
said, according to prepared remarks. ``I have no doubt they will hand
the country over to those dangerous forces.''

Mr. Alvarez, who immigrated in 1961 as part of ``Operation Pedro Pan,''
a United States-backed effort meant to transport young people opposed to
Fidel Castro's government out of Cuba, said that while Mr. Trump ``may
not always care about being politically correct,'' he was keeping the
``far left'' out of power.

''I'm speaking to you today because I've seen people like this before,''
he said. ``I've seen movements like this before. I've seen ideas like
this before and I'm here to tell you, we cannot let them take over our
country.''

``Those false promises --- spread the wealth, defund the police, trust a
socialist state more than your family and community --- don't sound
radical to my ears,'' he said. ``They sound familiar.''

Between a quarter and a third of Hispanic voters have chosen the
Republican presidential candidate in elections since 1972, and
Cuban-Americans have long been the most influential and prominent
example of such voters, particularly in Florida, which is known for
tight presidential races.

Even as Mr. Trump pursued harsh immigration policies and made
inflammatory remarks about Latinos, he has made
\href{https://www.nytimes3xbfgragh.onion/2020/07/19/us/goya-trump-hispanic-vote.html}{some
attempts to maintain} his small but durable support among those voters.

Both Ronald Reagan and George H.W. Bush made serious efforts to attract
more Hispanic voters to the Republican Party. In 1986, Mr. Reagan
granted amnesty to roughly three million undocumented immigrants,
including many from Mexico and Central America.

And both Mr. Bush and his son engaged in significant Hispanic outreach.
In 2000, George W. Bush received nearly 40 percent of the Latino vote,
more than any Republican candidate before or since.

---
\href{https://www.nytimes3xbfgragh.onion/by/jennifer-medina}{Jennifer
Medina}

\hypertarget{sean-parnell-the-rare-republican-in-a-battleground-district-to-speak-made-the-case-for-trump-in-pennsylvania}{%
\subsection{\texorpdfstring{\protect\hyperlink{sean-parnell-the-rare-republican-in-a-battleground-district-to-speak-made-the-case-for-trump-in-pennsylvania}{Sean
Parnell, the rare Republican in a battleground district to speak, made
the case for Trump in
Pennsylvania.}}{Sean Parnell, the rare Republican in a battleground district to speak, made the case for Trump in Pennsylvania.}}\label{sean-parnell-the-rare-republican-in-a-battleground-district-to-speak-made-the-case-for-trump-in-pennsylvania}}

\includegraphics{https://static01.graylady3jvrrxbe.onion/images/2020/08/24/us/politics/24-live-parnell-18112/24-live-parnell-18112-articleLarge.jpg?quality=75\&auto=webp\&disable=upscale}

In a convention populated by President Trump's family members, employees
and ardent loyalists, Sean Parnell is the rare candidate with a
competitive election this fall who is making the case for Mr. Trump's
re-election.

Mr. Parnell, a former Army Ranger who wrote a best-selling book about
his time in Afghanistan, is running for a House district Pittsburgh
suburbs
\href{https://www.nytimes3xbfgragh.onion/2020/08/16/us/politics/joe-biden-conor-lamb-trump.html}{represented
by Conor Lamb, a Democrat}.

Mr. Parnell made clear his intention to be tied to Mr. Trump, announcing
the start of his campaign last October on ``Fox and Friends,'' the
president's favorite morning cable news program.

Since then Mr. Parnell has run a credible campaign, raising more money
than did Mr. Lamb during the three-month period ending June 30. And
while his race is considered competitive, Mr. Parnell remains an
underdog. The Cook Political Report has the district rated ``likely
Democratic.''

On Monday Mr. Parnell described the Democratic Party as an organization
that has moved too far to the left. Mr. Trump, he said, oversaw a
booming economy and allowed Americans to make decisions for themselves
without interference from the federal government.

``President Trump unleashed the economic might of this nation like no
other president in our history,'' Mr. Parnell said. ``He triggered the
rising tide of working families, brought us energy independence,
reclaimed jobs from overseas that Democrats said would never return. He
has fiercely defended the besieged First and Second Amendments. That's
just a start. With four more years, imagine what we can achieve by
simply working with our president.''

--- \href{https://www.nytimes3xbfgragh.onion/by/reid-j-epstein}{Reid J.
Epstein}

\hypertarget{republicans-rewrote-history-on-the-coronavirus}{%
\subsection{\texorpdfstring{\protect\hyperlink{republicans-rewrote-history-on-the-coronavirus}{Republicans
rewrote history on the
coronavirus.}}{Republicans rewrote history on the coronavirus.}}\label{republicans-rewrote-history-on-the-coronavirus}}

\includegraphics{https://static01.graylady3jvrrxbe.onion/images/2020/08/24/us/politics/24vid-rnc-trump-wh/merlin_176139594_89317cb9-1adc-451f-9001-8ee192e91995-videoSixteenByNine3000.jpg}

Throughout the first night of the Republican convention, multiple
speakers, including President Trump, occasionally referred to the
pandemic in the past tense, as the event undertook a significant
rewriting of the history of the coronavirus pandemic response and tried
to portray the president's response as swift and decisive with a
particular focus on his decision to ban travel from China in January.

In his second appearance of the day, Mr. Trump broke with years of
tradition and appeared from the White House for a political event,
hosting a segment on his administration's response to the coronavirus
featuring some frontline workers.

Yet in a segment intended to show some empathy from the president, it
was clear that Mr. Trump's tone would remain combative and defensive
about the virus. He repeatedly referred to the virus with a racist name,
calling it the ``China virus,'' and alluding that he wanted to call it
something else but that he wouldn't so as not to upset people.

While Mr. Trump did institute a limited ban on travel from China early
on, it was essentially his final policy on the coronavirus for a month.
In that time, he declined to warn Americans, played down the virus,
failed to expand testing and refused to publicly wear a mask, even as
some members of his own party were calling on Americans to wear masks to
help slow the spread of the virus.

Standing in the center of a semi-socially distanced group, none of whom
were wearing masks, Mr. Trump quickly ran through each person's
background and how they had helped in the national effort to fight the
pandemic. He thanked them for their service, but kept it framed in
support of his presidency.

``I love the truckers, they're on my side,'' Mr. Trump said.

He also continued to try to portray his administration's response as a
success: ``We just have to make this virus go away,'' Mr. Trump said,
``and it's happening.''

The United States leads the world in coronavirus cases and deaths: more
than 5.7 million known infections, and more than 177,000 deaths.

--- \href{https://www.nytimes3xbfgragh.onion/by/nick-corasaniti}{Nick
Corasaniti}

\hypertarget{advertisement-2}{%
\subsubsection{Advertisement}\label{advertisement-2}}

\protect\hyperlink{after-dfp-ad-mid3}{Continue reading the main story}

\hypertarget{steve-scalise-offered-praise-for-trump-and-repeated-the-falsehood-that-biden-plans-to-defund-the-police}{%
\subsection{\texorpdfstring{\protect\hyperlink{steve-scalise-offered-praise-for-trump-and-repeated-the-falsehood-that-biden-plans-to-defund-the-police}{Steve
Scalise offered praise for Trump, and repeated the falsehood that Biden
plans to `defund' the
police.}}{Steve Scalise offered praise for Trump, and repeated the falsehood that Biden plans to `defund' the police.}}\label{steve-scalise-offered-praise-for-trump-and-repeated-the-falsehood-that-biden-plans-to-defund-the-police}}

\includegraphics{https://static01.graylady3jvrrxbe.onion/images/2020/08/24/us/politics/24vid-rnc-scalise/merlin_176141001_2d1af8f0-eea6-4410-981b-3401269b0d8d-videoSixteenByNineJumbo1600.jpg}

Some speakers tried to portray President Trump as a man of deep
compassion who is no less empathetic than Mr. Biden, no matter what the
news media (or his own Twitter account) might say.

One of the most emotional testimonials of the night came from
Representative Steve Scalise of Louisiana --- a close ally of Mr.
Trump's who was grievously wounded by a gunman in 2017 --- who modulated
warm memories of Mr. Trump with a flamethrower attack on Mr. Biden as
leader of ``a party that wants to burn the foundations of our country to
the ground.''

Mr. Scalise began with the aftermath of the shooting, an attack by a
gunman who sought revenge against Mr. Trump by firing on Republicans as
they practiced for the annual congressional softball game.

``That same night, Donald Trump came to the hospital, along with first
lady Melania Trump. They consoled my wife Jennifer --- they were there
for my family in my darkest hours,'' said Mr. Scalise, who nearly died
from a severe wound to his hip.

``Donald Trump would call to check on me throughout the following weeks,
just to see how I was doing. That's the kind of person he is. That's the
side of Donald Trump that the media will never show you.''

Then, without any real transition, Mr. Scalise, shifted tone, and
repeated the falsehood --- often uttered by the president --- that Mr.
Biden plans to ``defund'' the police.

``This is personal --- I wouldn't be here without the bravery and
heroism of the men and women in law enforcement who saved my life,'' he
added. ``President Trump stands with those brave men and women. Joe
Biden has embraced the left's insane mission to defund them.''

--- \href{https://www.nytimes3xbfgragh.onion/by/glenn-thrush}{Glenn
Thrush}

\hypertarget{kimberly-guilfoyle-delivered-a-dark-denunciation-of-democrats}{%
\subsection{\texorpdfstring{\protect\hyperlink{kimberly-guilfoyle-delivered-a-dark-denunciation-of-democrats}{Kimberly
Guilfoyle delivered a dark denunciation of
Democrats.}}{Kimberly Guilfoyle delivered a dark denunciation of Democrats.}}\label{kimberly-guilfoyle-delivered-a-dark-denunciation-of-democrats}}

\includegraphics{https://static01.graylady3jvrrxbe.onion/images/2020/08/24/us/politics/24-live-kimguilfoyle-1419/merlin_176140416_e181de28-ff73-44ce-b9f8-ec423e464c6f-articleLarge.jpg?quality=75\&auto=webp\&disable=upscale}

Kimberly Guilfoyle, a top fund-raiser for President Trump and the
girlfriend of Donald Trump Jr., delivered a blistering speech on Monday
at the Republican convention that contrasted the ``socialist agenda'' of
the Democrats with the president's role in delivering what she called
``the greatest economy the world had ever known.''

Ms. Guilfoyle, a former Fox News host, painted a dismaying picture of a
future in which rioters would destroy America's cities and the Democrats
would steal Americans' liberty. She cited California as a cautionary
tale, calling it a ``land of discarded needles in parks.''

(Her first husband was the Democratic politician Gavin Newsom, now the
governor of California and a foe of the president.)

During her speech, Ms. Guilfoyle called herself a ``first-generation
American'' and said both of her parents were immigrants. Neither claim
was quite right --- Ms. Guilfoyle was born in the United States, and the
Census Bureau uses the term ``first-generation'' to designate people who
are born in a foreign country and immigrate to the United States. And
while her father was born in Ireland and immigrated to the United
States, her mother was born in Puerto Rico, a United States territory,
making her an American from birth.

Ms. Guilfoyle, now 51, started dating the younger Mr. Trump, now 42, in
2018, and has been a constant figure at his side ever since. They made
their relationship public shortly after his wife, Vanessa, filed for
divorce.

``Some of you may have heard I recently started dating'' the president's
son, Ms. Guilfoyle
\href{https://www.washingtonpost.com/lifestyle/style/kimberly-guilfoyle-was-once-compared-to-jackie-kennedy-now-shes-basically-a-trump/2018/08/22/eed842f0-9756-11e8-810c-5fa705927d54_story.html}{told
a crowd} of high-school Republicans in 2018. ``I mean, what can I tell
you? Mama's a closer, you know what I mean?''

She drew some unwelcome publicity in March when a
\href{https://www.nytimes3xbfgragh.onion/2020/03/14/us/politics/trump-coronavirus-mar-a-lago.html}{number
of guests tested positive} for Covid-19 after going to her lavish 51st
birthday party at Mar-a-Lago, the president's club in Florida. In July,
she herself tested positive for the virus, but said that she was
asymptomatic.

She worked as a prosecutor in San Francisco years ago and is a
formidable defender not only of the president, but also of her
boyfriend. At an event last winter to promote his book ``Triggered: How
the Left Thrives on Hate and Wants to Silence Us,'' she and Mr. Trump
were heckled by a crowd of right-wing conservatives apparently angry at
the evening's format. Mr. Trump started to respond, when Ms. Guilfoyle
angrily
\href{.\%20https:/www.washingtonpost.com/nation/2019/11/11/donald-trump-jr-book-talk-ucla-derailed-by-far-right-protesters}{cut
in}.

``You're not making your parents proud by being rude and disruptive and
discourteous,'' she scolded. Then she questioned the dating prowess of
the mostly male crowd.

``Let me tell you something, I bet you engage and go on online dating,''
she said, ``because you're impressing no one here to get a date in
person.''

--- \href{https://www.nytimes3xbfgragh.onion/by/sarah-lyall}{Sarah
Lyall} and
\href{https://www.nytimes3xbfgragh.onion/by/katie-benner}{Katie Benner}

\hypertarget{mark-and-patricia-mccloskey-who-brandished-guns-at-peaceful-protesters-made-a-fear-mongering-pitch}{%
\subsection{\texorpdfstring{\protect\hyperlink{mark-and-patricia-mccloskey-who-brandished-guns-at-peaceful-protesters-made-a-fear-mongering-pitch}{Mark
and Patricia McCloskey, who brandished guns at peaceful protesters, made
a fear-mongering
pitch.}}{Mark and Patricia McCloskey, who brandished guns at peaceful protesters, made a fear-mongering pitch.}}\label{mark-and-patricia-mccloskey-who-brandished-guns-at-peaceful-protesters-made-a-fear-mongering-pitch}}

\includegraphics{https://static01.graylady3jvrrxbe.onion/images/2020/08/24/us/politics/24-live-mccloskeys2/24-live-mccloskeys2-articleLarge.png?quality=75\&auto=webp\&disable=upscale}

Mark and Patricia McCloskey, the St. Louis couple who threatened
peaceful protesters in June, echoed President Trump's claims that
Democratic policies would put Americans' lives in danger and made
several false claims about those policies.

Most notably, Ms. McCloskey said that Democrats wanted to ``abolish the
suburbs altogether by ending single-family home zoning,'' which is not
true.

Mr. Trump and his supporters have inaccurately described a
\href{https://www.federalregister.gov/documents/2015/07/16/2015-17032/affirmatively-furthering-fair-housing}{regulation}
issued by the Department of Housing and Urban Development in 2015,
casting it as a threat to the lifestyles of white suburbanites. The
Trump administration indefinitely delayed implementing it in 2018.

The regulation required communities that received federal housing
funding to have plans to ensure housing access regardless of race, but
``it doesn't dictate how they have to do that,'' Julián Castro, who was
the secretary of housing and urban development when the regulation was
finalized, said Monday night, calling the misrepresentations ``a
shameful, deceitful and calculated ploy to drum up racial resentment and
white fear.''

``The federal government does not have authority to dictate zoning
decisions of local communities,'' Mr. Castro, who ran in the 2020
Democratic presidential primary, added. ``That's very explicit, that's
settled, and this rule in no way requires communities to make specific
decisions about zoning.''

Mr. McCloskey also described efforts to ``defund the police'' as
evidence that Democrats ``no longer view the government's job as
protecting honest citizens from criminals, but rather protecting
criminals from honest citizens.''

While some Democrats support defunding the police, the party's nominee,
Joseph R. Biden Jr., does not --- and supporters generally
\href{https://www.nytimes3xbfgragh.onion/2020/06/08/us/what-does-defund-police-mean.html}{use
the term} to refer not to abolishing policing altogether, but to
redistributing some police funding to other public services.

The McCloskeys came to national attention in June after Mr. Trump
\href{https://www.nytimes3xbfgragh.onion/2020/06/29/us/politics/trump-white-couple-protesters.html}{retweeted
a video} of them pointing guns at Black protesters in St. Louis. The
demonstrators had been marching past the McCloskeys' mansion, which is
on a private street, on their way to Mayor Lyda Krewson's house to
protest police violence and systemic racism.

The McCloskeys --- who
\href{https://www.stltoday.com/news/local/metro/portland-place-couple-who-confronted-protesters-have-a-long-history-of-not-backing-down/article_281d9989-373e-53c3-abcb-ecd0225dd287.html}{have
repeatedly sued people} over a wide variety of grievances, including
land disputes --- said afterward that they had feared for their lives
and their property. In reality, the protesters were unarmed and
peaceful. Multiple videos show that they did not threaten or approach
the McCloskeys, and some can be heard telling one another to ``keep
moving.''

Last month, the McCloskeys
\href{https://www.nytimes3xbfgragh.onion/2020/07/20/us/mark-patricia-mccloskey-charges.html}{were
charged with unlawful use of a weapon}, which is a felony, for
brandishing a semiautomatic rifle ``in an angry or threatening manner.''

Mr. McCloskey expressed anger at that on Monday, saying: ``Not a single
person in the out-of-control mob you saw at our house was charged with a
crime. But you know who was? We were.''

Videos show that there was no ``out-of-control mob.''

--- \href{https://www.nytimes3xbfgragh.onion/by/maggie-astor}{Maggie
Astor}

\hypertarget{advertisement-3}{%
\subsubsection{Advertisement}\label{advertisement-3}}

\protect\hyperlink{after-dfp-ad-mid4}{Continue reading the main story}

\hypertarget{andrew-pollack-a-parkland-father-tried-to-portray-biden-as-a-threat-to-the-nation}{%
\subsection{\texorpdfstring{\protect\hyperlink{andrew-pollack-a-parkland-father-tried-to-portray-biden-as-a-threat-to-the-nation}{Andrew
Pollack, a Parkland father, tried to portray Biden as a threat to the
nation.}}{Andrew Pollack, a Parkland father, tried to portray Biden as a threat to the nation.}}\label{andrew-pollack-a-parkland-father-tried-to-portray-biden-as-a-threat-to-the-nation}}

Andrew Pollack, whose daughter, Meadow, was killed in the 2018 school
shooting in Parkland, Fla., praised President Trump's response to the
massacre and argued that Mr. Trump would do a better job than Joe Biden
at protecting students from gun violence.

Mr. Pollack was the only parent of a Parkland victim
\href{https://www.nytimes3xbfgragh.onion/2018/02/21/us/politics/trump-guns-school-shooting.html}{to
attend the listening session} President Trump held at the White House a
week after the shooting, and he said in his speech on Monday that the
event had shown him that Mr. Trump was ``a great listener'' who ``cuts
through the BS.''

He praised Mr. Trump's formation of the Federal Commission on School
Safety, which issued recommendations including tighter building security
and programs to arm school personnel. The commission endorsed so-called
red-flag laws, which allow the temporary confiscation of guns from
people deemed to pose an imminent threat to themselves or others, but
otherwise did not focus on gun restrictions.

``Gun control laws didn't fail my daughter. People did,'' Mr. Pollack
said, arguing that Parkland officials' failure to respond to warnings
that the gunman was dangerous was a result of Obama-era policies on
school discipline.

``I was just fine with the old approach to discipline and safety --- it
was called discipline and safety,'' Mr. Pollack said. ``But the
Obama-Biden administration took Parkland's bad policies and forced them
into schools across America.''

The policies he referred to, known as restorative justice, are intended
to reduce suspensions, expulsions and arrests, forms of discipline that
are used disproportionately against students of color. Among other
things, restorative justice programs push administrators to rely on
school personnel rather than the police to handle student discipline,
and to reject
``\href{https://www.nytimes3xbfgragh.onion/2016/10/03/us/the-unintended-consequences-of-taking-a-hard-line-on-school-discipline.html}{zero
tolerance}'' policies in favor of rehabilitation.

Mr. Pollack noted, correctly, that officials failed to act on numerous
warning signs that the gunman posed a threat, and that school districts
in Broward County, which includes Parkland, were early adopters of
restorative justice policies. But there is no evidence that those
policies led to the failures. The gunman was, in fact, removed from
Marjory Stoneman Douglas High School before the shooting, and his
worrisome behavior was repeatedly reported to the Parkland police and
the Broward County Sheriff's Office.

After the shooting, Mr. Pollack helped pass a Florida law that, among
other measures, allowed some school personnel to be armed. He has also
pushed for tighter security, such as metal detectors, in schools.

His views set him apart from many other Parkland survivors and victims'
families. Among those who have been politically active, most have
aligned themselves with Democrats and groups supporting gun
restrictions, which Mr. Pollack opposes.
\href{https://twitter.com/AndrewPollackFL/status/1297552296742658050}{He
tweeted} on Sunday that Joseph R. Biden Jr.'s proposed policies would
leave schools ``defenseless.''

Fred Guttenberg --- whose daughter, Jaime, was also killed in the
Parkland shooting --- participated in the roll call vote at the
Democratic National Convention last week, and many students who survived
the shooting have been involved in the
\href{https://www.nytimes3xbfgragh.onion/2018/08/15/us/politics/parkland-students-voting.html}{March
for Our Lives movement} for stricter gun laws.

--- \href{https://www.nytimes3xbfgragh.onion/by/maggie-astor}{Maggie
Astor}

\hypertarget{the-republican-convention-is-a-far-lower-tech-production-than-the-democratic-one}{%
\subsection{\texorpdfstring{\protect\hyperlink{the-republican-convention-is-a-far-lower-tech-production-than-the-democratic-one}{The
Republican convention is a far lower-tech production than the Democratic
one.}}{The Republican convention is a far lower-tech production than the Democratic one.}}\label{the-republican-convention-is-a-far-lower-tech-production-than-the-democratic-one}}

Last week, the Democratic convention was a triumph of technology, with
feeds from dozens of cities piped into a two-hour nightly program that
was seamless television production.

The first night of the Republican convention, by contrast, looks more
like a college media event. A series of speakers are appearing from a
single stage inside the Mellon Auditorium in Washington --- with the
pre-taped segments from outside of Washington the exception breaking up
the live speeches from the capital.

While Democrats produced images from around the country, the main scene
change from the Republican convention was the sign on the lectern from
which Mr. Trump's supporters spoke. Even Mr. Trump's first appearance
was an in-person chat with a half-dozen supporters gathered inside the
White House --- while former Vice President Joseph R. Biden Jr. last
week spoke with Democrats via videoconference last week.

And the Republican speeches are longer.

Vernon Jones, a Democratic Georgia state legislator who endorsed Mr.
Trump, was given a seven-minute time slot --- an amount of time afforded
last week only to the party's biggest stars: Senator Kamala Harris,
Barack and Michelle Obama and Mr. Biden and his wife, Jill Biden.

The lower production values reflect a shorter runway for Republicans to
plan their virtual convention. Until late July, Mr. Trump had insisted
the event take place in person, first in Charlotte, N.C., and later in
Jacksonville, Fla.

But Democratic convention planners began plotting an all-virtual
convention in early April, when it first became likely that it would not
be possible to gather large numbers of people together this summer.

--- \href{https://www.nytimes3xbfgragh.onion/by/reid-j-epstein}{Reid J.
Epstein}

\hypertarget{vernon-jones-a-democratic-state-legislator-crossed-party-lines-to-back-trump}{%
\subsection{\texorpdfstring{\protect\hyperlink{vernon-jones-a-democratic-state-legislator-crossed-party-lines-to-back-trump}{Vernon
Jones, a Democratic state legislator, crossed party lines to back
Trump.}}{Vernon Jones, a Democratic state legislator, crossed party lines to back Trump.}}\label{vernon-jones-a-democratic-state-legislator-crossed-party-lines-to-back-trump}}

\includegraphics{https://static01.graylady3jvrrxbe.onion/images/2020/08/24/us/politics/24-live-vernonjones2/24-live-vernonjones2-articleLarge.jpg?quality=75\&auto=webp\&disable=upscale}

Vernon Jones, a Black Democratic state representative from Georgia,
appeared at the Republican National Convention on Monday night and
accused his party of taking Black voters for granted.

``The Democratic Party does not want Black people to leave their mental
plantation,'' Mr. Jones said, according to prepared remarks. ``We've
been forced to be there for decades and generations.''

Mr. Jones, who was elected to the Georgia House of Representatives in
2016, voted twice for George W. Bush and endorsed President Trump in
April. He later announced he would not seek another term for his seat.

``Now you know, when I made the public announcement of my support for
President Trump, all hell broke loose,'' Mr. Jones said Monday,
according to his prepared remarks. ``I was threatened, called an
embarrassment, and asked to resign by my own party. Unfortunately,
that's consistent with the Democratic Party and how they view
independent-thinking Black men and women.''

For decades, polls have shown that an overwhelming majority of Black
voters favor Democrats.

Mr. Jones was one of several people of color the Trump campaign
highlighted on Monday, as the Republican Party tried to portray itself
as inclusive despite the president's continued messaging disparaging
African-Americans, Latinos and Muslims.

Mr. Jones has been a key surrogate for the Trump campaign in Georgia,
which traditionally votes Republican but has emerged as a key target for
Democrats in this year's presidential race.

During the 2008 Senate primary, Mr. Jones sent out a campaign mailer
picturing himself with Barack Obama, a senator at the time, standing
under the presidential campaign slogan ``Yes we can!'' But Mr. Obama had
not backed Mr. Jones, whose career was mired in controversy including an
accusation of rape, and the images of the two had been taken from
different photos, using digital manipulation to place them on a single
background. (No charges were filed related to the rape accusation.)

When the mailer caused a stir, Mr. Jones suggested that his Senate bid
would give Mr. Obama a
\href{https://www.nytimes3xbfgragh.onion/2008/07/16/us/politics/16georgia.html?searchResultPosition=1}{lift
in Georgia}.

---
\href{https://www.nytimes3xbfgragh.onion/by/jennifer-medina}{Jennifer
Medina}

\hypertarget{advertisement-4}{%
\subsubsection{Advertisement}\label{advertisement-4}}

\protect\hyperlink{after-dfp-ad-mid5}{Continue reading the main story}

\hypertarget{herschel-walker-said-he-takes-the-claim-that-trump-is-racist-as-a-personal-insult}{%
\subsection{\texorpdfstring{\protect\hyperlink{herschel-walker-said-he-takes-the-claim-that-trump-is-racist-as-a-personal-insult}{Herschel
Walker said he takes the claim that Trump is racist as a `personal
insult.'}}{Herschel Walker said he takes the claim that Trump is racist as a `personal insult.'}}\label{herschel-walker-said-he-takes-the-claim-that-trump-is-racist-as-a-personal-insult}}

\includegraphics{https://static01.graylady3jvrrxbe.onion/images/2020/08/24/us/politics/24-live-walker-18022/24-live-walker-18022-articleLarge.jpg?quality=75\&auto=webp\&disable=upscale}

President Trump has hired the fewest number of Black people to positions
of authority in his White House and on his presidential campaign of any
chief executive in decades.

But convention planners, trying to counter the display of diversity
showcased by Democrats last week, have offered airtime to several
high-profile people of color this week in an attempt to soften the
negative impression voters of all stripes have of
\href{https://abcnews.go.com/Politics/trump-trouble-thirds-americans-disapprove-handling-covid-19/story?id=72088513}{his
handling of race relations}.

One of the first such speakers was Herschel Walker, the Heisman
Trophy-winning running back Mr. Trump signed to his short-lived United
States Football League franchise, the New Jersey Generals, in the early
1980s.

The president has remained close with Mr. Walker, a multisport standout
who went on to play with the Dallas Cowboys and even dabbled in
bobsledding, martial arts and ballet.

``It hurts my soul to hear the terrible names that people call Donald
--- the worst is one `racist,''' said Mr. Walker, who grew up in rural
Georgia.

``I take it as a personal insult that people would think I would have a
37-year friendship with a racist,'' he said. ``People who think that
don't know what they are talking about. Growing up in the Deep South, I
have seen racism up close. I know what it is. And it isn't Donald
Trump.''

\hypertarget{jim-jordan-sought-to-paint-trump-as-empathetic-with-a-personal-story}{%
\subsection{\texorpdfstring{\protect\hyperlink{jim-jordan-sought-to-paint-trump-as-empathetic-with-a-personal-story}{Jim
Jordan sought to paint Trump as empathetic with a personal
story.}}{Jim Jordan sought to paint Trump as empathetic with a personal story.}}\label{jim-jordan-sought-to-paint-trump-as-empathetic-with-a-personal-story}}

\includegraphics{https://static01.graylady3jvrrxbe.onion/images/2020/08/24/us/politics/24-live-jimjordan/24-live-jimjordan-articleLarge.png?quality=75\&auto=webp\&disable=upscale}

Last week, Democrats painted a picture of Joseph R. Biden Jr. as a
compassionate man full of empathy for those who are hurting.

The obvious and unspoken contrast was to President Trump, who only on
rare occasions mentions the
\href{https://www.nytimes3xbfgragh.onion/interactive/2020/us/coronavirus-us-cases.html}{more
than 175,000 people} in the United States who have died from the
coronavirus and who has a tendency to make statements about the dead
largely about himself.

On Monday night during the Republican National Convention,
Representative Jim Jordan of Ohio offered something of a rebuttal: Mr.
Trump, he said, can also have empathy, albeit in private and for
political allies.

Mr. Jordan's nephew, a college wrestler, was killed two years ago in a
car accident, he said.

``It was a Saturday morning, three days after the accident,'' Mr. Jordan
said. ``I walked to the car, to head up to Eli's parents' home, when the
president called. We talked about a few issues. And then he asked how
the family was doing. I said they're doing `OK, but it's tough.' ''

Mr. Trump, Mr. Jordan said, then spent five minutes speaking to Mr.
Jordan's nephew's father. Mr. Jordan did not relay any particularly
soothing words the president conveyed, but said it helped the family in
its mourning.

``That's the president I've gotten to know the last four years,'' said
Mr. Jordan, who was the president's
\href{https://www.nytimes3xbfgragh.onion/2019/11/15/us/politics/jim-jordan-impeachment-hearings.html}{chief
defender during the impeachment inquiry}. ``That's the individual who's
made America great again and who knows America's best days are still in
front of us.''

--- \href{https://www.nytimes3xbfgragh.onion/by/reid-j-epstein}{Reid J.
Epstein}

\hypertarget{ronna-mcdaniel-a-real-housewife-of-michigan-doubled-down-on-a-trump-tweet}{%
\subsection{\texorpdfstring{\protect\hyperlink{ronna-mcdaniel-a-real-housewife-of-michigan-doubled-down-on-a-trump-tweet}{Ronna
McDaniel, `a real housewife' of Michigan, doubled down on a Trump
tweet.}}{Ronna McDaniel, `a real housewife' of Michigan, doubled down on a Trump tweet.}}\label{ronna-mcdaniel-a-real-housewife-of-michigan-doubled-down-on-a-trump-tweet}}

\includegraphics{https://static01.graylady3jvrrxbe.onion/images/2020/08/24/us/politics/24vid-rnc-mcdaniel/24vid-rnc-mcdaniel-videoSixteenByNine3000.jpg}

Ronna McDaniel, the chairwoman of the Republican National Committee,
embraced President Trump's (in)famous
\href{https://www.nytimes3xbfgragh.onion/2020/07/30/upshot/trump-suburban-voters.html}{``suburban
housewife'' tweet} at her party's convention on Monday, describing
herself as a proud ``housewife'' and ``mom'' who had risen to the top of
the G.O.P. on her merits --- rather than her gender.

``I'm actually a real housewife and a mom from Michigan with two
wonderful kids in public school who happens to be only the second woman
in 164 years to run the Republican Party,'' said Ms. McDaniel, a niece
of Mitt Romney, the 2012 Republican nominee.

She went on to take a swipe at Senator Kamala Harris, the first woman of
color to appear on a major party's ticket.

``Unlike Joe Biden, President Trump didn't choose me because I'm a woman
--- he chose me because I was the best person for the job,'' said Ms.
McDaniel, who has sought to marginalize Trump skeptics in the party
during her three-plus years at the helm.

In her introductory address Monday night, Ms. McDaniel, the former party
chairwoman in Michigan and a formidable fund-raiser, blasted Democrats
for ``talking about how much they despise our president'' while
revealing ``very little about their actual policies.''

\href{https://prod-cdn-static.gop.com/docs/Resolution_Platform_2020.pdf?_ga=2.165306300.2055661719.1598124638-455285808.1584478680}{The
Republican Party does not have a new platform this year}, for the first
time in recent memory.

Instead, the convention over which Ms. McDaniel presides unanimously
adopted a resolution that simply expressed ``the party's strong support
for President Donald Trump'' and his administration.

Ms. McDaniel, echoing the combative culture-war theme struck by the
president, took aim at the quartet of actresses and activists who hosted
the Democrats' slickly produced digital convention --- and used an
increasingly common, but deceptive, description of Democrats' stances on
abortion, saying they would ``allow abortion up until the point of
birth.''

Only about 1 percent of abortions are performed after 20 weeks'
gestation,
\href{https://www.cdc.gov/reproductivehealth/data_stats/abortion.htm}{according
to the Centers for Disease Control and Prevention}. The Democratic
presidential candidates mostly refused to endorse limits in the third
trimester --- noting that such abortions generally relate to threats to
the woman's health or severe fetal abnormalities, and arguing that the
decision in those cases should be between women and their doctors ---
but abortions just before labor would have started don't happen.

Ms. McDaniel also rejected the notion that her candidate was destined to
lose the empathy war against Mr. Biden.

``Their argument for Joe Biden boiled down to the fact that they think
he's a nice guy,'' she said.

``In the nearly four years I've worked on behalf of President Trump,
I've seen up close a man who has a deep love for family,'' she went on.
``A man who has reverence for the office of the presidency. A man with
an incredible respect for law enforcement and our military. I've seen
private moments where he comforts Americans in times of pain and
sadness.''

--- \href{https://www.nytimes3xbfgragh.onion/by/glenn-thrush}{Glenn
Thrush}

\hypertarget{advertisement-5}{%
\subsubsection{Advertisement}\label{advertisement-5}}

\protect\hyperlink{after-dfp-ad-mid6}{Continue reading the main story}

\hypertarget{kimberly-klacik-whose-campaign-ad-went-viral-accused-democrats-of-taking-black-voters-for-granted}{%
\subsection{\texorpdfstring{\protect\hyperlink{kimberly-klacik-whose-campaign-ad-went-viral-accused-democrats-of-taking-black-voters-for-granted}{Kimberly
Klacik, whose campaign ad went viral, accused Democrats of taking Black
voters for
granted.}}{Kimberly Klacik, whose campaign ad went viral, accused Democrats of taking Black voters for granted.}}\label{kimberly-klacik-whose-campaign-ad-went-viral-accused-democrats-of-taking-black-voters-for-granted}}

\includegraphics{https://static01.graylady3jvrrxbe.onion/images/2020/08/24/us/politics/24-live-klacik-1745/merlin_176138043_2d90e622-83f4-45ea-b77c-f6a5d1f8cd7d-articleLarge.jpg?quality=75\&auto=webp\&disable=upscale}

Kimberly Klacik, the Republican candidate in a solidly Democratic
congressional district in Maryland, appeared on a national convention
stage just a week after her first introduction to a national audience
and accused Democrats of destroying cities and taking Black voters for
granted.

``Abandoned buildings, liquor stores on every corner, drug addicts and
guns on the street --- that is now the norm in many neighborhoods,'' she
said of Baltimore, part of which is in Maryland's Seventh Congressional
District, where she is running. ``Sadly, this same cycle of decay exists
in many of America's Democrat-run cities. And yet, the Democrats still
assume that Black people will vote for them, no matter how much they let
us down and take us for granted.''

Her description of parts of Baltimore was reminiscent of President
Trump's, who last year
\href{https://www.nytimes3xbfgragh.onion/2019/07/27/us/politics/trump-elijah-cummings.html}{described
the Seventh District} as a ``disgusting, rat- and rodent-infested mess''
where ``no human being would want to live.'' That was part of a racist
attack by Mr. Trump on the district's representative at the time, Elijah
Cummings,
\href{https://www.nytimes3xbfgragh.onion/2019/10/17/us/politics/elijah-cummings-death-illness.html}{who
died last year}.

In her speech on Monday, Ms. Klacik, who is Black, continued: ``Joe
Biden believes we can't think for ourselves --- that the color of
someone's skin dictates their political views. We're not buying the lies
anymore. You and your party have neglected us for far too long.''

Polls show that an overwhelming majority of Black voters plan to vote
for Mr. Biden, who won the Democratic primary on the strength of their
support.

The district where Ms. Klacik is running is 26 percentage points more
Democratic than the nation as a whole, according to the Cook Partisan
Voting Index, and it is not considered competitive. Mr. Cummings held
the seat for more than 20 years.

But Ms. Klacik was invited to speak at the convention after
\href{https://twitter.com/kimKBaltimore/status/1295461903268040707}{one
of her campaign ads} --- in which she made largely the same arguments
she made in her speech --- went viral.

--- \href{https://www.nytimes3xbfgragh.onion/by/maggie-astor}{Maggie
Astor}

\hypertarget{arguing-biden-would-be-beholden-to-the-left-matt-gaetz-accused-him-of-supporting-policies-he-doesnt-back}{%
\subsection{\texorpdfstring{\protect\hyperlink{arguing-biden-would-be-beholden-to-the-left-matt-gaetz-accused-him-of-supporting-policies-he-doesnt-back}{Arguing
Biden would be beholden to the left, Matt Gaetz accused him of
supporting policies he doesn't
back.}}{Arguing Biden would be beholden to the left, Matt Gaetz accused him of supporting policies he doesn't back.}}\label{arguing-biden-would-be-beholden-to-the-left-matt-gaetz-accused-him-of-supporting-policies-he-doesnt-back}}

\includegraphics{https://static01.graylady3jvrrxbe.onion/images/2020/08/24/us/politics/24vid-rnc-gaetz/merlin_176138079_c04013fb-45c8-46a5-a5fb-aff9bd28f8fc-videoSixteenByNineJumbo1600.jpg}

In just two terms in Congress, Representative Matt Gaetz of Florida has
become one of President Trump's highest-profile surrogates, thanks to
his
\href{https://www.nytimes3xbfgragh.onion/2019/03/30/us/politics/matt-gaetz-trump.html}{relentless
efforts defending Mr. Trump} on cable television.

Mr. Gaetz, as much as any member of Congress, has channeled the spirit
of Trumpism, and on Monday night he delivered a series of searing yet
baseless attacks on former Vice President Joseph R. Biden Jr.

Mr. Biden, Mr. Gaetz argued, would be a mere avatar of the Democratic
Party's left wing, beholden to supporters of Senator Bernie Sanders ---
whom Mr. Biden handily defeated in the Democratic presidential primary.

``It's a horror film, really,'' he said. ``They'll disarm you, empty the
prisons, lock you in your home, and invite MS-13 to live next door. And
the defunded police aren't on their way.''

Mr. Biden, of course, does not support any of those things. While he has
called for putting in place universal background checks on new gun
purchases and
\href{https://www.nytimes3xbfgragh.onion/2019/08/11/opinion/joe-biden-ban-assault-weapons.html}{banning
assault-style weapons}, Mr. Biden did not support the confiscation of
existing weapons when it was proposed by former Representative Beto
O'Rourke of Texas last year. And Mr. Biden supports
\href{https://www.nytimes3xbfgragh.onion/2020/08/19/us/politics/democrats-biden-defund-police.html}{spending
more money} on law enforcement, not less.

For Mr. Gaetz, using the national convention stage to attack Mr. Biden
is his latest effort to not just defend Mr. Trump but also attack the
president's enemies.

Mr. Gaetz was an early and regular aggressor toward Robert S. Mueller
III's investigation into the 2016 Trump campaign's ties to Russia; he
\href{https://www.nytimes3xbfgragh.onion/2019/06/28/us/politics/matt-gaetz-ethics.html}{threatened
Michael D. Cohen}, Mr. Trump's former fixer, a day before Mr. Cohen was
to testify before the House Oversight Committee; and he mocked early
coronavirus precautions by wearing a gas mask during a House debate on
funding to address the pandemic.

In June, Twitter
\href{https://www.nytimes3xbfgragh.onion/2020/06/01/technology/twitter-matt-gaetz-warning.html}{affixed
a warning label} to a tweet of Mr. Gaetz's for glorifying violence after
he wrote: ``Now that we clearly see Antifa as terrorists, can we hunt
them down like we do those in the Middle East?''

But Mr. Gaetz did little to talk up Mr. Trump during his convention
speech, beyond calling him a ``visionary.'' His pitch for a second Trump
term amounted to little more than tearing into Mr. Biden and his running
mate, Senator Kamala Harris, while acknowledging that Mr. Trump isn't
always what people expect to see in a president.

``President Trump sometimes raises his voice --- and a ruckus,'' Mr.
Gaetz said. ``He knows that's what it takes to raise an army of patriots
who love America and will protect her.''

--- \href{https://www.nytimes3xbfgragh.onion/by/reid-j-epstein}{Reid J.
Epstein}

\hypertarget{charlie-kirk-made-a-nativist-pitch-for-four-more-years-of-trump}{%
\subsection{\texorpdfstring{\protect\hyperlink{charlie-kirk-made-a-nativist-pitch-for-four-more-years-of-trump}{Charlie
Kirk made a nativist pitch for four more years of
Trump.}}{Charlie Kirk made a nativist pitch for four more years of Trump.}}\label{charlie-kirk-made-a-nativist-pitch-for-four-more-years-of-trump}}

\includegraphics{https://static01.graylady3jvrrxbe.onion/images/2020/08/24/us/politics/24vid-rnc-charliekirk/merlin_176137818_394503d0-259d-4ee2-8579-5e2a0ec01d6b-videoSixteenByNineJumbo1600.jpg}

Charlie Kirk, the founder of the conservative group Turning Point USA
and a staunch defender of President Trump online, said at the Republican
National Convention on Monday night that the coming election posed a
threat to ``Western civilization,'' framing his support of Mr. Trump in
the same nativist spirit that powered the president's 2016 election.

``I am here tonight to tell you --- to warn you --- that this election
is a decision between preserving America as we know it, and eliminating
everything that we love,'' Mr. Kirk said.

In a laudatory speech, Mr. Kirk praised Mr. Trump as the ``the bodyguard
of Western civilization,'' repeatedly invoking his own Christian faith
and claiming the country's creation was ``centered around central
biblical ideals,'' despite the fact that the First Amendment of the
Constitution expressly prohibits a national religion.

Few speakers embody the social-media-fueled, Trump-aligned conservative
industrial complex more than Mr. Kirk. Now 26, he founded Turning Point
USA in 2012, when he was just 18, and it became one of many groups
courting young conservatives.

But his rapid embrace of Mr. Trump during the 2016 election quickly
turned Turning Point USA into a sprawling pro-Trump enterprise and made
Mr. Kirk a favorite of the president's. He consistently garnered
retweets from Mr. Trump and built an audience of more than 1.8 million
followers on Twitter.

Mr. Kirk has leaned into the provocative, sometimes inflammatory nature
of Mr. Trump's brand of politics, with occasional falsehoods and
misrepresentations.

``All of this is under attack by a group of bitter, deceitful, vengeful,
arrogant activists who wish to tear down this gift we have been given,''
Mr. Kirk said on Monday. He spoke of ``monuments,'' though he didn't
directly refer to the conservative push to preserve Confederate statues.
He also spoke of ``kicking doctors off of social media,'' though he
didn't mention that this had happened because they had
\href{https://www.nytimes3xbfgragh.onion/2020/07/28/technology/virus-video-trump.html}{spread
dangerous disinformation} about the coronavirus pandemic.

--- \href{https://www.nytimes3xbfgragh.onion/by/nick-corasaniti}{Nick
Corasaniti}

\hypertarget{advertisement-6}{%
\subsubsection{Advertisement}\label{advertisement-6}}

\protect\hyperlink{after-dfp-ad-mid7}{Continue reading the main story}

\hypertarget{biden-called-for-an-inquiry-into-the-police-shooting-of-jacob-blake}{%
\subsection{\texorpdfstring{\protect\hyperlink{biden-called-for-an-inquiry-into-the-police-shooting-of-jacob-blake}{Biden
called for an inquiry into the police shooting of Jacob
Blake.}}{Biden called for an inquiry into the police shooting of Jacob Blake.}}\label{biden-called-for-an-inquiry-into-the-police-shooting-of-jacob-blake}}

\includegraphics{https://static01.graylady3jvrrxbe.onion/images/2020/08/24/us/politics/24-live-kenosha/merlin_176127846_7a305a57-9efb-4548-910f-3566c8f25280-articleLarge.jpg?quality=75\&auto=webp\&disable=upscale}

Joseph R. Biden Jr. on Monday called for an immediate investigation into
\href{https://www.nytimes3xbfgragh.onion/2020/08/24/us/kenosha-police-shooting.html}{the
shooting of a Black man}, Jacob Blake, by the police in Kenosha, Wis.,
saying that ``these shots pierce the soul of our nation.''

``This morning, the nation wakes up yet again with grief and outrage
that yet another Black American is a victim of excessive force,'' Mr.
Biden said in a statement. ``This calls for an immediate, full and
transparent investigation and the officers must be held accountable.''

Mr. Biden added that the country ``must dismantle systemic racism,''
saying that ``equal justice has not been real for Black Americans and so
many others.''

Policing has emerged as an issue in the presidential campaign,
particularly after the killing of
\href{https://www.nytimes3xbfgragh.onion/article/george-floyd-who-is.html}{George
Floyd} in police custody in May
\href{https://www.nytimes3xbfgragh.onion/news-event/george-floyd-protests-minneapolis-new-york-los-angeles}{incited
protests around the nation}.

As Republicans gathered in Charlotte, N.C., for their convention on
Monday, Vice President Mike Pence told the delegates that ``four more
years means more support for our troops and our cops.'' And President
Trump's pledge to the delegates that ``we are going to fully fund law
enforcement and hire more police'' was greeted with chants and cheers.

--- \href{https://www.nytimes3xbfgragh.onion/by/thomas-kaplan}{Thomas
Kaplan} and
\href{https://www.nytimes3xbfgragh.onion/by/michael-cooper}{Michael
Cooper}

\hypertarget{mike-pompeos-convention-speech-appears-to-violate-rules-he-sent-to-state-department-employees}{%
\subsection{\texorpdfstring{\protect\hyperlink{mike-pompeos-convention-speech-appears-to-violate-rules-he-sent-to-state-department-employees}{Mike
Pompeo's convention speech appears to violate rules he sent to State
Department
employees.}}{Mike Pompeo's convention speech appears to violate rules he sent to State Department employees.}}\label{mike-pompeos-convention-speech-appears-to-violate-rules-he-sent-to-state-department-employees}}

\includegraphics{https://static01.graylady3jvrrxbe.onion/images/2020/08/24/us/politics/24-live-astor-1843/merlin_175941342_bac33849-5088-4f57-b028-3679aa098dee-articleLarge.jpg?quality=75\&auto=webp\&disable=upscale}

Secretary of State Mike Pompeo, who is scheduled to speak at the
Republican convention on Tuesday, instructed State Department employees
last month not to participate in political activities, and his own plans
appear to violate department regulations.

Employees ``may not engage in partisan political activity'' even outside
of work hours, Mr. Pompeo wrote in an internal cable on July 24.

``Similarly,'' he added, ``presidential and political appointees'' ---
of which he is one --- ``are subject to significant restrictions on
their political activity; they may not engage in any partisan political
activity in concert with a partisan campaign, political party, or
partisan political group, even on personal time and outside of the
federal workplace.''

According to State Department guidance from December 2019, department
employees are not allowed to ``speak for or against a partisan
candidate, political party, or partisan political group at a convention,
rally, or similar gathering sponsored by such entities.''

A State Department official said that Mr. Pompeo would ``address the
convention in his personal capacity'' and added: ``No State Department
resources will be used. Staff are not involved in preparing the remarks
or in the arrangements for Secretary Pompeo's appearance. The State
Department will not bear any costs in conjunction with this
appearance.''

But the official guidelines and Mr. Pompeo's cable state clearly that
such partisan activities are prohibited even on employees' personal
time.

The guidance also says that ``Senate-confirmed presidential appointees
\emph{may not even attend} a political party convention or
convention-related event.'' Mr. Pompeo is a Senate-confirmed
presidential appointee.

Mr. Pompeo will not be physically present at the convention, which is
being held in Charlotte, N.C., but speaking for Mr. Trump would violate
the December 2019 guidance. State Department representatives did not
respond to multiple requests for comment about whether those rules were
current.

--- \href{https://www.nytimes3xbfgragh.onion/by/maggie-astor}{Maggie
Astor} and \href{https://www.nytimes3xbfgragh.onion/by/lara-jakes}{Lara
Jakes}

\hypertarget{trumps-penchant-for-falsehoods-makes-covering-a-live-convention-a-unique-challenge-for-tv-broadcasters}{%
\subsection{\texorpdfstring{\protect\hyperlink{trumps-penchant-for-falsehoods-makes-covering-a-live-convention-a-unique-challenge-for-tv-broadcasters}{Trump's
penchant for falsehoods makes covering a live convention a unique
challenge for TV
broadcasters.}}{Trump's penchant for falsehoods makes covering a live convention a unique challenge for TV broadcasters.}}\label{trumps-penchant-for-falsehoods-makes-covering-a-live-convention-a-unique-challenge-for-tv-broadcasters}}

\includegraphics{https://static01.graylady3jvrrxbe.onion/images/2020/08/24/us/politics/24-live-trumptv-1847/merlin_176119890_ae62ab18-45ad-46b0-844f-59ab28570ebf-articleLarge.jpg?quality=75\&auto=webp\&disable=upscale}

Television's ability to handle a Trump-centric convention faced an early
test on Monday, when the president delivered a kickoff speech that was
filled with false claims about the integrity of mail-in voting and the
policy positions of his Democratic opponent, Joseph R. Biden Jr.

Broadcasters were already bracing for a week of tough editorial
decisions. Political conventions, at heart, are forms of propaganda, and
TV networks typically give a long leash to candidates as they formally
make their case to the nation.

But President Trump's
\href{https://www.nytimes3xbfgragh.onion/2018/12/29/us/politics/trump-fact-check.html}{well-documented}
\href{https://www.nytimes3xbfgragh.onion/2020/06/03/us/politics/trump-twitter-fact-check.html}{penchant}
\href{https://www.nytimes3xbfgragh.onion/2020/06/24/us/politics/trump-vote-by-mail.html}{for}
\href{https://www.nytimes3xbfgragh.onion/2020/07/14/us/politics/trump-fact-check-biden-police-coronavirus-china.html}{falsehoods}
presents a unique challenge, according to network executives.

Producers are trying to balance voters' right to hear directly from
their president while preventing misinformation from spreading unchecked
on their channels to millions of viewers. The president's allies say
that Mr. Trump deserves the same deference as past presidents, and any
intervention on the networks' part is a sign of editorial bias.

As the president spoke on Monday, a hodgepodge of journalistic
strategies emerged.

CNN cut away from Mr. Trump in the middle of his remarks. MSNBC carried
the entirety of Mr. Trump's speech live, opting for real-time analysis
in on-screen graphics. Fox News carried the speech live, but did not
offer a correction to Mr. Trump's false claims.

In general, TV producers say they are inclined to air Mr. Trump's
remarks live, with clarifications and corrections offered as necessary.

``There are certain speeches in the political life of the country that
the news networks treat as events the audience deserves to see: the
State of the Union, an inaugural address, and convention speeches by the
nominee and the running mate,'' said Mark Lukasiewicz, who was an
executive producer for coverage of six conventions at NBC News.

``These are singular events,'' Mr. Lukasiewicz added. ``But the networks
are going to struggle. How do you maintain an appearance of fairness and
equity between the two parties' political events, but deal with the fact
that one candidate, you have every reason to believe, will not tell the
truth?''

---
\href{https://www.nytimes3xbfgragh.onion/by/michael-m-grynbaum}{Michael
M. Grynbaum}

\hypertarget{advertisement-7}{%
\subsubsection{Advertisement}\label{advertisement-7}}

\protect\hyperlink{after-dfp-ad-mid8}{Continue reading the main story}

\hypertarget{in-lieu-of-a-party-platform-republicans-express-strong-support-of-trump}{%
\subsection{\texorpdfstring{\protect\hyperlink{in-lieu-of-a-party-platform-republicans-express-strong-support-of-trump}{In
lieu of a party platform, Republicans express `strong support' of
Trump.}}{In lieu of a party platform, Republicans express `strong support' of Trump.}}\label{in-lieu-of-a-party-platform-republicans-express-strong-support-of-trump}}

\includegraphics{https://static01.graylady3jvrrxbe.onion/images/2020/08/24/us/politics/24-live-rncnoplatform/merlin_176113404_3419297b-0042-4139-8fa6-ab9b3cbefca9-articleLarge.jpg?quality=75\&auto=webp\&disable=upscale}

The extent to which President Trump has bent the Republican Party to his
will was underscored this week when the party announced that it would
not adopt a new platform this year, but would ``continue to
enthusiastically support the president's America-first agenda.''

The decision not to adopt a new Republican Party platform, the party's
main statement of policy, was extraordinary. The
\href{https://prod-cdn-static.gop.com/docs/Resolution_Platform_2020.pdf?_ga=2.165306300.2055661719.1598124638-455285808.1584478680}{resolution
that the Republican National Committee passed over the weekend} forgoing
a new one anticipated criticism, claiming that the ``media has
outrageously misrepresented the implications'' of not adopting a new
platform and calling on the media to accurately report the party's
``strong support'' for the president.

Criticism came swiftly. William Kristol, a former chief of staff to Vice
President Dan Quayle who went on to serve as the editor of The Weekly
Standard, a conservative magazine, and who has emerged as one of Mr.
Trump's most prominent Republican critics, wrote on Twitter: ``It's no
longer the Republican party. It's a Trump cult.''

\begin{quote}
The Republicans, in 2020, for the first time, have no platform. Instead:
"RESOLVED, That the Republican Party has and will continue to
enthusiastically support the President's America-first agenda." It's no
longer the Republican party. It's a Trump
cult.\url{https://t.co/BATeUiXRYu}

--- Bill Kristol (@BillKristol)
\href{https://twitter.com/BillKristol/status/1297656400022208512?ref_src=twsrc\%5Etfw}{August
23, 2020}
\end{quote}

Party platforms are nonbinding documents that tend to
\href{https://www.nytimes3xbfgragh.onion/2012/08/29/us/politics/republican-platform-takes-turn-to-right.html}{lay
out policy positions and principles}. A new Republican Party platform
would have been instructive at a moment when Mr. Trump has broken with
party orthodoxy on a host of issues, including
\href{https://www.nytimes3xbfgragh.onion/2019/08/21/us/politics/deficit-will-reach-1-trillion-next-year-budget-office-predicts.html}{his
opposition to free trade agreements}; a foreign policy that has
attempted to
\href{https://www.nytimes3xbfgragh.onion/2018/07/16/us/politics/republicans-trump-putin.html}{forge
closer ties with Russia} even as he
\href{https://www.nytimes3xbfgragh.onion/2020/06/06/world/europe/germany-troop-withdrawal-america.html}{has
antagonized longstanding European allies}; and a fiscal policy under
which
\href{https://www.nytimes3xbfgragh.onion/2019/08/21/us/politics/deficit-will-reach-1-trillion-next-year-budget-office-predicts.html}{deficits
were rising even before the pandemic forced more federal spending}.

The Republican National Committee said that it was forgoing a new
platform because fewer people were attending the convention this year
because of coronavirus restrictions, and it ``did not want a small
contingent of delegates formulating a new platform without the breadth
of perspectives within the ever-growing Republican movement.'' The
Democrats, who held their convention remotely,
\href{https://www.nytimes3xbfgragh.onion/2020/08/19/us/elections/the-democrats-approved-a-platform-setting-out-the-partys-agenda.html}{nonetheless
adopted a new platform last week}.

On Sunday, Mr. Trump released a list of broad statements about his
agenda for a second term, under the heading ``President Trump: Fighting
for You!'' They included promises of millions of new jobs, a vow to
``hold China fully accountable for allowing the virus to spread around
the world'' and a ``return to normal in 2021.''

--- \href{https://www.nytimes3xbfgragh.onion/by/michael-cooper}{Michael
Cooper}

\hypertarget{democrats-are-using-covid-to-steal-the-election-trump-says-in-an-inflammatory-rnc-speech}{%
\subsection{\texorpdfstring{\protect\hyperlink{democrats-are-using-covid-to-steal-the-election-trump-says-in-an-inflammatory-rnc-speech}{Democrats
are `using Covid to steal the election,' Trump says in an inflammatory
R.N.C.
speech.}}{Democrats are `using Covid to steal the election,' Trump says in an inflammatory R.N.C. speech.}}\label{democrats-are-using-covid-to-steal-the-election-trump-says-in-an-inflammatory-rnc-speech}}

\includegraphics{https://static01.graylady3jvrrxbe.onion/images/2020/08/24/us/politics/24-live-trumprenominated2/24-live-trumprenominated2-articleLarge.jpg?quality=75\&auto=webp\&disable=upscale}

President Trump was nominated for a second term on Monday as the
Republican National Convention got underway in Charlotte, N.C., and he
used a surprise speech at the convention not to preview a second-term
agenda, but to cast doubt in advance on the November election and attack
mail-in voting, accusing Democrats of ``using Covid to steal the
election.''

Mr. Trump --- who took the stage as the crowd chanted ``Four more
years!'' --- began with a provocation.

``If you want to really drive them crazy, you say 12 more years,'' Mr.
Trump said.

Mr. Trump, who is seeking re-election amid a pandemic that his
administration has failed to contain, widespread economic pain and
racial unrest, used his speech to rally the party by focusing on the
strength of the stock market and attacking Democratic officials who
imposed coronavirus restrictions.

He repeated his unfounded allegations that President Barack Obama and
Vice President Joseph R. Biden Jr., his opponent in the coming election,
had spied on his campaign in 2016. ``We caught them doing really bad
things,'' he said. ``Let's see what happens. They're trying it again.''

Mr. Trump criticized Roy Cooper, the Democratic governor of North
Carolina, telling the crowd in Charlotte that Mr. Cooper and other
Democratic governors had enacted virus restrictions simply to hurt his
re-election chances and would lift them after Election Day.

``You have a governor who is in a total shutdown mood,'' he said. ``I
guarantee you on November 4, it will all open up.''

Though shutdowns caused by the pandemic have left millions of Americans
unemployed, and new rounds of relief have been held up in Washington,
Mr. Trump focused on his economic successes.

``We're just about ready to break the all-time stock market record,'' he
said.

Mr. Trump offered his remarks to a crowd that frequently broke into
applause, a dramatic contrast with last week's Democratic convention,
which was held largely remotely out of concerns that indoor gatherings
could spread the coronavirus. The Republicans have made their decision
to hold an in-person convention a political statement in itself.

With tens of millions of Americans expected to vote by mail in order to
avoid contracting the virus at polling places, the president continued
his monthslong assault on voting by mail and repeated unfounded
accusations that it was part of a Democratic plot to hand the election
to Mr. Biden.

``They're using Covid to steal the election,'' he said.

And he continued to try to paint Mr. Biden, an establishment figure in
politics for decades who has been running a centrist campaign, as
radical. He demanded that Mr. Biden put out a list of judges he would
appoint, as Mr. Trump did in 2016.

``He can't do it,'' he said. ``The radical left will demand he appoints
super-radical-left wild crazy justices going into the Supreme Court.''

If that happens, Mr. Trump said, ``Your American dream will be dead.''

While the Democrats at their convention last week made the death toll
from the pandemic ---
\href{https://www.nytimes3xbfgragh.onion/interactive/2020/us/coronavirus-us-cases.html?action=click\&pgtype=Article\&state=default\&module=styln-coronavirus\&variant=show\&region=TOP_BANNER\&context=storylines_menu}{now
past 175,000} --- a centerpiece of their case, and tried to lay the
blame for it at Mr. Trump's feet, the president mentioned the virus's
victims almost as an afterthought at the end of his rambling, nearly
hourlong speech.

``We will never forget the 175,000 people --- that will go up,'' he
said, adding the toll would have been millions more without travel bans
he implemented.

\begin{center}\rule{0.5\linewidth}{\linethickness}\end{center}

--- \href{https://www.nytimes3xbfgragh.onion/by/annie-karni}{Annie
Karni} and
\href{https://www.nytimes3xbfgragh.onion/by/michael-cooper}{Michael
Cooper}

\hypertarget{our-2020-election-guide}{%
\section{Our 2020 Election Guide}\label{our-2020-election-guide}}

Updated ~Sept. 9, 2020

\begin{center}\rule{0.5\linewidth}{\linethickness}\end{center}

\begin{itemize}
\item ~
  \hypertarget{the-latest}{%
  \subsection{The Latest}\label{the-latest}}

  \begin{itemize}
  \item
    Joe Biden heads today to Michigan, a battleground state where
    President Trump has resumed advertising ahead of a visit there on
    Thursday.
    \href{https://www.nytimes3xbfgragh.onion/live/2020/09/09/us/trump-vs-biden?action=click\&pgtype=Article\&state=default\&region=BELOW_MAIN_CONTENT\&context=storylines_guide}{Read
    live updates}.
  \end{itemize}
\item ~
  \hypertarget{how-to-win-270}{%
  \subsection{How to Win 270}\label{how-to-win-270}}

  \begin{itemize}
  \item
    Joe Biden and Donald Trump need 270 electoral votes to reach the
    White House. Try building
    \href{https://www.nytimes3xbfgragh.onion/interactive/2020/us/elections/election-states-biden-trump.html?action=click\&pgtype=Article\&state=default\&region=BELOW_MAIN_CONTENT\&context=storylines_guide}{your
    own coalition of battleground states}~to see potential outcomes.
  \end{itemize}
\item ~
  \hypertarget{voting-by-mail}{%
  \subsection{Voting by Mail}\label{voting-by-mail}}

  \begin{itemize}
  \item
    Will you have enough time to vote by mail in your state? Yes, but
    it's risky to procrastinate.
    \href{https://www.nytimes3xbfgragh.onion/interactive/2020/08/31/us/politics/vote-by-mail-deadlines.html?action=click\&pgtype=Article\&state=default\&region=BELOW_MAIN_CONTENT\&context=storylines_guide}{Check
    your state's deadline.}
  \item
    \href{https://www.nytimes3xbfgragh.onion/interactive/2020/us/elections/joe-biden.html?action=click\&pgtype=Article\&state=default\&region=BELOW_MAIN_CONTENT\&context=storylines_guide}{}

    \hypertarget{joe-biden}{%
    \section{Joe Biden}\label{joe-biden}}

    \hypertarget{democrat}{%
    \subsection{Democrat}\label{democrat}}

    \href{https://www.nytimes3xbfgragh.onion/interactive/2020/us/elections/donald-trump.html?action=click\&pgtype=Article\&state=default\&region=BELOW_MAIN_CONTENT\&context=storylines_guide}{}

    \hypertarget{donald-trump}{%
    \section{Donald Trump}\label{donald-trump}}

    \hypertarget{republican}{%
    \subsection{Republican}\label{republican}}
  \end{itemize}
\item
  \hypertarget{keep-up-with-our-coverage}{%
  \subsection{Keep Up With Our
  Coverage}\label{keep-up-with-our-coverage}}

  \begin{itemize}
  \item
    Get an
    \href{https://www.nytimes3xbfgragh.onion/newsletters/politics?action=click\&pgtype=Article\&state=default\&region=BELOW_MAIN_CONTENT\&context=storylines_guide}{email}~recapping
    the day's news
  \item
    Download our mobile app on
    \href{https://apps.apple.com/us/app/nytimes/id284862083?ls=1\&mat_click_id=5c79ae7455014fd1bd66b5610c05b8f2-20191112-16948\&referrer=mat_click_id\%3D5c79ae7455014fd1bd66b5610c05b8f2-20191112-16948\%26link_click_id\%3D722930677036718082}{iOS}~and
    \href{http://a.localytics.com/android?id=com.nytimes.android\&referrer=utm_source\%3Dother_nyt_mobile_web\%26utm_medium\%3DWeb\%2520page\%26utm_term\%3DGeneral\%2520Mobile\%2520Page\%26utm_campaign\%3DNYT\%2520Mobile\%2520General\%2520Page}{Android}~and
    turn on Breaking News and Politics alerts
  \end{itemize}
\end{itemize}

\hypertarget{site-index}{%
\subsection{Site Index}\label{site-index}}

\hypertarget{site-information-navigation}{%
\subsection{Site Information
Navigation}\label{site-information-navigation}}

\begin{itemize}
\tightlist
\item
  \href{https://help.nytimes3xbfgragh.onion/hc/en-us/articles/115014792127-Copyright-notice}{©~2020~The
  New York Times Company}
\end{itemize}

\begin{itemize}
\tightlist
\item
  \href{https://www.nytco.com/}{NYTCo}
\item
  \href{https://help.nytimes3xbfgragh.onion/hc/en-us/articles/115015385887-Contact-Us}{Contact
  Us}
\item
  \href{https://www.nytco.com/careers/}{Work with us}
\item
  \href{https://nytmediakit.com/}{Advertise}
\item
  \href{http://www.tbrandstudio.com/}{T Brand Studio}
\item
  \href{https://www.nytimes3xbfgragh.onion/privacy/cookie-policy\#how-do-i-manage-trackers}{Your
  Ad Choices}
\item
  \href{https://www.nytimes3xbfgragh.onion/privacy}{Privacy}
\item
  \href{https://help.nytimes3xbfgragh.onion/hc/en-us/articles/115014893428-Terms-of-service}{Terms
  of Service}
\item
  \href{https://help.nytimes3xbfgragh.onion/hc/en-us/articles/115014893968-Terms-of-sale}{Terms
  of Sale}
\item
  \href{https://spiderbites.nytimes3xbfgragh.onion}{Site Map}
\item
  \href{https://help.nytimes3xbfgragh.onion/hc/en-us}{Help}
\item
  \href{https://www.nytimes3xbfgragh.onion/subscription?campaignId=37WXW}{Subscriptions}
\end{itemize}
