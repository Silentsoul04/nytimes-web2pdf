Sections

SEARCH

\protect\hyperlink{site-content}{Skip to
content}\protect\hyperlink{site-index}{Skip to site index}

\href{https://myaccount.nytimes3xbfgragh.onion/auth/login?response_type=cookie\&client_id=vi}{}

\href{https://www.nytimes3xbfgragh.onion/section/todayspaper}{Today's
Paper}

\hypertarget{the-coronavirus-outbreak}{%
\subsubsection{\texorpdfstring{\href{https://www.nytimes3xbfgragh.onion/news-event/coronavirus?name=styln-coronavirus-markets\&region=TOP_BANNER\&block=storyline_menu_recirc\&action=click\&pgtype=LegacyCollection\&impression_id=ccf39550-f4cd-11ea-8a20-6328acb0d0c9\&variant=undefined}{The
Coronavirus
Outbreak}}{The Coronavirus Outbreak}}\label{the-coronavirus-outbreak}}

\begin{itemize}
\tightlist
\item
  live\href{https://www.nytimes3xbfgragh.onion/2020/09/11/world/covid-19-coronavirus.html?name=styln-coronavirus-markets\&region=TOP_BANNER\&block=storyline_menu_recirc\&action=click\&pgtype=LegacyCollection\&impression_id=ccf39551-f4cd-11ea-8a20-6328acb0d0c9\&variant=undefined}{Latest
  Updates}
\item
  \href{https://www.nytimes3xbfgragh.onion/interactive/2020/us/coronavirus-us-cases.html?name=styln-coronavirus-markets\&region=TOP_BANNER\&block=storyline_menu_recirc\&action=click\&pgtype=LegacyCollection\&impression_id=ccf3bc60-f4cd-11ea-8a20-6328acb0d0c9\&variant=undefined}{Maps
  and Cases}
\item
  \href{https://www.nytimes3xbfgragh.onion/interactive/2020/science/coronavirus-vaccine-tracker.html?name=styln-coronavirus-markets\&region=TOP_BANNER\&block=storyline_menu_recirc\&action=click\&pgtype=LegacyCollection\&impression_id=ccf3bc61-f4cd-11ea-8a20-6328acb0d0c9\&variant=undefined}{Vaccine
  Tracker}
\item
  \href{https://www.nytimes3xbfgragh.onion/2020/09/10/us/politics/fda-coronavirus-vaccine.html?name=styln-coronavirus-markets\&region=TOP_BANNER\&block=storyline_menu_recirc\&action=click\&pgtype=LegacyCollection\&impression_id=ccf3bc62-f4cd-11ea-8a20-6328acb0d0c9\&variant=undefined}{F.D.A.
  Regulators' Self-Defense}
\item
  \href{https://www.nytimes3xbfgragh.onion/2020/09/09/upshot/coronavirus-surprise-test-fees.html?name=styln-coronavirus-markets\&region=TOP_BANNER\&block=storyline_menu_recirc\&action=click\&pgtype=LegacyCollection\&impression_id=ccf3bc63-f4cd-11ea-8a20-6328acb0d0c9\&variant=undefined}{Surprise
  Test Fees}
\end{itemize}

\hypertarget{airlines-drop-most-change-fees-permanently}{%
\section{Airlines Drop Most Change Fees
Permanently}\label{airlines-drop-most-change-fees-permanently}}

Last Updated

Sept. 1, 2020, 4:26 a.m. ET

Sept. 1, 2020, 4:26 a.m. ET

This briefing is no longer being updated.

\hypertarget{heres-what-you-need-to-know}{%
\subsubsection{Here's what you need to
know:}\label{heres-what-you-need-to-know}}

\begin{itemize}
\item
  \protect\hyperlink{delta-and-american-follow-united-in-permanently-dropping-some-change-fees}{}

  Delta and American follow United in permanently dropping some change
  fees.
\item
  \protect\hyperlink{the-sp-ended-its-run-of-gains-but-had-its-second-best-month-this-year}{}

  The S\&P ended its run of gains but had its second-best month this
  year.
\item
  \protect\hyperlink{why-the-dow-revamped-its-lineup-of-stocks}{}

  Why the Dow revamped its lineup of stocks.
\item
  \protect\hyperlink{zooms-profit-jumped-3276-percent-in-the-2nd-quarter}{}

  Zoom's profit jumped 3,276 percent in the 2nd quarter.
\item
  \protect\hyperlink{jc-penney-hits-stalemate-in-buyer-talks}{}

  J.C. Penney hits stalemate in buyer talks.
\item
  \protect\hyperlink{indias-24-percent-decline-in-gdp-last-quarter-was-the-worst-of-any-major-economy}{}

  India's 24 percent decline in G.D.P. last quarter was the worst of any
  major economy.
\end{itemize}

\hypertarget{delta-and-american-follow-united-in-permanently-dropping-some-change-fees}{%
\subsection{\texorpdfstring{\protect\hyperlink{delta-and-american-follow-united-in-permanently-dropping-some-change-fees}{Delta
and American follow United in permanently dropping some change
fees.}}{Delta and American follow United in permanently dropping some change fees.}}\label{delta-and-american-follow-united-in-permanently-dropping-some-change-fees}}

\includegraphics{https://static01.graylady3jvrrxbe.onion/images/2020/08/31/business/31markets-brf-united/merlin_170119626_2e77d909-77ea-4cd6-95e5-bff5494e8a0a-articleLarge.jpg?quality=75\&auto=webp\&disable=upscale}

For months, airlines have been waiving change fees to encourage hesitant
travelers to fly again. Now, they're doing away with the charges
altogether, at least for flights within the United States.

On Sunday, \textbf{United Airlines} said it was permanently dropping
change fees for most customers flying domestically. \textbf{American
Airlines} and \textbf{Delta Air Lines} followed suit a day later. The
changes, effective immediately, apply to all standard economy and
premium seats, but not to any of the airlines' low-price basic economy
seats, which come with additional restrictions.

``When we hear from customers about where we can improve, getting rid of
this fee is often the top request,'' United's chief executive, Scott
Kirby, said in a recorded message to customers.

American said it would also drop change fees for short-distance
international trips to Canada, Mexico, the Caribbean, and the U.S.
Virgin Islands. \textbf{Southwest Airlines} does not charge change fees.

American, Delta and United all said that they would continue to waive
change fees through at least the end of the year for international
travel and for passengers holding basic economy tickets.

Starting in 2021, every United customer will be allowed to fly standby
for free on an earlier flight on their scheduled day of travel if a seat
is available, the airline said. American said it would do the same,
starting Oct. 1.

The announcements come as airlines prepare themselves for a recovery
that is expected to take years to unfold. Air travel is currently down
about 70 percent compared to last year, according to federal data, and
carriers have been doing all that they can to stand out from one another
and attract what few customers remain. Delta and Southwest, for example,
have been limiting seating on flights, while all of the major airlines
have waived change fees, touted cleaning regimens and imposed stringent
\href{https://www.nytimes3xbfgragh.onion/live/2020/08/27/business/stock-market-today-coronavirus/delta-has-barred-about-240-people-from-flying-for-not-wearing-masks}{mask
requirements}.

--- \href{https://www.nytimes3xbfgragh.onion/by/niraj-chokshi}{Niraj
Chokshi}

\hypertarget{the-sp-ended-its-run-of-gains-but-had-its-second-best-month-this-year}{%
\subsection{\texorpdfstring{\protect\hyperlink{the-sp-ended-its-run-of-gains-but-had-its-second-best-month-this-year}{The
S\&P ended its run of gains but had its second-best month this
year.}}{The S\&P ended its run of gains but had its second-best month this year.}}\label{the-sp-ended-its-run-of-gains-but-had-its-second-best-month-this-year}}

\begin{itemize}
\item
  The S\&P 500 fell 0.22 percent in the final trading day of August,
  ending a seven-day run of gains into record territory. Even so, the
  S\&P had its second-best month of the year, and its fourth-best month
  in five years, notching a gain of 7.01 percent.
\item
  Investors expressed some relief at the slowdown of the spread of the
  virus in the United States, which remains an epicenter of the
  pandemic. Stocks sensitive to the outlook for the virus, such as
  \textbf{Royal Caribbean Cruises} and the hotel and casino company
  \textbf{MGM Resorts International}, posted some of the biggest gains
  of the month. But technology giants and chip makers were the largest
  contributors to the rally.
\item
  On Monday, shares of \textbf{Tesla} rose 12.57 percent and
  \textbf{Apple} rose 3.39 percent as their
  \href{https://www.nytimes3xbfgragh.onion/reuters/2020/08/28/business/apple-tesla-stocksplit.html}{stock
  splits} took effect. The stock splits will make it less expensive to
  own individual stocks in the companies, putting them in reach of
  retail investors. Tesla, which announced a 5-for-1 split earlier this
  month, closed at \$2,213.40 on Friday. Apple, which is doing a 4-for-1
  split, closed Friday at \$499.23.
\item
  \textbf{European markets rose on Monday}, with indexes in Germany and
  France up about 1 percent. The markets in Britain were closed for a
  public holiday.
\item
  Asian markets were mixed on Monday, with \textbf{Japan's Nikkei ending
  the day up more than 1 percent}, while other benchmark indexes were in
  negative territory. Japanese stocks jumped after \textbf{Berkshire
  Hathaway}, which is owned by the legendary investor Warren Buffett,
  \href{https://www.wsj.com/articles/berkshire-hathaway-buys-stakes-in-five-japanese-trading-companies-11598841219?mod=hp_lead_pos3}{bought
  stakes in five of Japan's biggest trading companies}, including
  \textbf{Mitsubishi} and \textbf{Mitsui}.
\item
  Investors were encouraged by the Federal Reserve's move last week to
  \textbf{\href{https://www.nytimes3xbfgragh.onion/2020/08/27/business/economy/federal-reserve-inflation-jerome-powell.html}{relax
  its approach to inflation}}. The announcement by the Fed chair, Jerome
  H. Powell, at the annual Jackson Hole summit signaled that the central
  bank would keep interest rates low while focusing on fostering a
  strong labor market.
\item
  But \textbf{coronavirus cases around the world continued to increase},
  with the United States on Sunday surpassing
  \href{https://www.nytimes3xbfgragh.onion/2020/08/30/world/coronavirus-covid.html}{six
  million confirmed infections}, almost a quarter of the 25 million
  cases recorded globally.
\end{itemize}

--- \href{https://www.nytimes3xbfgragh.onion/by/matt-phillips}{Matt
Phillips}

\hypertarget{advertisement}{%
\subsubsection{Advertisement}\label{advertisement}}

\protect\hyperlink{after-dfp-ad-mid1}{Continue reading the main story}

\hypertarget{why-the-dow-revamped-its-lineup-of-stocks}{%
\subsection{\texorpdfstring{\protect\hyperlink{why-the-dow-revamped-its-lineup-of-stocks}{Why
the Dow revamped its lineup of
stocks.}}{Why the Dow revamped its lineup of stocks.}}\label{why-the-dow-revamped-its-lineup-of-stocks}}

\includegraphics{https://static01.graylady3jvrrxbe.onion/images/2020/08/31/business/31markets-brf-matt/merlin_175594770_da32f36e-65bf-4363-9a1e-5f3dfd48bc44-articleLarge.jpg?quality=75\&auto=webp\&disable=upscale}

It's a new day for the venerable \textbf{Dow Jones} industrial average.

Before the open of trading on Monday, the lineup of the index was
rejiggered, with \textbf{Exxon Mobil}, \textbf{Raytheon} and
\textbf{Pfizer} jettisoned, and \textbf{Amgen}, \textbf{Salesforce.com}
and \textbf{Honeywell} added to the 30-stock menu.

Why? Blame \textbf{Apple}.

Apple on Monday executed a 4-for-1 stock split, which essentially
chopped one share of Apple, which had been trading at more than \$500,
into four shares priced at a bit more than \$125. (Companies sometimes
split shares if their prices are getting high as a way to ensure that
they can be easily traded.)

But the stock split was a problem for the Dow, an index that weights
stocks depending on their share price.

The highest priced shares have the largest influence on the movement of
the index. And at more than \$500 a share, Apple was the most
influential stock in the Dow, accounting for roughly 12 percent of the
index. Apple is up more than 70 percent this year and was responsible
for a substantial part of the Dow's performance.

Apple's size in the Dow also puffed up the index's exposure to the
high-flying large-cap technology sector, which has soared this year and
has helped pull the overall markets up, despite the global coronavirus
pandemic.

After the split, Apple accounted for about 3 percent of the Dow, so the
stock's romp higher won't move the overall index as much as it used to.

It also means that the index's weight toward the all important
technology sector has also fallen.

So much of the market's gains this year are because of technology
stocks, and the Dow is already lagging behind other indexes, which have
heavier weighting to that sector.

For example, the S\&P 500 --- which sees companies with the largest
stock market valuation as the most influential --- has outperformed the
Dow significantly this year, thanks largely to the weighting the index
gives to giant tech firms like \textbf{Apple}, \textbf{Microsoft} and
\textbf{Amazon}, which have massive market capitalizations.

To lift the Dow's technology weighting after the Apple split, the
committee that controls membership in the index --- it is majority owned
by the financial information and credit rating giant S\&P Global ---
added Salesforce.com.

They also removed the pharmaceutical giant Pfizer, which was one of the
lowest-priced members of the Dow, adding Amgen, a biotech firm, to make
sure the health care sector was reflected in the index.

Raytheon was replaced with Honeywell International to increase the Dow's
weighting toward industrials, which had shrunk with Raytheon's lagging
share price.

And Exxon, which was the longest serving member of the Dow 30, having
joined as Standard Oil of New Jersey in 1928, was removed in an apparent
nod to the shrinking role of the energy sector of the market.
\textbf{Chevron} is now the last energy stock in the index.

None of these changes mean the actual level of the Dow Jones industrial
average will change, as their prices are adjusted using a formula that
ensures there's no sudden break with past trends of the index.

--- \href{https://www.nytimes3xbfgragh.onion/by/matt-phillips}{Matt
Phillips}

\hypertarget{zooms-profit-jumped-3276-percent-in-the-2nd-quarter}{%
\subsection{\texorpdfstring{\protect\hyperlink{zooms-profit-jumped-3276-percent-in-the-2nd-quarter}{Zoom's
profit jumped 3,276 percent in the 2nd
quarter.}}{Zoom's profit jumped 3,276 percent in the 2nd quarter.}}\label{zooms-profit-jumped-3276-percent-in-the-2nd-quarter}}

\textbf{Zoom's} revenue and profit continued to soar in the second
quarter as the company's videoconferencing software cemented its status
as a key tool for work, education and socialization since the start of
the coronavirus pandemic in March.

The San Jose, Calif., company beat even lofty industry expectations
Monday, reporting that it had \$663.5 million in revenue, up 355 percent
from the same period last year. Net income for the quarter leapt to
\$185.7 million, up from \$5.5 million a year ago. The company said it
now has more than 370,000 customers that have more than 10 employees, up
458 percent from a year ago.

Zoom, whose video-calling software has grown so ubiquitous that it has
become a
\href{https://www.nytimes3xbfgragh.onion/2020/03/17/style/zoom-parties-coronavirus-memes.html}{cultural}
\href{https://www.nytimes3xbfgragh.onion/2020/06/29/business/zoom-shirt.html}{phenomenon},
said that its rosy financial outlook was due in part to companies
shifting from a short-term reaction to the pandemic to long-term plans
for remote work.

--- \href{https://www.nytimes3xbfgragh.onion/by/kellen-browning}{Kellen
Browning}

\hypertarget{jc-penney-hits-stalemate-in-buyer-talks}{%
\subsection{\texorpdfstring{\protect\hyperlink{jc-penney-hits-stalemate-in-buyer-talks}{J.C.
Penney hits stalemate in buyer
talks.}}{J.C. Penney hits stalemate in buyer talks.}}\label{jc-penney-hits-stalemate-in-buyer-talks}}

\includegraphics{https://static01.graylady3jvrrxbe.onion/images/2020/08/31/business/31markets-brf-jcpenney/merlin_172118085_e802c17a-5393-4490-bfde-762cdf8cfa46-articleLarge.jpg?quality=75\&auto=webp\&disable=upscale}

A lawyer for \textbf{J.C. Penney} told a bankruptcy judge in Texas on
Monday that the retailer had hit a stalemate in its talks with buyers.
Now, to avoid liquidation, the company will focus on a sale to its top
lenders in a deal that would convert its debt to equity stakes. J.C.
Penney has set Sept. 10 as a new deadline to reach an agreement --- or
else liquidation becomes increasingly likely.

Hanging in the balance of the negotiations is the future of hundreds of
stores and tens of thousands of jobs. The 118-year-old department store
filed for bankruptcy in May with more than 800 stores and nearly 85,000
employees. It has since announced layoffs and store closures.

J.C. Penney had been talking to a consortium of \textbf{Brookfield
Property Partners} and \textbf{Simon Property Group} about a sale that
could have saved the retailer. Despite having worked through the
weekend, J.C. Penney's bankruptcy attorney, Joshua Sussberg, said Monday
that the company was unable to come to an agreement on a sale.

--- Lauren Hirsch

\hypertarget{advertisement-1}{%
\subsubsection{Advertisement}\label{advertisement-1}}

\protect\hyperlink{after-dfp-ad-mid2}{Continue reading the main story}

\hypertarget{indias-24-percent-decline-in-gdp-last-quarter-was-the-worst-of-any-major-economy}{%
\subsection{\texorpdfstring{\protect\hyperlink{indias-24-percent-decline-in-gdp-last-quarter-was-the-worst-of-any-major-economy}{India's
24 percent decline in G.D.P. last quarter was the worst of any major
economy.}}{India's 24 percent decline in G.D.P. last quarter was the worst of any major economy.}}\label{indias-24-percent-decline-in-gdp-last-quarter-was-the-worst-of-any-major-economy}}

\includegraphics{https://static01.graylady3jvrrxbe.onion/images/2020/08/31/business/31india-econ-covid-copyForBriefing/merlin_176023563_ef73700b-2c7f-4746-9088-f44fd007a58e-articleLarge.jpg?quality=75\&auto=webp\&disable=upscale}

The Indian economy contracted by 23.9 percent in the second quarter, the
worst decline among the world's top economies.

Data released by the Indian government on Monday showed that consumer
spending, private investment and exports had all suffered tremendously.
The sector including trade, hotel and transport dipped 47 percent.
India's once mighty manufacturing industry shrank 39 percent. The
figures reflect the onset of India's deepest recession since 1996, when
the country first began publishing its G.D.P. numbers.

The only bright spot, though relatively faint, was agriculture. Thanks
to strong rains this monsoon season, the sector grew 3.4 percent versus
3 percent in the previous quarter.

India's picture is further complicated by the fact that so many people
here are ``informally'' employed, working in jobs that are not covered
by contracts and often fall beyond government reach, such as rickshaw
driver, tailor, day laborer and farmhand. Economists say that official
numbers are bound to underestimate that part of the economy and that the
full damage could be even greater.

``My estimate is after the government takes the unorganized sector into
account,'' the overall economic slide will be ``minus 40 percent,'' said
Arun Kumar, a professor at New Delhi's Institute of Social Sciences.

Economists said that the surging coronavirus cases in the country might
push recovery further away and that the central bank would increasingly
come under pressure for additional stimulus payments and rate cuts.

The U.S. economy shrank 9.5 percent last quarter, and Japan's economy
shrank 7.6 percent.

--- \href{https://www.nytimes3xbfgragh.onion/by/sameer-yasir}{Sameer
Yasir} and
\href{https://www.nytimes3xbfgragh.onion/by/jeffrey-gettleman}{Jeffrey
Gettleman}

\hypertarget{chinas-exports-are-surging-despite-tariffs-and-the-pandemic}{%
\subsection{\texorpdfstring{\protect\hyperlink{chinas-exports-are-surging-despite-tariffs-and-the-pandemic}{China's
exports are surging, despite tariffs and the
pandemic.}}{China's exports are surging, despite tariffs and the pandemic.}}\label{chinas-exports-are-surging-despite-tariffs-and-the-pandemic}}

\includegraphics{https://static01.graylady3jvrrxbe.onion/images/2020/09/01/business/01China-Exports-2/merlin_175581288_07f5174c-b5d7-4c2c-bf34-049ffa250426-articleLarge.jpg?quality=75\&auto=webp\&disable=upscale}

This was supposed to be the year that China's export machine began to
stall. President Trump had imposed broad tariffs on Chinese goods.
Countries like Japan and France pushed companies to shift production
from China. The pandemic had crippled China's factories by the end of
January.

Instead, China Inc. has come roaring back.

After reopening in late February and early March, China's factories
began an export blitz that is still gaining steam. Exports soared in
July to their second-highest level ever, nearly matching the record
Christmas rush last December. The country has grabbed a much larger
share of global markets this summer from other manufacturing nations and
entrenched a dominance in trade that could last long after the world
begins to recover from the pandemic.

China is showing that its export machine cannot be stopped --- not by
the coronavirus and not by the Trump administration. Its resilience lies
not only in the country's low-cost skilled labor and efficient
infrastructure but also a state-controlled banking system that has been
offering small and large businesses extra loans to cope with the
pandemic.

The pandemic has also found China better placed than other exporting
nations. It is making what the world's hospitals and housebound families
need right now: personal protective gear, home improvement products and
lots of consumer electronics.

At the same time, demand has withered for many big-ticket items exported
by the United States and Europe, like Boeing and Airbus jets. And it has
also faltered for the commodities that most developing countries export,
particularly oil.

--- \href{https://www.nytimes3xbfgragh.onion/by/keith-bradsher}{Keith
Bradsher}

\hypertarget{beijing-complicates-tiktoks-deal-talks-with-new-export-restrictions}{%
\subsection{\texorpdfstring{\protect\hyperlink{beijing-complicates-tiktoks-deal-talks-with-new-export-restrictions}{Beijing
complicates TikTok's deal talks with new export
restrictions.}}{Beijing complicates TikTok's deal talks with new export restrictions.}}\label{beijing-complicates-tiktoks-deal-talks-with-new-export-restrictions}}

\includegraphics{https://static01.graylady3jvrrxbe.onion/images/2020/08/31/business/31markets-brf-tiktok/merlin_176102118_faa4daf9-f4fd-49e6-9c5b-184ed0fa8935-articleLarge.jpg?quality=75\&auto=webp\&disable=upscale}

The Chinese government over the weekend imposed
\href{https://www.nytimes3xbfgragh.onion/2020/08/29/technology/china-tiktok-export-controls.html}{new
restrictions on technology exports}, including what sound like the
algorithms that underpin \textbf{TikTok}, and the move has thrown a
wrench into negotiations to sell the video app to an American company.

The surprise move may be China's attempt to dictate terms of the sale,
which is happening under orders from President Trump.
\textbf{ByteDance}, TikTok's China-based parent company, has said that
it will comply with the new rules.

Or the new rules could be an effort to block the sale. China effectively
killed \textbf{Qualcomm's} 2018 bid to buy the Dutch chip maker
\textbf{NXP}
\href{https://www.nytimes3xbfgragh.onion/2018/07/25/technology/qualcomm-nxp-china-deadline.html}{by
withholding approval}. If Beijing blocks the sale of TikTok, it
effectively would be calling the Trump administration's bluff, daring it
to shut the app down.

People briefed on the talks have warned that Beijing's approval was
always important, and appeasing both Mr. Trump and Chinese officials was
a top priority for TikTok's main suitors, \textbf{Microsoft} and
\textbf{Oracle}. (Given their extensive business interests in China, the
buyers now have to tread even more carefully.) A deal, which had been
expected to be announced
\href{https://www.cnbc.com/2020/08/31/tiktok-deal-to-sell-us-business-could-be-announced-as-soon-as-tomorrow.html}{as
soon as this week}, may be delayed by the new rules.

---
\href{https://www.nytimes3xbfgragh.onion/by/michael-j-de-la-merced}{Michael
J. de la Merced}

\hypertarget{advertisement-2}{%
\subsubsection{Advertisement}\label{advertisement-2}}

\protect\hyperlink{after-dfp-ad-mid1}{Continue reading the main story}

\hypertarget{a-top-fed-official-says-low-unemployment-alone-wont-trigger-higher-rates}{%
\subsection{\texorpdfstring{\protect\hyperlink{a-top-fed-official-says-low-unemployment-alone-wont-trigger-higher-rates}{A
top Fed official says low unemployment alone won't trigger higher
rates.}}{A top Fed official says low unemployment alone won't trigger higher rates.}}\label{a-top-fed-official-says-low-unemployment-alone-wont-trigger-higher-rates}}

\includegraphics{https://static01.graylady3jvrrxbe.onion/images/2020/09/21/world/21markets-briefing-clarida/merlin_161538147_3984bfcf-2edc-4197-a8d4-e7af5cbe3f8a-articleLarge.jpg?quality=75\&auto=webp\&disable=upscale}

Richard H. Clarida, the Federal Reserve vice chair, said in a speech on
Monday that the central bank was in the middle of a ``robust evolution''
and that it would no longer raise interest rates to cool off economic
growth based solely on the level of the unemployment rate.

Low joblessness by itself ``will not, under our new framework, be a
sufficient trigger for policy action,'' Mr. Clarida said, noting that
other considerations --- like financial stability concerns or evidence
that inflation is likely to run hot --- could still prompt rate
increases.

``This is a robust evolution in the Federal Reserve's policy
framework,'' he said, adding that economic models that guess the labor
market's limits ``can be and have been wrong.''

The Fed chair, Jerome H. Powell,
\href{https://www.nytimes3xbfgragh.onion/2020/08/27/business/economy/federal-reserve-inflation-jerome-powell.html}{announced
last week} that he and his colleagues were ending a year-and-a-half-long
review of their monetary policy strategy, and that the Fed updated a
guiding document that sets out its long-run policy framework. Among the
most important changes, the Fed said that rather than worrying about
``deviations'' from full employment, the central bank would now be
concerned about ``shortfalls.''

While that might seem like semantics, it indicated a significant shift:
The Fed is formally ditching its long-held practice of lifting borrowing
costs to cool down the economy and fend off future inflation when the
unemployment rate drops below a certain level, the one that economists
use to signify ``full employment.''

Officials have become increasingly modest about their ability to guess
where that line in the sand might be, after joblessness fell to 50-year
lows but inflation stagnated below the central bank's 2 percent goal.
The move sets the stage for long periods of very low interest rates, and
codifies a transition that has taken hold over the last two years.

The Fed also announced that it would pursue a strategy to hit 2 percent
inflation as an average over time. That's a change from the Fed's old
approach, which was always trying to return inflation to 2 percent.

Mr. Clarida didn't offer exact details in his prepared remarks about how
high above 2 percent the Fed would allow prices to run. He hinted that
the Fed would update
\href{https://www.federalreserve.gov/monetarypolicy/files/fomcprojtabl20200610.pdf}{its
economic forecasts} --- which regularly predict inflation stopping at
exactly 2 percent --- before the end of the year.

--- \href{https://www.nytimes3xbfgragh.onion/by/jeanna-smialek}{Jeanna
Smialek}

\hypertarget{heres-the-business-news-to-watch-in-the-week-ahead}{%
\subsection{\texorpdfstring{\protect\hyperlink{heres-the-business-news-to-watch-in-the-week-ahead}{Here's
the business news to watch in the week
ahead.}}{Here's the business news to watch in the week ahead.}}\label{heres-the-business-news-to-watch-in-the-week-ahead}}

💰 Wall Street is eager for \textbf{Zoom} to report earnings after the
market closes today. Last quarter will be
\href{https://www.marketwatch.com/story/what-can-zoom-do-for-a-sequel-to-one-of-the-most-astounding-earnings-blowouts-of-all-time-11598724433}{a
tough act to follow} for the videoconferencing company: One analyst
called it the ``the greatest quarter in enterprise software history.''
Investors also want to see how the pandemic is affecting
\textbf{Campbell Soup}, which reports earnings on Thursday.

🏛 In a series of speeches, Fed officials will explain the implications
of the central bank's
\href{https://www.nytimes3xbfgragh.onion/2020/08/27/business/economy/federal-reserve-inflation-jerome-powell.html}{momentous
announcement} last week that it will tolerate higher inflation to foster
a stronger labor market. \textbf{Richard Clarida}, the Fed's vice chair,
speaks
\href{https://www.piie.com/events/fed-vice-chair-richard-h-clarida-us-monetary-policy}{today};
\textbf{Lael Brainard}, a Fed governor, speaks on
\href{https://www.brookings.edu/events/how-the-fed-will-respond-to-the-covid-19-recession-in-an-era-of-low-rates-and-low-inflation/}{Tuesday}
(followed by a panel discussion featuring the former Fed chairs
\textbf{Ben Bernanke} and \textbf{Janet Yellen}); and the New York Fed
president \textbf{John Williams} speaks on
\href{https://www.brettonwoods.org/event/in-conversation-ny-fed-presidents-on-covid19}{Wednesday}.

📈 The biggest economic news is due on Friday, with the release of the
monthly \textbf{U.S. jobs report}. Economists expect that the U.S.
economy added 1.4 million jobs in August, and that the unemployment rate
dropped below 10 percent. Both would be big improvements, but far from
restoring all the jobs lost during the pandemic. Similar trends are
playing out in the \textbf{European Union}, which releases its latest
jobs numbers on Tuesday, and in \textbf{Canada}, which reports on
Friday.

--- \href{https://www.nytimes3xbfgragh.onion/by/jason-karaian}{Jason
Karaian}

\hypertarget{the-latest-ford-finishes-making-ventilators}{%
\subsection{\texorpdfstring{\protect\hyperlink{the-latest-ford-finishes-making-ventilators}{The
latest: Ford finishes making
ventilators.}}{The latest: Ford finishes making ventilators.}}\label{the-latest-ford-finishes-making-ventilators}}

\begin{itemize}
\item
  \textbf{Ford Motor} said Monday that it ended production of
  ventilators after delivering the 50,000 it had promised to make, in a
  partnership with \textbf{General Electric}, when the coronavirus
  pandemic took hold in the spring. Ford said it shipped the last
  ventilator to the Department of Health and Human Services on Aug. 28
  from its plant in Rawsonville, Mich. \textbf{General Motors,} which
  had been assembling ventilators with \textbf{Ventec Life Systems,} is
  also nearing the end of its production run.
\item
  Companies
  \href{https://www.nytimes3xbfgragh.onion/2020/08/28/us/politics/trump-tax-holiday-bill-due.html}{can
  stop withholding payroll taxes} from employees' paychecks beginning
  Sept. 1. But those employees would still have to pay the tax through
  larger withholdings --- and less take-home pay --- by April. That
  guidance, released by the Treasury Department in coordination with the
  Internal Revenue Service on Friday evening, said that ``the affected
  taxpayer may make arrangements to otherwise collect the total
  applicable taxes from the employee,'' suggesting companies can hold
  workers liable for the tax even if they leave the company.
\end{itemize}

\hypertarget{site-index}{%
\subsection{Site Index}\label{site-index}}

\hypertarget{site-information-navigation}{%
\subsection{Site Information
Navigation}\label{site-information-navigation}}

\begin{itemize}
\tightlist
\item
  \href{https://help.nytimes3xbfgragh.onion/hc/en-us/articles/115014792127-Copyright-notice}{©~2020~The
  New York Times Company}
\end{itemize}

\begin{itemize}
\tightlist
\item
  \href{https://www.nytco.com/}{NYTCo}
\item
  \href{https://help.nytimes3xbfgragh.onion/hc/en-us/articles/115015385887-Contact-Us}{Contact
  Us}
\item
  \href{https://www.nytco.com/careers/}{Work with us}
\item
  \href{https://nytmediakit.com/}{Advertise}
\item
  \href{http://www.tbrandstudio.com/}{T Brand Studio}
\item
  \href{https://www.nytimes3xbfgragh.onion/privacy/cookie-policy\#how-do-i-manage-trackers}{Your
  Ad Choices}
\item
  \href{https://www.nytimes3xbfgragh.onion/privacy}{Privacy}
\item
  \href{https://help.nytimes3xbfgragh.onion/hc/en-us/articles/115014893428-Terms-of-service}{Terms
  of Service}
\item
  \href{https://help.nytimes3xbfgragh.onion/hc/en-us/articles/115014893968-Terms-of-sale}{Terms
  of Sale}
\item
  \href{https://spiderbites.nytimes3xbfgragh.onion}{Site Map}
\item
  \href{https://help.nytimes3xbfgragh.onion/hc/en-us}{Help}
\item
  \href{https://www.nytimes3xbfgragh.onion/subscription?campaignId=37WXW}{Subscriptions}
\end{itemize}
