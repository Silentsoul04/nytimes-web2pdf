Sections

SEARCH

\protect\hyperlink{site-content}{Skip to
content}\protect\hyperlink{site-index}{Skip to site index}

\href{https://myaccount.nytimes3xbfgragh.onion/auth/login?response_type=cookie\&client_id=vi}{}

\href{https://www.nytimes3xbfgragh.onion/section/todayspaper}{Today's
Paper}

\hypertarget{the-coronavirus-outbreak}{%
\subsubsection{\texorpdfstring{\href{https://www.nytimes3xbfgragh.onion/news-event/coronavirus?name=styln-coronavirus-markets\&region=TOP_BANNER\&block=storyline_menu_recirc\&action=click\&pgtype=LegacyCollection\&impression_id=711d7830-f1db-11ea-9371-2fd28fa48a5f\&variant=undefined}{The
Coronavirus
Outbreak}}{The Coronavirus Outbreak}}\label{the-coronavirus-outbreak}}

\begin{itemize}
\tightlist
\item
  live\href{https://www.nytimes3xbfgragh.onion/2020/09/08/world/covid-19-coronavirus.html?name=styln-coronavirus-markets\&region=TOP_BANNER\&block=storyline_menu_recirc\&action=click\&pgtype=LegacyCollection\&impression_id=711d7831-f1db-11ea-9371-2fd28fa48a5f\&variant=undefined}{Latest
  Updates}
\item
  \href{https://www.nytimes3xbfgragh.onion/interactive/2020/us/coronavirus-us-cases.html?name=styln-coronavirus-markets\&region=TOP_BANNER\&block=storyline_menu_recirc\&action=click\&pgtype=LegacyCollection\&impression_id=711d7832-f1db-11ea-9371-2fd28fa48a5f\&variant=undefined}{Maps
  and Cases}
\item
  \href{https://www.nytimes3xbfgragh.onion/interactive/2020/science/coronavirus-vaccine-tracker.html?name=styln-coronavirus-markets\&region=TOP_BANNER\&block=storyline_menu_recirc\&action=click\&pgtype=LegacyCollection\&impression_id=711d9f40-f1db-11ea-9371-2fd28fa48a5f\&variant=undefined}{Vaccine
  Tracker}
\item
  \href{https://www.nytimes3xbfgragh.onion/2020/09/02/your-money/eviction-moratorium-covid.html?name=styln-coronavirus-markets\&region=TOP_BANNER\&block=storyline_menu_recirc\&action=click\&pgtype=LegacyCollection\&impression_id=711d9f41-f1db-11ea-9371-2fd28fa48a5f\&variant=undefined}{Eviction
  Moratorium}
\item
  \href{https://www.nytimes3xbfgragh.onion/interactive/2020/09/02/magazine/food-insecurity-hunger-us.html?name=styln-coronavirus-markets\&region=TOP_BANNER\&block=storyline_menu_recirc\&action=click\&pgtype=LegacyCollection\&impression_id=711d9f42-f1db-11ea-9371-2fd28fa48a5f\&variant=undefined}{American
  Hunger}
\end{itemize}

\hypertarget{amc-to-reopen-140-theaters-ahead-of-tenet-release}{%
\section{AMC to Reopen 140 Theaters Ahead of `Tenet'
Release}\label{amc-to-reopen-140-theaters-ahead-of-tenet-release}}

Last Updated

Sept. 2, 2020, 4:11 a.m. ET

Sept. 2, 2020, 4:11 a.m. ET

This briefing is no longer being updated.

\hypertarget{heres-what-you-need-to-know}{%
\subsubsection{Here's what you need to
know:}\label{heres-what-you-need-to-know}}

\begin{itemize}
\item
  \protect\hyperlink{amc-plans-to-reopen-140-additional-theaters-by-friday}{}

  AMC plans to reopen 140 additional theaters by Friday.
\item
  \protect\hyperlink{mnuchin-tells-congress-there-is-more-work-to-be-done-to-help-the-economy-recover}{}

  Mnuchin tells Congress `there is more work to be done' to help the
  economy recover.
\item
  \protect\hyperlink{a-top-fed-official-says-the-economy-still-faces-big-risks}{}

  A top Fed official says the economy still faces big risks.
\item
  \protect\hyperlink{stocks-rise-following-augusts-highs}{}

  Stocks rise, following August's highs.
\item
  \protect\hyperlink{the-labor-department-will-start-counting-unemployment-claims-in-a-new-way}{}

  The Labor Department will start counting unemployment claims in a new
  way.
\end{itemize}

\hypertarget{amc-plans-to-reopen-140-additional-theaters-by-friday}{%
\subsection{\texorpdfstring{\protect\hyperlink{amc-plans-to-reopen-140-additional-theaters-by-friday}{AMC
plans to reopen 140 additional theaters by
Friday.}}{AMC plans to reopen 140 additional theaters by Friday.}}\label{amc-plans-to-reopen-140-additional-theaters-by-friday}}

\includegraphics{https://static01.graylady3jvrrxbe.onion/images/2020/09/01/business/01markets-brf-amc/merlin_176273319_27b12dd5-1967-47b4-914c-9a8f5967d8ff-articleLarge.jpg?quality=75\&auto=webp\&disable=upscale}

\textbf{AMC Entertainment} announced that it would reopen roughly 140
additional theaters by Sept. 4, bringing a total of 70 percent of its
movie theaters in the United States back into operation. Shares of the
company jumped nearly 10 percent after-hours following the announcement
Tuesday afternoon.

The majority of the theaters will reopen on Sept. 3, the same day that
Christopher Nolan's ``Tenet'' comes out. Theater executives are
\href{https://www.nytimes3xbfgragh.onion/2020/08/28/business/media/coronavirus-movie-theaters-new-mutants.html}{hoping}
the \$200 million thriller will draw crowds of viewers out of their
homes to experience the film on the big screen. But it
\href{https://www.nytimes3xbfgragh.onion/2020/08/28/business/media/coronavirus-movie-theaters-new-mutants.html}{remains
to be seen} whether people will want to sit in an enclosed space next to
other moviegoers at a time when the virus continues to spread across the
country.

AMC says it is consulting with ``top scientists and experts in public
health'' to make the reopening process safe for patrons.

``Our comprehensive commitment to operating our theaters safely now
includes social distancing through limiting ticket sales and automatic
seat blocking, seamless contactless ticketing, greatly enhanced cleaning
procedures, the availability of hand sanitizer and disinfecting wipes
throughout our theaters, as well as a mandatory mask policy for all
guests and crew members,'' the company said in a
\href{http://investor.amctheatres.com/file/Index?KeyFile=405157778}{statement}.

On Sept. 4, AMC plans to start reopening its first California theaters,
including seven theaters in the San Diego area.

--- \href{http://nytimes3xbfgragh.onion/by/gillian-friedman}{Gillian
Friedman}

\hypertarget{the-federal-government-barred-evictions-through-dec-31-citing-the-virus-risk}{%
\subsection{\texorpdfstring{\protect\hyperlink{the-federal-government-barred-evictions-through-dec-31-citing-the-virus-risk}{The
federal government barred evictions through Dec. 31, citing the virus
risk.}}{The federal government barred evictions through Dec. 31, citing the virus risk.}}\label{the-federal-government-barred-evictions-through-dec-31-citing-the-virus-risk}}

\includegraphics{https://static01.graylady3jvrrxbe.onion/images/2020/09/01/business/01markets-brf-evictionssub/merlin_175674180_a4b1a363-91ff-4bc8-abb2-c5af8f2bbff9-articleLarge.jpg?quality=75\&auto=webp\&disable=upscale}

The Trump administration issued an order on Tuesday barring evictions of
most renters in the country for the rest of the year as the nation
grapples with the coronavirus pandemic.

The order, put forward by the Centers for Disease Control and
Prevention, said the action was needed to stop the spread of the virus
and to avoid having renters lose their homes and wind up in shelters or
other crowded living conditions, compounding the crisis.

The moratorium would go further than
\href{https://www.nytimes3xbfgragh.onion/2020/07/23/business/evictions-moratorium-cares-act.html}{the
eviction ban under the CARES Act,} which covered as many as 12.3 million
renters living in apartment complexes or single-family homes financed
with federally backed mortgages. That provision expired in July.

To apply for the new moratorium, tenants would have to attest to a
substantial loss of household income, the inability to pay full rent and
best efforts to pay partial rent. Tenants would also need to stipulate
that eviction would be likely to leave them homeless or force them to
live with others at close quarters.

The order does not relieve tenants of their ultimate obligation to pay
rent. It applies to those who expect to earn no more than \$99,000 this
year or who meet other income limits.

---
\href{https://www.nytimes3xbfgragh.onion/by/matthew-goldstein}{Matthew
Goldstein}

\hypertarget{advertisement}{%
\subsubsection{Advertisement}\label{advertisement}}

\protect\hyperlink{after-dfp-ad-mid1}{Continue reading the main story}

\hypertarget{mnuchin-tells-congress-there-is-more-work-to-be-done-to-help-the-economy-recover}{%
\subsection{\texorpdfstring{\protect\hyperlink{mnuchin-tells-congress-there-is-more-work-to-be-done-to-help-the-economy-recover}{Mnuchin
tells Congress `there is more work to be done' to help the economy
recover.}}{Mnuchin tells Congress `there is more work to be done' to help the economy recover.}}\label{mnuchin-tells-congress-there-is-more-work-to-be-done-to-help-the-economy-recover}}

\includegraphics{https://static01.graylady3jvrrxbe.onion/images/2020/09/01/business/01DC-MNUCHIN-01/01DC-MNUCHIN-01-videoSixteenByNine3000.jpg}

Treasury Secretary Steven Mnuchin told a congressional committee on
Tuesday that the economy is recovering from the pandemic-induced
recession but said ``there is more work to be done'' and that he would
continue pushing for a ``bipartisan agreement'' on another round of
economic stimulus from Congress.

``While we continue to see signs of a strong economic recovery, we are
sensitive to the fact that there is more work to be done, and certain
areas of the economy require additional relief,'' Mr. Mnuchin told the
Select Subcommittee on the Coronavirus Crisis.

Mr. Mnuchin suggested that lawmakers focus on a smaller, more targeted
package of relief, saying ``we need support quickly and if we need to do
more we can come back.''

But while the Treasury secretary called on Congress to act, the chairman
of the select subcommittee, Representative James E. Clyburn, Democrat of
South Carolina, scolded Mr. Mnuchin for the administration's handling of
the pandemic and the recession and said the White House needs to take
additional steps to help workers and businesses.

Mr. Clyburn challenged Mr. Mnuchin on President Trump's claims that the
economy is ``roaring back,'' saying unemployment remains high and
millions are still out of work.

Deep divisions remain between the White House and congressional
Democrats and it is unclear whether another stimulus package might pass.
In a sign of those ongoing differences, Mr. Mnuchin said the next round
of stimulus should include ``liability protection for universities,
schools, and businesses.''

That proved to be a
\href{https://www.nytimes3xbfgragh.onion/2020/08/05/us/politics/liability-shield-business-coronavirus.html}{stumbling
point} the last time, with Mitch McConnell, the Senate majority leader,
insisting that any additional relief include protection against
virus-related lawsuits for companies and other institutions. Democrats,
along with unions and workers' rights advocates, objected to the
proposal, saying it would result in negligent behavior on the part of
businesses and schools and lead to more coronavirus cases and more
deaths.

--- \href{https://www.nytimes3xbfgragh.onion/by/jim-tankersley}{Jim
Tankersley}

\hypertarget{a-top-fed-official-says-the-economy-still-faces-big-risks}{%
\subsection{\texorpdfstring{\protect\hyperlink{a-top-fed-official-says-the-economy-still-faces-big-risks}{A
top Fed official says the economy still faces big
risks.}}{A top Fed official says the economy still faces big risks.}}\label{a-top-fed-official-says-the-economy-still-faces-big-risks}}

\includegraphics{https://static01.graylady3jvrrxbe.onion/images/2020/09/01/business/01markets-brf-brainard/merlin_162129732_e45d6e23-8cfd-404e-b30b-e73c8d5cb621-articleLarge.jpg?quality=75\&auto=webp\&disable=upscale}

Lael Brainard, a Federal Reserve governor, said the U.S. economy
remained at risk as the coronavirus pandemic wears on --- and support
from Congress and the White House was crucial to cushioning the blow.

``The economy continues to face considerable uncertainty associated with
the vagaries of the Covid-19 pandemic, and risks are tilted to the
downside,'' Ms. Brainard said in remarks prepared for delivery at a
Brookings Institution event on Tuesday. ``As was true in the first phase
of the crisis, fiscal support will remain essential to sustaining many
families and businesses.''

Her comments came as the future of another government support package
looked uncertain. Ms. Brainard, the last person on the Fed's board in
Washington to have been picked for her job by the Obama administration,
said that monetary policy would also play a role as pandemic uncertainty
persisted, and that central bankers would need to pivot from stabilizing
markets to supporting economic growth in the coming months.

``It will be important to provide the requisite accommodation to achieve
maximum employment and average inflation of 2 percent over time,'' she
said.

The Fed last week unveiled a
\href{https://www.nytimes3xbfgragh.onion/2020/08/27/business/economy/federal-reserve-inflation-jerome-powell.html}{new
long-run policy statement}, making critical updates to its strategy for
achieving its goals of full employment and stable inflation. Ms.
Brainard said the tweaks, which together lay the groundwork for long
periods of very low interest rates, will help to guide the central
bank's policies coming out of the pandemic.

One key change --- the Fed will now aim for 2 percent inflation on
average over time, instead of as a more or less absolute goal --- will
allow the Fed to keep rates low even as prices climb slightly, she said.

``I would expect the Committee to accommodate rather than offset
inflationary pressures moderately above 2 percent, in a process of
opportunistic reflation,'' she said.

--- \href{https://www.nytimes3xbfgragh.onion/by/jeanna-smialek}{Jeanna
Smialek}

\hypertarget{stocks-rise-following-augusts-highs}{%
\subsection{\texorpdfstring{\protect\hyperlink{stocks-rise-following-augusts-highs}{Stocks
rise, following August's
highs.}}{Stocks rise, following August's highs.}}\label{stocks-rise-following-augusts-highs}}

\begin{itemize}
\item
  \textbf{U.S. stocks fluctuated} in early trading Tuesday before
  turning positive. The S\&P 500 rose 0.75 percent, closing at another
  record high. The index ended August up about 7 percent for
  \href{https://www.nytimes3xbfgragh.onion/live/2020/08/31/business/stock-market-today-coronavirus/the-sp-ended-its-run-of-gains-but-had-its-second-best-month-this-year}{its
  second-best month of the year}.
\item
  \textbf{Zoom's} stock
  \href{https://www.nytimes3xbfgragh.onion/live/2020/09/01/business/stock-market-today-coronavirus/zoom-shares-are-soaring}{shot
  up 40 percent} after the video conferencing company reported that its
  revenue had quadrupled in the most recent quarter.
\item
  \textbf{European stocks were lower} on Tuesday, after Germany,
  Europe's largest economy, lowered its
  \href{https://www.nytimes3xbfgragh.onion/reuters/2020/09/01/world/europe/01reuters-health-coronavirus-germany-economy.html}{economic
  growth forecast for 2021}, though it revised upward its 2020 estimate.
\item
  Most Asian markets ended the day slightly higher, but Japan's Nikkei
  was flat. Markets in China were
  \href{https://www.reuters.com/article/us-china-economy-pmi/chinas-august-export-orders-shake-off-covid-gloom-fuelling-factory-expansion-caixin-pmi-idUSKBN25S3A7}{lifted
  by data} showing that the country's factory activity last month
  \textbf{expanded at the fastest rate since 2011}, signaling a
  continued recovery from the economic damage wrought by the pandemic.
\item
  The U.S. dollar continued its decline, dropping to a two-year low,
  while the euro rose to just below \$1.20 and China's yuan also
  strengthened.
\item
  The \textbf{Dow Jones industrial average} introduced on Monday its
  \href{https://www.nytimes3xbfgragh.onion/live/2020/08/31/business/stock-market-today-coronavirus/why-the-dow-revamped-its-lineup-of-stocks}{revamped
  lineup of stocks}, with \textbf{Amgen}, \textbf{Salesforce.com} and
  \textbf{Honeywell} replacing \textbf{Exxon Mobil}, \textbf{Raytheon}
  and \textbf{Pfizer} in the 30-stock menu. The rejiggering came after
  \textbf{Apple's} 4-for-1 stock split.
\end{itemize}

\hypertarget{advertisement-1}{%
\subsubsection{Advertisement}\label{advertisement-1}}

\protect\hyperlink{after-dfp-ad-mid2}{Continue reading the main story}

\hypertarget{the-labor-department-will-start-counting-unemployment-claims-in-a-new-way}{%
\subsection{\texorpdfstring{\protect\hyperlink{the-labor-department-will-start-counting-unemployment-claims-in-a-new-way}{The
Labor Department will start counting unemployment claims in a new
way.}}{The Labor Department will start counting unemployment claims in a new way.}}\label{the-labor-department-will-start-counting-unemployment-claims-in-a-new-way}}

\includegraphics{https://static01.graylady3jvrrxbe.onion/images/2020/09/01/business/01-markets-brf-jobless-adjustmen/merlin_174916647_7c65ab1c-2d78-486c-a04a-7946686b8b45-articleLarge.jpg?quality=75\&auto=webp\&disable=upscale}

Watch out: Weekly data on unemployment filings are about to get even
more confusing.

The Labor Department has
\href{https://oui.doleta.gov/press/2020/082720.pdf}{announced} that it
is changing the way it adjusts jobless claims figures for seasonal
patterns. Economists say the change will make the data more accurate,
but it will also complicate comparisons over time.

The seasonal adjustment process is meant to account for regular,
predictable patterns in layoffs. Hundreds of thousands of seasonal
retail workers are let go after the holidays each year, for example.

The surge of layoffs during the pandemic, however, threw off seasonal
patterns and led the seasonal adjustment process to exaggerate
week-to-week changes.

Until now, seasonal adjustments have taken past patterns into account by
offsetting the total by the percentage by which claims ordinarily rose
or fell that week. The new methodology will base the adjustments on the
number of people who filed claims in a given week in prior years.

Heidi Shierholz, a former chief economist for the Labor Department under
President Barack Obama and now a senior economist at the left-leaning
Economic Policy Institute, said the change in methodology should make
the seasonally adjusted numbers more accurate.

The Labor Department does not plan to revise its estimates for previous
weeks, however. That means that the next set of numbers, which will be
released on Thursday, will not be directly comparable to earlier data.
The report will almost certainly show a big drop in seasonally adjusted
claims, but that will reflect the change in methodology, not a
real-world decline in layoffs.

As a result, in our coverage, The Times plans to emphasize unadjusted
figures, which will not be affected by the change in methodology and are
comparable over time. We will continue to use the unadjusted figures at
least until weekly claims fall to a level where normal seasonal patterns
become relevant again.

None of this will change the big picture. Both adjusted and unadjusted
data showed a huge spike in unemployment filings beginning in March, and
a much more gradual decline since then. Both show progress stalling in
recent weeks.

``The broad brush strokes are the same no matter what numbers you use
here,'' Ms. Shierholz said.

--- \href{https://www.nytimes3xbfgragh.onion/by/ben-casselman}{Ben
Casselman}

\hypertarget{the-white-house-will-delay-payroll-tax-collection-for-more-than-a-million-federal-workers}{%
\subsection{\texorpdfstring{\protect\hyperlink{the-white-house-will-delay-payroll-tax-collection-for-more-than-a-million-federal-workers}{The
White House will delay payroll tax collection for more than a million
federal
workers.}}{The White House will delay payroll tax collection for more than a million federal workers.}}\label{the-white-house-will-delay-payroll-tax-collection-for-more-than-a-million-federal-workers}}

\includegraphics{https://static01.graylady3jvrrxbe.onion/images/2020/09/01/business/01markets-brf-omb/merlin_170583768_790ed45a-3e19-41b7-aaf5-8b70e22437dc-articleLarge.jpg?quality=75\&auto=webp\&disable=upscale}

The Trump administration plans to delay the collection of payroll taxes
for more than one million federal workers through the end of the year, a
move that could result in a sharp reduction in pay in the early months
of 2021.

The plan, which stems from an executive order issued by President Trump
in August, would force some federal employees into a complicated
deferral of tax liability that
\href{https://www.nytimes3xbfgragh.onion/2020/08/27/us/politics/trump-payroll-tax-coronavirus.html}{few
private-sector workers are likely to face}. Many companies and business
groups have said they don't plan to suspend the collection of payroll
taxes, which is voluntary, calling it unnecessary and overly complex.

Mr. Trump's executive order aims to boost the economy by delaying the
collection of the tax workers pay to help fund Social Security. But
because Mr. Trump does not have the authority to eliminate the tax
without the consent of Congress, workers will still owe that money next
year. Mr. Trump has promised to sign a bill that would eliminate the
taxes owed but Congress has shown little appetite for such legislation,
in part because the money is used to fund entitlement programs that are
already facing future insolvency.

Last week, the Treasury Department
\href{https://www.irs.gov/pub/irs-drop/n-20-65.pdf}{issued guidance} to
implement the delay, which affects workers earning less than \$104,000
per year. That guidance effectively gives employers the ability to
suspend payroll tax collections from Sept. 1 through Dec. 31. If no
additional measures are passed by Congress, those deferred taxes would
be due in the first quarter of 2021. As a result, employees would see
larger-than-normal paychecks for the end of this year, and smaller
paychecks at the start of next year.

Few companies have indicated they would participate in the deferral, but
a spokeswoman for the Office of Management and Budget said on Tuesday in
an email that the White House was moving to implement the guidance for
its employees.

While the federal government is the nation's largest employer, the move
by itself will not provide much of a boost to economic growth. The
Committee for a Responsible Federal Budget estimates that the overall
tax deferral for eligible workers through year's end would add up to
about \$5 billion over four months.

--- \href{https://www.nytimes3xbfgragh.onion/by/jim-tankersley}{Jim
Tankersley}

\hypertarget{extra-unemployment-pay-deters-few-from-seeking-work-a-survey-finds}{%
\subsection{\texorpdfstring{\protect\hyperlink{extra-unemployment-pay-deters-few-from-seeking-work-a-survey-finds}{Extra
unemployment pay deters few from seeking work, a survey
finds.}}{Extra unemployment pay deters few from seeking work, a survey finds.}}\label{extra-unemployment-pay-deters-few-from-seeking-work-a-survey-finds}}

\includegraphics{https://static01.graylady3jvrrxbe.onion/images/2020/09/01/business/01markets-brf-gallup-benefit/merlin_174644919_3946835c-62e1-4fa6-a8a3-4591fe325036-articleLarge.jpg?quality=75\&auto=webp\&disable=upscale}

Most unemployed Americans would go back to work if given the
opportunity, even if the government made jobless benefits more generous,
according to a new survey.

\href{https://news.gallup.com/poll/318452/broad-bipartisan-support-additional-stimulus.aspx}{The
Gallup survey} was conducted in early August, days after the
\href{https://www.nytimes3xbfgragh.onion/2020/07/29/business/economy/unemployment-benefits-coronavirus.html}{expiration
of the \$600 a week} in extra benefits that the federal government had
been paying out to jobless workers during the pandemic. More than 400
respondents who were receiving unemployment benefits were asked whether
they would return to their previous jobs if the payments were reinstated
at a lower level. More than 80 percent said they were ``very likely'' or
``somewhat likely'' to go back to work.

The amount of money offered made little difference to people's
decisions. About a third of the respondents were asked about a
prospective \$150 weekly add-on to their unemployment benefits. Another
third were asked about \$300, and the remaining third were asked about
\$450. The responses looked almost identical across the three groups.

Sonal Desai, chief investment officer of Franklin Templeton Fixed
Income, a partner with Gallup on the survey, said the results might look
surprising at first. But jobless Americans have good reason to prefer
going back to work. The expiration of the earlier \$600 supplement was a
potent reminder that benefits are temporary. And with the unemployment
rate still above 10 percent, there is lots of competition for available
jobs.

``You've got literally millions of people who have been sidelined, so
especially if you're in the restaurant or hospitality business, you
would be worried that if you didn't go back that someone else would take
your job,'' Ms. Desai said.

\href{https://news.yale.edu/2020/07/27/yale-study-finds-expanded-jobless-benefits-did-not-reduce-employment}{Other
recent research} has also found that the extra jobless benefits did not
discourage people from returning to work in significant numbers. And
recent economic data does not suggest that jobless Americans have rushed
back to work since the \$600 benefit expired.

--- \href{https://www.nytimes3xbfgragh.onion/by/ben-casselman}{Ben
Casselman}

\hypertarget{advertisement-2}{%
\subsubsection{Advertisement}\label{advertisement-2}}

\protect\hyperlink{after-dfp-ad-mid3}{Continue reading the main story}

\hypertarget{a-new-round-of-stimulus-checks-americans-say-yes}{%
\subsection{\texorpdfstring{\protect\hyperlink{a-new-round-of-stimulus-checks-americans-say-yes}{A
new round of stimulus checks? Americans say
yes.}}{A new round of stimulus checks? Americans say yes.}}\label{a-new-round-of-stimulus-checks-americans-say-yes}}

\includegraphics{https://static01.graylady3jvrrxbe.onion/images/2020/09/01/business/01markets-brf-gallup-stimulus-sub/merlin_174645003_0b941ef2-5364-4c6e-8598-3b37c07781ac-articleLarge.jpg?quality=75\&auto=webp\&disable=upscale}

Democrats and Republicans don't agree on much these days. But they agree
on this: They would like the government to send them money.

According to a
\href{https://news.gallup.com/poll/318452/broad-bipartisan-support-additional-stimulus.aspx}{survey
of 5,000 adults} conducted in early August by Gallup and Franklin
Templeton, the investment firm, 70 percent of Americans believe the
federal government should send a second round of direct cash
payments.About 82 percent of Democrats and 64 percent of Republicans
supported such a move.

The partisan divide over the size of a potential stimulus payment was
even smaller. Among those who support another round of checks, about
two-thirds across all partisan groups said the payments should be \$900
or more, the largest option offered in the survey.

``At this point, with unemployment still quite high, it's obvious that
there's not going to be an immediate recovery, so there's still a lot of
interest among both parties in continuing some form of relief,'' said
Jonathan Rothwell, principal economist for Gallup.

The earlier payments, which
\href{https://www.nytimes3xbfgragh.onion/article/where-is-my-stimulus-payment.html}{sent
\$1,200 per adult} and \$500 per child to most American households, were
among the most popular components of the CARES Act, the emergency
spending package passed in March.

But prospects for further payments are uncertain. House Democrats in May
\href{https://www.nytimes3xbfgragh.onion/2020/05/15/us/politics/house-simulus-vote.html}{passed
a bill} that included another round of \$1,200 checks, but Senate
Republicans have refused to take up the measure and are divided over an
alternative.

--- \href{https://www.nytimes3xbfgragh.onion/by/ben-casselman}{Ben
Casselman}

\hypertarget{jc-penney-has-10-days-to-avoid-liquidation}{%
\subsection{\texorpdfstring{\protect\hyperlink{jc-penney-has-10-days-to-avoid-liquidation}{J.C.
Penney has 10 days to avoid
liquidation.}}{J.C. Penney has 10 days to avoid liquidation.}}\label{jc-penney-has-10-days-to-avoid-liquidation}}

\includegraphics{https://static01.graylady3jvrrxbe.onion/images/2020/09/01/business/01markets-brf-jcpenney/merlin_172533198_d9a5617d-2b18-46d7-9f63-8ee7745958a2-articleLarge.jpg?quality=75\&auto=webp\&disable=upscale}

\textbf{J.C. Penney}'s advisers
\href{https://www.nytimes3xbfgragh.onion/live/2020/08/31/business/stock-market-today-coronavirus/jc-penney-hits-stalemate-in-buyer-talks}{warned
a bankruptcy judge} in Texas on Monday that talks with buyers have hit a
stalemate. The retailer now has until Sept. 10 to make a deal with a
buyer, sell to its creditors or liquidate,
\href{https://www.nytimes3xbfgragh.onion/2020/09/01/business/dealbook/tiktok-trump-china-cold-war.html}{today's
DealBook newsletter explains}.

The department store operator's survival hinges on a plan to carve out
some of its best properties into a real estate investment trust, or
REIT, and sell its retail business to a buyer that would keep stores
open. Its lenders have steered the process since it filed for bankruptcy
in May.

It thought it had found salvation in \textbf{Brookfield Property
Partners} and \textbf{Simon Property Group}, after \textbf{Hudson's Bay
Group} and \textbf{Sycamore Partners} dropped out of the running.
Brookfield and Simon both own malls with J.C. Penney stores as tenants,
so a liquidation would hurt them. Still, the consortium of mall owners
and J.C. Penney's creditors have butted heads. Key sticking points
include valuation and who has the right to redevelop mall space:
Brookfield and Simon or the creditors. If creditors lose that right, any
REIT would have less value.

Talks have been dragging for weeks**.** The bankruptcy judge overseeing
the case told both sides that they were trying the court's patience. The
rebuke wasn't enough: J.C. Penney's lawyer, \textbf{Kirkland \& Ellis}'s
bankruptcy guru Josh Sussberg, told the court yesterday that the
discussions with potential buyers had stalled, and the company would
instead focus on a bid by lenders. It is unclear, though, whether the
hedge funds that own J.C. Penney's debt want to take over an ailing
retail business during a pandemic.

Also of note: Mr. Sussberg said in the hearing that the retailer would
shut even more stores.

At risk are some 70,000 jobs. A liquidation would also likely bring bad
publicity for the hedge funds that have funded J.C. Penney's bankruptcy.
(Mr. Sussberg made sure to list the funds' names, which included
\textbf{H/2 Capital}, at an earlier hearing.) It would also be costly
for Brookfield and Simon, but they may simply decide to take the hit and
adjust to a new world in which malls are reborn
\href{https://www.wsj.com/articles/amazon-and-giant-mall-operator-look-at-turning-sears-j-c-penney-stores-into-fulfillment-centers-11596992863}{as
distribution centers}.

--- Lauren Hirsch

\hypertarget{zoom-shares-are-soaring}{%
\subsection{\texorpdfstring{\protect\hyperlink{zoom-shares-are-soaring}{Zoom
shares are
soaring.}}{Zoom shares are soaring.}}\label{zoom-shares-are-soaring}}

If you like charts that go up and to the right, there is a lot to like
in recent market moves, notes
\href{https://www.nytimes3xbfgragh.onion/2020/09/01/business/dealbook/tiktok-trump-china-cold-war.html}{today's
DealBook newsletter}.

\textbf{Zoom}'s ****
\href{https://www.nytimes3xbfgragh.onion/live/2020/08/31/business/stock-market-today-coronavirus\#zooms-profit-jumped-3276-percent-in-the-2nd-quarter}{latest
quarterly earnings} beat already high expectations --- and raised them
further. The videoconferencing company reported yesterday that revenue
more than quadrupled in its most recent quarter, while profit was 30
times higher than a year ago. At the close on Tuesday, Zoom's shares
were up more than 40 percent,
\href{https://www.bloomberg.com/news/articles/2020-09-01/zoom-s-record-quarter-adds-4-2-billion-to-ceo-yuan-s-fortune?sref=0w5HLLb3}{adding
billions to the net worth} of its chief executive, Eric Yuan.

\textbf{Tesla}'s stock is also soaring. Monday's five-for-one stock
split had no effect on the company's valuation, but the electric
carmaker's shares gained more than 12 percent on the day. Before the
market opened on Tuesday, the company
\href{https://ir.tesla.com/node/21016/html}{announced} that it will
raise up to \$5 billion by selling new shares ``from time to time.'' Now
seems like a good time to take advantage of the run-up in its stock
price, which has made Tesla the
\href{https://www.marketwatch.com/story/tesla-passes-visa-to-become-seventh-largest-us-company-by-market-cap-2020-08-31}{seventh-largest
listed company} in the U.S. and Elon Musk, its chief executive, the
world's third-richest man
(\href{https://www.bloomberg.com/news/articles/2020-08-31/elon-musk-is-now-richer-than-mark-zuckerberg-on-tesla-surge?sref=0w5HLLb3}{ahead
of Mark Zuckerberg} and closing in on Bill Gates).

Its shares were down nearly 5 percent on Tuesday, but have risen by more
than 400 percent so far this year.

--- \href{https://www.nytimes3xbfgragh.onion/by/jason-karaian}{Jason
Karaian}

\hypertarget{advertisement-3}{%
\subsubsection{Advertisement}\label{advertisement-3}}

\protect\hyperlink{after-dfp-ad-mid4}{Continue reading the main story}

\hypertarget{52-former-mcdonalds-franchisees-sue-the-fast-food-chain-for-racial-discrimination}{%
\subsection{\texorpdfstring{\protect\hyperlink{52-former-mcdonalds-franchisees-sue-the-fast-food-chain-for-racial-discrimination}{52
former McDonald's franchisees sue the fast-food chain for racial
discrimination.}}{52 former McDonald's franchisees sue the fast-food chain for racial discrimination.}}\label{52-former-mcdonalds-franchisees-sue-the-fast-food-chain-for-racial-discrimination}}

\includegraphics{https://static01.graylady3jvrrxbe.onion/images/2020/09/01/business/01-markets-brf-mcdonalds/01-markets-brf-mcdonalds-articleLarge.jpg?quality=75\&auto=webp\&disable=upscale}

Dozens of former McDonald's franchisees are suing the company for racial
discrimination, saying that the fast food giant placed Black-owned
franchises in subpar locations with higher operating and insurance costs
and less opportunity for profit than locations owned by white
franchisees.

``Revenue at McDonald's is determined by one thing and one thing only,
and that's location,'' said James Ferraro, attorney for the plaintiffs,
in an interview. ``When you want a Big Mac, you go to the nearest
McDonald's location.''

In the lawsuit, which was filed Tuesday in a federal court in Illinois,
the 52 plaintiffs claimed that McDonald's had impeded the efforts of
Black franchisees to acquire additional stores and pushed Black
franchisees out of the system by refusing to offer the same support,
including rent relief, offered to white franchisees experiencing
financial hardship.

The lawsuit said the plaintiffs' average annual revenue, at \$2 million,
was at least \$700,000 less than the company's national franchisee
average between 2011 and 2016. Last year, national average sales for its
franchisees was \$2.9 million, according to the suit.

McDonald's denied the racial discrimination allegations, saying that
while the company might recommend locations, the franchisees themselves
ultimately chose the location they wished to purchase.

``We are confident that the facts will show how committed we are to the
diversity and equal opportunity of the McDonald's system, including
across our franchisees, suppliers and employees,'' the company said in a
statement.

--- \href{http://nytimes3xbfgragh.onion/by/gillian-friedman}{Gillian
Friedman}

\hypertarget{the-latest-old-navy-to-give-workers-a-paid-day-for-election-day}{%
\subsection{\texorpdfstring{\protect\hyperlink{the-latest-old-navy-to-give-workers-a-paid-day-for-election-day}{The
latest: Old Navy to give workers a paid day for Election
Day.}}{The latest: Old Navy to give workers a paid day for Election Day.}}\label{the-latest-old-navy-to-give-workers-a-paid-day-for-election-day}}

\begin{itemize}
\item
  \textbf{Old Navy}, one of the largest U.S. apparel chains, said that
  it would give its employees a day of pay for serving as poll workers
  on Election Day this year, whether or not they are scheduled to work
  in stores on Nov. 3. The compensation will add to payment from their
  county's election commissioner. The chain said in a release that it
  wanted to engage its field employees in the democratic process,
  especially given that about 64 percent are between the ages of 18 and
  29. Retail is the second-biggest private employer in the U.S. after
  health care, and a series of chains have recently started announcing
  days off and other initiatives to encourage voter turnout this year in
  a tight presidential race.
\item
  \textbf{General Motors} said on Tuesday that it would stop making
  ventilators after delivering 30,000 of them to the federal government.
  The company's partner, \textbf{Ventec Life Systems}, will take control
  of an assembly line at a G.M. electronics plant in Kokomo, Ind. The
  automaker started building the assembly line in March to meet surging
  demand for ventilators in the early days of the coronavirus pandemic.
  Ventec will continue making ventilators in Kokomo and at its own plant
  in Bothell, Wash.
\item
  \textbf{Walmart} is
  \href{https://www.nytimes3xbfgragh.onion/2020/09/01/business/walmart-plus-membership.html}{rolling
  out a membership service} that will give customers free shipping on
  tens of thousands of items, including produce and groceries. The
  service, Walmart+, will cost \$98 a year. That is lower than the \$119
  charged for \textbf{Amazon Prime}, which has set the bar for
  e-commerce membership services, but Walmart+ will require an order of
  at least \$35 to qualify for the free shipping, while Prime does not
  have a minimum. Walmart said many of the 160,000 items that would
  qualify for the free shipping would be delivered directly from its
  stores to customers' homes.
\item
  On Sunday, \textbf{United Airlines} said it was
  \href{https://www.nytimes3xbfgragh.onion/live/2020/08/31/business/stock-market-today-coronavirus\#delta-and-american-follow-united-in-permanently-dropping-some-change-fees}{permanently
  dropping change fees} for most customers flying domestically.
  \textbf{American Airlines} and \textbf{Delta Air Lines} followed suit
  a day later. The changes, effective immediately, apply to all standard
  economy and premium seats, but not to any of the airlines' low-price
  basic economy seats, which come with additional restrictions.
\end{itemize}

\hypertarget{europes-unemployment-rate-climbs-even-as-furlough-benefits-are-extended}{%
\subsection{\texorpdfstring{\protect\hyperlink{europes-unemployment-rate-climbs-even-as-furlough-benefits-are-extended}{Europe's
unemployment rate climbs even as furlough benefits are
extended.}}{Europe's unemployment rate climbs even as furlough benefits are extended.}}\label{europes-unemployment-rate-climbs-even-as-furlough-benefits-are-extended}}

\includegraphics{https://static01.graylady3jvrrxbe.onion/images/2020/09/01/business/01-markets-brf-eurozone/01-markets-brf-eurozone-articleLarge.jpg?quality=75\&auto=webp\&disable=upscale}

The eurozone's
\href{https://ec.europa.eu/eurostat/documents/2995521/10568643/3-01092020-BP-EN.pdf/39668e66-2fd4-4ec0-9fd4-4d7c99306c98}{unemployment
rate} rose slightly in July to 7.9 percent, up from 7.2 percent in
March, which was the lowest on record, according to data published on
Tuesday.

Though
\href{https://www.nytimes3xbfgragh.onion/2020/08/24/business/europe-economy-layoffs.html}{government
programs} protected the jobs of a substantial portion of Europe's work
force during the height of the pandemic, unemployment is still rising.
In July, 12.8 million people were unemployed, 500,000 more than a year
ago. Several countries, including
\href{https://www.bloomberg.com/news/articles/2020-06-24/france-outlines-new-virus-furloughs-that-could-run-for-two-years\#:~:text=The\%20new\%20program\%20emulates\%20the,jobs\%20during\%20the\%20pandemic\%20lockdown.\&text=If\%20unions\%20and\%20business\%20agree,at\%20the\%20Elysee\%20Palace\%20said.}{France}
and
\href{https://www.nytimes3xbfgragh.onion/live/2020/08/26/business/stock-market-updates-coronavirus\#germany-extends-worker-benefits-as-it-prepares-for-a-long-recovery}{Germany},
have said recently that they would extend some of the wage-protection
benefits as large
employers\href{https://www.nytimes3xbfgragh.onion/2020/08/24/business/europe-economy-layoffs.html}{continued
to announce the layoffs} of thousands of workers.

The widespread use of furlough programs might also conceal
\href{https://www.nytimes3xbfgragh.onion/2020/08/13/business/europe-precarious-workers.html}{the
true impact} of the pandemic on Europe's labor market in the official
data, which only records people who are unemployed and currently looking
for new work.

The gradual increase in unemployment is difficult to interpret, said
Claus Vistesen, an economist at Pantheon Macroeconomics, in a note. ``It
is just as likely that unemployment rose because people returned to the
labor market --- and were classified as job seekers --- as it is that
previously employed or furloughed workers have lost their job,'' Mr.
Vistesen said.

What can be seen is that unemployment is higher for young people and
women. The eurozone unemployment rate for those under 25 was 17.3
percent, the highest since early 2018. For women, the unemployment rate
rose to 8.3 percent in July, from 8 percent the previous month. For men,
the rate rose to 7.6 percent, from 7.5 percent in June.

Of the countries that have reported so far, Spain, which is grappling
with
\href{https://www.nytimes3xbfgragh.onion/2020/08/31/world/europe/coronavirus-covid-spain-second-wave.html}{a
resurgence in coronavirus cases}, had the highest unemployment rate of
15.8 percent.

Separate data showed the annual rate of
\href{https://ec.europa.eu/eurostat/documents/2995521/10545459/2-01092020-AP-EN.pdf/7c0db6bb-3974-ce20-a7f0-6281743d0d7c}{inflation
turned negative} in the eurozone --- those nations that use the euro as
their currency --- for the first time since 2016.

--- \href{https://www.nytimes3xbfgragh.onion/by/eshe-nelson}{Eshe
Nelson}

\hypertarget{forty-one-states-have-been-approved-for-the-extra-unemployment-benefit}{%
\subsection{\texorpdfstring{\protect\hyperlink{forty-one-states-have-been-approved-for-the-extra-unemployment-benefit}{Forty-one
states have been approved for the extra unemployment
benefit.}}{Forty-one states have been approved for the extra unemployment benefit.}}\label{forty-one-states-have-been-approved-for-the-extra-unemployment-benefit}}

\includegraphics{https://static01.graylady3jvrrxbe.onion/images/2020/09/01/business/01markets-brf-supplement/merlin_174916812_81bfe728-a30f-40ef-818f-d536d8d5df85-articleLarge.jpg?quality=75\&auto=webp\&disable=upscale}

Forty-one states are now
\href{https://www.nytimes3xbfgragh.onion/article/stimulus-unemployment-payment-benefit.html}{signed
up} to provide their residents with an extra \$300 or \$400 in
unemployment, according to the Federal Emergency Management Agency.

The benefit was originally envisioned by President Trump as an extra
\$400 to unemployed workers, with the federal government providing \$300
and the states providing \$100. But states balked at the additional
cost, and now the states' standard unemployment benefit is counted as
their contribution. Workers who are not eligible for at least \$100 in
unemployment will not receive the additional benefit.

So far, only three states,
\href{https://www.courier-journal.com/story/news/local/2020/08/21/kentucky-unemployment-benefits-feds-approve-400-weekly-boost/3407444001/}{Kentucky},
\href{https://apnews.com/c74b1d3f46341434e61f19b4c824aaf2}{Montana} and
\href{https://wvmetronews.com/2020/08/28/w-va-approved-for-federal-enhanced-unemployment-benefit-but-questions-arise-over-how-far-that-goes/}{West
Virginia}, have decided to supply the extra \$100. Vermont's plan to
bring the total payment to \$400
\href{https://labor.vermont.gov/press-release/press-release-vermont-secures-federal-funding-increased-unemployment-benefits-through}{is
awaiting approval} from the state's legislature.

South Dakota's governor has said the state will not apply.

That leaves eight other states that have either not applied or have not
been approved: Delaware, Illinois, Kansas, Nebraska, New Jersey, Nevada,
Wisconsin and South Carolina.
\href{https://news.delaware.gov/2020/08/21/137968/}{Delaware},
\href{https://www.chicagotribune.com/coronavirus/ct-coronavirus-illinois-extra-300-weekly-unemployment-benefits-20200826-jkcrsirtlbbjthck6yzs3tzqwu-story.html}{Illinois},
\href{https://www.njherald.com/news/20200826/nj-applying-for-extra-300-week-in-federal-covid-19-unemployment-benefits}{New
Jersey},
\href{https://lasvegassun.com/news/2020/aug/25/nevada-to-seek-300-weekly-jobless-aid-wont-add-100/}{Nevada},
\href{https://www.wistv.com/2020/08/26/sc-employment-officials-apply-grant-that-could-provide-extra-unemployed/}{South
Carolina} and \href{https://dwd.wisconsin.gov/uiben/lwa/}{Wisconsin} say
they have applied or will apply. Kansas
\href{https://www.kshb.com/news/coronavirus/kansas-applies-for-program-to-bring-400-in-weekly-assistance-to-unemployed-residents}{says
it has applied} and intends to supply the extra \$100 to bring the total
payment to \$400.

Most states won't be able to start paying the benefit until
mid-September or even October. And the payments are expected to last
only four or five weeks.

--- William P. Davis

\hypertarget{site-index}{%
\subsection{Site Index}\label{site-index}}

\hypertarget{site-information-navigation}{%
\subsection{Site Information
Navigation}\label{site-information-navigation}}

\begin{itemize}
\tightlist
\item
  \href{https://help.nytimes3xbfgragh.onion/hc/en-us/articles/115014792127-Copyright-notice}{©~2020~The
  New York Times Company}
\end{itemize}

\begin{itemize}
\tightlist
\item
  \href{https://www.nytco.com/}{NYTCo}
\item
  \href{https://help.nytimes3xbfgragh.onion/hc/en-us/articles/115015385887-Contact-Us}{Contact
  Us}
\item
  \href{https://www.nytco.com/careers/}{Work with us}
\item
  \href{https://nytmediakit.com/}{Advertise}
\item
  \href{http://www.tbrandstudio.com/}{T Brand Studio}
\item
  \href{https://www.nytimes3xbfgragh.onion/privacy/cookie-policy\#how-do-i-manage-trackers}{Your
  Ad Choices}
\item
  \href{https://www.nytimes3xbfgragh.onion/privacy}{Privacy}
\item
  \href{https://help.nytimes3xbfgragh.onion/hc/en-us/articles/115014893428-Terms-of-service}{Terms
  of Service}
\item
  \href{https://help.nytimes3xbfgragh.onion/hc/en-us/articles/115014893968-Terms-of-sale}{Terms
  of Sale}
\item
  \href{https://spiderbites.nytimes3xbfgragh.onion}{Site Map}
\item
  \href{https://help.nytimes3xbfgragh.onion/hc/en-us}{Help}
\item
  \href{https://www.nytimes3xbfgragh.onion/subscription?campaignId=37WXW}{Subscriptions}
\end{itemize}
