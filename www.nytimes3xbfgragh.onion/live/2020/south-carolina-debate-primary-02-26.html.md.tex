Sections

SEARCH

\protect\hyperlink{site-content}{Skip to
content}\protect\hyperlink{site-index}{Skip to site index}

\href{https://myaccount.nytimes3xbfgragh.onion/auth/login?response_type=cookie\&client_id=vi}{}

\href{https://www.nytimes3xbfgragh.onion/section/todayspaper}{Today's
Paper}

\begin{itemize}
\item
  \href{https://www.nytimes3xbfgragh.onion/live/2020/09/08/us/trump-vs-biden?action=click\&pgtype=Article\&state=default\&region=TOP_BANNER\&context=storylines_menu}{Election
  Updates}
\item
  \href{https://www.nytimes3xbfgragh.onion/interactive/2020/us/elections/election-states-biden-trump.html?action=click\&pgtype=Article\&state=default\&region=TOP_BANNER\&context=storylines_menu}{Paths
  to 270}
\item
  \href{https://www.nytimes3xbfgragh.onion/interactive/2020/08/31/us/politics/vote-by-mail-deadlines.html?action=click\&pgtype=Article\&state=default\&region=TOP_BANNER\&context=storylines_menu}{Voting
  by Mail}
\item
  \href{https://www.nytimes3xbfgragh.onion/interactive/2019/us/elections/2020-presidential-election-calendar.html?action=click\&pgtype=Article\&state=default\&region=TOP_BANNER\&context=storylines_menu}{Key
  Dates}
\item
  \href{https://www.nytimes3xbfgragh.onion/newsletters/politics?action=click\&pgtype=Article\&state=default\&region=TOP_BANNER\&context=storylines_menu}{Politics
  Newsletter}
\end{itemize}

\hypertarget{highlights-from-south-carolina-cnn-town-halls-and-clyburns-endorsement}{%
\section{Highlights From South Carolina: CNN Town Halls and Clyburn's
Endorsement}\label{highlights-from-south-carolina-cnn-town-halls-and-clyburns-endorsement}}

Last Updated

Feb. 29, 2020, 11:00 a.m. ET

Feb. 29, 2020, 11:00 a.m. ET

\includegraphics{https://static01.graylady3jvrrxbe.onion/images/2020/02/26/us/politics/live-blog-south-3/live-blog-south-3-articleLarge.jpg?quality=75\&auto=webp\&disable=upscale}

\textbf{\href{https://www.nytimes3xbfgragh.onion/live/2020/south-carolina-primary-02-29}{\emph{Follow
our live coverage of the 2020 South Carolina primary}}\emph{.}}

\begin{itemize}
\item
  There are three days to go before Saturday's South Carolina primary
  and the Democratic candidates are campaigning hard, mostly around
  Charleston and the Lowcountry.
\item
  Representative James E. Clyburn, the highest-ranking African-American
  in Congress and a political powerhouse in South Carolina,
  \href{https://www.nytimes3xbfgragh.onion/live/2020/south-carolina-debate-primary-02-26\#jim-clyburn-endorsement}{threw
  his support} to former Vice President Joseph R. Biden Jr. on
  Wednesday.
\item
  After months of substantial leads in South Carolina polls, Mr. Biden
  is now in the fight of his political life there against Senator Bernie
  Sanders of Vermont, the rising front-runner, and Tom Steyer, the hedge
  fund billionaire whose heavy spending and direct appeals to black
  voters has made him a wild card.
\item
  Six candidates appeared at the Rev. Al Sharpton's National Action
  Network breakfast in North Charleston to make their pitches to the
  black voters who comprise a majority of the Democratic electorate in
  the state. The candidates then fanned out, making stops in Myrtle
  Beach and Georgetown. Senator Elizabeth Warren of Massachusetts held
  get-out-the-vote events with the singer John Legend in Orangeburg and
  Charleston.
\item
  Four candidates were taking part in CNN town halls Wednesday night:
  Former Mayor Michael R. Bloomberg of New York, Mr. Biden, Senator Amy
  Klobuchar of Minnesota, and Ms. Warren.
\end{itemize}

\href{https://www.nytimes3xbfgragh.onion/by/shane-goldmacher}{\includegraphics{https://static01.graylady3jvrrxbe.onion/images/2018/07/27/multimedia/author-shane-goldmacher/author-shane-goldmacher-thumbLarge.png}}

Feb. 27, 2020, 12:20 a.m. ET

Feb. 27, 2020, 12:20 a.m. ET

By \href{https://www.nytimes3xbfgragh.onion/by/shane-goldmacher}{Shane
Goldmacher}

\hypertarget{warren-says-she-would-stay-in-the-race-without-a-delegate-majority}{%
\subsection{\texorpdfstring{\protect\hyperlink{elizabeth-warren-delegates}{Warren
says she would stay in the race without a delegate
majority.}}{Warren says she would stay in the race without a delegate majority.}}\label{warren-says-she-would-stay-in-the-race-without-a-delegate-majority}}

Elizabeth Warren reiterated on Tuesday that she would be open to staying
in the presidential primary even if someone else had amassed an
insurmountable plurality --- but not a majority --- of pledged
delegates. That was the position of every other Democrat on the debate
stage in Las Vegas last week
\href{https://www.nytimes3xbfgragh.onion/2020/02/22/us/politics/democratic-primary-dnc-superdelegates.html}{besides
Senator Bernie Sanders} --- who, as the current front-runner, appears to
be the most likely candidate to wind up in the lead in such a situation.

``You do know that that was Bernie's position in 2016?'' Ms. Warren
said, pushing back on a Sanders supporter in the audience at a CNN town
hall who had asked about the issue. She said Mr. Sanders's ``last play''
in that race was to win over the party's so-called superdelegates.

\begin{quote}
Here is the video of the Warren exchange with the Sanders supporter over
going to the convention a delegate plurality
\href{https://t.co/P73LMdW3mV}{pic.twitter.com/P73LMdW3mV}

--- Shane Goldmacher (@ShaneGoldmacher)
\href{https://twitter.com/ShaneGoldmacher/status/1232885322586951680?ref_src=twsrc\%5Etfw}{February
27, 2020}
\end{quote}

She said she would stick to the rules outlined by the Democratic
National Committee.

``Bernie had a big role in writing the rules. I didn't,'' Warren said.
``I don't see how come you get to change it just because you see an
advantage.''

\href{https://twitter.com/tysonbrody}{Tyson Brody}, an adviser to Mr.
Sanders, responded to Ms. Warren's remarks on Twitter. ``So the plan
isn't to win then,'' he wrote, before deleting the tweet.

Read more

\href{https://www.nytimes3xbfgragh.onion/by/shane-goldmacher}{\includegraphics{https://static01.graylady3jvrrxbe.onion/images/2018/07/27/multimedia/author-shane-goldmacher/author-shane-goldmacher-thumbLarge.png}}

Feb. 26, 2020, 10:53 p.m. ET

Feb. 26, 2020, 10:53 p.m. ET

By \href{https://www.nytimes3xbfgragh.onion/by/shane-goldmacher}{Shane
Goldmacher}

\hypertarget{warren-calls-for-moving-trumps-wall-funds-to-battle-coronavirus}{%
\subsection{\texorpdfstring{\protect\hyperlink{elizabeth-warren-cnn}{Warren
calls for moving Trump's wall funds to battle
coronavirus.}}{Warren calls for moving Trump's wall funds to battle coronavirus.}}\label{warren-calls-for-moving-trumps-wall-funds-to-battle-coronavirus}}

Image

Senator Elizabeth Warren at a get-out-the-vote rally on Wednesday at
South Carolina State University in Orangeburg.Credit...Ruth Fremson/The
New York Times

Senator Elizabeth Warren said during a CNN town hall program Wednesday
night that she would introduce a proposal the next day to provide money
for the government's coronavirus response by stripping allocations from
one of President Trump's top priorities.

``I'm going to be introducing a plan tomorrow to take every dime that
the president is now spending on his racist wall at our southern border
and divert it to work on the coronavirus,'' she said in the opening
minutes of her hourlong appearance.

\begin{quote}
"'I'm going to be introducing a plan tomorrow to take every dime that
the President is now spending on his racist wall at our southern border
and divert it to work on the coronavirus," Elizabeth Warren says
\href{https://twitter.com/hashtag/CNNTownHall?src=hash\&ref_src=twsrc\%5Etfw}{\#CNNTownHall}
\href{https://t.co/I3WTbNlBzV}{pic.twitter.com/I3WTbNlBzV}

--- CNN Politics (@CNNPolitics)
\href{https://twitter.com/CNNPolitics/status/1232865628178067458?ref_src=twsrc\%5Etfw}{February
27, 2020}
\end{quote}

She went on to criticize the choice of Vice President Mike Pence by Mr.
Trump to coordinate the government's response, citing his response as
governor of Indiana to a public health crisis involving an H.I.V.
outbreak. For about two months in 2015,
\href{https://www.nytimes3xbfgragh.onion/2016/08/08/us/politics/mike-pence-needle-exchanges-indiana.html}{Mr.
Pence opposed efforts to distribute clean needles} to slow the spread of
the virus.

``We also need someone in the White House who is coordinating all of the
working and all of the messaging and all of the information,'' Ms.
Warren said. ``And we need someone who is not actively disqualified from
doing that the way the vice president is.''

Later in the program, Ms. Warren reiterated that she would be open to
staying in the presidential primary even if someone else had amassed a
plurality --- but not a majority --- of the delegates. That was the
position of every other Democrat on the debate stage in Las Vegas last
week
\href{https://www.nytimes3xbfgragh.onion/2020/02/22/us/politics/democratic-primary-dnc-superdelegates.html}{besides
Senator Bernie Sanders} --- who, as the current front-runner, appears to
be the most likely candidate to wind up in the lead in such a situation.

``You do know that that was Bernie's position in 2016?'' Ms. Warren
pushed back on a Sanders supporter in the audience who had asked about
the issue. She said Mr. Sanders's ``last play'' in that race was to win
over the party's so-called superdelegates.

\begin{quote}
Here is the video of the Warren exchange with the Sanders supporter over
going to the convention a delegate plurality
\href{https://t.co/P73LMdW3mV}{pic.twitter.com/P73LMdW3mV}

--- Shane Goldmacher (@ShaneGoldmacher)
\href{https://twitter.com/ShaneGoldmacher/status/1232885322586951680?ref_src=twsrc\%5Etfw}{February
27, 2020}
\end{quote}

She said she would stick to the rules outlined by the Democratic
National Committee.

``Bernie had a big role in writing the rules. I didn't,'' Warren said.
``I don't see how come you get to change it just because you see an
advantage.''

\href{https://twitter.com/tysonbrody}{Tyson Brody}, an adviser to Mr.
Sanders, responded to Ms. Warren's remarks on Twitter. ``So the plan
isn't to win then,'' he wrote, before deleting the tweet.

Read more

\hypertarget{advertisement}{%
\subsubsection{Advertisement}\label{advertisement}}

\protect\hyperlink{after-dfp-ad-mid1}{Continue reading the main story}

\href{https://www.nytimes3xbfgragh.onion/by/nick-corasaniti}{\includegraphics{https://static01.graylady3jvrrxbe.onion/images/2018/06/13/multimedia/author-nick-corasaniti/author-nick-corasaniti-thumbLarge-v2.png}}

Feb. 26, 2020, 10:41 p.m. ET

Feb. 26, 2020, 10:41 p.m. ET

By \href{https://www.nytimes3xbfgragh.onion/by/nick-corasaniti}{Nick
Corasaniti}

\hypertarget{klobuchar-says-her-debate-targets-were-sanders-and-warren}{%
\subsection{\texorpdfstring{\protect\hyperlink{amy-klobuchar-cnn}{Klobuchar
says her debate targets were Sanders and
Warren.}}{Klobuchar says her debate targets were Sanders and Warren.}}\label{klobuchar-says-her-debate-targets-were-sanders-and-warren}}

Image

Ms. Klobuchar addressed voters during a campaign event at Founders Hall
in Charleston on Wednesday.Credit...Travis Dove for The New York Times

Twenty-four hours after the chaotic Democratic debate in South Carolina
--- in which it seemed at times that everyone was shouting at everyone
else --- Senator Amy Klobuchar told a CNN audience who her two targets
were: Senators Bernie Sanders and Elizabeth Warren.

``I felt the one differentiation I wanted to make on that debate was the
difference between me as a leader of our party and my colleague Senator
Sanders, and actually Senator Warren,'' Ms. Klobuchar said Wednesday,
making a case that she would have the longest coattails to help
down-ballot Democrats running for Congress. ``I think that we really
need someone that can bring people with her and lead, and I am the only
one with the track record up there.''

The down-ballot argument is one Ms. Klobuchar makes during nearly every
stump speech, town hall event, campaign rally or television appearance.
But she had not mentioned both Ms. Warren and Mr. Sanders as the targets
for her criticism before.

In her hourlong appearance on CNN, broadcast from Charleston, Ms.
Klobuchar faced questions about her presidential campaign's viability,
including her ability to win support from black voters after just 2
percent of black voters caucused for her in Nevada, according to
entrance polls. She said she would push to protect voting rights and
create economic opportunity.

She was also asked about her ability to rally the liberal wing of the
party after making the contrasts with Mr. Sanders and Ms. Warren.
``Elizabeth and Bernie and I are all in leadership together --- I bet
you wish you were in those meetings --- in the U.S. Senate,'' she said.
``We have worked together on many, many issues. I admire both of them.''

Like the other candidates who appeared on CNN on Wednesday, Ms.
Klobuchar was asked at the beginning of her town hall about the
coronavirus, and particularly the announcement that
\href{https://www.nytimes3xbfgragh.onion/2020/02/26/us/politics/trump-coronavirus-cdc.html}{President
Trump had tapped Vice President Mike Pence} to oversee the public health
response.

``I would think usually you might put a medical professional in
charge,'' Ms. Klobuchar said, before adding that perhaps entrusting the
response to the vice president could help elevate the urgency.
(President Barack Obama
\href{https://www.nytimes3xbfgragh.onion/2014/10/18/us/ebola-cruise-ship-dallas.html}{named
Ron Klain}, a Democratic operative, to coordinate the government's
response to the Ebola virus in 2014.)

Returning to the subject of the debate, Ms. Klobuchar told the story of
the viral photo of her standing in between a shouting Tom Steyer and an
equally animated Joseph R. Biden Jr.

\begin{quote}
Klobuchar talks about this famous photo, jokes that Steyer "was so
heated he was moving into my little area." Said she was worried his
flailing arms might knock her over.

Joked: "I thought to myself, well, if I fall over, and i'm hit, at least
Steyer has deep pockets."
\href{https://t.co/PaF5aVwPP2}{pic.twitter.com/PaF5aVwPP2}

--- Nick Corasaniti (@NYTnickc)
\href{https://twitter.com/NYTnickc/status/1232857064965320705?ref_src=twsrc\%5Etfw}{February
27, 2020}
\end{quote}

``That moment with those two, they were going at it, and what was
somewhat amusing about it was Tom Steyer was so heated he was moving
into my little area,'' Ms. Klobuchar said. She said that she had a
little stool for her 5-foot-4 frame on the debate stage, and that she
worried briefly that Mr. Steyer's gesticulating might knock her off her
perch.

``I thought to myself, `Well, if I fall over, and I'm hit, at least
Steyer has deep pockets,''' she joked. ``I can get something out of
this.''

Read more

\href{https://www.nytimes3xbfgragh.onion/by/katie-glueck}{\includegraphics{https://static01.graylady3jvrrxbe.onion/images/2020/01/29/reader-center/author-katie-glueck/author-katie-glueck-thumbLarge.png}}

Feb. 26, 2020, 10:19 p.m. ET

Feb. 26, 2020, 10:19 p.m. ET

By \href{https://www.nytimes3xbfgragh.onion/by/katie-glueck}{Katie
Glueck}

\hypertarget{biden-hits-sanders-on-guns-and-democratic-socialism}{%
\subsection{\texorpdfstring{\protect\hyperlink{joe-biden-cnn-town-hall}{Biden
hits Sanders on guns and democratic
socialism.}}{Biden hits Sanders on guns and democratic socialism.}}\label{biden-hits-sanders-on-guns-and-democratic-socialism}}

Image

Mr. Biden at a community event in Georgetown.Credit...Maddie McGarvey
for The New York Times

CHARLESTON --- Joseph R. Biden Jr. came out swinging at one of his chief
rivals, Bernie Sanders, at a televised town hall event on Wednesday,
swiping at him over everything from his record on guns to the damage
that he suggested Mr. Sanders would do to Democrats running in
down-ballot races.

Mr. Sanders identifies as a democratic socialist, and Mr. Biden was
asked on the CNN program if he ``would support a socialist at the top of
the ticket.''

``We have moved in a direction that in fact, the progressive --- now
`progressive' means `Bernie,''' said Mr. Biden, who is a relative
centrist but added that throughout his career he had been considered a
``liberal liberal.''

``It means democratic socialism or whatever the phrase is,'' Mr. Biden
said. ``I think Bernie is a decent, honorable man who means what he
says.''

But, he suggested, Mr. Sanders is also a candidate who would endanger
the Democratic House majority and jeopardize more seats in the Senate
should he be the party's nominee.

His remarks come as Mr. Sanders has appeared to gain ground here in
South Carolina, a state Mr. Biden has considered his electoral firewall,
after the Vermont senator soundly defeated Mr. Biden in the first three
nominating contests.

``It's not a criticism of him as a man, it's a criticism of whether or
not you think you're going to be able to help elect a Democratic senator
here against Lindsey Graham, which I'm going to help do,'' he said of
the South Carolina Senate race this year.

His remarks about Mr. Sanders on Wednesday amounted to some of his most
sustained criticism of his rival to date, part of an energetic
performance in which Mr. Biden was by turns punchy --- when discussing
politics --- and emotional, such as when he discussed grief or cancer.

He questioned whether Democratic Senate candidates in a red-leaning
states would believe it was in their political interest to have ``a
self-proclaimed socialist at the top of the ticket.''

And as he often does, Mr. Biden expressed confidence that if he is the
presidential nominee, he would help Democrats who are running in a wide
range of competitive contests. He noted his efforts to do just that by
campaigning in districts and states across the country in 2018, when his
party recaptured the House of Representatives.

``Did anybody ask Bernie to come in?'' he said. ``It doesn't mean he's a
bad guy. It means it's going to be hard holding on to the United States
Congress and the United States Senate.''

Mr. Sanders did, in fact, campaign for many candidates in the midterm
elections.

Mr. Biden has repeatedly said he will support whoever the Democratic
nominee is, though he did not reiterate that point in his answers on
Wednesday.

He also repeated criticisms he has been making about Mr. Sanders for
weeks over his record on gun control. Previously, Mr. Biden has
acknowledged that Mr. Sanders has changed his views on gun control ---
but on Wednesday he said Mr. Sanders's past votes on the subject remain
important.

Mr. Sanders voted in 2005 for a law that gave immunity to gun
manufacturers in wrongful death lawsuits. Confronted with that record at
the debate on Tuesday night,
\href{https://www.nytimes3xbfgragh.onion/2020/02/25/us/politics/bernie-sanders-brady-bill-guns.html}{he
called it a ``bad vote.''}

Mr. Biden pressed the point on Wednesday. ``He's gone after every
corporation in the world --- I don't disagree on all of it with him,''
Mr. Biden said. ``But I have not seen him go after the gun
manufacturers.''

\begin{quote}
Joe Biden on Bernie Sanders' gun legislation record: "He goes after
every corporation in the world. But I have not seen him go after the gun
manufacturers"
\href{https://twitter.com/hashtag/CNNTownHall?src=hash\&ref_src=twsrc\%5Etfw}{\#CNNTownHall}
\href{https://t.co/TDaExUd69k}{pic.twitter.com/TDaExUd69k}

--- CNN Politics (@CNNPolitics)
\href{https://twitter.com/CNNPolitics/status/1232837394132611072?ref_src=twsrc\%5Etfw}{February
27, 2020}
\end{quote}

He went on to suggest that in previous elections, Mr. Sanders made
political calculations about how much to discuss gun control.

Mr. Biden was also pushed to defend some of his own record, including
his support for the 1994 crime bill. He repeated a claim he has made
before --- that the crime bill ``did not put more people in jail like
it's argued.'' Many experts have linked the measure to an acceleration
in mass incarceration.

Asked if he would support the measure today, he said that times have
changed --- but that such a standard should not apply to Mr. Sanders's
votes on guns.

``It was the right bill then,'' he said. ``Unlike voting to give
exemptions to the gun manufacturers, never a right vote under any
circumstances. Being against the Brady bill was never right under any
circumstances. It was right at the time.''

(Read more about Mr. Biden's record on the crime bill and criminal
justice
\href{https://www.nytimes3xbfgragh.onion/2019/06/25/us/joe-biden-crime-laws.html}{here},
\href{https://www.nytimes3xbfgragh.onion/2019/06/27/us/politics/joe-biden-crime-bill-mass-incarceration.html}{here}
and
\href{https://www.nytimes3xbfgragh.onion/2019/07/23/us/politics/biden-criminal-justice.html}{here}.)

Read more

\href{https://www.nytimes3xbfgragh.onion/by/sydney-ember}{\includegraphics{https://static01.graylady3jvrrxbe.onion/images/2018/06/12/multimedia/author-sydney-ember/author-sydney-ember-thumbLarge.png}}

Feb. 26, 2020, 10:00 p.m. ET

Feb. 26, 2020, 10:00 p.m. ET

By \href{https://www.nytimes3xbfgragh.onion/by/sydney-ember}{Sydney
Ember}

\hypertarget{sanders-bashes-the-media-and-defends-press-freedoms}{%
\subsection{\texorpdfstring{\protect\hyperlink{bernie-sanders-media}{Sanders
bashes the media (and defends press
freedoms).}}{Sanders bashes the media (and defends press freedoms).}}\label{sanders-bashes-the-media-and-defends-press-freedoms}}

GOLDSBORO, N.C. --- Even as Bernie Sanders came to the defense of The
New York Times,
\href{https://www.nytimes3xbfgragh.onion/2020/02/26/business/media/trump-new-york-times-lawsuit.html}{which
was sued by President Trump's re-election campaign} on Wednesday, he
spent much of the day bashing the news media's political coverage.

Pointing to the reporters who were covering him at a rally in Myrtle
Beach, S.C., he said the news media, ``which determines what you see and
read,'' was owned by the ``corporate media.''

And when asked at a forum in Goldsboro why poverty was not a bigger
focus in the presidential campaign, Mr. Sanders blamed the country's
news organizations.

``Do you know how many times the media has asked me, what am I going to
do about poverty?'' he said. ``I don't think they've ever asked. I'm
followed by all this media. They will not ask me about income and wealth
inequality.''

Instead, he said, ``They will ask me about some dumb statement I may
have made the other day or something or what do I think of this or
that.''

Still, Mr. Sanders defended The Times in a statement Wednesday
addressing the Trump campaign's libel lawsuit, which alleged that an
Op-Ed in the newspaper in March had falsely asserted there was a ``quid
pro quo'' between Mr. Trump's 2016 campaign and Russia.

Mr. Trump is ``trying to dismantle the right to a free press in the
First Amendment by suing The New York Times for publishing an opinion
column about his dangerous relationship with Russia,'' Mr. Sanders said
in the statement.

And at one point while campaigning on Wednesday, Mr. Sanders seemed to
stop himself. ``They are good people,'' Mr. Sanders said, referring to
journalists and drawing a contrast with Mr. Trump. ``These people try
hard. They are not `enemies of the people.'''

But he quickly resumed his diatribe.

``They have been educated in their jobs to think that politics is, if I
attack you or you attack me, that's a big deal or if I slip on a banana
peel, that becomes a big story.''

Read more

\hypertarget{advertisement-1}{%
\subsubsection{Advertisement}\label{advertisement-1}}

\protect\hyperlink{after-dfp-ad-mid2}{Continue reading the main story}

\href{https://www.nytimes3xbfgragh.onion/by/rebecca-r-ruiz}{\includegraphics{https://static01.graylady3jvrrxbe.onion/images/2018/06/12/multimedia/author-rebecca-r-ruiz/author-rebecca-r-ruiz-thumbLarge.png}}

Feb. 26, 2020, 8:21 p.m. ET

Feb. 26, 2020, 8:21 p.m. ET

By \href{https://www.nytimes3xbfgragh.onion/by/rebecca-r-ruiz}{Rebecca
R. Ruiz}

\hypertarget{bloomberg-seems-more-at-ease-in-a-town-hall-format}{%
\subsection{\texorpdfstring{\protect\hyperlink{michael-bloomberg-cnn-town-hall}{Bloomberg
seems more at ease in a town hall
format.}}{Bloomberg seems more at ease in a town hall format.}}\label{bloomberg-seems-more-at-ease-in-a-town-hall-format}}

Image

Former Mayor Michael R. Bloomberg of New York during the Democratic
debate in Charleston on Tuesday.Credit...Erin Schaff/The New York Times

Michael R. Bloomberg watched the beginning of
\href{https://www.nytimes3xbfgragh.onion/2020/02/26/us/politics/trump-coronavirus-cdc.html}{President
Trump's news conference on the coronavirus outbreak} backstage at a CNN
``town hall'' in Charleston, S.C., on Wednesday.

Minutes later, addressing the audience in the first such televised event
of his presidential campaign, he took aim at Mr. Trump for both his
handling of the virus threat and his reaction to
\href{https://www.nytimes3xbfgragh.onion/2020/02/26/us/milwaukee-shooting-miller-coors.html}{a
mass shooting in Milwaukee} on Wednesday.

``The bottom line is, we are not ready for this kind of thing,'' Mr.
Bloomberg said of the public health threat. ``The president is not ready
for this kind of thing.'' (Mr. Trump on Wednesday named Vice President
Mike Pence to oversee the government's response to the coronavirus, even
as he played down the threat of a widespread domestic outbreak.)

Mr. Bloomberg appeared to relish the town hall format, seizing the
chance to speak in greater detail about some of the causes he has put
his vast personal fortune toward --- without interjections or challenges
from his opponents.

``They talk over each other again and again. I found that difficult,''
Mr. Bloomberg said of the primary debates. ``I didn't grow up where you
step on people, and that's what they do all the time.''

Appearing more at ease than he had in his recent debate appearances, Mr.
Bloomberg invoked his experience presiding over New York City in the
aftermath of Sept. 11, Hurricane Sandy and the swine flu outbreak.

``Pulling people together, making them feel that they're part of the
solution, is what management is all about --- it's what I do,'' he said.
``New York is a microcosm of the country, and we've gone through a lot
of this stuff already.''

The multibillionaire addressed a range of issues, fielding questions on
climate change, gun control, and the controversial stop-and-frisk
policing policy during his tenure in New York, for which he again
apologized. ``I made a mistake,'' he said.

He criticized the Democratic race's front-runner, Senator Bernie Sanders
of Vermont, on gun control, suggesting that Mr. Sanders, who voted in
favor of a 2005 bill to shield gun manufacturers from liability
lawsuits, was beholden to the gun lobby. But he also reiterated that he
planned to throw financial support and his ``main campaign offices''
behind whoever the Democratic nominee might be.

``I've always thought it's ridiculous to say, `I will always support the
candidate no matter who it is', because that's how we got Donald
Trump,'' Mr. Bloomberg said. ``Having said that, it's easy for me to say
I'll support any of the Democratic candidates,'' he added. ``Because the
alternative is Donald Trump, and that we don't want that.''

\begin{quote}
"It's easy for me to make the commitment that I will support any of the
Democratic candidates if they get the nomination," says Michael
Bloomberg. "But it's easy to do it because the alternative is Donald
Trump and that we don't want"
\href{https://twitter.com/hashtag/CNNTownHall?src=hash\&ref_src=twsrc\%5Etfw}{\#CNNTownHall}
\url{https://t.co/wTLdt4cmJk}
\href{https://t.co/21adslB1ZD}{pic.twitter.com/21adslB1ZD}

--- CNN Politics (@CNNPolitics)
\href{https://twitter.com/CNNPolitics/status/1232830906479599617?ref_src=twsrc\%5Etfw}{February
27, 2020}
\end{quote}

Asked about sexual harassment in the workplace, Mr. Bloomberg denounced
forced arbitration and said his company had never required it. He went
on to praise the \#MeToo movement, though in 2018 he
\href{https://www.nytimes3xbfgragh.onion/2018/09/17/us/politics/bloomberg-president-2020-democrat.html}{cast
doubt} on harassment allegations against Charlie Rose, the television
anchor who broadcast his talk show from the Bloomberg L.P. offices.

``Most of the nondisclosure agreements every company has have to do with
severance,'' Mr. Bloomberg said, repeating that his lawyers had released
three women from agreements they had reached with his company in past
years, adding that he didn't know if they would speak out, and saying
that the company would never use such agreements again.

``We've changed our policy so we no longer use nondisclosure agreements
in the company, anywhere in the world, going forward,'' he said.

Read more

\href{https://www.nytimes3xbfgragh.onion/by/taylor-lorenz}{\includegraphics{https://static01.graylady3jvrrxbe.onion/images/icons/t_logo_291_black.png}}\href{https://www.nytimes3xbfgragh.onion/by/sheera-frenkel}{\includegraphics{https://static01.graylady3jvrrxbe.onion/images/2018/06/14/multimedia/author-sheera-frenkel/author-sheera-frenkel-thumbLarge.png}}

Feb. 26, 2020, 6:29 p.m. ET

Feb. 26, 2020, 6:29 p.m. ET

By \href{https://www.nytimes3xbfgragh.onion/by/taylor-lorenz}{Taylor
Lorenz} and
\href{https://www.nytimes3xbfgragh.onion/by/sheera-frenkel}{Sheera
Frenkel}

\hypertarget{bloombergs-sponsored-content-on-meme-accounts-pushes-instagrams-boundaries}{%
\subsection{\texorpdfstring{\protect\hyperlink{bloombergs-sponsored-content-on-meme-accounts-pushes-instagrams-boundaries}{Bloomberg's
sponsored content on meme accounts pushes Instagram's
boundaries.}}{Bloomberg's sponsored content on meme accounts pushes Instagram's boundaries.}}\label{bloombergs-sponsored-content-on-meme-accounts-pushes-instagrams-boundaries}}

A fresh batch of Bloomberg campaign ads flooded users Instagram feeds on
Wednesday when at least five high profile meme accounts posted more
sponsored content on behalf of the campaign. The ads
\href{https://twitter.com/TaylorLorenz/status/1232717081667391493}{took
the form of} fake ``relatable'' tweets and edited videos and were posted
on accounts including World Star Hip Hop,
\href{https://www.instagram.com/funnyhoodvidz/?hl=en}{Funny Hood Vidz},
\href{https://www.instagram.com/bangerbuddy/}{Banger Buddy},
\href{https://www.instagram.com/nugget/}{Nugget} and
\href{https://www.instagram.com/wasted/}{Wasted}.

None of the posts used Instagram's official system for disclosing
branded content.
\href{https://www.nytimes3xbfgragh.onion/2020/02/14/style/facebook-political-memes.html}{The
company has said that} all creators posting sponsored content on behalf
of presidential campaigns must use the official branded content tool,
but not every meme account is currently eligible to use it. Many of the
accounts the Bloomberg campaign has advertised on are also set to
private, which means that followers must request to see the accounts and
be approved by the account owners.

For weeks, Facebook has been scrambling to respond to the Bloomberg
campaign's new social media tactics.

\href{https://www.nytimes3xbfgragh.onion/by/maggie-astor}{\includegraphics{https://static01.graylady3jvrrxbe.onion/images/2018/07/18/multimedia/author-maggie-astor/author-maggie-astor-thumbLarge.png}}

Feb. 26, 2020, 5:51 p.m. ET

Feb. 26, 2020, 5:51 p.m. ET

By \href{https://www.nytimes3xbfgragh.onion/by/maggie-astor}{Maggie
Astor}

\hypertarget{obama-denounces-a-republican-super-pacs-deceptive-anti-biden-ad}{%
\subsection{\texorpdfstring{\protect\hyperlink{obama-biden-ad}{Obama
denounces a Republican super PAC's deceptive anti-Biden
ad.}}{Obama denounces a Republican super PAC's deceptive anti-Biden ad.}}\label{obama-denounces-a-republican-super-pacs-deceptive-anti-biden-ad}}

Former President Barack Obama is calling on television networks to stop
airing a pro-Trump super PAC's deceptive ad against Joseph R. Biden Jr.

The ad, from the Committee to Defend the President, uses audio of Mr.
Obama reading an excerpt from his memoir ``Dreams From My Father.'' In
that passage, a barber describes ``black committeemen'' in Chicago
seeking black voters' support while not actually helping them. The text
on the screen mentions Mr. Biden's support for the 1994 crime bill and
his past work with segregationist senators.

The implication is that Mr. Obama is denouncing Mr. Biden. But in
reality, the passage has nothing to do with Mr. Biden and is in the
words of a person Mr. Obama met, not Mr. Obama himself.

Lawyers for Mr. Obama sent
\href{https://int.graylady3jvrrxbe.onion/data/documenthelper/6788-obama-letter-on-anti-biden-ad/7ec054e132c6eadffa88/optimized/full.pdf\#page=1}{a
cease-and-desist letter} to the super PAC on Wednesday.

Mr. Obama's spokeswoman, Katie Hill, said in a statement:

\begin{quote}
President Obama has several friends in this race, including, of course,
his own esteemed vice president. He has said he has no plans to endorse
in the primary because he believes that in order for Democrats to be
successful this fall, voters must choose their nominee. But this
despicable ad is straight out of the Republican disinformation playbook,
and it's clearly designed to suppress turnout among minority voters in
South Carolina by taking President Obama's voice out of context and
twisting his words to mislead viewers. In the interest of truth in
advertising, we are calling on TV stations to take this ad down and stop
playing into the hands of bad actors who seek to sow division and
confusion among the electorate.
\end{quote}

Andrew Bates, a spokesman for Mr. Biden, cast the ad as evidence that
the president is afraid of running against him --- something Mr. Biden
has long argued.

``Donald Trump and his allies are absolutely terrified that Joe Biden
will defeat him in November,'' Mr. Bates said. ``This latest
intervention in the Democratic primary is one of the most desperate yet,
a despicable torrent of misinformation by the president's lackeys.''

Read more

\hypertarget{advertisement-2}{%
\subsubsection{Advertisement}\label{advertisement-2}}

\protect\hyperlink{after-dfp-ad-mid3}{Continue reading the main story}

\includegraphics{https://static01.graylady3jvrrxbe.onion/images/icons/t_logo_291_black.png}

Feb. 26, 2020, 4:09 p.m. ET

Feb. 26, 2020, 4:09 p.m. ET

The New York Times

\hypertarget{elizabeth-warren-through-a-different-lens}{%
\subsection{\texorpdfstring{\protect\hyperlink{elizabeth-warren-john-legend}{Elizabeth
Warren through a different
lens.}}{Elizabeth Warren through a different lens.}}\label{elizabeth-warren-through-a-different-lens}}

Image

Elizabeth Warren held a ``get out the vote'' rally featuring John Legend
at South Carolina State University, a historically black university in
Orangeburg.Credit...Ruth Fremson/The New York Times

\href{https://www.nytimes3xbfgragh.onion/by/katie-glueck}{\includegraphics{https://static01.graylady3jvrrxbe.onion/images/2020/01/29/reader-center/author-katie-glueck/author-katie-glueck-thumbLarge.png}}

Feb. 26, 2020, 3:19 p.m. ET

Feb. 26, 2020, 3:19 p.m. ET

By \href{https://www.nytimes3xbfgragh.onion/by/katie-glueck}{Katie
Glueck}

\hypertarget{biden-lashes-sanders-on-health-care-and-guns}{%
\subsection{\texorpdfstring{\protect\hyperlink{joe-biden-bernie-sanders-guns}{Biden
lashes Sanders on health care and
guns.}}{Biden lashes Sanders on health care and guns.}}\label{biden-lashes-sanders-on-health-care-and-guns}}

GEORGETOWN --- Joseph R. Biden Jr. laced into Bernie Sanders's health
care platform on Wednesday afternoon with particular vigor, dismissing
the senator's plan to pay for his sweeping ``Medicare for all'' proposal
and casting it as politically untenable and impractical.

``As my mother would say, God bless her soul, God bless Bernie,'' Mr.
Biden said, claiming that while Mr. Sanders
\href{https://www.nytimes3xbfgragh.onion/2020/02/24/us/politics/bernie-sanders-medicare-for-all.html}{had
finally shared more details about how he would fund Medicare for all},
he had so far failed to be straightforward about the tax implications
for working people.

``I think we ought to level with you all,'' Mr. Biden said. The former
vice president wasn't ``picking on'' Mr. Sanders or other supporters of
Medicare for all, he said, but he believes there should be ``a little
bit of honesty'' about ``what things are going to cost, who's going to
pay for it.''

As he did on the debate stage on Tuesday, Mr. Biden also criticized Mr.
Sanders
\href{https://www.nytimes3xbfgragh.onion/2020/02/25/us/politics/bernie-sanders-brady-bill-guns.html}{for
his past votes on issues like background checks for gun purchases}.

``He says it's because he's from Vermont,'' Mr. Biden said. ``I'm from
Delaware, one of the largest gun-owning states in the nation. You've got
to decide what you're for.''

The sharpened criticism of Mr. Sanders comes after the senator
outperformed Mr. Biden in each of the first three nominating contests,
and as he poses a potentially significant threat to his standing in
South Carolina, a state Mr. Biden has said he expects to win.

Read more

\href{https://www.nytimes3xbfgragh.onion/by/astead-w-herndon}{\includegraphics{https://static01.graylady3jvrrxbe.onion/images/2018/09/14/us/author-head-astead/author-head-astead-thumbLarge-v2.png}}

Feb. 26, 2020, 2:37 p.m. ET

Feb. 26, 2020, 2:37 p.m. ET

By \href{https://www.nytimes3xbfgragh.onion/by/astead-w-herndon}{Astead
W. Herndon}

\hypertarget{john-legend-lends-elizabeth-warren-star-power-in-south-carolina}{%
\subsection{\texorpdfstring{\protect\hyperlink{john-legend-lends-elizabeth-warren-star-power-in-south-carolina}{John
Legend lends Elizabeth Warren star power in South
Carolina.}}{John Legend lends Elizabeth Warren star power in South Carolina.}}\label{john-legend-lends-elizabeth-warren-star-power-in-south-carolina}}

Image

John Legend with Elizabeth Warren at a ``get out the vote'' rally in
South Carolina State University in Orangeburg.Credit...Ruth Fremson/The
New York Times

ORANGEBURG --- The Grammy Award-winning recording artist John Legend
rallied for Elizabeth Warren at South Carolina State University on
Wednesday, lending his star power to a candidate who needs a spark
before Saturday's primary.

Mr. Legend, whose real name is John Stephens, gave an impassioned speech
in support of Ms. Warren, saying he did not intend to publicly support a
presidential candidate until he was so moved by Ms. Warren's candidacy.

``The reason she ran is to give this democracy back to its rightful
owners --- that's you,'' he said. To the large group of college students
at the historically black university, Mr. Legend highlighted Ms.
Warren's plans to correct racial inequities.

However, in a slight nod to the challenges Ms. Warren faces in the
state, he implored the students to change the course of her candidacy.
``Elizabeth Warren needs you,'' he said. ``This country needs you.
Everyone is watching South Carolina. Everyone is asking what will South
Carolina do.''

``You have the power to send a message that will resound across this
nation,'' he added.

Mr. Legend performed immediately after Ms. Warren's speech, bringing the
crowd to its feet. He offered two of his biggest songs: ``Ordinary
People'' and ``All of Me.''

Read more

\hypertarget{advertisement-3}{%
\subsubsection{Advertisement}\label{advertisement-3}}

\protect\hyperlink{after-dfp-ad-mid4}{Continue reading the main story}

\href{https://www.nytimes3xbfgragh.onion/by/matt-stevens}{\includegraphics{https://static01.graylady3jvrrxbe.onion/images/2019/04/03/multimedia/author-matt-stevens/author-matt-stevens-thumbLarge.png}}

Feb. 26, 2020, 2:20 p.m. ET

Feb. 26, 2020, 2:20 p.m. ET

By \href{https://www.nytimes3xbfgragh.onion/by/matt-stevens}{Matt
Stevens}

\hypertarget{bloomberg-continues-to-attack-trump-about-coronavirus-response-in-new-ad}{%
\subsection{\texorpdfstring{\protect\hyperlink{bloomberg-continues-to-attack-trump-about-coronavirus-response-in-new-ad}{Bloomberg
continues to attack Trump about coronavirus response in new
ad.}}{Bloomberg continues to attack Trump about coronavirus response in new ad.}}\label{bloomberg-continues-to-attack-trump-about-coronavirus-response-in-new-ad}}

Michael R. Bloomberg continued to attack President Trump on Wednesday
over his handling of the coronavirus and the widespread concern it has
generated, releasing a new ad that painted the former mayor as a leader
much better equipped to handle the outbreak.

In the
\href{https://www.youtube.com/watch?v=gTaE_JlCev8\&feature=youtu.be}{30-second
ad, titled ``Pandemic,''} a narrator says that ``managing a crisis is
what Mike Bloomberg does,'' and cites the actions he took to rebuild New
York after 9/11. It also highlights the stock market plunge spurred by
fears about the spread of coronavirus and superimposes a picture of Mr.
Trump next to
\href{https://www.theguardian.com/world/2020/jan/31/us-coronavirus-budget-cuts-trump-underprepared}{a
headline from The Guardian} that reads: ``U.S. underprepared for
coronavirus due to Trump cuts, say health experts.''

The ad will soon begin airing nationally on cable, broadcast and digital
platforms, Mr. Bloomberg's campaign said.

Ahead of Tuesday night's debate, Mr. Bloomberg
\href{https://www.mikebloomberg.com/news/mike-bloomberg-statement-on-coronavirus-outbreak}{issued
a statement} accusing Mr. Trump of having ``buried his head in the sand
as people around the world were dying,'' and having ``responded only
after seeing the TV coverage of yesterday's stock market plunge.'' He
and other candidates then
\href{https://www.nytimes3xbfgragh.onion/2020/02/25/us/politics/trump-coronavirus.html}{continued
to criticize Mr. Trump's response} to the health crisis at the debate,
raising concerns about the way the Trump administration has funded the
Centers for Disease Control and Prevention.

For his part, Mr. Trump
\href{https://twitter.com/realDonaldTrump/status/1232652371832004608}{tweeted
on Wednesday} that he had scheduled a news conference for 6 p.m. to
discuss the topic
\href{https://www.nytimes3xbfgragh.onion/2020/02/26/us/politics/trump-coronavirus-cdc.html?action=click\&module=Top\%20Stories\&pgtype=Homepage}{while
also criticizing the news media} for what he said was an attempt to make
the coronavirus ``look as bad as possible.''

Read more

\href{https://www.nytimes3xbfgragh.onion/by/matt-stevens}{\includegraphics{https://static01.graylady3jvrrxbe.onion/images/2019/04/03/multimedia/author-matt-stevens/author-matt-stevens-thumbLarge.png}}

Feb. 26, 2020, 1:28 p.m. ET

Feb. 26, 2020, 1:28 p.m. ET

By \href{https://www.nytimes3xbfgragh.onion/by/matt-stevens}{Matt
Stevens}

\hypertarget{warren-is-criticized-again-over-past-ancestry-claims-and-writes-a-12-page-response}{%
\subsection{\texorpdfstring{\protect\hyperlink{warren-is-criticized-again-over-past-ancestry-claims-and-writes-a-12-page-response}{Warren
is criticized again over past ancestry claims (and writes a 12-page
response).}}{Warren is criticized again over past ancestry claims (and writes a 12-page response).}}\label{warren-is-criticized-again-over-past-ancestry-claims-and-writes-a-12-page-response}}

A group of more than 200 Cherokees and other Native Americans published
an open letter to Elizabeth Warren on Wednesday morning reiterating
concerns about the way she once presented her family history and about
her widely criticized decision
\href{https://www.nytimes3xbfgragh.onion/2018/12/06/us/politics/elizabeth-warren-dna-test-2020.html}{to
take a DNA test to prove Native American ancestry}.

In the letter, which was
\href{https://www.latimes.com/politics/story/2020-02-26/elizabeth-warren-again-is-pressed-on-past-claims-of-native-american-heritage}{first
reported by The Los Angeles Times}, the group of concerned citizens of
the Cherokee Nation, United Keetoowah Band of Cherokee Indians and the
Eastern Band of Cherokee Indians said Ms. Warren had ``perpetuated a
dangerous misunderstanding of tribal sovereignty.'' The group also
asserted that more broadly, Ms. Warren had ``set a harmful example'' at
a time when white people have
\href{https://www.latimes.com/local/lanow/la-na-cherokee-minority-contracts-20190626-story.html}{routinely
sought to falsely claim Native identity to advantage themselves}.

Though they acknowledged that Ms. Warren had
\href{https://www.nytimes3xbfgragh.onion/2019/02/01/us/politics/elizabeth-warren-cherokee-dna.html}{apologized}
for the
\href{https://www.nytimes3xbfgragh.onion/2019/08/19/us/politics/elizabeth-warren-native-american.html}{harm
she had caused}, they asked her to publicly clarify various aspects of
her family story.

Ms. Warren quickly responded with a heavily footnoted 12-page letter
bearing her signature, in which she wrote bluntly: ``I am not a person
of color; I am a white woman, and that is how I identify. In addition, I
am not a tribal citizen. Tribal Nations --- and only Tribal Nations ---
determine tribal citizenship.''

``I was wrong to have identified as a Native American, and, without
qualification or excuse, I apologize for the harm I caused,'' she wrote
in the letter.

She also sought to make clear that she had ``never benefited financially
or professionally'' by identifying herself as Native American, an
assertion backed up by an extensive
\href{https://www.bostonglobe.com/news/nation/2018/09/01/did-claiming-native-american-heritage-actually-help-elizabeth-warren-get-ahead-but-complicated/wUZZcrKKEOUv5Spnb7IO0K/story.html}{Boston
Globe investigation}.

Later in the letter she added: ``Regardless of whether you forgive me
--- and again, that is up to you and you alone --- I will continue to
try my hardest to be the best champion for Indian Country I can be.''

Last year, Ms. Warren rolled out a
\href{https://www.nytimes3xbfgragh.onion/2019/08/16/us/politics/elizabeth-warren-native-american.html}{policy
agenda aimed at helping Native Americans}, just days
\href{https://www.nytimes3xbfgragh.onion/2019/08/19/us/politics/elizabeth-warren-native-american.html}{before
appearing at a presidential forum in Sioux City, Iowa,} that was
dedicated to Native American issues. She drew cheers at various points
during her speech there, even as some citizens of the Cherokee Nation
continued to argue that her efforts were insufficient.

Read more

\includegraphics{https://static01.graylady3jvrrxbe.onion/images/icons/t_logo_291_black.png}

Feb. 26, 2020, 12:57 p.m. ET

Feb. 26, 2020, 12:57 p.m. ET

By
\href{https://www.nytimes3xbfgragh.onion/by/richard-a-oppel-jr}{Richard
A. Oppel Jr.} and
\href{https://www.nytimes3xbfgragh.onion/by/richard-fausset}{Richard
Fausset}

\hypertarget{klobuchars-prosecutorial-record-remains-a-liability}{%
\subsection{\texorpdfstring{\protect\hyperlink{klobuchars-prosecutorial-record-remains-a-liability}{Klobuchar's
prosecutorial record remains a
liability.}}{Klobuchar's prosecutorial record remains a liability.}}\label{klobuchars-prosecutorial-record-remains-a-liability}}

Image

Amy Klobuchar at the ministers' breakfast in North Charleston,
S.C.Credit...Travis Dove for The New York Times

During her eight-year tenure as the chief prosecutor in Minneapolis, Amy
Klobuchar earned a tough-on-crime reputation by seeking stiffer
sentences, tougher plea deals and more trials, and she vowed to call out
judges for ``letting offenders off the hook too easily.''

Those tactics served her well during her political rise, as Ms.
Klobuchar parlayed her record into a Senate seat and now a run for the
Democratic presidential nomination. But that record has also come under
attack from civil rights activists who say she pursued policies that
shored up her support in white suburbs at the cost of unfairly targeting
minorities and declining to prosecute police shootings.

``There is an entire community that suffered under her leadership, and
she has refused to accept accountability for the harm that she caused,''
said Nekima Levy Armstrong, a civil rights lawyer and former president
of the Minneapolis N.A.A.C.P.

In recent weeks, the candidate has found herself
\href{https://www.nytimes3xbfgragh.onion/2020/02/11/us/politics/amy-klobuchar-the-view.html}{forced
to answer questions} about the case of Myon Burrell, who was 16 when her
office prosecuted him in the death of an 11-year-old girl, Tyesha
Edwards, in 2002. Now she also faces questions about her office's choice
in 2001 to push to add two days to the sentence handed down against a
legal immigrant --- the difference between a misdemeanor and a felony,
which would result in his likely deportation.

Read more about Ms. Klobuchar's record as a prosecutor:

Read more

\hypertarget{advertisement-4}{%
\subsubsection{Advertisement}\label{advertisement-4}}

\protect\hyperlink{after-dfp-ad-mid5}{Continue reading the main story}

\href{https://www.nytimes3xbfgragh.onion/by/sydney-ember}{\includegraphics{https://static01.graylady3jvrrxbe.onion/images/2018/06/12/multimedia/author-sydney-ember/author-sydney-ember-thumbLarge.png}}

Feb. 26, 2020, 12:37 p.m. ET

Feb. 26, 2020, 12:37 p.m. ET

By \href{https://www.nytimes3xbfgragh.onion/by/sydney-ember}{Sydney
Ember}

\hypertarget{sanders-stays-on-message-after-debate}{%
\subsection{\texorpdfstring{\protect\hyperlink{sanders-stays-on-message-after-debate}{Sanders
stays on message after
debate.}}{Sanders stays on message after debate.}}\label{sanders-stays-on-message-after-debate}}

Image

Bernie Sanders at a campaign rally in North Charleston,
S.C.Credit...Erin Schaff/The New York Times

NORTH CHARLESTON --- After a debate during which he came under sustained
attack from his rivals but emerged relatively unscathed yet again, how
did Bernie Sanders feel? Was he riding high? Was he triumphant?

By all appearances, he was neither, delivering largely the same stump
speech he has for decades at a rally in North Charleston on Wednesday.
Before an enthusiastic but largely white crowd, Mr. Sanders laced into
President Trump --- as he often does.

``My view is that people all across the political spectrum understand we
cannot continue to have a president who is a pathological liar, we
cannot continue having a president who is running a corrupt
administration,'' he said.

As he often does, he knocked the establishment for asking whether he
could defeat Mr. Trump in the general election in November, and cited
recent polls showing him ahead.

He said he was ``bringing people together'' unlike Mr. Trump, who he
said was trying to ``divide us all up.''

He took a quick shot at Joseph R. Biden Jr. for a voting record that he
said would not excite voters as he could.

He pledged to expand the electorate and said that in order to beat Mr.
Trump, ``we are going to need the largest voter turnout in the history
of the United States.''

Three days before the South Carolina primary, Mr. Sanders offered the
same message he had before the nominating contests in Iowa, New
Hampshire and Nevada.

One of the few differences with his speech: He was able to mention those
three states as victories, offering them up as proof of the strength of
his grass-roots movement.

Read more

\href{https://www.nytimes3xbfgragh.onion/by/matt-stevens}{\includegraphics{https://static01.graylady3jvrrxbe.onion/images/2019/04/03/multimedia/author-matt-stevens/author-matt-stevens-thumbLarge.png}}

Feb. 26, 2020, 12:20 p.m. ET

Feb. 26, 2020, 12:20 p.m. ET

By \href{https://www.nytimes3xbfgragh.onion/by/matt-stevens}{Matt
Stevens}

\hypertarget{candidates-make-their-case-with-new-ads-and-merchandise}{%
\subsection{\texorpdfstring{\protect\hyperlink{candidates-make-their-case-with-new-ads-and-merchandise}{Candidates
make their case with new ads (and
merchandise).}}{Candidates make their case with new ads (and merchandise).}}\label{candidates-make-their-case-with-new-ads-and-merchandise}}

Joseph R. Biden Jr. wants to highlight a big endorsement. Pete Buttigieg
seeks to get the word out about his plan to invest in African-Americans.
And Michael R. Bloomberg has released new campaign swag that takes a
shot at Bernie Sanders.

With the South Carolina primary just days away, the Democratic
candidates have honed their closing arguments and tailored another round
of advertising to voters they hope to win over in the 11th hour.

Just minutes after receiving a significant endorsement from James E.
Clyburn, the South Carolina representative who is the highest-ranking
African-American in Congress, Mr. Biden's campaign quickly
\href{https://twitter.com/JoeBiden/status/1232683784954359808}{turned
around a video featuring his praise} and posted it on Twitter.

Meanwhile, Mr. Buttigieg released a new series of television and digital
ads that \href{https://www.youtube.com/watch?v=SUbMJprZ548}{feature his
Douglass Plan}, which seeks to empower black Americans, and
\href{https://www.youtube.com/watch?v=SUbMJprZ548}{Gladys Muhammad}, a
black community leader from South Bend, Ind., where he was mayor.

For his part, Mr. Bloomberg's team has released a new piece of campaign
merchandise that takes a shot at Mr. Sanders.

``We've got something in the shop for our \#CAPitalist friends,'' Team
Bloomberg tweeted on Wednesday. The item?
\href{https://twitter.com/Mike2020/status/1232668650253029378}{A
baseball cap emblazoned with the phrase ``NOT A SOCIALIST.''}

Read more

\href{https://www.nytimes3xbfgragh.onion/by/matt-stevens}{\includegraphics{https://static01.graylady3jvrrxbe.onion/images/2019/04/03/multimedia/author-matt-stevens/author-matt-stevens-thumbLarge.png}}

Feb. 26, 2020, 11:56 a.m. ET

Feb. 26, 2020, 11:56 a.m. ET

By \href{https://www.nytimes3xbfgragh.onion/by/matt-stevens}{Matt
Stevens}

\hypertarget{biden-campaign-tries-to-explain-his-arrest-abroad}{%
\subsection{\texorpdfstring{\protect\hyperlink{biden-campaign-tries-to-explain-his-arrest-abroad}{Biden
campaign tries to explain his `arrest'
abroad.}}{Biden campaign tries to explain his `arrest' abroad.}}\label{biden-campaign-tries-to-explain-his-arrest-abroad}}

An official with Joseph R. Biden Jr.'s campaign said on Tuesday that he
had been separated from black colleagues during a trip to South Africa
in the 1970s, an explanation that amounts to his team's first attempt to
clarify comments that Mr. Biden has repeatedly made about having been
arrested while trying to visit Nelson Mandela in prison.

Mr. Biden's assertion,
\href{https://www.nytimes3xbfgragh.onion/2020/02/21/us/politics/biden-south-africa-arrest-mandela.html}{which
was rebutted by a former U.S. ambassador to the United Nations}, has
drawn skepticism in recent days as Mr. Biden's campaign declined to
answer questions about his remarks. But after the Democratic debate on
Tuesday, Kate Bedingfield, a deputy campaign manager for Mr. Biden, said
he had been referring to an episode in which he was separated from black
colleagues in Johannesburg while on a congressional delegation trip to
South Africa several decades ago.

``He was separated from his party at the airport,'' Ms. Bedingfield
said, a point she reiterated when a reporter noted that a separation did
not did not constitute an arrest.

As recently as this month, Mr. Biden told an audience in South Carolina:
``I had the great honor of meeting'' Mr. Mandela, and ``I had the great
honor of being arrested with our U.N. ambassador on the streets of
Soweto trying to get to see him.''

\href{https://www.nytimes3xbfgragh.onion/2020/02/26/us/politics/joe-biden-arrest-mandela.html}{Here
is the full story} on the campaign's explanation by my colleague Katie
Glueck.

Read more

\hypertarget{advertisement-5}{%
\subsubsection{Advertisement}\label{advertisement-5}}

\protect\hyperlink{after-dfp-ad-mid6}{Continue reading the main story}

\href{https://www.nytimes3xbfgragh.onion/by/sheryl-gay-stolberg}{\includegraphics{https://static01.graylady3jvrrxbe.onion/images/2018/11/26/multimedia/author-sheryl-gay-stolberg/author-sheryl-gay-stolberg-thumbLarge.png}}

Feb. 26, 2020, 11:28 a.m. ET

Feb. 26, 2020, 11:28 a.m. ET

By
\href{https://www.nytimes3xbfgragh.onion/by/sheryl-gay-stolberg}{Sheryl
Gay Stolberg}

\hypertarget{pelosi-says-she-is-comfortable-with-sanders-as-a-potential-nominee}{%
\subsection{\texorpdfstring{\protect\hyperlink{pelosi-says-she-is-comfortable-with-sanders-as-a-potential-nominee}{Pelosi
says she is `comfortable' with Sanders as a potential
nominee.}}{Pelosi says she is `comfortable' with Sanders as a potential nominee.}}\label{pelosi-says-she-is-comfortable-with-sanders-as-a-potential-nominee}}

Image

Speaker Nancy Pelosi at the Capitol in Washington.Credit...Anna
Moneymaker/The New York Times

WASHINGTON --- Speaker Nancy Pelosi said on Wednesday that she was
``comfortable'' with Bernie Sanders at the top of the Democratic ticket
in November, and did not believe his candidacy would jeopardize her
majority.

Ms. Pelosi spoke in a brief hallway interview after her caucus had its
weekly meeting, amid growing concern among moderate Democrats --- many
of them in swing districts won by President Trump --- about a Sanders
nomination.

One such moderate, Representative Tom Malinowski of New Jersey, told
reporters on Wednesday that he had endorsed Joseph R. Biden Jr., and
suggested that by selecting Mr. Sanders, Democrats would fritter away
their chance to wrest the presidency from Mr. Trump. ``Why we would risk
this extraordinary opportunity by nominating somebody who has a tendency
to divide our own side is beyond me,'' he said.

Ms. Pelosi has scheduled a briefing on Thursday at which officials from
the Democratic National Committee will brief Democratic lawmakers on the
delegate selection process. Members of Congress are ``superdelegates,''
who are entitled to support whomever they want at the party convention,
but new rules this year effectively bar them from participating in the
first ballot.

Read more

\href{https://www.nytimes3xbfgragh.onion/by/matt-stevens}{\includegraphics{https://static01.graylady3jvrrxbe.onion/images/2019/04/03/multimedia/author-matt-stevens/author-matt-stevens-thumbLarge.png}}

Feb. 26, 2020, 11:03 a.m. ET

Feb. 26, 2020, 11:03 a.m. ET

By \href{https://www.nytimes3xbfgragh.onion/by/matt-stevens}{Matt
Stevens}

\hypertarget{boston-globe-endorses-warren-ahead-of-super-tuesday}{%
\subsection{\texorpdfstring{\protect\hyperlink{boston-globe-endorses-warren-ahead-of-super-tuesday}{Boston
Globe endorses Warren ahead of Super
Tuesday.}}{Boston Globe endorses Warren ahead of Super Tuesday.}}\label{boston-globe-endorses-warren-ahead-of-super-tuesday}}

Image

Elizabeth Warren arrived for an event in Charleston, S.C.Credit...Ruth
Fremson/The New York Times

The Boston Globe editorial board announced on Wednesday that it would
endorse Elizabeth Warren, its home-state senator, in the Democratic
primary, less than a week before Massachusetts voters go to the polls on
Super Tuesday.

In an
\href{https://www.bostonglobe.com/2020/02/26/opinion/globe-endorses-elizabeth-warren/}{editorial
published} early Wednesday morning, The Globe called Ms. Warren ``a
leader with the qualifications, the track record, and the tenacity to
defend the principles of democracy, bring fairness to an economy that is
excluding too many Americans, and advance a progressive agenda.''

``She would fight the corruption and corporate influence that distort
our politics, lift up working families, and combat gun violence and
climate change,'' the editorial said, adding that her ``diagnosis of
what ails the democratic process is sound'' and that she is ``fearless
and brilliant on her feet.''

The newspaper endorsement gives Ms. Warren a timely boost in a state
that is all but a must-win for her campaign.

The Globe's editorial board
\href{https://www.bostonglobe.com/2020/02/26/opinion/gop-primaries-nominating-bill-weld-would-restore-principle-probity-republican-party/}{also
endorsed William F. Weld}, the former governor of Massachusetts, in the
Republican primary.

Read more

\href{https://www.nytimes3xbfgragh.onion/by/stephanie-saul}{\includegraphics{https://static01.graylady3jvrrxbe.onion/images/2020/02/06/reader-center/author-stephanie-saul/author-stephanie-saul-thumbLarge.png}}

Feb. 26, 2020, 10:38 a.m. ET

Feb. 26, 2020, 10:38 a.m. ET

By \href{https://www.nytimes3xbfgragh.onion/by/stephanie-saul}{Stephanie
Saul}

\hypertarget{sanders-invokes-his-long-history-with-civil-rights-leaders}{%
\subsection{\texorpdfstring{\protect\hyperlink{sanders-invokes-his-long-history-with-civil-rights-leaders}{Sanders
invokes his long history with civil rights
leaders.}}{Sanders invokes his long history with civil rights leaders.}}\label{sanders-invokes-his-long-history-with-civil-rights-leaders}}

NORTH CHARLESTON --- Bernie Sanders, speaking at the Rev. Al Sharpton's
breakfast, invoked his association with a number of civil rights leaders
and made reference to his work fighting housing discrimination in
Chicago, his involvement with the Student Nonviolent Coordinating
Committee, his participation in the 1965 March on Washington and his
endorsement in 1988 of the Rev. Jesse Jackson's presidential bid.

``Dr. King talked about socialism during his life, and what he said back
then resonates today,'' Mr. Sanders said. ```This country has socialism
for the rich and rugged individualism for the poor,''' he continued,
quoting the Rev. Dr. Martin Luther King Jr. ``You all know what that
means? It means that people like Donald Trump when he was a private real
estate developer received \$800 million in subsidies to build housing.
That's called socialism for the rich.''

Mr. Sharpton introduced Mr. Sanders as the front-runner and talked about
how opponents of the civil rights movement accused Dr. King of being a
Communist. ``Whatever you decide to do on Saturday, do not go by those
who try to use the socialist tag to separate us over what we need to do
in this country. Because we are not that stupid. Socialism, capitalism,
all of them have not worked for black folks.''

\hypertarget{our-2020-election-guide}{%
\section{Our 2020 Election Guide}\label{our-2020-election-guide}}

Updated ~Sept. 8, 2020

\begin{center}\rule{0.5\linewidth}{\linethickness}\end{center}

\begin{itemize}
\item ~
  \hypertarget{the-latest}{%
  \subsection{The Latest}\label{the-latest}}

  \begin{itemize}
  \item
    The campaign
    \href{https://www.nytimes3xbfgragh.onion/live/2020/09/08/us/trump-vs-biden?action=click\&pgtype=Article\&state=default\&region=BELOW_MAIN_CONTENT\&context=storylines_guide}{shifts
    to a higher gear this week}, with President Trump set to visit
    Florida and North Carolina today and Joseph R. Biden heading to
    Michigan tomorrow.
  \end{itemize}
\item ~
  \hypertarget{how-to-win-270}{%
  \subsection{How to Win 270}\label{how-to-win-270}}

  \begin{itemize}
  \item
    Joe Biden and Donald Trump need 270 electoral votes to reach the
    White House. Try building
    \href{https://www.nytimes3xbfgragh.onion/interactive/2020/us/elections/election-states-biden-trump.html?action=click\&pgtype=Article\&state=default\&region=BELOW_MAIN_CONTENT\&context=storylines_guide}{your
    own coalition of battleground states}~to see potential outcomes.
  \end{itemize}
\item ~
  \hypertarget{voting-by-mail}{%
  \subsection{Voting by Mail}\label{voting-by-mail}}

  \begin{itemize}
  \item
    Will you have enough time to vote by mail in your state? Yes, but
    it's risky to procrastinate.
    \href{https://www.nytimes3xbfgragh.onion/interactive/2020/08/31/us/politics/vote-by-mail-deadlines.html?action=click\&pgtype=Article\&state=default\&region=BELOW_MAIN_CONTENT\&context=storylines_guide}{Check
    your state's deadline.}
  \item
    \href{https://www.nytimes3xbfgragh.onion/interactive/2020/us/elections/joe-biden.html?action=click\&pgtype=Article\&state=default\&region=BELOW_MAIN_CONTENT\&context=storylines_guide}{}

    \hypertarget{joe-biden}{%
    \section{Joe Biden}\label{joe-biden}}

    \hypertarget{democrat}{%
    \subsection{Democrat}\label{democrat}}

    \href{https://www.nytimes3xbfgragh.onion/interactive/2020/us/elections/donald-trump.html?action=click\&pgtype=Article\&state=default\&region=BELOW_MAIN_CONTENT\&context=storylines_guide}{}

    \hypertarget{donald-trump}{%
    \section{Donald Trump}\label{donald-trump}}

    \hypertarget{republican}{%
    \subsection{Republican}\label{republican}}
  \end{itemize}
\item
  \hypertarget{keep-up-with-our-coverage}{%
  \subsection{Keep Up With Our
  Coverage}\label{keep-up-with-our-coverage}}

  \begin{itemize}
  \item
    Get an
    \href{https://www.nytimes3xbfgragh.onion/newsletters/politics?action=click\&pgtype=Article\&state=default\&region=BELOW_MAIN_CONTENT\&context=storylines_guide}{email}~recapping
    the day's news
  \item
    Download our mobile app on
    \href{https://apps.apple.com/us/app/nytimes/id284862083?ls=1\&mat_click_id=5c79ae7455014fd1bd66b5610c05b8f2-20191112-16948\&referrer=mat_click_id\%3D5c79ae7455014fd1bd66b5610c05b8f2-20191112-16948\%26link_click_id\%3D722930677036718082}{iOS}~and
    \href{http://a.localytics.com/android?id=com.nytimes.android\&referrer=utm_source\%3Dother_nyt_mobile_web\%26utm_medium\%3DWeb\%2520page\%26utm_term\%3DGeneral\%2520Mobile\%2520Page\%26utm_campaign\%3DNYT\%2520Mobile\%2520General\%2520Page}{Android}~and
    turn on Breaking News and Politics alerts
  \end{itemize}
\end{itemize}

\hypertarget{site-index}{%
\subsection{Site Index}\label{site-index}}

\hypertarget{site-information-navigation}{%
\subsection{Site Information
Navigation}\label{site-information-navigation}}

\begin{itemize}
\tightlist
\item
  \href{https://help.nytimes3xbfgragh.onion/hc/en-us/articles/115014792127-Copyright-notice}{©~2020~The
  New York Times Company}
\end{itemize}

\begin{itemize}
\tightlist
\item
  \href{https://www.nytco.com/}{NYTCo}
\item
  \href{https://help.nytimes3xbfgragh.onion/hc/en-us/articles/115015385887-Contact-Us}{Contact
  Us}
\item
  \href{https://www.nytco.com/careers/}{Work with us}
\item
  \href{https://nytmediakit.com/}{Advertise}
\item
  \href{http://www.tbrandstudio.com/}{T Brand Studio}
\item
  \href{https://www.nytimes3xbfgragh.onion/privacy/cookie-policy\#how-do-i-manage-trackers}{Your
  Ad Choices}
\item
  \href{https://www.nytimes3xbfgragh.onion/privacy}{Privacy}
\item
  \href{https://help.nytimes3xbfgragh.onion/hc/en-us/articles/115014893428-Terms-of-service}{Terms
  of Service}
\item
  \href{https://help.nytimes3xbfgragh.onion/hc/en-us/articles/115014893968-Terms-of-sale}{Terms
  of Sale}
\item
  \href{https://spiderbites.nytimes3xbfgragh.onion}{Site Map}
\item
  \href{https://help.nytimes3xbfgragh.onion/hc/en-us}{Help}
\item
  \href{https://www.nytimes3xbfgragh.onion/subscription?campaignId=37WXW}{Subscriptions}
\end{itemize}
