Sections

SEARCH

\protect\hyperlink{site-content}{Skip to
content}\protect\hyperlink{site-index}{Skip to site index}

\href{https://www.nytimes3xbfgragh.onion/section/business}{Business}

\href{https://myaccount.nytimes3xbfgragh.onion/auth/login?response_type=cookie\&client_id=vi}{}

\href{https://www.nytimes3xbfgragh.onion/section/todayspaper}{Today's
Paper}

\href{/section/business}{Business}\textbar{}U.S. Factory Activity
Accelerates as Orders Jump to More Than 16-1/2-Year High

\url{https://nyti.ms/2YUGqKU}

\begin{itemize}
\item
\item
\item
\item
\item
\end{itemize}

Advertisement

\protect\hyperlink{after-top}{Continue reading the main story}

Supported by

\protect\hyperlink{after-sponsor}{Continue reading the main story}

\hypertarget{us-factory-activity-accelerates-as-orders-jump-to-more-than-16-12-year-high}{%
\section{U.S. Factory Activity Accelerates as Orders Jump to More Than
16-1/2-Year
High}\label{us-factory-activity-accelerates-as-orders-jump-to-more-than-16-12-year-high}}

By Reuters

\begin{itemize}
\item
  Sept. 1, 2020
\item
  \begin{itemize}
  \item
  \item
  \item
  \item
  \item
  \end{itemize}
\end{itemize}

WASHINGTON --- U.S. manufacturing activity increased more than expected
in August as new orders surged to their highest level in over 16-1/2
years, but employment at factories continued to lag amid safety
restrictions intended to slow the spread of COVID-19.

The upbeat report from the Institute for Supply Management (ISM)
strengthened expectations for a sharp rebound in economic activity this
quarter, though the outlook is uncertain as money from the government
dries up. Manufacturing is not out of the woods yet as the coronavirus
crisis lingers.

Factory activity is being driven by demand for goods used at home like
electronics while capital investment remains weak. The ISM said
manufacturers described sentiment as "generally optimistic, though to a
lesser degree compared to July." The economy, which slipped into
recession in February, suffered its deepest contraction in at least 73
years in the second quarter.

"The economy isn't back to normal regardless of what these upbeat
surveys of purchasing manager executives are telling us," said Chris
Rupkey, chief economist at MUFG in New York. "Time will tell if
companies should be more vigilant about expansion plans. Government
pandemic stimulus programs are starting to sputter, and attempts to
limit bankruptcies and foreclosures, and those paycheck protection
schemes, are all set to fade."

The ISM said its index of national factory activity increased to a
reading of 56.0 last month from 54.2 in July. That was the highest level
since November 2018 and marked three straight months of growth.

A reading above 50 indicates expansion in manufacturing, which accounts
for 11\% of the U.S. economy. Economists polled by Reuters had forecast
the index rising to 54.5 in August.

Underscoring the uneven landscape, manufacturers of transportation
equipment said "airline industry continues to be under great pressure."
Makers of machinery said "capital equipment new orders have slowed
again."

In contrast, manufacturers of electrical equipment, appliances and
components reported "strong demand from existing and new customers."
Similar sentiments were echoed by makers of chemical, wood and
fabricated metal products.

Fifteen industries including wood, primary metals and computer and
electronic products reported growth in August. Printing and related
support activities, petroleum and coal, and furniture and related
products contracted.

"Manufacturers tied to services hit hard by the pandemic continue to
struggle," said Sarah House, a senior economist at Wells Fargo
Securities in Charlotte North Carolina. "As the pandemic drags on...
manufacturers, even those tied most tightly to goods spending, are
hardly in the clear."

For now manufacturing is chugging along, and also picked up across the
world in August. Factory activity in China expanded at the fastest clip
in nearly a decade last month, while manufacturing remained on a
recovery path in the euro zone.

Stocks on Wall Street were trading higher. The dollar dipped against a
basket of currencies. U.S. Treasury prices rose.

EMPLOYMENT TRAILING

The ISM's forward-looking new orders sub-index increased to a reading of
67.6 in August, the strongest since January 2004, from 61.5 in July.
Fifteen industries reported an increase in demand. Only one industry
said orders declined. A measure of customers' inventories dropped to its
lowest level since June 2010, an indication that orders could rise
further.

The survey's measure of order backlogs at factories increased as did
orders for exports.

Though factory employment continued to improve last month, it remained
in contraction territory. The ISM's manufacturing employment measure
rose to a reading of 46.4 from 44.3 in July. This measure has now
contracted for 13 straight months.

According to the ISM "companies and suppliers operated in reconfigured
factories, with limited labor application due to safety restrictions."
It said "long-term labor market growth remains uncertain."

Factory employment was already in decline before the coronavirus crisis
because of the Trump administration's trade war with China. Its struggle
to rebound even as orders received by factories are rising fits in with
economists' views that the labor market is losing steam after being
boosted by the reopening of businesses in May.

The government's closely followed employment report to be released on
Friday is expected to show 1.4 million jobs created in August after
1.763 million were added in July, according to a Reuters survey of
economists. That would leave nonfarm payrolls about 11.5 million below
their pre-pandemic level.

Even with manufacturing now in what some economists say is a V-shaped
recovery, business investment remains lackluster.

A separate report from the Commerce Department on Tuesday showed
construction spending edged up 0.1\% in July after falling 0.5\% in
June, confounding economists' expectations for a 1.0\% rebound. While
low mortgage rates boosted investment in homebuilding, outlays on
nonresidential construction projects such as office infrastructure and
power fell, potential headwinds to manufacturing.

Spending on public construction projects tumbled 1.3\%, with big
declines in education, highways and streets.

"Public construction will continue to be held down as state and local
budgets are squeezed by the pandemic," said Nancy Vanden Houten, lead
U.S. economist~at Oxford Economics in New York.

(Reporting By Lucia Mutikani; Editing by Chizu Nomiyama and Andrea
Ricci)

Advertisement

\protect\hyperlink{after-bottom}{Continue reading the main story}

\hypertarget{site-index}{%
\subsection{Site Index}\label{site-index}}

\hypertarget{site-information-navigation}{%
\subsection{Site Information
Navigation}\label{site-information-navigation}}

\begin{itemize}
\tightlist
\item
  \href{https://help.nytimes3xbfgragh.onion/hc/en-us/articles/115014792127-Copyright-notice}{©~2020~The
  New York Times Company}
\end{itemize}

\begin{itemize}
\tightlist
\item
  \href{https://www.nytco.com/}{NYTCo}
\item
  \href{https://help.nytimes3xbfgragh.onion/hc/en-us/articles/115015385887-Contact-Us}{Contact
  Us}
\item
  \href{https://www.nytco.com/careers/}{Work with us}
\item
  \href{https://nytmediakit.com/}{Advertise}
\item
  \href{http://www.tbrandstudio.com/}{T Brand Studio}
\item
  \href{https://www.nytimes3xbfgragh.onion/privacy/cookie-policy\#how-do-i-manage-trackers}{Your
  Ad Choices}
\item
  \href{https://www.nytimes3xbfgragh.onion/privacy}{Privacy}
\item
  \href{https://help.nytimes3xbfgragh.onion/hc/en-us/articles/115014893428-Terms-of-service}{Terms
  of Service}
\item
  \href{https://help.nytimes3xbfgragh.onion/hc/en-us/articles/115014893968-Terms-of-sale}{Terms
  of Sale}
\item
  \href{https://spiderbites.nytimes3xbfgragh.onion}{Site Map}
\item
  \href{https://help.nytimes3xbfgragh.onion/hc/en-us}{Help}
\item
  \href{https://www.nytimes3xbfgragh.onion/subscription?campaignId=37WXW}{Subscriptions}
\end{itemize}
