\href{https://www.nytimes3xbfgragh.onion/wirecutter/}{Wirecutter}

\protect\hyperlink{main}{Skip To Content}

 Search Wirecutter For:

 Search

Reviews for the real world

\begin{itemize}
\tightlist
\item
  \href{/wirecutter/electronics/}{Electronics}

  \begin{itemize}
  \tightlist
  \item
    \href{/wirecutter/electronics/}{All Electronics}
  \item
    \href{/wirecutter/electronics/accessories/}{Accessories}
  \item
    \href{/wirecutter/electronics/audio/}{Audio}
  \item
    \href{/wirecutter/electronics/batteries/}{Batteries and Charging}
  \item
    \href{/wirecutter/electronics/cameras/}{Cameras}
  \item
    \href{/wirecutter/electronics/camera-accessories/}{Camera
    Accessories}
  \item
    \href{/wirecutter/electronics/computers/}{Computers}
  \item
    \href{/wirecutter/health-fitness/exercise/}{Exercise}
  \item
    \href{/wirecutter/electronics/gaming/}{Gaming}
  \item
    \href{/wirecutter/electronics/headphones/}{Headphones}
  \item
    \href{/wirecutter/electronics/home-theater/}{Home Theater}
  \item
    \href{/wirecutter/electronics/laptops/}{Laptops}
  \item
    \href{/wirecutter/electronics/networking/}{Networking}
  \item
    \href{/wirecutter/office/printers-scanners/}{Printers \& Scanners}
  \item
    \href{/wirecutter/electronics/projectors/}{Projectors}
  \item
    \href{/wirecutter/home-garden/smart-home/}{Smart Home Devices}
  \item
    \href{/wirecutter/electronics/smartphones/}{Smartphones}
  \item
    \href{/wirecutter/electronics/storage-devices/}{Storage}
  \item
    \href{/wirecutter/electronics/tablets/}{Tablets}
  \item
    \href{/wirecutter/electronics/tvs/}{TVs}
  \item
    \href{/wirecutter/electronics/usb-c/}{USB-C}
  \end{itemize}
\item
  \href{/wirecutter/home-garden/}{Home \& Garden}

  \begin{itemize}
  \tightlist
  \item
    \href{/wirecutter/home-garden/}{All Home \& Garden}
  \item
    \href{/wirecutter/home-garden/bathroom/}{Bathroom}
  \item
    \href{/wirecutter/home-garden/bedroom/}{Bedroom}
  \item
    \href{/wirecutter/home-garden/cleaning/}{Cleaning}
  \item
    \href{/wirecutter/home-garden/closet-laundry/}{Closet \& Laundry}
  \item
    \href{/wirecutter/home-garden/garage/}{Garage}
  \item
    \href{/wirecutter/home-garden/entertaining-home/}{Entertaining}
  \item
    \href{/wirecutter/home-garden/furniture/}{Furniture}
  \item
    \href{/wirecutter/home-garden/gardening/}{Gardening}
  \item
    \href{/wirecutter/home-garden/safety-home-garden/}{Safety}
  \item
    \href{/wirecutter/home-garden/smart-home/}{Smart Home Devices}
  \item
    \href{/wirecutter/home-garden/storage-home-garden/}{Storage}
  \item
    \href{/wirecutter/home-garden/tools/}{Tools}
  \item
    \href{/wirecutter/home-garden/weather/}{Weather}
  \end{itemize}
\item
  \href{/wirecutter/kitchen-dining/}{Kitchen \& Dining}

  \begin{itemize}
  \tightlist
  \item
    \href{/wirecutter/kitchen-dining/}{All Kitchen \& Dining}
  \item
    \href{/wirecutter/kitchen-dining/baking/}{Baking}
  \item
    \href{/wirecutter/kitchen-dining/coffee/}{Coffee}
  \item
    \href{/wirecutter/kitchen-dining/cookware/}{Cookware}
  \item
    \href{/wirecutter/kitchen-dining/entertaining/}{Dining \&
    Entertaining}
  \item
    \href{/wirecutter/kitchen-dining/large-appliances/}{Large
    Appliances}
  \item
    \href{/wirecutter/kitchen-dining/small-appliances/}{Small
    Appliances}
  \item
    \href{/wirecutter/kitchen-dining/storage/}{Storage}
  \item
    \href{/wirecutter/kitchen-dining/tools-kitchen-dining/}{Tools}
  \item
    \href{/wirecutter/kitchen-dining/wine-bar/}{Wine \& Bar}
  \end{itemize}
\item
  \href{/wirecutter/money/}{Money}

  \begin{itemize}
  \tightlist
  \item
    \href{/wirecutter/money/}{All Money}
  \item
    \href{/wirecutter/money/shopping/}{Shopping}
  \item
    \href{/wirecutter/money/managing-money/}{Managing Your Money}
  \end{itemize}
\item
  More categories...

  \begin{itemize}
  \tightlist
  \item
    \href{/wirecutter/travel/}{Travel}

    \begin{itemize}
    \tightlist
    \item
      \href{/wirecutter/travel/}{All Travel}
    \item
      \href{/wirecutter/travel/backpacks/}{Backpacks}
    \item
      \href{/wirecutter/travel/bags/}{Bags}
    \item
      \href{/wirecutter/travel/gear-travel/}{Gear}
    \item
      \href{/wirecutter/travel/luggage/}{Luggage}
    \end{itemize}
  \item
    \href{/wirecutter/gifts/}{Gifts}

    \begin{itemize}
    \tightlist
    \item
      \href{/wirecutter/gifts/}{All Gifts}
    \item
      \href{/wirecutter/gifts/guide/}{Gift Guide}
    \item
      \href{/wirecutter/gifts/kids/}{Gifts for Babies \& Kids}
    \item
      \href{/wirecutter/gifts/adults/}{Gifts for Grown-Ups}
    \item
      \href{/wirecutter/gifts/mothers-day/}{Mother's Day Gifts}
    \item
      \href{/wirecutter/gifts/graduation/}{Graduation Gifts}
    \end{itemize}
  \item
    More categories...

    \begin{itemize}
    \tightlist
    \item
      \href{/wirecutter/appliances/}{Appliances}

      \begin{itemize}
      \tightlist
      \item
        \href{/wirecutter/appliances/}{All Appliances}
      \item
        \href{/wirecutter/appliances/large/}{Large Appliances}
      \item
        \href{/wirecutter/appliances/small/}{Small Appliances}
      \item
        \href{/wirecutter/appliances/vacuums/}{Vacuum Cleaners}
      \end{itemize}
    \item
      More categories...

      \begin{itemize}
      \tightlist
      \item
        \href{/wirecutter/health-fitness/}{Health \& Fitness}

        \begin{itemize}
        \tightlist
        \item
          \href{/wirecutter/health-fitness/}{All Health \& Fitness}
        \item
          \href{/wirecutter/health-fitness/cycling/}{Cycling}
        \item
          \href{/wirecutter/health-fitness/exercise/}{Exercise}
        \item
          \href{/wirecutter/health-fitness/medical-supplies/}{Medical
          Supplies}
        \item
          \href{/wirecutter/health-fitness/personal-care/}{Personal
          Care}
        \item
          \href{/wirecutter/health-fitness/wearables/}{Wearables}
        \end{itemize}
      \item
        \href{/wirecutter/baby-kid/}{Baby \& Kid}

        \begin{itemize}
        \tightlist
        \item
          \href{/wirecutter/baby-kid/}{All Baby \& Kid}
        \item
          \href{/wirecutter/baby-kid/baby/}{Baby}
        \item
          \href{/wirecutter/baby-kid/pregnancy-nursing/}{Pregnancy \&
          Nursing}
        \item
          \href{/wirecutter/baby-kid/school/}{School}
        \item
          \href{/wirecutter/baby-kid/toys/}{Toys}
        \end{itemize}
      \item
        \href{/wirecutter/outdoors/}{Outdoors}

        \begin{itemize}
        \tightlist
        \item
          \href{/wirecutter/outdoors/}{All Outdoors}
        \item
          \href{/wirecutter/outdoors/apparel/}{Apparel}
        \item
          \href{/wirecutter/outdoors/camping/}{Camping}
        \item
          \href{/wirecutter/outdoors/gear/}{Gear}
        \item
          \href{/wirecutter/outdoors/hiking/}{Hiking}
        \item
          \href{/wirecutter/outdoors/snow/}{Snow}
        \item
          \href{/wirecutter/outdoors/swim/}{Swim}
        \item
          \href{/wirecutter/home-garden/weather/}{Weather}
        \end{itemize}
      \item
        \href{/wirecutter/pets/}{Pets}

        \begin{itemize}
        \tightlist
        \item
          \href{/wirecutter/pets/}{All Pets}
        \item
          \href{/wirecutter/pets/cats/}{Cats}
        \item
          \href{/wirecutter/pets/dogs/}{Dogs}
        \item
          \href{/wirecutter/pets/gear-pets/}{Gear}
        \end{itemize}
      \item
        \href{/wirecutter/hobby-crafts/}{Hobby \& Crafts}

        \begin{itemize}
        \tightlist
        \item
          \href{/wirecutter/hobby-crafts/}{All Hobby \& Crafts}
        \item
          \href{/wirecutter/hobby-crafts/music/}{Music}
        \end{itemize}
      \item
        \href{/wirecutter/software/}{Software \& Apps}
      \item
        \href{/wirecutter/office/}{Office}

        \begin{itemize}
        \tightlist
        \item
          \href{/wirecutter/office/}{All Office}
        \item
          \href{/wirecutter/office/furniture-office/}{Furniture}
        \item
          \href{/wirecutter/office/home-office/}{Home Office}
        \item
          \href{/wirecutter/office/printers-scanners/}{Printers \&
          Scanners}
        \end{itemize}
      \item
        \href{/wirecutter/cars/}{Cars}

        \begin{itemize}
        \tightlist
        \item
          \href{/wirecutter/cars/}{All Cars}
        \item
          \href{/wirecutter/cars/accessories-auto/}{Accessories}
        \end{itemize}
      \item
        \href{/wirecutter/adult/}{Adult}
      \end{itemize}
    \end{itemize}
  \end{itemize}
\end{itemize}

\begin{itemize}
\tightlist
\item
  \href{/wirecutter/deals/}{Deals}
\item
  \href{/wirecutter/blog/}{Blog}
\end{itemize}

Reviews for the real world.

Wirecutter is reader-supported. When you buy through links on our site,
we may earn an affiliate commission. \href{/wirecutter/about/}{Learn
more}

\includegraphics{https://cdn.thewirecutter.com/wp-content/uploads/2020/01/blenders-lowres-2x1--320x160.jpg}

\begin{enumerate}
\def\labelenumi{\arabic{enumi}.}
\tightlist
\item
  \href{/wirecutter/kitchen-dining/}{Kitchen \& Dining}
\item
  \href{/wirecutter/kitchen-dining/small-appliances/}{Small Kitchen
  Appliances}
\end{enumerate}

\hypertarget{the-best-blender}{%
\section{The Best Blender}\label{the-best-blender}}

Updated April 1, 2020

\begin{itemize}
\tightlist
\item
  We've updated this guide to include information on
  \protect\hyperlink{blender-vs-food-processor-which-one-should-you-get}{how
  to choose between a blender and a food processor}.
\end{itemize}

Your guide

\begin{itemize}
\item
  \includegraphics{https://cdn.thewirecutter.com/wp-content/uploads/2015/12/stockton-lesley.jpg}

  \href{https://www.nytimes3xbfgragh.onion/wirecutter/authors/lesley-stockton/}{Lesley
  Stockton}
\end{itemize}

Share this review

A blender is the only machine in your kitchen that can produce a
beverage from chunks of ice and fruit in less than 60 seconds. And no
other blender we've tested since 2014 can reliably produce silky soups,
spoon-thick smoothies, and stable emulsifications like the
\href{https://www.nytimes3xbfgragh.onion/wirecutter/out/link/7761/112178/4/69782/?merchant=Amazon}{Vitamix
5200}. Yes, it's pricey, but we think it's worth the investment for its
powerful motor, nuanced controls, and long-lasting reliability.

\hypertarget{our-pick}{%
\paragraph{Our pick}\label{our-pick}}

\href{https://www.nytimes3xbfgragh.onion/wirecutter/out/link/7761/112178/4/109194?merchant=Amazon}{}

\hypertarget{vitamix-5200}{%
\subsubsection{\texorpdfstring{\href{https://www.nytimes3xbfgragh.onion/wirecutter/out/link/7761/112178/4/109194?merchant=Amazon}{Vitamix
5200}}{Vitamix 5200}}\label{vitamix-5200}}

\hypertarget{the-best-blender-1}{%
\subsection{\texorpdfstring{\href{https://www.nytimes3xbfgragh.onion/wirecutter/out/link/7761/112178/4/109194?merchant=Amazon}{The
best blender}}{The best blender}}\label{the-best-blender-1}}

This high-performance machine blends more gracefully than any of the
others we've tested. Its motor powers through thick mixtures, and it
comes with a seven-year warranty backed by excellent customer service.

\hypertarget{buying-options}{%
\paragraph{Buying Options}\label{buying-options}}

\href{https://www.nytimes3xbfgragh.onion/wirecutter/out/link/7761/112178/4/109194?merchant=Amazon}{\$438*
from Amazon}

\href{https://www.nytimes3xbfgragh.onion/wirecutter/out/link/7761/149373/4/109194?merchant=Vitamix}{\$450
from Vitamix}

*At the time of publishing, the price was \$440.

In our tests, from 2012 to now, Vitamix blenders have always performed
the best overall. The classic
\href{https://www.nytimes3xbfgragh.onion/wirecutter/out/link/7761/112178/4/69782/?merchant=Amazon}{Vitamix
5200} is the only one we've tried that can make creamy peanut butter and
puree soup without spewing molten liquid up the sides of the jar. It
doesn't have any preset buttons, but it does offer the widest range of
speeds (far wider than on the comparably priced
\protect\hyperlink{the-competition}{Blendtec Designer 675}) of any brand
we've tested. It's a favorite in many (if not most) professional
kitchens and juice bars. We've also found the Vitamix 5200 to be one of
the most reliable and durable blenders we've tested, and if the motor
burns out within the seven-year warranty period, Vitamix will promptly
replace the machine.

Advertisement

\hypertarget{runner-up}{%
\paragraph{Runner-up}\label{runner-up}}

\href{https://www.nytimes3xbfgragh.onion/wirecutter/out/link/3754/162049/4/109195?merchant=Amazon}{}

\hypertarget{oster-versa-pro-series-blender}{%
\subsubsection{\texorpdfstring{\href{https://www.nytimes3xbfgragh.onion/wirecutter/out/link/3754/162049/4/109195?merchant=Amazon}{Oster
Versa Pro Series
Blender}}{Oster Versa Pro Series Blender}}\label{oster-versa-pro-series-blender}}

\hypertarget{a-more-affordable-but-less-durable-blender}{%
\subsection{\texorpdfstring{\href{https://www.nytimes3xbfgragh.onion/wirecutter/out/link/3754/162049/4/109195?merchant=Amazon}{A
more affordable but less durable
blender}}{A more affordable but less durable blender}}\label{a-more-affordable-but-less-durable-blender}}

This Oster Versa model offers the best balance of performance and price
that we've found. It's not as powerful or durable as the Vitamix 5200,
but it holds its own against many blenders that are twice the price.

\hypertarget{buying-options-1}{%
\paragraph{Buying Options}\label{buying-options-1}}

\href{https://www.nytimes3xbfgragh.onion/wirecutter/out/link/3754/162049/4/109195?merchant=Amazon}{\$148*
from Amazon}

*At the time of publishing, the price was \$155.

The
\href{https://www.nytimes3xbfgragh.onion/wirecutter/out/link/3754/162049/4/69780/?merchant=Amazon}{Oster
Versa Pro Series Blender} is the best of a new breed of more
budget-friendly high-powered blenders. Compared with equally priced
blenders, this 1,400-watt model offers more speed variations and runs
more quietly; it's also one of the few models that come with a tamper
for bursting air pockets in thick mixtures. At 17½ inches tall, it will
fit better on a counter under a cabinet than most other high-performance
models. We don't think this is the absolute best blender out there. It
doesn't compare to Vitamix blenders in terms of power and longevity (we
burned out our Oster after two and a half years), but it does have
serious blending skills, a user-friendly design, and a solid, seven-year
warranty. If you don't want to throw down almost half a grand on a
powerful blender, the Oster is your best bet.

\hypertarget{also-great}{%
\paragraph{Also great}\label{also-great}}

\href{https://www.nytimes3xbfgragh.onion/wirecutter/out/link/8947/25620/4/109196?merchant=Amazon}{}

\hypertarget{cleanblend-blender}{%
\subsubsection{\texorpdfstring{\href{https://www.nytimes3xbfgragh.onion/wirecutter/out/link/8947/25620/4/109196?merchant=Amazon}{Cleanblend
Blender}}{Cleanblend Blender}}\label{cleanblend-blender}}

\hypertarget{powerful-blender-motor-shorter-warranty}{%
\subsection{\texorpdfstring{\href{https://www.nytimes3xbfgragh.onion/wirecutter/out/link/8947/25620/4/109196?merchant=Amazon}{Powerful
blender motor, shorter
warranty}}{Powerful blender motor, shorter warranty}}\label{powerful-blender-motor-shorter-warranty}}

A relative newcomer, this blender makes some of the silkiest smoothies,
but its speeds aren't as nuanced as those on our top pick.

\hypertarget{buying-options-2}{%
\paragraph{Buying Options}\label{buying-options-2}}

\href{https://www.nytimes3xbfgragh.onion/wirecutter/out/link/8947/25620/4/109196?merchant=Amazon}{\$180
from Amazon}

If you're not ready to spring for the Vitamix, and you don't mind
trading the Oster's longer warranty to get a little more power, go for
the 1,800-watt
\href{https://www.nytimes3xbfgragh.onion/wirecutter/out/link/8947/25620/4/69781/?merchant=Amazon}{Cleanblend
Blender}. The Cleanblend's strong motor helps pulverize berry seeds and
ice, creating creamier smoothies and piña coladas than even the Vitamix.
This model's jar is made of thick, durable Tritan plastic and has a
comfortable, grippy handle. Unlike the Oster model, the Cleanblend
doesn't have any preset buttons and doesn't offer much variance between
the low and high speeds. In our testing, the Cleanblend's motor has held
up better than the Oster's and is still going strong after four years of
regular use. But this blender comes with only a five-year warranty, in
contrast to the seven years of coverage from both Vitamix and Oster. And
since Cleanblend has been around only since 2013, we're still a little
uncertain of the company's staying power and the reliability of its
customer service.

\hypertarget{budget-pick}{%
\paragraph{Budget pick}\label{budget-pick}}

\href{https://www.nytimes3xbfgragh.onion/wirecutter/out/link/35968/159119/4/109197?merchant=Bed\%20Bath\%20\%26\%20Beyond}{}

\hypertarget{kitchenaid-5-speed-diamond-blender}{%
\subsubsection{\texorpdfstring{\href{https://www.nytimes3xbfgragh.onion/wirecutter/out/link/35968/159119/4/109197?merchant=Bed\%20Bath\%20\%26\%20Beyond}{KitchenAid
5-Speed Diamond
Blender}}{KitchenAid 5-Speed Diamond Blender}}\label{kitchenaid-5-speed-diamond-blender}}

\hypertarget{for-occasional-blending}{%
\subsection{\texorpdfstring{\href{https://www.nytimes3xbfgragh.onion/wirecutter/out/link/35968/159119/4/109197?merchant=Bed\%20Bath\%20\%26\%20Beyond}{For
occasional
blending}}{For occasional blending}}\label{for-occasional-blending}}

Although this blender isn't as powerful as more expensive models, it can
produce a great emulsification and decent (but chunkier) smoothies.

\hypertarget{buying-options-3}{%
\paragraph{Buying Options}\label{buying-options-3}}

\href{https://shop-links.co/1696305516080111511}{\$90 from Bed Bath \&
Beyond}

\href{https://www.nytimes3xbfgragh.onion/wirecutter/out/link/35968/159787/4/109197?merchant=Home\%20Depot}{\$130
from Home Depot}

\href{https://www.nytimes3xbfgragh.onion/wirecutter/out/link/35968/158229/4/109197?merchant=Amazon}{Buy
from Amazon}

Not everyone wants to spend \$200, let alone over \$400, on a blender.
If you want a blender for the occasional sauce or smoothie, the
\href{https://www.nytimes3xbfgragh.onion/wirecutter/out/link/35968/159119/4/109206/?merchant=Bed\%20Bath\%20\%26\%20Beyond}{KitchenAid
5-Speed Diamond Blender} is the best model we've seen for under \$100.
In our tests, we found that the Diamond model blended thick smoothies
better and faster than its predecessor, the
\href{https://www.nytimes3xbfgragh.onion/wirecutter/out/link/7763/22821/4/109199/?merchant=Amazon}{KitchenAid
5-Speed Classic Blender} (our former budget pick). It produced coarser
textures than any of our other picks did, and its motor isn't nearly as
powerful (and so is more likely to burn out if overtaxed). But it's a
good, all-purpose blender that's small enough to fit on the counter
under most kitchen cabinets.

\hypertarget{everything-we-recommend}{%
\subsubsection{Everything we recommend}\label{everything-we-recommend}}

\hypertarget{our-pick-1}{%
\paragraph{Our pick}\label{our-pick-1}}

\href{https://www.nytimes3xbfgragh.onion/wirecutter/out/link/7761/112178/4/109194?merchant=Amazon}{}

\hypertarget{vitamix-5200-1}{%
\subsubsection{\texorpdfstring{\href{https://www.nytimes3xbfgragh.onion/wirecutter/out/link/7761/112178/4/109194?merchant=Amazon}{Vitamix
5200}}{Vitamix 5200}}\label{vitamix-5200-1}}

\hypertarget{the-best-blender-2}{%
\subsection{\texorpdfstring{\href{https://www.nytimes3xbfgragh.onion/wirecutter/out/link/7761/112178/4/109194?merchant=Amazon}{The
best blender}}{The best blender}}\label{the-best-blender-2}}

This high-performance machine blends more gracefully than any of the
others we've tested. Its motor powers through thick mixtures, and it
comes with a seven-year warranty backed by excellent customer service.

\hypertarget{buying-options-4}{%
\paragraph{Buying Options}\label{buying-options-4}}

\href{https://www.nytimes3xbfgragh.onion/wirecutter/out/link/7761/112178/4/109194?merchant=Amazon}{\$438*
from Amazon}

\href{https://www.nytimes3xbfgragh.onion/wirecutter/out/link/7761/149373/4/109194?merchant=Vitamix}{\$450
from Vitamix}

*At the time of publishing, the price was \$440.

\hypertarget{runner-up-1}{%
\paragraph{Runner-up}\label{runner-up-1}}

\href{https://www.nytimes3xbfgragh.onion/wirecutter/out/link/3754/162049/4/109195?merchant=Amazon}{}

\hypertarget{oster-versa-pro-series-blender-1}{%
\subsubsection{\texorpdfstring{\href{https://www.nytimes3xbfgragh.onion/wirecutter/out/link/3754/162049/4/109195?merchant=Amazon}{Oster
Versa Pro Series
Blender}}{Oster Versa Pro Series Blender}}\label{oster-versa-pro-series-blender-1}}

\hypertarget{a-more-affordable-but-less-durable-blender-1}{%
\subsection{\texorpdfstring{\href{https://www.nytimes3xbfgragh.onion/wirecutter/out/link/3754/162049/4/109195?merchant=Amazon}{A
more affordable but less durable
blender}}{A more affordable but less durable blender}}\label{a-more-affordable-but-less-durable-blender-1}}

This Oster Versa model offers the best balance of performance and price
that we've found. It's not as powerful or durable as the Vitamix 5200,
but it holds its own against many blenders that are twice the price.

\hypertarget{buying-options-5}{%
\paragraph{Buying Options}\label{buying-options-5}}

\href{https://www.nytimes3xbfgragh.onion/wirecutter/out/link/3754/162049/4/109195?merchant=Amazon}{\$148*
from Amazon}

*At the time of publishing, the price was \$155.

\hypertarget{also-great-1}{%
\paragraph{Also great}\label{also-great-1}}

\href{https://www.nytimes3xbfgragh.onion/wirecutter/out/link/8947/25620/4/109196?merchant=Amazon}{}

\hypertarget{cleanblend-blender-1}{%
\subsubsection{\texorpdfstring{\href{https://www.nytimes3xbfgragh.onion/wirecutter/out/link/8947/25620/4/109196?merchant=Amazon}{Cleanblend
Blender}}{Cleanblend Blender}}\label{cleanblend-blender-1}}

\hypertarget{powerful-blender-motor-shorter-warranty-1}{%
\subsection{\texorpdfstring{\href{https://www.nytimes3xbfgragh.onion/wirecutter/out/link/8947/25620/4/109196?merchant=Amazon}{Powerful
blender motor, shorter
warranty}}{Powerful blender motor, shorter warranty}}\label{powerful-blender-motor-shorter-warranty-1}}

A relative newcomer, this blender makes some of the silkiest smoothies,
but its speeds aren't as nuanced as those on our top pick.

\hypertarget{buying-options-6}{%
\paragraph{Buying Options}\label{buying-options-6}}

\href{https://www.nytimes3xbfgragh.onion/wirecutter/out/link/8947/25620/4/109196?merchant=Amazon}{\$180
from Amazon}

\hypertarget{budget-pick-1}{%
\paragraph{Budget pick}\label{budget-pick-1}}

\href{https://www.nytimes3xbfgragh.onion/wirecutter/out/link/35968/159119/4/109197?merchant=Bed\%20Bath\%20\%26\%20Beyond}{}

\hypertarget{kitchenaid-5-speed-diamond-blender-1}{%
\subsubsection{\texorpdfstring{\href{https://www.nytimes3xbfgragh.onion/wirecutter/out/link/35968/159119/4/109197?merchant=Bed\%20Bath\%20\%26\%20Beyond}{KitchenAid
5-Speed Diamond
Blender}}{KitchenAid 5-Speed Diamond Blender}}\label{kitchenaid-5-speed-diamond-blender-1}}

\hypertarget{for-occasional-blending-1}{%
\subsection{\texorpdfstring{\href{https://www.nytimes3xbfgragh.onion/wirecutter/out/link/35968/159119/4/109197?merchant=Bed\%20Bath\%20\%26\%20Beyond}{For
occasional
blending}}{For occasional blending}}\label{for-occasional-blending-1}}

Although this blender isn't as powerful as more expensive models, it can
produce a great emulsification and decent (but chunkier) smoothies.

\hypertarget{buying-options-7}{%
\paragraph{Buying Options}\label{buying-options-7}}

\href{https://shop-links.co/1696305516080111511}{\$90 from Bed Bath \&
Beyond}

\href{https://www.nytimes3xbfgragh.onion/wirecutter/out/link/35968/159787/4/109197?merchant=Home\%20Depot}{\$130
from Home Depot}

\href{https://www.nytimes3xbfgragh.onion/wirecutter/out/link/35968/158229/4/109197?merchant=Amazon}{Buy
from Amazon}

\hypertarget{the-research}{%
\subsection{The research}\label{the-research}}

Collapse all

\begin{itemize}
\tightlist
\item
  \protect\hyperlink{why-you-should-trust-us}{Why you should trust us}
\item
  \protect\hyperlink{blender-vs-food-processor-which-one-should-you-get}{Blender
  vs. food processor: Which one should you get?}
\item
  \protect\hyperlink{what-type-of-blender-should-you-get}{What type of
  blender should you get?}
\item
  \protect\hyperlink{how-we-picked}{How we picked}
\item
  \protect\hyperlink{how-we-tested}{How we tested}
\item
  \protect\hyperlink{our-pick-vitamix-5200}{Our pick: Vitamix 5200}
\item
  \protect\hyperlink{flaws-but-not-dealbreakers}{Flaws but not
  dealbreakers}
\item
  \protect\hyperlink{what-about-other-vitamix-models}{What about other
  Vitamix models?}
\item
  \protect\hyperlink{runner-up-oster-versa-pro-series-blender}{Runner-up:
  Oster Versa Pro Series Blender}
\item
  \protect\hyperlink{also-great-cleanblend-blender}{Also great:
  Cleanblend Blender}
\item
  \protect\hyperlink{budget-pick-kitchenaid-5-speed-diamond-blender}{Budget
  pick: KitchenAid 5-Speed Diamond Blender}
\item
  \protect\hyperlink{blender-care-and-maintenance}{Blender care and
  maintenance}
\item
  \protect\hyperlink{the-competition}{The competition}
\item
  \protect\hyperlink{what-to-look-forward-to}{What to look forward to}
\item
  \protect\hyperlink{sources}{Sources}
\end{itemize}

\hypertarget{why-you-should-trust-us}{%
\subsection{Why you should trust us}\label{why-you-should-trust-us}}

As a senior staff writer for Wirecutter, I've covered everything from
\href{https://www.nytimes3xbfgragh.onion/wirecutter/reviews/the-best-chefs-knife-for-most-cooks/}{chef's
knives} to
\href{https://www.nytimes3xbfgragh.onion/wirecutter/reviews/best-stand-mixer/}{stand
mixers}, and I've tested every blender worth testing since 2014. I also
have a breadth of cooking and entertaining knowledge from decades of
working in restaurants and magazine test kitchens. This guide builds on
the work of Christine Cyr Clisset, now a deputy editor at Wirecutter.

We reached out to Jonathan Cochran, a former blender salesperson who now
runs the site \href{http://blenderdude.com/}{Blender Dude}, for his take
on the best Vitamix and Blendtec models to test (his site has affiliate
partnerships with both companies). For our original guide, authored by
Seamus Bellamy, we consulted with
\href{http://www.americastestkitchen.com/tour}{Lisa McManus}, an
executive editor in charge of equipment testing at Cook's Illustrated
and Cook's Country magazines.

\hypertarget{blender-vs-food-processor-which-one-should-you-get}{%
\subsection{Blender vs. food processor: Which one should you
get?}\label{blender-vs-food-processor-which-one-should-you-get}}

Although there's some overlap in what they can do, blenders and food
processors aren't interchangeable appliances. A countertop blender is a
better tool for making purees, quick sauces, and emulsifications (like
mayonnaise and vinaigrette), and it's the only appliance that will whip
berries and fibrous veggies into a silky-smooth texture. Because a
blender's jar is narrow and usually angled at the base, it creates a
vortex that helps pass ingredients through the blades more frequently
than in a food processor, yielding smoother textures.

\begin{itemize}
\item
  \href{https://www.nytimes3xbfgragh.onion/wirecutter/reviews/the-best-food-processor/}{\includegraphics{https://cdn.thewirecutter.com/wp-content/uploads/2020/01/05-food-processor-cuisinart-custom-630-120x80.jpg}}

  \hypertarget{the-best-food-processor}{%
  \subsubsection{\texorpdfstring{\href{https://www.nytimes3xbfgragh.onion/wirecutter/reviews/the-best-food-processor/}{The
  Best Food
  Processor}}{The Best Food Processor}}\label{the-best-food-processor}}

  Of all the models we've tested since 2013, the
  \href{https://wclink.co/link/13566/152745/4/108512/?merchant=Amazon}{Cuisinart
  Custom 14 Food Processor} remains our favorite because it's simple,
  powerful, and durable.
\end{itemize}

With a little effort, you can also puree wet ingredients (such as
tomatoes for sauce) in a food processor, but the doughnut-shaped
container doesn't handle liquids as well as a blender's jar does---it
tends to leak. A food processor will work fine for thick purees like
hummus and is great for sauces with a coarser texture like pesto. But it
won't make a good smoothie and---since you can't control the speed of
the blades---is liable to shoot hot soup everywhere. Instead, a food
processor is best for chopping, slicing, and grating. With the right
attachment, it will even mix and knead dough. Many people use food
processors for mincing vegetables, but this appliance is also your best
friend for easily grating cheese, slicing potatoes for a gratin,
grinding fresh bread crumbs, or quickly cutting butter into flour to
make pie dough.

In short, blenders liquefy, food processors chop and slice. Depending on
your needs, you might choose one over the other, or you might want both.
We have a guide to the
\href{https://www.nytimes3xbfgragh.onion/wirecutter/reviews/the-best-food-processor/}{best
food processors} too, if you're interested.

\hypertarget{what-type-of-blender-should-you-get}{%
\subsection{What type of blender should you
get?}\label{what-type-of-blender-should-you-get}}

A countertop blender delivers the silkiest smoothies, daiquiris, soups,
and sauces of any style of blender you can buy. It's more versatile than
a
\href{https://www.nytimes3xbfgragh.onion/wirecutter/reviews/best-personal-blender/}{personal
blender} (which is meant mainly for smoothies) because it holds more and
can handle hot liquids. It's also more powerful than an
\href{https://www.nytimes3xbfgragh.onion/wirecutter/reviews/best-immersion-blender/}{immersion
blender}, which is great for pureeing soups directly in the pot or
making a quick mayo but doesn't yield the velvety textures you get from
a good countertop blender.

That said, a blender's performance and longevity are usually
proportional to its cost. High-end blenders are more powerful and
designed to puree the thickest mixtures without burning out, something
that inexpensive blenders simply can't do. If you want a kitchen
workhorse---a machine that can tackle everything from hot soups and
sauces to thick frozen concoctions---a full-size, high-powered blender
is the best choice. How much you should spend on one depends on exactly
what you'll use it for. Below is a breakdown of what each of our picks
will do for you.

\textbf{Get our budget pick, the KitchenAid, if:}

\begin{itemize}
\tightlist
\item
  You use your blender only for the occasional smoothie, frozen drink,
  or soup.
\item
  You don't blend nut butters or other motor-taxing mixtures.
\item
  A short, limited one-year warranty isn't a concern.
\end{itemize}

\textbf{Get our runner-up, the Oster, or our also-great pick, the
Cleanblend, if:}

\begin{itemize}
\tightlist
\item
  You blend no more than a few times a week.
\item
  You rarely make nut butters.
\item
  A five- or seven-year warranty is important to you.
\end{itemize}

\textbf{Get our top pick, the Vitamix, if:}

\begin{itemize}
\tightlist
\item
  Blending is part of your daily lifestyle.
\item
  You frequently blend thick, motor-taxing mixtures like nut butters and
  spoonable smoothies.
\item
  You want a blender with the widest range of speeds for easily doing
  everything from blending hot liquids to pulverizing ice cubes.
\item
  A seven-year warranty is important to you.
\end{itemize}

Alternatively, if you just want to make a daily smoothie, you might be
better off with a NutriBullet
(\href{https://www.nytimes3xbfgragh.onion/wirecutter/reviews/nutribullet/}{we've
tested them all}). And if you're still not sure whether to get a
countertop blender or another style of blender (or maybe even a food
processor), we have a
\href{https://www.nytimes3xbfgragh.onion/wirecutter/reviews/blender-food-processor-mixer/}{guide}
that breaks down the pros and cons of each type.

\hypertarget{how-we-picked}{%
\subsection{How we picked}\label{how-we-picked}}

\includegraphics{https://cdn.thewirecutter.com/wp-content/uploads/2017/12/blenders-lowres-9764-120x80.jpg}

For height comparison, from left to right: the Cleanblend, the Oster,
the Vitamix, and the KitchenAid 5-Speed Classic (our former budget
pick). Photo: Michael Hession

Since 2013, we've researched or tested almost every decent household
blender available, from budget models starting as low as \$40 to
powerful, high-performance models topping out at \$700. In all this
testing, we've found the following criteria to be the most important to
look for in a blender:

\hypertarget{jar-shape-and-motor-strength}{%
\subsubsection{Jar shape and motor
strength}\label{jar-shape-and-motor-strength}}

A great blender should be able to smoothly process tough items like
fibrous kale, frozen berries, and ice without burning out the motor. How
efficiently a blender does this depends on a combination of the blade
length and position, the shape of the mixing jar, and the motor
strength. All three of those elements combine to create a vortex that
pulls food down around the blade.

In our testing, we've found that tall, tapered jars with a curved bottom
develop a more consistent vortex than short, wide ones with a flat
bottom. But the better blending that you get from a taller, tapered jar
comes with a trade-off: A fully assembled blender might be too tall to
fit under low-hanging cabinets. Blenders with wide, short jars are
better for countertop storage, but you're sacrificing performance for
that convenience.

A more powerful motor also helps to create a better vortex and blends
thick mixtures more easily than a weaker one. But a blender's power
rating isn't easy information to come by. Most blender companies
advertise only ``peak horsepower,'' a spec that's misleading if you're
trying to determine a motor's strength. A motor works at peak horsepower
for just a fraction of a second, when you start the blender, in order to
overcome inertia. Immediately after, the motor drops to its ``rated
horsepower,'' which is the amount of power it can sustain without
burning out. As explained on
\href{http://www.cookingforengineers.com/article/287/Understanding-Blender-Specifications/print}{Cooking
For Engineers}, you can get a ballpark estimate of a blender's rated
horsepower by dividing its wattage by 746 (because 746 watts equals
approximately one unit of electrical horsepower). This equation doesn't
account for efficiency, but it does offer a more realistic approximation
of the blender's power output.

We've found that tall, tapered jars with a curved bottom develop a more
consistent vortex than short, wide ones with a flat bottom.

\hypertarget{jar-material}{%
\subsubsection{Jar material}\label{jar-material}}

Most of the blenders we've tested come with plastic jars. All of our
picks have jars made of BPA-free Tritan plastic, which is very durable.
Many of the lower-end blenders we've tested don't advertise which
material their jars are made of beyond a ``BPA-free'' note. (For the
record, BPA
\href{https://www.nytimes3xbfgragh.onion/wirecutter/blog/bpa-panic-accomplished-nothing/}{isn't
as much of a health risk} as it's been made out to be.) But the majority
of these jars are probably made of polycarbonate, which is more rigid
than Tritan but also very strong. Both materials will crack if heated
too high, which is why these jars should not go in the dishwasher.

We understand that some folks prefer metal or glass jars. But you'd be
hard-pressed to find a powerful blender with a glass jar, and there's
probably a good reason for this. As April Jones explains
\href{http://www.cookingforengineers.com/article/287/Understanding-Blender-Specifications/print}{in
her article on Cooking For Engineers}: ``Due to the high-speed blades
and high horsepower motors, glass isn't the safest option for
professional-grade blenders. If a metal object, such as a spoon or
knife, were accidentally left in the blender, a glass pitcher could
shatter and potentially cause an injury. Using polycarbonate plastics or
copolyester is a much safer option to avoid the hazard of broken
glass.'' Stainless steel jars are durable but opaque, and we like to
monitor the progress of purees and emulsifications without having to
remove the lid.

\hypertarget{price}{%
\subsubsection{Price}\label{price}}

Judging from buyer reviews, the holy grail for many home cooks seems to
be a \$50 or \$100 blender that performs like a \$500 Vitamix or
Blendtec. But that isn't realistic. High-end blenders (\$150 and
up)---often called high-performance blenders---offer more power, produce
much smoother textures, and generally last a lot longer than lower-end,
under-\$100 blenders. High-performance blenders also tackle tasks that
you'd never want to try in a cheap blender, such as making peanut butter
or milling grains.

That said, there's nothing wrong with a cheap blender as long as you
understand its limitations. Some folks want an affordable mid-range
blender to make the occasional daiquiri or smoothie. So we've tested
blenders in a wide range of prices with the understanding that, for the
most part, you get what you pay for.

\hypertarget{warranty}{%
\subsubsection{Warranty}\label{warranty}}

The most common complaint we've found about cheap blenders is that their
motors burn out easily and their jars crack or start leaking. But it's
not impossible for even higher-end blenders to experience burnout. As
\href{https://www.americastestkitchen.com/tour}{Lisa McManus}, executive
editor in charge of equipment testing at Cook's Illustrated and Cook's
Country magazines, told Seamus Bellamy in our 2012 review, ``Blenders
have a really hard job to do in that little space. The motor is only so
big. If you make it do something difficult every day, a lot of them burn
out. It's a lot of stress to put on a little machine.'' This is why a
long warranty is important, especially if you're paying a lot for a
blender. Vitamix, Oster, and Cleanblend models all come with warranties
of five to seven years, and---at least for Vitamix machines---we've read
plenty of owner reviews saying the blender lasts much longer. You can't
expect that level of performance from dirt-cheap models, which is
probably why most of them come with only one-year limited warranties.

\includegraphics{https://cdn.thewirecutter.com/wp-content/uploads/2018/06/blenders-lowres-2971-120x80.jpg}

In 2016, we tested our picks against new models from Cuisinart and
Braun. Clockwise from left: Braun Puremix, Cuisinart Hurricane, and
Cleanblend (with the early-style jar). Photo: Michael Hession

\hypertarget{speed-control}{%
\subsubsection{Speed control}\label{speed-control}}

Whether you want a blender with manual controls or preset functions is
largely a personal preference. But we appreciate a powerful blender with
a simple interface that includes an on/off switch, a pulse button, and a
variable-speed dial. These easy controls allow you to quickly adjust the
speed or turn off the machine if things get messy.

Preset programs for making smoothies, mixing soups, or crushing ice can
be great if you want to multitask in the kitchen while blending. But
we've also found that these functions rarely deliver purees as smooth as
when we control the speed and time with the manual setting.

\hypertarget{tamper}{%
\subsubsection{Tamper}\label{tamper}}

In our five years of testing, we've found that a tamper---a small
plastic bat that lets you push food down into the blades---separates the
great blenders from the good ones. When a blender is really cranking,
air pockets tend to form around the blade, and a tamper allows you to
burst them without having to stop the machine. The tamper that comes
with a blender is designed to safely clear the blades of that particular
model, as long as you use it with the lid on. Using a different tamper
or another tool that might hit the moving blades is dangerous and could
damage the machine. If your blender doesn't come with a tamper, the only
way you should burst air pockets is to turn the machine off, remove the
jar from the base, and stir the mixture with a spoon.

\includegraphics{https://cdn.thewirecutter.com/wp-content/uploads/2018/06/blenders-lowres-2997-120x80.jpg}

Don't fear the tamper! Blender jars and lids are designed to let you use
the tamper without the risk of jamming it in the blades. Photo: Michael
Hession

So why don't all blenders come with a tamper? Because forcing frozen and
thick mixtures into the blades puts a lot of stress on the motor.
Performance blenders that include tampers have powerful motors that can
handle this stress---they're designed for it. But cheaper blenders have
weaker motors. If they were to include tampers, people would probably
push these machines past their limits, ultimately leading to the motor
burning out.

\hypertarget{how-we-tested}{%
\subsection{How we tested}\label{how-we-tested}}

\includegraphics{https://cdn.thewirecutter.com/wp-content/uploads/2018/06/blenders-lowres-2995-120x80.jpg}

We made green smoothies in each blender to see how well the machines
processed fibrous raw kale. Photo: Michael Hession

We gauged how each model performed everyday blending jobs like making
thick frozen smoothies and hot soups. We also wanted to see which
blenders could emulsify eggs and oil into mayonnaise and pulverize nuts
into a smooth butter. In each blender, we made a thick green smoothie
packed with frozen bananas and berries, kale, and coconut water. We
looked at each blender's ability to create a consistent vortex without
taxing the motor or needing additional liquid. Afterward, we tasted the
smoothies to assess mouthfeel, and then we strained the remainder
through a fine-mesh sieve to see how well the blenders pulverized tough
greens and berry seeds.

A blender can be a useful tool for making emulsified sauces like
mayonnaise, hollandaise, vinaigrettes, and Caesar dressing, so we tested
each model's ability to emulsify mayonnaise made with one egg yolk.
Making successful blender mayonnaise (or hollandaise or Caesar) hinges
on the blades sitting low enough in the jar that they start whipping the
egg yolk before you add a drop of oil.

To see how the motors handled dense purees, we processed raw peanuts
into peanut butter. With our finalists, we made rounds of piña coladas
to see how well they blended ice into slush.

Additionally, we noted how easy or difficult each machine was to clean,
how noisy each model was, whether any of them produced a burning smell
while the motor ran, whether the jars were difficult to attach to the
bases, and how easy the interfaces were to use.

\hypertarget{our-pick-vitamix-5200}{%
\subsection{Our pick: Vitamix 5200}\label{our-pick-vitamix-5200}}

Photo: Michael Hession

\hypertarget{our-pick-2}{%
\paragraph{Our pick}\label{our-pick-2}}

\href{https://www.nytimes3xbfgragh.onion/wirecutter/out/link/7761/112178/4/109194?merchant=Amazon}{}

\hypertarget{vitamix-5200-2}{%
\subsubsection{\texorpdfstring{\href{https://www.nytimes3xbfgragh.onion/wirecutter/out/link/7761/112178/4/109194?merchant=Amazon}{Vitamix
5200}}{Vitamix 5200}}\label{vitamix-5200-2}}

\hypertarget{the-best-blender-3}{%
\subsection{\texorpdfstring{\href{https://www.nytimes3xbfgragh.onion/wirecutter/out/link/7761/112178/4/109194?merchant=Amazon}{The
best blender}}{The best blender}}\label{the-best-blender-3}}

This high-performance machine blends more gracefully than any of the
others we've tested. Its motor powers through thick mixtures, and it
comes with a seven-year warranty backed by excellent customer service.

\hypertarget{buying-options-8}{%
\paragraph{Buying Options}\label{buying-options-8}}

\href{https://www.nytimes3xbfgragh.onion/wirecutter/out/link/7761/112178/4/109194?merchant=Amazon}{\$438*
from Amazon}

\href{https://www.nytimes3xbfgragh.onion/wirecutter/out/link/7761/149373/4/109194?merchant=Vitamix}{\$450
from Vitamix}

*At the time of publishing, the price was \$440.

The
\href{https://www.nytimes3xbfgragh.onion/wirecutter/out/link/7761/112178/4/69782/?merchant=Amazon}{Vitamix
5200} offers the best performance you can get in a home blender. This
model has been one of our favorite blenders since 2014, and it's the
classic Vitamix that has remained the standard for pro chefs and blender
enthusiasts. It consistently performed at the top of the pack in our
tests, and it came recommended to us by multiple experts because it
powerfully purees and pulverizes food more reliably, thoroughly, and
elegantly than most.

The Vitamix 5200 did not make the absolute smoothest smoothies of all
the blenders we tested---that prize went to the Blendtec and Cleanblend
machines. But when it came to consistent and graceful performance, the
Vitamix won every time. This model was the only blender we tested that
smoothly blended peanuts and almonds into butter. And whereas other
blenders, such as the Blendtec, Cleanblend, and Oster, spit bits of mayo
up the sides of the jar and out the lid's center hole, the Vitamix kept
the mixture smoothly and evenly moving around the base of the blade.

We found Vitamix's variable-speed dial to have the best range among any
of the blenders we tried. Its low is really low, and there is a
noticeable shift as you advance through each number. In our tests, this
range of speeds made the Vitamix the best blender for hot liquids: You
can start blending at a lazy swirl and slowly increase the speed so that
the hot liquid is less likely to shoot up toward the lid and risk a
volcanic, trip-to-the-burn-unit situation. In comparison, the Cleanblend
has a forceful start on the lowest setting, which increases the chances
of a painful eruption when you're blending hot soups. The same goes for
the Blendtec Designer 675, which in our tests was so powerful that the
soup setting created a cyclone in a jar.

\begin{itemize}
\item
  \href{https://www.nytimes3xbfgragh.onion/wirecutter/reviews/vitamix-vs-blendtec/}{\includegraphics{https://cdn.thewirecutter.com/wp-content/uploads/2018/06/blenders-lowres-9471-120x80.jpg}}

  \hypertarget{blend-off-vitamix-vs-blendtec}{%
  \subsubsection{\texorpdfstring{\href{https://www.nytimes3xbfgragh.onion/wirecutter/reviews/vitamix-vs-blendtec/}{Blend-Off:
  Vitamix vs.
  Blendtec}}{Blend-Off: Vitamix vs. Blendtec}}\label{blend-off-vitamix-vs-blendtec}}

  We pitted a Blendtec blender against a Vitamix model in a series of
  head-to-head tests, and the winner was clear:
  \href{https://wclink.co/link/7761/0/4/76786/}{Vitamix} beat Blendtec
  every time.
\end{itemize}

The Vitamix's tamper is essential for breaking up air pockets and
pushing ingredients down toward the blade while the machine is running.
When using a model without a tamper, we often needed to stop the machine
to burst air pockets or scrape ingredients down the sides of the jar
with a spatula. In some cases, we also had to add more water to the
smoothie to get all the ingredients to move around the blades without
the help of a tamper. For all these reasons, blending in the Vitamix
with a tamper took about half the time as it took in the Blendtec with
no included tamper. By keeping the ingredients moving, we were able to
whip up a smoothie in about 30 seconds.

\includegraphics{https://cdn.thewirecutter.com/wp-content/uploads/2018/06/blenders-lowres-3761-120x80.jpg}

The Vitamix 5200 lacks preset speeds, but its variable speed was the
smoothest and most pleasant to use of all the blenders we tried. Photo:
Michael Hession

The Vitamix's Tritan-plastic jar feels sturdier than those of the other
blenders we recommend, and the grippy handle is comfortable to hold. We
also found the tall, narrow, tapered shape of the jar to be ideal for
creating a strong vortex that pulls ingredients down toward the blade.
That feature helped the Vitamix blend more efficiently than the Oster,
with its wider jar, and the result was vastly superior to what we got
from the wide, blocky jar of the Blendtec. As with the jars of most
other high-powered blenders, the jar of the Vitamix (which has the blade
attached) is very easy to clean. After you make a smoothie or something
similar, you should find it's sufficient to just pour in a bit of hot
water, add a couple of drops of dish soap, blend for 30 seconds or so,
and then rinse out the jar.

No high-powered blender we tested could be described as quiet, but we
found the noise from the Vitamix to be much less offensive than the
high-pitched whine of the Blendtec, and it was quieter than the roar of
our runner-up, the
\protect\hyperlink{runner-up-oster-versa-pro-series-blender}{Oster
Versa}.

Should its motor overheat, the Vitamix is equipped with an automatic
shutoff feature to keep it from burning out. In our experience, the
Vitamix should be able to handle a lot before it gets to that point, but
if your Vitamix does shut off, it's best to let the machine rest for an
hour before you try to use it again.

One thing that softens the blow of spending more than \$400 on a Vitamix
is the comfort of knowing that it's backed by a seven-year warranty. We
called Vitamix's customer service and learned that the approximate time
between filing a claim and receiving your blender back in working order
(or a certified refurb) is six to 10 days. For an additional \$75, you
can also buy a three-year extended warranty for the 5200. If you
purchase a new Vitamix from the company's site or from a certified
third-party retailer, such as Amazon, you have 30 days from the date of
purchase to buy the extended warranty directly from Vitamix. After 30
days is up, you can purchase the extended warranty up until the original
one expires for around \$120.

You can save some money on a Vitamix if you opt for a
certified-refurbished model. Jonathan Cochran of Blender Dude highly
recommends them. ``My pick for `best bang for the buck' continues to be
the Certified Refurbished (Blendtec) and Certified Reconditioned
(Vitamix) models. I have personally inspected hundreds of each, and for
all intents and purposes they are indistinguishable from the new models
at a significantly reduced price point,'' he told us. A certified
reconditioned Vitamix comes with a five-year warranty, with the option
to extend coverage three more years for an additional \$75.

\hypertarget{long-term-test-notes}{%
\subsubsection{Long-term test notes}\label{long-term-test-notes}}

We used the same Vitamix 5200 in our test kitchen for five years with
nothing but excellent results. It finally did burn out, but only after
we put it through strenuous use over the course of many tests for both
this guide and others. Still, it easily outlasted the Oster, and it made
many more (and better) batches of nut butter and extra-thick smoothies
before we pushed it to its limit. Since our Vitamix was still under
warranty when it burned out, we contacted customer service, and the
representatives promptly replaced it.

I've also used a Vitamix at home for years, and it's still my favorite
household blender, period. I long-term tested the runner-up, the Oster,
for six months and noticed some glaring differences: The Vitamix can
handle more without its motor straining, and the Vitamix's tamper is
much better than the Oster's, which is really hard to get down in there.

Other Wirecutter staffers also love their Vitamix blenders. Special
projects editor Ganda Suthivarakom has used hers since 2015 without
issue and says: ``I love that I can make a lot of vegan recipes for
cashew creams without having to soak the nuts beforehand.'' Senior staff
writer Chris Heinonen, who has had his Vitamix since 2018, guesses that
he has ``used it more than all my blenders in the past combined.'' The
only minor complaint we've heard is from senior editor Kalee Thompson,
who notes: ``It's so tall, it doesn't fit under the upper shelves over
my counters ... so I'm less inclined to leave it out, and once it's
away, I don't use it as much.'' That said, others have told us how much
they appreciate the Vitamix's large capacity.

\hypertarget{flaws-but-not-dealbreakers}{%
\subsection{Flaws but not
dealbreakers}\label{flaws-but-not-dealbreakers}}

We know that for many people, the biggest issue with the Vitamix 5200 is
its steep price. At around \$400 or so, it's at least twice the price of
our runner-up, the Oster Versa Pro Series Blender. In the past, we've
even made the Oster our top pick because of its comparatively affordable
price. But after years of testing the Vitamix and using it in our test
kitchen, we think it's really worth the investment. It's more durable
and all-around more effective than any other model we've found, and if
you plan on using a blender regularly, it will make your life a lot
easier. Plus, consider the cost of buying a smoothie rather than making
it at home: A morning smoothie can run from about \$5 to \$13, so in two
to four months you will have paid the same amount as for a 5200. A
Vitamix, by contrast, will last you at least seven years (and it makes a
lot more than smoothies).

At more than 20 inches tall, the Vitamix 5200 is a big appliance---too
big to fit under some kitchen cabinets. But none of the other
high-powered blenders we tested were much smaller. Though the Oster is a
couple of inches shorter, it also has a beefier base. If size is really
an issue for you, Vitamix makes other lines of blenders (as mentioned
below) that have a shorter profile. But we've found that the tall,
narrow shape of the 5200's blending jar is one of the components that
help this machine create such an effective vortex.

Finally, the Vitamix 5200 doesn't come with any presets, just a
variable-speed dial. But even though it's nice to be able to press a
button and have your blender run through a smoothie-making program, it's
not really essential. You'll probably want to stick close to your
blender anyway in order to use the tamper to get things moving, and it's
not hard to adjust the dial if you feel the need to. With the Vitamix
it's also easy to get good results without any presets.

\hypertarget{what-about-other-vitamix-models}{%
\subsection{What about other Vitamix
models?}\label{what-about-other-vitamix-models}}

The 5200 isn't the only blender in Vitamix's quiver---if you want the
blending power of the 5200 but strongly prefer presets, or if you need a
shorter jar that will fit your space, consider looking into other
models. (If you want a good breakdown of the different Vitamix models,
Jonathan Cochran of Blender Dude
\href{http://blenderdude.com/articles/vitamix-models/}{compares them}.)

That said, the original 5200 remains our favorite, because every new
blender from Vitamix comes with a squat jar that doesn't blend small
amounts as well as the 5200's tall and tapered pitcher. We tested the
5300 and found that the base of its short jar is too wide to develop and
keep a vortex if you're making, say, a thick smoothie for one or two
people. Check out the \protect\hyperlink{the-competition}{Competition
section} for more detailed testing notes on the 5300.

We haven't tested any models from the new
\href{https://www.vitamix.com/us/en_us/shop/smart-system-blenders}{Vitamix
Ascent Series}, but we suspect we'd have the same issue with the
shorter, squatter jars. According to owner reviews, the Ascent blenders
seem to come with some other issues, too, such as a complicated adapter
for the personal blending cup and a sensor that shuts off the machine if
it detects that the mixture in the jar is too thick. Our favorite
feature of the 5200 is that it blends absurdly thick concoctions!

\hypertarget{runner-up-oster-versa-pro-series-blender}{%
\subsection{Runner-up: Oster Versa Pro Series
Blender}\label{runner-up-oster-versa-pro-series-blender}}

\includegraphics{https://cdn.thewirecutter.com/wp-content/uploads/2017/08/blenders-lowres-3754-120x80.jpg}

Photo: Michael Hession

\hypertarget{runner-up-2}{%
\paragraph{Runner-up}\label{runner-up-2}}

\href{https://www.nytimes3xbfgragh.onion/wirecutter/out/link/3754/162049/4/109195?merchant=Amazon}{}

\hypertarget{oster-versa-pro-series-blender-2}{%
\subsubsection{\texorpdfstring{\href{https://www.nytimes3xbfgragh.onion/wirecutter/out/link/3754/162049/4/109195?merchant=Amazon}{Oster
Versa Pro Series
Blender}}{Oster Versa Pro Series Blender}}\label{oster-versa-pro-series-blender-2}}

\hypertarget{a-more-affordable-but-less-durable-blender-2}{%
\subsection{\texorpdfstring{\href{https://www.nytimes3xbfgragh.onion/wirecutter/out/link/3754/162049/4/109195?merchant=Amazon}{A
more affordable but less durable
blender}}{A more affordable but less durable blender}}\label{a-more-affordable-but-less-durable-blender-2}}

This Oster Versa model offers the best balance of performance and price
that we've found. It's not as powerful or durable as the Vitamix 5200,
but it holds its own against many blenders that are twice the price.

\hypertarget{buying-options-9}{%
\paragraph{Buying Options}\label{buying-options-9}}

\href{https://www.nytimes3xbfgragh.onion/wirecutter/out/link/3754/162049/4/109195?merchant=Amazon}{\$148*
from Amazon}

*At the time of publishing, the price was \$155.

We don't think you can beat the value of the
\href{https://www.nytimes3xbfgragh.onion/wirecutter/out/link/3754/162049/4/69780/?merchant=Amazon}{Oster
Versa Pro Series Blender}. It isn't quite as powerful as the Vitamix
5200, but it is half the price, and it beat out most of the other
blenders in its price range at making silky smoothies, purees, and
blended cocktails. It has one of the best combinations of variable and
preset speeds we've found, and its settings are more intuitive to use
than those on other models we've tried. It also comes with features,
such as a tamper and overheating protection, that are usually available
only on more expensive machines. We don't think the Oster is as durable
as the Vitamix (ours burned out after two and a half years). But it does
come with a seven-year warranty, and it's a great option if you're not
ready to spring for the
\protect\hyperlink{our-pick-vitamix-5200}{Vitamix}.

The Oster passed almost every challenge we threw at it. And although it
failed to achieve the absolute smoothest drink textures---it left whole
raspberry seeds in smoothies and made a slightly grainy piña
colada---compared with the Blendtec or the Cleanblend, its smoothies
were still much smoother than any of the results from lower-priced
blenders. As long as the Oster had about 2 cups of nuts to work with, it
made a decent nut butter (albeit one that was slightly crunchier than
the batch we made in the Vitamix). And it whipped up a velvety puree.
The only thing the Oster really struggled to do was make mayonnaise; we
were able to make an emulsification only once out of four tries.

We found the Oster easier to control than other blenders of a similar
price, thanks to its wide range of speeds. Though not as varied as those
on the Vitamix, the speeds on the Oster are far more diverse than those
on the Cleanblend, which, despite its variable-speed dial, seems to have
only two settings: high and higher. In comparison, the Oster's low speed
is sane enough that you can start pureeing a batch of soup without
having hot liquid shoot up the sides of the jar (a problem with the
Cleanblend).

\includegraphics{https://cdn.thewirecutter.com/wp-content/uploads/2018/06/blenders-lowres-2969-120x80.jpg}

The Oster made one of the smoothest smoothies. Its mixture left only a
small amount of pulp and berry seeds in our fine-mesh sieve. Photo:
Michael Hession

The Oster is the only one of our picks to have both manual speed
controls and preset programs for soup, dip, and smoothies. This makes it
more versatile than the more expensive entry-level models from Vitamix
and Blendtec, which have only variable or preset speeds, respectively.
To get presets with a Vitamix, or a variable-speed ``touch slider'' with
a Blendtec model, you need to spend even more.

The tamper that comes with the Oster is a little too short and oddly
shaped. In contrast to the smooth cylindrical tampers of the
\protect\hyperlink{our-pick-vitamix-5200}{Vitamix} and
\protect\hyperlink{also-great-cleanblend-blender}{Cleanblend} models,
the Oster's tamper has three flat pieces of plastic that meet in the
middle. But the design works sufficiently to burst air bubbles and help
move things like peanuts around the blades, which makes the Oster's
tamper better than no tamper at all.

This Oster model, like other high-performance blenders, is a beefy
machine. The base takes up 8 by 9 inches of counter space. But at 17½
inches tall to the top of the lid, the Oster will fit better on a
counter under most kitchen cabinets than the Vitamix or the Cleanblend,
both of which are more than 19 inches tall.

Also, like all the other high-powered blenders we tested, the Oster gets
loud when the motor is turned up all the way---much louder than the
Vitamix but not as annoying or high-pitched as the Blendtec. For now,
this is just the way it is with high-performance blenders.

Like the Vitamix, the Oster will shut off if the motor is in danger of
overheating. If the Oster's overload protection stops the motor, you
should allow it to cool for 45 minutes and press the reset button on the
bottom of the base before you run the blender again. This procedure
reduces the risk of permanent motor burnout.

The Oster Versa passed almost every challenge we threw at it.

Should it burn out, the Oster comes with a limited seven-year warranty
that covers ``defects in material and workmanship,'' including the motor
and the Tritan jar. That policy is about the same as the coverage from
Blendtec and Vitamix, which offer eight- and seven-year warranties,
respectively, on their models. In our experience, Oster's customer
service is courteous and quickly addresses any issues with a blender
while it's under warranty.

But if you're thinking that the Oster Versa will deliver the longevity
and performance of a Vitamix 5200 at a fraction of the cost, think
again. The Oster model's biggest flaw is its durability: We found
through personal experience that the Versa can burn out after two to
three years of moderate to frequent use (see our long-term test notes
for this model, below). We've seen some reviews on Amazon (as well as
comments from our own readers) that mention the same problem. But Oster
honors its seven-year warranty and is quick to send a replacement model
(we got ours in about a week). Although it took three attempts for us to
get through to customer service by phone during the busy holiday
shopping season, we're assuming that hiccup was due to the unusually
high call volume that occurs at that time of year.

The blending jar, lid, and controls on the Oster also feel cheaper
compared with what you get on the Vitamix. But given that this machine
is typically almost \$250 less, we're comfortable with the lower-quality
hardware.

\hypertarget{long-term-test-notes-1}{%
\subsubsection{Long-term test notes}\label{long-term-test-notes-1}}

For three years, we averaged using the Versa twice a week to make
smoothies and soup, and it never quit on us during that time---although
we occasionally detected a faint burning smell from the motor while we
were blending thick smoothies. But the motor permanently died when we
formally tested the three-year-old Versa again for our 2017 update. One
minute into our blending the nut butter, the overload protection cut the
motor. We should've let the motor rest for 45 minutes before restarting,
but we let it cool for only 10 minutes before our second attempt---and
that's when the motor burned out completely. However, our blender was
still under warranty, and Oster quickly sent a replacement model.

Wirecutter's community lead, Erin Price, uses the Oster Versa and so far
has no complaints. She told us: ``I've had the Oster Versa for three
years, and it's still going strong (though it sat in storage for one of
those years). I mostly use it for smoothies, and it handles ice and
greens so well.''

\hypertarget{also-great-cleanblend-blender}{%
\subsection{Also great: Cleanblend
Blender}\label{also-great-cleanblend-blender}}

\includegraphics{https://cdn.thewirecutter.com/wp-content/uploads/2017/12/blenders-lowres-9777-120x80.jpg}

Photo: Michael Hession

\hypertarget{also-great-2}{%
\paragraph{Also great}\label{also-great-2}}

\href{https://www.nytimes3xbfgragh.onion/wirecutter/out/link/8947/25620/4/109196?merchant=Amazon}{}

\hypertarget{cleanblend-blender-2}{%
\subsubsection{\texorpdfstring{\href{https://www.nytimes3xbfgragh.onion/wirecutter/out/link/8947/25620/4/109196?merchant=Amazon}{Cleanblend
Blender}}{Cleanblend Blender}}\label{cleanblend-blender-2}}

\hypertarget{powerful-blender-motor-shorter-warranty-2}{%
\subsection{\texorpdfstring{\href{https://www.nytimes3xbfgragh.onion/wirecutter/out/link/8947/25620/4/109196?merchant=Amazon}{Powerful
blender motor, shorter
warranty}}{Powerful blender motor, shorter warranty}}\label{powerful-blender-motor-shorter-warranty-2}}

A relative newcomer, this blender makes some of the silkiest smoothies,
but its speeds aren't as nuanced as those on our top pick.

\hypertarget{buying-options-10}{%
\paragraph{Buying Options}\label{buying-options-10}}

\href{https://www.nytimes3xbfgragh.onion/wirecutter/out/link/8947/25620/4/109196?merchant=Amazon}{\$180
from Amazon}

If you're willing to take a chance on a shorter warranty from a newer
company, the 1,800-watt
\href{https://www.nytimes3xbfgragh.onion/wirecutter/out/link/8947/25620/4/69781/?merchant=Amazon}{Cleanblend
Blender} costs about the same as the
\protect\hyperlink{runner-up-oster-versa-pro-series-blender}{Oster
Versa} and produces finer purees. In our tests, it blended silkier
smoothies and piña coladas than many models that cost more than twice as
much. This model comes with a durable Tritan-plastic jar and a tamper
for you to help move thick mixtures while it's blending. The Cleanblend
doesn't have any preset buttons, and its variable speeds aren't as
nuanced as the \protect\hyperlink{our-pick-vitamix-5200}{Vitamix's}, but
its interface is simple and intuitive to use. Judging from our long-term
testing, the Cleanblend's motor is durable and able to handle tough jobs
like nut butter better than the Oster. It's also backed by a complete
five-year warranty.

The Cleanblend made some of the smoothest smoothies in our tests,
performing better than the Oster and even the Vitamix in that regard.
When we strained our kale and berry smoothie, barely any raspberry seeds
remained in our fine-mesh sieve; the only blender that did better was
the Blendtec. The Cleanblend also came in second, behind the Blendtec,
in blending a silky-smooth piña colada. We're talking restaurant-worthy
blended drinks here.

For blending other things, the Cleanblend has a few limitations. It
doesn't have as wide a range of speeds as the Oster or the Vitamix, and
it kicks into high gear even at the 1 setting, which in our soup test
sent hot liquid shooting up to the lid. Although the Cleanblend was
better at making mayonnaise than the Oster, this model's motor also
seemed to produce a lot of heat; its mayo was noticeably warm. Like our
other picks (except the KitchenAid, our budget pick), the Cleanblend
comes with a tamper, but the bat is a little short. It works fine for
most tasks, but don't attempt to make nut butter from fewer than 2 cups
of nuts, because the shorter tamper won't reach the mixture once the
nuts are finely ground.

The Cleanblend made some of the smoothest smoothies in our tests.

Judging from our long-term testing, the Cleanblend's motor is more
durable than the Oster's, though we're not sure it's a match for the
motor of the time-tested Vitamix. In our 2017 testing, our four-year-old
Cleanblend and Vitamix blenders both powered through two rounds of nut
butter without quitting. That same test fried our three-year-old Oster.
That said, Oster offers a seven-year warranty on the Versa Pro Series
Blender, but Cleanblend offers only a five-year total warranty.

For an extra \$75, you can extend the warranty on your Cleanblend
Blender to a total of 10 years. This is a great value when you consider
that the blender, including the decade of coverage, is still about \$200
less than a Vitamix. If you're looking for the all-around great
performance of a Vitamix for less than half the cost, you won't find
that here (or anywhere else for that matter), but the Cleanblend is a
good value when you compare the numbers.

However, Cleanblend's customer service is reachable only by email or a
form on its website, and that might not inspire confidence in some
people. Both Vitamix and Oster have a customer service phone number that
connects you to a representative. Even though the Cleanblend seems more
durable than the Oster, Cleanblend is such a new company that we're not
yet confident in its machine's long-term reliability.

The Cleanblend's base takes up 9½ by 8 inches of counter space, about
the same as our other high-performance picks (our budget pick, the
KitchenAid, is smaller). And at 19 inches high to the top of the lid,
the Cleanblend is taller than the Oster, but it has just slightly more
clearance under most kitchen cabinets than the Vitamix (which measures
closer to 20 inches). Also, like all of the other high-performance
blenders we tested, the Cleanblend is loud. But compared with the Ninja
Chef's thunderous roar and the Blendtec's high-pitched whine, the
Cleanblend is far easier on the ears.

\hypertarget{long-term-test-notes-2}{%
\subsubsection{Long-term test notes}\label{long-term-test-notes-2}}

Staff writer Michael Sullivan has used the Cleanblend at home for about
two years and says he's never had an issue with it. He pulls it out
about six times a month to make smoothies, sauces, soup, or occasionally
emulsifications like mayonnaise. He has even crushed ice in it a few
times, and he says that so far it has never stalled out.

Sabrina Imbler, a Wirecutter staff writer at the time of our tests, used
the Cleanblend in her home for more than a year. She used it three to
four times a week and never experienced stalling or burnout. She told
us: ``{[}My{]} only minor complaint is that sometimes the blender
rattles a bit on top of the base, which makes me a little wary, but
otherwise it's great. I only use it for smoothies and mixed drinks,
never any kind of nuts, but it pulverizes ice pretty quick. It's also
the perfect size for two smoothies. I tend to use the middle range of
speeds, as I rarely need the highest, and the lowest is less effective
for my needs. And I really like that it's a dial as opposed to number
buttons---easier to {[}crank{]} up if my stuff isn't blending fast.''

\hypertarget{budget-pick-kitchenaid-5-speed-diamond-blender}{%
\subsection{Budget pick: KitchenAid 5-Speed Diamond
Blender}\label{budget-pick-kitchenaid-5-speed-diamond-blender}}

\includegraphics{https://cdn.thewirecutter.com/wp-content/uploads/2020/01/blenders-lowres-1175-120x80.jpg}

Photo: Sarah Kobos

\hypertarget{budget-pick-2}{%
\paragraph{Budget pick}\label{budget-pick-2}}

\href{https://www.nytimes3xbfgragh.onion/wirecutter/out/link/35968/159119/4/109197?merchant=Bed\%20Bath\%20\%26\%20Beyond}{}

\hypertarget{kitchenaid-5-speed-diamond-blender-2}{%
\subsubsection{\texorpdfstring{\href{https://www.nytimes3xbfgragh.onion/wirecutter/out/link/35968/159119/4/109197?merchant=Bed\%20Bath\%20\%26\%20Beyond}{KitchenAid
5-Speed Diamond
Blender}}{KitchenAid 5-Speed Diamond Blender}}\label{kitchenaid-5-speed-diamond-blender-2}}

\hypertarget{for-occasional-blending-2}{%
\subsection{\texorpdfstring{\href{https://www.nytimes3xbfgragh.onion/wirecutter/out/link/35968/159119/4/109197?merchant=Bed\%20Bath\%20\%26\%20Beyond}{For
occasional
blending}}{For occasional blending}}\label{for-occasional-blending-2}}

Although this blender isn't as powerful as more expensive models, it can
produce a great emulsification and decent (but chunkier) smoothies.

\hypertarget{buying-options-11}{%
\paragraph{Buying Options}\label{buying-options-11}}

\href{https://shop-links.co/1696305516080111511}{\$90 from Bed Bath \&
Beyond}

\href{https://www.nytimes3xbfgragh.onion/wirecutter/out/link/35968/159787/4/109197?merchant=Home\%20Depot}{\$130
from Home Depot}

\href{https://www.nytimes3xbfgragh.onion/wirecutter/out/link/35968/158229/4/109197?merchant=Amazon}{Buy
from Amazon}

If you just want a simple and inexpensive blender for occasional use, we
like the
\href{https://www.nytimes3xbfgragh.onion/wirecutter/out/link/35968/159119/4/109206/?merchant=Bed\%20Bath\%20\%26\%20Beyond}{KitchenAid
5-Speed Diamond Blender}. It's even better than its predecessor, the
KitchenAid 5-Speed Classic (our former budget pick), at maintaining a
consistent vortex, which is the key to faster blending. The KitchenAid
is a good starter blender for beginner cooks outfitting their first
kitchen. It doesn't produce the smoothest purees, nor does it grind nuts
into creamy butter. But for the price, the KitchenAid is a very good
medium-duty blender.

In our tests, the KitchenAid easily blended frozen berries and kale. It
left a much more pulpy texture to the smoothies than our other picks
(testers crunched on whole berry seeds with each sip). But the
KitchenAid's performance far outpaced that of any other blender we
tested in its price range.

The vortex on the KitchenAid is really impressive, due in large part to
its narrow and tapered 60-ounce jar. This jar shape helps draw food down
into the blades and keep the mixture moving. It blends a slightly icier
piña colada than our other picks, but unless you were to compare it side
by side with a Vitamix, you wouldn't notice the difference.

\includegraphics{https://cdn.thewirecutter.com/wp-content/uploads/2020/01/blenders-lowres-1162-120x80.jpg}

The only noticeable difference between the KitchenAid 5-Speed Diamond
(left) and our former budget pick, the 5-Speed Classic (right), is the
taller pitcher. We like the Diamond's tall, tapered jar because it
develops and holds a vortex faster and better than that of the 5-Speed
Classic. Photo: Sarah Kobos

To be clear, this model is no Oster---and it's certainly not a Vitamix.
The KitchenAid doesn't have the power to pulverize nuts into smooth
butter, and its speed settings are not as varied as those on our other
picks. But it does a decent enough job if you need something that will
make the occasional smoothie or pureed soup for a fraction of the price.
The other blenders we tested in this price range felt cheap or were very
loud, or they produced a gross, burning-motor smell while running. We
didn't love the way the jar clipped onto the base (it took some getting
used to), but beyond that we had no complaints. Like our other picks,
the KitchenAid has a jar made of BPA-free Tritan plastic.

The vortex on the KitchenAid is really impressive.

We've seen some owner reviews of the KitchenAid that complain about
leaking jars and burned-out motors. But all of the regular blenders
we've looked at have similar reviews, and the comments on the KitchenAid
aren't any worse. If you use it heavily, the KitchenAid may not last you
many years, but we think it's the best in its price range. There are
other complaints about the blender not fitting under upper cabinets. But
we found that the KitchenAid, at 17 inches tall, fit the standard,
18-inch gap between a countertop and upper cabinets.

The KitchenAid 5-Speed Diamond Blender comes with a one-year limited
warranty, which is the shortest warranty of all of our picks by far. But
as we said before, if you want a blender only to make the occasional
smoothie, sauce, or soup---and you want to spend less than \$100---the
KitchenAid should reliably do just that for years to come.

\hypertarget{blender-care-and-maintenance}{%
\subsection{Blender care and
maintenance}\label{blender-care-and-maintenance}}

If you find that your blender is having a difficult time processing
ingredients, don't be afraid to be aggressive (within reason) with the
tamper to get the mixture moving around the blades. Also, make sure the
blender jar is at least 25 percent full. Although high speeds will help
process smoother mixtures, a
\href{http://demandware.edgesuite.net/aamb_prd/on/demandware.static/Sites-oster-Site/Sites-oster-Library/default/v1429181234748/documents/instruction-manuals/BLSTVB-RV0-000_IB\%20Versa\%201400\%20Series\%20Blender.pdf}{lower
speed} (PDF) may also help ingredients start circulating if they just
aren't moving. When you're following a recipe, it's also good to add
ingredients in the order listed; blender recipe books tend to be
specific with the order (Vitamix, for example, generally lists ice as
the last ingredient).

To limit the risk of hot liquids shooting out the top of a blending jar,
always start on a low setting and slowly increase the speed (presets
generally do this automatically for you). Never fill the jar past the
hot-liquid fill line. And for good measure, to limit the risk of the lid
popping off, place a dish towel over the lid, with your hand firmly
holding the lid down, while you blend.

Hand-wash the blending jar with warm, soapy water rather than running it
through the dishwasher. This will help extend the life of the jar. In
our own testing, we found that the best way to clean a blender jar is to
use a bottle brush or a scrub brush; processing water and a little soap
in the blender jar will help loosen up tough ingredients like peanut
butter, and the brush should do the rest.

\hypertarget{the-competition}{%
\subsection{The competition}\label{the-competition}}

The
\href{https://www.nytimes3xbfgragh.onion/wirecutter/out/link/34557/156463/4/109200/?merchant=Amazon}{Breville
Super Q} is a performance blender that's packed with lots of bells and
whistles. In our tests, with its squat jar and powerful motor, the Super
Q performed a lot like the Blendtec Designer 675, throwing smoothie up
the sides and into the lid. At one point, the Breville shot bits of a
smoothie in my face when I opened the cap to add more liquid. The Super
Q pulverizes tough foods, but the Vitamix also does that for less
money---and with less drama inside the jar. The Super Q also generated a
lot of heat when we made peanut butter---so much that we had to stop the
test early when we noticed steam coming out of the jar. Although the
Super Q blended the silkiest piña coladas and came with lots of extra
goodies (a 68-ounce jar, a personal blending jar, preset blending
programs, and a vacuum attachment that's supposed to slow the oxidation
of raw foods), we don't think it's worth the \$100-plus over the
Vitamix's price, especially since most of those goodies would just
clutter your cabinets.

The
\href{https://www.nytimes3xbfgragh.onion/wirecutter/out/link/35381/157534/4/103356/?merchant=Amazon}{KitchenAid
K400} blender performed slightly better than the
\protect\hyperlink{budget-pick-kitchenaid-5-speed-diamond-blender}{KitchenAid
5-Speed Diamond} (our budget pick), but not enough to warrant its
\$150-plus price jump. And the K400 wasn't nearly as good at blending
fibrous kale as the less expensive Oster and Cleanblend blenders.

\includegraphics{https://cdn.thewirecutter.com/wp-content/uploads/2018/11/instantpotblender-lowres--120x80.jpg}

The Instant Pot Ace 60 Cooking Blender includes a heating element in the
base, which makes it unique but also bulky. Photo: Sarah Kobos

The
\href{https://www.nytimes3xbfgragh.onion/wirecutter/out/link/28373/147455/4/86231/?merchant=Walmart}{Instant
Pot Ace 60 Cooking Blender} is unique among the blenders we tested in
that it has a heating element in its base, so it can both cook and puree
foods (some high-powered blenders also claim to heat soup, but they do
so only with friction). Although it's a nifty feature, as we watched
squash soup gently burp and bubble inside the blender's glass jar, the
Ace reminded us of a lava lamp: novel but impractical. After performing
extensive testing, we don't think the Ace is better than any of our
picks.

The Ace works well at some basic tasks: It whipped up smooth peanut
butter and did a slightly better job of pulverizing tough kale leaves,
ice cubes, and chewy dates than our former budget pick, the
\href{https://www.nytimes3xbfgragh.onion/wirecutter/out/link/7763/22821/4/109201/?merchant=Amazon}{KitchenAid
5-Speed Classic Blender}. But the Ace is huge and loud compared with the
KitchenAid, and it doesn't offer as much control. And though we like
that the Ace comes with a tamper, its glass jar is heavier and less
durable than the Tritan plastic jars of our picks. The jar's wide base
also makes creating mayonnaise impossible and makes it difficult for the
Ace to form a powerful vortex (instead flinging ingredients all over the
jar).

The Ace cooks soups quickly, but it can't make much food at once. Video:
Sarah Kobos

As for the Ace's cooking abilities, we were able to make a satisfying
broccoli cheese soup and a smooth butternut squash soup from raw
ingredients, but we had to blend for longer than the programmed setting
to get a creamy texture. And we were disappointed that we couldn't
adjust the temperature or sauté in the machine, since the heating
element doesn't start if it doesn't detect liquid in the jar. As such,
the Ace doesn't produce the same nuanced flavors that you'd get if you
were to start with a little caramelization. The heating element also
introduces another possible point of failure into a type of appliance
that is already prone to burning out.

Compared with our top pick, the Vitamix 5200, the
\href{https://www.nytimes3xbfgragh.onion/wirecutter/out/link/29475/149476/4/83450/?merchant=Amazon}{Vitamix
5300} has the same 64-ounce capacity and speed-control dial, but it
lacks the ultra-high-speed switch available on the 5200. It has a
slightly higher peak horsepower, but any extra power is negated by the
shape of the jar. In testing, we found that the 5300's squatter jar
didn't maintain a vortex as well as the 5200's narrow, tapered one.
Also, for smaller volumes---2 cups or less---the 5300's tamper didn't
reach down quite far enough to burst air pockets. We had to add more
liquid to thicker mixtures, such as date puree and hummus, because the
tamper wasn't cutting it.

The
\href{https://www.nytimes3xbfgragh.onion/wirecutter/out/link/29477/159139/4/109202/?merchant=Costco}{Vitamix
Explorian E320}, available at Costco, is 99 percent identical to the
5300. A Vitamix customer service representative told us that the two
blenders have the same motor base, jar, tamper, and functionality. The
main difference between the blenders is that the 5300 has a small on/off
switch located just below the control panel. On top of that, the E320 is
available only as part of a package with two personal cups and an
adapter.

Vitamix added the
\href{https://www.nytimes3xbfgragh.onion/wirecutter/out/link/26098/137761/4/74159/?merchant=Amazon}{Explorian
Series E310} variable-speed blender to its lineup in 2017. We chose not
to test this model because we don't think it's a good value. Although
it's typically three-quarters the price of the
\protect\hyperlink{our-pick-vitamix-5200}{Vitamix 5200}, the cost
difference is directly proportional to the E310's smaller blending jar
(48 ounces versus 64 ounces) and shorter warranty (five versus seven
years). On the E310, Vitamix also replaced the switch that flips the
machine from variable speed to high power with a pulse switch, thus
eliminating the option for one-touch high-power blending. If you have
limited storage space in your kitchen, you might like the E310 for its
shorter height (about 17 inches tall, compared with the Vitamix 5200,
which is about 20 inches tall). But if you're going to shell out the
cash for a Vitamix blender, we still think spending a little more on the
5200 is the best choice.

The
\href{https://www.nytimes3xbfgragh.onion/wirecutter/out/link/24354/130080/4/109203/?merchant=Amazon}{Ninja
Chef} 1,500-watt blender is the first high-performance model from this
company that doesn't have sets of blades throughout the jar. Instead,
the Ninja Chef's blades sit in the base of the jar, as in normal
blenders. This model also performed better than its predecessors. But
it's extremely loud, and our top picks---the Vitamix, the Oster, and the
Cleanblend---still blended silkier smoothies in our tests.

The 1,800-watt
\href{https://www.nytimes3xbfgragh.onion/wirecutter/out/link/24355/130081/4/69785/?merchant=Amazon}{Hamilton
Beach Professional Blender} performed well in our tests. When we used
the manual speeds, the blender's digital readout showed a countdown
timer, which was helpful because the instruction manual advises against
continuously running the motor for more than two minutes. But the
preprogrammed settings didn't effectively keep the mixture moving when
air pockets occurred. In addition, the on/off buttons are angled upward
at the top of the base and thus susceptible to food and grime buildup
over time.

The
\href{https://www.nytimes3xbfgragh.onion/wirecutter/out/link/15879/130108/4/57879/?merchant=Amazon}{KitchenAid
Pro Line Series Blender} is expensive, and it's also the heaviest
blender we've tested (22 pounds). It blended silky-smooth textures,
though not quite as easily as the Vitamix 5200, and it also didn't do
well at emulsification. Its performance intrigued us, but after a year
of long-term testing this model, we found that it delivered similar
results to the Vitamix. And the heft and size of this KitchenAid model
make it a difficult-to-move space hog.

The
\href{https://www.nytimes3xbfgragh.onion/wirecutter/out/link/13556/38700/4/57877/?merchant=Amazon}{Cuisinart
CBT-1500 Hurricane} struggled to process foods in our tests. Blending
thick smoothies and peanut butter required more liquid, a lot of
starting and stopping, and banging the jar on the counter. It did make
mayonnaise on the first try, though, unlike the more powerful Cuisinart
CBT-2000 Hurricane Pro. But without the Turbo button of the Hurricane
Pro (more on that below), this model is just another middle-of-the-road
blender.

The
\href{https://www.nytimes3xbfgragh.onion/wirecutter/out/link/13557/38703/4/57878/?merchant=Amazon}{Cuisinart
CBT-2000 Hurricane Pro} performed similarly to the Cuisinart CBT-1500
Hurricane, except it didn't make mayonnaise as well (we achieved
emulsification on the third try only). We did find the Turbo button
useful for creating a fine puree. But again, without a tamper to burst
air pockets, this blender needed a lot of tending to produce uniform,
smooth purees.

The
\href{https://www.nytimes3xbfgragh.onion/wirecutter/out/link/15878/47240/4/57876/?merchant=Amazon}{Braun
PureMix} is a small, tamper-less blender, and it didn't impress us in
the least, with a flimsy jug and a lightweight base. The PureMix had a
hard time blending our smoothie, and we needed to add so much liquid to
the mixture that the texture was way too thin---yuck! We disqualified
the Braun after our first test.

Will the
\href{https://www.nytimes3xbfgragh.onion/wirecutter/out/link/26968/141467/4/76667/?merchant=Amazon}{Blendtec
Designer 675} blend? Yes, but not as well as our top picks. Despite
Blendtec's clever (if at times
\href{http://www.willitblend.com/videos/minecraft-frozen}{mildly
sinister}) marketing campaign of blending everything from
\href{https://www.youtube.com/watch?v=aM94aorYVS4}{rake handles} to
\href{https://www.youtube.com/watch?v=lBUJcD6Ws6s}{iPhones}, we've found
its blenders wanting (we also tested the Total model in 2012). Although
in our tests the Designer 675 killed it on making smoothies and blended
drinks, its lack of a tamper limits its usefulness. It didn't make
peanut butter (a tamper would have helped), and the preset speed for
soup was frightening, with hot liquid flying wildly around the jar. We
do think this particular model is quite beautiful, with a sleek black,
illuminated base. It's a great blender if you want something that looks
slick on your counter and can make amazingly smooth mixed drinks and
smoothies. But we think a blender that's this expensive should perform
well at more than just those two tasks. For more on how the Blendtec
stacks up against the Vitamix 5200, read our piece on
\href{https://www.nytimes3xbfgragh.onion/wirecutter/reviews/vitamix-vs-blendtec/}{testing
the two blenders head-to-head}.

The
\href{https://www.nytimes3xbfgragh.onion/wirecutter/out/link/8948/25621/4/57888?merchant=Amazon}{Waring
Commercial Xtreme} made very smooth smoothies, and it felt substantial.
But ultimately it didn't perform better than our picks from Vitamix,
Oster, or Cleanblend. If we were willing to pay this much for a blender,
we'd instead go for a reconditioned Vitamix 5200. We do like that Waring
has a
\href{https://www.nytimes3xbfgragh.onion/wirecutter/out/link/8951/25631/4/57889?merchant=Amazon}{metal
jar} that you can purchase for this machine.

We tested the
\href{https://www.nytimes3xbfgragh.onion/wirecutter/out/link/8518/24610/4/57891?merchant=Amazon}{Blendtec
Total Blender} for our 2012 review but found that it couldn't compete
with the Vitamix we tested at the time. The lid felt flimsy, and this
model's panel controls seemed cheap.

When it came to blending green and berry smoothies, we thought the
\href{https://www.nytimes3xbfgragh.onion/wirecutter/out/link/36679/159116/4/109204/?merchant=Amazon}{Oster
Beehive Blender} did a pretty good job. It left a lot of pulp behind,
and we kept having to open the lid to tamp down ingredients for our ice
and bean-spread tests. It was also especially loud (at a really annoying
frequency).

For the price, the
\href{https://www.nytimes3xbfgragh.onion/wirecutter/out/link/8526/24649/4/57895?merchant=Amazon}{Ninja
Master Prep Professional} is a decent machine, but we don't think it
compares to any of our other picks. It did a surprisingly good job of
making smoothies, mixing bean spread, and blending margaritas, but the
design is terrible for making mayonnaise (the motor is top-mounted, so
you can't drizzle anything into the jar). The stacked blades are also
dangerously sharp, making them difficult to clean. The Ninja Master Prep
Professional comes with three blending jars in various sizes; we thought
it added up to too many parts and that they would just end up cluttering
our cupboards. Overall, the machine feels really cheap.

The
\href{https://www.nytimes3xbfgragh.onion/wirecutter/out/link/8527/24650/4/57896?merchant=Amazon}{Ninja
Professional Blender 1000} didn't perform well. The green smoothies we
made in this blender had a weird, confetti-like texture. And the mayo
this model made was especially loose, which meant that too much air was
getting whipped in. Every time we ran the Ninja Professional, we
detected a strong, burning-motor smell. The jar was hard to get on the
base, and the lid was tricky to clamp on. Also, the base was big and
clunky and felt cheap.

\hypertarget{what-to-look-forward-to}{%
\subsection{What to look forward to}\label{what-to-look-forward-to}}

KitchenAid released the
\href{https://www.nytimes3xbfgragh.onion/wirecutter/out/link/36680/159117/4/109205/?merchant=KitchenAid}{K150
blender} in early 2020. It's priced comparably to the
\protect\hyperlink{budget-pick-kitchenaid-5-speed-diamond-blender}{5-Speed
Diamond} (our budget pick), but the K150 has only three speeds (which is
not necessarily a bad thing). We think the K150 may have budget-pick
potential, and we'll test it as soon as we can.

\hypertarget{sources}{%
\subsection{Sources}\label{sources}}

\begin{enumerate}
\def\labelenumi{\arabic{enumi}.}
\item
  \href{http://www.americastestkitchen.com/equipment_reviews/1383-blenders}{Midpriced
  Blenders (subscription required)}, America's Test Kitchen
\item
  Andrew Gebhart, Ry Crist,
  \href{http://www.cnet.com/news/from-smoothies-to-pesto-seven-blenders-reviewed/}{From
  smoothies to pesto to almond butter: 13 blenders reviewed}, CNET,
  August 22, 2014
\item
  Lisa McManus,
  \href{https://www.americastestkitchen.com/tour}{executive editor of
  equipment testing at America's Test Kitchen}, interview
\item
  Jonathan Cochran, \href{http://blenderdude.com}{author of the Blender
  Dude blog}, interview
\item
  J. Kenji López-Alt,
  \href{http://www.seriouseats.com/2014/12/high-end-blender-test-equipment-breville-vitamix-blendtec-review.html}{Vitamix
  vs. Blendtec vs. Breville: Who Makes the Best High-End Blender?},
  Serious Eats, December 16, 2014
\end{enumerate}

\hypertarget{about-your-guide}{%
\subsection{About your guide}\label{about-your-guide}}

\includegraphics{https://cdn.thewirecutter.com/wp-content/uploads/2015/12/stockton-lesley.jpg}

Lesley Stockton

Lesley Stockton is a senior staff writer reporting on all things cooking
and entertaining for Wirecutter. Her expertise builds on a lifelong
career in the culinary world---from a restaurant cook and caterer to a
food editor at Martha Stewart. She is perfectly happy to leave all that
behind to be a full-time kitchen-gear nerd.

\hypertarget{further-reading}{%
\subsection{Further reading}\label{further-reading}}

\begin{itemize}
\item
  \href{https://www.nytimes3xbfgragh.onion/wirecutter/reviews/best-blender-for-smoothies/}{}

  \includegraphics{https://cdn.thewirecutter.com/wp-content/uploads/2018/06/blenderforsmoothies-lowres-2997-120x80.jpg}

  \hypertarget{whats-the-best-blender-for-smoothies}{%
  \subsubsection{\texorpdfstring{\href{https://www.nytimes3xbfgragh.onion/wirecutter/reviews/best-blender-for-smoothies/}{What's
  the Best Blender for
  Smoothies?}}{What's the Best Blender for Smoothies?}}\label{whats-the-best-blender-for-smoothies}}

  by Lesley Stockton

  A thick, silky smoothie is one of the hardest things to make in a
  blender, so we think the best blender for smoothies is the best,
  period: the
  \href{https://www.nytimes3xbfgragh.onion/wirecutter/out/link/7761/0/4/76626/}{Vitamix
  5200}.
\item
  \href{https://www.nytimes3xbfgragh.onion/wirecutter/reviews/best-personal-blender/}{}

  \includegraphics{https://cdn.thewirecutter.com/wp-content/uploads/2017/08/personalblenders-lowres-6582-120x80.jpg}

  \hypertarget{the-best-personal-blender}{%
  \subsubsection{\texorpdfstring{\href{https://www.nytimes3xbfgragh.onion/wirecutter/reviews/best-personal-blender/}{The
  Best Personal
  Blender}}{The Best Personal Blender}}\label{the-best-personal-blender}}

  by Lesley Stockton

  After spending 20 hours researching two dozen personal blenders and
  testing ten models with 20+ pounds of ingredients, we think the
  \href{https://www.nytimes3xbfgragh.onion/wirecutter/out/link/17375/162800/4/59216/?merchant=Nutribullet}{NutriBullet
  Pro 900 Series} offers the best balance of power, simplicity, and
  price for most people.
\item
  \href{https://www.nytimes3xbfgragh.onion/wirecutter/reviews/best-immersion-blender/}{}

  \includegraphics{https://cdn.thewirecutter.com/wp-content/uploads/2018/08/immersionblenders-lowres-0683-120x80.jpg}

  \hypertarget{the-best-immersion-blender}{%
  \subsubsection{\texorpdfstring{\href{https://www.nytimes3xbfgragh.onion/wirecutter/reviews/best-immersion-blender/}{The
  Best Immersion
  Blender}}{The Best Immersion Blender}}\label{the-best-immersion-blender}}

  by Christine Cyr Clisset, Michael Sullivan, and Sharon Franke

  We've spent over 50 hours researching and testing immersion blenders,
  and after pureeing gallons of soup, we think the
  \href{https://www.nytimes3xbfgragh.onion/wirecutter/out/link/7616/0/4/79148/}{Breville
  Control Grip} is the best.
\item
  \href{https://www.nytimes3xbfgragh.onion/wirecutter/lists/pro-kitchen-tools-to-level-up-your-home-cooking/}{}

  \includegraphics{https://cdn.thewirecutter.com/wp-content/uploads/2018/11/pro-kitchen-tools-3x2-1-120x80.jpg}

  \hypertarget{pro-kitchen-tools-to-level-up-your-home-cooking}{%
  \subsubsection{\texorpdfstring{\href{https://www.nytimes3xbfgragh.onion/wirecutter/lists/pro-kitchen-tools-to-level-up-your-home-cooking/}{Pro
  Kitchen Tools to Level Up Your Home
  Cooking}}{Pro Kitchen Tools to Level Up Your Home Cooking}}\label{pro-kitchen-tools-to-level-up-your-home-cooking}}

  by Raphael Brion

  If you're ready to make the jump from occasional home cook to home
  chef, these are the pro-level tools we recommend.
\end{itemize}

\href{https://thewirecutter.com/wp-admin/post.php?post=10824\&action=edit}{Edit}

\begin{itemize}
\tightlist
\item
  \href{/wirecutter/electronics/}{Electronics}
\item
  \href{/wirecutter/home-garden/}{Home \& Garden}
\item
  \href{/wirecutter/kitchen-dining/}{Kitchen \& Dining}
\item
  \href{/wirecutter/money/}{Money}
\item
  \href{/wirecutter/travel/}{Travel}
\item
  \href{/wirecutter/gifts/}{Gifts}
\item
  \href{/wirecutter/appliances/}{Appliances}
\item
  \href{/wirecutter/health-fitness/}{Health \& Fitness}
\item
  \href{/wirecutter/baby-kid/}{Baby \& Kid}
\item
  \href{/wirecutter/outdoors/}{Outdoors}
\item
  \href{/wirecutter/pets/}{Pets}
\item
  \href{/wirecutter/hobby-crafts/}{Hobby \& Crafts}
\item
  \href{/wirecutter/software/}{Software \& Apps}
\item
  \href{/wirecutter/office/}{Office}
\item
  \href{/wirecutter/cars/}{Cars}
\item
  \href{/wirecutter/adult/}{Adult}
\item
  \href{/wirecutter/deals/}{Deals}
\item
  \href{/wirecutter/lists/}{Lists}
\item
  \href{/wirecutter/blog/}{Blog}
\end{itemize}

\begin{itemize}
\tightlist
\item
  \href{/wirecutter/about/}{About Wirecutter}
\item
  \href{/wirecutter/masthead/}{Masthead}
\item
  \href{/wirecutter/jobs/}{Jobs}
\item
  \href{/wirecutter/contact-us/}{Contact Us}
\end{itemize}

\begin{itemize}
\tightlist
\item
  \href{https://www.nytimes3xbfgragh.onion/subscription/privacy-policy\#/privacy}{Privacy
  Policy}
\item
  \href{https://help.nytimes3xbfgragh.onion/hc/en-us/articles/115014893428-Terms-of-service}{Terms
  of Use}
\item
  \href{https://www.nytimes3xbfgragh.onion/subscription/privacy-policy\#/cookie}{Cookie
  Policy}
\item
  \href{/wirecutter/partners/}{Partnerships \& Advertising}
\item
  \href{/wirecutter/feed/}{RSS Feed}
\end{itemize}

Let's be friends!

\begin{itemize}
\tightlist
\item
  \href{https://www.facebookcorewwwi.onion/thewirecutter/}{}
\item
  \href{https://www.instagram.com/wirecutter/}{}
\item
  \href{https://twitter.com/wirecutter/}{}
\end{itemize}

You can \href{/wirecutter/contact-us/}{send us a note} too.

© 2020 Wirecutter, Inc., \href{https://nytimes3xbfgragh.onion/}{A New
York Times Company}
